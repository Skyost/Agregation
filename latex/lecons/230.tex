\documentclass[12pt, a4paper]{report}

% LuaLaTeX :

\RequirePackage{iftex}
\RequireLuaTeX

% Packages :

\usepackage[french]{babel}
%\usepackage[utf8]{inputenc}
%\usepackage[T1]{fontenc}
\usepackage[pdfencoding=auto, pdfauthor={Hugo Delaunay}, pdfsubject={Mathématiques}, pdfcreator={agreg.skyost.eu}]{hyperref}
\usepackage{amsmath}
\usepackage{amsthm}
%\usepackage{amssymb}
\usepackage{stmaryrd}
\usepackage{tikz}
\usepackage{tkz-euclide}
\usepackage{fourier-otf}
\usepackage{fontspec}
\usepackage{titlesec}
\usepackage{fancyhdr}
\usepackage{catchfilebetweentags}
\usepackage[french, capitalise, noabbrev]{cleveref}
\usepackage[fit, breakall]{truncate}
\usepackage[top=2.5cm, right=2cm, bottom=2.5cm, left=2cm]{geometry}
\usepackage{enumerate}
\usepackage{tocloft}
\usepackage{microtype}
%\usepackage{mdframed}
%\usepackage{thmtools}
\usepackage{xcolor}
\usepackage{tabularx}
\usepackage{aligned-overset}
\usepackage[subpreambles=true]{standalone}
\usepackage{environ}
\usepackage[normalem]{ulem}
\usepackage{marginnote}
\usepackage{etoolbox}
\usepackage{setspace}
\usepackage[bibstyle=reading, citestyle=draft]{biblatex}
\usepackage{xpatch}
\usepackage[many, breakable]{tcolorbox}
\usepackage[backgroundcolor=white, bordercolor=white, textsize=small]{todonotes}

% Bibliographie :

\newcommand{\overridebibliographypath}[1]{\providecommand{\bibliographypath}{#1}}
\overridebibliographypath{../bibliography.bib}
\addbibresource{\bibliographypath}
\defbibheading{bibliography}[\bibname]{%
	\newpage
	\section*{#1}%
}
\renewbibmacro*{entryhead:full}{\printfield{labeltitle}}
\DeclareFieldFormat{url}{\newline\footnotesize\url{#1}}
\AtEndDocument{\printbibliography}

% Police :

\setmathfont{Erewhon Math}

% Tikz :

\usetikzlibrary{calc}

% Longueurs :

\setlength{\parindent}{0pt}
\setlength{\headheight}{15pt}
\setlength{\fboxsep}{0pt}
\titlespacing*{\chapter}{0pt}{-20pt}{10pt}
\setlength{\marginparwidth}{1.5cm}
\setstretch{1.1}

% Métadonnées :

\author{agreg.skyost.eu}
\date{\today}

% Titres :

\setcounter{secnumdepth}{3}

\renewcommand{\thechapter}{\Roman{chapter}}
\renewcommand{\thesubsection}{\Roman{subsection}}
\renewcommand{\thesubsubsection}{\arabic{subsubsection}}
\renewcommand{\theparagraph}{\alph{paragraph}}

\titleformat{\chapter}{\huge\bfseries}{\thechapter}{20pt}{\huge\bfseries}
\titleformat*{\section}{\LARGE\bfseries}
\titleformat{\subsection}{\Large\bfseries}{\thesubsection \, - \,}{0pt}{\Large\bfseries}
\titleformat{\subsubsection}{\large\bfseries}{\thesubsubsection. \,}{0pt}{\large\bfseries}
\titleformat{\paragraph}{\bfseries}{\theparagraph. \,}{0pt}{\bfseries}

\setcounter{secnumdepth}{4}

% Table des matières :

\renewcommand{\cftsecleader}{\cftdotfill{\cftdotsep}}
\addtolength{\cftsecnumwidth}{10pt}

% Redéfinition des commandes :

\renewcommand*\thesection{\arabic{section}}
\renewcommand{\ker}{\mathrm{Ker}}

% Nouvelles commandes :

\newcommand{\website}{https://agreg.skyost.eu}

\newcommand{\tr}[1]{\mathstrut ^t #1}
\newcommand{\im}{\mathrm{Im}}
\newcommand{\rang}{\operatorname{rang}}
\newcommand{\trace}{\operatorname{trace}}
\newcommand{\id}{\operatorname{id}}
\newcommand{\stab}{\operatorname{Stab}}

\providecommand{\newpar}{\\[\medskipamount]}

\providecommand{\lesson}[3]{%
	\title{#3}%
	\hypersetup{pdftitle={#3}}%
	\setcounter{section}{\numexpr #2 - 1}%
	\section{#3}%
	\fancyhead[R]{\truncate{0.73\textwidth}{#2 : #3}}%
}

\providecommand{\development}[3]{%
	\title{#3}%
	\hypersetup{pdftitle={#3}}%
	\section*{#3}%
	\fancyhead[R]{\truncate{0.73\textwidth}{#3}}%
}

\providecommand{\summary}[1]{%
	\textit{#1}%
	\medskip%
}

\tikzset{notestyleraw/.append style={inner sep=0pt, rounded corners=0pt, align=center}}

%\newcommand{\booklink}[1]{\website/bibliographie\##1}
\newcommand{\citelink}[2]{\hyperlink{cite.\therefsection @#1}{#2}}
\newcommand{\previousreference}{}
\providecommand{\reference}[2][]{%
	\notblank{#1}{\renewcommand{\previousreference}{#1}}{}%
	\todo[noline]{%
		\protect\vspace{16pt}%
		\protect\par%
		\protect\notblank{#1}{\cite{[\previousreference]}\\}{}%
		\protect\citelink{\previousreference}{p. #2}%
	}%
}

\definecolor{devcolor}{HTML}{00695c}
\newcommand{\dev}[1]{%
	\reversemarginpar%
	\todo[noline]{
		\protect\vspace{16pt}%
		\protect\par%
		\bfseries\color{devcolor}\href{\website/developpements/#1}{DEV}
	}%
	\normalmarginpar%
}

% En-têtes :

\pagestyle{fancy}
\fancyhead[L]{\truncate{0.23\textwidth}{\thepage}}
\fancyfoot[C]{\scriptsize \href{\website}{\texttt{agreg.skyost.eu}}}

% Couleurs :

\definecolor{property}{HTML}{fffde7}
\definecolor{proposition}{HTML}{fff8e1}
\definecolor{lemma}{HTML}{fff3e0}
\definecolor{theorem}{HTML}{fce4f2}
\definecolor{corollary}{HTML}{ffebee}
\definecolor{definition}{HTML}{ede7f6}
\definecolor{notation}{HTML}{f3e5f5}
\definecolor{example}{HTML}{e0f7fa}
\definecolor{cexample}{HTML}{efebe9}
\definecolor{application}{HTML}{e0f2f1}
\definecolor{remark}{HTML}{e8f5e9}
\definecolor{proof}{HTML}{e1f5fe}

% Théorèmes :

\theoremstyle{definition}
\newtheorem{theorem}{Théorème}

\newtheorem{property}[theorem]{Propriété}
\newtheorem{proposition}[theorem]{Proposition}
\newtheorem{lemma}[theorem]{Lemme}
\newtheorem{corollary}[theorem]{Corollaire}

\newtheorem{definition}[theorem]{Définition}
\newtheorem{notation}[theorem]{Notation}

\newtheorem{example}[theorem]{Exemple}
\newtheorem{cexample}[theorem]{Contre-exemple}
\newtheorem{application}[theorem]{Application}

\theoremstyle{remark}
\newtheorem{remark}[theorem]{Remarque}

\counterwithin*{theorem}{section}

\newcommand{\applystyletotheorem}[1]{
	\tcolorboxenvironment{#1}{
		enhanced,
		breakable,
		colback=#1!98!white,
		boxrule=0pt,
		boxsep=0pt,
		left=8pt,
		right=8pt,
		top=8pt,
		bottom=8pt,
		sharp corners,
		after=\par,
	}
}

\applystyletotheorem{property}
\applystyletotheorem{proposition}
\applystyletotheorem{lemma}
\applystyletotheorem{theorem}
\applystyletotheorem{corollary}
\applystyletotheorem{definition}
\applystyletotheorem{notation}
\applystyletotheorem{example}
\applystyletotheorem{cexample}
\applystyletotheorem{application}
\applystyletotheorem{remark}
\applystyletotheorem{proof}

% Environnements :

\NewEnviron{whitetabularx}[1]{%
	\renewcommand{\arraystretch}{2.5}
	\colorbox{white}{%
		\begin{tabularx}{\textwidth}{#1}%
			\BODY%
		\end{tabularx}%
	}%
}

% Maths :

\DeclareFontEncoding{FMS}{}{}
\DeclareFontSubstitution{FMS}{futm}{m}{n}
\DeclareFontEncoding{FMX}{}{}
\DeclareFontSubstitution{FMX}{futm}{m}{n}
\DeclareSymbolFont{fouriersymbols}{FMS}{futm}{m}{n}
\DeclareSymbolFont{fourierlargesymbols}{FMX}{futm}{m}{n}
\DeclareMathDelimiter{\VERT}{\mathord}{fouriersymbols}{152}{fourierlargesymbols}{147}



\begin{document}
	%<*content>
	\lesson{analysis}{230}{Séries de nombres réels ou complexes. Comportement des restes ou des sommes partielles des séries numériques. Exemples.}
	
	\subsection{Séries réelles et complexes}
	
	\subsubsection{Notion de série et convergence}
	
	Soit $\mathbb{K} = \mathbb{R}$ ou $\mathbb{C}$. Muni de sa norme usuelle $\mid . \mid$, $\mathbb{K}$ est un espace de Banach.
	
	\reference[GOU20]{208}
	
	\begin{definition}
		Soit $(u_n)$ une suite à valeurs dans $\mathbb{K}$.
		\begin{itemize}
			\item On appelle \textbf{série} de terme général $u_n$ la suite $(S_n)$ définie par
			\[ \forall n \in \mathbb{N}, \, S_n = u_0 + \dots + u_n \]
			On note cette série $\sum u_n$.
			\item $u_n$ s'appelle le \textbf{terme} d'indice $n$.
			\item $S_n$ s'appelle la \textbf{somme partielle} d'indice $n$.
		\end{itemize}
	\end{definition}
	
	\begin{definition}
		En reprenant les notations précédentes, on dit que $\sum u_n$ \textbf{converge} si la suite $(S_n)$ converge. Dans ce cas, la limite s'appelle la \textbf{somme} de la série, et on la note $\sum_{n=0}^{+\infty} u_n$.
	\end{definition}
	
	\begin{definition}
		On appelle \textbf{reste} d'ordre $n$ d'une série convergente $\sum u_n$ l'élément $R_n$ défini par
		\[ R_n = \sum_{k=0}^{+\infty} u_k - \sum_{k=0}^{n} u_k = \sum_{k=n+1}^{+\infty} u_k \]
	\end{definition}
	
	\begin{example}
		Soit $q \in \mathbb{C}$. Alors $\sum q^n \text{ converge} \iff |q| < 1$. Dans ce cas :
		\begin{itemize}
			\item La somme partielle d'indice $n$ est égale à $\frac{1-q^{n+1}}{1-q}$.
			\item La somme de la série est égale à $\frac{1}{1-q}$.
			\item Le reste d'ordre $n$ de $\sum q^n$ est égal à $\frac{q^n}{1-q}$.
		\end{itemize}
	\end{example}
	
	\reference[AMR11]{81}
	
	\begin{proposition}
		Si $\sum u_n$ converge, alors $\lim_{n \rightarrow +\infty} u_n = 0$.
	\end{proposition}
	
	\begin{cexample}
		La réciproque est fausse, par exemple en considérant la suite $(u_n)$ définie pour tout $n \in \mathbb{N}$ par $u_n = \ln(1 + \frac{1}{n})$, on a $\sum_{k=1}^{n} u_k = \ln(n+1) \longrightarrow_{n \rightarrow +\infty} +\infty$.
	\end{cexample}
	
	\begin{proposition}
		Muni des opérations :
		\begin{itemize}
			\item $\forall (u_n), (v_n) \in \mathbb{K}^{\mathbb{N}}, \, \sum u_n + \sum v_n = \sum (u_n + v_n)$,
			\item $\forall \lambda \in \mathbb{K}, \, \forall (u_n) \in \mathbb{K}^{\mathbb{N}}, \, \lambda \sum u_n = \sum (\lambda u_n)$,
		\end{itemize}
		l'ensemble des séries numériques est un espace vectoriel sur $\mathbb{K}$ dont l'ensemble des séries convergentes est un sous-espace vectoriel.
	\end{proposition}
	
	\reference[GOU20]{209}
	
	\begin{proposition}[Critère de Cauchy pour les séries]
		Une série $\sum u_n$ converge si et seulement si
		\[ \forall \epsilon > 0, \, \exists N \in \mathbb{N}, \, \forall n \geq N, \, \forall p \in \mathbb{N}, \, \left| \sum_{k=0}^{p} u_{n+k} \right| < \epsilon \]
	\end{proposition}
	
	\begin{definition}
		On dit que $\sum u_n$ est \textbf{absolument convergente} si $\sum |u_n|$ est convergente.
	\end{definition}
	
	\begin{theorem}
		Tout série à valeurs dans $\mathbb{K}$ absolument convergente est convergente.
	\end{theorem}
	
	Ce dernier théorème justifie de s'intéresser plus particulièrement aux sommes à termes positifs.
	
	\subsubsection{Séries à termes positifs}
	
	\paragraph{Comparaison}
	
	\begin{proposition}
		Une série à termes positifs converge si et seulement si la suite des sommes partielles est majorée.
	\end{proposition}
	
	\begin{corollary}
		On considère deux séries réelles $\sum u_n$ et $\sum v_n$ telles que $\forall n \in \mathbb{N}, \, 0 \leq u_n \leq v_n$. Alors :
		\begin{enumerate}[(i)]
			\item Si $\sum v_n$ converge, $\sum u_n$ converge.
			\item Si $\sum u_n$ diverge, $\sum v_n$ diverge.
		\end{enumerate}
	\end{corollary}
	
	\begin{proposition}
		On considère deux séries $\sum u_n$ et $\sum v_n$ à termes positifs.
		\begin{enumerate}[(i)]
			\item Si $v_n = O(u_n)$ et si $\sum u_n$ converge, alors $\sum v_n$ converge.
			\item Si $u_n \sim v_n$, alors les séries $\sum u_n$ et $\sum v_n$ sont de même nature.
			\begin{itemize}
				\item En cas de convergence, les restes vérifient $\sum_{k=n}^{+\infty} u_k \sim \sum_{k=n}^{+\infty} v_k$.
				\item En cas de divergence, les sommes partielles vérifient $\sum_{k=0}^{n} u_k \sim \sum_{k=0}^{n} v_k$.
			\end{itemize}
		\end{enumerate}
	\end{proposition}
	
	\reference{219}
	
	\begin{application}[Formule de Stirling]
		\[ \exists k > 0 \text{ tel que } n! \sim k \sqrt{n} \left( \frac{n}{e} \right)^n \]
	\end{application}
	
	\reference{228}
	
	\begin{application}[Développement asymptotique de la suite des sinus itérés]
		Soit $(u_n)$ une suite vérifiant
		\[ u_0 \in \left] 0, \frac{\pi}{2} \right], \quad \text{ et } \quad \forall n \in \mathbb{N}, u_{n+1} = \sin(u_n) \]
		Alors
		\[ u_n = \sqrt{\frac{3}{n}} - \frac{3\sqrt{3}}{10} \frac{\ln(n)}{n\sqrt{n}} + o \left( \frac{\ln(n)}{n\sqrt{n}} \right) \]
	\end{application}
	
	\reference{211}
	
	\begin{application}[Développement asymptotique de la série harmonique]
		\[ \sum_{k=1}^n \frac{1}{k} = \ln(n) + \gamma + \frac{1}{2n} + o \left( \frac{1}{2n} \right) \]
		où $\gamma$ désigne la constante d'Euler.
	\end{application}
	
	\begin{proposition}[Comparaison série - intégrale]
		Soit $f : \mathbb{R}^+ \rightarrow \mathbb{R}^+$ une fonction positive, continue par morceaux et décroissante sur $\mathbb{R}^+$. Alors la suite $(U_n)$ définie par
		\[ \forall n \in \mathbb{N}, \, \sum_{k=0}^n f(k) - \int_0^n f(t) \, \mathrm{d}t \]
		est convergente. En particulier, la série $\sum f(n)$ et l'intégrale $\int_0^{+\infty} f(t) \, \mathrm{d}t$ sont de même nature.
	\end{proposition}
	
	\begin{example}
		La série de Riemann $\sum \frac{1}{n^\alpha}$ converge si et seulement si $\alpha > 1$.
	\end{example}
	
	\begin{example}
		La série de Bertrand $\sum \frac{1}{n^\alpha \ln(n)^\beta}$ converge si et seulement si $\alpha > 1$ ou si $\alpha = 1$ et $\beta > 1$.
	\end{example}
	
	\reference[AMR11]{109}
	
	\begin{proposition}
		Soit $\sum f(n)$ une série relevant d'une comparaison série - intégrale. On note $R_n$ le reste d'ordre $n$ de cette série. Alors,
		\[ \forall n \geq 1, \, \int_{n+1}^{+\infty} f(t) \, \mathrm{d}t \leq |R_n| \leq \int_{n}^{+\infty} f(t) \, \mathrm{d}t \]
	\end{proposition}
	
	\begin{example}
		La somme $\sum_{n=1}^{20} \frac{1}{n^3}$ donne une approximation de $\zeta(3) = \sum_{n=1}^{+\infty} \frac{1}{n^3}$ à moins de $125 \times 10^{-5}$ près.
	\end{example}
	
	\reference[GOU20]{213}
	
	\begin{proposition}
		Soient deux séries réelles $\sum u_n$ et $\sum v_n$ à termes strictement positifs telles que $\frac{u_{n+1}}{u_n} \geq \frac{v_{n+1}}{v_n}$ à partir d'un certain rang. Alors :
		\begin{enumerate}[(i)]
			\item Si $\sum u_n$ converge, $\sum v_n$ converge.
			\item Si $\sum v_n$ diverge, $\sum u_n$ diverge.
		\end{enumerate}
	\end{proposition}
	
	\paragraph{Critères}
	
	\begin{proposition}[Règle de d'Alembert]
		Soit $\sum u_n$ une série à termes strictement positifs telle que
		\[ \lim_{n \rightarrow +\infty} \frac{u_{n+1}}{u_n} = \lambda \in [0, +\infty] \]
		Alors :
		\begin{enumerate}[(i)]
			\item Si $\lambda < 1$, $\sum u_n$ converge.
			\item Si $\lambda > 1$, $\sum u_n$ diverge.
		\end{enumerate}
	\end{proposition}
	
	\reference[AMR11]{94}
	
	\begin{example}
		$\sum \left( 1 - \frac{1}{n} \right)^{n^2}$ converge.
	\end{example}
	
	\reference{108}
	
	\begin{proposition}
		Soit $\sum u_n$ une série relevant de la règle de D'Alembert. On note $R_n$ le reste d'ordre $n$ de cette série. Alors il existe $N \in \mathbb{N}$ et $\alpha \in ]0,1[$ tels que
		\[ \forall n \geq N, \, |R_n| \leq \frac{\alpha}{1-\alpha} \]
	\end{proposition}
	
	\begin{example}
		$\sum_{k=0}^{10} \frac{1}{n!}$ donne une valeur approchée de $e$ à moins de $3 \times 10^{-8}$ près par défaut.
	\end{example}
	
	\reference[GOU20]{214}
	
	\begin{proposition}[Règle de Cauchy]
		Soit $\sum u_n$ une série à termes strictement positifs telle que
		\[ \lim_{n \rightarrow +\infty} \sqrt[n]{u_n} = \lambda \in [0, +\infty] \]
		Alors :
		\begin{enumerate}[(i)]
			\item Si $\lambda < 1$, $\sum u_n$ converge.
			\item Si $\lambda > 1$, $\sum u_n$ diverge.
		\end{enumerate}
	\end{proposition}
	
	\reference[AMR11]{112}
	
	\begin{example}
		$\sum \left( \frac{4n+1}{3n+2} \right)^{n}$ converge.
	\end{example}
	
	\reference{107}
	
	\begin{proposition}
		Soit $\sum u_n$ une série relevant de la règle de Cauchy. On note $R_n$ le reste d'ordre $n$ de cette série. Alors il existe $N \in \mathbb{N}$ et $\alpha \in ]0,1[$ tels que
		\[ \forall n \geq N, \, |R_n| \leq \frac{\alpha^{n+1}}{1-\alpha} \]
	\end{proposition}
	
	\begin{example}
		En reprenant les notations précédentes, pour $u_n = n^{-n}$, on a $R_4 < 0,00035$.
	\end{example}
	
	\subsubsection{Séries semi-convergentes}
	
	\reference{214}
	
	\begin{definition}
		On appelle \textbf{séries semi-convergentes} les séries convergentes mais non absolument convergentes.
	\end{definition}
	
	\begin{theorem}[Critère de Leibniz]
		Soit $(a_n)$ une suite à termes positifs, décroissantes, tendant vers $0$. Alors
		\[ \sum (-1)^n a_n \text{ converge} \quad \text{ et } \quad \forall n \in \mathbb{N}, \, |R_n| = \left| \sum_{k=n+1}^{+\infty} (-1)^k a_k \right| \leq a_{n+1} \]
	\end{theorem}
	
	\reference[AMR11]{97}
	
	\begin{example}
		La série $\sum (-1)^{n-1} n^{-\alpha}$ est convergente pour $\alpha > 0$. De plus, les restes $R_n$ vérifient
		\[ |R_n| \leq \frac{1}{(n+1)^\alpha} \]
	\end{example}
	
	\reference[GOU20]{215}
	
	\begin{proposition}[Transformation d'Abel]
		Soit une série $\sum u_n$ où $\forall n \in \mathbb{N}, \, u_n = \alpha_n v_n$. On note $\forall n \in \mathbb{N}, \, S_n = \sum_{k=0}^n v_k$. Alors,
		\[ \sum_{k=0}^n u_k = \alpha_n S_n + \sum_{k=0}^{n-1} (\alpha_k - \alpha_{k+1}) S_k \]
	\end{proposition}
	
	\reference[AMR11]{99}
	
	\begin{corollary}[Critère d'Abel]
		Soit une série $\sum u_n$ où $\forall n \in \mathbb{N}, \, u_n = \alpha_n v_n$. On suppose :
		\begin{itemize}
			\item $(\alpha_n)$ est une suite réelle positive, décroissante et qui tend vers $0$.
			\item La série $\sum v_n$ est bornée par une constante $M$.
		\end{itemize}
		Alors $\sum u_n$ est convergente, et les restes $R_n$ vérifient $\forall n \in \mathbb{N}, \, |R_n| \leq M a_{n+1}$.
	\end{corollary}
	
	\reference[GOU20]{216}
	
	\begin{remark}
		En reprenant les notations précédentes, avec $v_n = (-1)^n$, on retrouve le critère de Leibniz. 
	\end{remark}
	
	\begin{example}
		La série $\sum \frac{e^{ni\theta}}{n^\alpha}$ converge pour tout $\alpha > 0, \theta \in \mathbb{R} \setminus 2 \pi \mathbb{Z}$.
	\end{example}
	
	\subsection{Calcul de sommes}
	
	\subsubsection{Séries de Fourier}
	
	\begin{definition}
		Soit $f : \mathbb{R} \rightarrow \mathbb{C}$ une application $2\pi$-périodique et continue par morceaux sur $\mathbb{R}$. On appelle \textbf{coefficients de Fourier} de $f$ les nombres complexes définis par
		\[ \forall n \in \mathbb{Z}, \, c_n(f) = \int_{0}^{2\pi} f(t) e^{-int} \, \mathrm{d}t \]
		La \textbf{série de Fourier} associée à $f$ est
		\[ \sum_{n \in \mathbb{Z}} c_n(f)e^{inx} \]
	\end{definition}
	
	\begin{theorem}[Parseval]
		Soit $f : \mathbb{R} \rightarrow \mathbb{C}$ une application $2\pi$-périodique et continue par morceaux sur $\mathbb{R}$. Alors la série de Fourier de $f$ est convergente et,
		\[ \sum_{-\infty}^{+\infty} |c_n(f)|^2 = \frac{1}{2\pi} \int_0^{2\pi} |f(t)| \, \mathrm{d}t \]
	\end{theorem}
	
	\begin{example}
		Avec $f : x \mapsto 1 - \frac{x^2}{\pi^2}$, on obtient $\sum_{n=1}^{+\infty} \frac{1}{n^4} = \frac{\pi^4}{90}$.
	\end{example}
	
	\begin{theorem}[Jordan-Dirichlet]
		Soit $f : \mathbb{R} \rightarrow \mathbb{C}$ une application $2\pi$-périodique et $\mathcal{C}^1$ par morceaux sur $\mathbb{R}$. Alors la série de Fourier de $f$ est convergente en tout point $x \in \mathbb{R}$ et sa somme en ce point vaut
		\[ \frac{f(x^+) + f(x^-)}{2} \]
	\end{theorem}
	
	\begin{example}
		Toujours avec $f : x \mapsto 1 - \frac{x^2}{\pi^2}$, on obtient $\sum_{n=1}^{+\infty} \frac{1}{n^2} = \frac{\pi^2}{6}$.
	\end{example}
	
	\subsubsection{Séries entières}
	
	\reference{247}
	
	\begin{definition}
		On appelle \textbf{série entière} toute série de fonctions de la forme $\sum a_n z^n$ où $z$ est une variable complexe et où $(a_n)$ est une suite complexe.
	\end{definition}
	
	\begin{lemma}[Abel]
		Soient $\sum a_n z^n$ une série entière et $z_0 \in \mathbb{C}$ tels que $(a_n z_0^n)$ soit bornée. Alors :
		\begin{enumerate}[(i)]
			\item $\forall z \in \mathbb{C}$ tel que $|z| < |z_0|$, $\sum a_n z^n$ converge absolument.
			\item $\forall r \in ]0,z_0[, \, \sum a_n z^n$ converge normalement dans $\overline{D}(0, r) = \{ z \in \mathbb{C} \mid |z| \leq r \}$.
		\end{enumerate}
	\end{lemma}
	
	\begin{definition}
		Soit $\sum a_n z^n$ une série entière. Le nombre
		\[ R = \sup \{ r \geq 0 \mid (|a_n|r^n) \text{ est bornée} \} \]
		est le \textbf{rayon de convergence} de $\sum a_n z^n$. On a :
		\begin{itemize}
			\item $\forall z \in \mathbb{C}$ tel que $|z| < R$, $\sum a_n z^n$ converge absolument.
			\item $\forall z \in \mathbb{C}$ tel que $|z| > R$, $\sum a_n z^n$ diverge.
			\item $\forall r \in [0,R[$, $\sum a_n z^n$ converge normalement sur $\overline{D}(0,r)$.
		\end{itemize}
		Le disque $D(0,R)$ est le \textbf{disque de convergence} de la série, le cercle $C(0,R)$ est le \textbf{cercle d'incertitude}.
	\end{definition}
	
	\begin{example}
		$\sum \frac{z^n}{n!}$ est une série entière de rayon de convergence infini.
	\end{example}
	
	\reference[GOU21]{314}
	\dev{nombres-de-bell}
	
	\begin{theorem}[Nombres de Bell]
		Pour tout $n \in \mathbb{N}^*$, on note $B_n$ le nombre de partitions de $\llbracket 1, n \rrbracket$. Par convention on pose $B_0 = 1$. Alors,
		\[ \forall k \in \mathbb{N}^*, \, B_k = \frac{1}{e} \sum_{n=0}^{+\infty} \frac{n^k}{n!} \]
	\end{theorem}

	\reference[GOU20]{263}
	\dev{theoreme-d-abel-angulaire}
	
	\begin{theorem}[Abel angulaire]
		\label{230-1}
		Soit $\sum a_n z^n$ une série entière de rayon de convergence supérieur ou égal à $1$ tel que $\sum a_n$ converge. On note $f$ la somme de cette série sur le disque unité $D$ de $\mathbb{C}$. On fixe $\theta_0 \in \left[ 0, \frac{\pi}{2} \right[$ et on pose $\Delta_{\theta_0} = \{ z \in D \mid \exists \rho > 0 \text{ et } \exists \theta \in [-\theta_0, \theta_0] \text{ tels que } z = 1 - \rho e^{i\theta} \}$.
		\begin{center}
			\begin{tikzpicture}
				\draw[->] (-3, 0) -- (3, 0) node[right] {$x$};
				\draw[->] (0, -3) -- (0, 3) node[above] {$y$};
				\draw (0,2) node {$\bullet$} node[above right]{$1$};
				\draw (2,0) node {$\bullet$} node[below right]{$1$};
				\draw (0,0) circle (2);
				\coordinate (A) at (130:3.5);
				\coordinate (B) at (230:3.5);
				\coordinate (C) at (2,0);
				\begin{scope}
					\path[clip] circle (2);
					\path[clip] (A) -- (B) -- (C) -- cycle;
					\draw [transparent, fill=blue!30, fill opacity=0.3] (C) circle (9);
				\end{scope}
				\begin{scope}
					\path[clip] (A) -- (180:3.5) -- (C) -- cycle;
					\draw (C) circle (1);
				\end{scope}
				\draw (0.7,0.35) node {$\theta_0$};
				\draw (C) -- (A);
				\draw (C) -- (B);
				\coordinate (S) at (210:3.5);
				\coordinate (E) at (-0.5,-0.5);
				\draw [->] (S) to [out=50] (E);
				\draw (212:3.7) node {$\Delta_{\theta_0}$};
			\end{tikzpicture}
		\end{center}
		Alors $\lim_{\substack{z \rightarrow 1 \\ z \in \Delta_{\theta_0}}} f(z) = \sum_{n=0}^{+\infty} a_n$.
	\end{theorem}
	
	\begin{application}
		\[ \sum_{n=0}^{+\infty} \frac{(-1)^n}{(2n+1)} = \frac{\pi}{4} \]
	\end{application}
	
	\begin{application}
		\[ \sum_{n=0}^{+\infty} \frac{(-1)^{n-1}}{n} = \ln(2) \]
	\end{application}
	
	\begin{theorem}[Taubérien faible]
		Soit $\sum a_n z^n$ une série entière de rayon de convergence $1$. On note $f$ la somme de cette série sur $D(0,1)$. On suppose que
		\[ \exists S \in \mathbb{C} \text{ tel que } \lim_{\substack{x \rightarrow 1 \\ x < 1}} f(x) = S \]
		Si $a_n = o \left( \frac{1}{n} \right)$, alors $\sum a_n$ converge et $\sum_{n=0}^{+\infty} a_n = S$.
	\end{theorem}
	
	\begin{remark}
		Ce dernier résultat est une réciproque partielle du \cref{230-1}. Il reste vrai en supposant $a_n = O \left( \frac{1}{n} \right)$ (c'est le théorème Taubérien fort).
	\end{remark}
	%</content>
\end{document}