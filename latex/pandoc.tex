% Théorèmes :

\theoremstyle{definition}
\newtheorem{theorem}{Théorème}

\newtheorem{property}[theorem]{Propriété}
\newtheorem{proposition}[theorem]{Proposition}
\newtheorem{lemma}[theorem]{Lemme}
\newtheorem{corollary}[theorem]{Corollaire}

\newtheorem{definition}[theorem]{Définition}
\newtheorem{notation}[theorem]{Notation}

\newtheorem{example}[theorem]{Exemple}
\newtheorem{cexample}[theorem]{Contre-exemple}
\newtheorem{application}[theorem]{Application}

\theoremstyle{remark}
\newtheorem{remark}[theorem]{Remarque}

\counterwithin*{theorem}{subsection}

% Commandes :

\newcommand{\lesson}[3]{%
	\begin{docname}%
		\section*{#3}%
	\end{docname}%
	\begin{doccategories}%
		#1%
	\end{doccategories}%
}

\newcommand{\development}[3]{%
	\begin{docname}%
		\section*{#3}%
	\end{docname}%
	\begin{doccategories}%
		#1%
	\end{doccategories}%
}

\newcommand{\summary}[1]{%
	\begin{docsummary}%
		#1%
	\end{docsummary}%
	\newpar%
}

\newcommand{\reference}[2][]{%
	\begin{bookref}%
		\textbf{[#1]} \\ p. #2%
	\end{bookref}%
}

\newcommand{\includelatexpicture}[1]{%
	\begin{center}%
		\includegraphics{/images/latex/#1.svg}%
	\end{center}%
}

\renewcommand{\newpar}{


}

% Environnements :

\renewenvironment{whitetabularx}[1]{\begin{tabularx}{\textwidth}{#1}}{\end{tabularx}}
