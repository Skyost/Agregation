\documentclass[12pt, a4paper]{report}

% LuaLaTeX :

\RequirePackage{iftex}
\RequireLuaTeX

% Packages :

\usepackage[french]{babel}
%\usepackage[utf8]{inputenc}
%\usepackage[T1]{fontenc}
\usepackage[pdfencoding=auto, pdfauthor={Hugo Delaunay}, pdfsubject={Mathématiques}, pdfcreator={agreg.skyost.eu}]{hyperref}
\usepackage{amsmath}
\usepackage{amsthm}
%\usepackage{amssymb}
\usepackage{stmaryrd}
\usepackage{tikz}
\usepackage{tkz-euclide}
\usepackage{fourier-otf}
\usepackage{fontspec}
\usepackage{titlesec}
\usepackage{fancyhdr}
\usepackage{catchfilebetweentags}
\usepackage[french, capitalise, noabbrev]{cleveref}
\usepackage[fit, breakall]{truncate}
\usepackage[top=2.5cm, right=2cm, bottom=2.5cm, left=2cm]{geometry}
\usepackage{enumerate}
\usepackage{tocloft}
\usepackage{microtype}
%\usepackage{mdframed}
%\usepackage{thmtools}
\usepackage{xcolor}
\usepackage{tabularx}
\usepackage{aligned-overset}
\usepackage[subpreambles=true]{standalone}
\usepackage{environ}
\usepackage[normalem]{ulem}
\usepackage{marginnote}
\usepackage{etoolbox}
\usepackage{setspace}
\usepackage[bibstyle=reading, citestyle=draft]{biblatex}
\usepackage{xpatch}
\usepackage[many, breakable]{tcolorbox}
\usepackage[backgroundcolor=white, bordercolor=white, textsize=small]{todonotes}

% Bibliographie :

\newcommand{\overridebibliographypath}[1]{\providecommand{\bibliographypath}{#1}}
\overridebibliographypath{../bibliography.bib}
\addbibresource{\bibliographypath}
\defbibheading{bibliography}[\bibname]{%
	\newpage
	\section*{#1}%
}
\renewbibmacro*{entryhead:full}{\printfield{labeltitle}}
\DeclareFieldFormat{url}{\newline\footnotesize\url{#1}}
\AtEndDocument{\printbibliography}

% Police :

\setmathfont{Erewhon Math}

% Tikz :

\usetikzlibrary{calc}

% Longueurs :

\setlength{\parindent}{0pt}
\setlength{\headheight}{15pt}
\setlength{\fboxsep}{0pt}
\titlespacing*{\chapter}{0pt}{-20pt}{10pt}
\setlength{\marginparwidth}{1.5cm}
\setstretch{1.1}

% Métadonnées :

\author{agreg.skyost.eu}
\date{\today}

% Titres :

\setcounter{secnumdepth}{3}

\renewcommand{\thechapter}{\Roman{chapter}}
\renewcommand{\thesubsection}{\Roman{subsection}}
\renewcommand{\thesubsubsection}{\arabic{subsubsection}}
\renewcommand{\theparagraph}{\alph{paragraph}}

\titleformat{\chapter}{\huge\bfseries}{\thechapter}{20pt}{\huge\bfseries}
\titleformat*{\section}{\LARGE\bfseries}
\titleformat{\subsection}{\Large\bfseries}{\thesubsection \, - \,}{0pt}{\Large\bfseries}
\titleformat{\subsubsection}{\large\bfseries}{\thesubsubsection. \,}{0pt}{\large\bfseries}
\titleformat{\paragraph}{\bfseries}{\theparagraph. \,}{0pt}{\bfseries}

\setcounter{secnumdepth}{4}

% Table des matières :

\renewcommand{\cftsecleader}{\cftdotfill{\cftdotsep}}
\addtolength{\cftsecnumwidth}{10pt}

% Redéfinition des commandes :

\renewcommand*\thesection{\arabic{section}}
\renewcommand{\ker}{\mathrm{Ker}}

% Nouvelles commandes :

\newcommand{\website}{https://agreg.skyost.eu}

\newcommand{\tr}[1]{\mathstrut ^t #1}
\newcommand{\im}{\mathrm{Im}}
\newcommand{\rang}{\operatorname{rang}}
\newcommand{\trace}{\operatorname{trace}}
\newcommand{\id}{\operatorname{id}}
\newcommand{\stab}{\operatorname{Stab}}

\providecommand{\newpar}{\\[\medskipamount]}

\providecommand{\lesson}[3]{%
	\title{#3}%
	\hypersetup{pdftitle={#3}}%
	\setcounter{section}{\numexpr #2 - 1}%
	\section{#3}%
	\fancyhead[R]{\truncate{0.73\textwidth}{#2 : #3}}%
}

\providecommand{\development}[3]{%
	\title{#3}%
	\hypersetup{pdftitle={#3}}%
	\section*{#3}%
	\fancyhead[R]{\truncate{0.73\textwidth}{#3}}%
}

\providecommand{\summary}[1]{%
	\textit{#1}%
	\medskip%
}

\tikzset{notestyleraw/.append style={inner sep=0pt, rounded corners=0pt, align=center}}

%\newcommand{\booklink}[1]{\website/bibliographie\##1}
\newcommand{\citelink}[2]{\hyperlink{cite.\therefsection @#1}{#2}}
\newcommand{\previousreference}{}
\providecommand{\reference}[2][]{%
	\notblank{#1}{\renewcommand{\previousreference}{#1}}{}%
	\todo[noline]{%
		\protect\vspace{16pt}%
		\protect\par%
		\protect\notblank{#1}{\cite{[\previousreference]}\\}{}%
		\protect\citelink{\previousreference}{p. #2}%
	}%
}

\definecolor{devcolor}{HTML}{00695c}
\newcommand{\dev}[1]{%
	\reversemarginpar%
	\todo[noline]{
		\protect\vspace{16pt}%
		\protect\par%
		\bfseries\color{devcolor}\href{\website/developpements/#1}{DEV}
	}%
	\normalmarginpar%
}

% En-têtes :

\pagestyle{fancy}
\fancyhead[L]{\truncate{0.23\textwidth}{\thepage}}
\fancyfoot[C]{\scriptsize \href{\website}{\texttt{agreg.skyost.eu}}}

% Couleurs :

\definecolor{property}{HTML}{fffde7}
\definecolor{proposition}{HTML}{fff8e1}
\definecolor{lemma}{HTML}{fff3e0}
\definecolor{theorem}{HTML}{fce4f2}
\definecolor{corollary}{HTML}{ffebee}
\definecolor{definition}{HTML}{ede7f6}
\definecolor{notation}{HTML}{f3e5f5}
\definecolor{example}{HTML}{e0f7fa}
\definecolor{cexample}{HTML}{efebe9}
\definecolor{application}{HTML}{e0f2f1}
\definecolor{remark}{HTML}{e8f5e9}
\definecolor{proof}{HTML}{e1f5fe}

% Théorèmes :

\theoremstyle{definition}
\newtheorem{theorem}{Théorème}

\newtheorem{property}[theorem]{Propriété}
\newtheorem{proposition}[theorem]{Proposition}
\newtheorem{lemma}[theorem]{Lemme}
\newtheorem{corollary}[theorem]{Corollaire}

\newtheorem{definition}[theorem]{Définition}
\newtheorem{notation}[theorem]{Notation}

\newtheorem{example}[theorem]{Exemple}
\newtheorem{cexample}[theorem]{Contre-exemple}
\newtheorem{application}[theorem]{Application}

\theoremstyle{remark}
\newtheorem{remark}[theorem]{Remarque}

\counterwithin*{theorem}{section}

\newcommand{\applystyletotheorem}[1]{
	\tcolorboxenvironment{#1}{
		enhanced,
		breakable,
		colback=#1!98!white,
		boxrule=0pt,
		boxsep=0pt,
		left=8pt,
		right=8pt,
		top=8pt,
		bottom=8pt,
		sharp corners,
		after=\par,
	}
}

\applystyletotheorem{property}
\applystyletotheorem{proposition}
\applystyletotheorem{lemma}
\applystyletotheorem{theorem}
\applystyletotheorem{corollary}
\applystyletotheorem{definition}
\applystyletotheorem{notation}
\applystyletotheorem{example}
\applystyletotheorem{cexample}
\applystyletotheorem{application}
\applystyletotheorem{remark}
\applystyletotheorem{proof}

% Environnements :

\NewEnviron{whitetabularx}[1]{%
	\renewcommand{\arraystretch}{2.5}
	\colorbox{white}{%
		\begin{tabularx}{\textwidth}{#1}%
			\BODY%
		\end{tabularx}%
	}%
}

% Maths :

\DeclareFontEncoding{FMS}{}{}
\DeclareFontSubstitution{FMS}{futm}{m}{n}
\DeclareFontEncoding{FMX}{}{}
\DeclareFontSubstitution{FMX}{futm}{m}{n}
\DeclareSymbolFont{fouriersymbols}{FMS}{futm}{m}{n}
\DeclareSymbolFont{fourierlargesymbols}{FMX}{futm}{m}{n}
\DeclareMathDelimiter{\VERT}{\mathord}{fouriersymbols}{152}{fourierlargesymbols}{147}



\begin{document}
	%<*content>
	\development{algebra}{theoreme-des-deux-carres-fermat}{Théorème des deux carrés de Fermat}

	\summary{Nous démontrons le théorème des deux carrés de Fermat (qui donne des conditions sur la décomposition en facteurs premiers d'un entier pour que celui-ci soit somme de deux carrés) à l'aide de l'anneau des entiers de Gauss $\mathbb{Z}[i]$.}

	\reference[I-P]{137}

	\begin{lemma}
		\label{theoreme-des-deux-carres-fermat-1}
		Soit $p \geq 3$ un nombre premier. Alors $x \in \mathbb{F}^*_p$ est un carré si et seulement si $x^{\frac{p-1}{2}} = 1$.
	\end{lemma}

	\begin{demonstration}
		On pose $X = \{ x \in \mathbb{F}_p \mid x^{\frac{p-1}{2}} = 1 \}$, et on note $S$ l'ensemble des carrés de $\mathbb{F}_p^*$. Comme un polynôme de degré $d$ sur $\mathbb{F}_p$ possède au plus $d$ racines, on a $|X| \leq \deg \left( X^{\frac{p-1}{2}} - 1 \right) = \frac{p-1}{2}$.
		\newpar
		D'autre part, si $x \in S$, on peut écrire $x = y^2$ et on a donc $x^{\frac{p-1}{2}} = y^{p-1} = 1$ car $|\mathbb{F}_p^*| = p-1$. Donc, $S \subset X$.
		\newpar
		Pour conclure, calculons le cardinal de $S$. Pour cela, considérons le morphisme
		\[
		\begin{array}{cl}
			\mathbb{F}_p^* &\rightarrow S \\
			x &\mapsto x^2
		\end{array}
		\]
		dont le noyau est $\{ x \in \mathbb{F}_p^* \mid x^2 = 1 \} = \{ \pm 1 \}$ qui est de cardinal $2$. En appliquant le premier théorème d'isomorphisme, et en considérant les cardinaux ; on obtient $|S| = \frac{p-1}{2}$. Donc $S = X$.
	\end{demonstration}

	Introduisons maintenant des notations qui seront utiles pour la suite.

	\begin{notation}
		On note \[ N :
		\begin{array}{cl}
			\mathbb{Z}[i] &\rightarrow \mathbb{N} \\
			a+ib &\mapsto a^2 + b^2
		\end{array}
		\] et $\Sigma$ l'ensemble des entiers qui sont somme de deux carrés.
	\end{notation}

	\begin{remark}
		\label{theoreme-des-deux-carres-fermat-2}
		$n \in \Sigma \iff \exists z \in \mathbb{Z}[i] \text{ tel que } N(z)=n$.
	\end{remark}

	\begin{lemma}
		\label{theoreme-des-deux-carres-fermat-3}
		Voici quelques propriétés sur $N$ et $\mathbb{Z}[i]$ dont nous aurons besoin :
		\begin{enumerate}[(i)]
			\item $N$ est multiplicative.
			\item $\mathbb{Z}[i]^* = \{ z \in \mathbb{Z}[i] \mid N(z) = 1 \} = \{ \pm 1, \pm i \}$.
			\item $\mathbb{Z}[i]$ est euclidien de stathme $N$.
		\end{enumerate}
	\end{lemma}

	\begin{demonstration}
		\begin{enumerate}[(i)]
			\item On a $\forall z, z' \in \mathbb{C}$, $|zz'|^2 = |z|^2 |z'|^2$ (par multiplicativité de $(.)^2$ et de $|.|$). Et $N$ n'est que la restriction de $|.|^2$ à $\mathbb{Z}[i]$. Il est également tout-à-fait possible de montrer cette propriété par un calcul direct.
			\item Soit $z \in \mathbb{Z}[i]^*$. On a $N(z)N(z^{-1}) = N(zz^{-1}) = N(1) = 1$. Comme $N$ est à valeurs dans $\mathbb{N}$, on a $N(z) = N(z^{-1}) = 1$. En écrivant $z = a+ib$, on a $N(z) = a^2 + b^2 = 1$, d'où $a = \pm 1$ ou $b = \pm 1$. Réciproquement, $\pm 1$ et $\pm i$ sont bien inversibles dans $\mathbb{Z}[i]$ et de module $1$.
			\item Soient $z, t \in \mathbb{Z}[i]$. On pose $\frac{z}{t} = x + iy \in \mathbb{C}$ avec $x, y \in \mathbb{R}$. Soient $a, b \in \mathbb{Z}$ tels que :
			\begin{itemize}
				\item $|x-a| \leq \frac{1}{2}$.
				\item $|y-b| \leq \frac{1}{2}$.
			\end{itemize}
			(Ces nombres existent bien, ne pas hésiter à faire un dessin pour s'en convaincre.) On pose $q = a+ib \in \mathbb{Z}[i]$, et on a
			\[ \left| \frac{z}{t} - q \right| = (x-a)^2 + (y-b)^2 \leq \frac{1}{4} + \frac{1}{4} < 1 \]
			On pose alors $r = z-qt$, et on a bien
			\[ z = tq+r \text{ et } N(r) = r^2 = |t^2| \left| \frac{z}{t} - q^2 \right| < |t|^2 = N(t) \]
		\end{enumerate}
	\end{demonstration}

	\begin{lemma}
		\label{theoreme-des-deux-carres-fermat-4}
		Soit $p$ un nombre premier. Si $p$ n'est pas irréductible dans $\mathbb{Z}[i]$, alors $p \in \Sigma$.
	\end{lemma}

	\begin{demonstration}
		On suppose que $p$ n'est pas irréductible dans $\mathbb{Z}[i]$. On peut donc écrire $p = uv$ avec $u, v \in \mathbb{Z}[i]$ non inversibles. Ainsi,
		\[ p^2 = N(p) = N(uv) = \underbrace{N(u)}_{\neq 1} \underbrace{N(v)}_{\neq 1} \overset{p \text{ premier}}{\implies} N(u) = N(v) = p \]
		Par la \cref{theoreme-des-deux-carres-fermat-2}, $p \in \Sigma$.
	\end{demonstration}

	\begin{theorem}[Deux carrés de Fermat]
		Soit $n \in \mathbb{N}^*$. Alors $n \in \Sigma$ si et seulement si $v_p(n)$ est pair pour tout $p$ premier tel que $p \equiv 3 \mod 4$ (où $v_p(n)$ désigne la valuation $p$-adique de $n$).
	\end{theorem}

	\begin{demonstration}
		\uline{Sens direct :} On écrit $n = a^2 + b^2$ avec $a, b \in \mathbb{Z}$. Soit $p \mid n$ tel que $p \equiv 3 \mod 4$. Montrons que $p \notin \Sigma$. On suppose par l'absurde que l'on peut écrire $p = c^2 + d^2$ avec $c, d \in \mathbb{Z}$. On va discerner les cas :
		\begin{itemize}
			\item Si $c \equiv \pm 1 \mod 4$, alors $c^2 \equiv 1 \mod 4$ (et de même pour $d^2$).
			\item Si $c \equiv \pm 2 \mod 4$, alors $c^2 \equiv 0 \mod 4$ (et de même pour $d^2$).
		\end{itemize}
		Donc $p = c^2 + d^2 \equiv 0, 1 \text{ ou } 2 \mod 4$ : absurde. En particulier, par le \cref{theoreme-des-deux-carres-fermat-4} (en prenant la contraposée), $p$ est irréductible dans $\mathbb{Z}[i]$. Comme $\mathbb{Z}[i]$ est euclidien (cf. \cref{theoreme-des-deux-carres-fermat-3}), $p$ est un élément premier de $\mathbb{Z}[i]$. Mais, $p \mid n = (a+ib)(a-ib)$. Donc $p \mid a+ib$ ou $p \mid a-ib$.
		Dans les deux cas, on a $p \mid a$ et $p \mid b$. Ainsi,
		\[ \left( \frac{a}{p} \right)^2 + \left( \frac{b}{p} \right)^2 = \frac{n}{p^2} \]
		donc de deux choses l'une ; on a :
		\[ p^2 \mid n \text{ et } \frac{n}{p^2} \in \Sigma \]
		Il suffit alors d'itérer le processus (en remplaçant $n$ par $\frac{n}{p^2}$) $k$ fois jusqu'à ce que $p$ ne divise plus $\frac{n}{p^{2k}}$. On a alors $n = p^{2k} u$ avec $p \nmid u$. D'où $v_p(n) = 2k$.
		\newpar
		\uline{Réciproque :} Soit $p$ premier tel que $p \equiv 3 \mod 4$. Alors $p^{v_p(n)} = \left( p^{\frac{v_p(n)}{2}} \right)^2$ est un carré, donc $p^{v_p(n)} \in \Sigma$.
		\newpar
		Soit maintenant $p$ premier tel que $p = 2$ ou $p \equiv 1  \mod 4$. Alors en conséquence du \cref{theoreme-des-deux-carres-fermat-1} (le cas $p = 2$ étant trivial), $-1$ est un carré de $\mathbb{F}_p$ ie. $\exists a \in \mathbb{Z}$ tel que $-1 \equiv a^2 \mod p$. Donc $p \mid a^2 + 1 = (a-i)(a+i)$. Oui mais, $p$ ne divise ni $a-i$, ni $a+i$. Donc $p$ n'est pas un élément premier de $\mathbb{Z}[i]$ et n'est donc pas irréductible dans $\mathbb{Z}[i]$ (toujours parce que $\mathbb{Z}[i]$ est euclidien, cf. \cref{theoreme-des-deux-carres-fermat-3}). En vertu du \cref{theoreme-des-deux-carres-fermat-4}, $p \in \Sigma$.
		\newpar
		Comme $N$ est multiplicative, par la \cref{theoreme-des-deux-carres-fermat-2}, on en déduit que $\Sigma$ est stable par multiplication. Donc $n \in \Sigma$ (en décomposant $n$ en produit de facteurs premiers).
	\end{demonstration}

	\reference[PER]{48}

	\begin{remark}
		Le fait qu'un élément irréductible d'un anneau euclidien est premier est une conséquence directe du lemme d'Euclide, vrai dans les anneaux factoriels (donc à fortiori aussi dans les anneaux euclidiens).
	\end{remark}
	%</content>
\end{document}
