\documentclass[12pt, a4paper]{report}

% LuaLaTeX :

\RequirePackage{iftex}
\RequireLuaTeX

% Packages :

\usepackage[french]{babel}
%\usepackage[utf8]{inputenc}
%\usepackage[T1]{fontenc}
\usepackage[pdfencoding=auto, pdfauthor={Hugo Delaunay}, pdfsubject={Mathématiques}, pdfcreator={agreg.skyost.eu}]{hyperref}
\usepackage{amsmath}
\usepackage{amsthm}
%\usepackage{amssymb}
\usepackage{stmaryrd}
\usepackage{tikz}
\usepackage{tkz-euclide}
\usepackage{fourier-otf}
\usepackage{fontspec}
\usepackage{titlesec}
\usepackage{fancyhdr}
\usepackage{catchfilebetweentags}
\usepackage[french, capitalise, noabbrev]{cleveref}
\usepackage[fit, breakall]{truncate}
\usepackage[top=2.5cm, right=2cm, bottom=2.5cm, left=2cm]{geometry}
\usepackage{enumerate}
\usepackage{tocloft}
\usepackage{microtype}
%\usepackage{mdframed}
%\usepackage{thmtools}
\usepackage{xcolor}
\usepackage{tabularx}
\usepackage{aligned-overset}
\usepackage[subpreambles=true]{standalone}
\usepackage{environ}
\usepackage[normalem]{ulem}
\usepackage{marginnote}
\usepackage{etoolbox}
\usepackage{setspace}
\usepackage[bibstyle=reading, citestyle=draft]{biblatex}
\usepackage{xpatch}
\usepackage[many, breakable]{tcolorbox}
\usepackage[backgroundcolor=white, bordercolor=white, textsize=small]{todonotes}

% Bibliographie :

\newcommand{\overridebibliographypath}[1]{\providecommand{\bibliographypath}{#1}}
\overridebibliographypath{../bibliography.bib}
\addbibresource{\bibliographypath}
\defbibheading{bibliography}[\bibname]{%
	\newpage
	\section*{#1}%
}
\renewbibmacro*{entryhead:full}{\printfield{labeltitle}}
\DeclareFieldFormat{url}{\newline\footnotesize\url{#1}}
\AtEndDocument{\printbibliography}

% Police :

\setmathfont{Erewhon Math}

% Tikz :

\usetikzlibrary{calc}

% Longueurs :

\setlength{\parindent}{0pt}
\setlength{\headheight}{15pt}
\setlength{\fboxsep}{0pt}
\titlespacing*{\chapter}{0pt}{-20pt}{10pt}
\setlength{\marginparwidth}{1.5cm}
\setstretch{1.1}

% Métadonnées :

\author{agreg.skyost.eu}
\date{\today}

% Titres :

\setcounter{secnumdepth}{3}

\renewcommand{\thechapter}{\Roman{chapter}}
\renewcommand{\thesubsection}{\Roman{subsection}}
\renewcommand{\thesubsubsection}{\arabic{subsubsection}}
\renewcommand{\theparagraph}{\alph{paragraph}}

\titleformat{\chapter}{\huge\bfseries}{\thechapter}{20pt}{\huge\bfseries}
\titleformat*{\section}{\LARGE\bfseries}
\titleformat{\subsection}{\Large\bfseries}{\thesubsection \, - \,}{0pt}{\Large\bfseries}
\titleformat{\subsubsection}{\large\bfseries}{\thesubsubsection. \,}{0pt}{\large\bfseries}
\titleformat{\paragraph}{\bfseries}{\theparagraph. \,}{0pt}{\bfseries}

\setcounter{secnumdepth}{4}

% Table des matières :

\renewcommand{\cftsecleader}{\cftdotfill{\cftdotsep}}
\addtolength{\cftsecnumwidth}{10pt}

% Redéfinition des commandes :

\renewcommand*\thesection{\arabic{section}}
\renewcommand{\ker}{\mathrm{Ker}}

% Nouvelles commandes :

\newcommand{\website}{https://agreg.skyost.eu}

\newcommand{\tr}[1]{\mathstrut ^t #1}
\newcommand{\im}{\mathrm{Im}}
\newcommand{\rang}{\operatorname{rang}}
\newcommand{\trace}{\operatorname{trace}}
\newcommand{\id}{\operatorname{id}}
\newcommand{\stab}{\operatorname{Stab}}

\providecommand{\newpar}{\\[\medskipamount]}

\providecommand{\lesson}[3]{%
	\title{#3}%
	\hypersetup{pdftitle={#3}}%
	\setcounter{section}{\numexpr #2 - 1}%
	\section{#3}%
	\fancyhead[R]{\truncate{0.73\textwidth}{#2 : #3}}%
}

\providecommand{\development}[3]{%
	\title{#3}%
	\hypersetup{pdftitle={#3}}%
	\section*{#3}%
	\fancyhead[R]{\truncate{0.73\textwidth}{#3}}%
}

\providecommand{\summary}[1]{%
	\textit{#1}%
	\medskip%
}

\tikzset{notestyleraw/.append style={inner sep=0pt, rounded corners=0pt, align=center}}

%\newcommand{\booklink}[1]{\website/bibliographie\##1}
\newcommand{\citelink}[2]{\hyperlink{cite.\therefsection @#1}{#2}}
\newcommand{\previousreference}{}
\providecommand{\reference}[2][]{%
	\notblank{#1}{\renewcommand{\previousreference}{#1}}{}%
	\todo[noline]{%
		\protect\vspace{16pt}%
		\protect\par%
		\protect\notblank{#1}{\cite{[\previousreference]}\\}{}%
		\protect\citelink{\previousreference}{p. #2}%
	}%
}

\definecolor{devcolor}{HTML}{00695c}
\newcommand{\dev}[1]{%
	\reversemarginpar%
	\todo[noline]{
		\protect\vspace{16pt}%
		\protect\par%
		\bfseries\color{devcolor}\href{\website/developpements/#1}{DEV}
	}%
	\normalmarginpar%
}

% En-têtes :

\pagestyle{fancy}
\fancyhead[L]{\truncate{0.23\textwidth}{\thepage}}
\fancyfoot[C]{\scriptsize \href{\website}{\texttt{agreg.skyost.eu}}}

% Couleurs :

\definecolor{property}{HTML}{fffde7}
\definecolor{proposition}{HTML}{fff8e1}
\definecolor{lemma}{HTML}{fff3e0}
\definecolor{theorem}{HTML}{fce4f2}
\definecolor{corollary}{HTML}{ffebee}
\definecolor{definition}{HTML}{ede7f6}
\definecolor{notation}{HTML}{f3e5f5}
\definecolor{example}{HTML}{e0f7fa}
\definecolor{cexample}{HTML}{efebe9}
\definecolor{application}{HTML}{e0f2f1}
\definecolor{remark}{HTML}{e8f5e9}
\definecolor{proof}{HTML}{e1f5fe}

% Théorèmes :

\theoremstyle{definition}
\newtheorem{theorem}{Théorème}

\newtheorem{property}[theorem]{Propriété}
\newtheorem{proposition}[theorem]{Proposition}
\newtheorem{lemma}[theorem]{Lemme}
\newtheorem{corollary}[theorem]{Corollaire}

\newtheorem{definition}[theorem]{Définition}
\newtheorem{notation}[theorem]{Notation}

\newtheorem{example}[theorem]{Exemple}
\newtheorem{cexample}[theorem]{Contre-exemple}
\newtheorem{application}[theorem]{Application}

\theoremstyle{remark}
\newtheorem{remark}[theorem]{Remarque}

\counterwithin*{theorem}{section}

\newcommand{\applystyletotheorem}[1]{
	\tcolorboxenvironment{#1}{
		enhanced,
		breakable,
		colback=#1!98!white,
		boxrule=0pt,
		boxsep=0pt,
		left=8pt,
		right=8pt,
		top=8pt,
		bottom=8pt,
		sharp corners,
		after=\par,
	}
}

\applystyletotheorem{property}
\applystyletotheorem{proposition}
\applystyletotheorem{lemma}
\applystyletotheorem{theorem}
\applystyletotheorem{corollary}
\applystyletotheorem{definition}
\applystyletotheorem{notation}
\applystyletotheorem{example}
\applystyletotheorem{cexample}
\applystyletotheorem{application}
\applystyletotheorem{remark}
\applystyletotheorem{proof}

% Environnements :

\NewEnviron{whitetabularx}[1]{%
	\renewcommand{\arraystretch}{2.5}
	\colorbox{white}{%
		\begin{tabularx}{\textwidth}{#1}%
			\BODY%
		\end{tabularx}%
	}%
}

% Maths :

\DeclareFontEncoding{FMS}{}{}
\DeclareFontSubstitution{FMS}{futm}{m}{n}
\DeclareFontEncoding{FMX}{}{}
\DeclareFontSubstitution{FMX}{futm}{m}{n}
\DeclareSymbolFont{fouriersymbols}{FMS}{futm}{m}{n}
\DeclareSymbolFont{fourierlargesymbols}{FMX}{futm}{m}{n}
\DeclareMathDelimiter{\VERT}{\mathord}{fouriersymbols}{152}{fourierlargesymbols}{147}


% Bibliographie :

\addbibresource{\bibliographypath}%
\defbibheading{bibliography}[\bibname]{%
	\newpage
	\section*{#1}%
}
\renewbibmacro*{entryhead:full}{\printfield{labeltitle}}%
\DeclareFieldFormat{url}{\newline\footnotesize\url{#1}}%

\AtEndDocument{\printbibliography}

\begin{document}
	%<*content>
	\development{algebra}{theoreme-de-maschke}{Théorème de Maschke}

	\summary{Dans ce développement, nous montrons le théorème de Maschke qui dit que toute représentation linéaire de degré non-nul est somme directe d'un nombre fini de représentations linéaires irréductibles.}

	\reference[SER]{18}

	Soit $G$ un groupe fini de cardinal $r$. Tous les espaces vectoriels considérés ici sont de dimension finie.

	\medskip

	\begin{lemma}
		\label{theoreme-de-maschke-1}
		Soit $\rho : G \rightarrow \mathrm{GL}(V)$ une représentation de $G$ et soit $W$ un sous-espace de $V$ stable par $\rho(g)$ pour tout $g \in G$. Alors il existe un supplémentaire de $W$ dans $V$ stable par $\rho(g)$ pour tout $g \in G$.
	\end{lemma}

	\begin{proof}
		Soit $p : V \mapsto W$ une projection de $V$ sur $W$. Formons la moyenne $p_0$ des transformés de $p$ par les éléments de $G$ :
		\[ p_0 = \frac{1}{r} \sum_{g \in G} \rho(g) p \rho(g)^{-1} \]
		Puisque $W$ est stable par $\rho(g)$ pour tout $g \in G$, on a $\mathrm{Im}(p_0) \subset W$.
		\newpar
		D'autre part, si $x \in W$, on a $\rho(g)^{-1}(x) = \rho(g^{-1})(x) \in W$. D'où :
		\[ (p \circ \rho(g)^{-1})(x) = \rho(g)^{-1}(x) \implies (\rho(g) \circ p \circ \rho(g)^{-1})(x) = x \]
		D'où $p_0(x) = \frac{r}{r} x = x$. Donc $\mathrm{Im}(p_0) = W$ et $p_0^2 = p_0$ ie. $p_0$ est le projecteur de $V$ sur $W$ parallèlement au supplémentaire $W_0 = \ker(p_0)$ de $W$.
		\newpar
		Si l'on calcule $\rho(h) p_0 \rho(h)^{-1}$, on trouve :
		\[ \rho(h) p_0 \rho(h)^{-1} = \frac{1}{r} \sum_{g \in G} \rho(h) \rho(g) p \rho(g)^{-1} \rho(h)^{-1} = \frac{1}{r} \sum_{g \in G} \rho(hg) p \rho(hg)^{-1} = p_0 \]
		car $g \mapsto hg$ est une bijection de $G$ dans $G$. Donc on a :
		\[ \rho(h) p_0 = p_0 \rho(h) \]
		Si maintenant $x \in W_0$, on a $p_0(x) = 0$. D'où $\forall g \in G, \, (p_0 \circ \rho(g))(x) = (\rho(g) \circ p_0)(x) = 0$ ie. $\rho(g)(x) \in W_0$, ce que l'on voulait.
	\end{proof}

	\begin{theorem}[Maschke]
		Toute représentation linéaire de degré non-nul est somme directe d'un nombre fini de représentations linéaires irréductibles.
	\end{theorem}

	\begin{proof}
		Soit $\rho : G \rightarrow \mathrm{GL}(V)$ une représentation linéaire de $G$. On raisonne par récurrence sur $n = \dim(V)$.
		\begin{itemize}
			\item \uline{Si $n = 1$ :} la représentation $\rho$ est irréductible, donc le résultat est évident.
			\item \uline{Supposons le résultat vrai à un rang $n \geq 1$ et montrons-le au rang $n+1$.} Si $\rho$ est irréductible, il n'y a rien à montrer. Dans le cas contraire, on note $W$ le sous-espace de $V$ laissé stable par $\rho(g)$ pour tout $g \in G$. Par le \cref{theoreme-de-maschke-1}, il existe $W_0$ tel que $E = W \oplus W_0$ avec $W_0$ laissé stable par $\rho(g)$ pour tout $g \in G$. Par l'hypothèse de récurrence, $\rho_W : g \mapsto \rho(g)_{|W}$ et $\rho_{W_0} : g \mapsto \rho(g)_{|{W_0}}$ sont sommes directes de représentations irréductibles, et comme $\rho = \rho_W \oplus \rho_{W_0}$, on a le résultat.
		\end{itemize}
	\end{proof}
	%</content>
\end{document}
