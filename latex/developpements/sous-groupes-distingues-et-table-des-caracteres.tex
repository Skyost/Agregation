\documentclass[12pt, a4paper]{report}

% LuaLaTeX :

\RequirePackage{iftex}
\RequireLuaTeX

% Packages :

\usepackage[french]{babel}
%\usepackage[utf8]{inputenc}
%\usepackage[T1]{fontenc}
\usepackage[pdfencoding=auto, pdfauthor={Hugo Delaunay}, pdfsubject={Mathématiques}, pdfcreator={agreg.skyost.eu}]{hyperref}
\usepackage{amsmath}
\usepackage{amsthm}
%\usepackage{amssymb}
\usepackage{stmaryrd}
\usepackage{tikz}
\usepackage{tkz-euclide}
\usepackage{fourier-otf}
\usepackage{fontspec}
\usepackage{titlesec}
\usepackage{fancyhdr}
\usepackage{catchfilebetweentags}
\usepackage[french, capitalise, noabbrev]{cleveref}
\usepackage[fit, breakall]{truncate}
\usepackage[top=2.5cm, right=2cm, bottom=2.5cm, left=2cm]{geometry}
\usepackage{enumerate}
\usepackage{tocloft}
\usepackage{microtype}
%\usepackage{mdframed}
%\usepackage{thmtools}
\usepackage{xcolor}
\usepackage{tabularx}
\usepackage{aligned-overset}
\usepackage[subpreambles=true]{standalone}
\usepackage{environ}
\usepackage[normalem]{ulem}
\usepackage{marginnote}
\usepackage{etoolbox}
\usepackage{setspace}
\usepackage[bibstyle=reading, citestyle=draft]{biblatex}
\usepackage{xpatch}
\usepackage[many, breakable]{tcolorbox}
\usepackage[backgroundcolor=white, bordercolor=white, textsize=small]{todonotes}

% Bibliographie :

\newcommand{\overridebibliographypath}[1]{\providecommand{\bibliographypath}{#1}}
\overridebibliographypath{../bibliography.bib}
\addbibresource{\bibliographypath}
\defbibheading{bibliography}[\bibname]{%
	\newpage
	\section*{#1}%
}
\renewbibmacro*{entryhead:full}{\printfield{labeltitle}}
\DeclareFieldFormat{url}{\newline\footnotesize\url{#1}}
\AtEndDocument{\printbibliography}

% Police :

\setmathfont{Erewhon Math}

% Tikz :

\usetikzlibrary{calc}

% Longueurs :

\setlength{\parindent}{0pt}
\setlength{\headheight}{15pt}
\setlength{\fboxsep}{0pt}
\titlespacing*{\chapter}{0pt}{-20pt}{10pt}
\setlength{\marginparwidth}{1.5cm}
\setstretch{1.1}

% Métadonnées :

\author{agreg.skyost.eu}
\date{\today}

% Titres :

\setcounter{secnumdepth}{3}

\renewcommand{\thechapter}{\Roman{chapter}}
\renewcommand{\thesubsection}{\Roman{subsection}}
\renewcommand{\thesubsubsection}{\arabic{subsubsection}}
\renewcommand{\theparagraph}{\alph{paragraph}}

\titleformat{\chapter}{\huge\bfseries}{\thechapter}{20pt}{\huge\bfseries}
\titleformat*{\section}{\LARGE\bfseries}
\titleformat{\subsection}{\Large\bfseries}{\thesubsection \, - \,}{0pt}{\Large\bfseries}
\titleformat{\subsubsection}{\large\bfseries}{\thesubsubsection. \,}{0pt}{\large\bfseries}
\titleformat{\paragraph}{\bfseries}{\theparagraph. \,}{0pt}{\bfseries}

\setcounter{secnumdepth}{4}

% Table des matières :

\renewcommand{\cftsecleader}{\cftdotfill{\cftdotsep}}
\addtolength{\cftsecnumwidth}{10pt}

% Redéfinition des commandes :

\renewcommand*\thesection{\arabic{section}}
\renewcommand{\ker}{\mathrm{Ker}}

% Nouvelles commandes :

\newcommand{\website}{https://agreg.skyost.eu}

\newcommand{\tr}[1]{\mathstrut ^t #1}
\newcommand{\im}{\mathrm{Im}}
\newcommand{\rang}{\operatorname{rang}}
\newcommand{\trace}{\operatorname{trace}}
\newcommand{\id}{\operatorname{id}}
\newcommand{\stab}{\operatorname{Stab}}

\providecommand{\newpar}{\\[\medskipamount]}

\providecommand{\lesson}[3]{%
	\title{#3}%
	\hypersetup{pdftitle={#3}}%
	\setcounter{section}{\numexpr #2 - 1}%
	\section{#3}%
	\fancyhead[R]{\truncate{0.73\textwidth}{#2 : #3}}%
}

\providecommand{\development}[3]{%
	\title{#3}%
	\hypersetup{pdftitle={#3}}%
	\section*{#3}%
	\fancyhead[R]{\truncate{0.73\textwidth}{#3}}%
}

\providecommand{\summary}[1]{%
	\textit{#1}%
	\medskip%
}

\tikzset{notestyleraw/.append style={inner sep=0pt, rounded corners=0pt, align=center}}

%\newcommand{\booklink}[1]{\website/bibliographie\##1}
\newcommand{\citelink}[2]{\hyperlink{cite.\therefsection @#1}{#2}}
\newcommand{\previousreference}{}
\providecommand{\reference}[2][]{%
	\notblank{#1}{\renewcommand{\previousreference}{#1}}{}%
	\todo[noline]{%
		\protect\vspace{16pt}%
		\protect\par%
		\protect\notblank{#1}{\cite{[\previousreference]}\\}{}%
		\protect\citelink{\previousreference}{p. #2}%
	}%
}

\definecolor{devcolor}{HTML}{00695c}
\newcommand{\dev}[1]{%
	\reversemarginpar%
	\todo[noline]{
		\protect\vspace{16pt}%
		\protect\par%
		\bfseries\color{devcolor}\href{\website/developpements/#1}{DEV}
	}%
	\normalmarginpar%
}

% En-têtes :

\pagestyle{fancy}
\fancyhead[L]{\truncate{0.23\textwidth}{\thepage}}
\fancyfoot[C]{\scriptsize \href{\website}{\texttt{agreg.skyost.eu}}}

% Couleurs :

\definecolor{property}{HTML}{fffde7}
\definecolor{proposition}{HTML}{fff8e1}
\definecolor{lemma}{HTML}{fff3e0}
\definecolor{theorem}{HTML}{fce4f2}
\definecolor{corollary}{HTML}{ffebee}
\definecolor{definition}{HTML}{ede7f6}
\definecolor{notation}{HTML}{f3e5f5}
\definecolor{example}{HTML}{e0f7fa}
\definecolor{cexample}{HTML}{efebe9}
\definecolor{application}{HTML}{e0f2f1}
\definecolor{remark}{HTML}{e8f5e9}
\definecolor{proof}{HTML}{e1f5fe}

% Théorèmes :

\theoremstyle{definition}
\newtheorem{theorem}{Théorème}

\newtheorem{property}[theorem]{Propriété}
\newtheorem{proposition}[theorem]{Proposition}
\newtheorem{lemma}[theorem]{Lemme}
\newtheorem{corollary}[theorem]{Corollaire}

\newtheorem{definition}[theorem]{Définition}
\newtheorem{notation}[theorem]{Notation}

\newtheorem{example}[theorem]{Exemple}
\newtheorem{cexample}[theorem]{Contre-exemple}
\newtheorem{application}[theorem]{Application}

\theoremstyle{remark}
\newtheorem{remark}[theorem]{Remarque}

\counterwithin*{theorem}{section}

\newcommand{\applystyletotheorem}[1]{
	\tcolorboxenvironment{#1}{
		enhanced,
		breakable,
		colback=#1!98!white,
		boxrule=0pt,
		boxsep=0pt,
		left=8pt,
		right=8pt,
		top=8pt,
		bottom=8pt,
		sharp corners,
		after=\par,
	}
}

\applystyletotheorem{property}
\applystyletotheorem{proposition}
\applystyletotheorem{lemma}
\applystyletotheorem{theorem}
\applystyletotheorem{corollary}
\applystyletotheorem{definition}
\applystyletotheorem{notation}
\applystyletotheorem{example}
\applystyletotheorem{cexample}
\applystyletotheorem{application}
\applystyletotheorem{remark}
\applystyletotheorem{proof}

% Environnements :

\NewEnviron{whitetabularx}[1]{%
	\renewcommand{\arraystretch}{2.5}
	\colorbox{white}{%
		\begin{tabularx}{\textwidth}{#1}%
			\BODY%
		\end{tabularx}%
	}%
}

% Maths :

\DeclareFontEncoding{FMS}{}{}
\DeclareFontSubstitution{FMS}{futm}{m}{n}
\DeclareFontEncoding{FMX}{}{}
\DeclareFontSubstitution{FMX}{futm}{m}{n}
\DeclareSymbolFont{fouriersymbols}{FMS}{futm}{m}{n}
\DeclareSymbolFont{fourierlargesymbols}{FMX}{futm}{m}{n}
\DeclareMathDelimiter{\VERT}{\mathord}{fouriersymbols}{152}{fourierlargesymbols}{147}



\begin{document}
	%<*content>
	\development{algebra}{sous-groupes-distingues-et-table-des-caracteres}{Sous-groupes distingués et table des caractères}

	\summary{Dans ce développement, on montre que tout sous-groupe distingué d'un groupe fini s'écrit comme intersection de noyaux de caractères irréductibles. On utilise ensuite ce résultat pour donner un critère de simplicité.}

	\reference[I-P]{68}

	Soit $G$ un groupe fini.

	\begin{notation}
		\begin{itemize}
			\item On note $\rho_1, \dots, \rho_r$ les représentations irréductibles de $G$. On suppose que $\rho_1$ est la représentation triviale.
			\item On note $\chi_1, \dots, \chi_r$ les caractères respectifs de $\rho_1, \dots, \rho_r$.
		\end{itemize}
	\end{notation}

	\begin{theorem}
		\label{sous-groupes-distingues-et-table-des-caracteres-1}
		Les sous-groupes distingués de $G$ sont exactement les
		\[ \bigcap_{i \in I} \ker(\rho_i) \text{ où } I \in \mathcal{P}(\llbracket 1, r \rrbracket) \]
	\end{theorem}

	\begin{demonstration}
		Pour tout $i \in \llbracket 1, r \rrbracket$, le noyau de $\rho_i$ est clairement distingué et donc toute intersection de noyaux de représentations irréductibles l'est aussi. Montrons que ce sont en fait les seuls sous-groupes distingués de $G$. Soit $N \lhd G$. On note $\rho : G/N \rightarrow \mathrm{GL}(V)$ la représentation régulière de $G/N$, de degré $[G : N]$. On pose $\widetilde{\rho} : g \mapsto \rho(\pi_N(g))$ (où $\pi_N : G \rightarrow G/N$ désigne la projection canonique sur le quotient, qui est un morphisme de groupes car $N$ est distingué). On a alors :
		\[ \forall g, h \in G, \, \widetilde{\rho}(g) \in \mathrm{GL}(V) \text{ et } \widetilde{\rho}(gh) = \widetilde{\rho}(g) \widetilde{\rho}(h) \]
		donc $\widetilde{\rho}$ est une représentation de $G$. De plus, comme $\rho$ est injective, on a
		\[ \ker(\widetilde{\rho}) = \{ g \in G \mid \rho(\pi_N(g)) = \id_V \} = \{ g \in G \mid \pi_N(g) = e_{G/N} \} = N \]
		Comme $\widetilde{\rho}$ est une représentation de $G$, on peut la décomposer en somme directe de représentations irréductibles de $G$ (par le théorème de Maschke) :
		\[ \widetilde{\rho} = \bigoplus_{i \in I} \rho_i \text{ où } I \in \mathcal{P}(\llbracket 1, r \rrbracket) \text{ et } \forall i \in I, \, \rho_i : G \rightarrow \mathrm{GL}(W_i) \]
		Soit $g \in G$. Dans une base adaptée à la décomposition $V = \bigoplus_{i \in I} W_i$, $\widetilde{\rho}(g)$ s'écrit comme une matrice diagonale par blocs (où chaque bloc correspond à une sous-représentation irréductible). Ainsi,
		\[ g \in \ker(\widetilde{\rho}) \iff \forall i \in I, \, g \in \ker(\rho_i) \]
		d'où $N = \ker(\widetilde{\rho}) = \bigcap_{i \in I} \ker(\rho_i)$.
	\end{demonstration}

	\begin{corollary}
		$G$ est simple si et seulement si $\forall i \neq 1$, $\forall g \neq e_G$, $\chi_i(g) \neq \chi_i(e_G)$.
	\end{corollary}

	\begin{demonstration}
		\underLine{Sens direct :} Supposons $G$ simple. Pour $i \neq 1$, $\ker(\chi_i) = \ker(\rho_i) \lhd G$, donc
		\[ \ker(\chi_i) = G \text{ ou } \ker(\chi_i) = \{ e_G \} \]
		Supposons par l'absurde que que $\ker(\chi_i) = G$ et notons $n_i$ le degré de $\rho_i$ de sorte que $\chi_i(g) = \chi_i(e_G) = n_i$ pour tout $g \in G$. En particulier,
		\[ \langle \chi_1, \chi_i \rangle = \frac{1}{|G|} \sum_{g \in G} \overline{\chi_1(g)} \chi_i(g) = n_i \]
		Mais, $\chi_1$ et $\chi_i$ sont orthogonaux, donc $\langle \chi_1, \chi_i \rangle = 0$ : absurde. D'où $\ker(\chi_i) = \{ e_G \}$. Donc $\forall g \neq e_G$, $\chi_i(g) \neq \chi_i(e_G)$.
		\newpar
		\underLine{Réciproque :} Supposons que $\forall i \neq 1$, $\forall g \neq e_G$, $\chi_i(g) \neq \chi_i(e_G)$. Alors $\forall i \neq 1$, $\ker(\chi_i) = \{ e_G \}$. De plus, comme $\ker(\chi_1) = G$, toute intersection de ces sous-groupes est triviale. Ainsi, $G$ est simple par le \cref{sous-groupes-distingues-et-table-des-caracteres-1}.
	\end{demonstration}
	%</content>
\end{document}
