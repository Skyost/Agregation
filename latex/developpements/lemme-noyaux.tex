\documentclass[12pt, a4paper]{report}

% LuaLaTeX :

\RequirePackage{iftex}
\RequireLuaTeX

% Packages :

\usepackage[french]{babel}
%\usepackage[utf8]{inputenc}
%\usepackage[T1]{fontenc}
\usepackage[pdfencoding=auto, pdfauthor={Hugo Delaunay}, pdfsubject={Mathématiques}, pdfcreator={agreg.skyost.eu}]{hyperref}
\usepackage{amsmath}
\usepackage{amsthm}
%\usepackage{amssymb}
\usepackage{stmaryrd}
\usepackage{tikz}
\usepackage{tkz-euclide}
\usepackage{fourier-otf}
\usepackage{fontspec}
\usepackage{titlesec}
\usepackage{fancyhdr}
\usepackage{catchfilebetweentags}
\usepackage[french, capitalise, noabbrev]{cleveref}
\usepackage[fit, breakall]{truncate}
\usepackage[top=2.5cm, right=2cm, bottom=2.5cm, left=2cm]{geometry}
\usepackage{enumerate}
\usepackage{tocloft}
\usepackage{microtype}
%\usepackage{mdframed}
%\usepackage{thmtools}
\usepackage{xcolor}
\usepackage{tabularx}
\usepackage{aligned-overset}
\usepackage[subpreambles=true]{standalone}
\usepackage{environ}
\usepackage[normalem]{ulem}
\usepackage{marginnote}
\usepackage{etoolbox}
\usepackage{setspace}
\usepackage[bibstyle=reading, citestyle=draft]{biblatex}
\usepackage{xpatch}
\usepackage[many, breakable]{tcolorbox}
\usepackage[backgroundcolor=white, bordercolor=white, textsize=small]{todonotes}

% Bibliographie :

\newcommand{\overridebibliographypath}[1]{\providecommand{\bibliographypath}{#1}}
\overridebibliographypath{../bibliography.bib}
\addbibresource{\bibliographypath}
\defbibheading{bibliography}[\bibname]{%
	\newpage
	\section*{#1}%
}
\renewbibmacro*{entryhead:full}{\printfield{labeltitle}}
\DeclareFieldFormat{url}{\newline\footnotesize\url{#1}}
\AtEndDocument{\printbibliography}

% Police :

\setmathfont{Erewhon Math}

% Tikz :

\usetikzlibrary{calc}

% Longueurs :

\setlength{\parindent}{0pt}
\setlength{\headheight}{15pt}
\setlength{\fboxsep}{0pt}
\titlespacing*{\chapter}{0pt}{-20pt}{10pt}
\setlength{\marginparwidth}{1.5cm}
\setstretch{1.1}

% Métadonnées :

\author{agreg.skyost.eu}
\date{\today}

% Titres :

\setcounter{secnumdepth}{3}

\renewcommand{\thechapter}{\Roman{chapter}}
\renewcommand{\thesubsection}{\Roman{subsection}}
\renewcommand{\thesubsubsection}{\arabic{subsubsection}}
\renewcommand{\theparagraph}{\alph{paragraph}}

\titleformat{\chapter}{\huge\bfseries}{\thechapter}{20pt}{\huge\bfseries}
\titleformat*{\section}{\LARGE\bfseries}
\titleformat{\subsection}{\Large\bfseries}{\thesubsection \, - \,}{0pt}{\Large\bfseries}
\titleformat{\subsubsection}{\large\bfseries}{\thesubsubsection. \,}{0pt}{\large\bfseries}
\titleformat{\paragraph}{\bfseries}{\theparagraph. \,}{0pt}{\bfseries}

\setcounter{secnumdepth}{4}

% Table des matières :

\renewcommand{\cftsecleader}{\cftdotfill{\cftdotsep}}
\addtolength{\cftsecnumwidth}{10pt}

% Redéfinition des commandes :

\renewcommand*\thesection{\arabic{section}}
\renewcommand{\ker}{\mathrm{Ker}}

% Nouvelles commandes :

\newcommand{\website}{https://agreg.skyost.eu}

\newcommand{\tr}[1]{\mathstrut ^t #1}
\newcommand{\im}{\mathrm{Im}}
\newcommand{\rang}{\operatorname{rang}}
\newcommand{\trace}{\operatorname{trace}}
\newcommand{\id}{\operatorname{id}}
\newcommand{\stab}{\operatorname{Stab}}

\providecommand{\newpar}{\\[\medskipamount]}

\providecommand{\lesson}[3]{%
	\title{#3}%
	\hypersetup{pdftitle={#3}}%
	\setcounter{section}{\numexpr #2 - 1}%
	\section{#3}%
	\fancyhead[R]{\truncate{0.73\textwidth}{#2 : #3}}%
}

\providecommand{\development}[3]{%
	\title{#3}%
	\hypersetup{pdftitle={#3}}%
	\section*{#3}%
	\fancyhead[R]{\truncate{0.73\textwidth}{#3}}%
}

\providecommand{\summary}[1]{%
	\textit{#1}%
	\medskip%
}

\tikzset{notestyleraw/.append style={inner sep=0pt, rounded corners=0pt, align=center}}

%\newcommand{\booklink}[1]{\website/bibliographie\##1}
\newcommand{\citelink}[2]{\hyperlink{cite.\therefsection @#1}{#2}}
\newcommand{\previousreference}{}
\providecommand{\reference}[2][]{%
	\notblank{#1}{\renewcommand{\previousreference}{#1}}{}%
	\todo[noline]{%
		\protect\vspace{16pt}%
		\protect\par%
		\protect\notblank{#1}{\cite{[\previousreference]}\\}{}%
		\protect\citelink{\previousreference}{p. #2}%
	}%
}

\definecolor{devcolor}{HTML}{00695c}
\newcommand{\dev}[1]{%
	\reversemarginpar%
	\todo[noline]{
		\protect\vspace{16pt}%
		\protect\par%
		\bfseries\color{devcolor}\href{\website/developpements/#1}{DEV}
	}%
	\normalmarginpar%
}

% En-têtes :

\pagestyle{fancy}
\fancyhead[L]{\truncate{0.23\textwidth}{\thepage}}
\fancyfoot[C]{\scriptsize \href{\website}{\texttt{agreg.skyost.eu}}}

% Couleurs :

\definecolor{property}{HTML}{fffde7}
\definecolor{proposition}{HTML}{fff8e1}
\definecolor{lemma}{HTML}{fff3e0}
\definecolor{theorem}{HTML}{fce4f2}
\definecolor{corollary}{HTML}{ffebee}
\definecolor{definition}{HTML}{ede7f6}
\definecolor{notation}{HTML}{f3e5f5}
\definecolor{example}{HTML}{e0f7fa}
\definecolor{cexample}{HTML}{efebe9}
\definecolor{application}{HTML}{e0f2f1}
\definecolor{remark}{HTML}{e8f5e9}
\definecolor{proof}{HTML}{e1f5fe}

% Théorèmes :

\theoremstyle{definition}
\newtheorem{theorem}{Théorème}

\newtheorem{property}[theorem]{Propriété}
\newtheorem{proposition}[theorem]{Proposition}
\newtheorem{lemma}[theorem]{Lemme}
\newtheorem{corollary}[theorem]{Corollaire}

\newtheorem{definition}[theorem]{Définition}
\newtheorem{notation}[theorem]{Notation}

\newtheorem{example}[theorem]{Exemple}
\newtheorem{cexample}[theorem]{Contre-exemple}
\newtheorem{application}[theorem]{Application}

\theoremstyle{remark}
\newtheorem{remark}[theorem]{Remarque}

\counterwithin*{theorem}{section}

\newcommand{\applystyletotheorem}[1]{
	\tcolorboxenvironment{#1}{
		enhanced,
		breakable,
		colback=#1!98!white,
		boxrule=0pt,
		boxsep=0pt,
		left=8pt,
		right=8pt,
		top=8pt,
		bottom=8pt,
		sharp corners,
		after=\par,
	}
}

\applystyletotheorem{property}
\applystyletotheorem{proposition}
\applystyletotheorem{lemma}
\applystyletotheorem{theorem}
\applystyletotheorem{corollary}
\applystyletotheorem{definition}
\applystyletotheorem{notation}
\applystyletotheorem{example}
\applystyletotheorem{cexample}
\applystyletotheorem{application}
\applystyletotheorem{remark}
\applystyletotheorem{proof}

% Environnements :

\NewEnviron{whitetabularx}[1]{%
	\renewcommand{\arraystretch}{2.5}
	\colorbox{white}{%
		\begin{tabularx}{\textwidth}{#1}%
			\BODY%
		\end{tabularx}%
	}%
}

% Maths :

\DeclareFontEncoding{FMS}{}{}
\DeclareFontSubstitution{FMS}{futm}{m}{n}
\DeclareFontEncoding{FMX}{}{}
\DeclareFontSubstitution{FMX}{futm}{m}{n}
\DeclareSymbolFont{fouriersymbols}{FMS}{futm}{m}{n}
\DeclareSymbolFont{fourierlargesymbols}{FMX}{futm}{m}{n}
\DeclareMathDelimiter{\VERT}{\mathord}{fouriersymbols}{152}{fourierlargesymbols}{147}



\begin{document}
	%<*content>
	\development{algebra}{lemme-noyaux}{Lemme des noyaux}

	\summary{On montre par récurrence le lemme des noyaux pour un endomorphisme d'un espace vectoriel de dimension finie, et on applique ce résultat pour obtenir un critère de diagonalisation.}

	\reference{GOU21}{185}
	
	Soit $E$ un espace vectoriel de dimension finie $n \geq 1$ sur un corps commutatif $\mathbb{K}$.

	\begin{lemma}[Lemme des noyaux]
		\label{lemme-noyaux-1}
		Soient $f \in \mathcal{L}(E)$ et $P = P_1 \dots P_k \in \mathbb{K}[X]$ (les $P_i$ étant supposés premiers entre-eux deux-à-deux). Alors,
		\[ \ker{P(f)} = \bigoplus_{i = 1}^k \ker{P_i(f)} \]
	\end{lemma}

	\begin{demonstration}
		On procède par récurrence sur $k \geq 2$.
		\begin{itemize}
			\item \underline{Pour $k = 2$ :} par le théorème de Bézout, il existe $U, V \in \mathbb{K}[X]$ tels que $UP_1 + VP_2 = 1$. Donc,
			\[ \forall x \in E, \, (UP_1 + VP_2)(f)(x) = (U(f) \circ P_1(f))(x) + (V(f) \circ P_2(f))(x) = x \tag{$*$} \]
			Soit $x \in \ker{P_1(f)} \cap \ker{P_2(f)}$. On a :
			\[ x \overset{(*)}{=} (U(f) \circ P_1(f))(x) + (V(f) \circ P_2(f))(x) \overset{x \in \ker{P_1(f)} \cap \ker{P_2(f)}}{=} 0 \]
			Donc $\ker{P_1(f)} \cap \ker{P_2(f)} = \{ 0 \}$ : la somme est directe.
			\newpar
			Soit maintenant $x \in \ker{P(f)}$. Par calcul,
			\[ P_2(f)(UP_1(f)(x)) = (UP_1P_2)(f)(x) = (U(f) \circ P(f))(x) = 0 \]
			ie. $UP_1(f)(x) \in \ker{P_2(f)}$. De même, $VP_2(f)(x) \in \ker{P_1(f)}$. Par $(*)$, $x \in \ker{P_1(f)} + \ker{P_2(f)}$. Donc $\ker{P(f)} \subset \ker{P_1(f)} \oplus \ker{P_2(f)}$.
			\newpar
			Et si $x \in \ker{P_1(f)}$,
			\[ P(f)(x) = (P_1(f) \circ P_2(f))(x) = (P_2(f) \circ P_1(f))(x) = 0 \]
			donc $x \in \ker{P(f)}$ et $\ker{P_1}(f) \subset \ker{P(f)}$. De même, on montre que $\ker{P_2}(f) \subset \ker{P(f)}$. Comme $\ker{P(f)}$ est un espace vectoriel, on a bien l'inclusion réciproque.
			\item \underline{On suppose le résultat vrai à un rang $k \geq 2$.} Montrons qu'il reste vrai au rang $k+1$. Écrivons
			\[ P = Q_1Q_2 \text{ avec } Q_1 = P_1 \dots P_k, Q_2 = P_{k+1} \]
			Les polynômes $Q_1$ et $Q_2$ sont premiers entre-eux, donc le cas $k = 2$ permet d'obtenir :
			\begin{align*}
				\ker{P(f)} &= \ker{Q_1(f)} \oplus \ker{Q_2(f)} \\
				&= \left( \bigoplus_{i = 1}^k \ker{P_i(f)} \right) \oplus \ker{P_{k+1}(f)} \text{ par hypothèse de récurrence} \\
				&= \bigoplus_{i = 1}^{k+1} \ker{P_i(f)}
			\end{align*}
			ce que l'on voulait.
		\end{itemize}
	\end{demonstration}

	\begin{application}
		Soit $f \in \mathcal{L}(E)$. Alors $f$ est diagonalisable si et seulement s'il existe $P \in \mathbb{K}[X]$ scindé sur $\mathbb{K}$ à racines simples tel que $P(f) = 0$.
	\end{application}

	\begin{demonstration}
		\underline{Sens direct :} Soient $\lambda_1, \dots, \lambda_k$ les valeurs propres distinctes de $f$ et $E_{\lambda_1}, \dots, E_{\lambda_k}$ les sous-espaces propres correspondants. On pose
		\[ P = (X-\lambda_1) \dots (X-\lambda_k) \in \mathbb{K}[X] \]
		On peut appliquer le \cref{lemme-noyaux-1} :
		\begin{align*}
			\ker{P(f)} &= \bigoplus_{i = 1}^k \ker{f - \lambda_i \operatorname{id}_E} \\
			&= \bigoplus_{i = 1}^k E_{\lambda_i} \\
			\overset{f \text{ diagonalisable}}&{=} E
		\end{align*}
		donc $P(f) = 0$ (et $P$ est bien scindé à racines simples).
		\newpar
		\underline{Réciproque :} On écrit
		\[ P = \alpha (X-\lambda_1) \dots (X-\lambda_k) \]
		avec les $\lambda_i \in \mathbb{K}$ distincts et $\alpha \neq 0$. On peut encore appliquer \cref{lemme-noyaux-1} :
		\begin{align*}
			E &= \ker{P(f)} \\
			&= \bigoplus_{i = 1}^k \ker{f - \lambda_i \operatorname{id}_E} \tag{*}
		\end{align*}
		Notons $I = \{ i \in \llbracket 1, k \rrbracket \mid \ker{f - \lambda_i \operatorname{id}_E} \neq \{ 0 \} \}$. $\forall i \in I$, $\lambda_i$ est valeur propre de $f$ et $E_{\lambda_i} = \ker{f - \lambda_i \operatorname{id}_E}$ n'est autre que le sous-espace propre correspondant. Par $(*)$,
		\[ E = \bigoplus_{i \in I} E_{\lambda_i} \]
		donc $f$ est diagonalisable.
	\end{demonstration}
	%</content>
\end{document}
