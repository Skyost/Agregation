\documentclass[12pt, a4paper]{report}

% LuaLaTeX :

\RequirePackage{iftex}
\RequireLuaTeX

% Packages :

\usepackage[french]{babel}
%\usepackage[utf8]{inputenc}
%\usepackage[T1]{fontenc}
\usepackage[pdfencoding=auto, pdfauthor={Hugo Delaunay}, pdfsubject={Mathématiques}, pdfcreator={agreg.skyost.eu}]{hyperref}
\usepackage{amsmath}
\usepackage{amsthm}
%\usepackage{amssymb}
\usepackage{stmaryrd}
\usepackage{tikz}
\usepackage{tkz-euclide}
\usepackage{fourier-otf}
\usepackage{fontspec}
\usepackage{titlesec}
\usepackage{fancyhdr}
\usepackage{catchfilebetweentags}
\usepackage[french, capitalise, noabbrev]{cleveref}
\usepackage[fit, breakall]{truncate}
\usepackage[top=2.5cm, right=2cm, bottom=2.5cm, left=2cm]{geometry}
\usepackage{enumerate}
\usepackage{tocloft}
\usepackage{microtype}
%\usepackage{mdframed}
%\usepackage{thmtools}
\usepackage{xcolor}
\usepackage{tabularx}
\usepackage{aligned-overset}
\usepackage[subpreambles=true]{standalone}
\usepackage{environ}
\usepackage[normalem]{ulem}
\usepackage{marginnote}
\usepackage{etoolbox}
\usepackage{setspace}
\usepackage[bibstyle=reading, citestyle=draft]{biblatex}
\usepackage{xpatch}
\usepackage[many, breakable]{tcolorbox}
\usepackage[backgroundcolor=white, bordercolor=white, textsize=small]{todonotes}

% Bibliographie :

\newcommand{\overridebibliographypath}[1]{\providecommand{\bibliographypath}{#1}}
\overridebibliographypath{../bibliography.bib}
\addbibresource{\bibliographypath}
\defbibheading{bibliography}[\bibname]{%
	\newpage
	\section*{#1}%
}
\renewbibmacro*{entryhead:full}{\printfield{labeltitle}}
\DeclareFieldFormat{url}{\newline\footnotesize\url{#1}}
\AtEndDocument{\printbibliography}

% Police :

\setmathfont{Erewhon Math}

% Tikz :

\usetikzlibrary{calc}

% Longueurs :

\setlength{\parindent}{0pt}
\setlength{\headheight}{15pt}
\setlength{\fboxsep}{0pt}
\titlespacing*{\chapter}{0pt}{-20pt}{10pt}
\setlength{\marginparwidth}{1.5cm}
\setstretch{1.1}

% Métadonnées :

\author{agreg.skyost.eu}
\date{\today}

% Titres :

\setcounter{secnumdepth}{3}

\renewcommand{\thechapter}{\Roman{chapter}}
\renewcommand{\thesubsection}{\Roman{subsection}}
\renewcommand{\thesubsubsection}{\arabic{subsubsection}}
\renewcommand{\theparagraph}{\alph{paragraph}}

\titleformat{\chapter}{\huge\bfseries}{\thechapter}{20pt}{\huge\bfseries}
\titleformat*{\section}{\LARGE\bfseries}
\titleformat{\subsection}{\Large\bfseries}{\thesubsection \, - \,}{0pt}{\Large\bfseries}
\titleformat{\subsubsection}{\large\bfseries}{\thesubsubsection. \,}{0pt}{\large\bfseries}
\titleformat{\paragraph}{\bfseries}{\theparagraph. \,}{0pt}{\bfseries}

\setcounter{secnumdepth}{4}

% Table des matières :

\renewcommand{\cftsecleader}{\cftdotfill{\cftdotsep}}
\addtolength{\cftsecnumwidth}{10pt}

% Redéfinition des commandes :

\renewcommand*\thesection{\arabic{section}}
\renewcommand{\ker}{\mathrm{Ker}}

% Nouvelles commandes :

\newcommand{\website}{https://agreg.skyost.eu}

\newcommand{\tr}[1]{\mathstrut ^t #1}
\newcommand{\im}{\mathrm{Im}}
\newcommand{\rang}{\operatorname{rang}}
\newcommand{\trace}{\operatorname{trace}}
\newcommand{\id}{\operatorname{id}}
\newcommand{\stab}{\operatorname{Stab}}

\providecommand{\newpar}{\\[\medskipamount]}

\providecommand{\lesson}[3]{%
	\title{#3}%
	\hypersetup{pdftitle={#3}}%
	\setcounter{section}{\numexpr #2 - 1}%
	\section{#3}%
	\fancyhead[R]{\truncate{0.73\textwidth}{#2 : #3}}%
}

\providecommand{\development}[3]{%
	\title{#3}%
	\hypersetup{pdftitle={#3}}%
	\section*{#3}%
	\fancyhead[R]{\truncate{0.73\textwidth}{#3}}%
}

\providecommand{\summary}[1]{%
	\textit{#1}%
	\medskip%
}

\tikzset{notestyleraw/.append style={inner sep=0pt, rounded corners=0pt, align=center}}

%\newcommand{\booklink}[1]{\website/bibliographie\##1}
\newcommand{\citelink}[2]{\hyperlink{cite.\therefsection @#1}{#2}}
\newcommand{\previousreference}{}
\providecommand{\reference}[2][]{%
	\notblank{#1}{\renewcommand{\previousreference}{#1}}{}%
	\todo[noline]{%
		\protect\vspace{16pt}%
		\protect\par%
		\protect\notblank{#1}{\cite{[\previousreference]}\\}{}%
		\protect\citelink{\previousreference}{p. #2}%
	}%
}

\definecolor{devcolor}{HTML}{00695c}
\newcommand{\dev}[1]{%
	\reversemarginpar%
	\todo[noline]{
		\protect\vspace{16pt}%
		\protect\par%
		\bfseries\color{devcolor}\href{\website/developpements/#1}{DEV}
	}%
	\normalmarginpar%
}

% En-têtes :

\pagestyle{fancy}
\fancyhead[L]{\truncate{0.23\textwidth}{\thepage}}
\fancyfoot[C]{\scriptsize \href{\website}{\texttt{agreg.skyost.eu}}}

% Couleurs :

\definecolor{property}{HTML}{fffde7}
\definecolor{proposition}{HTML}{fff8e1}
\definecolor{lemma}{HTML}{fff3e0}
\definecolor{theorem}{HTML}{fce4f2}
\definecolor{corollary}{HTML}{ffebee}
\definecolor{definition}{HTML}{ede7f6}
\definecolor{notation}{HTML}{f3e5f5}
\definecolor{example}{HTML}{e0f7fa}
\definecolor{cexample}{HTML}{efebe9}
\definecolor{application}{HTML}{e0f2f1}
\definecolor{remark}{HTML}{e8f5e9}
\definecolor{proof}{HTML}{e1f5fe}

% Théorèmes :

\theoremstyle{definition}
\newtheorem{theorem}{Théorème}

\newtheorem{property}[theorem]{Propriété}
\newtheorem{proposition}[theorem]{Proposition}
\newtheorem{lemma}[theorem]{Lemme}
\newtheorem{corollary}[theorem]{Corollaire}

\newtheorem{definition}[theorem]{Définition}
\newtheorem{notation}[theorem]{Notation}

\newtheorem{example}[theorem]{Exemple}
\newtheorem{cexample}[theorem]{Contre-exemple}
\newtheorem{application}[theorem]{Application}

\theoremstyle{remark}
\newtheorem{remark}[theorem]{Remarque}

\counterwithin*{theorem}{section}

\newcommand{\applystyletotheorem}[1]{
	\tcolorboxenvironment{#1}{
		enhanced,
		breakable,
		colback=#1!98!white,
		boxrule=0pt,
		boxsep=0pt,
		left=8pt,
		right=8pt,
		top=8pt,
		bottom=8pt,
		sharp corners,
		after=\par,
	}
}

\applystyletotheorem{property}
\applystyletotheorem{proposition}
\applystyletotheorem{lemma}
\applystyletotheorem{theorem}
\applystyletotheorem{corollary}
\applystyletotheorem{definition}
\applystyletotheorem{notation}
\applystyletotheorem{example}
\applystyletotheorem{cexample}
\applystyletotheorem{application}
\applystyletotheorem{remark}
\applystyletotheorem{proof}

% Environnements :

\NewEnviron{whitetabularx}[1]{%
	\renewcommand{\arraystretch}{2.5}
	\colorbox{white}{%
		\begin{tabularx}{\textwidth}{#1}%
			\BODY%
		\end{tabularx}%
	}%
}

% Maths :

\DeclareFontEncoding{FMS}{}{}
\DeclareFontSubstitution{FMS}{futm}{m}{n}
\DeclareFontEncoding{FMX}{}{}
\DeclareFontSubstitution{FMX}{futm}{m}{n}
\DeclareSymbolFont{fouriersymbols}{FMS}{futm}{m}{n}
\DeclareSymbolFont{fourierlargesymbols}{FMX}{futm}{m}{n}
\DeclareMathDelimiter{\VERT}{\mathord}{fouriersymbols}{152}{fourierlargesymbols}{147}



\begin{document}
	%<*content>
	\development{algebra}{decomposition-polaire}{Décomposition polaire}
	
	\summary{On montre que toute matrice $M \in \mathrm{GL}_n(\mathbb{R})$ peut s'écrire de manière unique $M = OS$ avec $O \in \mathcal{O}_n(\mathbb{R})$ et $S \in \mathcal{S}_n^{++}(\mathbb{R})$, et que l'application $(O, S) \mapsto M$ est un homéomorphisme.}
	
	\begin{lemma}
		\label{decomposition-polaire-1}
		Soit $S \in \mathcal{S}_n(\mathbb{R})$. Alors $S \in \mathcal{S}_n^{++}(\mathbb{R})$ si et seulement si toutes ses valeurs propres sont strictement positives.
	\end{lemma}
	
	\begin{demonstration}
		Par le théorème spectral, on peut écrire $S = \tr P \operatorname{Diag}(\lambda_1, \dots, \lambda_n) P$ avec $P \in \mathcal{O}_n(\mathbb{R})$. Si on suppose $\lambda_1, \dots, \lambda_n > 0$, on a $\forall x \neq 0$,
		\[ \tr x S x = \tr (Px) \operatorname{Diag}(\lambda_1, \dots, \lambda_n) (Px) > 0 \text{ car } \operatorname{Diag}(\lambda_1, \dots, \lambda_n) \in \mathcal{S}_n^{++}(\mathbb{R}) \]
		d'où le résultat.
		\newpar
		Réciproquement, on suppose $\forall x \neq 0$, $\tr x S x > 0$. Avec $x = \tr P e_1$ (où $e_1$ désigne le vecteur dont la première coordonnée vaut $1$ et les autres sont nulles),
		\[ \tr x S x = \tr (Px) D (Px) = \tr e_1 D e_1 = \lambda_1 > 0 \]
		Et on peut faire de même pour montrer que $\forall i \in \llbracket 1, n \rrbracket$, $\lambda_i > 0$.
	\end{demonstration}
	
	\begin{lemma}
		\label{decomposition-polaire-2}
		$\mathcal{S}_n^+(\mathbb{R})$ est un fermé de $\mathcal{M}_n(\mathbb{R})$ et $\mathrm{GL}_n(\mathbb{R}) \, \cap \, \mathcal{S}_n^{+}(\mathbb{R}) \subset \mathcal{S}_n^{++}(\mathbb{R})$.
	\end{lemma}
	
	\begin{demonstration}
		Pour la première assertion, il suffit de constater que
		\[ \mathcal{S}_n^+(\mathbb{R}) = \{ M \in \mathcal{M}_n(\mathbb{R}) \mid \tr M = M \} \, \cap \, \left( \bigcap_{x \in \mathbb{R}^n} \{ M \in \mathcal{M}_n(\mathbb{R}) \mid \tr x M x \geq 0 \} \right) \]
		qui est une intersection de fermés (par image réciproque). Maintenant, si $M \in \mathrm{GL}_n(\mathbb{R}) \, \cap \, \mathcal{S}_n^{+}(\mathbb{R})$, alors $M$ est diagonalisable avec des valeurs propres positives ou nulles (par le théorème spectral). Mais comme $\det(M) \neq 0$, toutes les valeurs propres de $M$ sont strictement positives. Donc par le \cref{decomposition-polaire-1}, $M \in \mathcal{S}_n^{++}(\mathbb{R})$.
	\end{demonstration}
	
	\reference{C-G}{348}
	
	\begin{theorem}[Décomposition polaire]
		L'application
		\[ \mu :
		\begin{array}{cl}
			\mathcal{O}_n(\mathbb{R}) \times \mathcal{S}_n^{++}(\mathbb{R}) & \rightarrow \mathrm{GL}_n(\mathbb{R}) \\
			(O, S) & \mapsto OS                            
		\end{array}
		\]
		est un homéomorphisme.
	\end{theorem}
	
	\begin{demonstration}
		Montrer qu'une application est un homéomorphisme se fait en $4$ étapes : on montre qu'elle est continue, injective, surjective, et que la réciproque est elle aussi continue.
		\begin{itemize}
			\item \underline{L'application est bien définie et continue :} Si $O \in \mathcal{O}_n(\mathbb{R})$ et $S \in \mathcal{S}_n^{++}(\mathbb{R})$, alors $OS \in \mathrm{GL}_n(\mathbb{R})$. De plus, $\mu$ est continue en tant que restriction de la multiplication matricielle.
			\item \underline{L'application est surjective :} Soit $M \in \mathrm{GL}_n(\mathbb{R})$. Si $x \neq 0$, on a
			\[ \tr x (\tr M M) x = \tr (Mx) (Mx) = \Vert Mx \Vert_2^2 > 0 \]
			En particulier, $\tr M M \in \mathcal{S}_n^{++}(\mathbb{R})$. Par le théorème spectral, il existe $P \in \mathcal{O}_n(\mathbb{R})$ et $\lambda_1, \dots, \lambda_n > 0$ tels que $\tr M M = P \operatorname{Diag}(\lambda_1, \dots, \lambda_n) P^{-1}$. On pose alors
			\[ D = \operatorname{Diag} \left(\sqrt{\lambda_1}, \dots, \sqrt{\lambda_n} \right) \text{ et } S = P D P^{-1} \]
			de sorte que $S^2 = \tr M M$. Mais de plus,
			\[ \tr S = \tr P^{-1} \tr D \tr P = S \implies S \in \mathcal{S}_n(\mathbb{R}) \]
			et par le \cref{decomposition-polaire-1},
			\[ \forall i \in \llbracket 1, n \rrbracket, \, \sqrt{\lambda_i} > 0 \implies S \in \mathcal{S}_n^{++}(\mathbb{R}) \]
			On pose donc $O = MS^{-1}$ (ie. $M = OS$), et on a
			\[ \tr O O = \tr (MS^{-1}) MS^{-1} = \tr S^{-1} \tr M M S^{-1} = S^{-1} S^2 S^{-1} = I_n \implies O \in \mathcal{O}_n(\mathbb{R}) \]
			Donc $\mu(O, S) = M$ et $\mu$ est surjective.
			\item \underline{L'application est injective :} Soit $M = OS \in \mathrm{GL}_n(\mathbb{R})$ (avec $O$ et $S$ comme précédemment). Soit $M = O'S'$ une autre décomposition polaire de $M$. Alors il vient,
			\[ S^2 = \tr M M = \tr (O'S') O'S' = \tr S' \tr O' O' S' = S'^{2} \]
			Soit $Q$ un polynôme tel que $\forall i \in \llbracket 1, n \rrbracket$, $Q(\lambda_i) = \sqrt{\lambda_i}$ (les polynômes d'interpolation de Lagrange conviennent parfaitement). Alors,
			\[\ S = PD \tr P = PQ \left(D^2 \right) \tr P = Q \left(PD^2 \tr P \right) = Q \left(\tr M M \right) = Q \left(S^2 \right) = Q \left(S'^2 \right) \]
			Mais $S'$ commute avec $S'^2$, donc avec $S = Q \left(S'^2 \right)$. En particulier, $S$ et $S'$ sont codiagonalisables, il existe $P_0 \in \mathbb{GL}_n(\mathbb{R})$ et $\mu_1, \dots, \mu_n, \mu'_1, \dots, \mu'_n \in \mathbb{R}$ tels que
			\[ S = P_0 \operatorname{Diag}(\mu_1, \dots, \mu_n) P_0^{-1} \text{ et } S' = P_0 \operatorname{Diag} \left (\mu'_1, \dots, \mu'_n \right) P_0^{-1} \]
			d'où :
			\begin{align*}
				S^2 = S'^2 & \implies P_0 \operatorname{Diag} \left(\mu^2_1, \dots, \mu^2_n \right) P_0^{-1} = P_0 \operatorname{Diag} \left (\mu'^2_1, \dots, \mu'^2_n \right) P_0^{-1} \\
				& \implies \mu^2_i = \mu'^2_i \qquad \forall i \in \llbracket 1, n \rrbracket                                                                                 \\
				& \implies \mu_i = \mu'_i \qquad \forall i \in \llbracket 1, n \rrbracket \text{ car } \forall i \in \llbracket 1, n \rrbracket, \, \mu_i > 0                 \\
				& \implies S = S'                                                                                                                                             
			\end{align*}
			Ainsi, $O = MS^{-1} = MS'^{-1} = O'$. Donc $\mu$ est injective.
			\item \underline{L'application inverse est continue :} Soit $(M_p) \in \mathrm{GL}_n(\mathbb{R})^{\mathbb{N}}$ qui converge vers $M \in \mathrm{GL}_n(\mathbb{R})$. Il s'agit de montrer que la suite $\left (\mu^{-1} \left (M_p \right) \right) = (O_p, S_p)$ converge vers $\mu^{-1}(M) = (O, S)$. Comme $\mathcal{O}_n(\mathbb{R})$ est compact, il existe $\varphi : \mathbb{N} \rightarrow \mathbb{N}$ strictement croissante telle que la suite extraite $(O_{\varphi(p)})$ converge vers une valeur d'adhérence $\overline{O} \in \mathcal{O}_n(\mathbb{R})$. Ainsi, la suite $(S_{\varphi(p)})$ converge vers $\overline{S} = \overline{O}^{-1} M$. 
			\newpar
			Mais, $\overline{S} = \overline{O}^{-1} M \in \mathrm{GL}_n(\mathbb{R}) \, \cap \, \overline{\mathcal{S}_n^{++}(\mathbb{R})}$. Donc par le \cref{decomposition-polaire-1},
			\[ \overline{S} \in \mathrm{GL}_n(\mathbb{R}) \, \cap \, \mathcal{S}_n^{+}(\mathbb{R}) \]
			et par le \cref{decomposition-polaire-2},
			\[ \overline{S} \in \mathcal{S}_n^{++}(\mathbb{R}) \]
			Par unicité de la décomposition polaire, on a $M = \overline{O} \overline{S}$, d'où $\overline{O} = O$ et $\overline{S} = S$.
		\end{itemize}
	\end{demonstration}
	
	\begin{remark}
		La preuve vaut encore dans le cas complexe.
	\end{remark}
	%</content>
\end{document}