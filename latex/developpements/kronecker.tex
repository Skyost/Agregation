\documentclass[12pt, a4paper]{report}

% LuaLaTeX :

\RequirePackage{iftex}
\RequireLuaTeX

% Packages :

\usepackage[french]{babel}
%\usepackage[utf8]{inputenc}
%\usepackage[T1]{fontenc}
\usepackage[pdfencoding=auto, pdfauthor={Hugo Delaunay}, pdfsubject={Mathématiques}, pdfcreator={agreg.skyost.eu}]{hyperref}
\usepackage{amsmath}
\usepackage{amsthm}
%\usepackage{amssymb}
\usepackage{stmaryrd}
\usepackage{tikz}
\usepackage{tkz-euclide}
\usepackage{fourier-otf}
\usepackage{fontspec}
\usepackage{titlesec}
\usepackage{fancyhdr}
\usepackage{catchfilebetweentags}
\usepackage[french, capitalise, noabbrev]{cleveref}
\usepackage[fit, breakall]{truncate}
\usepackage[top=2.5cm, right=2cm, bottom=2.5cm, left=2cm]{geometry}
\usepackage{enumerate}
\usepackage{tocloft}
\usepackage{microtype}
%\usepackage{mdframed}
%\usepackage{thmtools}
\usepackage{xcolor}
\usepackage{tabularx}
\usepackage{aligned-overset}
\usepackage[subpreambles=true]{standalone}
\usepackage{environ}
\usepackage[normalem]{ulem}
\usepackage{marginnote}
\usepackage{etoolbox}
\usepackage{setspace}
\usepackage[bibstyle=reading, citestyle=draft]{biblatex}
\usepackage{xpatch}
\usepackage[many, breakable]{tcolorbox}
\usepackage[backgroundcolor=white, bordercolor=white, textsize=small]{todonotes}

% Bibliographie :

\newcommand{\overridebibliographypath}[1]{\providecommand{\bibliographypath}{#1}}
\overridebibliographypath{../bibliography.bib}
\addbibresource{\bibliographypath}
\defbibheading{bibliography}[\bibname]{%
	\newpage
	\section*{#1}%
}
\renewbibmacro*{entryhead:full}{\printfield{labeltitle}}
\DeclareFieldFormat{url}{\newline\footnotesize\url{#1}}
\AtEndDocument{\printbibliography}

% Police :

\setmathfont{Erewhon Math}

% Tikz :

\usetikzlibrary{calc}

% Longueurs :

\setlength{\parindent}{0pt}
\setlength{\headheight}{15pt}
\setlength{\fboxsep}{0pt}
\titlespacing*{\chapter}{0pt}{-20pt}{10pt}
\setlength{\marginparwidth}{1.5cm}
\setstretch{1.1}

% Métadonnées :

\author{agreg.skyost.eu}
\date{\today}

% Titres :

\setcounter{secnumdepth}{3}

\renewcommand{\thechapter}{\Roman{chapter}}
\renewcommand{\thesubsection}{\Roman{subsection}}
\renewcommand{\thesubsubsection}{\arabic{subsubsection}}
\renewcommand{\theparagraph}{\alph{paragraph}}

\titleformat{\chapter}{\huge\bfseries}{\thechapter}{20pt}{\huge\bfseries}
\titleformat*{\section}{\LARGE\bfseries}
\titleformat{\subsection}{\Large\bfseries}{\thesubsection \, - \,}{0pt}{\Large\bfseries}
\titleformat{\subsubsection}{\large\bfseries}{\thesubsubsection. \,}{0pt}{\large\bfseries}
\titleformat{\paragraph}{\bfseries}{\theparagraph. \,}{0pt}{\bfseries}

\setcounter{secnumdepth}{4}

% Table des matières :

\renewcommand{\cftsecleader}{\cftdotfill{\cftdotsep}}
\addtolength{\cftsecnumwidth}{10pt}

% Redéfinition des commandes :

\renewcommand*\thesection{\arabic{section}}
\renewcommand{\ker}{\mathrm{Ker}}

% Nouvelles commandes :

\newcommand{\website}{https://agreg.skyost.eu}

\newcommand{\tr}[1]{\mathstrut ^t #1}
\newcommand{\im}{\mathrm{Im}}
\newcommand{\rang}{\operatorname{rang}}
\newcommand{\trace}{\operatorname{trace}}
\newcommand{\id}{\operatorname{id}}
\newcommand{\stab}{\operatorname{Stab}}

\providecommand{\newpar}{\\[\medskipamount]}

\providecommand{\lesson}[3]{%
	\title{#3}%
	\hypersetup{pdftitle={#3}}%
	\setcounter{section}{\numexpr #2 - 1}%
	\section{#3}%
	\fancyhead[R]{\truncate{0.73\textwidth}{#2 : #3}}%
}

\providecommand{\development}[3]{%
	\title{#3}%
	\hypersetup{pdftitle={#3}}%
	\section*{#3}%
	\fancyhead[R]{\truncate{0.73\textwidth}{#3}}%
}

\providecommand{\summary}[1]{%
	\textit{#1}%
	\medskip%
}

\tikzset{notestyleraw/.append style={inner sep=0pt, rounded corners=0pt, align=center}}

%\newcommand{\booklink}[1]{\website/bibliographie\##1}
\newcommand{\citelink}[2]{\hyperlink{cite.\therefsection @#1}{#2}}
\newcommand{\previousreference}{}
\providecommand{\reference}[2][]{%
	\notblank{#1}{\renewcommand{\previousreference}{#1}}{}%
	\todo[noline]{%
		\protect\vspace{16pt}%
		\protect\par%
		\protect\notblank{#1}{\cite{[\previousreference]}\\}{}%
		\protect\citelink{\previousreference}{p. #2}%
	}%
}

\definecolor{devcolor}{HTML}{00695c}
\newcommand{\dev}[1]{%
	\reversemarginpar%
	\todo[noline]{
		\protect\vspace{16pt}%
		\protect\par%
		\bfseries\color{devcolor}\href{\website/developpements/#1}{DEV}
	}%
	\normalmarginpar%
}

% En-têtes :

\pagestyle{fancy}
\fancyhead[L]{\truncate{0.23\textwidth}{\thepage}}
\fancyfoot[C]{\scriptsize \href{\website}{\texttt{agreg.skyost.eu}}}

% Couleurs :

\definecolor{property}{HTML}{fffde7}
\definecolor{proposition}{HTML}{fff8e1}
\definecolor{lemma}{HTML}{fff3e0}
\definecolor{theorem}{HTML}{fce4f2}
\definecolor{corollary}{HTML}{ffebee}
\definecolor{definition}{HTML}{ede7f6}
\definecolor{notation}{HTML}{f3e5f5}
\definecolor{example}{HTML}{e0f7fa}
\definecolor{cexample}{HTML}{efebe9}
\definecolor{application}{HTML}{e0f2f1}
\definecolor{remark}{HTML}{e8f5e9}
\definecolor{proof}{HTML}{e1f5fe}

% Théorèmes :

\theoremstyle{definition}
\newtheorem{theorem}{Théorème}

\newtheorem{property}[theorem]{Propriété}
\newtheorem{proposition}[theorem]{Proposition}
\newtheorem{lemma}[theorem]{Lemme}
\newtheorem{corollary}[theorem]{Corollaire}

\newtheorem{definition}[theorem]{Définition}
\newtheorem{notation}[theorem]{Notation}

\newtheorem{example}[theorem]{Exemple}
\newtheorem{cexample}[theorem]{Contre-exemple}
\newtheorem{application}[theorem]{Application}

\theoremstyle{remark}
\newtheorem{remark}[theorem]{Remarque}

\counterwithin*{theorem}{section}

\newcommand{\applystyletotheorem}[1]{
	\tcolorboxenvironment{#1}{
		enhanced,
		breakable,
		colback=#1!98!white,
		boxrule=0pt,
		boxsep=0pt,
		left=8pt,
		right=8pt,
		top=8pt,
		bottom=8pt,
		sharp corners,
		after=\par,
	}
}

\applystyletotheorem{property}
\applystyletotheorem{proposition}
\applystyletotheorem{lemma}
\applystyletotheorem{theorem}
\applystyletotheorem{corollary}
\applystyletotheorem{definition}
\applystyletotheorem{notation}
\applystyletotheorem{example}
\applystyletotheorem{cexample}
\applystyletotheorem{application}
\applystyletotheorem{remark}
\applystyletotheorem{proof}

% Environnements :

\NewEnviron{whitetabularx}[1]{%
	\renewcommand{\arraystretch}{2.5}
	\colorbox{white}{%
		\begin{tabularx}{\textwidth}{#1}%
			\BODY%
		\end{tabularx}%
	}%
}

% Maths :

\DeclareFontEncoding{FMS}{}{}
\DeclareFontSubstitution{FMS}{futm}{m}{n}
\DeclareFontEncoding{FMX}{}{}
\DeclareFontSubstitution{FMX}{futm}{m}{n}
\DeclareSymbolFont{fouriersymbols}{FMS}{futm}{m}{n}
\DeclareSymbolFont{fourierlargesymbols}{FMX}{futm}{m}{n}
\DeclareMathDelimiter{\VERT}{\mathord}{fouriersymbols}{152}{fourierlargesymbols}{147}



\begin{document}
	%<*content>
	\development{algebra}{kronecker}{Théorème de Kronecker}

	\summary{En utilisant les polynômes symétriques, nous montrons ici que toutes les racines d'un polynôme unitaire à coefficients entiers dont les racines sont dans $D(0, 1) \setminus \{ 0 \}$, sont en fait des racines de l'unité.}

	\begin{lemma}[Relations de Viète]
		\label{kronecker-1}
		Soient $A$ un anneau commutatif unitaire intègre et $P = \sum_{i=1}^n a_iX^i \in K[X]$ que l'on suppose scindé dans $A[X]$ et tel que $a_n \in A^*$. Si on note $\Sigma_k(X_1, \dots, X_n)$ le $k$-ième polynôme symétrique élémentaire en $n$ variables et $\alpha_1$, ..., $\alpha_n$ les racines de $P$ (comptées avec multiplicité), alors $\Sigma_k(\alpha_1, \dots, \alpha_n) = (-1)^k a_{n-k} a_n^{-1}$.
	\end{lemma}

	\begin{demonstration}
		On a $P = a_n \prod_{i=1}^n (X-\alpha_i)$. En développant partiellement $P$, on a de même :
		\[ P = a_n X^n - a_n (\alpha_1 + \dots + \alpha_n)X^{n-1} + \dots + (-1)^n a_n \alpha_1 \dots \alpha_n \]
		Par identification avec la forme développée, les coefficients de $X^{n-1}$ doivent être égaux. En particulier :
		\[ a_{n-1} = -a_n (\alpha_1 + \dots + \alpha_n) \iff \underbrace{\alpha_1 + \dots + \alpha_n}_{= \Sigma_1(\alpha_1, \dots \alpha_n)} = - a_{n-1} a_n^{-1} \]
		Et on procède de même pour trouver les autres coefficients. Par exemple, $a_0 = (-1)^n a_n \alpha_1 \dots \alpha_n \iff \Sigma_n(\alpha_1, \dots \alpha_n) = (-1)^n a_0 a_n^{-1}$.
	\end{demonstration}

	\begin{remark}
		Tout au long de ce développement, nous utiliserons implicitement le fait que tout polynôme à coefficient dans $\mathbb{C}$ (dont à fortiori aussi dans $\mathbb{Z}$) admet $n$ racines complexes comptées avec multiplicité. Il s'agit du théorème de d'Alembert-Gauss.
	\end{remark}

	\reference{I-P}{279}

	\begin{theorem}[Kronecker]
		\label{kronecker-2}
		Soit $P \in \mathbb{Z}[X]$ unitaire tel que toutes ses racines complexes appartiennent au disque unité épointé en l'origine (que l'on note $D$). Alors toutes ses racines sont des racines de l'unité.
	\end{theorem}

	\begin{demonstration}
		Notons par $\Omega_n$ l'ensemble des polynômes unitaires à coefficients dans $\mathbb{Z}$ tels que toutes leurs racines complexes appartiennent à $D$. Soit $P \in \Omega_n$ dont on note $a_0, \dots, a_n$ les coefficients et $z_1, \dots, z_n$ les racines complexes. On note $\forall k \in \llbracket 0, n \rrbracket$, $\sigma_k = \Sigma_k(z_1, \dots, z_n)$. D'après le \cref{kronecker-1}, on a :
		\[ \forall k \in \llbracket 0, n \rrbracket, \, \sigma_k = (-1)^k a_{n-k} \tag{$*$} \]
		D'où $\forall k \in \llbracket 0, n \rrbracket$ :
		\begin{align*}
			|\sigma_k| &= \left| \sum_{I \in \mathcal{P}_k(\llbracket 1, n \rrbracket)} \prod_{i \in I} z_i \right| \\
			&\leq \sum_{I \in \mathcal{P}_k(\llbracket 1, n \rrbracket)} \prod_{i \in I} |z_i| \\
			&\leq |\mathcal{P}_k(\llbracket 1, n \rrbracket)| \times 1 \\
			&= \binom{n}{k}
		\end{align*}
		Et par $(*)$,
		\[ \forall k \in \llbracket 0, n \rrbracket, \, |a_k| \leq \binom{n}{n-k} = \binom{n}{k} \]
		$\Omega_n$ est donc un ensemble fini (car on n'a qu'un nombre limité de choix possibles pour les coefficients $a_k$).
		\newpar
		On pose maintenant
		\[ \forall k \in \mathbb{N}, \, P_k = \prod_{j=0}^n (X-z_j^k) \]
		qui sont des polynômes unitaires de degré $n$ dont les racines $z_1^k, \dots, z_n^k$ appartiennent toutes à $D$. Soient $k \in \mathbb{N}$ et $r \in \llbracket 0, n \rrbracket$. D'après le \cref{kronecker-1}, le coefficient de $X^{n-r}$ de $P_k$ est $(-1)^r \Sigma_r(z_1^k, \dots, z_n^k)$. Mais, $\Sigma_r(X_1^k, \dots, X_n^k) \in \mathbb{Z}[X]$, donc on peut y appliquer le théorème fondamental des polynômes symétriques :
		\[ \exists Q_{r,k} \in \mathbb{Z}[X] \text{ tel que } \Sigma_r(X_1^k, \dots, X_n^k) = Q_{r,k}(\Sigma_1(X_1, \dots, X_n), \dots, \Sigma_n(X_1, \dots, X_n)) \]
		Or, comme $P \in \mathbb{Z}[X]$, on a $\forall j \in \llbracket 0, n \rrbracket$, $\Sigma_j(z_1, \dots, z_n) \in \mathbb{Z}$ d'après le \cref{kronecker-1}. En particulier, on a $\Sigma_r(X_1^k, \dots, X_n^k) \in \mathbb{Z}[X]$ car $Q_{r,k} \in \mathbb{Z}[X]$. On en déduit que $\forall k \in \mathbb{N}$, $P_k \in \Omega_n$.
		\newpar
		Comme $\Omega_n$ est fini, l'ensemble des racines de tous les $P_k$ ; qui est $\{ z \in \mathbb{C} \mid \exists k \in \mathbb{N}, \, P_k(z) = 0 \}$ est fini. Soit $j \in \llbracket 1, n \rrbracket$. L'ensemble $\{ z_j^k \mid k \in \mathbb{N} \}$ est inclus dans l'ensemble de ces racines, qui est fini ; il est donc lui-même fini :
		\[ \exists k \neq k' \text{ tel que } z_j^k = z_j^{k'} \]
		Quitte à échanger les deux, on peut supposer $k \geq k'$. Comme $z_j \neq 0$, on a $z_j^{k-k'} = 1$. Donc $z_j$ est une racine de l'unité ; ce que l'on voulait.
	\end{demonstration}

	\begin{corollary}
		Soit $P \in \mathbb{Z}[X]$ unitaire et irréductible sur $\mathbb{Q}$ tel que toutes ses racines complexes soient de module inférieur ou égal à $1$. Alors $P = X$ ou $P$ est un polynôme cyclotomique.
	\end{corollary}

	\begin{demonstration}
		Si $0$ est racine de $P$, alors $X \mid P$, donc $P = X$ par irréductibilité et unitarité. Sinon, $0$ n'est pas racine de $P$. On peut donc appliquer le \cref{kronecker-2} à $P$ ; et donc les racines de $P$ sont des racines de l'unité. Ainsi, en notant $N$ le maximum des ordres des racines de $P$, on a :
		\[ P \mid (X^N - 1)^n \text{ où } n = \deg(P) \]
		Or, la décomposition en irréductibles de $X^N - 1$ est
		\[ X^N - 1 = \prod_{d \mid N} \Phi_d \]
		Puisque $\mathbb{Q}[X]$ est un anneau factoriel, $P$ est premier. Donc d'après le lemme de Gauss, comme $P \mid X^N - 1$ :
		\[ \exists d \mid N \text{ tel que } P = \Phi_d \]
	\end{demonstration}
	%</content>
\end{document}
