\documentclass[12pt, a4paper]{report}

% LuaLaTeX :

\RequirePackage{iftex}
\RequireLuaTeX

% Packages :

\usepackage[french]{babel}
%\usepackage[utf8]{inputenc}
%\usepackage[T1]{fontenc}
\usepackage[pdfencoding=auto, pdfauthor={Hugo Delaunay}, pdfsubject={Mathématiques}, pdfcreator={agreg.skyost.eu}]{hyperref}
\usepackage{amsmath}
\usepackage{amsthm}
%\usepackage{amssymb}
\usepackage{stmaryrd}
\usepackage{tikz}
\usepackage{tkz-euclide}
\usepackage{fourier-otf}
\usepackage{fontspec}
\usepackage{titlesec}
\usepackage{fancyhdr}
\usepackage{catchfilebetweentags}
\usepackage[french, capitalise, noabbrev]{cleveref}
\usepackage[fit, breakall]{truncate}
\usepackage[top=2.5cm, right=2cm, bottom=2.5cm, left=2cm]{geometry}
\usepackage{enumerate}
\usepackage{tocloft}
\usepackage{microtype}
%\usepackage{mdframed}
%\usepackage{thmtools}
\usepackage{xcolor}
\usepackage{tabularx}
\usepackage{aligned-overset}
\usepackage[subpreambles=true]{standalone}
\usepackage{environ}
\usepackage[normalem]{ulem}
\usepackage{marginnote}
\usepackage{etoolbox}
\usepackage{setspace}
\usepackage[bibstyle=reading, citestyle=draft]{biblatex}
\usepackage{xpatch}
\usepackage[many, breakable]{tcolorbox}
\usepackage[backgroundcolor=white, bordercolor=white, textsize=small]{todonotes}

% Bibliographie :

\newcommand{\overridebibliographypath}[1]{\providecommand{\bibliographypath}{#1}}
\overridebibliographypath{../bibliography.bib}
\addbibresource{\bibliographypath}
\defbibheading{bibliography}[\bibname]{%
	\newpage
	\section*{#1}%
}
\renewbibmacro*{entryhead:full}{\printfield{labeltitle}}
\DeclareFieldFormat{url}{\newline\footnotesize\url{#1}}
\AtEndDocument{\printbibliography}

% Police :

\setmathfont{Erewhon Math}

% Tikz :

\usetikzlibrary{calc}

% Longueurs :

\setlength{\parindent}{0pt}
\setlength{\headheight}{15pt}
\setlength{\fboxsep}{0pt}
\titlespacing*{\chapter}{0pt}{-20pt}{10pt}
\setlength{\marginparwidth}{1.5cm}
\setstretch{1.1}

% Métadonnées :

\author{agreg.skyost.eu}
\date{\today}

% Titres :

\setcounter{secnumdepth}{3}

\renewcommand{\thechapter}{\Roman{chapter}}
\renewcommand{\thesubsection}{\Roman{subsection}}
\renewcommand{\thesubsubsection}{\arabic{subsubsection}}
\renewcommand{\theparagraph}{\alph{paragraph}}

\titleformat{\chapter}{\huge\bfseries}{\thechapter}{20pt}{\huge\bfseries}
\titleformat*{\section}{\LARGE\bfseries}
\titleformat{\subsection}{\Large\bfseries}{\thesubsection \, - \,}{0pt}{\Large\bfseries}
\titleformat{\subsubsection}{\large\bfseries}{\thesubsubsection. \,}{0pt}{\large\bfseries}
\titleformat{\paragraph}{\bfseries}{\theparagraph. \,}{0pt}{\bfseries}

\setcounter{secnumdepth}{4}

% Table des matières :

\renewcommand{\cftsecleader}{\cftdotfill{\cftdotsep}}
\addtolength{\cftsecnumwidth}{10pt}

% Redéfinition des commandes :

\renewcommand*\thesection{\arabic{section}}
\renewcommand{\ker}{\mathrm{Ker}}

% Nouvelles commandes :

\newcommand{\website}{https://agreg.skyost.eu}

\newcommand{\tr}[1]{\mathstrut ^t #1}
\newcommand{\im}{\mathrm{Im}}
\newcommand{\rang}{\operatorname{rang}}
\newcommand{\trace}{\operatorname{trace}}
\newcommand{\id}{\operatorname{id}}
\newcommand{\stab}{\operatorname{Stab}}

\providecommand{\newpar}{\\[\medskipamount]}

\providecommand{\lesson}[3]{%
	\title{#3}%
	\hypersetup{pdftitle={#3}}%
	\setcounter{section}{\numexpr #2 - 1}%
	\section{#3}%
	\fancyhead[R]{\truncate{0.73\textwidth}{#2 : #3}}%
}

\providecommand{\development}[3]{%
	\title{#3}%
	\hypersetup{pdftitle={#3}}%
	\section*{#3}%
	\fancyhead[R]{\truncate{0.73\textwidth}{#3}}%
}

\providecommand{\summary}[1]{%
	\textit{#1}%
	\medskip%
}

\tikzset{notestyleraw/.append style={inner sep=0pt, rounded corners=0pt, align=center}}

%\newcommand{\booklink}[1]{\website/bibliographie\##1}
\newcommand{\citelink}[2]{\hyperlink{cite.\therefsection @#1}{#2}}
\newcommand{\previousreference}{}
\providecommand{\reference}[2][]{%
	\notblank{#1}{\renewcommand{\previousreference}{#1}}{}%
	\todo[noline]{%
		\protect\vspace{16pt}%
		\protect\par%
		\protect\notblank{#1}{\cite{[\previousreference]}\\}{}%
		\protect\citelink{\previousreference}{p. #2}%
	}%
}

\definecolor{devcolor}{HTML}{00695c}
\newcommand{\dev}[1]{%
	\reversemarginpar%
	\todo[noline]{
		\protect\vspace{16pt}%
		\protect\par%
		\bfseries\color{devcolor}\href{\website/developpements/#1}{DEV}
	}%
	\normalmarginpar%
}

% En-têtes :

\pagestyle{fancy}
\fancyhead[L]{\truncate{0.23\textwidth}{\thepage}}
\fancyfoot[C]{\scriptsize \href{\website}{\texttt{agreg.skyost.eu}}}

% Couleurs :

\definecolor{property}{HTML}{fffde7}
\definecolor{proposition}{HTML}{fff8e1}
\definecolor{lemma}{HTML}{fff3e0}
\definecolor{theorem}{HTML}{fce4f2}
\definecolor{corollary}{HTML}{ffebee}
\definecolor{definition}{HTML}{ede7f6}
\definecolor{notation}{HTML}{f3e5f5}
\definecolor{example}{HTML}{e0f7fa}
\definecolor{cexample}{HTML}{efebe9}
\definecolor{application}{HTML}{e0f2f1}
\definecolor{remark}{HTML}{e8f5e9}
\definecolor{proof}{HTML}{e1f5fe}

% Théorèmes :

\theoremstyle{definition}
\newtheorem{theorem}{Théorème}

\newtheorem{property}[theorem]{Propriété}
\newtheorem{proposition}[theorem]{Proposition}
\newtheorem{lemma}[theorem]{Lemme}
\newtheorem{corollary}[theorem]{Corollaire}

\newtheorem{definition}[theorem]{Définition}
\newtheorem{notation}[theorem]{Notation}

\newtheorem{example}[theorem]{Exemple}
\newtheorem{cexample}[theorem]{Contre-exemple}
\newtheorem{application}[theorem]{Application}

\theoremstyle{remark}
\newtheorem{remark}[theorem]{Remarque}

\counterwithin*{theorem}{section}

\newcommand{\applystyletotheorem}[1]{
	\tcolorboxenvironment{#1}{
		enhanced,
		breakable,
		colback=#1!98!white,
		boxrule=0pt,
		boxsep=0pt,
		left=8pt,
		right=8pt,
		top=8pt,
		bottom=8pt,
		sharp corners,
		after=\par,
	}
}

\applystyletotheorem{property}
\applystyletotheorem{proposition}
\applystyletotheorem{lemma}
\applystyletotheorem{theorem}
\applystyletotheorem{corollary}
\applystyletotheorem{definition}
\applystyletotheorem{notation}
\applystyletotheorem{example}
\applystyletotheorem{cexample}
\applystyletotheorem{application}
\applystyletotheorem{remark}
\applystyletotheorem{proof}

% Environnements :

\NewEnviron{whitetabularx}[1]{%
	\renewcommand{\arraystretch}{2.5}
	\colorbox{white}{%
		\begin{tabularx}{\textwidth}{#1}%
			\BODY%
		\end{tabularx}%
	}%
}

% Maths :

\DeclareFontEncoding{FMS}{}{}
\DeclareFontSubstitution{FMS}{futm}{m}{n}
\DeclareFontEncoding{FMX}{}{}
\DeclareFontSubstitution{FMX}{futm}{m}{n}
\DeclareSymbolFont{fouriersymbols}{FMS}{futm}{m}{n}
\DeclareSymbolFont{fourierlargesymbols}{FMX}{futm}{m}{n}
\DeclareMathDelimiter{\VERT}{\mathord}{fouriersymbols}{152}{fourierlargesymbols}{147}



\begin{document}
	%<*content>
	\development{analysis}{methode-de-newton}{Méthode de Newton}

	\summary{On démontre ici la méthode de Newton qui permet de trouver numériquement une approximation précise d'un zéro d'une fonction réelle d'une variable réelle.}

	\reference[ROU]{142}

	\begin{theorem}
		Soit $f : [c, d] \rightarrow \mathbb{R}$ une fonction de classe $\mathcal{C}^2$ strictement croissante sur $[c, d]$. On considère la fonction
		\[ \varphi :
		\begin{array}{cl}
			[c, d] &\rightarrow \mathbb{R} \\
			x &\mapsto x - \frac{f(x)}{f'(x)}
		\end{array}
		\]
		(qui est bien définie car $f' > 0$). Alors :
		\begin{enumerate}[(i)]
			\item $\exists! a \in [c, d]$ tel que $f(a) = 0$.
			\item $\exists \alpha > 0$ tel que $I = [a - \alpha, a + \alpha]$ est stable par $\varphi$.
			\item La suite $(x_n)$ des itérés (définie par récurrence par $x_{n+1} = \varphi(x_n)$ pour tout $n \geq 0$) converge quadratiquement vers $a$ pour tout $x_0 \in I$.
		\end{enumerate}
	\end{theorem}

	\begin{proof}
		Soit $x \in [c, d]$. Comme $f(a) = 0$, on peut écrire :
		\begin{align*}
			\varphi(x) - a &= x - a - \frac{f(x) - f(a)}{f'(x)} \\
			&= \frac{f(a) - f(x) - (a-x)f'(x)}{f'(x)}
		\end{align*}
		Or, la formule de Taylor-Lagrange à l'ordre $2$ donne l'existence d'un $z \in ]a, x[$ tel que
		\begin{align*}
			&f(a) = f(x) + f'(x)(a-x) + \frac{1}{2} f''(z)(a-x)^2 \\
			\iff& f(a) - f(x) - f'(x)(a-x)  = \frac{1}{2} f''(z)(a-x)^2
		\end{align*}
		D'où :
		\[ \varphi(x) - a = \frac{f''(z)}{2f'(x)}(x-a)^2 \tag{$*$} \]
		Soit $C = \frac{\max_{x \in [c, d]} |f''(x)|}{2\min_{x \in [c, d]} |f'(x)|}$. Par $(*)$, on a :
		\[ \forall x \in [c, d], \, |\varphi(x)-a| \leq C |x-a|^2 \]
		Soit maintenant $\alpha > 0$ suffisamment petit pour que $C\alpha < 1$ et que $I = [a - \alpha, a + \alpha] \subset [c, d]$. Alors :
		\[ x \in I \implies |\varphi(x) - a| \leq C\alpha^2 < \alpha \]
		(la première inégalité se voit en faisant un dessin, et la seconde vient du fait que $C\alpha < 1$). D'où $\varphi(I) \subset I$. Et si $x_0 \in I$, on a donc $\forall n \in \mathbb{N}$, $x_n \in I$ et
		\begin{align*}
			|x_{n+1} - a| &= |\varphi(x_n) - a| \\
			&\leq C |x_n - a|^2
		\end{align*}
		D'où $c |x_n - a| \leq (C |x_0 - a|)^{2^n} \leq (C \alpha)^{2^n}$ où $C \alpha < 1$. On a donc bien convergence quadratique de la suite $(x_n)$ vers le réel $a$.
	\end{proof}

	\reference[DEM]{100}

	\begin{remark}
		On suppose que l'on connaisse une approximation grossière du point que l'on nomme $x_0$.
		\includelatexpicture{methode-de-newton}
		L'idée de la méthode est de remplacer la courbe représentative de $f$ par sa tangente au point $x_0$ :
		\[ y = f'(x_0)(x-x_0) + f(x_0) \]
		L'abscisse $x_1$ du point d'intersection de cette tangente avec l'axe des abscisses est donnée par
		\[ x_1 = x_0 - \frac{f(x_0)}{f'(x_0)} \]
		d'où le fait d'itérer la fonction $\varphi : x \mapsto x - \frac{f(x)}{f'(x)}$.
	\end{remark}

	\reference[ROU]{142}

	\begin{corollary}
		En reprenant les hypothèses et notations du théorème précédent, et en supposant de plus $f$ strictement convexe sur $[c, d]$, le résultat du théorème est vrai sur $I = [a, d]$. De plus :
		\begin{enumerate}[(i)]
			\item $(x_n)$ est strictement décroissante (ou constante).
			\item $x_{n+1} - a \sim \frac{f''(a)}{f'(a)} (x_n - a)^2$ pour $x_0 > a$.
		\end{enumerate}
	\end{corollary}

	\begin{proof}
		La dérivée $f'$ est strictement croissante (car $f$ est strictement convexe) sur $]c, d[$. Ainsi, soit $x \in [a, d]$. Si $x = a$, on a $\varphi(x) = x$, et la suite $(x_n)$ est alors constante. Supposons maintenant $x > a$. On a :
		\[ \varphi(x) = x - \frac{\overbrace{f(x)}^{> 0}}{\underbrace{f'(x)}_{> 0}} < x \]
		Et par $(*)$ (de la démonstration précédente), $\exists z \in ]a, x[$ :
		\[ \varphi(x) - a = \frac{f''(z)}{2f'(z)} (x-a)^2 > 0 \iff \varphi(x) < a \]
		Ainsi, $I = [a, d]$ est stable par $\varphi$ et pour $x_0 \in ]a, d]$, on a $x_n \in ]a, d]$ pour tout $n \in \mathbb{N}$ et la suite $(x_n)$ est strictement décroissante. La suite $(x_n)$ admet donc une limite $\ell$ vérifiant $\varphi(\ell) = \ell \iff f(\ell) = 0$ ie. $\ell = a$ par unicité. Comme dans le théorème précédent, la convergence est quadratique :
		\[ 0 \leq x_{n+1} - a \leq C (x_n - a)^2 \]
		Enfin, si $x_0 \in ]a, d]$, on a comme dans $(*)$ :
		\[ \forall n \in \mathbb{N}, \, x_n > a \text{ et } \frac{x_{n+1} - a}{(x_n - a)^2} = \frac{f''(z_n)}{f'(z)} \]
		(en faisant la même démarche que pour $(*)$ on obtient $z_n \in ]a, x_n[$). On fait tendre $n$ vers l'infini et la fraction de droite tend vers $\frac{f''(a)}{f'(a)}$; d'où le résultat.
	\end{proof}

	\begin{remark}
		L'ajout de l'hypothèse de convexité à la méthode de Newton, nous permet de nous affranchir de l'intervalle $I$ tout en gardant la même vitesse de convergence.
	\end{remark}
	%</content>
\end{document}
