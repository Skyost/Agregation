\documentclass[12pt, a4paper]{report}

% LuaLaTeX :

\RequirePackage{iftex}
\RequireLuaTeX

% Packages :

\usepackage[french]{babel}
%\usepackage[utf8]{inputenc}
%\usepackage[T1]{fontenc}
\usepackage[pdfencoding=auto, pdfauthor={Hugo Delaunay}, pdfsubject={Mathématiques}, pdfcreator={agreg.skyost.eu}]{hyperref}
\usepackage{amsmath}
\usepackage{amsthm}
%\usepackage{amssymb}
\usepackage{stmaryrd}
\usepackage{tikz}
\usepackage{tkz-euclide}
\usepackage{fourier-otf}
\usepackage{fontspec}
\usepackage{titlesec}
\usepackage{fancyhdr}
\usepackage{catchfilebetweentags}
\usepackage[french, capitalise, noabbrev]{cleveref}
\usepackage[fit, breakall]{truncate}
\usepackage[top=2.5cm, right=2cm, bottom=2.5cm, left=2cm]{geometry}
\usepackage{enumerate}
\usepackage{tocloft}
\usepackage{microtype}
%\usepackage{mdframed}
%\usepackage{thmtools}
\usepackage{xcolor}
\usepackage{tabularx}
\usepackage{aligned-overset}
\usepackage[subpreambles=true]{standalone}
\usepackage{environ}
\usepackage[normalem]{ulem}
\usepackage{marginnote}
\usepackage{etoolbox}
\usepackage{setspace}
\usepackage[bibstyle=reading, citestyle=draft]{biblatex}
\usepackage{xpatch}
\usepackage[many, breakable]{tcolorbox}
\usepackage[backgroundcolor=white, bordercolor=white, textsize=small]{todonotes}

% Bibliographie :

\newcommand{\overridebibliographypath}[1]{\providecommand{\bibliographypath}{#1}}
\overridebibliographypath{../bibliography.bib}
\addbibresource{\bibliographypath}
\defbibheading{bibliography}[\bibname]{%
	\newpage
	\section*{#1}%
}
\renewbibmacro*{entryhead:full}{\printfield{labeltitle}}
\DeclareFieldFormat{url}{\newline\footnotesize\url{#1}}
\AtEndDocument{\printbibliography}

% Police :

\setmathfont{Erewhon Math}

% Tikz :

\usetikzlibrary{calc}

% Longueurs :

\setlength{\parindent}{0pt}
\setlength{\headheight}{15pt}
\setlength{\fboxsep}{0pt}
\titlespacing*{\chapter}{0pt}{-20pt}{10pt}
\setlength{\marginparwidth}{1.5cm}
\setstretch{1.1}

% Métadonnées :

\author{agreg.skyost.eu}
\date{\today}

% Titres :

\setcounter{secnumdepth}{3}

\renewcommand{\thechapter}{\Roman{chapter}}
\renewcommand{\thesubsection}{\Roman{subsection}}
\renewcommand{\thesubsubsection}{\arabic{subsubsection}}
\renewcommand{\theparagraph}{\alph{paragraph}}

\titleformat{\chapter}{\huge\bfseries}{\thechapter}{20pt}{\huge\bfseries}
\titleformat*{\section}{\LARGE\bfseries}
\titleformat{\subsection}{\Large\bfseries}{\thesubsection \, - \,}{0pt}{\Large\bfseries}
\titleformat{\subsubsection}{\large\bfseries}{\thesubsubsection. \,}{0pt}{\large\bfseries}
\titleformat{\paragraph}{\bfseries}{\theparagraph. \,}{0pt}{\bfseries}

\setcounter{secnumdepth}{4}

% Table des matières :

\renewcommand{\cftsecleader}{\cftdotfill{\cftdotsep}}
\addtolength{\cftsecnumwidth}{10pt}

% Redéfinition des commandes :

\renewcommand*\thesection{\arabic{section}}
\renewcommand{\ker}{\mathrm{Ker}}

% Nouvelles commandes :

\newcommand{\website}{https://agreg.skyost.eu}

\newcommand{\tr}[1]{\mathstrut ^t #1}
\newcommand{\im}{\mathrm{Im}}
\newcommand{\rang}{\operatorname{rang}}
\newcommand{\trace}{\operatorname{trace}}
\newcommand{\id}{\operatorname{id}}
\newcommand{\stab}{\operatorname{Stab}}

\providecommand{\newpar}{\\[\medskipamount]}

\providecommand{\lesson}[3]{%
	\title{#3}%
	\hypersetup{pdftitle={#3}}%
	\setcounter{section}{\numexpr #2 - 1}%
	\section{#3}%
	\fancyhead[R]{\truncate{0.73\textwidth}{#2 : #3}}%
}

\providecommand{\development}[3]{%
	\title{#3}%
	\hypersetup{pdftitle={#3}}%
	\section*{#3}%
	\fancyhead[R]{\truncate{0.73\textwidth}{#3}}%
}

\providecommand{\summary}[1]{%
	\textit{#1}%
	\medskip%
}

\tikzset{notestyleraw/.append style={inner sep=0pt, rounded corners=0pt, align=center}}

%\newcommand{\booklink}[1]{\website/bibliographie\##1}
\newcommand{\citelink}[2]{\hyperlink{cite.\therefsection @#1}{#2}}
\newcommand{\previousreference}{}
\providecommand{\reference}[2][]{%
	\notblank{#1}{\renewcommand{\previousreference}{#1}}{}%
	\todo[noline]{%
		\protect\vspace{16pt}%
		\protect\par%
		\protect\notblank{#1}{\cite{[\previousreference]}\\}{}%
		\protect\citelink{\previousreference}{p. #2}%
	}%
}

\definecolor{devcolor}{HTML}{00695c}
\newcommand{\dev}[1]{%
	\reversemarginpar%
	\todo[noline]{
		\protect\vspace{16pt}%
		\protect\par%
		\bfseries\color{devcolor}\href{\website/developpements/#1}{DEV}
	}%
	\normalmarginpar%
}

% En-têtes :

\pagestyle{fancy}
\fancyhead[L]{\truncate{0.23\textwidth}{\thepage}}
\fancyfoot[C]{\scriptsize \href{\website}{\texttt{agreg.skyost.eu}}}

% Couleurs :

\definecolor{property}{HTML}{fffde7}
\definecolor{proposition}{HTML}{fff8e1}
\definecolor{lemma}{HTML}{fff3e0}
\definecolor{theorem}{HTML}{fce4f2}
\definecolor{corollary}{HTML}{ffebee}
\definecolor{definition}{HTML}{ede7f6}
\definecolor{notation}{HTML}{f3e5f5}
\definecolor{example}{HTML}{e0f7fa}
\definecolor{cexample}{HTML}{efebe9}
\definecolor{application}{HTML}{e0f2f1}
\definecolor{remark}{HTML}{e8f5e9}
\definecolor{proof}{HTML}{e1f5fe}

% Théorèmes :

\theoremstyle{definition}
\newtheorem{theorem}{Théorème}

\newtheorem{property}[theorem]{Propriété}
\newtheorem{proposition}[theorem]{Proposition}
\newtheorem{lemma}[theorem]{Lemme}
\newtheorem{corollary}[theorem]{Corollaire}

\newtheorem{definition}[theorem]{Définition}
\newtheorem{notation}[theorem]{Notation}

\newtheorem{example}[theorem]{Exemple}
\newtheorem{cexample}[theorem]{Contre-exemple}
\newtheorem{application}[theorem]{Application}

\theoremstyle{remark}
\newtheorem{remark}[theorem]{Remarque}

\counterwithin*{theorem}{section}

\newcommand{\applystyletotheorem}[1]{
	\tcolorboxenvironment{#1}{
		enhanced,
		breakable,
		colback=#1!98!white,
		boxrule=0pt,
		boxsep=0pt,
		left=8pt,
		right=8pt,
		top=8pt,
		bottom=8pt,
		sharp corners,
		after=\par,
	}
}

\applystyletotheorem{property}
\applystyletotheorem{proposition}
\applystyletotheorem{lemma}
\applystyletotheorem{theorem}
\applystyletotheorem{corollary}
\applystyletotheorem{definition}
\applystyletotheorem{notation}
\applystyletotheorem{example}
\applystyletotheorem{cexample}
\applystyletotheorem{application}
\applystyletotheorem{remark}
\applystyletotheorem{proof}

% Environnements :

\NewEnviron{whitetabularx}[1]{%
	\renewcommand{\arraystretch}{2.5}
	\colorbox{white}{%
		\begin{tabularx}{\textwidth}{#1}%
			\BODY%
		\end{tabularx}%
	}%
}

% Maths :

\DeclareFontEncoding{FMS}{}{}
\DeclareFontSubstitution{FMS}{futm}{m}{n}
\DeclareFontEncoding{FMX}{}{}
\DeclareFontSubstitution{FMX}{futm}{m}{n}
\DeclareSymbolFont{fouriersymbols}{FMS}{futm}{m}{n}
\DeclareSymbolFont{fourierlargesymbols}{FMX}{futm}{m}{n}
\DeclareMathDelimiter{\VERT}{\mathord}{fouriersymbols}{152}{fourierlargesymbols}{147}



\begin{document}
	%<*content>
	\development{analysis}{theoreme-d-abel-angulaire}{Théorème d'Abel angulaire}

	\summary{On montre le théorème d'Abel ``angulaire'', qui permet d'intervertir certaines sommes et limites, et on l'applique justement au calcul de deux sommes.}

	\reference[GOU20]{263}

	\begin{theorem}[Abel angulaire]
		\label{theoreme-d-abel-angulaire-1}
		Soit $\sum a_n z^n$ une série entière de rayon de convergence supérieur ou égal à $1$ tel que $\sum a_n$ converge. On note $f$ la somme de cette série sur le disque unité $D$ de $\mathbb{C}$. On fixe $\theta_0 \in \left[ 0, \frac{\pi}{2} \right[$ et on pose $\Delta_{\theta_0} = \{ z \in D \mid \exists \rho > 0 \text{ et } \exists \theta \in [-\theta_0, \theta_0] \text{ tels que } z = 1 - \rho e^{i\theta} \}$.
		\includelatexpicture{theoreme-d-abel-angulaire}
		Alors $\lim_{\substack{z \rightarrow 1 \\ z \in \Delta_{\theta_0}}} f(z) = \sum_{n=0}^{+\infty} a_n$.
	\end{theorem}

	\begin{demonstration}
		On note $\forall n \in \mathbb{N}$, $S = \sum_{n=0}^{+\infty} a_n$, $S_n = \sum_{k=0}^n a_k$ et $R_n = S - S_n$. On chercher à majorer $|f(z) - S|$ ; on va effectuer une transformation d'Abel en écrivant $\forall n \geq 1$, $a_n = R_{n-1} - R_n$. Soit $z \in D \setminus \{ 0 \}$. $\forall N \in \mathbb{N}^*$, on a
		\begin{align*}
			\sum_{n=0}^N a_n z^n &= \sum_{n=1}^N (R_{n-1} - R_n)(z^n - 1) \\
			&= \sum_{n=0}^{N-1} R_n(z^{n+1} - 1) - \sum_{n=1}^N R_n(z^n - 1) \\
			&= \sum_{n=0}^{N-1} R_n(z^{n+1} - z^n) - R_N(z^N - 1) \\
			&= (z-1) \sum_{n=0}^{N-1} R_nz^n - R_N(z^N - 1)
		\end{align*}
		Donc en faisant $N \rightarrow +\infty$ :
		\[ f(z) - S = (z-1) \sum_{n=0}^{+\infty} R_nz^n \tag{$*$} \]
		Soit $\epsilon > 0$. $\exists N \in \mathbb{N}$ tel que $\forall n \geq N$, $|R_n| < \epsilon$. D'après $(*)$, $\forall z \in D$,
		\begin{align*}
			|f(z)-S| &\leq |z-1| \left| \sum_{n=0}^N R_n z^n \right| + \epsilon |z-1| \left( \sum_{n=N+1}^{+\infty} |z|^n \right) \\
			&\leq |z-1| \left( \sum_{n=0}^N |R_n| \right) + \epsilon \frac{|z-1|}{1-|z|} \tag{$**$}
		\end{align*}
		Soit $z \in \Delta_{\theta_0}$ de sorte que $z = 1-\rho e^{i\theta}$ avec $\rho > 0$ et $|\theta| \leq \theta_0$. Notons avant toute chose que $|z-1| = \rho$. Cherchons maintenant des conditions sur $z$ pour majorer les deux termes :
		\begin{itemize}
			\item On a :
			\begin{align*}
				|z|^2 &= (1 - \rho \cos(\theta))^2 + (\rho \sin(\theta))^2 \\
				&= 1 - 2 \rho \cos(\theta) + \rho^2 (\cos(\theta)^2 + \sin(\theta)^2) \\
				&= 1 - 2 \rho \cos(\theta) + \rho^2
			\end{align*}
			En supposant $\rho \leq \cos(\theta_0)$, cela permet de majorer le deuxième terme de $(**)$ :
			\begin{align*}
				\frac{|z-1|}{1-|z|} &= \frac{|z-1|}{1-|z|^2}(1+|z|) \\
				&= \frac{\rho}{2 \rho \cos(\theta) - \rho^2}(1+|z|) \\
				&\leq \frac{2}{2\cos(\theta) - \rho} \\
				&\leq \frac{2}{2\cos(\theta_0) - \cos(\theta_0)} \\
				&= \frac{2}{\cos(\theta_0)}
			\end{align*}
			\item Soit $\alpha > 0$ suffisamment petit pour que $\alpha \sum_{n=0}^N |R_n| < \epsilon$. Si $z \in \Delta_{\theta_0}$ tel que $|z-1| \leq \alpha$, alors on peut majorer le premier terme de $(**)$ :
			\[ |z-1| \left( \sum_{n=0}^N |R_n| \right) \leq \alpha \left( \sum_{n=0}^N |R_n| \right) < \epsilon \]
		\end{itemize}
		Donc, en faisant $z \longrightarrow 1$ tel que $z \in \Delta_{\theta_0}$ (on aura bien $\rho = |z-1| \leq \inf \{ \alpha, \cos(\theta_0) \}$), et en injectant les deux majorations trouvées dans $(**)$ :
		\[ |f(z)-S| \leq \epsilon + \epsilon \frac{2}{\cos(\theta_0)} = \epsilon \left(1 + \frac{2}{\cos(\theta_0)} \right) \]
		d'où le résultat.
	\end{demonstration}

	\begin{application}
		\[ \sum_{n=0}^{+\infty} \frac{(-1)^n}{(2n+1)} = \frac{\pi}{4} \]
	\end{application}

	\begin{demonstration}
		En appliquant le \cref{theoreme-d-abel-angulaire-1} :
		\begin{align*}
			\sum_{n=0}^{+\infty} \frac{(-1)^n}{(2n+1)} &= \lim_{\substack{x \rightarrow 1 \\ x < 1}} \sum_{n=0}^{+\infty} \frac{(-1)^n}{(2n+1)} x^n \\
			&= \lim_{\substack{x \rightarrow 1 \\ x < 1}} \arctan(x) \\
			&= \arctan(1) \\
			&= \frac{\pi}{4}
		\end{align*}
	\end{demonstration}

	\begin{application}
		\[ \sum_{n=0}^{+\infty} \frac{(-1)^{n-1}}{n} = \ln(2) \]
	\end{application}

	\begin{demonstration}
		Toujours en appliquant le \cref{theoreme-d-abel-angulaire-1} :
		\begin{align*}
			\sum_{n=0}^{+\infty} \frac{(-1)^{n-1}}{n} &= \lim_{\substack{x \rightarrow 1 \\ x < 1}} \sum_{n=0}^{+\infty} \frac{(-1)^{n-1}}{n} x^n \\
			&= \lim_{\substack{x \rightarrow 1 \\ x < 1}} \ln(1 + x) \\
			&= \ln(2)
		\end{align*}
	\end{demonstration}
	%</content>
\end{document}
