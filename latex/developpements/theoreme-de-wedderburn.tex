\documentclass[12pt, a4paper]{report}

% LuaLaTeX :

\RequirePackage{iftex}
\RequireLuaTeX

% Packages :

\usepackage[french]{babel}
%\usepackage[utf8]{inputenc}
%\usepackage[T1]{fontenc}
\usepackage[pdfencoding=auto, pdfauthor={Hugo Delaunay}, pdfsubject={Mathématiques}, pdfcreator={agreg.skyost.eu}]{hyperref}
\usepackage{amsmath}
\usepackage{amsthm}
%\usepackage{amssymb}
\usepackage{stmaryrd}
\usepackage{tikz}
\usepackage{tkz-euclide}
\usepackage{fourier-otf}
\usepackage{fontspec}
\usepackage{titlesec}
\usepackage{fancyhdr}
\usepackage{catchfilebetweentags}
\usepackage[french, capitalise, noabbrev]{cleveref}
\usepackage[fit, breakall]{truncate}
\usepackage[top=2.5cm, right=2cm, bottom=2.5cm, left=2cm]{geometry}
\usepackage{enumerate}
\usepackage{tocloft}
\usepackage{microtype}
%\usepackage{mdframed}
%\usepackage{thmtools}
\usepackage{xcolor}
\usepackage{tabularx}
\usepackage{aligned-overset}
\usepackage[subpreambles=true]{standalone}
\usepackage{environ}
\usepackage[normalem]{ulem}
\usepackage{marginnote}
\usepackage{etoolbox}
\usepackage{setspace}
\usepackage[bibstyle=reading, citestyle=draft]{biblatex}
\usepackage{xpatch}
\usepackage[many, breakable]{tcolorbox}
\usepackage[backgroundcolor=white, bordercolor=white, textsize=small]{todonotes}

% Bibliographie :

\newcommand{\overridebibliographypath}[1]{\providecommand{\bibliographypath}{#1}}
\overridebibliographypath{../bibliography.bib}
\addbibresource{\bibliographypath}
\defbibheading{bibliography}[\bibname]{%
	\newpage
	\section*{#1}%
}
\renewbibmacro*{entryhead:full}{\printfield{labeltitle}}
\DeclareFieldFormat{url}{\newline\footnotesize\url{#1}}
\AtEndDocument{\printbibliography}

% Police :

\setmathfont{Erewhon Math}

% Tikz :

\usetikzlibrary{calc}

% Longueurs :

\setlength{\parindent}{0pt}
\setlength{\headheight}{15pt}
\setlength{\fboxsep}{0pt}
\titlespacing*{\chapter}{0pt}{-20pt}{10pt}
\setlength{\marginparwidth}{1.5cm}
\setstretch{1.1}

% Métadonnées :

\author{agreg.skyost.eu}
\date{\today}

% Titres :

\setcounter{secnumdepth}{3}

\renewcommand{\thechapter}{\Roman{chapter}}
\renewcommand{\thesubsection}{\Roman{subsection}}
\renewcommand{\thesubsubsection}{\arabic{subsubsection}}
\renewcommand{\theparagraph}{\alph{paragraph}}

\titleformat{\chapter}{\huge\bfseries}{\thechapter}{20pt}{\huge\bfseries}
\titleformat*{\section}{\LARGE\bfseries}
\titleformat{\subsection}{\Large\bfseries}{\thesubsection \, - \,}{0pt}{\Large\bfseries}
\titleformat{\subsubsection}{\large\bfseries}{\thesubsubsection. \,}{0pt}{\large\bfseries}
\titleformat{\paragraph}{\bfseries}{\theparagraph. \,}{0pt}{\bfseries}

\setcounter{secnumdepth}{4}

% Table des matières :

\renewcommand{\cftsecleader}{\cftdotfill{\cftdotsep}}
\addtolength{\cftsecnumwidth}{10pt}

% Redéfinition des commandes :

\renewcommand*\thesection{\arabic{section}}
\renewcommand{\ker}{\mathrm{Ker}}

% Nouvelles commandes :

\newcommand{\website}{https://agreg.skyost.eu}

\newcommand{\tr}[1]{\mathstrut ^t #1}
\newcommand{\im}{\mathrm{Im}}
\newcommand{\rang}{\operatorname{rang}}
\newcommand{\trace}{\operatorname{trace}}
\newcommand{\id}{\operatorname{id}}
\newcommand{\stab}{\operatorname{Stab}}

\providecommand{\newpar}{\\[\medskipamount]}

\providecommand{\lesson}[3]{%
	\title{#3}%
	\hypersetup{pdftitle={#3}}%
	\setcounter{section}{\numexpr #2 - 1}%
	\section{#3}%
	\fancyhead[R]{\truncate{0.73\textwidth}{#2 : #3}}%
}

\providecommand{\development}[3]{%
	\title{#3}%
	\hypersetup{pdftitle={#3}}%
	\section*{#3}%
	\fancyhead[R]{\truncate{0.73\textwidth}{#3}}%
}

\providecommand{\summary}[1]{%
	\textit{#1}%
	\medskip%
}

\tikzset{notestyleraw/.append style={inner sep=0pt, rounded corners=0pt, align=center}}

%\newcommand{\booklink}[1]{\website/bibliographie\##1}
\newcommand{\citelink}[2]{\hyperlink{cite.\therefsection @#1}{#2}}
\newcommand{\previousreference}{}
\providecommand{\reference}[2][]{%
	\notblank{#1}{\renewcommand{\previousreference}{#1}}{}%
	\todo[noline]{%
		\protect\vspace{16pt}%
		\protect\par%
		\protect\notblank{#1}{\cite{[\previousreference]}\\}{}%
		\protect\citelink{\previousreference}{p. #2}%
	}%
}

\definecolor{devcolor}{HTML}{00695c}
\newcommand{\dev}[1]{%
	\reversemarginpar%
	\todo[noline]{
		\protect\vspace{16pt}%
		\protect\par%
		\bfseries\color{devcolor}\href{\website/developpements/#1}{DEV}
	}%
	\normalmarginpar%
}

% En-têtes :

\pagestyle{fancy}
\fancyhead[L]{\truncate{0.23\textwidth}{\thepage}}
\fancyfoot[C]{\scriptsize \href{\website}{\texttt{agreg.skyost.eu}}}

% Couleurs :

\definecolor{property}{HTML}{fffde7}
\definecolor{proposition}{HTML}{fff8e1}
\definecolor{lemma}{HTML}{fff3e0}
\definecolor{theorem}{HTML}{fce4f2}
\definecolor{corollary}{HTML}{ffebee}
\definecolor{definition}{HTML}{ede7f6}
\definecolor{notation}{HTML}{f3e5f5}
\definecolor{example}{HTML}{e0f7fa}
\definecolor{cexample}{HTML}{efebe9}
\definecolor{application}{HTML}{e0f2f1}
\definecolor{remark}{HTML}{e8f5e9}
\definecolor{proof}{HTML}{e1f5fe}

% Théorèmes :

\theoremstyle{definition}
\newtheorem{theorem}{Théorème}

\newtheorem{property}[theorem]{Propriété}
\newtheorem{proposition}[theorem]{Proposition}
\newtheorem{lemma}[theorem]{Lemme}
\newtheorem{corollary}[theorem]{Corollaire}

\newtheorem{definition}[theorem]{Définition}
\newtheorem{notation}[theorem]{Notation}

\newtheorem{example}[theorem]{Exemple}
\newtheorem{cexample}[theorem]{Contre-exemple}
\newtheorem{application}[theorem]{Application}

\theoremstyle{remark}
\newtheorem{remark}[theorem]{Remarque}

\counterwithin*{theorem}{section}

\newcommand{\applystyletotheorem}[1]{
	\tcolorboxenvironment{#1}{
		enhanced,
		breakable,
		colback=#1!98!white,
		boxrule=0pt,
		boxsep=0pt,
		left=8pt,
		right=8pt,
		top=8pt,
		bottom=8pt,
		sharp corners,
		after=\par,
	}
}

\applystyletotheorem{property}
\applystyletotheorem{proposition}
\applystyletotheorem{lemma}
\applystyletotheorem{theorem}
\applystyletotheorem{corollary}
\applystyletotheorem{definition}
\applystyletotheorem{notation}
\applystyletotheorem{example}
\applystyletotheorem{cexample}
\applystyletotheorem{application}
\applystyletotheorem{remark}
\applystyletotheorem{proof}

% Environnements :

\NewEnviron{whitetabularx}[1]{%
	\renewcommand{\arraystretch}{2.5}
	\colorbox{white}{%
		\begin{tabularx}{\textwidth}{#1}%
			\BODY%
		\end{tabularx}%
	}%
}

% Maths :

\DeclareFontEncoding{FMS}{}{}
\DeclareFontSubstitution{FMS}{futm}{m}{n}
\DeclareFontEncoding{FMX}{}{}
\DeclareFontSubstitution{FMX}{futm}{m}{n}
\DeclareSymbolFont{fouriersymbols}{FMS}{futm}{m}{n}
\DeclareSymbolFont{fourierlargesymbols}{FMX}{futm}{m}{n}
\DeclareMathDelimiter{\VERT}{\mathord}{fouriersymbols}{152}{fourierlargesymbols}{147}



\begin{document}
	%<*content>
	\development{algebra}{theoreme-de-wedderburn}{Théorème de Wedderburn}

	\summary{En utilisant les polynômes cyclotomiques, nous montrons que tout corps fini est commutatif.}

	\reference[GOU21]{100}

	\begin{lemma}
		\label{theoreme-de-wedderburn-1}
		Soient $\mathbb{K}$ et $\mathbb{L}$ deux corps tels que $\mathbb{K}$ est commutatif et $\mathbb{K} \subset \mathbb{L}$. Alors $\exists d \in \mathbb{N}^*$ tel que $|\mathbb{L}| = |\mathbb{K}|^d$.
	\end{lemma}

	\begin{demonstration}
		$\mathbb{L}$ est un espace vectoriel sur $\mathbb{K}$ de dimension finie $d$ (car $\mathbb{L}$ est fini). Donc $\mathbb{L}$ est isomorphe en tant que $\mathbb{K}$-espace vectoriel à $\mathbb{K}^d$. En particulier, $|\mathbb{L}| = |\mathbb{K}|^d$.
	\end{demonstration}

	\begin{theorem}[Wedderburn]
		Tout corps fini est commutatif.
	\end{theorem}

	\begin{demonstration}
		Soit $\mathbb{K}$ un corps. L'idée va être de procéder par récurrence sur le cardinal du corps.
		\begin{itemize}
			\item \underLine{Si $|\mathbb{K}| = 2$ :} alors $\mathbb{K} = \{0, 1\}$ est commutatif.
			\item \underLine{On suppose le résultat vrai pour tout corps fini de cardinal strictement inférieur à $\mathbb{K}$.} On veut montrer que $\mathbb{K}$ est commutatif. Supposons par l'absurde que $\mathbb{K}$ ne l'est pas. On pose
			\[ Z = Z(\mathbb{K}) = \{ x \in \mathbb{K} \mid \forall y \in \mathbb{K}, \, xy = yx \} \]
			le centre de $\mathbb{K}$ dont on note $q$ le cardinal. C'est un sous-corps de $\mathbb{K}$ qui est (par hypothèse) inclus strictement dans $\mathbb{K}$. Donc $Z$ est commutatif, et par le \cref{theoreme-de-wedderburn-1}, on peut écrire $|\mathbb{K}| = q^n$ où $n \in \mathbb{N}^*$. Si $x \in \mathbb{K}$, on pose
			\[ \mathbb{K}_x = Z_{\mathbb{K}}(\{ x \}) = \{ y \in \mathbb{K} \mid xy = yx \} \]
			Montrons que
			\[ \exists d \mid n \text{ tel que } |\mathbb{K}_x| = q^d \tag{$*$} \]
			Notons déjà encore une fois que $\mathbb{K}_x$ est un sous-corps de $\mathbb{K}$.
			\begin{itemize}
				\item Si $\mathbb{K}_x = \mathbb{K}$, on a $|\mathbb{K}_x| = |\mathbb{K}| = q^n$. Il suffit donc de prendre $d = n$.
				\item Sinon, $\mathbb{K}_x \subsetneq \mathbb{K}$, donc $\mathbb{K}_x$ est commutatif par hypothèse. Par le \cref{theoreme-de-wedderburn-1}, il existe $k \in \mathbb{N}^*$ tel que $|\mathbb{K}| = |\mathbb{K}_x|^k$.
				\newpar
				Mais, $Z$ est un sous-corps (commutatif) de $\mathbb{K}_x$, donc d'après le \cref{theoreme-de-wedderburn-1}, il existe $d \in \mathbb{N}^*$ tel que $|\mathbb{K}_x| = |Z|^d$. Donc on a
				\[ q^n = |\mathbb{K}| = |\mathbb{K}_x|^k = (q^d)^k = q^{dk} \]
				d'où $d \mid n$.
			\end{itemize}
			On considère l'action par conjugaison $\cdot$ de $\mathbb{K}$ sur lui-même $x \cdot y = xyx^{-1}$. Si $y \in \mathbb{K}^*$, alors
			\[ \mathrm{Stab}_y = \{ x \in \mathbb{K}^* \mid x.y = y \} = \mathbb{K}_y^* \]
			Soit $\Omega$ un système de représentants associé à la relation d'équivalence ``être dans la même orbite''. L'équation aux classes donne alors
			\[ |\mathbb{K}^*| = \sum_{\omega \in \Omega} \frac{|\mathbb{K}^*|}{|\mathrm{Stab}_\omega|} \]
			Or,
			\[ \mathrm{Stab}_\omega = \mathbb{K}^* \iff \forall x \in \mathbb{K}^*, \, \omega x = x \omega \iff \omega \in Z^* \]
			donc en notant $\Omega' = \Omega \setminus Z^*$, on a :
			\[ |\mathbb{K}^*| = \sum_{\omega \in Z^*} \frac{|\mathbb{K}^*|}{|\mathrm{Stab}_\omega|} + \sum_{\omega \in \Omega'} \frac{|\mathbb{K}^*|}{|\mathrm{Stab}_\omega|} = |Z^*| + \sum_{\omega \in \Omega} \frac{|\mathbb{K}^*|}{|\mathbb{K}^*_\omega|} \tag{$**$} \]
			Soit $\omega \in \Omega'$. Par $(*)$,
			\[ \exists d \mid n \text{ tel que } |\mathrm{Stab}_\omega| = |\mathbb{K}^*_\omega| = q^d - 1 \]
			De plus, $d \neq n$ (car $\omega \notin Z^*$). Si maintenant on pose
			\[ \forall d \mid n, \, \lambda_d = |\{ \omega \in \Omega' \mid |\mathrm{Stab}_\omega| = q^d - 1 \}| \]
			on peut alors écrire en remplaçant dans $(**)$ :
			\[ q^n - 1 = |K^*| = (q - 1) + \sum_{d \mid \mid n} \lambda_d \left( \frac{q^n - 1}{q^d - 1} \right) \tag{$***$} \]
			Si $d \mid \mid n$, on a
			\[ X^n-1 = \prod_{k \mid n} \Phi_k = \Phi_n \left ( \prod_{k \mid d} \Phi_k \right ) \left ( \prod_{\substack{k \mid \mid n \\ k \nmid d}} \Phi_k \right ) = \Phi_n (X^m - 1) \left ( \prod_{\substack{k \mid \mid n \\ k \nmid d}} \Phi_k \right ) \]
			Donc, $\Phi_n \mid \frac{X^n - 1}{X^d - 1}$ dans $\mathbb{Z}[X]$. Ceci étant vrai quelque soit $d$ divisant strictement $n$, on en déduit
			\[ \Phi_n \mid \sum_{d \mid \mid n} \lambda_d \frac{X^n - 1}{X^d - 1} \text{ dans } \mathbb{Z}[X] \]
			Comme de plus, $\Phi_n \mid X^n - 1$ dans $\mathbb{Z}[X]$, on conclut que
			\[ \Phi_n \mid X^n - 1 - \sum_{d \mid \mid n} \lambda_d \frac{X^n - 1}{X^d - 1} \text{ dans } \mathbb{Z}[X] \]
			ce qui donne, une fois évalué en $q$ :
			\[ \Phi_n(q) \mid q^n - 1 - \sum_{d \mid \mid n} \lambda_d \frac{q^n - 1}{q^d - 1} \overset{(***)}{=} q-1 \implies |\Phi_n(q)| \leq q-1 \]
			Mais $n \geq 2$, donc
			\begin{align*}
				|\Phi_n(q)| &= \prod_{\xi \in \mu_n^*} |q - \xi| \\
				&> \prod_{i=1}^{\varphi(n)} |q - 1| \\
				&\geq |q-1|
			\end{align*}
			On peut en effet interpréter $|q - \xi|$ comme la distance du complexe $q$ au complexe $\xi$ ; le premier est sur l'axe réel et est $\geq 2$, le second est sur le cercle unité mais n'est pas sur l'axe réel :
			\includelatexpicture{theoreme-de-wedderburn}
			cela nous permet de justifier l'inégalité stricte. On a donc une contradiction.
		\end{itemize}
	\end{demonstration}
	%</content>
\end{document}
