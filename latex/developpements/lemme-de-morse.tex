\documentclass[12pt, a4paper]{report}

% LuaLaTeX :

\RequirePackage{iftex}
\RequireLuaTeX

% Packages :

\usepackage[french]{babel}
%\usepackage[utf8]{inputenc}
%\usepackage[T1]{fontenc}
\usepackage[pdfencoding=auto, pdfauthor={Hugo Delaunay}, pdfsubject={Mathématiques}, pdfcreator={agreg.skyost.eu}]{hyperref}
\usepackage{amsmath}
\usepackage{amsthm}
%\usepackage{amssymb}
\usepackage{stmaryrd}
\usepackage{tikz}
\usepackage{tkz-euclide}
\usepackage{fourier-otf}
\usepackage{fontspec}
\usepackage{titlesec}
\usepackage{fancyhdr}
\usepackage{catchfilebetweentags}
\usepackage[french, capitalise, noabbrev]{cleveref}
\usepackage[fit, breakall]{truncate}
\usepackage[top=2.5cm, right=2cm, bottom=2.5cm, left=2cm]{geometry}
\usepackage{enumerate}
\usepackage{tocloft}
\usepackage{microtype}
%\usepackage{mdframed}
%\usepackage{thmtools}
\usepackage{xcolor}
\usepackage{tabularx}
\usepackage{aligned-overset}
\usepackage[subpreambles=true]{standalone}
\usepackage{environ}
\usepackage[normalem]{ulem}
\usepackage{marginnote}
\usepackage{etoolbox}
\usepackage{setspace}
\usepackage[bibstyle=reading, citestyle=draft]{biblatex}
\usepackage{xpatch}
\usepackage[many, breakable]{tcolorbox}
\usepackage[backgroundcolor=white, bordercolor=white, textsize=small]{todonotes}

% Bibliographie :

\newcommand{\overridebibliographypath}[1]{\providecommand{\bibliographypath}{#1}}
\overridebibliographypath{../bibliography.bib}
\addbibresource{\bibliographypath}
\defbibheading{bibliography}[\bibname]{%
	\newpage
	\section*{#1}%
}
\renewbibmacro*{entryhead:full}{\printfield{labeltitle}}
\DeclareFieldFormat{url}{\newline\footnotesize\url{#1}}
\AtEndDocument{\printbibliography}

% Police :

\setmathfont{Erewhon Math}

% Tikz :

\usetikzlibrary{calc}

% Longueurs :

\setlength{\parindent}{0pt}
\setlength{\headheight}{15pt}
\setlength{\fboxsep}{0pt}
\titlespacing*{\chapter}{0pt}{-20pt}{10pt}
\setlength{\marginparwidth}{1.5cm}
\setstretch{1.1}

% Métadonnées :

\author{agreg.skyost.eu}
\date{\today}

% Titres :

\setcounter{secnumdepth}{3}

\renewcommand{\thechapter}{\Roman{chapter}}
\renewcommand{\thesubsection}{\Roman{subsection}}
\renewcommand{\thesubsubsection}{\arabic{subsubsection}}
\renewcommand{\theparagraph}{\alph{paragraph}}

\titleformat{\chapter}{\huge\bfseries}{\thechapter}{20pt}{\huge\bfseries}
\titleformat*{\section}{\LARGE\bfseries}
\titleformat{\subsection}{\Large\bfseries}{\thesubsection \, - \,}{0pt}{\Large\bfseries}
\titleformat{\subsubsection}{\large\bfseries}{\thesubsubsection. \,}{0pt}{\large\bfseries}
\titleformat{\paragraph}{\bfseries}{\theparagraph. \,}{0pt}{\bfseries}

\setcounter{secnumdepth}{4}

% Table des matières :

\renewcommand{\cftsecleader}{\cftdotfill{\cftdotsep}}
\addtolength{\cftsecnumwidth}{10pt}

% Redéfinition des commandes :

\renewcommand*\thesection{\arabic{section}}
\renewcommand{\ker}{\mathrm{Ker}}

% Nouvelles commandes :

\newcommand{\website}{https://agreg.skyost.eu}

\newcommand{\tr}[1]{\mathstrut ^t #1}
\newcommand{\im}{\mathrm{Im}}
\newcommand{\rang}{\operatorname{rang}}
\newcommand{\trace}{\operatorname{trace}}
\newcommand{\id}{\operatorname{id}}
\newcommand{\stab}{\operatorname{Stab}}

\providecommand{\newpar}{\\[\medskipamount]}

\providecommand{\lesson}[3]{%
	\title{#3}%
	\hypersetup{pdftitle={#3}}%
	\setcounter{section}{\numexpr #2 - 1}%
	\section{#3}%
	\fancyhead[R]{\truncate{0.73\textwidth}{#2 : #3}}%
}

\providecommand{\development}[3]{%
	\title{#3}%
	\hypersetup{pdftitle={#3}}%
	\section*{#3}%
	\fancyhead[R]{\truncate{0.73\textwidth}{#3}}%
}

\providecommand{\summary}[1]{%
	\textit{#1}%
	\medskip%
}

\tikzset{notestyleraw/.append style={inner sep=0pt, rounded corners=0pt, align=center}}

%\newcommand{\booklink}[1]{\website/bibliographie\##1}
\newcommand{\citelink}[2]{\hyperlink{cite.\therefsection @#1}{#2}}
\newcommand{\previousreference}{}
\providecommand{\reference}[2][]{%
	\notblank{#1}{\renewcommand{\previousreference}{#1}}{}%
	\todo[noline]{%
		\protect\vspace{16pt}%
		\protect\par%
		\protect\notblank{#1}{\cite{[\previousreference]}\\}{}%
		\protect\citelink{\previousreference}{p. #2}%
	}%
}

\definecolor{devcolor}{HTML}{00695c}
\newcommand{\dev}[1]{%
	\reversemarginpar%
	\todo[noline]{
		\protect\vspace{16pt}%
		\protect\par%
		\bfseries\color{devcolor}\href{\website/developpements/#1}{DEV}
	}%
	\normalmarginpar%
}

% En-têtes :

\pagestyle{fancy}
\fancyhead[L]{\truncate{0.23\textwidth}{\thepage}}
\fancyfoot[C]{\scriptsize \href{\website}{\texttt{agreg.skyost.eu}}}

% Couleurs :

\definecolor{property}{HTML}{fffde7}
\definecolor{proposition}{HTML}{fff8e1}
\definecolor{lemma}{HTML}{fff3e0}
\definecolor{theorem}{HTML}{fce4f2}
\definecolor{corollary}{HTML}{ffebee}
\definecolor{definition}{HTML}{ede7f6}
\definecolor{notation}{HTML}{f3e5f5}
\definecolor{example}{HTML}{e0f7fa}
\definecolor{cexample}{HTML}{efebe9}
\definecolor{application}{HTML}{e0f2f1}
\definecolor{remark}{HTML}{e8f5e9}
\definecolor{proof}{HTML}{e1f5fe}

% Théorèmes :

\theoremstyle{definition}
\newtheorem{theorem}{Théorème}

\newtheorem{property}[theorem]{Propriété}
\newtheorem{proposition}[theorem]{Proposition}
\newtheorem{lemma}[theorem]{Lemme}
\newtheorem{corollary}[theorem]{Corollaire}

\newtheorem{definition}[theorem]{Définition}
\newtheorem{notation}[theorem]{Notation}

\newtheorem{example}[theorem]{Exemple}
\newtheorem{cexample}[theorem]{Contre-exemple}
\newtheorem{application}[theorem]{Application}

\theoremstyle{remark}
\newtheorem{remark}[theorem]{Remarque}

\counterwithin*{theorem}{section}

\newcommand{\applystyletotheorem}[1]{
	\tcolorboxenvironment{#1}{
		enhanced,
		breakable,
		colback=#1!98!white,
		boxrule=0pt,
		boxsep=0pt,
		left=8pt,
		right=8pt,
		top=8pt,
		bottom=8pt,
		sharp corners,
		after=\par,
	}
}

\applystyletotheorem{property}
\applystyletotheorem{proposition}
\applystyletotheorem{lemma}
\applystyletotheorem{theorem}
\applystyletotheorem{corollary}
\applystyletotheorem{definition}
\applystyletotheorem{notation}
\applystyletotheorem{example}
\applystyletotheorem{cexample}
\applystyletotheorem{application}
\applystyletotheorem{remark}
\applystyletotheorem{proof}

% Environnements :

\NewEnviron{whitetabularx}[1]{%
	\renewcommand{\arraystretch}{2.5}
	\colorbox{white}{%
		\begin{tabularx}{\textwidth}{#1}%
			\BODY%
		\end{tabularx}%
	}%
}

% Maths :

\DeclareFontEncoding{FMS}{}{}
\DeclareFontSubstitution{FMS}{futm}{m}{n}
\DeclareFontEncoding{FMX}{}{}
\DeclareFontSubstitution{FMX}{futm}{m}{n}
\DeclareSymbolFont{fouriersymbols}{FMS}{futm}{m}{n}
\DeclareSymbolFont{fourierlargesymbols}{FMX}{futm}{m}{n}
\DeclareMathDelimiter{\VERT}{\mathord}{fouriersymbols}{152}{fourierlargesymbols}{147}



\begin{document}
	%<*content>
	\development{analysis}{lemme-de-morse}{Lemme de Morse}

	\summary{En usant (certains diront plutôt ``en abusant'') du théorème d'inversion locale, on montre le lemme de Morse et on l'applique à l'étude de la position d’une surface par rapport à son plan tangent.}

	\begin{notation}
		Si $f : \mathbb{R}^n \rightarrow \mathbb{R}$ est une application dont toutes les dérivées secondes existent, on note $\mathrm{H}(f)_a$ la hessienne de $f$ au point $a$.
	\end{notation}

	\reference[ROU]{201}

	\begin{lemma}
		\label{lemme-de-morse-1}
		Soit $A_0 \in \mathcal{S}_n(\mathbb{R})$ inversible. Alors il existe un voisinage $V$ de $A_0$ dans $\mathcal{S}_n(\mathbb{R})$ et une application $\psi : V \rightarrow \mathrm{GL}_n(\mathbb{R})$ de classe $\mathcal{C}^1$ telle que
		\[ \forall A \in V, \, A = \tr \psi(A) A_0 \psi(A) \]
	\end{lemma}

	\begin{demonstration}
		On définit l'application
		\[ \varphi :
		\begin{array}{cl}
			\mathcal{M}_n(\mathbb{R}) &\rightarrow \mathcal{S}_n(\mathbb{R}) \\
			M &\mapsto \tr M A_0 M
		\end{array}
		\]
		qui est une application polynômiale en les coefficients de $M$, donc de classe $\mathcal{C}^1$. Soit $H \in \mathcal{M}_n(\mathbb{R})$. On calcule :
		\begin{align*}
			\varphi(I_n + H) - \varphi(I_n) &= \tr H A_0 + A_0 H + \tr H A_0 + H \\
			&= \tr (A_0 H) + A_0 H + o(\Vert H \Vert^2)
		\end{align*}
		où ($\Vert . \Vert$ désigne une quelconque norme d'algèbre sur $\mathcal{M}_n(\mathbb{R})$). Ainsi, on a $\mathrm{d} \varphi_{I_n}(H) = \tr(A_0 H) + A_0 H$. D'où
		\[ \ker(\mathrm{d} \varphi_{I_n}) = \{ M \in \mathcal{M}_n(\mathbb{R}) \mid A_0 M \in \mathcal{A}_n(\mathbb{R}) \} = A_0^{-1} \mathcal{A}_n(\mathbb{R}) \]
		On définit donc
		\[ F = \{ M \in \mathcal{M}_n(\mathbb{R}) \mid A_0 M \in \mathcal{S}_n(\mathbb{R}) \} = A_0^{-1} \mathcal{S}_n(\mathbb{R})  \]
		et on a $\mathcal{M}_n(\mathbb{R}) = F \oplus \ker(\mathrm{d} \varphi_{I_n})$. Ainsi, la différentielle $\mathrm{d} (\varphi_{|F})_{I_n}$ est bijective (car $\ker(\mathrm{d} (\varphi_{|F})_{I_n}) = \ker(\mathrm{d} \varphi_{I_n}) \, \cap \, F = \{ 0 \}$).
		\newpar
		On peut donc appliquer le théorème d'inversion locale à $\varphi_{|F}$ : il existe $U$ un voisinage ouvert de $I_n$ dans $F$ tel que $(\varphi_{|U})$ soit $\mathcal{C}^1$-difféomorphisme de $U$ sur $V = \varphi(U)$. De plus, on peut supposer $U \subset \mathrm{GL}_n(\mathbb{R})$ (quitte à considérer $U \, \cap \, U'$ où $U'$ est un voisinage ouvert de $I_n$ dans $\mathrm{GL}_n(\mathbb{R})$ ; qui existe par continuité de $\det$).
		\newpar
		Ainsi, $V$ est un voisinage ouvert de $A_0 = \varphi(I_n)$ dans $\mathcal{S}_n(\mathbb{R})$ vérifiant :
		\[ \forall A \in V, \, A = \tr (\varphi_{|U})^{-1}(A) A_0 (\varphi_{|U})^{-1}(A) \]
		Il suffit alors de poser $\psi = (\varphi_{|U})^{-1}$ (qui est bien une application de classe $\mathcal{C}^1$) pour avoir le résultat demandé.
	\end{demonstration}

	\reference{344}

	\begin{lemma}[Morse]
		\label{lemme-de-morse-2}
		Soit $f : U \rightarrow \mathbb{R}$ une fonction de classe $\mathcal{C}^3$ (où $U$ désigne un ouvert de $\mathbb{R}^n$ contenant l'origine). On suppose :
		\begin{itemize}
			\item $\mathrm{d} f_0 = 0$.
			\item La matrice symétrique $\mathrm{H} (f)_0$ est inversible.
			\item La signature de $\mathrm{H}(f)_0$ est $(p, n-p)$.
		\end{itemize}
		Alors il existe un difféomorphisme $\phi = (\phi_1, \dots, \phi_n)$ de classe $\mathcal{C}^1$ entre deux voisinage de l'origine de $\mathbb{R}^n$ $V \subset U$ et $W$ tel que $\varphi(0) = 0$ et
		\[ \forall x \in U, \, f(x) - f(0) = \sum_{k=1}^p \phi_k^2(x) - \sum_{k=p+1}^n \phi_k^2(x) \]
	\end{lemma}

	\begin{demonstration}
		On écrit la formule de Taylor à l'ordre $1$ avec reste intégral au voisinage de $0$, qui donne :
		\begin{align*}
			&f(x) = f(0) + \mathrm{d} f_0(x) + \int_0^1 (1-t) \mathrm{d}^2 (1-t) f_{tx} (x, x) \, \mathrm{d}t \\
			\iff& f(x) - f(0) = \tr x Q(x) x \tag{$*$}
		\end{align*}
		où $Q(x)$ est la matrice symétrique définie par $Q(x) = \int_0^1 (1-t) \mathrm{H} f_{tx} \, \mathrm{d}t$ (qui est une application $\mathcal{C}^1$ sur $U$ car $f$ est $\mathcal{C}^3$ sur $U$).
		\newpar
		Par hypothèse, $Q(0) = \frac{\mathrm{H} (f)_0}{2}$ est une matrice symétrique inversible, donc en vertu du \cref{lemme-de-morse-1}, il existe un voisinage $V_1$ de $Q(0)$ dans $\mathcal{S}_n(\mathbb{R})$ et une application $\psi : V_1 \rightarrow \mathrm{GL}_n(\mathbb{R})$ de classe $\mathcal{C}^1$ tels que :
		\[ \forall A \in V_1, \, Q(0) = \psi(A) Q(0) \psi(A) \]
		Mais, l'application $x \mapsto Q(x)$ est continue sur $U$ (puisque $f$ est de classe $\mathcal{C}^3$ sur $U$), donc il existe $V_2$ voisinage de $0$ dans $U$ tel que $\forall x \in V_2$, $Q(x) \in V_1$. On peut donc définir l'application $M = \psi \circ Q_{|V_2}$, qui nous permet d'écrire
		\[ \forall x \in V_2, \, Q(x) = \tr M (x) Q(0) M (x) \tag{$*$} \]
		Or, $Q(0)$ est de signature $(p, n-p)$, donc d'après la loi d'inertie de Sylvester, il existe $P \in \mathrm{GL}_n(\mathbb{R})$ telle que
		\[ Q(0) = \tr P \underbrace{\begin{pmatrix} I_p & \\ & - I_{n-p} \end{pmatrix}}_{= D} P \tag{$***$} \]
		Finalement en combinant $(*)$ avec $(**)$ et $(***)$, cela donne
		\begin{align*}
			&\forall x \in V_2, \, f(x) - f(0) = \tr (P M(x) x) D (P M(x) x) \\
			\overset{\varphi(x) = P M(x) x}{\iff}& \forall x \in V_2, \, f(x) - f(0) = \tr \varphi(x) D \varphi(x)
		\end{align*}
		ce qui est bien l'expression voulue.
		\newpar
		Il reste à montrer que $\varphi$ définit bien un difféomorphisme de classe $\mathcal{C}^1$ entre deux voisinages de l'origine. Notons déjà que $\varphi$ est de classe $\mathcal{C}^1$ car $M$ l'est. Calculons la différentielle en $0$ de $\varphi$. Soit $h \in V_2$ ;
		\begin{align*}
			\varphi(h) - \varphi(0) &= P M(h) h \\
			&= P( M(0) + \mathrm{d} M_0 (h) + o(\Vert h \Vert))h \\
			&= P M(0) h + o(\Vert h \Vert)
		\end{align*}
		d'où $\mathrm{d} \varphi_0 (h) = P M(0) h$. Or, $P M(0)$ est inversible, donc en particulier, $\mathrm{d} \varphi_0 (h)$ l'est aussi. On peut appliquer le théorème d'inversion locale à $\varphi$, qui donne l'existence de deux ouverts $V$ et $W$ contenant l'origine (car $\varphi(0) = 0$) tel que $\phi = \varphi_{|V}$ soit un $\mathcal{C}^1$-difféomorphisme de $V$ sur $W$.
	\end{demonstration}

	\reference{331}

	\begin{application}
		Soit $S$ la surface d'équation $z = f(x, y)$ où $f$ est de classe $\mathcal{C}^3$ au voisinage de l'origine. On suppose la forme quadratique $\mathrm{d}^2 f_0$ non dégénérée. Alors, en notant $P$ le plan tangent à $S$ en $0$ :
		\begin{enumerate}[(i)]
			\item Si $\mathrm{d}^2 f_0$ est de signature $(2, 0)$, alors $S$ est au-dessus de $P$ au voisinage de $0$.
			\item Si $\mathrm{d}^2 f_0$ est de signature $(0, 2)$, alors $S$ est en-dessous de $P$ au voisinage de $0$.
			\item Si $\mathrm{d}^2 f_0$ est de signature $(1, 1)$, alors $S$ traverse $P$ selon une courbe admettant un point double en $(0, f(0))$.
		\end{enumerate}
	\end{application}

	\begin{demonstration}
		Une équation cartésienne de $P$ est donnée par
		\[ z - 0 = f(0) + \mathrm{d} f_0(x, y) \]
		La différence d'altitude entre la surface $S$ et le plan tangent $P$ au point $h \in \mathbb{R}^2$ est donc donnée par
		\[ \delta(h) = f(h) - (f(0) + \mathrm{d}f_0(h)) \]
		et le \cref{lemme-de-morse-2} permet d'écrire
		\[ \delta(h) = \alpha \phi_1(h)^2 + \beta \phi_2(h)^2 \]
		où $(\alpha, \beta)$ désigne la signature de $\mathrm{d}^2 f_0$ et $\phi = (\phi_1, \phi_2)$ est un $\mathcal{C}^1$-difféomorphisme entre deux voisinages de l'origine dans $\mathbb{R}^2$. En particulier, $\phi_1$ et $\phi_2$ ne s'annulent simultanément qu'en $0$.
		\begin{enumerate}[(i)]
			\item Si $\mathrm{d}^2 f_a$ est de signature $(2, 0)$, on a $\delta(h) > 0$ pour $h$ voisin de $0$ et $h \neq 0$.
			\item Si $\mathrm{d}^2 f_a$ est de signature $(0, 2)$, on a $\delta(h) < 0$ pour $h$ voisin de $0$ et $h \neq 0$.
			\item Si $\mathrm{d}^2 f_a$ est de signature $(1, 1)$, on a $\delta(h) = \phi_1(h)^2 - \phi_2(h)^2$ et $S$ traverse $P$ selon une courbe admettant un point double en $(0, f(0))$.
		\end{enumerate}
	\end{demonstration}
	%</content>
\end{document}
