\documentclass[12pt, a4paper]{report}

% LuaLaTeX :

\RequirePackage{iftex}
\RequireLuaTeX

% Packages :

\usepackage[french]{babel}
%\usepackage[utf8]{inputenc}
%\usepackage[T1]{fontenc}
\usepackage[pdfencoding=auto, pdfauthor={Hugo Delaunay}, pdfsubject={Mathématiques}, pdfcreator={agreg.skyost.eu}]{hyperref}
\usepackage{amsmath}
\usepackage{amsthm}
%\usepackage{amssymb}
\usepackage{stmaryrd}
\usepackage{tikz}
\usepackage{tkz-euclide}
\usepackage{fourier-otf}
\usepackage{fontspec}
\usepackage{titlesec}
\usepackage{fancyhdr}
\usepackage{catchfilebetweentags}
\usepackage[french, capitalise, noabbrev]{cleveref}
\usepackage[fit, breakall]{truncate}
\usepackage[top=2.5cm, right=2cm, bottom=2.5cm, left=2cm]{geometry}
\usepackage{enumerate}
\usepackage{tocloft}
\usepackage{microtype}
%\usepackage{mdframed}
%\usepackage{thmtools}
\usepackage{xcolor}
\usepackage{tabularx}
\usepackage{aligned-overset}
\usepackage[subpreambles=true]{standalone}
\usepackage{environ}
\usepackage[normalem]{ulem}
\usepackage{marginnote}
\usepackage{etoolbox}
\usepackage{setspace}
\usepackage[bibstyle=reading, citestyle=draft]{biblatex}
\usepackage{xpatch}
\usepackage[many, breakable]{tcolorbox}
\usepackage[backgroundcolor=white, bordercolor=white, textsize=small]{todonotes}

% Bibliographie :

\newcommand{\overridebibliographypath}[1]{\providecommand{\bibliographypath}{#1}}
\overridebibliographypath{../bibliography.bib}
\addbibresource{\bibliographypath}
\defbibheading{bibliography}[\bibname]{%
	\newpage
	\section*{#1}%
}
\renewbibmacro*{entryhead:full}{\printfield{labeltitle}}
\DeclareFieldFormat{url}{\newline\footnotesize\url{#1}}
\AtEndDocument{\printbibliography}

% Police :

\setmathfont{Erewhon Math}

% Tikz :

\usetikzlibrary{calc}

% Longueurs :

\setlength{\parindent}{0pt}
\setlength{\headheight}{15pt}
\setlength{\fboxsep}{0pt}
\titlespacing*{\chapter}{0pt}{-20pt}{10pt}
\setlength{\marginparwidth}{1.5cm}
\setstretch{1.1}

% Métadonnées :

\author{agreg.skyost.eu}
\date{\today}

% Titres :

\setcounter{secnumdepth}{3}

\renewcommand{\thechapter}{\Roman{chapter}}
\renewcommand{\thesubsection}{\Roman{subsection}}
\renewcommand{\thesubsubsection}{\arabic{subsubsection}}
\renewcommand{\theparagraph}{\alph{paragraph}}

\titleformat{\chapter}{\huge\bfseries}{\thechapter}{20pt}{\huge\bfseries}
\titleformat*{\section}{\LARGE\bfseries}
\titleformat{\subsection}{\Large\bfseries}{\thesubsection \, - \,}{0pt}{\Large\bfseries}
\titleformat{\subsubsection}{\large\bfseries}{\thesubsubsection. \,}{0pt}{\large\bfseries}
\titleformat{\paragraph}{\bfseries}{\theparagraph. \,}{0pt}{\bfseries}

\setcounter{secnumdepth}{4}

% Table des matières :

\renewcommand{\cftsecleader}{\cftdotfill{\cftdotsep}}
\addtolength{\cftsecnumwidth}{10pt}

% Redéfinition des commandes :

\renewcommand*\thesection{\arabic{section}}
\renewcommand{\ker}{\mathrm{Ker}}

% Nouvelles commandes :

\newcommand{\website}{https://agreg.skyost.eu}

\newcommand{\tr}[1]{\mathstrut ^t #1}
\newcommand{\im}{\mathrm{Im}}
\newcommand{\rang}{\operatorname{rang}}
\newcommand{\trace}{\operatorname{trace}}
\newcommand{\id}{\operatorname{id}}
\newcommand{\stab}{\operatorname{Stab}}

\providecommand{\newpar}{\\[\medskipamount]}

\providecommand{\lesson}[3]{%
	\title{#3}%
	\hypersetup{pdftitle={#3}}%
	\setcounter{section}{\numexpr #2 - 1}%
	\section{#3}%
	\fancyhead[R]{\truncate{0.73\textwidth}{#2 : #3}}%
}

\providecommand{\development}[3]{%
	\title{#3}%
	\hypersetup{pdftitle={#3}}%
	\section*{#3}%
	\fancyhead[R]{\truncate{0.73\textwidth}{#3}}%
}

\providecommand{\summary}[1]{%
	\textit{#1}%
	\medskip%
}

\tikzset{notestyleraw/.append style={inner sep=0pt, rounded corners=0pt, align=center}}

%\newcommand{\booklink}[1]{\website/bibliographie\##1}
\newcommand{\citelink}[2]{\hyperlink{cite.\therefsection @#1}{#2}}
\newcommand{\previousreference}{}
\providecommand{\reference}[2][]{%
	\notblank{#1}{\renewcommand{\previousreference}{#1}}{}%
	\todo[noline]{%
		\protect\vspace{16pt}%
		\protect\par%
		\protect\notblank{#1}{\cite{[\previousreference]}\\}{}%
		\protect\citelink{\previousreference}{p. #2}%
	}%
}

\definecolor{devcolor}{HTML}{00695c}
\newcommand{\dev}[1]{%
	\reversemarginpar%
	\todo[noline]{
		\protect\vspace{16pt}%
		\protect\par%
		\bfseries\color{devcolor}\href{\website/developpements/#1}{DEV}
	}%
	\normalmarginpar%
}

% En-têtes :

\pagestyle{fancy}
\fancyhead[L]{\truncate{0.23\textwidth}{\thepage}}
\fancyfoot[C]{\scriptsize \href{\website}{\texttt{agreg.skyost.eu}}}

% Couleurs :

\definecolor{property}{HTML}{fffde7}
\definecolor{proposition}{HTML}{fff8e1}
\definecolor{lemma}{HTML}{fff3e0}
\definecolor{theorem}{HTML}{fce4f2}
\definecolor{corollary}{HTML}{ffebee}
\definecolor{definition}{HTML}{ede7f6}
\definecolor{notation}{HTML}{f3e5f5}
\definecolor{example}{HTML}{e0f7fa}
\definecolor{cexample}{HTML}{efebe9}
\definecolor{application}{HTML}{e0f2f1}
\definecolor{remark}{HTML}{e8f5e9}
\definecolor{proof}{HTML}{e1f5fe}

% Théorèmes :

\theoremstyle{definition}
\newtheorem{theorem}{Théorème}

\newtheorem{property}[theorem]{Propriété}
\newtheorem{proposition}[theorem]{Proposition}
\newtheorem{lemma}[theorem]{Lemme}
\newtheorem{corollary}[theorem]{Corollaire}

\newtheorem{definition}[theorem]{Définition}
\newtheorem{notation}[theorem]{Notation}

\newtheorem{example}[theorem]{Exemple}
\newtheorem{cexample}[theorem]{Contre-exemple}
\newtheorem{application}[theorem]{Application}

\theoremstyle{remark}
\newtheorem{remark}[theorem]{Remarque}

\counterwithin*{theorem}{section}

\newcommand{\applystyletotheorem}[1]{
	\tcolorboxenvironment{#1}{
		enhanced,
		breakable,
		colback=#1!98!white,
		boxrule=0pt,
		boxsep=0pt,
		left=8pt,
		right=8pt,
		top=8pt,
		bottom=8pt,
		sharp corners,
		after=\par,
	}
}

\applystyletotheorem{property}
\applystyletotheorem{proposition}
\applystyletotheorem{lemma}
\applystyletotheorem{theorem}
\applystyletotheorem{corollary}
\applystyletotheorem{definition}
\applystyletotheorem{notation}
\applystyletotheorem{example}
\applystyletotheorem{cexample}
\applystyletotheorem{application}
\applystyletotheorem{remark}
\applystyletotheorem{proof}

% Environnements :

\NewEnviron{whitetabularx}[1]{%
	\renewcommand{\arraystretch}{2.5}
	\colorbox{white}{%
		\begin{tabularx}{\textwidth}{#1}%
			\BODY%
		\end{tabularx}%
	}%
}

% Maths :

\DeclareFontEncoding{FMS}{}{}
\DeclareFontSubstitution{FMS}{futm}{m}{n}
\DeclareFontEncoding{FMX}{}{}
\DeclareFontSubstitution{FMX}{futm}{m}{n}
\DeclareSymbolFont{fouriersymbols}{FMS}{futm}{m}{n}
\DeclareSymbolFont{fourierlargesymbols}{FMX}{futm}{m}{n}
\DeclareMathDelimiter{\VERT}{\mathord}{fouriersymbols}{152}{fourierlargesymbols}{147}



\begin{document}
	%<*content>
	\development{algebra}{invariants-de-similitude}{Invariants de similitude}

	\summary{Nous montrons l'existence et l'unicité des invariants de similitude d'un endomorphisme d'un espace de dimension finie en utilisant la dualité.}

	\reference[GOU21]{398}

	Soit $E$ un espace vectoriel de dimension finie $n \geq 1$ sur un corps commutatif $\mathbb{K}$. Soit $f \in \mathcal{L}(E)$.

	\begin{notation}
		Soit $x \in E$. On note $P_x$ le polynôme unitaire engendrant l'idéal $\{ P \in \mathbb{K}[X] \mid P(f)(x) = 0 \}$ (un tel polynôme existe car $\mathbb{K}[X]$ est principal et cet idéal est non réduit à $\{ 0 \}$) et $E_x = \{ P(f)(x) \mid P \in \mathbb{K}[X] \}$.
	\end{notation}

	\begin{lemma}
		\label{invariants-de-similitude-1}
		\begin{enumerate}[(i)]
			\item Si $k = \deg(\mu_f)$, alors $\mathbb{K}[f]$ est un sous-espace vectoriel de $\mathcal{L}(E)$ de dimension $k$, dont une base est $(f^i)_{i \in \llbracket 0, k-1 \rrbracket}$.
			\item Soit $x \in E$. Si $l = \deg(P_x)$, alors $E_x$ est un sous-espace vectoriel de $E$ de dimension $l$, dont une base est $(f^i(x))_{i \in \llbracket 0, l-1 \rrbracket}$.
		\end{enumerate}
	\end{lemma}

	\reference{61}

	\begin{demonstration}
		\begin{enumerate}[(i)]
			\item Montrons que la famille $(f^i)_{i \in \llbracket 0, k-1 \rrbracket}$ est à la fois libre et génératrice.
			\begin{itemize}
				\item Soit $P(f) \in \mathbb{K}[f]$. On fait la division euclidienne de $P$ par $\mu_f$ dans $\mathbb{K}[X]$ pour écrire $P = \mu_f Q + R$ avec $Q, R \in \mathbb{K}[X]$ et $\deg(R) < k = \deg(\mu_f)$. En évaluant en $f$, cela donne $P(f) = R(f) \in \operatorname{Vect}(\id_E, \dots, f^{k-1})$. Donc la famille est génératrice.
				\item Si $\sum_{i=0}^{k-1} \lambda_i f^i = 0$, alors le polynôme $P = \sum_{i=0}^{k-1} \lambda_i X^i$ vérifie $P(f) = 0$. Donc $\mu_f \mid P$, et comme $\deg(P) < \deg(\mu_f)$, on a $P = 0$. Donc $\lambda_0 = \dots = \lambda_{k-1} = 0$. Donc la famille est libre.
			\end{itemize}
			\item La deuxième assertion se montre sensiblement de la même manière.
		\end{enumerate}
	\end{demonstration}

	\reference{290}

	\begin{lemma}
		\label{invariants-de-similitude-2}
		Il existe $x \in E$ tel que $P_x = \mu_f$.
	\end{lemma}

	\begin{remark}
		La démonstration est un peu trop longue pour être incluse ici : c'est un résultat qui demande du temps pour le démontrer (et pourrait constituer un vrai développement à part entière). Nous vous renvoyons vers \cite{[GOU21]} p. 178 pour la démonstration.
	\end{remark}

	\begin{theorem}[Frobenius]
		Il existe des sous-espaces vectoriels $F_1, \dots, F_r$ de $E$ tous stables par $f$ tels que :
		\begin{enumerate}[(i)]
			\item $E = \bigoplus_{i = 1}^r F_i$.
			\item $\forall i \in \llbracket 1, r \rrbracket$, la restriction $f_i = f_{|F_i}$ est un endomorphisme cyclique de $F_i$.
			\item Si $P_i = \mu_{f_i}$ est le polynôme minimal de $f_i$, on a $P_{i+1} \mid P_i$ $\forall i \in \llbracket 1, r-1 \rrbracket$.
		\end{enumerate}
		La suite $(P_i)_{i \in \llbracket 1, r \rrbracket}$ ne dépend que de $f$ et non du choix de la décomposition (elle est donc unique). On l'appelle \textbf{suite des invariants de $f$}.
	\end{theorem}

	\begin{demonstration}
		\begin{itemize}
			\item \underLine{Existence :} Soit $k = \deg(\mu_f)$. Par le \cref{invariants-de-similitude-2}, il existe $x \in E$ tel que $P_x = \mu_f$. Par le \cref{invariants-de-similitude-1}, le sous-espace $F = E_x$ est de dimension $k$ et est stable par $f$ et comme $\deg(P_x) = k$, la famille de vecteurs
			\[ (\underbrace{x}_{= e_1}, \dots, \underbrace{f^{k-1}(x)}_{= e_k}) \]
			forme une base de $F$. Complétons cette base en une base $(e_1, \dots, e_n)$ de $E$. En désignant par $(e_1^*, \dots, e_n^*)$ la base duale associée et en notant $\Gamma = \{ e_k^* \circ f^i \mid i \in \mathbb{N} \}$, on pose
			\begin{align*}
				G &= \Gamma^\circ \\
				&= \{ x \in E \mid \forall i \in \mathbb{N}, \, (e_k^* \circ f^i)(x) = 0 \}
			\end{align*}
			Ainsi, $G$ est l'ensemble des $x \in E$ tel que la $k$-ième coordonnée de $f^i(x)$ (dans la base $(e_1, \dots, e_n)$)  est nulle $\forall i \in \mathbb{N}$ ; $G$ est donc un sous-espace de $E$ stable par $f$. Montrons que $F \oplus G = E$.
			\newpar
			Montrons que $F \, \cap \, G = \{ 0 \}$. Soit $y \in F \, \cap \, G$. Si $y \neq 0$, on peut écrire $y = \lambda_1 e_1 + \dots + \lambda_p e_p$ avec $\lambda_p \neq 0$ et $p \leq k$. En composant par $e_k^* \circ f^{k-p}$, on obtient
			\begin{align*}
				0 &\underset{y \in G}{=} e_k^* \circ f^{k-p}(y) \\
				&= e_k^* (\lambda_1 f^{k-p}(e_1) + \dots + \lambda_p f^{k-p}(e_p)) \\
				&= e_k^* (\lambda_1 f^{k-p}(x) + \dots + \lambda_p f^{k-p}(x)) \\
				&= \lambda_p
			\end{align*}
			Ce qui est absurde.
			\newpar
			Montrons que $\dim(F) + \dim(G) = n$. Cela revient à montrer que $\dim(G) = n - k$. On sait que $G = \Gamma^\circ = (\operatorname{Vect}(\Gamma))^\circ$ et $\dim(\operatorname{Vect}(\Gamma)) + \dim(\operatorname{Vect}(\Gamma)^\circ) = n$. Montrons donc que $\dim(\operatorname{Vect}(\Gamma)) = k$. Posons
			\[
			\varphi :
			\begin{array}{cl}
				\mathbb{K}[f] &\rightarrow \operatorname{Vect}(\Gamma) \\
				g &\mapsto e_k^* \circ g
			\end{array}
			\]
			Par définition de $\Gamma$, $\varphi$ est surjective. Soit $g \in \ker(\varphi)$. On a alors $e_k^* \circ g = 0$, et comme $g \in \mathbb{K}[f]$,
			\[ g = \lambda_1 \id + \dots \lambda_p f^{p-1} \text{ avec } \lambda_p \neq 0 \text{ et } p \leq k \]
			On a donc $0 = e_k^* \circ g(f^{k-p} (x)) = \lambda_p \neq 0$. Ainsi, $g = 0$ et $\phi$ est un isomorphisme. Donc $\dim(\operatorname{Vect}(\Gamma)) = \dim(\mathbb{K}[f]) = k$ par le \cref{invariants-de-similitude-1}, ce que l'on voulait.
			\newpar
			Soit $P_1$ le polynôme minimal de $f_{|F}$ (qui est le polynôme minimal de $f$ car $P_1 = \mu_{f_{|F}} = \underset{\mu_f = P_x}{=} \mu_f$). Soit $P_2$ le polynôme minimal de $f_{|G}$. Comme $G$ est stable par $f$, on a $P_1(f_{|G}) = \mu_f(f_{|G}) = 0$, donc $P_2 \mid P_1$. Il suffit alors de réitérer en remplaçant $f$ par $f_{|G}$ et $E$ par $G$ pour obtenir la décomposition voulu.
			\item \underLine{Unicité :} Soient $F_1, \dots, F_r$ et $G_1, \dots G_s$ des sous-espaces vectoriels stables par $f$ qui vérifient $(i)$, $(ii)$ et $(iii)$. On note pour tout $i$, $P_i = \mu_{f_{|F_i}}$ et $Q_i = \mu_{f_{|G_i}}$. On suppose par l'absurde $(P_1, \dots, P_r) \neq (Q_1, \dots, Q_s)$. Soit $j = \min\{ i \mid P_i \neq Q_i \}$.
			Comme $E = \bigoplus_{i = 1}^r F_i$ (où $\forall i \in \llbracket 1, r \rrbracket$, $F_i$ est stable par $f$ et $\forall k \geq j \geq 1$, $P_j(f)(F_k) = 0$) :
			\[ P_j(f)(F_1) \oplus \dots \oplus P_j(f)(F_{j-1}) = P_j(f)(E) \tag{$*$} \]
			De même,
			\[ P_j(f)(G_1) \oplus \dots \oplus P_j(f)(G_{j-1}) \oplus P_j(f)(G_j) \oplus \dots \oplus P_j(f)(G_s) = P_j(f)(E) \tag{$**$} \]
			Notons que l'on a $\forall i \in \llbracket 1, j-1 \rrbracket$, $\dim(P_j(f)(F_i)) = \dim(P_j(f)(G_i))$. En effet, on peut trouver une base $\mathcal{B}_i$ de $F_i$ et une base $\mathcal{B}'_i$ de $G_i$ telles que $\operatorname{Mat}(f_{|F_i}, \mathcal{B}_i) = \operatorname{Mat}(f_{|G_i}, \mathcal{B}'_i)$ par cyclicité de $f_{|F_i}$ et $f_{|G_i}$. En prenant les dimensions dans $(*)$ et $(**)$, on en déduit :
			\[ 0 = \dim(P_j(f)(G_j)) = \dots = \dim(P_j(f)(G_s)) \implies Q_j \mid P_j \]
			Par symétrie, on a de même $P_j \mid Q_j$. D'où $P_j = Q_j$ : absurde.
		\end{itemize}
	\end{demonstration}

	\begin{remark}
		Dans ce développement, il est courant de ne montrer que l'existence (à cause de la contrainte de temps, mais aussi de la contrainte d'espace).
	\end{remark}
	%</content>
\end{document}
