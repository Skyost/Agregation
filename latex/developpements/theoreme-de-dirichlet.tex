\documentclass[12pt, a4paper]{report}

% LuaLaTeX :

\RequirePackage{iftex}
\RequireLuaTeX

% Packages :

\usepackage[french]{babel}
%\usepackage[utf8]{inputenc}
%\usepackage[T1]{fontenc}
\usepackage[pdfencoding=auto, pdfauthor={Hugo Delaunay}]{hyperref}
\usepackage{amsmath}
\usepackage{amsthm}
%\usepackage{amssymb}
\usepackage{stmaryrd}
\usepackage{tikz}
\usepackage{tkz-euclide}
\usepackage{fourier-otf}
\usepackage{fontspec}
\usepackage{titlesec}
\usepackage{fancyhdr}
\usepackage{catchfilebetweentags}
\usepackage[french, capitalise, noabbrev]{cleveref}
\usepackage[fit, breakall]{truncate}
\usepackage[margin=2.5cm]{geometry}
\usepackage{enumerate}
\usepackage{tocloft}
\usepackage{microtype}
\usepackage{mdframed}
\usepackage{thmtools}
\usepackage{xcolor}
\usepackage{tabularx}
\usepackage{aligned-overset}
\usepackage[subpreambles=true]{standalone}
\usepackage{environ}
\usepackage[normalem]{ulem}
\usepackage{marginnote}
\usepackage{etoolbox}
\usepackage{setspace}
\usepackage[bibstyle=reading, citestyle=draft]{biblatex}
\usepackage{xpatch}

% Bibliographie :

\providecommand{\bibliographypath}{../bibliography.bib}
\addbibresource{\bibliographypath}
\defbibheading{bibliography}[\bibname]{%
	\newpage
	\section*{#1}%
}
\renewbibmacro*{entryhead:full}{\printfield{labeltitle}}
\DeclareFieldFormat{url}{\newline\footnotesize\url{#1}}
\AtEndDocument{\printbibliography}

% Police :

\setmathfont{Erewhon Math}

% Tikz :

\usetikzlibrary{calc}

% Longueurs :

\setlength{\parindent}{0pt}
\setlength{\headheight}{15pt}
\setlength{\fboxsep}{0pt}
\titlespacing*{\chapter}{0pt}{-20pt}{10pt}
\setlength{\marginparwidth}{1.5cm}
\setstretch{1.1}

% Métadonnées :

\author{agreg.skyost.eu}
\date{\today}

% Titres :

\setcounter{secnumdepth}{3}

\renewcommand{\thechapter}{\Roman{chapter}}
\renewcommand{\thesubsection}{\Roman{subsection}}
\renewcommand{\thesubsubsection}{\arabic{subsubsection}}

\titleformat{\chapter}{\huge\bfseries}{\thechapter}{20pt}{\huge\bfseries}
\titleformat*{\section}{\LARGE\bfseries}
\titleformat{\subsection}{\Large\bfseries}{\thesubsection \, - \,}{0pt}{\Large\bfseries}
\titleformat{\subsubsection}{\large\bfseries}{\thesubsubsection. \,}{0pt}{\large\bfseries}

% Table des matières :

\renewcommand{\cftsecleader}{\cftdotfill{\cftdotsep}}
\addtolength{\cftsecnumwidth}{10pt}

% Redéfinition des commandes :

\renewcommand*\thesection{\arabic{section}}
\renewcommand{\ker}{\mathrm{Ker}}

% Nouvelles commandes :

\newcommand{\website}{https://agreg.skyost.eu}

\newcommand{\tr}[1]{\mathstrut ^t #1}
\newcommand{\im}{\mathrm{Im}}
\newcommand{\rang}{\operatorname{rang}}
\newcommand{\trace}{\operatorname{trace}}
\newcommand{\id}{\operatorname{id}}
\newcommand{\stab}{\operatorname{Stab}}

\providecommand{\newpar}{\\[\medskipamount]}

\providecommand{\lesson}[3]{%
	\title{#3}%
	\hypersetup{pdftitle={#3}}%
	\setcounter{section}{\numexpr #2 - 1}%
	\section{#3}%
	\fancyhead[R]{\truncate{0.73\textwidth}{#2 : #3}}%
}

\providecommand{\development}[3]{%
	\title{#3}%
	\hypersetup{pdftitle={#3}}%
	\section*{#3}%
	\fancyhead[R]{\truncate{0.73\textwidth}{#3}}%
}

\providecommand{\summary}[1]{%
	\textit{#1}%
	\medskip%
}

%\newcommand{\booklink}[1]{\website/bibliographie\##1}
\newcommand{\citelink}[2]{\hyperlink{cite.\therefsection @#1}{#2}}
\newcommand{\previousreference}{}
\providecommand{\reference}[2][]{%
	\notblank{#1}{\renewcommand{\previousreference}{#1}}{}%
	\marginnote{%
		\centering%
		\notblank{#1}{\cite{[\previousreference]}\\}{}%
		\citelink{\previousreference}{p. #2}%
	}%
}

\newcommand{\imagespath}{../images/}

\providecommand{\includelatexpicture}[1]{%
	\begin{center}%
		\input{\imagespath#1}%
	\end{center}%
	\medskip%
}

\definecolor{devcolor}{HTML}{00695c}
\newcommand{\dev}[1]{%
	\reversemarginpar%
	\marginnote[\bfseries\color{devcolor}\href{\website/developpements/#1}{DEV}]{}%
	\normalmarginpar%
}

% En-têtes :

\pagestyle{fancy}
\fancyhead[L]{\truncate{0.23\textwidth}{\thepage}}
\fancyfoot[C]{\scriptsize \href{\website}{\texttt{agreg.skyost.eu}}}

% Couleurs :

\definecolor{property}{HTML}{fffde7}
\definecolor{proposition}{HTML}{fff8e1}
\definecolor{lemma}{HTML}{fff3e0}
\definecolor{theorem}{HTML}{fce4f2}
\definecolor{corollary}{HTML}{ffebee}
\definecolor{definition}{HTML}{ede7f6}
\definecolor{notation}{HTML}{f3e5f5}
\definecolor{example}{HTML}{e0f7fa}
\definecolor{cexample}{HTML}{efebe9}
\definecolor{application}{HTML}{e0f2f1}
\definecolor{remark}{HTML}{e8f5e9}
\definecolor{demonstration}{HTML}{e1f5fe}

% Théorèmes :

\declaretheoremstyle[bodyfont=\normalfont]{theorem}
\declaretheoremstyle[headfont=\itshape]{remark}

\newcounter{thm}[section]

\newcommand{\newth}[3][style=theorem]{
	\declaretheorem[%
		name=#3,%
		%within=section,
		sibling=thm,%
		mdframed={%
			hidealllines=true,%
			backgroundcolor={#2!98!white},%
			innermargin=8pt,%
			splittopskip=18pt,%
			splitbottomskip=16pt,%
		},%
		#1%
	]{#2}%
}

\newth{property}{Propriété}
\newth{proposition}{Proposition}
\newth{lemma}{Lemme}
\newth{theorem}{Théorème}
\newth{corollary}{Corollaire}
\newth{definition}{Définition}
\newth{notation}{Notation}
\newth{example}{Exemple}
\newth{cexample}{Contre-exemple}
\newth{application}{Application}
\newth[style=remark]{remark}{Remarque}
\newth[style=remark, numbered=no, qed=\textsquare]{demonstration}{Démonstration}

% Environnements :

\NewEnviron{whitetabularx}[1]{%
	\renewcommand{\arraystretch}{2.5}
	\colorbox{white}{%
		\begin{tabularx}{\textwidth}{#1}%
			\BODY%
		\end{tabularx}%
	}%
}

% Maths :

\DeclareFontEncoding{FMS}{}{}
\DeclareFontSubstitution{FMS}{futm}{m}{n}
\DeclareFontEncoding{FMX}{}{}
\DeclareFontSubstitution{FMX}{futm}{m}{n}
\DeclareSymbolFont{fouriersymbols}{FMS}{futm}{m}{n}
\DeclareSymbolFont{fourierlargesymbols}{FMX}{futm}{m}{n}
\DeclareMathDelimiter{\VERT}{\mathord}{fouriersymbols}{152}{fourierlargesymbols}{147}



\begin{document}
	%<*content>
	\development{analysis}{theoreme-de-dirichlet}{Théorème de Dirichlet}

	\summary{On montre le théorème de Dirichlet, qui permet sous certaines hypothèse de régularité, de montrer la convergence ponctuelle d'une série de Fourier. On l'applique ensuite pour calculer $\sum_{n=1}^{+\infty} \frac{1}{n^2}$.}
	
	\begin{notation}
		Soit $f : \mathbb{R} \rightarrow \mathbb{C}$ une fonction $2 \pi$-périodique.
		\begin{itemize}
			\item On note $\forall n \in \mathbb{Z}$, $c_n(f)$ le $n$-ième coefficient de Fourier.
			\item On note $\forall n \in \mathbb{Z}$, $a_n(f)$ et $b_n(f)$ les $n$-ièmes coefficients de Fourier réels.
		\end{itemize}
	\end{notation}

	\reference[GOU20]{271}
	
	\begin{lemma}
		\label{theoreme-de-dirichlet-1}
		Soit $D_n$ le noyau de Dirichlet, défini par
		\[ \forall n \in \mathbb{N}, \, \forall t \in \mathbb{R}, \, D_n(t) = \sum_{k=-n}^n e^{ikt} \]
		Alors :
		\[ \forall n \in \mathbb{N}, \, \forall t \in \mathbb{R} \setminus 2\pi\mathbb{Z}, \, D_n(t) = \frac{\sin \left ( \frac{(2n + 1)t}{2} \right)}{\sin \left ( \frac{t}{2} \right)} \]
	\end{lemma}

	\begin{demonstration}
		Soient $n \in \mathbb{Z}$ et $t \in \mathbb{R}\setminus 2\pi\mathbb{Z}$. On calcul :
		\begin{align*}
			D_n(t) &= e^{-int} \sum_{k=0}^{2n} e^{ikt} \\
			&= e^{-int} \frac{e^{(2n+1)it} - 1}{e^{it} - 1} \\
			&= e^{-int} \frac{e^{\frac{i (2n+1) t}{2}}}{e^{\frac{i t}{2}}} \frac{e^{\frac{i (2n+1) t}{2}} - e^{\frac{-i (2n+1) t}{2}}}{e^{i\frac{t}{2}} - e^{-i \frac{t}{2}}} \\
			&= \frac{2i \sin \left ( \frac{(2n + 1)t}{2} \right)}{2i \sin \left ( \frac{t}{2} \right)} \\
			&= \frac{\sin \left ( \frac{(2n + 1)t}{2} \right)}{\sin \left ( \frac{t}{2} \right)}
		\end{align*}
	\end{demonstration}

	\begin{theorem}[Dirichlet]
		\label{theoreme-de-dirichlet-2}
		Soit $f : \mathbb{R} \rightarrow \mathbb{C}$ une fonction $2 \pi$-périodique et de classe $\mathcal{C}^1$ par morceaux sur $\mathbb{R}$. Soit $t_0 \in \mathbb{R}$. Alors $\sum_{n \in \mathbb{Z}} c_n(f) e^{int_0}$ converge et
		\[ \sum_{n = -\infty}^{+\infty} c_n(f) e^{int_0} = \frac{f(t_0+) + f(t_0-)}{2} \]
	\end{theorem}

	\begin{demonstration}
		Quitte à composer par une translation $t \mapsto t + t_0$, on peut supposer $t_0 = 0$. Pour tout $n \in \mathbb{N}$, on note $s_n = \sum_{k=-n}^n c_k(f)$. Il s'agit de montrer que la suite $(u_n)$ définie $\forall n \in \mathbb{N}$ par $u_n = s_n - \frac{f(0+) + f(0-)}{2}$ tend vers $0$. Or,
		\[ 2\pi s_n = \sum_{k=-n}^n \int_{-\pi}^{\pi} f(t) e^{-int} \, \mathrm{d}t = \int_{-\pi}^{\pi} f(t) D_n(t) \, \mathrm{d}t = \int^0_{-\pi} f(t) D_n(t) \, \mathrm{d}t + \int_0^{\pi} f(t) D_n(t) \, \mathrm{d}t \]
		Comme $D_n$ est une fonction paire, on a
		\[ \int^0_{-\pi} f(t) D_n(t) \, \mathrm{d}t = \int_0^{\pi} f(-t) D_n(t) \, \mathrm{d}t \]
		D'où,
		\[ 2 \pi s_n = \int_0^{\pi} (f(t) + f(-t)) D_n(t) \, \mathrm{d}t \]
		On en déduit à l'aide du \cref{theoreme-de-dirichlet-1}, que :
		\[ 2 \pi u_n = \int_0^{\pi} (f(t) + f(-t) - f(0+) - f(0-)) D_n(t) \, \mathrm{d}t = \int_0^{\pi} g(t) \sin \left ( \frac{(2n + 1)t}{2} \right) \, \mathrm{d}t \]
		où $g : t \mapsto \frac{f(t) + f(-t) - f(0+) - f(0-)}{\sin \left ( \frac{t}{2} \right)}$ est continue par morceaux sur $]0, \pi[$ et bornée sur un voisinage de $0$ par hypothèse. La fonction $g$ est donc intégrable sur $]0, \pi]$ et le lemme de Riemann-Lebesgue entraîne $\lim_{n \rightarrow +\infty} 2 \pi u_n = 0$.
	\end{demonstration}

	\begin{application}
		\[ \sum_{n=1}^{+\infty} \frac{1}{n^2} = \frac{\pi^2}{6} \]
	\end{application}

	\begin{demonstration}
		On définit $f : x \mapsto 1 - \frac{x^2}{\pi^2}$. La fonction $f$ est $2\pi$-périodique et est paire. Les coefficients $b_n$ sont donc nuls. Par ailleurs,
		\[ a_0(f) = \frac{1}{\pi} \int_{-\pi}^{\pi} \left(1 - \frac{t^2}{\pi^2} \right) \, \mathrm{d}t = \frac{4}{3} \]
		et $\forall n \in \mathbb{N}^*$,
		\[ a_n(f) = \frac{1}{\pi} \int_{-\pi}^{\pi} \left(1 - \frac{t^2}{\pi^2} \right) \cos(nt) \, \mathrm{d}t = -\frac{2}{\pi^3} \int_{-\pi}^{\pi} t^2 \cos(nt) \, \mathrm{d}t \]
		Après une double intégration par parties, on obtient :
		\[ a_n(f) = (-1)^{n+1} \frac{4}{n^2 \pi^2} \]
		Comme la fonction $f$ est $\mathcal{C}^{\infty}$, on peut appliquer le \cref{theoreme-de-dirichlet-2} en $t_0 = \pi$, qui donne :
		\begin{align*}
			0 &= f(\pi) \\
			&= \frac{a_0(f)}{2} \sum_{n = 1}^{+\infty} a_n(f) \cos(n\pi) \\
			&= \frac{2}{3} - \frac{4}{\pi^2} \sum_{n = 1}^{+\infty} (-1)^n \frac{\cos(n\pi)}{n^2} \\
			&= \frac{2}{3} - \frac{4}{\pi^2} \sum_{n = 1}^{+\infty} \frac{1}{n^2}
		\end{align*}
		On en déduit :
		\[ \sum_{n=1}^{+\infty} \frac{1}{n^2} = \frac{\pi^2}{6} \]
	\end{demonstration}
	%</content>
\end{document}
