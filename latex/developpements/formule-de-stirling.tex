\documentclass[12pt, a4paper]{report}

% LuaLaTeX :

\RequirePackage{iftex}
\RequireLuaTeX

% Packages :

\usepackage[french]{babel}
%\usepackage[utf8]{inputenc}
%\usepackage[T1]{fontenc}
\usepackage[pdfencoding=auto, pdfauthor={Hugo Delaunay}, pdfsubject={Mathématiques}, pdfcreator={agreg.skyost.eu}]{hyperref}
\usepackage{amsmath}
\usepackage{amsthm}
%\usepackage{amssymb}
\usepackage{stmaryrd}
\usepackage{tikz}
\usepackage{tkz-euclide}
\usepackage{fourier-otf}
\usepackage{fontspec}
\usepackage{titlesec}
\usepackage{fancyhdr}
\usepackage{catchfilebetweentags}
\usepackage[french, capitalise, noabbrev]{cleveref}
\usepackage[fit, breakall]{truncate}
\usepackage[top=2.5cm, right=2cm, bottom=2.5cm, left=2cm]{geometry}
\usepackage{enumerate}
\usepackage{tocloft}
\usepackage{microtype}
%\usepackage{mdframed}
%\usepackage{thmtools}
\usepackage{xcolor}
\usepackage{tabularx}
\usepackage{aligned-overset}
\usepackage[subpreambles=true]{standalone}
\usepackage{environ}
\usepackage[normalem]{ulem}
\usepackage{marginnote}
\usepackage{etoolbox}
\usepackage{setspace}
\usepackage[bibstyle=reading, citestyle=draft]{biblatex}
\usepackage{xpatch}
\usepackage[many, breakable]{tcolorbox}
\usepackage[backgroundcolor=white, bordercolor=white, textsize=small]{todonotes}

% Bibliographie :

\newcommand{\overridebibliographypath}[1]{\providecommand{\bibliographypath}{#1}}
\overridebibliographypath{../bibliography.bib}
\addbibresource{\bibliographypath}
\defbibheading{bibliography}[\bibname]{%
	\newpage
	\section*{#1}%
}
\renewbibmacro*{entryhead:full}{\printfield{labeltitle}}
\DeclareFieldFormat{url}{\newline\footnotesize\url{#1}}
\AtEndDocument{\printbibliography}

% Police :

\setmathfont{Erewhon Math}

% Tikz :

\usetikzlibrary{calc}

% Longueurs :

\setlength{\parindent}{0pt}
\setlength{\headheight}{15pt}
\setlength{\fboxsep}{0pt}
\titlespacing*{\chapter}{0pt}{-20pt}{10pt}
\setlength{\marginparwidth}{1.5cm}
\setstretch{1.1}

% Métadonnées :

\author{agreg.skyost.eu}
\date{\today}

% Titres :

\setcounter{secnumdepth}{3}

\renewcommand{\thechapter}{\Roman{chapter}}
\renewcommand{\thesubsection}{\Roman{subsection}}
\renewcommand{\thesubsubsection}{\arabic{subsubsection}}
\renewcommand{\theparagraph}{\alph{paragraph}}

\titleformat{\chapter}{\huge\bfseries}{\thechapter}{20pt}{\huge\bfseries}
\titleformat*{\section}{\LARGE\bfseries}
\titleformat{\subsection}{\Large\bfseries}{\thesubsection \, - \,}{0pt}{\Large\bfseries}
\titleformat{\subsubsection}{\large\bfseries}{\thesubsubsection. \,}{0pt}{\large\bfseries}
\titleformat{\paragraph}{\bfseries}{\theparagraph. \,}{0pt}{\bfseries}

\setcounter{secnumdepth}{4}

% Table des matières :

\renewcommand{\cftsecleader}{\cftdotfill{\cftdotsep}}
\addtolength{\cftsecnumwidth}{10pt}

% Redéfinition des commandes :

\renewcommand*\thesection{\arabic{section}}
\renewcommand{\ker}{\mathrm{Ker}}

% Nouvelles commandes :

\newcommand{\website}{https://agreg.skyost.eu}

\newcommand{\tr}[1]{\mathstrut ^t #1}
\newcommand{\im}{\mathrm{Im}}
\newcommand{\rang}{\operatorname{rang}}
\newcommand{\trace}{\operatorname{trace}}
\newcommand{\id}{\operatorname{id}}
\newcommand{\stab}{\operatorname{Stab}}

\providecommand{\newpar}{\\[\medskipamount]}

\providecommand{\lesson}[3]{%
	\title{#3}%
	\hypersetup{pdftitle={#3}}%
	\setcounter{section}{\numexpr #2 - 1}%
	\section{#3}%
	\fancyhead[R]{\truncate{0.73\textwidth}{#2 : #3}}%
}

\providecommand{\development}[3]{%
	\title{#3}%
	\hypersetup{pdftitle={#3}}%
	\section*{#3}%
	\fancyhead[R]{\truncate{0.73\textwidth}{#3}}%
}

\providecommand{\summary}[1]{%
	\textit{#1}%
	\medskip%
}

\tikzset{notestyleraw/.append style={inner sep=0pt, rounded corners=0pt, align=center}}

%\newcommand{\booklink}[1]{\website/bibliographie\##1}
\newcommand{\citelink}[2]{\hyperlink{cite.\therefsection @#1}{#2}}
\newcommand{\previousreference}{}
\providecommand{\reference}[2][]{%
	\notblank{#1}{\renewcommand{\previousreference}{#1}}{}%
	\todo[noline]{%
		\protect\vspace{16pt}%
		\protect\par%
		\protect\notblank{#1}{\cite{[\previousreference]}\\}{}%
		\protect\citelink{\previousreference}{p. #2}%
	}%
}

\definecolor{devcolor}{HTML}{00695c}
\newcommand{\dev}[1]{%
	\reversemarginpar%
	\todo[noline]{
		\protect\vspace{16pt}%
		\protect\par%
		\bfseries\color{devcolor}\href{\website/developpements/#1}{DEV}
	}%
	\normalmarginpar%
}

% En-têtes :

\pagestyle{fancy}
\fancyhead[L]{\truncate{0.23\textwidth}{\thepage}}
\fancyfoot[C]{\scriptsize \href{\website}{\texttt{agreg.skyost.eu}}}

% Couleurs :

\definecolor{property}{HTML}{fffde7}
\definecolor{proposition}{HTML}{fff8e1}
\definecolor{lemma}{HTML}{fff3e0}
\definecolor{theorem}{HTML}{fce4f2}
\definecolor{corollary}{HTML}{ffebee}
\definecolor{definition}{HTML}{ede7f6}
\definecolor{notation}{HTML}{f3e5f5}
\definecolor{example}{HTML}{e0f7fa}
\definecolor{cexample}{HTML}{efebe9}
\definecolor{application}{HTML}{e0f2f1}
\definecolor{remark}{HTML}{e8f5e9}
\definecolor{proof}{HTML}{e1f5fe}

% Théorèmes :

\theoremstyle{definition}
\newtheorem{theorem}{Théorème}

\newtheorem{property}[theorem]{Propriété}
\newtheorem{proposition}[theorem]{Proposition}
\newtheorem{lemma}[theorem]{Lemme}
\newtheorem{corollary}[theorem]{Corollaire}

\newtheorem{definition}[theorem]{Définition}
\newtheorem{notation}[theorem]{Notation}

\newtheorem{example}[theorem]{Exemple}
\newtheorem{cexample}[theorem]{Contre-exemple}
\newtheorem{application}[theorem]{Application}

\theoremstyle{remark}
\newtheorem{remark}[theorem]{Remarque}

\counterwithin*{theorem}{section}

\newcommand{\applystyletotheorem}[1]{
	\tcolorboxenvironment{#1}{
		enhanced,
		breakable,
		colback=#1!98!white,
		boxrule=0pt,
		boxsep=0pt,
		left=8pt,
		right=8pt,
		top=8pt,
		bottom=8pt,
		sharp corners,
		after=\par,
	}
}

\applystyletotheorem{property}
\applystyletotheorem{proposition}
\applystyletotheorem{lemma}
\applystyletotheorem{theorem}
\applystyletotheorem{corollary}
\applystyletotheorem{definition}
\applystyletotheorem{notation}
\applystyletotheorem{example}
\applystyletotheorem{cexample}
\applystyletotheorem{application}
\applystyletotheorem{remark}
\applystyletotheorem{proof}

% Environnements :

\NewEnviron{whitetabularx}[1]{%
	\renewcommand{\arraystretch}{2.5}
	\colorbox{white}{%
		\begin{tabularx}{\textwidth}{#1}%
			\BODY%
		\end{tabularx}%
	}%
}

% Maths :

\DeclareFontEncoding{FMS}{}{}
\DeclareFontSubstitution{FMS}{futm}{m}{n}
\DeclareFontEncoding{FMX}{}{}
\DeclareFontSubstitution{FMX}{futm}{m}{n}
\DeclareSymbolFont{fouriersymbols}{FMS}{futm}{m}{n}
\DeclareSymbolFont{fourierlargesymbols}{FMX}{futm}{m}{n}
\DeclareMathDelimiter{\VERT}{\mathord}{fouriersymbols}{152}{fourierlargesymbols}{147}


% Bibliographie :

\addbibresource{\bibliographypath}%
\defbibheading{bibliography}[\bibname]{%
	\newpage
	\section*{#1}%
}
\renewbibmacro*{entryhead:full}{\printfield{labeltitle}}%
\DeclareFieldFormat{url}{\newline\footnotesize\url{#1}}%

\AtEndDocument{\printbibliography}

\begin{document}
	%<*content>
	\development{analysis}{formule-de-stirling}{Formule de Stirling}

	\summary{Dans ce développement un peu technique, nous démontrons la formule de Stirling $n! \sim \sqrt{2n\pi} \left(\frac{n}{e} \right)^n$ à l'aide du théorème central limite et de la fonction $\Gamma$ d'Euler.}

	\begin{lemma}
		\label{formule-de-stirling-1}
		Soit $Y$ une variable aléatoire réelle à densité. Alors $\forall n \geq 1$, $\frac{Y - n}{\sqrt{n}}$ est à densité et,
		\[ f_{\frac{Y - n}{\sqrt{n}}}(x) = \sqrt{n} f_{Y}(n + x \sqrt{n}) \text{ p.p. en } x \in \mathbb{R} \]
	\end{lemma}

	\begin{proof}
		$\forall x \in \mathbb{R}$,
		\begin{align*}
			F_{\frac{Y - n}{\sqrt{n}}}(x) & = \mathbb{P} \left(\frac{Y - n}{\sqrt{n}} \leq x \right) \\
			&= \mathbb{P} (Y \leq x \sqrt{n} + n) \\
			&= F_Y (x \sqrt{n} + n)
		\end{align*}
		Or, la fonction de répartition d'une variable aléatoire réelle à densité est dérivable presque partout, et sa dérivée est presque partout égale à sa densité. Donc :
		\[ f_{\frac{Y - n}{\sqrt{n}}}(x) = \sqrt{n} f_Y (x \sqrt{n} + n) \text{ p.p. en } x \in \mathbb{R} \]
	\end{proof}

	\begin{remark}
		Il ne s'agit ni plus ni moins qu'une version affaiblie du théorème de changement de variable.
	\end{remark}

	\reference[G-K]{180}

	\begin{lemma}
		\label{formule-de-stirling-2}
		Soient $X$ et $Y$ deux variables aléatoires indépendantes telles que $X \sim \Gamma(a, \gamma)$ et $Y \sim \Gamma(b, \gamma)$. Alors $Z = X + Y \sim \Gamma(a+b, \gamma)$.
	\end{lemma}

	\begin{proof}
		Soit $f_{a,\gamma} : x \mapsto \frac{\gamma^a}{\Gamma(a)} x^{a-1} e^{-\gamma x} \mathbb{1}_{\mathbb{R}^+}(x)$ la densité de la loi $\Gamma(a, \gamma)$. $\forall x \geq 0$, on a :
		\begin{align*}
			f_Z(x) & = \int_0^x f_{a, \gamma}(x-t)f_{b, \gamma}(t) \, \mathrm{d}t \\
			& = \int_0^x \frac{\gamma^a}{\Gamma(a)} t^{a-1} e^{-\gamma t} \frac{\gamma^b}{\Gamma(b)} (x-t)^{b-1} e^{-\gamma (x-t)} \, \mathrm{d}t \\
			& = \frac{\gamma^{a+b} e^{-\gamma x}}{\Gamma(a) \Gamma(b)} \int_0^x t^{a-1} (x-t)^{b-1} \, \mathrm{d}t \\
			& \overset{t=ux}{=} \frac{\gamma^{a+b} e^{-\gamma x}}{\Gamma(a) \Gamma(b)} x^{a+b-1} \int_0^1 u^{a-1} (x-t)^{b-1} \, \mathrm{d}t \\
			& = K_{a,b} f_{a+b, \gamma}(x)
		\end{align*}
		où $K_{a,b} = \frac{\Gamma(a+b)}{\Gamma(a) \Gamma(b)} \int_0^1 u^{a-1} (1-u)^{b-1} \, \mathrm{d}u$. Notons par ailleurs que $f_Z$ est nulle sur $\mathbb{R}^-$ et coïncide donc avec $K_{a,b} f_{a+b, \gamma}$ sur $\mathbb{R}^-$.
		\newpar
		Pour conclure, on utilise la condition de normalisation :
		\[ 1 = \int_{\mathbb{R}} f_Z(x) \, \mathrm{d}x = K_{a,b} \int_{\mathbb{R}} f_{a+b, \gamma}(x) \, \mathrm{d}x = K_{a,b} \]
		On obtient ainsi $f_Z = f_{a+b, \gamma}$, ce que l'on voulait.
	\end{proof}

	\reference{556}

	\begin{theorem}[Formule de Stirling]
		\[ n! \sim \sqrt{2n\pi} \left(\frac{n}{e} \right)^n \]
	\end{theorem}

	\begin{proof}
		Soit $(X_n)$ une suite de variable aléatoires indépendantes de même loi $\mathcal{E}(1)$. On pose $S_n = \sum_{k=0}^n X_k$. Montrons par récurrence que $S_n \sim \Gamma(n+1, 1)$.
		\begin{itemize}
			\item \uline{Pour $n = 0$ :} c'est clair car $\mathcal{E}(1) = \Gamma(1, 1)$.
			\item \uline{On suppose le résultat vrai à un rang $n \geq 0$.} Pour montrer qu'il reste vrai au rang $n+1$, il suffit d'appliquer le \cref{formule-de-stirling-2} à $S_n \sim \Gamma(n, 1)$ et $X_{n+1} \sim \Gamma(1, 1)$ (qui sont bien indépendantes).
		\end{itemize}
		Par le \cref{formule-de-stirling-1} appliqué à $S_n$, p.p. en $x \in \mathbb{R}$,
		\begin{align*}
			\overbrace{f_{\frac{S_n - n}{\sqrt{n}}}(x)}^{= g_n(x)} & = \sqrt{n} f_{S_n} (n + x \sqrt{n}) \\
			& = \frac{\sqrt{n}}{\Gamma(n+1)} n^n \left(1 + \frac{x}{\sqrt{n}} \right)^n e^{-(n + x\sqrt{n})} \mathbb{1}_{[-\sqrt{n}, +\infty[}(x) \\
			& = a_n h_n(x)
		\end{align*}
		avec :
		\begin{itemize}
			\item $a_n = \frac{n^{n+\frac{1}{2}} e^{-n} \sqrt{2 \pi}}{\Gamma(n+1)}$ (ce qui nous intéresse).
			\item $h_n : x \mapsto \frac{e^{-\sqrt{n} x}}{\sqrt{2\pi}} \left( 1 + \frac{x}{\sqrt{n}} \right)^n \mathbb{1}_{[-\sqrt{n}, +\infty[}(x)$ (ce qui nous intéresse moins).
		\end{itemize}
		\medskip
		Montrons maintenant que $\frac{S_n - n}{\sqrt{n}}$ converge en loi vers $\mathcal{N}(0,1)$. D'après le théorème central limite,
		\[ \frac{S_n - \mathbb{E}(S_n)}{\operatorname{Var}(S_n)} \overset{(d)}{\longrightarrow} \mathcal{N}(0,1) \]
		où :
		\begin{itemize}
			\item $\mathbb{E}(S_n) = (n+1) \mathbb{E}(X_0) = n+1$.
			\item $\operatorname{Var}(S_n) = (n+1) \operatorname{Var}(X_0) = n+1$ par indépendance.
		\end{itemize}
		On applique maintenant le théorème de Slutsky :
		\[ \frac{S_n - n}{\sqrt{n}} = \underbrace{\frac{\sqrt{n+1}}{\sqrt{n}}}_{\longrightarrow 1} \left( \underbrace{\frac{S_n - (n+1)}{\sqrt{n+1}}}_{\overset{(d)}{\longrightarrow} \mathcal{N}(0,1)} + \underbrace{\frac{1}{\sqrt{n+1}}}_{\longrightarrow 0} \right) \overset{(d)}{\longrightarrow} \mathcal{N}(0,1) \]
		Tout cela pour dire que,
		\[ \int_0^1 g_n(x) \, \mathrm{d}x = \mathbb{P} \left( \frac{S_n - n}{\sqrt{n}} \in [0,1] \right) \longrightarrow \int_0^1 \frac{e^{-\frac{x^2}{2}}}{\sqrt{2 \pi}} \, \mathrm{d}x \]
		De plus :
		\begin{itemize}
			\item $\forall n \in \mathbb{N}$, $h_n$ est mesurable.
			\item $\forall x \in \mathbb{R}$, $h_n(x) = \frac{e^{-x^2 \varphi \left( \frac{x}{\sqrt{n}} \right)}}{\sqrt{2 \pi}} \mathbb{1}_{]-1, +\infty[} \left( \frac{x}{\sqrt{n}} \right)$ où $\forall x > -1$, $\varphi(x) = \frac{x - \ln(1+x)}{x^2}$. Par développement limité, on a $\lim_{x \rightarrow 0} \varphi(x) = \frac{1}{2}$. Donc $\forall x \in \mathbb{R}$, $h_n(x) \longrightarrow \frac{e^{-\frac{x^2}{2}}}{\sqrt{2 \pi}}$.
			\item Comme $\forall x > -1$, $\varphi(x) \geq 0$, alors $h_n$ est dominée par $x \mapsto \frac{1}{\sqrt{2 \pi}}$.
		\end{itemize}
		\medskip
		Donc par le théorème de convergence dominée,
		\[ \int_0^1 h_n(x) \, \mathrm{d}x \longrightarrow \int_0^1 \frac{e^{-\frac{x^2}{2}}}{\sqrt{2 \pi}} \, \mathrm{d}x \]
		Pour conclure, on écrit :
		\[ \int_0^1 g_n(x) \, \mathrm{d}x = a_n \int_0^1 h_n(x) \, \mathrm{d}x \implies \lim_{n \rightarrow +\infty} a_n = \frac{\lim_{n \rightarrow +\infty} \int_0^1 g_n(x) \, \mathrm{d}x}{\lim_{n \rightarrow +\infty} \int_0^1 h_n(x) \, \mathrm{d}x} = 1 \]
		et comme $\Gamma(n+1) = n!$, par définition de $a_n$ :
		\[ 1 = \lim_{n \rightarrow +\infty} a_n = \lim_{n \rightarrow +\infty} \frac{n^{n + \frac{1}{2}} e^{-n} \sqrt{2\pi}}{n!} \]
	\end{proof}
	%</content>
\end{document}
