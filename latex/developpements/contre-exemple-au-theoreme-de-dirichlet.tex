\documentclass[12pt, a4paper]{report}

% LuaLaTeX :

\RequirePackage{iftex}
\RequireLuaTeX

% Packages :

\usepackage[french]{babel}
%\usepackage[utf8]{inputenc}
%\usepackage[T1]{fontenc}
\usepackage[pdfencoding=auto, pdfauthor={Hugo Delaunay}, pdfsubject={Mathématiques}, pdfcreator={agreg.skyost.eu}]{hyperref}
\usepackage{amsmath}
\usepackage{amsthm}
%\usepackage{amssymb}
\usepackage{stmaryrd}
\usepackage{tikz}
\usepackage{tkz-euclide}
\usepackage{fourier-otf}
\usepackage{fontspec}
\usepackage{titlesec}
\usepackage{fancyhdr}
\usepackage{catchfilebetweentags}
\usepackage[french, capitalise, noabbrev]{cleveref}
\usepackage[fit, breakall]{truncate}
\usepackage[top=2.5cm, right=2cm, bottom=2.5cm, left=2cm]{geometry}
\usepackage{enumerate}
\usepackage{tocloft}
\usepackage{microtype}
%\usepackage{mdframed}
%\usepackage{thmtools}
\usepackage{xcolor}
\usepackage{tabularx}
\usepackage{aligned-overset}
\usepackage[subpreambles=true]{standalone}
\usepackage{environ}
\usepackage[normalem]{ulem}
\usepackage{marginnote}
\usepackage{etoolbox}
\usepackage{setspace}
\usepackage[bibstyle=reading, citestyle=draft]{biblatex}
\usepackage{xpatch}
\usepackage[many, breakable]{tcolorbox}
\usepackage[backgroundcolor=white, bordercolor=white, textsize=small]{todonotes}

% Bibliographie :

\newcommand{\overridebibliographypath}[1]{\providecommand{\bibliographypath}{#1}}
\overridebibliographypath{../bibliography.bib}
\addbibresource{\bibliographypath}
\defbibheading{bibliography}[\bibname]{%
	\newpage
	\section*{#1}%
}
\renewbibmacro*{entryhead:full}{\printfield{labeltitle}}
\DeclareFieldFormat{url}{\newline\footnotesize\url{#1}}
\AtEndDocument{\printbibliography}

% Police :

\setmathfont{Erewhon Math}

% Tikz :

\usetikzlibrary{calc}

% Longueurs :

\setlength{\parindent}{0pt}
\setlength{\headheight}{15pt}
\setlength{\fboxsep}{0pt}
\titlespacing*{\chapter}{0pt}{-20pt}{10pt}
\setlength{\marginparwidth}{1.5cm}
\setstretch{1.1}

% Métadonnées :

\author{agreg.skyost.eu}
\date{\today}

% Titres :

\setcounter{secnumdepth}{3}

\renewcommand{\thechapter}{\Roman{chapter}}
\renewcommand{\thesubsection}{\Roman{subsection}}
\renewcommand{\thesubsubsection}{\arabic{subsubsection}}
\renewcommand{\theparagraph}{\alph{paragraph}}

\titleformat{\chapter}{\huge\bfseries}{\thechapter}{20pt}{\huge\bfseries}
\titleformat*{\section}{\LARGE\bfseries}
\titleformat{\subsection}{\Large\bfseries}{\thesubsection \, - \,}{0pt}{\Large\bfseries}
\titleformat{\subsubsection}{\large\bfseries}{\thesubsubsection. \,}{0pt}{\large\bfseries}
\titleformat{\paragraph}{\bfseries}{\theparagraph. \,}{0pt}{\bfseries}

\setcounter{secnumdepth}{4}

% Table des matières :

\renewcommand{\cftsecleader}{\cftdotfill{\cftdotsep}}
\addtolength{\cftsecnumwidth}{10pt}

% Redéfinition des commandes :

\renewcommand*\thesection{\arabic{section}}
\renewcommand{\ker}{\mathrm{Ker}}

% Nouvelles commandes :

\newcommand{\website}{https://agreg.skyost.eu}

\newcommand{\tr}[1]{\mathstrut ^t #1}
\newcommand{\im}{\mathrm{Im}}
\newcommand{\rang}{\operatorname{rang}}
\newcommand{\trace}{\operatorname{trace}}
\newcommand{\id}{\operatorname{id}}
\newcommand{\stab}{\operatorname{Stab}}

\providecommand{\newpar}{\\[\medskipamount]}

\providecommand{\lesson}[3]{%
	\title{#3}%
	\hypersetup{pdftitle={#3}}%
	\setcounter{section}{\numexpr #2 - 1}%
	\section{#3}%
	\fancyhead[R]{\truncate{0.73\textwidth}{#2 : #3}}%
}

\providecommand{\development}[3]{%
	\title{#3}%
	\hypersetup{pdftitle={#3}}%
	\section*{#3}%
	\fancyhead[R]{\truncate{0.73\textwidth}{#3}}%
}

\providecommand{\summary}[1]{%
	\textit{#1}%
	\medskip%
}

\tikzset{notestyleraw/.append style={inner sep=0pt, rounded corners=0pt, align=center}}

%\newcommand{\booklink}[1]{\website/bibliographie\##1}
\newcommand{\citelink}[2]{\hyperlink{cite.\therefsection @#1}{#2}}
\newcommand{\previousreference}{}
\providecommand{\reference}[2][]{%
	\notblank{#1}{\renewcommand{\previousreference}{#1}}{}%
	\todo[noline]{%
		\protect\vspace{16pt}%
		\protect\par%
		\protect\notblank{#1}{\cite{[\previousreference]}\\}{}%
		\protect\citelink{\previousreference}{p. #2}%
	}%
}

\definecolor{devcolor}{HTML}{00695c}
\newcommand{\dev}[1]{%
	\reversemarginpar%
	\todo[noline]{
		\protect\vspace{16pt}%
		\protect\par%
		\bfseries\color{devcolor}\href{\website/developpements/#1}{DEV}
	}%
	\normalmarginpar%
}

% En-têtes :

\pagestyle{fancy}
\fancyhead[L]{\truncate{0.23\textwidth}{\thepage}}
\fancyfoot[C]{\scriptsize \href{\website}{\texttt{agreg.skyost.eu}}}

% Couleurs :

\definecolor{property}{HTML}{fffde7}
\definecolor{proposition}{HTML}{fff8e1}
\definecolor{lemma}{HTML}{fff3e0}
\definecolor{theorem}{HTML}{fce4f2}
\definecolor{corollary}{HTML}{ffebee}
\definecolor{definition}{HTML}{ede7f6}
\definecolor{notation}{HTML}{f3e5f5}
\definecolor{example}{HTML}{e0f7fa}
\definecolor{cexample}{HTML}{efebe9}
\definecolor{application}{HTML}{e0f2f1}
\definecolor{remark}{HTML}{e8f5e9}
\definecolor{proof}{HTML}{e1f5fe}

% Théorèmes :

\theoremstyle{definition}
\newtheorem{theorem}{Théorème}

\newtheorem{property}[theorem]{Propriété}
\newtheorem{proposition}[theorem]{Proposition}
\newtheorem{lemma}[theorem]{Lemme}
\newtheorem{corollary}[theorem]{Corollaire}

\newtheorem{definition}[theorem]{Définition}
\newtheorem{notation}[theorem]{Notation}

\newtheorem{example}[theorem]{Exemple}
\newtheorem{cexample}[theorem]{Contre-exemple}
\newtheorem{application}[theorem]{Application}

\theoremstyle{remark}
\newtheorem{remark}[theorem]{Remarque}

\counterwithin*{theorem}{section}

\newcommand{\applystyletotheorem}[1]{
	\tcolorboxenvironment{#1}{
		enhanced,
		breakable,
		colback=#1!98!white,
		boxrule=0pt,
		boxsep=0pt,
		left=8pt,
		right=8pt,
		top=8pt,
		bottom=8pt,
		sharp corners,
		after=\par,
	}
}

\applystyletotheorem{property}
\applystyletotheorem{proposition}
\applystyletotheorem{lemma}
\applystyletotheorem{theorem}
\applystyletotheorem{corollary}
\applystyletotheorem{definition}
\applystyletotheorem{notation}
\applystyletotheorem{example}
\applystyletotheorem{cexample}
\applystyletotheorem{application}
\applystyletotheorem{remark}
\applystyletotheorem{proof}

% Environnements :

\NewEnviron{whitetabularx}[1]{%
	\renewcommand{\arraystretch}{2.5}
	\colorbox{white}{%
		\begin{tabularx}{\textwidth}{#1}%
			\BODY%
		\end{tabularx}%
	}%
}

% Maths :

\DeclareFontEncoding{FMS}{}{}
\DeclareFontSubstitution{FMS}{futm}{m}{n}
\DeclareFontEncoding{FMX}{}{}
\DeclareFontSubstitution{FMX}{futm}{m}{n}
\DeclareSymbolFont{fouriersymbols}{FMS}{futm}{m}{n}
\DeclareSymbolFont{fourierlargesymbols}{FMX}{futm}{m}{n}
\DeclareMathDelimiter{\VERT}{\mathord}{fouriersymbols}{152}{fourierlargesymbols}{147}



\begin{document}
	%<*content>
	\development{analysis}{contre-exemple-au-theoreme-de-dirichlet}{Contre-exemple au théorème de Dirichlet}

	\summary{On construit un contre-exemple au théorème de Dirichlet qui montre l'importance de l'hypothèse $\mathcal{C}^1$ par morceaux.}

	\reference[GOU20]{275}
	
	\begin{cexample}
		Soit $f : \mathbb{R} \rightarrow \mathbb{R}$ paire, $2\pi$-périodique telle que :
		\[ \forall x \in [0, \pi], f(x) = \sum_{p=1}^{+\infty} \frac{1}{p^2} \sin \left( (2^{p^3} + 1) \frac{x}{2} \right)
		\]
		Alors $f$ est bien définie et continue sur $\mathbb{R}$. Cependant, sa série de Fourier diverge en $0$.
	\end{cexample}

	\begin{proof}
		\[ \forall x \in [0, \pi], \, \left| \frac{1}{p^2} \sin \left( (2^{p^3} + 1) \frac{x}{2} \right) \right| \leq \frac{1}{p^2} \]
		donc la série converge normalement sur $[0, \pi]$. Comme la fonction $x \mapsto \frac{1}{p^2} \sin \left( (2^{p^3} + 1) \frac{x}{2} \right)$ est continue (en tant que composée de fonctions continues), la fonction $f$ est continue sur $[0, \pi]$. On prolonge $f$ par parité en posant
		\[ \forall x \in [-\pi, 0[, \, f(x) = f(-x) \]
		Ainsi prolongée, $f$ est continue sur l'intervalle $[-\pi, \pi]$. Comme $f(\pi) = f(-\pi)$, on en déduit que $f$ est $2\pi$-périodique, et est donc continue sur $\mathbb{R}$ tout entier.
		\newpar
		Posons $\forall k \in \mathbb{N}$ :
		\[ \forall n \in \mathbb{N}, \, a_{n,k} = \int_0^{\pi} \cos(nt) \sin \left( \frac{(2k+1) t}{2} \, \mathrm{d}t \right) \text{ et } \forall q \in \mathbb{N}, \, s_{q,k} = \sum_{n=0}^{q} a_{n,k} \]
		Soient $k, n \in \mathbb{N}$. On va chercher à minorer $s_{n,k}$. Pour cela, calculons explicitement $a_{n,k}$ :
		\begin{align*}
			a_{n,k} &= \frac{1}{2} \int_0^{\pi} \sin \left( \left( \frac{2k+1}{2} + n \right) t \right) + \sin \left( \left( \frac{2k+1}{2} - n \right) t \right) \, \mathrm{d}t \\
			&= \frac{1}{2} \left( \frac{1}{k+n+\frac{1}{2}} + \frac{1}{k-n+\frac{1}{2}} \right) \\
			&= \frac{k+\frac{1}{2}}{\left( k + \frac{1}{2} \right)^2 - n^2}
		\end{align*}
		Par conséquent, à $k$ fixé, $a_{n,k} \geq 0$ pour tout $n \leq k$. Donc $s_{q,k} \geq 0$ pour tout $q \leq k$. Pour le cas $q > k$, on remarque que les $a_{q,k}$ sont, à un facteur $\frac{2}{\pi}$ près, les coefficients de Fourier $a_n(g_k)$ de la fonction paire
		\[ g_k : t \mapsto \left| \sin \left( (k + \frac{1}{2}) t \right) \right| \]
		qui est continue et $\mathcal{C}^1$ par morceaux. D'après le théorème de Dirichlet, sa série de Fourier converge simplement vers $g_k$ sur $\mathbb{R}$. En particulier, en $0$, cela donne :
		\[ \frac{a_{0,k}}{2} + \sum_{n=1}^{+\infty} a_{n,k} = \frac{\pi}{2} g_k(0) = 0 \]
		En faisant tendre $q$ vers $+\infty$, on a ainsi :
		\[ s_{q,k} \longrightarrow \frac{a_{0,k}}{2} \]
		Or, $a_{n,k}$ est positif pour $n \leq k$ et négatif pour $n > k$. Donc la suite $(s_{q,k})$ est décroissante à partir de l'indice $q = k$. Comme elle converge vers $\frac{a_{0,k}}{2}$, on en déduit que
		\[ \forall q > k, \, s_{q, k} \geq \frac{a_{0,k}}{2} \geq 0 \]
		Il nous reste à obtenir une minoration de $s_{k,k}$. Or, pour tout $k \in \mathbb{N}^*$,
		\[ s_{k,k} \geq \sum_{n=1}^k \frac{k + \frac{1}{2}}{(k + \frac{1}{2})^2 - n^2} \]
		Mais, la fonction $t \mapsto \frac{k + \frac{1}{2}}{(k + \frac{1}{2})^2 - t^2}$ est croissante. Donc par comparaison série-intégrale,
		\begin{align*}
			s_{k,k} &\geq \sum_{n=1}^k \int_{n-1}^n \frac{k + \frac{1}{2}}{(k + \frac{1}{2})^2 - t^2} \, \mathrm{d}t \\
			&= \int_0^k \frac{k + \frac{1}{2}}{(k + \frac{1}{2})^2 - t^2} \, \mathrm{d}t \\
			&= \frac{\ln(4k+3)}{2} \\
			&\geq \frac{\ln(k)}{2}
		\end{align*}
		Comme $f$ est paire, les coefficients de Fourier $b_n(f)$ sont nuls. Par ailleurs, $\forall n \in \mathbb{N}$,
		\begin{align*}
			a_n(f) &= \frac{2}{\pi} \int_0^{\pi} f(t) \cos(nt) \, \mathrm{d}t \\
			&= \frac{2}{\pi} \int_0^{\pi} \cos(nt) \sum_{p=1}^{+\infty} \frac{1}{p^2} \sin \left( (2^{p^3} + 1) \frac{t}{2} \right) \, \mathrm{d}t \\
			&= \frac{2}{\pi} \sum_{p=1}^{+\infty} \frac{1}{p^2} \int_0^{\pi} \sin \left( (2^{p^3} + 1) \frac{t}{2} \right) \, \mathrm{d}t
		\end{align*}
		l'interversion somme-intégrale étant licite par convergence normale sur un segment. Donc,
		\[ \forall n \in \mathbb{N}, \, a_n(f) = \frac{2}{\pi} \sum_{p=1}^{+\infty} \frac{1}{p^2} a_{n,2^{p^3 - 1}} \implies \forall n \in \mathbb{N}, \, S_n = \frac{\pi}{2} \sum_{k=0}^{n} a_k(f) = \sum_{p=1}^{+\infty} \frac{1}{p^2} s_{n,2^{p^3 - 1}} \]
		Comme les $s_{q,k}$ sont positifs et que $s_{k,k} \geq \frac{\ln(k)}{2}$, on en déduit
		\[ \forall p \in \mathbb{N}, \, S_{2^{p^3 - 1}} \geq \frac{1}{p^2} s_{2^{p^3 - 1}, 2^{p^3 - 1}} \geq \frac{1}{2p^2} \ln(2^{p^3 - 1}) = \frac{p^3 - 1}{2 p^2} \ln(2) \longrightarrow +\infty \]
	\end{proof}
	%</content>
\end{document}
