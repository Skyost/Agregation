\documentclass[12pt, a4paper]{report}

% LuaLaTeX :

\RequirePackage{iftex}
\RequireLuaTeX

% Packages :

\usepackage[french]{babel}
%\usepackage[utf8]{inputenc}
%\usepackage[T1]{fontenc}
\usepackage[pdfencoding=auto, pdfauthor={Hugo Delaunay}, pdfsubject={Mathématiques}, pdfcreator={agreg.skyost.eu}]{hyperref}
\usepackage{amsmath}
\usepackage{amsthm}
%\usepackage{amssymb}
\usepackage{stmaryrd}
\usepackage{tikz}
\usepackage{tkz-euclide}
\usepackage{fourier-otf}
\usepackage{fontspec}
\usepackage{titlesec}
\usepackage{fancyhdr}
\usepackage{catchfilebetweentags}
\usepackage[french, capitalise, noabbrev]{cleveref}
\usepackage[fit, breakall]{truncate}
\usepackage[top=2.5cm, right=2cm, bottom=2.5cm, left=2cm]{geometry}
\usepackage{enumerate}
\usepackage{tocloft}
\usepackage{microtype}
%\usepackage{mdframed}
%\usepackage{thmtools}
\usepackage{xcolor}
\usepackage{tabularx}
\usepackage{aligned-overset}
\usepackage[subpreambles=true]{standalone}
\usepackage{environ}
\usepackage[normalem]{ulem}
\usepackage{marginnote}
\usepackage{etoolbox}
\usepackage{setspace}
\usepackage[bibstyle=reading, citestyle=draft]{biblatex}
\usepackage{xpatch}
\usepackage[many, breakable]{tcolorbox}
\usepackage[backgroundcolor=white, bordercolor=white, textsize=small]{todonotes}

% Bibliographie :

\newcommand{\overridebibliographypath}[1]{\providecommand{\bibliographypath}{#1}}
\overridebibliographypath{../bibliography.bib}
\addbibresource{\bibliographypath}
\defbibheading{bibliography}[\bibname]{%
	\newpage
	\section*{#1}%
}
\renewbibmacro*{entryhead:full}{\printfield{labeltitle}}
\DeclareFieldFormat{url}{\newline\footnotesize\url{#1}}
\AtEndDocument{\printbibliography}

% Police :

\setmathfont{Erewhon Math}

% Tikz :

\usetikzlibrary{calc}

% Longueurs :

\setlength{\parindent}{0pt}
\setlength{\headheight}{15pt}
\setlength{\fboxsep}{0pt}
\titlespacing*{\chapter}{0pt}{-20pt}{10pt}
\setlength{\marginparwidth}{1.5cm}
\setstretch{1.1}

% Métadonnées :

\author{agreg.skyost.eu}
\date{\today}

% Titres :

\setcounter{secnumdepth}{3}

\renewcommand{\thechapter}{\Roman{chapter}}
\renewcommand{\thesubsection}{\Roman{subsection}}
\renewcommand{\thesubsubsection}{\arabic{subsubsection}}
\renewcommand{\theparagraph}{\alph{paragraph}}

\titleformat{\chapter}{\huge\bfseries}{\thechapter}{20pt}{\huge\bfseries}
\titleformat*{\section}{\LARGE\bfseries}
\titleformat{\subsection}{\Large\bfseries}{\thesubsection \, - \,}{0pt}{\Large\bfseries}
\titleformat{\subsubsection}{\large\bfseries}{\thesubsubsection. \,}{0pt}{\large\bfseries}
\titleformat{\paragraph}{\bfseries}{\theparagraph. \,}{0pt}{\bfseries}

\setcounter{secnumdepth}{4}

% Table des matières :

\renewcommand{\cftsecleader}{\cftdotfill{\cftdotsep}}
\addtolength{\cftsecnumwidth}{10pt}

% Redéfinition des commandes :

\renewcommand*\thesection{\arabic{section}}
\renewcommand{\ker}{\mathrm{Ker}}

% Nouvelles commandes :

\newcommand{\website}{https://agreg.skyost.eu}

\newcommand{\tr}[1]{\mathstrut ^t #1}
\newcommand{\im}{\mathrm{Im}}
\newcommand{\rang}{\operatorname{rang}}
\newcommand{\trace}{\operatorname{trace}}
\newcommand{\id}{\operatorname{id}}
\newcommand{\stab}{\operatorname{Stab}}

\providecommand{\newpar}{\\[\medskipamount]}

\providecommand{\lesson}[3]{%
	\title{#3}%
	\hypersetup{pdftitle={#3}}%
	\setcounter{section}{\numexpr #2 - 1}%
	\section{#3}%
	\fancyhead[R]{\truncate{0.73\textwidth}{#2 : #3}}%
}

\providecommand{\development}[3]{%
	\title{#3}%
	\hypersetup{pdftitle={#3}}%
	\section*{#3}%
	\fancyhead[R]{\truncate{0.73\textwidth}{#3}}%
}

\providecommand{\summary}[1]{%
	\textit{#1}%
	\medskip%
}

\tikzset{notestyleraw/.append style={inner sep=0pt, rounded corners=0pt, align=center}}

%\newcommand{\booklink}[1]{\website/bibliographie\##1}
\newcommand{\citelink}[2]{\hyperlink{cite.\therefsection @#1}{#2}}
\newcommand{\previousreference}{}
\providecommand{\reference}[2][]{%
	\notblank{#1}{\renewcommand{\previousreference}{#1}}{}%
	\todo[noline]{%
		\protect\vspace{16pt}%
		\protect\par%
		\protect\notblank{#1}{\cite{[\previousreference]}\\}{}%
		\protect\citelink{\previousreference}{p. #2}%
	}%
}

\definecolor{devcolor}{HTML}{00695c}
\newcommand{\dev}[1]{%
	\reversemarginpar%
	\todo[noline]{
		\protect\vspace{16pt}%
		\protect\par%
		\bfseries\color{devcolor}\href{\website/developpements/#1}{DEV}
	}%
	\normalmarginpar%
}

% En-têtes :

\pagestyle{fancy}
\fancyhead[L]{\truncate{0.23\textwidth}{\thepage}}
\fancyfoot[C]{\scriptsize \href{\website}{\texttt{agreg.skyost.eu}}}

% Couleurs :

\definecolor{property}{HTML}{fffde7}
\definecolor{proposition}{HTML}{fff8e1}
\definecolor{lemma}{HTML}{fff3e0}
\definecolor{theorem}{HTML}{fce4f2}
\definecolor{corollary}{HTML}{ffebee}
\definecolor{definition}{HTML}{ede7f6}
\definecolor{notation}{HTML}{f3e5f5}
\definecolor{example}{HTML}{e0f7fa}
\definecolor{cexample}{HTML}{efebe9}
\definecolor{application}{HTML}{e0f2f1}
\definecolor{remark}{HTML}{e8f5e9}
\definecolor{proof}{HTML}{e1f5fe}

% Théorèmes :

\theoremstyle{definition}
\newtheorem{theorem}{Théorème}

\newtheorem{property}[theorem]{Propriété}
\newtheorem{proposition}[theorem]{Proposition}
\newtheorem{lemma}[theorem]{Lemme}
\newtheorem{corollary}[theorem]{Corollaire}

\newtheorem{definition}[theorem]{Définition}
\newtheorem{notation}[theorem]{Notation}

\newtheorem{example}[theorem]{Exemple}
\newtheorem{cexample}[theorem]{Contre-exemple}
\newtheorem{application}[theorem]{Application}

\theoremstyle{remark}
\newtheorem{remark}[theorem]{Remarque}

\counterwithin*{theorem}{section}

\newcommand{\applystyletotheorem}[1]{
	\tcolorboxenvironment{#1}{
		enhanced,
		breakable,
		colback=#1!98!white,
		boxrule=0pt,
		boxsep=0pt,
		left=8pt,
		right=8pt,
		top=8pt,
		bottom=8pt,
		sharp corners,
		after=\par,
	}
}

\applystyletotheorem{property}
\applystyletotheorem{proposition}
\applystyletotheorem{lemma}
\applystyletotheorem{theorem}
\applystyletotheorem{corollary}
\applystyletotheorem{definition}
\applystyletotheorem{notation}
\applystyletotheorem{example}
\applystyletotheorem{cexample}
\applystyletotheorem{application}
\applystyletotheorem{remark}
\applystyletotheorem{proof}

% Environnements :

\NewEnviron{whitetabularx}[1]{%
	\renewcommand{\arraystretch}{2.5}
	\colorbox{white}{%
		\begin{tabularx}{\textwidth}{#1}%
			\BODY%
		\end{tabularx}%
	}%
}

% Maths :

\DeclareFontEncoding{FMS}{}{}
\DeclareFontSubstitution{FMS}{futm}{m}{n}
\DeclareFontEncoding{FMX}{}{}
\DeclareFontSubstitution{FMX}{futm}{m}{n}
\DeclareSymbolFont{fouriersymbols}{FMS}{futm}{m}{n}
\DeclareSymbolFont{fourierlargesymbols}{FMX}{futm}{m}{n}
\DeclareMathDelimiter{\VERT}{\mathord}{fouriersymbols}{152}{fourierlargesymbols}{147}



\begin{document}
	%<*content>
	\development{analysis}{transformee-de-fourier-d-une-gaussienne}{Transformée de Fourier d'une gaussienne}

	\summary{On calcule la transformée de Fourier d'une fonction de type gaussienne $x \mapsto e^{-ax^2}$ à l'aide du théorème intégral de Cauchy.}

	\reference[AMR08]{156}

	\begin{proposition}
		On définit $\forall a \in \mathbb{R}^+_*$,
		\[ \gamma_a :
		\begin{array}{cl}
			\mathbb{R} &\rightarrow \mathbb{R} \\
			x &\mapsto e^{-ax^2}
		\end{array}
		\]
		Alors,
		\[ \forall \xi \in \mathbb{R}, \, \widehat{\gamma_a}(\xi) = \sqrt{\frac{\pi}{a}} e^{\frac{- \xi^2}{4a}} \]
	\end{proposition}

	\begin{proof}
		Soit $a \in \mathbb{R}^+_*$. On a
		\[ \forall \xi \in \mathbb{R}, \, \widehat{\gamma_a}(\xi) = \int_{-\infty}^{+\infty} e^{-ax^2} e^{-ix\xi} \, \mathrm{d}x \]
		et en écrivant
		\[ ax^2 + ix\xi = a \left( x^2 + i \frac{x \xi}{a} \right) = a \left( \left( x + i \frac{\xi}{2a} \right)^2 + \frac{\xi^2}{4a^2} \right) \]
		on en déduit que
		\[ \forall \xi \in \mathbb{R}, \, \widehat{\gamma_a}(\xi) = e^{-\frac{\xi^2}{4a}} \int_{-\infty}^{+\infty} e^{-a \left( x + i \frac{\xi}{2a} \right)^2} \, \mathrm{d}x \tag{$*$} \]
		On va considérer la fonction
		\[
		\begin{array}{cl}
			\mathbb{C} &\rightarrow \mathbb{C} \\
			z &\mapsto e^{-az^2}
		\end{array}
		\]
		Pour $R > 0$ et $\xi \in \mathbb{R}$, on note $\Gamma(R)$ le rectangle de sommets $-R, R, R + i\frac{\xi}{2a}, -R + i\frac{\xi}{2a}$ parcouru dans le sens direct :
		\includelatexpicture{transformee-de-fourier-d-une-gaussienne}
		On a,
		\[ \underbrace{\int_{\Gamma(R)} e^{-az^2} \, \mathrm{d}z}_{= I(R)} = \underbrace{\int_{-R}^R e^{-az^2} \, \mathrm{d}z}_{= I_1(R)} + \underbrace{\int_R^{R + i\frac{\xi}{2a}} e^{-az^2} \, \mathrm{d}z}_{= I_2(R)} + \underbrace{\int_{R + i\frac{\xi}{2a}}^{-R + i\frac{\xi}{2a}} e^{-az^2} \, \mathrm{d}z}_{= I_3(R)} + \underbrace{\int_{-R + i\frac{\xi}{2a}}^{-R} e^{-az^2} \, \mathrm{d}z}_{= I_4(R)} \]
		Nous allons traiter les intégrales séparément.
		\begin{itemize}
			\item \underline{Pour $I_1(R)$ :} On a affaire à une intégrale sur l'axe réel. Or, on connait la valeur de l'intégrale de Gauss :
			\[ \int_{-\infty}^{+\infty} e^{-y^2} \, \mathrm{d}y = \sqrt{\pi} \]
			Donc en faisant le changement de variable $y = \sqrt{a}x$, on obtient :
			\[ \sqrt{a} \int_{-\infty}^{+\infty} e^{-ax^2} \, \mathrm{d}x = \sqrt{\pi} \iff \int_{-\infty}^{+\infty} e^{-ax^2} \, \mathrm{d}x = \sqrt{\frac{\pi}{a}} \]
			D'où :
			\[ I_1(R) \longrightarrow \sqrt{\frac{\pi}{a}} \]
			quand $R \longrightarrow +\infty$.
			\item \underline{Pour $I_2(R)$ :} On a :
			\begin{align*}
				&\forall z \in \left[ R, R + i \frac{\xi}{2a} \right], \, z = R + it \text{ avec $t \in \left[ 0, \frac{\xi}{2a} \right]$} \\
				\implies& \mathrm{d}z = i\mathrm{d}t
			\end{align*}
			D'où :
			\[ I_2(R) = i \int_0^{\frac{\xi}{2a}} e^{-a (R+it)^2} \, \mathrm{d}t \]
			On en déduit,
			\begin{align*}
				|I_2(R)| &\leq \int_0^{\frac{\xi}{2a}} \left| e^{-a (R+it)^2} \right| \, \mathrm{d}t \\
				&= \int_0^{\frac{\xi}{2a}} \left| e^{-a(R^2 - t^2)} \right| \underbrace{\left| e^{i 2aRt} \right|}_{= 1} \, \mathrm{d}t \\
				&= \int_0^{\frac{\xi}{2a}} e^{-a(R^2 - t^2)} \, \mathrm{d}t \\
				&= e^{-aR^2} \int_0^{\frac{\xi}{2a}} e^{at^2} \, \mathrm{d}t \\
				&\longrightarrow 0
			\end{align*}
			quand $R \longrightarrow +\infty$.
			\item \underline{Pour $I_3(R)$ :} On a :
			\begin{align*}
				&\forall z \in \left[ R + i \frac{\xi}{2a}, -R + i \frac{\xi}{2a} \right], \, z = t + i\frac{\xi}{2a} \text{ avec $t \in \left[ R, -R \right]$} \\
				\implies& \mathrm{d}z = \mathrm{d}t
			\end{align*}
			D'où :
			\[ I_3(R) = \int_R^{-R} e^{-a \left(t + i \frac{\xi}{2a} \right)^2} \, \mathrm{d}t = - \int_{-R}^R e^{-a \left(t + i \frac{\xi}{2a} \right)^2} \, \mathrm{d}t = - e^{\frac{\xi^2}{4a}} \int_{-R}^R e^{-a \left( t + i \frac{\xi}{2a} \right)^2} \, \mathrm{d}t \]
			qui est une intégrale généralisée absolument convergente. Ainsi par $(*)$,
			\[ I_3(R) \longrightarrow - e^{\frac{\xi^2}{4a}} \widehat{\gamma_a}(\xi) \]
			quand $R \longrightarrow +\infty$.
			\item \underline{Pour $I_4(R)$ :} Ce cas-ci se traite exactement comme $I_2(R)$. On a :
			\begin{align*}
				&\forall z \in \left[ -R + i \frac{\xi}{2a}, -R \right], \, z = -R + it \text{ avec $t \in \left[ \frac{\xi}{2a}, 0 \right]$} \\
				\implies& \mathrm{d}z = i\mathrm{d}t
			\end{align*}
			D'où :
			\[ I_4(R) = i \int_{\frac{\xi}{2a}}^0 e^{-a (-R+it)^2} \, \mathrm{d}t = -i \int_0^{\frac{\xi}{2a}} e^{-a (-R+it)^2} \, \mathrm{d}t \]
			On en déduit,
			\[ |I_4(R)| \leq \int_0^{\frac{\xi}{2a}} \left| e^{-a (-R+it)^2} \right| \, \mathrm{d}t = e^{-aR^2} \int_0^{\frac{\xi}{2a}} e^{at^2} \, \mathrm{d}t \longrightarrow 0 \]
			quand $R \longrightarrow +\infty$.
			\item \underline{Pour $I(R)$ :} La fonction $z \mapsto e^{-az^2}$ est holomorphe et le contour $\Gamma(R)$ est fermé. Donc $I(R) = 0$ en vertu du théorème intégral de Cauchy.
		\end{itemize}
		En passant à la limite, on obtient ainsi :
		\[ 0 = \sqrt{\frac{\pi}{a}} + 0 - e^{\frac{\xi^2}{4a}} \widehat{\gamma_a}(\xi) + 0 \iff \widehat{\gamma_a}(\xi) = \sqrt{\frac{\pi}{a}} e^{\frac{- \xi^2}{4a}} \]
	\end{proof}
	%</content>
\end{document}
