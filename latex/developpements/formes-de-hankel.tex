\documentclass[12pt, a4paper]{report}

% LuaLaTeX :

\RequirePackage{iftex}
\RequireLuaTeX

% Packages :

\usepackage[french]{babel}
%\usepackage[utf8]{inputenc}
%\usepackage[T1]{fontenc}
\usepackage[pdfencoding=auto, pdfauthor={Hugo Delaunay}, pdfsubject={Mathématiques}, pdfcreator={agreg.skyost.eu}]{hyperref}
\usepackage{amsmath}
\usepackage{amsthm}
%\usepackage{amssymb}
\usepackage{stmaryrd}
\usepackage{tikz}
\usepackage{tkz-euclide}
\usepackage{fourier-otf}
\usepackage{fontspec}
\usepackage{titlesec}
\usepackage{fancyhdr}
\usepackage{catchfilebetweentags}
\usepackage[french, capitalise, noabbrev]{cleveref}
\usepackage[fit, breakall]{truncate}
\usepackage[top=2.5cm, right=2cm, bottom=2.5cm, left=2cm]{geometry}
\usepackage{enumerate}
\usepackage{tocloft}
\usepackage{microtype}
%\usepackage{mdframed}
%\usepackage{thmtools}
\usepackage{xcolor}
\usepackage{tabularx}
\usepackage{aligned-overset}
\usepackage[subpreambles=true]{standalone}
\usepackage{environ}
\usepackage[normalem]{ulem}
\usepackage{marginnote}
\usepackage{etoolbox}
\usepackage{setspace}
\usepackage[bibstyle=reading, citestyle=draft]{biblatex}
\usepackage{xpatch}
\usepackage[many, breakable]{tcolorbox}
\usepackage[backgroundcolor=white, bordercolor=white, textsize=small]{todonotes}

% Bibliographie :

\newcommand{\overridebibliographypath}[1]{\providecommand{\bibliographypath}{#1}}
\overridebibliographypath{../bibliography.bib}
\addbibresource{\bibliographypath}
\defbibheading{bibliography}[\bibname]{%
	\newpage
	\section*{#1}%
}
\renewbibmacro*{entryhead:full}{\printfield{labeltitle}}
\DeclareFieldFormat{url}{\newline\footnotesize\url{#1}}
\AtEndDocument{\printbibliography}

% Police :

\setmathfont{Erewhon Math}

% Tikz :

\usetikzlibrary{calc}

% Longueurs :

\setlength{\parindent}{0pt}
\setlength{\headheight}{15pt}
\setlength{\fboxsep}{0pt}
\titlespacing*{\chapter}{0pt}{-20pt}{10pt}
\setlength{\marginparwidth}{1.5cm}
\setstretch{1.1}

% Métadonnées :

\author{agreg.skyost.eu}
\date{\today}

% Titres :

\setcounter{secnumdepth}{3}

\renewcommand{\thechapter}{\Roman{chapter}}
\renewcommand{\thesubsection}{\Roman{subsection}}
\renewcommand{\thesubsubsection}{\arabic{subsubsection}}
\renewcommand{\theparagraph}{\alph{paragraph}}

\titleformat{\chapter}{\huge\bfseries}{\thechapter}{20pt}{\huge\bfseries}
\titleformat*{\section}{\LARGE\bfseries}
\titleformat{\subsection}{\Large\bfseries}{\thesubsection \, - \,}{0pt}{\Large\bfseries}
\titleformat{\subsubsection}{\large\bfseries}{\thesubsubsection. \,}{0pt}{\large\bfseries}
\titleformat{\paragraph}{\bfseries}{\theparagraph. \,}{0pt}{\bfseries}

\setcounter{secnumdepth}{4}

% Table des matières :

\renewcommand{\cftsecleader}{\cftdotfill{\cftdotsep}}
\addtolength{\cftsecnumwidth}{10pt}

% Redéfinition des commandes :

\renewcommand*\thesection{\arabic{section}}
\renewcommand{\ker}{\mathrm{Ker}}

% Nouvelles commandes :

\newcommand{\website}{https://agreg.skyost.eu}

\newcommand{\tr}[1]{\mathstrut ^t #1}
\newcommand{\im}{\mathrm{Im}}
\newcommand{\rang}{\operatorname{rang}}
\newcommand{\trace}{\operatorname{trace}}
\newcommand{\id}{\operatorname{id}}
\newcommand{\stab}{\operatorname{Stab}}

\providecommand{\newpar}{\\[\medskipamount]}

\providecommand{\lesson}[3]{%
	\title{#3}%
	\hypersetup{pdftitle={#3}}%
	\setcounter{section}{\numexpr #2 - 1}%
	\section{#3}%
	\fancyhead[R]{\truncate{0.73\textwidth}{#2 : #3}}%
}

\providecommand{\development}[3]{%
	\title{#3}%
	\hypersetup{pdftitle={#3}}%
	\section*{#3}%
	\fancyhead[R]{\truncate{0.73\textwidth}{#3}}%
}

\providecommand{\summary}[1]{%
	\textit{#1}%
	\medskip%
}

\tikzset{notestyleraw/.append style={inner sep=0pt, rounded corners=0pt, align=center}}

%\newcommand{\booklink}[1]{\website/bibliographie\##1}
\newcommand{\citelink}[2]{\hyperlink{cite.\therefsection @#1}{#2}}
\newcommand{\previousreference}{}
\providecommand{\reference}[2][]{%
	\notblank{#1}{\renewcommand{\previousreference}{#1}}{}%
	\todo[noline]{%
		\protect\vspace{16pt}%
		\protect\par%
		\protect\notblank{#1}{\cite{[\previousreference]}\\}{}%
		\protect\citelink{\previousreference}{p. #2}%
	}%
}

\definecolor{devcolor}{HTML}{00695c}
\newcommand{\dev}[1]{%
	\reversemarginpar%
	\todo[noline]{
		\protect\vspace{16pt}%
		\protect\par%
		\bfseries\color{devcolor}\href{\website/developpements/#1}{DEV}
	}%
	\normalmarginpar%
}

% En-têtes :

\pagestyle{fancy}
\fancyhead[L]{\truncate{0.23\textwidth}{\thepage}}
\fancyfoot[C]{\scriptsize \href{\website}{\texttt{agreg.skyost.eu}}}

% Couleurs :

\definecolor{property}{HTML}{fffde7}
\definecolor{proposition}{HTML}{fff8e1}
\definecolor{lemma}{HTML}{fff3e0}
\definecolor{theorem}{HTML}{fce4f2}
\definecolor{corollary}{HTML}{ffebee}
\definecolor{definition}{HTML}{ede7f6}
\definecolor{notation}{HTML}{f3e5f5}
\definecolor{example}{HTML}{e0f7fa}
\definecolor{cexample}{HTML}{efebe9}
\definecolor{application}{HTML}{e0f2f1}
\definecolor{remark}{HTML}{e8f5e9}
\definecolor{proof}{HTML}{e1f5fe}

% Théorèmes :

\theoremstyle{definition}
\newtheorem{theorem}{Théorème}

\newtheorem{property}[theorem]{Propriété}
\newtheorem{proposition}[theorem]{Proposition}
\newtheorem{lemma}[theorem]{Lemme}
\newtheorem{corollary}[theorem]{Corollaire}

\newtheorem{definition}[theorem]{Définition}
\newtheorem{notation}[theorem]{Notation}

\newtheorem{example}[theorem]{Exemple}
\newtheorem{cexample}[theorem]{Contre-exemple}
\newtheorem{application}[theorem]{Application}

\theoremstyle{remark}
\newtheorem{remark}[theorem]{Remarque}

\counterwithin*{theorem}{section}

\newcommand{\applystyletotheorem}[1]{
	\tcolorboxenvironment{#1}{
		enhanced,
		breakable,
		colback=#1!98!white,
		boxrule=0pt,
		boxsep=0pt,
		left=8pt,
		right=8pt,
		top=8pt,
		bottom=8pt,
		sharp corners,
		after=\par,
	}
}

\applystyletotheorem{property}
\applystyletotheorem{proposition}
\applystyletotheorem{lemma}
\applystyletotheorem{theorem}
\applystyletotheorem{corollary}
\applystyletotheorem{definition}
\applystyletotheorem{notation}
\applystyletotheorem{example}
\applystyletotheorem{cexample}
\applystyletotheorem{application}
\applystyletotheorem{remark}
\applystyletotheorem{proof}

% Environnements :

\NewEnviron{whitetabularx}[1]{%
	\renewcommand{\arraystretch}{2.5}
	\colorbox{white}{%
		\begin{tabularx}{\textwidth}{#1}%
			\BODY%
		\end{tabularx}%
	}%
}

% Maths :

\DeclareFontEncoding{FMS}{}{}
\DeclareFontSubstitution{FMS}{futm}{m}{n}
\DeclareFontEncoding{FMX}{}{}
\DeclareFontSubstitution{FMX}{futm}{m}{n}
\DeclareSymbolFont{fouriersymbols}{FMS}{futm}{m}{n}
\DeclareSymbolFont{fourierlargesymbols}{FMX}{futm}{m}{n}
\DeclareMathDelimiter{\VERT}{\mathord}{fouriersymbols}{152}{fourierlargesymbols}{147}



\begin{document}
	%<*content>
	\development{algebra}{formes-de-hankel}{Formes de Hankel}

	\summary{Le but de ce développement est de construire une forme quadratique permettant de dénombrer les racines réelles distinctes d'un polynôme en fonction de ses racines complexes.}

	\reference[C-G]{339}

	Soit $P \in \mathbb{R}[X]$ un polynôme de degré $n$.

	\begin{theorem}[Formes de Hankel]
		On note $x_1, \dots, x_t$ les racines complexes de $P$ de multiplicités respectives $m_1, \dots, m_t$. On pose
		\[ s_0 = n \text{ et } \forall k \geq 1, \, s_k = \sum_{i=1}^t m_i x_i^k \]
		Alors :
		\begin{enumerate}[(i)]
			\item $\sigma = \sum_{i, j \in \llbracket 0, n-1 \rrbracket} s_{i+j} X_i X_j$ est définit forme quadratique sur $\mathbb{C}^n$ ainsi qu'une forme quadratique $\sigma_{\mathbb{R}}$ sur $\mathbb{R}^n$.
			\item Si on note $(p,q)$ la signature de $\sigma_{\mathbb{R}}$, on a :
			\begin{itemize}
				\item $t = p + q$.
				\item Le nombre de racines réelles distinctes de $P$ est $p-q$.
			\end{itemize}
		\end{enumerate}
	\end{theorem}

	\begin{proof}
		$\sigma$ est un polynôme homogène de degré $2$ sur $\mathbb{C}$ (car la somme des exposants est $2$ pour chacun des monômes), qui définit donc une forme quadratique sur $\mathbb{C}^n$. De plus, on peut écrire :
		\[ \forall k \geq 1, \, s_k = \sum_{\substack{x \text{ racine de P} \\ x \in \mathbb{R}}} m_k x^k + \sum_{\substack{x \text{ racine de P} \\ x \in \mathbb{C}}} m_k (x^k + \overline{x}^k) \]
		donc $s_k = \overline{s_k}$ ie. $s_k \in \mathbb{R}$. Donc $\sigma$ définit une forme quadratique $\sigma_{\mathbb{R}}$ sur $\mathbb{R}^n$. D'où le point $(i)$.
		\newpar
		Soit $\varphi_k$ la forme linéaire sur $\mathbb{C}^n$ définie par le polynôme homogène de degré $1$
		\[ P_k(X_0, \dots, X_{n-1}) = X_0 + x_k X_1 + \dots + x_k^{n-1} X_{n-1} \]
		pour $k \in \llbracket 0, t \rrbracket$. Dans la base duale $(e_i^*)_{i \in \llbracket 0, n-1 \rrbracket}$ de la base canonique $(e_i)_{i \in \llbracket 0, n-1 \rrbracket}$ de $\mathbb{C}^n$, on a
		\[ \varphi_k = e_0^* + x_k e_1^* + \dots + x_k^{n-1} e_{n-1}^* \]
		Et comme
		\[ \det((\varphi_k)_{k \in \llbracket 0, t \rrbracket}) = \begin{vmatrix} 1 & 1 & \dots & 1 \\ x_0 & x_1 & \dots & x_t \\ \vdots & \ddots & \ddots & \vdots \\ x_0^{n-1} & x_1^{n-1} & \dots & x_t^{n-1} \end{vmatrix} \overset{\text{Vandermonde}}{\neq} 0 \]
		la famille $(\varphi_k)_{k \in \llbracket 0, t \rrbracket}$ est de rang $t$ sur $\mathbb{C}$. Or, le coefficient de $X_i X_j$ dans $\sum_{k=1}^t m_k \varphi_k^2$ vaut
		\[ \begin{cases} \sum_{k=1}^t m_k x_k^{2i} = s_{i+j} &\text{ si } i=j \\ \sum_{k=1}^t 2 m_k x_k^i x_k^j = \sum_{k=1}^t 2 m_k x_k^{i+j} = 2s_{i+j} &\text{ sinon} \end{cases} \]
		donc, $\sigma = \sum_{k=1}^t m_k \varphi_k^2$. En particulier, $\rang (\sigma) = t$ par indépendant des $\varphi_k$. On en déduit,
		\[ p+q = \rang(\sigma) = \rang(\sigma_{\mathbb{R}}) = t \]
		(le rang est invariant par extension de corps).
		\newpar
		Soit $k \in \llbracket 0, t \rrbracket$. Calculons la signature de la forme quadratique $\varphi_k^2 + \overline{\varphi_k}^2$ :
		\begin{itemize}
			\item Si $x_k \in \mathbb{R}$, on a $\varphi_k^2 + \overline{\varphi_k}^2 = 2 \varphi_k^2$, qui est de signature $(1, 0)$ car $\varphi_k \neq 0$.
			\item Si $x_k \notin \mathbb{R}$, on a $\varphi_k^2 + \overline{\varphi_k}^2 = 2 \operatorname{Re}(\varphi_k)^2 - 2 \operatorname{Im}(\varphi_k)^2$ qui est bien une forme quadratique réelle. Et $x_k = \overline{x_k}$, donc la matrice
			\[ \begin{pmatrix} 1 & 1 \\ x_k & \overline{x_k} \\ \vdots & \vdots \\ x_k^{n-1} & \overline{x_k}^{n-1} \end{pmatrix} \]
			est de rang $2$ (cf. le mineur correspondant aux deux premières lignes). Donc $\varphi_k$ et $\overline{\varphi_k}$ sont indépendantes. Ainsi, $\rang(\varphi_k^2 + \overline{\varphi_k}^2) = 2$ sur $\mathbb{C}$, donc sur $\mathbb{R}$ aussi (toujours par invariance du rang par extension de corps). Donc la signature de $\varphi_k^2 + \overline{\varphi_k}^2$ est $(1, 1)$.
		\end{itemize}
		Maintenant, regroupons les $\varphi_k$ conjuguées entre elles lorsqu'elles ne sont pas réelles :
		\[ \sigma = \sum_{\substack{k=1 \\ x_k \in \mathbb{R}}}^t m_k \varphi_k^2 + \sum_{\substack{k=1 \\ x_k \notin \mathbb{R}}}^t m_k (\varphi_k^2 + \overline{\varphi_k}^2) \]
		En passant à la signature, on obtient :
		\[ (p, q) = (r, 0) + \left( \frac{t-r}{2}, \frac{t-r}{2} \right) = \left( \frac{t+r}{2}, \frac{t-r}{2} \right) \]
		où $r$ désigne le nombre de racines réelles distinctes de $P$. Par unicité de la signature d’une forme quadratique réelle, on a bien $p-q=r$. D'où le point $(ii)$.
	\end{proof}

	\begin{remark}
		Tout l'intérêt de ces formes quadratiques est qu'on peut calculer les $s_k$ par récurrence en utilisant les polynômes symétriques élémentaires, sans avoir besoin des racines.
	\end{remark}

	\reference[GOU21]{80}

	\begin{proposition}[Sommes de Newton]
		On pose $P = \sum_{k=0}^n a_k X^k$. Les sommes de Newton vérifient les relations suivantes :
		\begin{enumerate}[(i)]
			\item $s_0 = n$.
			\item $\forall k \geq 1$, $s_k = -k a_{n-k} \sum_{i=1}^{k-1} s_i a_{n-k+i}$.
			\item $\forall m \in \mathbb{N}^*$, $s_{n+m} = \sum_{i=0}^{n-1} s_{m+i} a_i$.
		\end{enumerate}
	\end{proposition}
	%</content>
\end{document}
