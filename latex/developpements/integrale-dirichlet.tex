\documentclass[12pt, a4paper]{report}

% LuaLaTeX :

\RequirePackage{iftex}
\RequireLuaTeX

% Packages :

\usepackage[french]{babel}
%\usepackage[utf8]{inputenc}
%\usepackage[T1]{fontenc}
\usepackage[pdfencoding=auto, pdfauthor={Hugo Delaunay}, pdfsubject={Mathématiques}, pdfcreator={agreg.skyost.eu}]{hyperref}
\usepackage{amsmath}
\usepackage{amsthm}
%\usepackage{amssymb}
\usepackage{stmaryrd}
\usepackage{tikz}
\usepackage{tkz-euclide}
\usepackage{fourier-otf}
\usepackage{fontspec}
\usepackage{titlesec}
\usepackage{fancyhdr}
\usepackage{catchfilebetweentags}
\usepackage[french, capitalise, noabbrev]{cleveref}
\usepackage[fit, breakall]{truncate}
\usepackage[top=2.5cm, right=2cm, bottom=2.5cm, left=2cm]{geometry}
\usepackage{enumerate}
\usepackage{tocloft}
\usepackage{microtype}
%\usepackage{mdframed}
%\usepackage{thmtools}
\usepackage{xcolor}
\usepackage{tabularx}
\usepackage{aligned-overset}
\usepackage[subpreambles=true]{standalone}
\usepackage{environ}
\usepackage[normalem]{ulem}
\usepackage{marginnote}
\usepackage{etoolbox}
\usepackage{setspace}
\usepackage[bibstyle=reading, citestyle=draft]{biblatex}
\usepackage{xpatch}
\usepackage[many, breakable]{tcolorbox}
\usepackage[backgroundcolor=white, bordercolor=white, textsize=small]{todonotes}

% Bibliographie :

\newcommand{\overridebibliographypath}[1]{\providecommand{\bibliographypath}{#1}}
\overridebibliographypath{../bibliography.bib}
\addbibresource{\bibliographypath}
\defbibheading{bibliography}[\bibname]{%
	\newpage
	\section*{#1}%
}
\renewbibmacro*{entryhead:full}{\printfield{labeltitle}}
\DeclareFieldFormat{url}{\newline\footnotesize\url{#1}}
\AtEndDocument{\printbibliography}

% Police :

\setmathfont{Erewhon Math}

% Tikz :

\usetikzlibrary{calc}

% Longueurs :

\setlength{\parindent}{0pt}
\setlength{\headheight}{15pt}
\setlength{\fboxsep}{0pt}
\titlespacing*{\chapter}{0pt}{-20pt}{10pt}
\setlength{\marginparwidth}{1.5cm}
\setstretch{1.1}

% Métadonnées :

\author{agreg.skyost.eu}
\date{\today}

% Titres :

\setcounter{secnumdepth}{3}

\renewcommand{\thechapter}{\Roman{chapter}}
\renewcommand{\thesubsection}{\Roman{subsection}}
\renewcommand{\thesubsubsection}{\arabic{subsubsection}}
\renewcommand{\theparagraph}{\alph{paragraph}}

\titleformat{\chapter}{\huge\bfseries}{\thechapter}{20pt}{\huge\bfseries}
\titleformat*{\section}{\LARGE\bfseries}
\titleformat{\subsection}{\Large\bfseries}{\thesubsection \, - \,}{0pt}{\Large\bfseries}
\titleformat{\subsubsection}{\large\bfseries}{\thesubsubsection. \,}{0pt}{\large\bfseries}
\titleformat{\paragraph}{\bfseries}{\theparagraph. \,}{0pt}{\bfseries}

\setcounter{secnumdepth}{4}

% Table des matières :

\renewcommand{\cftsecleader}{\cftdotfill{\cftdotsep}}
\addtolength{\cftsecnumwidth}{10pt}

% Redéfinition des commandes :

\renewcommand*\thesection{\arabic{section}}
\renewcommand{\ker}{\mathrm{Ker}}

% Nouvelles commandes :

\newcommand{\website}{https://agreg.skyost.eu}

\newcommand{\tr}[1]{\mathstrut ^t #1}
\newcommand{\im}{\mathrm{Im}}
\newcommand{\rang}{\operatorname{rang}}
\newcommand{\trace}{\operatorname{trace}}
\newcommand{\id}{\operatorname{id}}
\newcommand{\stab}{\operatorname{Stab}}

\providecommand{\newpar}{\\[\medskipamount]}

\providecommand{\lesson}[3]{%
	\title{#3}%
	\hypersetup{pdftitle={#3}}%
	\setcounter{section}{\numexpr #2 - 1}%
	\section{#3}%
	\fancyhead[R]{\truncate{0.73\textwidth}{#2 : #3}}%
}

\providecommand{\development}[3]{%
	\title{#3}%
	\hypersetup{pdftitle={#3}}%
	\section*{#3}%
	\fancyhead[R]{\truncate{0.73\textwidth}{#3}}%
}

\providecommand{\summary}[1]{%
	\textit{#1}%
	\medskip%
}

\tikzset{notestyleraw/.append style={inner sep=0pt, rounded corners=0pt, align=center}}

%\newcommand{\booklink}[1]{\website/bibliographie\##1}
\newcommand{\citelink}[2]{\hyperlink{cite.\therefsection @#1}{#2}}
\newcommand{\previousreference}{}
\providecommand{\reference}[2][]{%
	\notblank{#1}{\renewcommand{\previousreference}{#1}}{}%
	\todo[noline]{%
		\protect\vspace{16pt}%
		\protect\par%
		\protect\notblank{#1}{\cite{[\previousreference]}\\}{}%
		\protect\citelink{\previousreference}{p. #2}%
	}%
}

\definecolor{devcolor}{HTML}{00695c}
\newcommand{\dev}[1]{%
	\reversemarginpar%
	\todo[noline]{
		\protect\vspace{16pt}%
		\protect\par%
		\bfseries\color{devcolor}\href{\website/developpements/#1}{DEV}
	}%
	\normalmarginpar%
}

% En-têtes :

\pagestyle{fancy}
\fancyhead[L]{\truncate{0.23\textwidth}{\thepage}}
\fancyfoot[C]{\scriptsize \href{\website}{\texttt{agreg.skyost.eu}}}

% Couleurs :

\definecolor{property}{HTML}{fffde7}
\definecolor{proposition}{HTML}{fff8e1}
\definecolor{lemma}{HTML}{fff3e0}
\definecolor{theorem}{HTML}{fce4f2}
\definecolor{corollary}{HTML}{ffebee}
\definecolor{definition}{HTML}{ede7f6}
\definecolor{notation}{HTML}{f3e5f5}
\definecolor{example}{HTML}{e0f7fa}
\definecolor{cexample}{HTML}{efebe9}
\definecolor{application}{HTML}{e0f2f1}
\definecolor{remark}{HTML}{e8f5e9}
\definecolor{proof}{HTML}{e1f5fe}

% Théorèmes :

\theoremstyle{definition}
\newtheorem{theorem}{Théorème}

\newtheorem{property}[theorem]{Propriété}
\newtheorem{proposition}[theorem]{Proposition}
\newtheorem{lemma}[theorem]{Lemme}
\newtheorem{corollary}[theorem]{Corollaire}

\newtheorem{definition}[theorem]{Définition}
\newtheorem{notation}[theorem]{Notation}

\newtheorem{example}[theorem]{Exemple}
\newtheorem{cexample}[theorem]{Contre-exemple}
\newtheorem{application}[theorem]{Application}

\theoremstyle{remark}
\newtheorem{remark}[theorem]{Remarque}

\counterwithin*{theorem}{section}

\newcommand{\applystyletotheorem}[1]{
	\tcolorboxenvironment{#1}{
		enhanced,
		breakable,
		colback=#1!98!white,
		boxrule=0pt,
		boxsep=0pt,
		left=8pt,
		right=8pt,
		top=8pt,
		bottom=8pt,
		sharp corners,
		after=\par,
	}
}

\applystyletotheorem{property}
\applystyletotheorem{proposition}
\applystyletotheorem{lemma}
\applystyletotheorem{theorem}
\applystyletotheorem{corollary}
\applystyletotheorem{definition}
\applystyletotheorem{notation}
\applystyletotheorem{example}
\applystyletotheorem{cexample}
\applystyletotheorem{application}
\applystyletotheorem{remark}
\applystyletotheorem{proof}

% Environnements :

\NewEnviron{whitetabularx}[1]{%
	\renewcommand{\arraystretch}{2.5}
	\colorbox{white}{%
		\begin{tabularx}{\textwidth}{#1}%
			\BODY%
		\end{tabularx}%
	}%
}

% Maths :

\DeclareFontEncoding{FMS}{}{}
\DeclareFontSubstitution{FMS}{futm}{m}{n}
\DeclareFontEncoding{FMX}{}{}
\DeclareFontSubstitution{FMX}{futm}{m}{n}
\DeclareSymbolFont{fouriersymbols}{FMS}{futm}{m}{n}
\DeclareSymbolFont{fourierlargesymbols}{FMX}{futm}{m}{n}
\DeclareMathDelimiter{\VERT}{\mathord}{fouriersymbols}{152}{fourierlargesymbols}{147}



\begin{document}
	%<*content>
	\development{analysis}{integrale-dirichlet}{Intégrale de Dirichlet}

	\summary{Il s'agit ici de calculer l'intégrale de Dirichlet en utilisant les théorèmes classiques d'intégration.}
	
	\begin{lemma}
		\label{integrale-dirichlet-1}
		\[ \forall y, t \in \mathbb{R}^+, \, |e^{-(y-i)t}| \leq 1 \]
	\end{lemma}

	\begin{demonstration}
		Soient $y, t \in \mathbb{R}^+$. On a :
		\[ |e^{-(y-i)t}| = |e^{-yt} e^{it}| = |e^{-yt}| |e^{it}| \]
		Or, $e^{it}$ est un complexe de module $1$ et $yt \geq 0$, donc $e^{-yt} \leq 1$. D'où le résultat.
	\end{demonstration}

	\reference{G-K}{107}

	\begin{theorem}[Intégrale de Dirichlet]
		On pose $\forall x \geq 0$,
		\[ F(x) = \int_0^{+\infty} \frac{\sin(t)}{t} e^{-xt} \, \mathrm{d}t \]
		alors :
		\begin{enumerate}[(i)]
			\item $F$ est bien définie et est continue sur $\mathbb{R}^+$.
			\item $F$ est dérivable sur $\mathbb{R}^+_*$ et $\forall x \in \mathbb{R}^+_*$, $F'(x) = -\frac{1}{1+x^2}$.
			\item $F(0) = \int_0^{+\infty} \frac{\sin(t)}{t} \, \mathrm{d}t = \frac{\pi}{2}$.
		\end{enumerate}
	\end{theorem}

	\reference{G-K}{478}

	\begin{demonstration}
		Posons $\forall x \in \mathbb{R}^+$ et $\forall t \in \mathbb{R}^+_*$, $f(x,t) = \frac{\sin(t)}{t} e^{-xt}$ ainsi que $\forall n \geq 1$, $F_n(x) = \int_0^n \frac{\sin(t)}{t} e^{-xt} \, \mathrm{d}t$. On a :
		\begin{itemize}
			\item $\forall x \geq 0$, $t \mapsto f(x, t)$ est mesurable.
			\item p.p. en $t > 0$, $x \mapsto f(x, t)$ est continue.
			\item $\forall x \geq 0$ et p.p. en $t > 0$, $|f(x,t)| \leq 1$, et $t \mapsto 1$ est intégrable sur $[0,n]$.
		\end{itemize}
		On peut donc appliquer le théorème de continuité sous l'intégrale pour conclure que $F_n$ est continue sur $\mathbb{R}^+$.
		\newpar
		Soient $x \geq 0$ et $q \geq p \geq N \geq 0$. On a :
		\begin{align*}
			|F_q(x) - F_p(x)| &= \left| \int_p^q f(x,t) \, \mathrm{d}t \right| \\
			&= \left| \operatorname{Im} \left( \int_p^q e^{-xt} \frac{e^{it}}{t} \, \mathrm{d}t \right) \right| \\
			&\leq \left| \int_p^q \frac{e^{-(x-i)t}}{t} \, \mathrm{d}t \right| \\
			&= \frac{1}{|x-i|} \left| \int_p^q (x-i) \frac{e^{-(x-i)t}}{t} \, \mathrm{d}t \right| \\
			&\leq \left| \int_p^q (x-i) \frac{e^{-(x-i)t}}{t} \, \mathrm{d}t \right| \\
			&= \left| \int_p^q -(x-i) e^{-(x-i)t} \frac{1}{t} \, \mathrm{d}t \right|
		\end{align*}
		Nous allons réaliser une intégration par parties. Pour cela, posons :
		\begin{itemize}
			\item $u'(t) = -(x-i) e^{-(x-i)t} \implies u(t) = e^{-(x-i)t}$
			\item $v(t) = \frac{1}{t} \implies v'(t) = -\frac{1}{t^2}$
		\end{itemize}
		Ce qui nous donne :
		\begin{align*}
			\left| \int_p^q (x-i) \frac{e^{-(x-i)t}}{t} \, \mathrm{d}t \right| &= \left| \left[ u(t)v(t) \right]_p^q - \int_p^q u(t) v'(t) \, \mathrm{d}t \right| \\
			&= \left| \frac{e^{-(q-i)t}}{q} - \frac{e^{-(p-i)t}}{p} +  \int_p^q \frac{e^{-(x-i)t}}{t^2} \, \mathrm{d}t \right|
		\end{align*}
		On applique maintenant le \cref{integrale-dirichlet-1} :
		\begin{align*}
			\left| \frac{e^{-(q-i)t}}{q} - \frac{e^{-(p-i)t}}{p} +  \int_p^q \frac{e^{-(x-i)t}}{t^2} \, \mathrm{d}t \right| &\leq \frac{1}{p} + \frac{1}{q} + \int_p^q \frac{1}{t^2} \, \mathrm{d}t \\
			&= \frac{1}{p} + \frac{1}{q} - \left[ \frac{1}{t} \right]^q_p \\
			&\leq \frac{2}{N}
		\end{align*}
		D'où :
		\[ |F_q(x) - F_p(x)| \leq \frac{2}{N} \]
		Donc la suite de fonctions continues $(F_n)$ vérifie le critère de Cauchy uniforme, et converge ainsi vers $F$ uniformément. En particulier, $F$ est continue sur $\mathbb{R}^+$.
		\newpar
		Soit $a > 0$. $f$ est dérivable par rapport à $x$ et pour tout $x \in ]a, +\infty|$ et $t \in \mathbb{R}^+$ :
		\[ \left| \frac{\partial f}{\partial x} (x, t) \right| = |-\sin(t) e^{-xt}| \leq e^{-at} \]
		On applique le théorème de dérivation sous l'intégrale, qui donne :
		\[ \forall x \in ]a, +\infty[, F'(x) = \int_0^{+\infty} -\sin(t) e^{-xt} \, \mathrm{d}t \]
		En particulier, c'est vrai sur $\mathbb{R}^+_*$ car la dérivabilité est une propriété locale. Or $\forall A > 0$, on a :
		\begin{align*}
			& \int_0^A e^{-(i+x)t} \, \mathrm{d}t = \frac{1-e^{-(i+x)A}}{i+x} \\
			\implies & \lim_{A \rightarrow +\infty} \int_0^A e^{-(i+x)t} \, \mathrm{d}t = \frac{1}{i+x} = \frac{-i+x}{1 + x^2} \\
			\implies & \operatorname{Im} \left( \lim_{A \rightarrow +\infty} \int_0^A e^{-(i+x)t} \, \mathrm{d}t \right) = \operatorname{Im} \left( \frac{-i+x}{1 + x^2} \right) = -\frac{1}{1 + x^2}
		\end{align*}
		Or,
		\[ \operatorname{Im} \left( \lim_{A \rightarrow +\infty} \int_0^A e^{-(i+x)t} \, \mathrm{d}t \right) = \lim_{A \rightarrow +\infty} \int_0^A \operatorname{Im} \left( e^{-(i+x)t} \right) \, \mathrm{d}t = \int_0^{+\infty} -\sin(t) e^{-xt} \, \mathrm{d}t = F'(x) \]
		En recollant les deux morceaux :
		\[ F'(x) = -\frac{1}{1+x^2} \tag{$*$} \]
		\newpar
		Soient $x, y \in \mathbb{R}^+_*$. En intégrant $(*)$ entre $x$ et $y$, on obtient :
		\[ F(x) - F(y) = \arctan(x) - \arctan(y) \]
		Mais,
		\begin{align*}
			|F(y)| &= \left| \int_0^{+\infty} \frac{\sin(t)}{t} e^{-yt} \, \mathrm{d}t \right| \\
			&\leq \int_0^{+\infty} \left| \frac{\sin(t)}{t} e^{-yt} \right| \, \mathrm{d}t \\
			&\leq \int_0^{+\infty} e^{-yt} \, \mathrm{d}t \\
			&= \frac{1}{y}
			&\longleftrightarrow 0
		\end{align*}
		Donc $\forall x > 0$, $F(x) = \frac{\pi}{2} - \arctan(x)$. En évaluant en $0$, on obtient :
		\[ F(0) = \int_0^{+\infty} \frac{\sin(t)}{t} \, \mathrm{d}t = \frac{\pi}{2} \]
	\end{demonstration}
	%</content>
\end{document}
