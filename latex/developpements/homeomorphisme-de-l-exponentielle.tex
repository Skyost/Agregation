\documentclass[12pt, a4paper]{report}

% LuaLaTeX :

\RequirePackage{iftex}
\RequireLuaTeX

% Packages :

\usepackage[french]{babel}
%\usepackage[utf8]{inputenc}
%\usepackage[T1]{fontenc}
\usepackage[pdfencoding=auto, pdfauthor={Hugo Delaunay}, pdfsubject={Mathématiques}, pdfcreator={agreg.skyost.eu}]{hyperref}
\usepackage{amsmath}
\usepackage{amsthm}
%\usepackage{amssymb}
\usepackage{stmaryrd}
\usepackage{tikz}
\usepackage{tkz-euclide}
\usepackage{fourier-otf}
\usepackage{fontspec}
\usepackage{titlesec}
\usepackage{fancyhdr}
\usepackage{catchfilebetweentags}
\usepackage[french, capitalise, noabbrev]{cleveref}
\usepackage[fit, breakall]{truncate}
\usepackage[top=2.5cm, right=2cm, bottom=2.5cm, left=2cm]{geometry}
\usepackage{enumerate}
\usepackage{tocloft}
\usepackage{microtype}
%\usepackage{mdframed}
%\usepackage{thmtools}
\usepackage{xcolor}
\usepackage{tabularx}
\usepackage{aligned-overset}
\usepackage[subpreambles=true]{standalone}
\usepackage{environ}
\usepackage[normalem]{ulem}
\usepackage{marginnote}
\usepackage{etoolbox}
\usepackage{setspace}
\usepackage[bibstyle=reading, citestyle=draft]{biblatex}
\usepackage{xpatch}
\usepackage[many, breakable]{tcolorbox}
\usepackage[backgroundcolor=white, bordercolor=white, textsize=small]{todonotes}

% Bibliographie :

\newcommand{\overridebibliographypath}[1]{\providecommand{\bibliographypath}{#1}}
\overridebibliographypath{../bibliography.bib}
\addbibresource{\bibliographypath}
\defbibheading{bibliography}[\bibname]{%
	\newpage
	\section*{#1}%
}
\renewbibmacro*{entryhead:full}{\printfield{labeltitle}}
\DeclareFieldFormat{url}{\newline\footnotesize\url{#1}}
\AtEndDocument{\printbibliography}

% Police :

\setmathfont{Erewhon Math}

% Tikz :

\usetikzlibrary{calc}

% Longueurs :

\setlength{\parindent}{0pt}
\setlength{\headheight}{15pt}
\setlength{\fboxsep}{0pt}
\titlespacing*{\chapter}{0pt}{-20pt}{10pt}
\setlength{\marginparwidth}{1.5cm}
\setstretch{1.1}

% Métadonnées :

\author{agreg.skyost.eu}
\date{\today}

% Titres :

\setcounter{secnumdepth}{3}

\renewcommand{\thechapter}{\Roman{chapter}}
\renewcommand{\thesubsection}{\Roman{subsection}}
\renewcommand{\thesubsubsection}{\arabic{subsubsection}}
\renewcommand{\theparagraph}{\alph{paragraph}}

\titleformat{\chapter}{\huge\bfseries}{\thechapter}{20pt}{\huge\bfseries}
\titleformat*{\section}{\LARGE\bfseries}
\titleformat{\subsection}{\Large\bfseries}{\thesubsection \, - \,}{0pt}{\Large\bfseries}
\titleformat{\subsubsection}{\large\bfseries}{\thesubsubsection. \,}{0pt}{\large\bfseries}
\titleformat{\paragraph}{\bfseries}{\theparagraph. \,}{0pt}{\bfseries}

\setcounter{secnumdepth}{4}

% Table des matières :

\renewcommand{\cftsecleader}{\cftdotfill{\cftdotsep}}
\addtolength{\cftsecnumwidth}{10pt}

% Redéfinition des commandes :

\renewcommand*\thesection{\arabic{section}}
\renewcommand{\ker}{\mathrm{Ker}}

% Nouvelles commandes :

\newcommand{\website}{https://agreg.skyost.eu}

\newcommand{\tr}[1]{\mathstrut ^t #1}
\newcommand{\im}{\mathrm{Im}}
\newcommand{\rang}{\operatorname{rang}}
\newcommand{\trace}{\operatorname{trace}}
\newcommand{\id}{\operatorname{id}}
\newcommand{\stab}{\operatorname{Stab}}

\providecommand{\newpar}{\\[\medskipamount]}

\providecommand{\lesson}[3]{%
	\title{#3}%
	\hypersetup{pdftitle={#3}}%
	\setcounter{section}{\numexpr #2 - 1}%
	\section{#3}%
	\fancyhead[R]{\truncate{0.73\textwidth}{#2 : #3}}%
}

\providecommand{\development}[3]{%
	\title{#3}%
	\hypersetup{pdftitle={#3}}%
	\section*{#3}%
	\fancyhead[R]{\truncate{0.73\textwidth}{#3}}%
}

\providecommand{\summary}[1]{%
	\textit{#1}%
	\medskip%
}

\tikzset{notestyleraw/.append style={inner sep=0pt, rounded corners=0pt, align=center}}

%\newcommand{\booklink}[1]{\website/bibliographie\##1}
\newcommand{\citelink}[2]{\hyperlink{cite.\therefsection @#1}{#2}}
\newcommand{\previousreference}{}
\providecommand{\reference}[2][]{%
	\notblank{#1}{\renewcommand{\previousreference}{#1}}{}%
	\todo[noline]{%
		\protect\vspace{16pt}%
		\protect\par%
		\protect\notblank{#1}{\cite{[\previousreference]}\\}{}%
		\protect\citelink{\previousreference}{p. #2}%
	}%
}

\definecolor{devcolor}{HTML}{00695c}
\newcommand{\dev}[1]{%
	\reversemarginpar%
	\todo[noline]{
		\protect\vspace{16pt}%
		\protect\par%
		\bfseries\color{devcolor}\href{\website/developpements/#1}{DEV}
	}%
	\normalmarginpar%
}

% En-têtes :

\pagestyle{fancy}
\fancyhead[L]{\truncate{0.23\textwidth}{\thepage}}
\fancyfoot[C]{\scriptsize \href{\website}{\texttt{agreg.skyost.eu}}}

% Couleurs :

\definecolor{property}{HTML}{fffde7}
\definecolor{proposition}{HTML}{fff8e1}
\definecolor{lemma}{HTML}{fff3e0}
\definecolor{theorem}{HTML}{fce4f2}
\definecolor{corollary}{HTML}{ffebee}
\definecolor{definition}{HTML}{ede7f6}
\definecolor{notation}{HTML}{f3e5f5}
\definecolor{example}{HTML}{e0f7fa}
\definecolor{cexample}{HTML}{efebe9}
\definecolor{application}{HTML}{e0f2f1}
\definecolor{remark}{HTML}{e8f5e9}
\definecolor{proof}{HTML}{e1f5fe}

% Théorèmes :

\theoremstyle{definition}
\newtheorem{theorem}{Théorème}

\newtheorem{property}[theorem]{Propriété}
\newtheorem{proposition}[theorem]{Proposition}
\newtheorem{lemma}[theorem]{Lemme}
\newtheorem{corollary}[theorem]{Corollaire}

\newtheorem{definition}[theorem]{Définition}
\newtheorem{notation}[theorem]{Notation}

\newtheorem{example}[theorem]{Exemple}
\newtheorem{cexample}[theorem]{Contre-exemple}
\newtheorem{application}[theorem]{Application}

\theoremstyle{remark}
\newtheorem{remark}[theorem]{Remarque}

\counterwithin*{theorem}{section}

\newcommand{\applystyletotheorem}[1]{
	\tcolorboxenvironment{#1}{
		enhanced,
		breakable,
		colback=#1!98!white,
		boxrule=0pt,
		boxsep=0pt,
		left=8pt,
		right=8pt,
		top=8pt,
		bottom=8pt,
		sharp corners,
		after=\par,
	}
}

\applystyletotheorem{property}
\applystyletotheorem{proposition}
\applystyletotheorem{lemma}
\applystyletotheorem{theorem}
\applystyletotheorem{corollary}
\applystyletotheorem{definition}
\applystyletotheorem{notation}
\applystyletotheorem{example}
\applystyletotheorem{cexample}
\applystyletotheorem{application}
\applystyletotheorem{remark}
\applystyletotheorem{proof}

% Environnements :

\NewEnviron{whitetabularx}[1]{%
	\renewcommand{\arraystretch}{2.5}
	\colorbox{white}{%
		\begin{tabularx}{\textwidth}{#1}%
			\BODY%
		\end{tabularx}%
	}%
}

% Maths :

\DeclareFontEncoding{FMS}{}{}
\DeclareFontSubstitution{FMS}{futm}{m}{n}
\DeclareFontEncoding{FMX}{}{}
\DeclareFontSubstitution{FMX}{futm}{m}{n}
\DeclareSymbolFont{fouriersymbols}{FMS}{futm}{m}{n}
\DeclareSymbolFont{fourierlargesymbols}{FMX}{futm}{m}{n}
\DeclareMathDelimiter{\VERT}{\mathord}{fouriersymbols}{152}{fourierlargesymbols}{147}



\begin{document}
	%<*content>
	\development{algebra}{homeomorphisme-de-l-exponentielle}{\texorpdfstring{$\exp : \mathcal{S}_n(\mathbb{R}) \rightarrow \mathcal{S}^{++}_n(\mathbb{R})$}{exp : Sn(R) -> Sn(R)++} est un homéomorphisme}

	\summary{Dans ce développement, on démontre que l'exponentielle de matrices induit un homéomorphisme de $\mathcal{S}_n(\mathbb{R})$ sur $\mathcal{S}^{++}_n(\mathbb{R})$.}

	\begin{lemma}
		\label{homeomorphisme-de-l-exponentielle-1}
		$\mathcal{S}_n(\mathbb{R})$ est un fermé de $\mathcal{M}_n(\mathbb{R})$.
	\end{lemma}

	\begin{demonstration}
		Il suffit d'écrire
		\[ \mathcal{S}_n(\mathbb{R}) = \{ M \in \mathcal{M}_n(\mathbb{R}) \mid \tr M = M \} = f^{-1}\{ 0 \} \]
		où $f : M \mapsto \tr M - M$ est continue, donc $\mathcal{S}_n(\mathbb{R})$ est fermé en tant qu'image réciproque d'un fermé par une application continue.
	\end{demonstration}

	\begin{lemma}
		\label{homeomorphisme-de-l-exponentielle-2}
		Une suite bornée d'un espace métrique qui admet une seule valeur d'adhérence converge vers cette valeur d'adhérence.
	\end{lemma}

	\begin{demonstration}
		Soit $(x_n)$ une suite bornée d'un espace métrique $(E, d)$ qui n'admet qu'une seule valeur d'adhérence $\ell \in E$. On suppose par l'absurde que $(x_n)$ ne converge pas vers $\ell$ :
		\[ \exists \epsilon > 0 \text{ tel que } \forall N \in \mathbb{N}, \, \exists n \geq N \text{ tel que } d(x_n, \ell) > \epsilon \tag{$*$} \]
		On va construire une sous-suite qui converge vers une valeur d'adhérence différente de $\ell$.
		\newpar
		Par $(*)$ appliqué à $N = 0$, $\exists n_0 \geq 0$ tel que $d(x_{n_0}, \ell) > \epsilon$. On définit donc $\varphi(0) = n_0$.
		\newpar
		Supposons construite $\varphi(i)$ jusqu'à un rang $k$ telle que $\forall i <\leq> k$, $\varphi(i+1) > \varphi(i)$ (lorsque cela à un sens) et $d(x_{\varphi(i)}, \ell) > \epsilon$. Il suffit alors d'appliquer $(*)$ à $N = \varphi(n) + 1$ pour obtenir un $n_k \geq \varphi(n) + 1 > \varphi(n)$ tel que $d(x_{n_k}, \ell) > \epsilon$ ; on définit alors $\varphi(k+1) = n_k$.
		\newpar
		Nous venons donc de construire par récurrence une application $\varphi : \mathbb{N} \rightarrow \mathbb{N}$ strictement croissante et telle que $\forall n \in \mathbb{N}$, $d(x_{\varphi(n)}, \ell) > \epsilon$. La suite $(x_{\varphi(n)})$ est bornée (par hypothèse) : elle est contenue dans un compact et admet une valeur d'adhérence $\ell'$ (par le théorème de Bolzano-Weierstrass). Soit donc $\phi : \mathbb{N} \rightarrow \mathbb{N}$ strictement croissante telle que $(x_{(\varphi \circ \psi)(n)})$ converge vers $\ell'$.
		\newpar
		On a $\forall n \in \mathbb{N}$, $d(x_{(\varphi \circ \psi)(n)}, \ell) > \epsilon$, qui donne $d(\ell', \ell) \geq \epsilon$ après un passage à la limite. Donc $\ell \neq \ell'$.	Et $\ell'$ est clairement valeur d'adhérence de $(x_n)$ : absurde.
	\end{demonstration}

	\reference[I-P]{182}

	\begin{lemma}
		\label{homeomorphisme-de-l-exponentielle-3}
		Soit $S \in \mathcal{S}_n(\mathbb{R})$. Alors,
		\[ \VERT S \VERT_2 = \rho(S) \]
		où $\rho$ est l'application qui a une matrice y associe son rayon spectral.
	\end{lemma}

	\begin{demonstration}
		D'après le théorème spectral, il existe $(e_1, \dots, e_n)$ une base orthonormée de $\mathbb{R}^n$ formée de vecteurs propres de $S$ associés aux valeurs propres $\lambda_1, \dots, \lambda_n$ de $S$, qui sont réelles car $S$ est symétrique. Soit $x \in \mathbb{R}^n$ dont on note $(x_1, \dots, x_n)$ ses coordonnées dans cette base. On a
		\[ \Vert Sx \Vert_2^2 = \left \Vert \sum_{i=1}^{n} \lambda_i x_i e_i \right \Vert_2^2 = \sum_{i=1}^n \lambda_i^2 x_i^2 \leq \rho(S)^2 \Vert x \Vert_2^2 \]
		D'où $\VERT S \VERT_2 \leq \rho(S)$. Pour obtenir l'inégalité inverse, il suffit de considérer $\lambda \in \mathbb{R}$ une valeur propre de $S$ telle que $|\lambda| = \rho(S)$ et $x \in \mathbb{R}^n$ un vecteur propre associé à $\lambda$. On a alors
		\[ \Vert Sx \Vert_2 = |\lambda| \Vert x \Vert_2 \]
		et on a bien $\rho(S) \leq \VERT S \VERT_2$.
	\end{demonstration}

	\begin{theorem}
		L'application $\exp : \mathcal{S}_n(\mathbb{R}) \rightarrow \mathcal{S}^{++}_n(\mathbb{R})$ est un homéomorphisme.
	\end{theorem}

	\begin{demonstration}
		Montrer qu'une application est un homéomorphisme se fait en $4$ étapes : on montre qu'elle est continue, injective, surjective, et que la réciproque est elle aussi continue.
		\begin{itemize}
			\item \uline{L'application est bien définie et continue :} Soit $S \in \mathcal{S}_n(\mathbb{R})$. D'après le théorème spectral,
			\[ \exists P \in \mathcal{O}_n(\mathbb{R}) \text{ telle que } S = P \underbrace{\operatorname{Diag}(\lambda_1, \dots, \lambda_n)}_{= D}P^{-1} \]
			où $\lambda_1, \dots, \lambda_n$ désignent les valeurs propres de $S$. On a donc
			\begin{align*}
				\exp(S) &= P^{-1} \exp(D) P \\
				&= P^{-1} \operatorname{Diag}(e^{\lambda_1}, \dots, \lambda_n) P
			\end{align*}
			Or, $P^{-1} = \tr P$, donc $\tr \exp(S) = \exp(S)$ et $\exp(S) \in \mathcal{S}_n(\mathbb{R})$. De plus, $\forall x \in \mathbb{R}^n$,
			\[ \tr x S x = \tr (Px) D (Px) > 0 \]
			car $D \in \mathcal{S}^{++}_n(\mathbb{R})$. Donc $S \in \mathcal{S}^{++}_n(\mathbb{R})$.
			\item \uline{L'application est surjective :} Soit $S \in \mathcal{S}^{++}_n(\mathbb{R})$. On peut écrire
			\[ S = P \operatorname{Diag}(\mu_1, \dots, \mu_n) P^{-1} \]
			Il suffit alors de poser $U = P^{-1} \operatorname{Diag}(\ln(\mu_1), \dots, \ln(\mu_n)) P \in \mathcal{S}_n(\mathbb{R})$ pour avoir $\exp(U) = S$ ; d'où la surjectivité.
			\item \uline{L'application est injective :} Soient $S, S' \in \mathcal{S}_n(\mathbb{R})$ telles que $\exp(S) = \exp(S')$. Montrons que $S = S'$. Comme avant, $\exists P, P' \in \mathcal{O}_n(\mathbb{R})$ telles que
			\[ S = P \operatorname{Diag}(\lambda_1, \dots, \lambda_n) P^{-1} \text{ et } S' = P' \operatorname{Diag}(\lambda'_1, \dots, \lambda'_n) P'^{-1} \]
			Soit $L \in \mathbb{R}[X]$ tel que $\forall i \in \llbracket 1, n \rrbracket$, $L(e^{\lambda_i}) = \lambda_i$ et $L(e^{\lambda'_i}) = \lambda'_i$ (les polynômes d'interpolation de Lagrange conviennent parfaitement et sont bien définis dans le cas présent car $e^{\lambda_i} = e^{\lambda_j} \implies \lambda_i = \lambda_j$ par injectivité de l'exponentielle). D'où
			\begin{align*}
				L(\exp(S)) &= L(P \operatorname{Diag}(\lambda_1, \dots, \lambda_n) P^{-1}) \\
				&= P L(\exp(\operatorname{Diag}(\lambda_1, \dots, \lambda_n))) P^{-1} \\
				&= P \operatorname{Diag}(\lambda_1, \dots, \lambda_n) P^{-1} \\
				&= S
			\end{align*}
			et de même, $L(\exp(S')) = S'$. D'où $S = S'$.
			\item \uline{L'application inverse est continue :} Soit $(A_k)$ une suite de $\mathcal{S}^{++}_n(\mathbb{R})$ qui converge vers $A \in \mathcal{S}^{++}_n(\mathbb{R})$. Il s'agit de montrer que la suite $(B_k)$ de terme général $B_k = \exp^{-1}(A_k)$ converge vers $B = \exp^{-1}(A)$. Supposons tout d'abord $(B_k)$ non bornée. Comme sur $\mathcal{S}_n(\mathbb{R})$, $\VERT . \VERT_2 = \rho(.)$ (par le \cref{homeomorphisme-de-l-exponentielle-3}), il existe $\varphi : \mathbb{N} \rightarrow \mathbb{N}$ strictement croissante telle que $\rho(B_\varphi(k)) \longrightarrow +\infty$. On peut donc extraire une suite de valeurs propres $(\lambda_k)$ telle que $|\lambda_k| \longrightarrow +\infty$. Encore une fois, quitte à extraire, on peut supposer $\lambda_k \longrightarrow +\infty$ ou $\lambda_k \longrightarrow -\infty$.
			\begin{itemize}
				\item Si $\lambda_k \longrightarrow +\infty$, $e^{\lambda_k} \longrightarrow +\infty$. Mais $\forall k \in \mathbb{N}$, $e^{\lambda_k}$ est valeur propre de $A_k$, donc $\rho(A_k) \longrightarrow +\infty$ : absurde car $(A_k)$ converge.
				\item Si $\lambda_k \longrightarrow -\infty$, $e^{-\lambda_k} \longrightarrow +\infty$. Mais $\forall k \in \mathbb{N}$, $e^{-\lambda_k}$ est valeur propre de $A_k^{-1}$, donc $\rho(A_k^{-1}) \longrightarrow +\infty$ : absurde car $(A_k^{-1})$ converge par continuité de $M \mapsto M^{-1}$.
			\end{itemize}
			Donc la suite $(B_k)$ est bornée. Par le théorème de Bolzano-Weierstrass, $(B_k)$ admet une valeur d'adhérence $\widetilde{B}$. Comme $\mathcal{S}_n(\mathbb{R})$ est fermé (c'est le \cref{homeomorphisme-de-l-exponentielle-1}), $\widetilde{B} \in \mathcal{S}_n(\mathbb{R})$.
			\newpar
			Soit $\varphi : \mathbb{N} \rightarrow \mathbb{N}$ strictement croissante telle que $B_{\varphi(k)} \longrightarrow \widetilde{B}$. Alors,
			\[ \exp(B) = A \longleftarrow A_{\varphi(k)} = \exp(B_{\varphi(k)}) \longrightarrow \exp(\widetilde{B}) \]
			ie. $\exp(B) = \exp(\widetilde{B})$ ; donc $B = \widetilde{B}$ par injectivité de $\exp$. Donc par le \cref{homeomorphisme-de-l-exponentielle-2}, $B_k \longrightarrow B$.
		\end{itemize}
	\end{demonstration}
	%</content>
\end{document}
