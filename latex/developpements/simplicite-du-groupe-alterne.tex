\documentclass[12pt, a4paper]{report}

% LuaLaTeX :

\RequirePackage{iftex}
\RequireLuaTeX

% Packages :

\usepackage[french]{babel}
%\usepackage[utf8]{inputenc}
%\usepackage[T1]{fontenc}
\usepackage[pdfencoding=auto, pdfauthor={Hugo Delaunay}, pdfsubject={Mathématiques}, pdfcreator={agreg.skyost.eu}]{hyperref}
\usepackage{amsmath}
\usepackage{amsthm}
%\usepackage{amssymb}
\usepackage{stmaryrd}
\usepackage{tikz}
\usepackage{tkz-euclide}
\usepackage{fourier-otf}
\usepackage{fontspec}
\usepackage{titlesec}
\usepackage{fancyhdr}
\usepackage{catchfilebetweentags}
\usepackage[french, capitalise, noabbrev]{cleveref}
\usepackage[fit, breakall]{truncate}
\usepackage[top=2.5cm, right=2cm, bottom=2.5cm, left=2cm]{geometry}
\usepackage{enumerate}
\usepackage{tocloft}
\usepackage{microtype}
%\usepackage{mdframed}
%\usepackage{thmtools}
\usepackage{xcolor}
\usepackage{tabularx}
\usepackage{aligned-overset}
\usepackage[subpreambles=true]{standalone}
\usepackage{environ}
\usepackage[normalem]{ulem}
\usepackage{marginnote}
\usepackage{etoolbox}
\usepackage{setspace}
\usepackage[bibstyle=reading, citestyle=draft]{biblatex}
\usepackage{xpatch}
\usepackage[many, breakable]{tcolorbox}
\usepackage[backgroundcolor=white, bordercolor=white, textsize=small]{todonotes}

% Bibliographie :

\newcommand{\overridebibliographypath}[1]{\providecommand{\bibliographypath}{#1}}
\overridebibliographypath{../bibliography.bib}
\addbibresource{\bibliographypath}
\defbibheading{bibliography}[\bibname]{%
	\newpage
	\section*{#1}%
}
\renewbibmacro*{entryhead:full}{\printfield{labeltitle}}
\DeclareFieldFormat{url}{\newline\footnotesize\url{#1}}
\AtEndDocument{\printbibliography}

% Police :

\setmathfont{Erewhon Math}

% Tikz :

\usetikzlibrary{calc}

% Longueurs :

\setlength{\parindent}{0pt}
\setlength{\headheight}{15pt}
\setlength{\fboxsep}{0pt}
\titlespacing*{\chapter}{0pt}{-20pt}{10pt}
\setlength{\marginparwidth}{1.5cm}
\setstretch{1.1}

% Métadonnées :

\author{agreg.skyost.eu}
\date{\today}

% Titres :

\setcounter{secnumdepth}{3}

\renewcommand{\thechapter}{\Roman{chapter}}
\renewcommand{\thesubsection}{\Roman{subsection}}
\renewcommand{\thesubsubsection}{\arabic{subsubsection}}
\renewcommand{\theparagraph}{\alph{paragraph}}

\titleformat{\chapter}{\huge\bfseries}{\thechapter}{20pt}{\huge\bfseries}
\titleformat*{\section}{\LARGE\bfseries}
\titleformat{\subsection}{\Large\bfseries}{\thesubsection \, - \,}{0pt}{\Large\bfseries}
\titleformat{\subsubsection}{\large\bfseries}{\thesubsubsection. \,}{0pt}{\large\bfseries}
\titleformat{\paragraph}{\bfseries}{\theparagraph. \,}{0pt}{\bfseries}

\setcounter{secnumdepth}{4}

% Table des matières :

\renewcommand{\cftsecleader}{\cftdotfill{\cftdotsep}}
\addtolength{\cftsecnumwidth}{10pt}

% Redéfinition des commandes :

\renewcommand*\thesection{\arabic{section}}
\renewcommand{\ker}{\mathrm{Ker}}

% Nouvelles commandes :

\newcommand{\website}{https://agreg.skyost.eu}

\newcommand{\tr}[1]{\mathstrut ^t #1}
\newcommand{\im}{\mathrm{Im}}
\newcommand{\rang}{\operatorname{rang}}
\newcommand{\trace}{\operatorname{trace}}
\newcommand{\id}{\operatorname{id}}
\newcommand{\stab}{\operatorname{Stab}}

\providecommand{\newpar}{\\[\medskipamount]}

\providecommand{\lesson}[3]{%
	\title{#3}%
	\hypersetup{pdftitle={#3}}%
	\setcounter{section}{\numexpr #2 - 1}%
	\section{#3}%
	\fancyhead[R]{\truncate{0.73\textwidth}{#2 : #3}}%
}

\providecommand{\development}[3]{%
	\title{#3}%
	\hypersetup{pdftitle={#3}}%
	\section*{#3}%
	\fancyhead[R]{\truncate{0.73\textwidth}{#3}}%
}

\providecommand{\summary}[1]{%
	\textit{#1}%
	\medskip%
}

\tikzset{notestyleraw/.append style={inner sep=0pt, rounded corners=0pt, align=center}}

%\newcommand{\booklink}[1]{\website/bibliographie\##1}
\newcommand{\citelink}[2]{\hyperlink{cite.\therefsection @#1}{#2}}
\newcommand{\previousreference}{}
\providecommand{\reference}[2][]{%
	\notblank{#1}{\renewcommand{\previousreference}{#1}}{}%
	\todo[noline]{%
		\protect\vspace{16pt}%
		\protect\par%
		\protect\notblank{#1}{\cite{[\previousreference]}\\}{}%
		\protect\citelink{\previousreference}{p. #2}%
	}%
}

\definecolor{devcolor}{HTML}{00695c}
\newcommand{\dev}[1]{%
	\reversemarginpar%
	\todo[noline]{
		\protect\vspace{16pt}%
		\protect\par%
		\bfseries\color{devcolor}\href{\website/developpements/#1}{DEV}
	}%
	\normalmarginpar%
}

% En-têtes :

\pagestyle{fancy}
\fancyhead[L]{\truncate{0.23\textwidth}{\thepage}}
\fancyfoot[C]{\scriptsize \href{\website}{\texttt{agreg.skyost.eu}}}

% Couleurs :

\definecolor{property}{HTML}{fffde7}
\definecolor{proposition}{HTML}{fff8e1}
\definecolor{lemma}{HTML}{fff3e0}
\definecolor{theorem}{HTML}{fce4f2}
\definecolor{corollary}{HTML}{ffebee}
\definecolor{definition}{HTML}{ede7f6}
\definecolor{notation}{HTML}{f3e5f5}
\definecolor{example}{HTML}{e0f7fa}
\definecolor{cexample}{HTML}{efebe9}
\definecolor{application}{HTML}{e0f2f1}
\definecolor{remark}{HTML}{e8f5e9}
\definecolor{proof}{HTML}{e1f5fe}

% Théorèmes :

\theoremstyle{definition}
\newtheorem{theorem}{Théorème}

\newtheorem{property}[theorem]{Propriété}
\newtheorem{proposition}[theorem]{Proposition}
\newtheorem{lemma}[theorem]{Lemme}
\newtheorem{corollary}[theorem]{Corollaire}

\newtheorem{definition}[theorem]{Définition}
\newtheorem{notation}[theorem]{Notation}

\newtheorem{example}[theorem]{Exemple}
\newtheorem{cexample}[theorem]{Contre-exemple}
\newtheorem{application}[theorem]{Application}

\theoremstyle{remark}
\newtheorem{remark}[theorem]{Remarque}

\counterwithin*{theorem}{section}

\newcommand{\applystyletotheorem}[1]{
	\tcolorboxenvironment{#1}{
		enhanced,
		breakable,
		colback=#1!98!white,
		boxrule=0pt,
		boxsep=0pt,
		left=8pt,
		right=8pt,
		top=8pt,
		bottom=8pt,
		sharp corners,
		after=\par,
	}
}

\applystyletotheorem{property}
\applystyletotheorem{proposition}
\applystyletotheorem{lemma}
\applystyletotheorem{theorem}
\applystyletotheorem{corollary}
\applystyletotheorem{definition}
\applystyletotheorem{notation}
\applystyletotheorem{example}
\applystyletotheorem{cexample}
\applystyletotheorem{application}
\applystyletotheorem{remark}
\applystyletotheorem{proof}

% Environnements :

\NewEnviron{whitetabularx}[1]{%
	\renewcommand{\arraystretch}{2.5}
	\colorbox{white}{%
		\begin{tabularx}{\textwidth}{#1}%
			\BODY%
		\end{tabularx}%
	}%
}

% Maths :

\DeclareFontEncoding{FMS}{}{}
\DeclareFontSubstitution{FMS}{futm}{m}{n}
\DeclareFontEncoding{FMX}{}{}
\DeclareFontSubstitution{FMX}{futm}{m}{n}
\DeclareSymbolFont{fouriersymbols}{FMS}{futm}{m}{n}
\DeclareSymbolFont{fourierlargesymbols}{FMX}{futm}{m}{n}
\DeclareMathDelimiter{\VERT}{\mathord}{fouriersymbols}{152}{fourierlargesymbols}{147}



\begin{document}
	%<*content>
	\development{algebra}{simplicite-du-groupe-alterne}{Simplicité de \texorpdfstring{$A_n$}{An} pour \texorpdfstring{$n \geq 5$}{n ≥ 5}}

	\summary{On montre que $A_n$ est simple pour $n \geq 5$ en montrant dans un premier temps le cas $n = 5$, puis en s'y ramenant.}

	\begin{lemma}
		\label{simplicite-du-groupe-alterne-1}
		Les $3$-cycles sont conjugués dans $A_n$ pour $n \geq 5$.
	\end{lemma}

	\begin{demonstration}
		Soient $\alpha = \begin{pmatrix} a_1 & a_2 & a_3 \end{pmatrix}$ et $\beta = \begin{pmatrix} b_1 & b_2 & b_3 \end{pmatrix}$ deux $3$-cycles. Soit $\sigma \in S_n$ telle que pour tout $i$, $\sigma(a_i) = b_i$. Alors on a $\sigma \alpha \sigma^{-1} = \begin{pmatrix} \sigma(a_1) & \sigma(a_2) & \sigma(a_3) \end{pmatrix} = \beta$. On a les deux cas suivants :
		\begin{itemize}
			\item Si $\sigma \in A_n$, alors $\alpha$ et $\beta$ sont bien conjuguées dans $A_n$.
			\item Si $\sigma \notin A_n$, il existe $\tau \in S_n$ une transposition dont le support est disjoint de celui de $\alpha$ (c'est vrai car $n \geq 5$). Notons $\gamma = \sigma \tau$ et remarquons que $\gamma \in A_n$. Dans ce cas, $\gamma \alpha \gamma^{-1} = \begin{pmatrix} \gamma(a_1) & \gamma(a_2) & \gamma(a_3) \end{pmatrix} = \begin{pmatrix} \sigma(a_1) & \sigma(a_2) & \sigma(a_3) \end{pmatrix} = \beta$. Donc $\alpha$ et $\beta$ sont bien conjuguées dans $A_n$.
		\end{itemize}
	\end{demonstration}

	\begin{lemma}
		\label{simplicite-du-groupe-alterne-2}
		$A_n$ est engendré par les $3$-cycles.
	\end{lemma}

	\begin{demonstration}
		Tout élément de $A_n$ est produit d'un nombre pair de transpositions ; il suffit donc de vérifier que tout produit de deux transpositions peut s'écrire comme produit de $3$-cycles. Soient $\tau_1$ et $\tau_2$ deux transpositions :
		\begin{itemize}
			\item Si $\tau_1 = \tau_2$, alors $\tau_1 \tau_2 = \operatorname{Id}$, d'où le résultat.
			\item Si leurs supports ont exactement un point commun, on écrit $\tau_1 = \begin{pmatrix} a & b \end{pmatrix}$ et $\tau_2 = \begin{pmatrix} b & c \end{pmatrix}$ et donc on a $\tau_1 \tau_2 =  \begin{pmatrix} a & b & c \end{pmatrix}$.
			\item Si leurs supports sont disjoints, on écrit $\tau_1 = \begin{pmatrix} a & b \end{pmatrix}$ et $\tau_2 = \begin{pmatrix} c & d \end{pmatrix}$ et donc on a $\tau_1 \tau_2 =  \begin{pmatrix} a & b & c \end{pmatrix}\begin{pmatrix} b & c & d \end{pmatrix}$.
		\end{itemize}
	\end{demonstration}

	\reference{PER}{28}

	\begin{lemma}
		\label{simplicite-du-groupe-alterne-3}
		$A_5$ est simple.
	\end{lemma}

	\begin{demonstration}
		Commençons par décrire les types possibles des permutations de $A_5$ (le ``type'' d'une permutation désigne
    les cardinaux des supports des cycles apparaissant dans sa décomposition en cycles disjoints).
		\newpar
		\begin{whitetabularx}{|X|X|}
			\hline
			\textbf{Type de permutation} & \textbf{Nombre de permutations} \\
			\hline
			$[1]$ & $1$ \\
			\hline
			$[3]$ & $\frac{5 \times 4 \times 3}{3} = 20$ \\
			\hline
			$[5]$ & $\frac{5 \times 4 \times 3 \times 2 \times 1}{5} = 24$ \\
			\hline
			$[2,2]$ & $\frac{1}{2} \frac{5 \times 4 \times 3 \times 2}{4} = 15$ \\
			\hline
		\end{whitetabularx}
		\newpar
		Montrons que les permutations de type $[2,2]$ sont conjuguées dans $A_5$. Soient $\alpha = \begin{pmatrix} a_1 & b_1 \end{pmatrix} \begin{pmatrix} c_1 & d_1 \end{pmatrix} \begin{pmatrix} e_1 \end{pmatrix}$ et $\beta = \begin{pmatrix} a_2 & b_2 \end{pmatrix} \begin{pmatrix} c_2 & d_2 \end{pmatrix} \begin{pmatrix} e_2 \end{pmatrix}$ deux permutations de type $[2,2]$. Il suffit de prendre $\sigma \in A_5$ telle que $\sigma(a_1) = a_2$, $\sigma(b_1) = b_2$ et $\sigma(e_1) = e_2$ pour avoir $\sigma \alpha \sigma^{-1} = \beta$.
		\newpar
		Soit $H \lhd A_5$ tel que $H \neq \{ \operatorname{Id} \}$. Montrons que $H = A_5$.
		\begin{itemize}
			\item Si $H$ contient une permutation de type $[2,2]$, alors par le point 1., il les contient toutes.
			\item Si $H$ contient une permutation de type $[3]$, alors par le \cref{simplicite-du-groupe-alterne-1}, il les contient toutes.
			\item Si $H$ contient une permutation de type $[5]$ (qui est donc d'ordre $5$), alors $H$ contient le $5$-Sylow engendré par cet élément. Or, on sait par les théorèmes de Sylow que les sous-groupes de Sylow sont conjugués entre-eux. Donc $H$ contient tous les $5$-Sylow et donc contient tous les éléments d'ordre $5$.
		\end{itemize}
		Or, $H$ ne peut pas vérifier qu'un seul des points précédents en vertu du théorème de Lagrange, car ni $16 = 15 + 1$, ni $21 = 20 + 1$, ni $25 = 24 + 1$ ne divisent $|A_5| = 60$. Donc $H$ vérifie au moins deux des points précédents, et ainsi $|H| \geq 1 + 15 + 20 = 36$. Donc $|H|=60$ et $H = A_5$.
	\end{demonstration}

	\begin{theorem}
		$A_n$ est simple pour $n \geq 5$.
	\end{theorem}

	\begin{demonstration}
		Soit $N \lhd A_n$ tel que $N \neq \{ \operatorname{Id} \}$. L'idée générale de la démonstration et de se ramener au cas $n = 5$ à l'aide d'une permutation bien spécifique.
		\newpar
		Soit $\sigma \in N \setminus \{ \operatorname{Id} \}$, il existe donc $a \in \llbracket 1, n \rrbracket$ tel que $\sigma(a) = b \neq a$. Soit $c \in \llbracket 1, n \rrbracket$ différent de $a$, $b$ et $\sigma(b)$. On pose $\tau = \begin{pmatrix} a & c & b \end{pmatrix} \in A_n$ (on a $\tau^{-1} = \begin{pmatrix} a & b & c \end{pmatrix})$. Soit $\rho = \tau \sigma \tau^{-1} \sigma^{-1}$. Par calcul :
		\[ \rho = \begin{pmatrix} a & c & b \end{pmatrix} \sigma \begin{pmatrix} a & b & c \end{pmatrix} \sigma^{-1} = \begin{pmatrix} a & c & b \end{pmatrix} \begin{pmatrix} \sigma(a) & \sigma(b) & \sigma(c) \end{pmatrix} \]
		Notons bien que $\rho \neq \operatorname{Id}$ (en tant que produit de $3$-cycles, car $\sigma(b) \neq c$). Or, $\tau \sigma \tau^{-1} \in N$ car $N$ est distingué et $\sigma^{-1}$ aussi car $N$ est un groupe, donc $\rho \in N$.
		\newpar
		Notons $\mathcal{F} = \{ a, b, c, \sigma(a), \sigma(b), \sigma(c) \}$. Comme $\sigma(a) = b$, $|\mathcal{F}| \leq 5$. Quitte à rajouter, au besoin, des éléments à $\mathcal{F}$, on peut supposer que $|\mathcal{F}| = 5$. On pose
		\[ A(\mathcal{F}) = \{ \alpha \in A_n \mid \forall i \in \llbracket 1, n \rrbracket \setminus \mathcal{F}, \, \alpha(i) = i \} \]
		le sous-groupe de $A_n$ contenant les éléments qui laissent fixes $\llbracket 1, n \rrbracket \setminus \mathcal{F}$. Si on pose $\mathcal{F} = \{ a_1, a_2, a_3, a_4, a_5 \}$, on a une bijection entre $\mathcal{F}$ et $\llbracket 1, 5 \rrbracket$ :
		\begin{align*}
			\mathcal{F} &\rightarrow \llbracket 1, 5 \rrbracket \\
			a_i &\mapsto i
		\end{align*}
		Donc $A(\mathcal{F})$ et $A_5$ sont deux groupes isomorphes (en effet, une permutation n'agissant que sur $\mathcal{F}$ peut s'identifier à une permutation n'agissant que sur $\llbracket 1, 5 \rrbracket$). De plus, par le \cref{simplicite-du-groupe-alterne-3}, comme $A_5$ est simple, $A(\mathcal{F})$ l'est aussi.
		\newpar
		Soit $N_0 = N \, \cap \, A(\mathcal{F})$. $N_0 \lhd A(\mathcal{F})$, en effet, soient $\alpha \in N_0$ et $\beta \in A(\mathcal{F})$ :
		\begin{itemize}
			\item $\beta \alpha \beta^{-1} \in A(\mathcal{F})$ car $A(\mathcal{F})$ est un groupe.
			\item $\beta \alpha \beta^{-1} \in N$ car $N \lhd A_5$.
		\end{itemize}
		En particulier, $N_0$ est distingué dans $A(\mathcal{F})$ qui est simple. De plus, $\rho \in N_0$ (car $\operatorname{Supp}(\rho) \subset \mathcal{F}$ et $\epsilon(\rho) = (-1)^{6} = 1$ donc $\rho \in A(\mathcal{F})$ et par 1., $\rho \in N$). Donc $N_0 \neq \{ \operatorname{Id} \}$, et ainsi $N_0 = A(\mathcal{F})$. On en déduit :
		\[ A(\mathcal{F}) = N \, \cap \, A(\mathcal{F}) \tag{$*$} \]
		Finalement, $\tau$ est un $3$-cycle qui n'agit que sur $\mathcal{F}$, donc $\tau \in A(\mathcal{F})$ et par $(*)$, $\tau \in N$. Or, $\tau$ est un $3$-cycle et les $3$-cycles sont conjugués dans $A_n$ (par le \cref{simplicite-du-groupe-alterne-3}) donc $N$ contient tous les $3$-cycles. Et comme ceux-ci engendrent $A_n$ (par le \cref{simplicite-du-groupe-alterne-2}), on a $N = A_n$.
	\end{demonstration}
	%</content>
\end{document}
