\documentclass[12pt, a4paper]{report}

% LuaLaTeX :

\RequirePackage{iftex}
\RequireLuaTeX

% Packages :

\usepackage[french]{babel}
%\usepackage[utf8]{inputenc}
%\usepackage[T1]{fontenc}
\usepackage[pdfencoding=auto, pdfauthor={Hugo Delaunay}, pdfsubject={Mathématiques}, pdfcreator={agreg.skyost.eu}]{hyperref}
\usepackage{amsmath}
\usepackage{amsthm}
%\usepackage{amssymb}
\usepackage{stmaryrd}
\usepackage{tikz}
\usepackage{tkz-euclide}
\usepackage{fourier-otf}
\usepackage{fontspec}
\usepackage{titlesec}
\usepackage{fancyhdr}
\usepackage{catchfilebetweentags}
\usepackage[french, capitalise, noabbrev]{cleveref}
\usepackage[fit, breakall]{truncate}
\usepackage[top=2.5cm, right=2cm, bottom=2.5cm, left=2cm]{geometry}
\usepackage{enumerate}
\usepackage{tocloft}
\usepackage{microtype}
%\usepackage{mdframed}
%\usepackage{thmtools}
\usepackage{xcolor}
\usepackage{tabularx}
\usepackage{aligned-overset}
\usepackage[subpreambles=true]{standalone}
\usepackage{environ}
\usepackage[normalem]{ulem}
\usepackage{marginnote}
\usepackage{etoolbox}
\usepackage{setspace}
\usepackage[bibstyle=reading, citestyle=draft]{biblatex}
\usepackage{xpatch}
\usepackage[many, breakable]{tcolorbox}
\usepackage[backgroundcolor=white, bordercolor=white, textsize=small]{todonotes}

% Bibliographie :

\newcommand{\overridebibliographypath}[1]{\providecommand{\bibliographypath}{#1}}
\overridebibliographypath{../bibliography.bib}
\addbibresource{\bibliographypath}
\defbibheading{bibliography}[\bibname]{%
	\newpage
	\section*{#1}%
}
\renewbibmacro*{entryhead:full}{\printfield{labeltitle}}
\DeclareFieldFormat{url}{\newline\footnotesize\url{#1}}
\AtEndDocument{\printbibliography}

% Police :

\setmathfont{Erewhon Math}

% Tikz :

\usetikzlibrary{calc}

% Longueurs :

\setlength{\parindent}{0pt}
\setlength{\headheight}{15pt}
\setlength{\fboxsep}{0pt}
\titlespacing*{\chapter}{0pt}{-20pt}{10pt}
\setlength{\marginparwidth}{1.5cm}
\setstretch{1.1}

% Métadonnées :

\author{agreg.skyost.eu}
\date{\today}

% Titres :

\setcounter{secnumdepth}{3}

\renewcommand{\thechapter}{\Roman{chapter}}
\renewcommand{\thesubsection}{\Roman{subsection}}
\renewcommand{\thesubsubsection}{\arabic{subsubsection}}
\renewcommand{\theparagraph}{\alph{paragraph}}

\titleformat{\chapter}{\huge\bfseries}{\thechapter}{20pt}{\huge\bfseries}
\titleformat*{\section}{\LARGE\bfseries}
\titleformat{\subsection}{\Large\bfseries}{\thesubsection \, - \,}{0pt}{\Large\bfseries}
\titleformat{\subsubsection}{\large\bfseries}{\thesubsubsection. \,}{0pt}{\large\bfseries}
\titleformat{\paragraph}{\bfseries}{\theparagraph. \,}{0pt}{\bfseries}

\setcounter{secnumdepth}{4}

% Table des matières :

\renewcommand{\cftsecleader}{\cftdotfill{\cftdotsep}}
\addtolength{\cftsecnumwidth}{10pt}

% Redéfinition des commandes :

\renewcommand*\thesection{\arabic{section}}
\renewcommand{\ker}{\mathrm{Ker}}

% Nouvelles commandes :

\newcommand{\website}{https://agreg.skyost.eu}

\newcommand{\tr}[1]{\mathstrut ^t #1}
\newcommand{\im}{\mathrm{Im}}
\newcommand{\rang}{\operatorname{rang}}
\newcommand{\trace}{\operatorname{trace}}
\newcommand{\id}{\operatorname{id}}
\newcommand{\stab}{\operatorname{Stab}}

\providecommand{\newpar}{\\[\medskipamount]}

\providecommand{\lesson}[3]{%
	\title{#3}%
	\hypersetup{pdftitle={#3}}%
	\setcounter{section}{\numexpr #2 - 1}%
	\section{#3}%
	\fancyhead[R]{\truncate{0.73\textwidth}{#2 : #3}}%
}

\providecommand{\development}[3]{%
	\title{#3}%
	\hypersetup{pdftitle={#3}}%
	\section*{#3}%
	\fancyhead[R]{\truncate{0.73\textwidth}{#3}}%
}

\providecommand{\summary}[1]{%
	\textit{#1}%
	\medskip%
}

\tikzset{notestyleraw/.append style={inner sep=0pt, rounded corners=0pt, align=center}}

%\newcommand{\booklink}[1]{\website/bibliographie\##1}
\newcommand{\citelink}[2]{\hyperlink{cite.\therefsection @#1}{#2}}
\newcommand{\previousreference}{}
\providecommand{\reference}[2][]{%
	\notblank{#1}{\renewcommand{\previousreference}{#1}}{}%
	\todo[noline]{%
		\protect\vspace{16pt}%
		\protect\par%
		\protect\notblank{#1}{\cite{[\previousreference]}\\}{}%
		\protect\citelink{\previousreference}{p. #2}%
	}%
}

\definecolor{devcolor}{HTML}{00695c}
\newcommand{\dev}[1]{%
	\reversemarginpar%
	\todo[noline]{
		\protect\vspace{16pt}%
		\protect\par%
		\bfseries\color{devcolor}\href{\website/developpements/#1}{DEV}
	}%
	\normalmarginpar%
}

% En-têtes :

\pagestyle{fancy}
\fancyhead[L]{\truncate{0.23\textwidth}{\thepage}}
\fancyfoot[C]{\scriptsize \href{\website}{\texttt{agreg.skyost.eu}}}

% Couleurs :

\definecolor{property}{HTML}{fffde7}
\definecolor{proposition}{HTML}{fff8e1}
\definecolor{lemma}{HTML}{fff3e0}
\definecolor{theorem}{HTML}{fce4f2}
\definecolor{corollary}{HTML}{ffebee}
\definecolor{definition}{HTML}{ede7f6}
\definecolor{notation}{HTML}{f3e5f5}
\definecolor{example}{HTML}{e0f7fa}
\definecolor{cexample}{HTML}{efebe9}
\definecolor{application}{HTML}{e0f2f1}
\definecolor{remark}{HTML}{e8f5e9}
\definecolor{proof}{HTML}{e1f5fe}

% Théorèmes :

\theoremstyle{definition}
\newtheorem{theorem}{Théorème}

\newtheorem{property}[theorem]{Propriété}
\newtheorem{proposition}[theorem]{Proposition}
\newtheorem{lemma}[theorem]{Lemme}
\newtheorem{corollary}[theorem]{Corollaire}

\newtheorem{definition}[theorem]{Définition}
\newtheorem{notation}[theorem]{Notation}

\newtheorem{example}[theorem]{Exemple}
\newtheorem{cexample}[theorem]{Contre-exemple}
\newtheorem{application}[theorem]{Application}

\theoremstyle{remark}
\newtheorem{remark}[theorem]{Remarque}

\counterwithin*{theorem}{section}

\newcommand{\applystyletotheorem}[1]{
	\tcolorboxenvironment{#1}{
		enhanced,
		breakable,
		colback=#1!98!white,
		boxrule=0pt,
		boxsep=0pt,
		left=8pt,
		right=8pt,
		top=8pt,
		bottom=8pt,
		sharp corners,
		after=\par,
	}
}

\applystyletotheorem{property}
\applystyletotheorem{proposition}
\applystyletotheorem{lemma}
\applystyletotheorem{theorem}
\applystyletotheorem{corollary}
\applystyletotheorem{definition}
\applystyletotheorem{notation}
\applystyletotheorem{example}
\applystyletotheorem{cexample}
\applystyletotheorem{application}
\applystyletotheorem{remark}
\applystyletotheorem{proof}

% Environnements :

\NewEnviron{whitetabularx}[1]{%
	\renewcommand{\arraystretch}{2.5}
	\colorbox{white}{%
		\begin{tabularx}{\textwidth}{#1}%
			\BODY%
		\end{tabularx}%
	}%
}

% Maths :

\DeclareFontEncoding{FMS}{}{}
\DeclareFontSubstitution{FMS}{futm}{m}{n}
\DeclareFontEncoding{FMX}{}{}
\DeclareFontSubstitution{FMX}{futm}{m}{n}
\DeclareSymbolFont{fouriersymbols}{FMS}{futm}{m}{n}
\DeclareSymbolFont{fourierlargesymbols}{FMX}{futm}{m}{n}
\DeclareMathDelimiter{\VERT}{\mathord}{fouriersymbols}{152}{fourierlargesymbols}{147}



\begin{document}
	%<*content>
	\development{analysis}{projection-sur-un-convexe-ferme}{Projection sur un convexe fermé}

	\summary{On montre le théorème de projection sur un convexe fermé dans un espace de Hilbert réel en utilisant les suites de Cauchy et des propriétés du produit scalaire.}

	Soit $H$ un espace de Hilbert réel de norme $\Vert . \Vert$ et dont on note $\langle ., . \rangle$ le produit scalaire associé.

	\begin{lemma}[Identité du parallélogramme]
		\label{projection-sur-un-convexe-ferme-1}
		Soient $x, y \in H$. Alors :
		\[ \Vert x + y \Vert^2 + \Vert x - y \Vert^2 = 2(\Vert x \Vert^2 + \Vert y \Vert^2) \]
	\end{lemma}

	\begin{demonstration}
		D'une part,
		\[ \Vert x + y \Vert^2 = \langle x + y, x + y \rangle = \Vert x \Vert^2 + \Vert y \Vert^2 + 2 \langle x, y \rangle \]
		D'autre part,
		\[ \Vert x - y \Vert^2 = \langle x - y, x - y \rangle = \Vert x \Vert^2 + \Vert y \Vert^2 - 2 \langle x, y \rangle \]
		En additionnant, on obtient bien l'égalité voulue.
	\end{demonstration}

	\begin{remark}
		L'interprétation géométrique de cette égalité est que dans le parallélogramme formé par les vecteurs $x$ et $y$, la somme des carrés des diagonales est égale à la somme des carrés des côtés.
		\includelatexpicture{projection-sur-un-convexe-ferme-1}
	\end{remark}

	\reference[GOU20]{427}

	\begin{theorem}[Projection sur un convexe fermé]
		\label{projection-sur-un-convexe-ferme-2}
		Soit $C \subset H$ un convexe fermé non-vide. Alors :
		\[ \forall x \in H, \exists! y \in C \text{ tel que } d(x, C) = \inf_{z \in C} \Vert x - z \Vert = d(x, y) \]
		On peut donc noter $y = P_C(x)$, le \textbf{projeté orthogonal de $x$ sur $C$}. Il s'agit de l'unique point de $C$ vérifiant
		\[ \forall z \in C, \, \langle x - P_C(x), z - P_C(x) \rangle \leq 0 \tag{$*$} \]
		\includelatexpicture{projection-sur-un-convexe-ferme-2}
	\end{theorem}

	\begin{demonstration}
		Soit $x \in H$. Posons $\delta = d(x, C)$. Par la caractérisation séquentielle de la borne inférieure, il existe $(y_n)$ une suite de $C$ telle que $\Vert x - y_n \Vert \longrightarrow \delta$. Montrons que $(y_n)$ est une suite de Cauchy. On applique le \cref{projection-sur-un-convexe-ferme-1} :
		\[ \forall p, q \in \mathbb{N}, \, \Vert (x - y_p) + (x - y_q) \Vert^2 + \Vert y_p - y_q \Vert^2 = 2(\Vert x - y_p \Vert^2 + \Vert x - y_q \Vert^2) \tag{$**$} \]
		Or, $C$ est convexe. Donc $\forall p, q \in \mathbb{N}$, $\frac{y_p + y_q}{2} \in C$.
		Par définition,
		\begin{align*}
			& \left\Vert x - \frac{y_p + y_q}{2} \right\Vert \geq \delta \\
			\iff& \frac{1}{2} \Vert (x - y_p) + (x - y_q) \Vert \geq \delta \\
			\iff& \Vert (x - y_p) + (x - y_q) \Vert^2 \geq 4 \delta^2
		\end{align*}
		Par $(**)$, quand $p, q \longrightarrow +\infty$ :
		\[ \Vert y_p - y_q \Vert \leq 2((\underbrace{\Vert x - y_p \Vert^2}_{\longrightarrow \delta^2} - \delta^2) + (\underbrace{\Vert x - y_q \Vert^2}_{\longrightarrow \delta^2} - \delta^2)) \longrightarrow 0 \]
		Ainsi $(y_n)$ est une suite de Cauchy de $H$ qui est complet, donc $(y_n)$ converge vers $y \in H$. Mais, $C$ est fermé et $(y_n)$ est une suite de $C$, donc $y \in C$.
		\newpar
		Montrons maintenant que $y$ est unique. Soit $z \in C$ tel que $\delta = d(x, C)$. On définit la suite $(z_n)$ par
		\[ \forall n \in \mathbb{N}, \, z_n =
		\begin{cases}
			y \text{ si } n \text{ est pair} \\
			z \text{ si } n \text{ est impair} \\
		\end{cases}
		\]
		Cette suite vérifie $\forall n \in \mathbb{N}$, $\Vert x - y_n \Vert = \delta$ donc en particulier $\Vert x - y_n \Vert \longrightarrow \delta$, et on peut tout-à-fait refaire le raisonnement précédent pour montrer que $(z_n)$ converge (vers $y = z$, donc). Ainsi, on a bien existence et unicité du projeté.
		\newpar
		Soit $y \in C$ vérifiant $(*)$. Montrons que $y = P_C(x)$. $\forall z \in C$,
		\begin{align*}
			\Vert z - x \Vert^2 &= \Vert (z - y) - (x - y) \Vert^2 \\
			&= \Vert z - y \Vert^2 + \Vert x - y \Vert^2 - 2 \langle z - y, x - y \rangle \\
			&\geq \Vert z - y \Vert^2 + \Vert x - y \Vert^2 \\
			&\geq \Vert x - y \Vert^2
		\end{align*}
		ie. $\Vert z - x \Vert \geq \Vert x - y \Vert$. De plus, $y \in C$, donc $d(y, C) = d(x, C)$. D'où $y = P_C(x)$.
		\newpar
		Montrons maintenant que $P_C(x)$ vérifie bien $(*)$. Et $\forall z \in C$, on a
		\[ \Vert x - z \Vert^2 \geq \Vert x - P_C(x) \Vert^2 \]
		Or, en développant :
		\begin{align*}
			\Vert x - z \Vert^2 &= \Vert (x - P_C(x)) - (z - P_C(x)) \Vert^2 \\
			&= \Vert x - P_C(x) \Vert^2 + \Vert z - P_C(x) \Vert^2 - 2 \langle x - P_C(x), z - P_C(x) \rangle \\
			&\geq \Vert x - P_C(x) \Vert^2
		\end{align*}
		D'où,
		\[ \Vert z - P_C(x) \Vert^2 - 2 \langle x - P_C(x), z - P_C(x) \rangle \geq 0 \tag{$***$} \]
		Soit maintenant $z_0 \in C$. On va appliquer $(***)$ à $z = \lambda z_0 + (1 - \lambda) z_0 \in C$ pour $\lambda \in ]0, 1]$ :
		\begin{align*}
			& \lambda^2 \Vert z_0 + P_C(x) \Vert^2 -2 \lambda \langle x - P_C(x), z_0 - P_C(x) \rangle \geq 0 \\
			\implies& \lambda \Vert z_0 + P_C(x) \Vert^2 -2 \langle x - P_C(x), z_0 - P_C(x) \rangle \geq 0 \\
			\overset{\lambda \longrightarrow 0}{\implies}& -2 \langle x - P_C(x), z_0 - P_C(x) \rangle \geq 0
		\end{align*}
		ce que l'on voulait.
	\end{demonstration}

	\begin{remark}
		$(*)$ traduit le fait géométrique que l'angle du vecteur $\overrightarrow{P_C(x)x}$ avec $\overrightarrow{P_C(x)z}$ est obtus pour tout $z \in C$. En effet, en notant cet angle $\theta$, on a pour $z \in C$ :
		\[ \langle x - P_C(x), z - P_C(x) \rangle = \Vert x - P_C(x) \Vert \Vert z - P_C(x) \Vert \cos (\theta) \]
		et si $\theta$ est obtus, on a donc $\cos (\theta) \leq 0$.
	\end{remark}

	\begin{corollary}
		Soit $F$ un sous-espace vectoriel fermé de $H$. Alors $F \oplus F^\perp = H$.
	\end{corollary}

	\begin{demonstration}
		Si $x \in F \, \cap \, F^\perp$, alors $\Vert x \Vert = \langle x, x \rangle = 0$, et donc $x = 0$. Montrons maintenant que $F + F^\perp = H$. Soit $x \in H$. Comme $F$ est un convexe fermé de $H$ (en tant que sous-espace vectoriel fermé), on peut appliquer le \cref{projection-sur-un-convexe-ferme-2}. Ainsi, il existe un unique $P_F(x) \in F$ tel que $d(x, F) = d(x, P_F(x))$ et
		\[ \forall z \in F, \, \langle x - P_F(x), z - P_F(x) \rangle \leq 0 \tag{$*$} \]
		Soit $z_0 \in F$. on peut appliquer $(*)$ à $z = z_0$ :
		\[ \langle x - P_F(x), z_0 - P_F(x) \rangle \leq 0 \]
		On va également appliquer $(*)$ à $z = -z_0 + 2P_F(x) \in F$ :
		\[ \langle x - P_F(x), -z_0 + P_F(x) \rangle \leq 0 \iff \langle x - P_F(x), z_0 - P_F(x) \rangle \geq 0 \]
		Ce qui montre que l'inégalité de $(*)$ est en fait une égalité. On en tire :
		\[ \forall z \in F, \, \langle x - P_F(x), z \rangle = \langle x - P_F(x), z - P_F(x) \rangle - \langle x - P_F(x), 0 - P_F(x) \rangle = 0 \]
		donc $x - P_F(x) \in F^\perp$. En conclusion, on a :
		\[ x = \underbrace{P_F(x)}_{\in F} + \underbrace{x - P_F(x)}_{\in F^\perp} \in F + F^\perp \]
		et on a donc bien la somme directe $H = F \oplus F^\perp$.
	\end{demonstration}
	%</content>
\end{document}
