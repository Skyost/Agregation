\documentclass[12pt, a4paper]{report}

% LuaLaTeX :

\RequirePackage{iftex}
\RequireLuaTeX

% Packages :

\usepackage[french]{babel}
%\usepackage[utf8]{inputenc}
%\usepackage[T1]{fontenc}
\usepackage[pdfencoding=auto, pdfauthor={Hugo Delaunay}, pdfsubject={Mathématiques}, pdfcreator={agreg.skyost.eu}]{hyperref}
\usepackage{amsmath}
\usepackage{amsthm}
%\usepackage{amssymb}
\usepackage{stmaryrd}
\usepackage{tikz}
\usepackage{tkz-euclide}
\usepackage{fourier-otf}
\usepackage{fontspec}
\usepackage{titlesec}
\usepackage{fancyhdr}
\usepackage{catchfilebetweentags}
\usepackage[french, capitalise, noabbrev]{cleveref}
\usepackage[fit, breakall]{truncate}
\usepackage[top=2.5cm, right=2cm, bottom=2.5cm, left=2cm]{geometry}
\usepackage{enumerate}
\usepackage{tocloft}
\usepackage{microtype}
%\usepackage{mdframed}
%\usepackage{thmtools}
\usepackage{xcolor}
\usepackage{tabularx}
\usepackage{aligned-overset}
\usepackage[subpreambles=true]{standalone}
\usepackage{environ}
\usepackage[normalem]{ulem}
\usepackage{marginnote}
\usepackage{etoolbox}
\usepackage{setspace}
\usepackage[bibstyle=reading, citestyle=draft]{biblatex}
\usepackage{xpatch}
\usepackage[many, breakable]{tcolorbox}
\usepackage[backgroundcolor=white, bordercolor=white, textsize=small]{todonotes}

% Bibliographie :

\newcommand{\overridebibliographypath}[1]{\providecommand{\bibliographypath}{#1}}
\overridebibliographypath{../bibliography.bib}
\addbibresource{\bibliographypath}
\defbibheading{bibliography}[\bibname]{%
	\newpage
	\section*{#1}%
}
\renewbibmacro*{entryhead:full}{\printfield{labeltitle}}
\DeclareFieldFormat{url}{\newline\footnotesize\url{#1}}
\AtEndDocument{\printbibliography}

% Police :

\setmathfont{Erewhon Math}

% Tikz :

\usetikzlibrary{calc}

% Longueurs :

\setlength{\parindent}{0pt}
\setlength{\headheight}{15pt}
\setlength{\fboxsep}{0pt}
\titlespacing*{\chapter}{0pt}{-20pt}{10pt}
\setlength{\marginparwidth}{1.5cm}
\setstretch{1.1}

% Métadonnées :

\author{agreg.skyost.eu}
\date{\today}

% Titres :

\setcounter{secnumdepth}{3}

\renewcommand{\thechapter}{\Roman{chapter}}
\renewcommand{\thesubsection}{\Roman{subsection}}
\renewcommand{\thesubsubsection}{\arabic{subsubsection}}
\renewcommand{\theparagraph}{\alph{paragraph}}

\titleformat{\chapter}{\huge\bfseries}{\thechapter}{20pt}{\huge\bfseries}
\titleformat*{\section}{\LARGE\bfseries}
\titleformat{\subsection}{\Large\bfseries}{\thesubsection \, - \,}{0pt}{\Large\bfseries}
\titleformat{\subsubsection}{\large\bfseries}{\thesubsubsection. \,}{0pt}{\large\bfseries}
\titleformat{\paragraph}{\bfseries}{\theparagraph. \,}{0pt}{\bfseries}

\setcounter{secnumdepth}{4}

% Table des matières :

\renewcommand{\cftsecleader}{\cftdotfill{\cftdotsep}}
\addtolength{\cftsecnumwidth}{10pt}

% Redéfinition des commandes :

\renewcommand*\thesection{\arabic{section}}
\renewcommand{\ker}{\mathrm{Ker}}

% Nouvelles commandes :

\newcommand{\website}{https://agreg.skyost.eu}

\newcommand{\tr}[1]{\mathstrut ^t #1}
\newcommand{\im}{\mathrm{Im}}
\newcommand{\rang}{\operatorname{rang}}
\newcommand{\trace}{\operatorname{trace}}
\newcommand{\id}{\operatorname{id}}
\newcommand{\stab}{\operatorname{Stab}}

\providecommand{\newpar}{\\[\medskipamount]}

\providecommand{\lesson}[3]{%
	\title{#3}%
	\hypersetup{pdftitle={#3}}%
	\setcounter{section}{\numexpr #2 - 1}%
	\section{#3}%
	\fancyhead[R]{\truncate{0.73\textwidth}{#2 : #3}}%
}

\providecommand{\development}[3]{%
	\title{#3}%
	\hypersetup{pdftitle={#3}}%
	\section*{#3}%
	\fancyhead[R]{\truncate{0.73\textwidth}{#3}}%
}

\providecommand{\summary}[1]{%
	\textit{#1}%
	\medskip%
}

\tikzset{notestyleraw/.append style={inner sep=0pt, rounded corners=0pt, align=center}}

%\newcommand{\booklink}[1]{\website/bibliographie\##1}
\newcommand{\citelink}[2]{\hyperlink{cite.\therefsection @#1}{#2}}
\newcommand{\previousreference}{}
\providecommand{\reference}[2][]{%
	\notblank{#1}{\renewcommand{\previousreference}{#1}}{}%
	\todo[noline]{%
		\protect\vspace{16pt}%
		\protect\par%
		\protect\notblank{#1}{\cite{[\previousreference]}\\}{}%
		\protect\citelink{\previousreference}{p. #2}%
	}%
}

\definecolor{devcolor}{HTML}{00695c}
\newcommand{\dev}[1]{%
	\reversemarginpar%
	\todo[noline]{
		\protect\vspace{16pt}%
		\protect\par%
		\bfseries\color{devcolor}\href{\website/developpements/#1}{DEV}
	}%
	\normalmarginpar%
}

% En-têtes :

\pagestyle{fancy}
\fancyhead[L]{\truncate{0.23\textwidth}{\thepage}}
\fancyfoot[C]{\scriptsize \href{\website}{\texttt{agreg.skyost.eu}}}

% Couleurs :

\definecolor{property}{HTML}{fffde7}
\definecolor{proposition}{HTML}{fff8e1}
\definecolor{lemma}{HTML}{fff3e0}
\definecolor{theorem}{HTML}{fce4f2}
\definecolor{corollary}{HTML}{ffebee}
\definecolor{definition}{HTML}{ede7f6}
\definecolor{notation}{HTML}{f3e5f5}
\definecolor{example}{HTML}{e0f7fa}
\definecolor{cexample}{HTML}{efebe9}
\definecolor{application}{HTML}{e0f2f1}
\definecolor{remark}{HTML}{e8f5e9}
\definecolor{proof}{HTML}{e1f5fe}

% Théorèmes :

\theoremstyle{definition}
\newtheorem{theorem}{Théorème}

\newtheorem{property}[theorem]{Propriété}
\newtheorem{proposition}[theorem]{Proposition}
\newtheorem{lemma}[theorem]{Lemme}
\newtheorem{corollary}[theorem]{Corollaire}

\newtheorem{definition}[theorem]{Définition}
\newtheorem{notation}[theorem]{Notation}

\newtheorem{example}[theorem]{Exemple}
\newtheorem{cexample}[theorem]{Contre-exemple}
\newtheorem{application}[theorem]{Application}

\theoremstyle{remark}
\newtheorem{remark}[theorem]{Remarque}

\counterwithin*{theorem}{section}

\newcommand{\applystyletotheorem}[1]{
	\tcolorboxenvironment{#1}{
		enhanced,
		breakable,
		colback=#1!98!white,
		boxrule=0pt,
		boxsep=0pt,
		left=8pt,
		right=8pt,
		top=8pt,
		bottom=8pt,
		sharp corners,
		after=\par,
	}
}

\applystyletotheorem{property}
\applystyletotheorem{proposition}
\applystyletotheorem{lemma}
\applystyletotheorem{theorem}
\applystyletotheorem{corollary}
\applystyletotheorem{definition}
\applystyletotheorem{notation}
\applystyletotheorem{example}
\applystyletotheorem{cexample}
\applystyletotheorem{application}
\applystyletotheorem{remark}
\applystyletotheorem{proof}

% Environnements :

\NewEnviron{whitetabularx}[1]{%
	\renewcommand{\arraystretch}{2.5}
	\colorbox{white}{%
		\begin{tabularx}{\textwidth}{#1}%
			\BODY%
		\end{tabularx}%
	}%
}

% Maths :

\DeclareFontEncoding{FMS}{}{}
\DeclareFontSubstitution{FMS}{futm}{m}{n}
\DeclareFontEncoding{FMX}{}{}
\DeclareFontSubstitution{FMX}{futm}{m}{n}
\DeclareSymbolFont{fouriersymbols}{FMS}{futm}{m}{n}
\DeclareSymbolFont{fourierlargesymbols}{FMX}{futm}{m}{n}
\DeclareMathDelimiter{\VERT}{\mathord}{fouriersymbols}{152}{fourierlargesymbols}{147}


% Bibliographie :

\addbibresource{\bibliographypath}%
\defbibheading{bibliography}[\bibname]{%
	\newpage
	\section*{#1}%
}
\renewbibmacro*{entryhead:full}{\printfield{labeltitle}}%
\DeclareFieldFormat{url}{\newline\footnotesize\url{#1}}%

\AtEndDocument{\printbibliography}

\begin{document}
  %<*content>
  \lesson{analysis}{226}{Suites vectorielles et réelles définies par une relation de récurrence \texorpdfstring{$u_{n+1} = f(u_n)$}{un+1=f(un)}. Exemples. Applications à la résolution approchée d'équations.}

  \subsection{Suites récurrentes}

  \subsubsection{Définition et premières propriétés}

  \reference[DAN]{145}

  \begin{definition}
    Soit $E$ un ensemble. On dit qu'une suite $(u_n)$ d'éléments de $E$ est \textbf{récurrente} d'ordre $h \in \mathbb{N}^*$ si on peut écrire
    \[ \forall n \geq h, \, u_{n+h} = f(u_{n-1}, \dots, u_{n-h}) \tag{$*$} \]
    où $f : E^h \rightarrow E$ et les premières valeurs $u_0, \dots, u_{h-1} \in E$ étant donnés.
  \end{definition}

  \begin{example}
    On considère la suite numérique $(u_n)$ définie par
    \[
      \begin{cases}
        u_0 = 0 \\
        u_1 = -1 \\
        \forall n \in \mathbb{N}, \, u_{n+2} = 5u_{n+1} - 6u_n
      \end{cases}
    \]
    et on a,
    \[ \forall n \in \mathbb{N}, \, u_n = 2^n - 3^n \]
  \end{example}

  \reference[GOU20]{206}

  \begin{example}
    On considère les suite numérique $(u_n)$ et $(v_n)$ définies par
    \[
      \begin{cases}
        u_0 \geq 0 \\
        \forall n \in \mathbb{N}, \, u_{n+1} = \sqrt{\frac{1+u_n}{2}}
      \end{cases}
      \text{ et }
      \forall n \in \mathbb{N}, \, v_n = \prod_{k=0}^{n} u_k
    \]
    Alors, pour $u_0 = \cos(\theta)$, on a
    \[ \forall n \in \mathbb{N}, \, v_n = \prod_{k=1}^n \cos \left( \frac{\theta}{2^k} \right) = \frac{\sin(\theta)}{2^n \sin \left( \frac{\theta}{2^n} \right)} \]
    donc
    \[ \lim_{n \rightarrow +\infty} v_n = \frac{\sin(\theta)}{\theta} \]
  \end{example}

  \begin{application}[Formule de Viète]
    \[ \frac{2}{\pi} = \sqrt{\frac{1}{2}} \times \sqrt{\frac{1}{2} + \frac{1}{2} \sqrt{\frac{1}{2}}} \times \dots \]
  \end{application}

  \reference[FGN3]{160}

  \begin{example}
    La suite de fonctions polynômiales $(P_n)$ définie par récurrence par :
    \[ P_0 : z \mapsto 1, \, P_1 : z \mapsto z, \text{ et } \forall n \geq 1, \, zP_n : z \mapsto P_{n-1}(z)-P_{n+1}(z) \]
    est une suite bornée si et seulement si $z = \pm 1$.
  \end{example}

  \reference[I-P]{116}

  \begin{theorem}
    Soit $(E, d)$ un espace métrique compact. Soit $(u_n)$ une suite de $E$ telle que $d(u_n,u_{n-1}) \longrightarrow 0$. Alors l'ensemble $\Gamma$ des valeurs d'adhérence de $(u_n)$ est connexe.
  \end{theorem}

  \begin{corollary}[Lemme de la grenouille]
    Soient $f : [0, 1] \rightarrow [0, 1]$ continue et $(x_n)$ une suite de $[0, 1]$ telle que
    \[ \begin{cases} x_0 \in [0, 1] \\ x_{n+1} = f(x_n) \end{cases} \]
    Alors $(x_n)$ converge si et seulement si $\lim_{n \rightarrow +\infty } x_{n+1} - x_n = 0$.
  \end{corollary}

  \subsubsection{Récurrences classiques}

  \reference[GOU20]{201}

  Soit $\mathbb{K} = \mathbb{R}$ ou $\mathbb{C}$. On fixe $(u_n)$ une suite récurrente d'ordre $1$ définie par $u_{n+1} = f(u_n)$ où $f : \mathbb{K} \rightarrow \mathbb{K}$.

  \begin{definition}
    \begin{itemize}
      \item Si $f$ est une translation (ie. $f$ est de la forme $f : x \mapsto x + b$ où $b \in \mathbb{K}$), alors $(u_n)$ est une suite \textbf{arithmétique} de raison $b$.
      \item Si $f$ est linéaire (ie. $f$ est de la forme $f : x \mapsto ax$ où $a \in \mathbb{K}$), alors $(u_n)$ est une suite \textbf{géométrique} de raison $a$.
      \item Si $f$ est affine (ie. $f$ est de la forme $f : x \mapsto ax + b$ où $a, b \in \mathbb{K}$), alors $(u_n)$ est une suite \textbf{arithmético-géométrique}.
      \item Si $f$ est homographique (ie. $f$ est de la forme $f : x \mapsto \frac{ax + b}{cx + d}$ où $a$, $b$, $c$, $d \in E$ et $ad - bc \neq 0$), alors $(u_n)$ vérifie une \textbf{récurrence homographique}.
    \end{itemize}
  \end{definition}

  \begin{proposition}
    \begin{enumerate}[label=(\roman*)]
      \item Si $(u_n)$ est arithmétique de raison $b$, alors $\forall n \in \mathbb{N}$, $u_n = u_0 + n b$.
      \item Si $(u_n)$ est géométrique de raison $a$, alors $\forall n \in \mathbb{N}$, $u_n = a^n u_0$.
      \item Si $(u_n)$ est arithmético-géométrique et si $1 - a \neq 0$, en posant $r = (1 - a)^{-1} b$, on a $\forall n \in \mathbb{N}$, $u_n = a^n (u_0 - r) + r$.
    \end{enumerate}
  \end{proposition}

  \begin{proposition}
    Supposons que $(u_n)$ vérifie une récurrence homographique. On considère l'équation
    \[ f(x) = x \iff cx^2 - (a-d)x - b = 0 \tag{$E$} \]
    Alors :
    \begin{enumerate}
      \item Si $(E)$ admet deux racines distinctes $r_1$ et $r_2$, on a
      $\forall n \in \mathbb{N}$, $\frac{u_n - r_1}{u_n - r_2} = k^n \frac{u_0 - r_1}{u_0 - r_2}$ où $k = \frac{a - r_1 c}{a - r_2 c}$.
      \item Si $(E)$ admet une racine double $r$, on a $\forall n \in \mathbb{N}$, $\frac{1}{u_n - r} = \frac{1}{u_0 - r} + kn$ où $k = \frac{c}{a - r c}$.
    \end{enumerate}
  \end{proposition}

  \begin{remark}
    Ces formules permettent de décider s'il existe un rang $n$ tel que le dénominateur de $f$ s'annule, auquel cas les termes ultérieurs de la suite ne sont pas définis.
  \end{remark}

  \begin{example}
    Pour la relation $u_{n+1} = \frac{2u_n + 1}{u_n + 2}$, l'équation $(E)$ admet $\pm 1$ pour solutions, donc $\frac{u_n + 1}{u_n - 1} = 3^n \frac{u_0 + 1}{u_0 - 1}$.
  \end{example}

  \subsubsection{Suites récurrentes vectorielles}

  \reference[GOU21]{153}

  \begin{proposition}[Déterminant circulant]
    Soient $n \in \mathbb{N}^*$ et $a_1, \dots, a_n \in \mathbb{C}$. On pose $\omega = e^{\frac{2i\pi}{n}}$. Alors
    \[ \begin{vmatrix} a_0 & a_1 & \dots & a_{n-1} \\ a_{n-1} & a_0 & \dots & a_{n-2}\\ \vdots & \vdots & \ddots & \vdots \\ a_1 & a_2 & \dots & a_0 \end{vmatrix} = \prod_{j=0}^{n-1} P(\omega^j) \]
    où $P = \sum_{k=0}^{n-1} a_k X^k$.
  \end{proposition}

  \reference[I-P]{389}
  \dev{suite-de-polygones}

  \begin{application}[Suite de polygones]
    Soit $P_0$ un polygone dont les sommets sont $\{ z_{0,1}, \dots, z_{0,n} \}$. On définit la suite de polygones $(P_k)$ par récurrence en disant que, pour tout $k \in \mathbb{N}^*$, les sommets de $P_{k+1}$ sont les milieux des arêtes de $P_k$.
    \newpar
    Alors la suite $(P_k)$ converge vers l'isobarycentre de $P_0$.
  \end{application}

  \subsection{Outils pour étudier les suites récurrentes}

  \subsubsection{Stabilité de l'intervalle et continuité}

  \reference[AMR11]{38}

  Soient $I \subseteq \mathbb{R}$ un intervalle de $\mathbb{R}$. On fixe $(u_n)$ une suite récurrente d'ordre $1$ définie par $u_{n+1} = f(u_n)$ où $f : I \rightarrow \mathbb{R}$.

  \begin{theorem}[Caractérisation séquentielle de la continuité]
    En reprenant les notations précédentes, une fonction $g : I \rightarrow \mathbb{R}$ est continue si et seulement si pour toute suite réelle convergente $(v_n) \in I^{\mathbb{N}}$ dont on note $\ell$ la limite, $g(v_n) \longrightarrow_{n \rightarrow +\infty} \ell$.
  \end{theorem}

  \begin{corollary}
    Si une suite récurrente d'ordre $1$ (dont on note $f$ la fonction) converge vers $\ell$, alors $f(\ell) = \ell$.
  \end{corollary}

  \begin{example}
    La suite $(u_n)$ définie par $u_0 \in \left[ -\frac{\pi}{2}, \frac{\pi}{2} \right]$ et $\forall n \geq 1, \, u_{n+1} = \sin(u_n)$ converge vers $0$.
  \end{example}

  \reference[GOU20]{200}

  \begin{proposition}
    \begin{enumerate}[label=(\roman*)]
      \item Si $f$ est croissante, alors $(u_n)$ est monotone et son sens de monotonie est donnée par le signe de $u_1 - u_0$.
      \item Si $f$ est décroissante, alors $(u_{2n})$ et $(u_{2n+1})$ sont monotones et leur sens de monotonie est opposé.
    \end{enumerate}
  \end{proposition}

  \begin{example}
    La suite réelle $(u_n)$ définie par récurrence par :
    \[ u_0 \in [0, 1[ \text{ et } \forall n \geq 0, \, u_{n+1} = \frac{1}{2 - \sqrt{u_n}} \]
    est une suite qui converge vers $1$.
  \end{example}

  \subsubsection{Équation caractéristique}

  \begin{definition}
    Une suite $(u_n)$ à valeurs dans $\mathbb{C}$ vérifie une \textbf{récurrence linéaire homogène} d'ordre $h$ si
    \[ \forall n \in \mathbb{N}, \quad u_{n+h} = a_{h-1} u_{n+h-1} + \dots + a_0 u_0 \tag{$*$} \]
    où $a_1, \dots, a_h \in \mathbb{C}$.
  \end{definition}

  \begin{proposition}
    Si on note $r_1, \dots, r_q$ les racines du polynôme caractéristique de $(*)$ (de multiplicités respectives $\alpha_1, \dots, \alpha_q$), alors l'ensemble des suites vérifiant $(*)$ est l'ensemble des suites $(u_n)$ telles que :
    \[ u_n = P_1(n) r_1^n + \dots + P_q(n) r_q^n \]
    où $\forall i \in \llbracket 1, q \rrbracket$, $P_i$ est un polynôme de degré strictement inférieur à $\alpha_i$.
  \end{proposition}

  \begin{example}
    Soit $(u_n)$ la suite définie par $\forall n \in \mathbb{N}, \, u_n = a u_{n-1} + b u_{n-2}$. Son polynôme caractéristique est $P = X^2 - aX - b$.
    \begin{enumerate}
      \item Si $P$ a deux racines distinctes $r_1$ et $r_2$, alors $\forall n \in \mathbb{N}, \, u_n = \lambda r_1^n + \mu r_2^n$ où $\lambda$ et $\mu$ sont tels que $u_0 = \lambda + \mu$ et $u_1 = \lambda r_1 + \mu r_2$.
      \item Si $P$ a une racine double $r$, alors $\forall n \in \mathbb{N}, \, u_n = (\lambda n + \mu) r^n$ où $\lambda$ et $\mu$ sont tels que $u_0 = \mu$ et $u_1 = (\lambda + \mu) r$.
    \end{enumerate}
  \end{example}

  \reference[AMR11]{47}

  \begin{example}
    Soit $(F_n)$ la suite de Fibonacci définie par $F_0 = 0$, $F_1 = 1$ et $\forall n \geq 2$, $F_n = F_{n-1} + F_{n-2}$. Alors,
    \[ \forall n \in \mathbb{N}, \, F_n = \frac{1}{\sqrt{5}} \left ( \left ( \frac{1 + \sqrt{5}}{2} \right)^n - \left ( \frac{1 - \sqrt{5}}{2} \right)^n \right) \]
  \end{example}

  \begin{example}
    La suite $(u_n)$ définie par $u_0 = 1$ et $u_{n+2} = u_n - u_{n+1}$ est à termes positifs si et seulement si $u_1 = \frac{1 - \sqrt{5}}{2}$.
  \end{example}

  \subsubsection{Développement asymptotique}

  \reference{53}

  \begin{definition}
    À toute suite numérique $(u_n)$ on y associe sa suite $(v_n)$ des \textbf{moyennes de Cesàro} où
    \[ \forall n \in \mathbb{N}, v_n = \frac{1}{n} \sum_{k=1}^{n} u_k \]
  \end{definition}

  \begin{theorem}
    Si $(u_n)$ converge vers $\ell \in \mathbb{K}$, alors sa suite des moyennes de Cesàro converge vers $\ell$. On dit que $(u_n)$ converge \textbf{au sens de Cesàro}.
  \end{theorem}

  \reference[FGN3]{142}

  \begin{proposition}
    Soit $f$ une application continue définie au voisinage de $0^+$ admettant
    un développement asymptotique en $0$ de la forme $f(x) = x - ax^\alpha + o(x^\alpha)$, où $a > 0$ et $\alpha > 1$. Alors pour $u_0 > 0$ assez petit, la suite $(u_n)$ définie par $u_{n+1} = f(u_n)$ pour $n \in \mathbb{N}$ vérifie
    \[ u_n \sim \frac{1}{(na(\alpha-1))^{\frac{1}{\alpha-1}}} \]
  \end{proposition}

  \begin{example}
    Si $f = \sin$ et $(u_n)$ est définie par $u_0 \in [0, 2\pi]$ et $\forall n \in \mathbb{N}, \, u_{n+1} = f(u_n)$, on a l'équivalent en $+\infty$ :
    \[ u_n \sim \sqrt{\frac{3}{n}} \]
  \end{example}

  \reference[GOU20]{228}

  \begin{proposition}
    En reprenant les notations précédentes, on a, pour $u_0 \in \left] 0, \frac{\pi}{2} \right]$,
    \[ u_n = \sqrt{\frac{3}{n}} - \frac{3 \sqrt{3}}{10} \frac{\ln(n)}{n\sqrt{n}} + o\left( \frac{\ln(n)}{n\sqrt{n}} \right) \]
  \end{proposition}

  \reference[FGN3]{148}

  \begin{example}
    On définit $(u_n)$ par $u_0 \in \mathbb{R}$ et $\forall n \in \mathbb{N}, \, u_{n+1} = u_n + e^{-u_n}$, on a l'équivalent en $+\infty$ :
    \[ u_n = n + \frac{\ln(n)}{2n} + o\left( \frac{\ln(n)}{n} \right) \]
  \end{example}

  \subsection{Applications à la résolution approchée d'équations}

  \subsubsection{Point fixe et itération}

  \reference[DAN]{146}

  \begin{theorem}[Point fixe de Banach]
    Soient $(E,d)$ un espace métrique complet et $f : E \rightarrow E$ une application contractant (ie. $\exists k \in ]0,1[ \text{ tel que } \forall x, y \in E, \, d(f(x), f(y)) \leq k d(x, y)$). Alors,
    \[ \exists! x \in E \text{ tel que } f(x) = x \]
    De plus la suite des itérés définie par $x_0 \in E$ et $\forall n \in \mathbb{N}, x_{n+1} = f(x_n)$ converge vers $x$.
  \end{theorem}

  \begin{theorem}[Point fixe dans un compact]
    Soit $(E,d)$ un espace métrique compact et $f : E \rightarrow E$ telle que
    \[ \forall x, y \in E, \, x \neq y \implies d(f(x), f(y)) < d(x,y) \]
    alors $f$ admet un unique point fixe et pour tout $x_0 \in E$, la suite des itérés
    \[ x_{n+1} = f(x_n) \]
    converge vers ce point fixe.
  \end{theorem}

  \reference[DEM]{95}

  \begin{application}
    Soient $a, b \in \mathbb{R}$ et $f : [a, b] \rightarrow \mathbb{R}$ dérivable, strictement croissante et telle que $f(a) < 0$, $f(b) > 0$ et $0 < m \leq f'(x) \leq M$ sur $[a, b]$. On pose $\varphi : x \mapsto x - \frac{1}{M} f(x)$. On considère l'équation :
    \[ f(x) = 0 \iff \varphi(x) = x \tag{$E$} \]
    Alors :
    \begin{enumerate}[label=(\roman*)]
      \item $(E)$ admet une unique solution $x$ et pour tout point initial $x_0 \in [a, b]$, la suite des itérés $(x_n)$ définie par $\forall n \in \mathbb{N}, \, x_{n+1} = \varphi(x_n)$ converge vers $x$.
      \item La vitesse de convergence est estimée par la suite géométrique $\left( 1 - \frac{m}{M} \right)$ : il faut que les bornes $m$ et $M$ soient proches.
    \end{enumerate}
  \end{application}

  \begin{remark}
    Cela marche aussi dans le cas où $f(a) > 0$, $f(b) < 0$ et $-M \leq f'(x) \leq -m < 0$ (il suffit alors de changer $f$ en $-f$).
  \end{remark}

  \begin{definition}
    Soient $I$ un intervalle fermé de $\mathbb{R}$ et $\varphi : I \rightarrow I$ une application de classe $\mathcal{C}^1$. Soit $a \in I$ un point fixe de $\varphi$.
    \begin{itemize}
      \item Si $|\varphi'(a)| < 1$, on dit que $a$ est \textbf{attractif}. Si de plus $\varphi'(a) = 0$, $a$ est \textbf{superattractif}.
      \item Si $|\varphi'(a)| > 1$, on dit que $a$ est \textbf{répulsif}.
    \end{itemize}
  \end{definition}

  \begin{proposition}
    On reprend les notations précédentes et on considère la suite des itérés $(x_n)$ (avec $x_0 \in I$ et $\forall n \in \mathbb{N}, \, x_{n+1} = \varphi(x_n)$). Alors :
    \begin{enumerate}[label=(\roman*)]
      \item Si $a$ est attractif, $(x_n)$ converge à une vitesse géométrique :
      \[ |x_n - a| \leq k^n |x_0 - a| \]
      \item Si $a$ est superattractif et $\varphi$ est $\mathcal{C}^2$ telle que $|\varphi''| < M$ sur $I$, alors la vitesse de convergence est hypergéométrique :
      \[ |x_n - a| \leq \frac{2}{M} 10^{-2^n} \]
      \item Si $a$ est répulsif, il existe $h > 0$ tel que $\varphi_{|[a-h, a+h]}$ admette une application réciproque $\varphi^{-1}$ définie sur $\varphi([a-h, a+h])$ et le point $a$ est attractif pour $\varphi^{-1}$.
    \end{enumerate}
  \end{proposition}

  \begin{example}
    Soit $f : x \mapsto x^3 - 4x + 1$. On pose $\varphi : x \mapsto \frac{1}{4} (x^3 + 1)$ et on considère
    \[ f(x) = 0 \iff \varphi(x) = x \tag{$E$} \]
    Alors $(E)$ possède trois solutions réelles $a_1 < a_2 < a_3$ telles que :
    \begin{itemize}
      \item $a_1 \in ]-2,5; -2[$.
      \item $a_2 \in ]0; 0,5[$ et $a_2$ est attractif.
      \item $a_3 \in ]1,5; 2[$.
    \end{itemize}
  \end{example}

  \subsubsection{Méthode de Newton}

  \reference[ROU]{152}
  \dev{methode-de-newton}

  \begin{theorem}[Méthode de Newton]
    Soit $f : [c, d] \rightarrow \mathbb{R}$ une fonction de classe $\mathcal{C}^2$ strictement croissante sur $[c, d]$. On considère la fonction
    \[ \varphi :
    \begin{array}{ccc}
      [c, d] &\rightarrow& \mathbb{R} \\
      x &\mapsto& x - \frac{f(x)}{f'(x)}
    \end{array}
    \]
    (qui est bien définie car $f' > 0$). Alors :
    \begin{enumerate}[label=(\roman*)]
      \item $\exists! a \in [c, d]$ tel que $f(a) = 0$.
      \item $\exists \alpha > 0$ tel que $I = [a - \alpha, a + \alpha]$ est stable par $\varphi$.
      \item La suite $(x_n)$ des itérés (définie par récurrence par $x_{n+1} = \varphi(x_n)$ pour tout $n \geq 0$) converge quadratiquement vers $a$ pour tout $x_0 \in I$.
    \end{enumerate}
  \end{theorem}

  \begin{corollary}
    En reprenant les hypothèses et notations du théorème précédent, et en supposant de plus $f$ strictement convexe sur $[c, d]$, le résultat du théorème est vrai sur $I = [a, d]$. De plus :
    \begin{enumerate}[label=(\roman*)]
      \item $(x_n)$ est strictement décroissante (ou constante).
      \item $x_{n+1} - a \sim \frac{f''(a)}{2f'(a)} (x_n - a)^2$ pour $x_0 > a$.
    \end{enumerate}
  \end{corollary}

  \begin{example}
    \begin{itemize}
      \item On fixe $y > 0$. En itérant la fonction $F : x \mapsto \frac{1}{2} \left( x + \frac{y}{x} \right)$ pour un nombre de départ compris entre $c$ et $d$ où $0 < c < d$ et $c^2 < 0 < d^2$, on peut obtenir une approximation du nombre $\sqrt{y}$.
      \item En itérant la fonction $F : x \mapsto \frac{x^2+1}{2x-1}$ pour un nombre de départ supérieur à $2$, on peut obtenir une approximation du nombre d'or $\varphi = \frac{1+\sqrt{5}}{2}$.
    \end{itemize}
  \end{example}

  \reference[DEM]{102}

  \begin{example}
    La méthode de Newton appliquée à la fonction $x \mapsto x^3 - 4x + 1$ dans le but d'approximer ses zéros donne :
    \begin{center}
      \begin{whitetabularx}{|X|l|l|l|}
        \hline
        $x_0$ & $-2$ & $0$ & $2$ \\
        \hline
        $x_1$ & $-2,125$ & $0,25$ & $1,875$ \\
        \hline
        $x_2$ & $-2,114975450$ & $0,254098361$ & $1,860978520$ \\
        \hline
        $x_3$ & $-2,114907545$ & $0,254101688$ & $1,860805877$ \\
        \hline
        $x_4$ & $-2,114907541$ & $= x_3$ & $1,860805853$ \\
        \hline
        $x_5$ & $= x_4$ & & $= x_4$ \\
        \hline
      \end{whitetabularx}
    \end{center}
  \end{example}

  \subsubsection{Généralisation à $\mathbb{R}^m$}

  \reference{110}

  \begin{theorem}[Méthode de Newton-Raphson]
    Soit $f : \Omega \rightarrow \mathbb{R}^m$ (où $\Omega \subset \mathbb{R}^m$ est un ouvert) de classe $\mathcal{C}^1$ telle que $f(a) = 0$. On suppose que $\mathrm{d}f_a$ est inversible. Alors il existe un voisinage $U$ de $a$ dans $\Omega$ tel que $\varphi : x \mapsto x - (\mathrm{d}f_x)^{-1}(f(x))$ soit bien définie sur $U$ et la suite des itérés $x_{n+1} = \varphi(x_n)$ converge quadratiquement vers $a$.
  \end{theorem}

  \begin{example}
    On considère le système
    \[ \begin{cases} x^2 + xy - 2y^2 = 4 \\ xe^x + ye^y = 0 \end{cases} \tag{$S$} \]
    On pose $X_0 = \begin{pmatrix} -2 \\ 0,2 \end{pmatrix}$ et $\Delta(x,y) = (2x+y)(y+1)e^y - (x-4y)(x+1)e^x$ ainsi que :
    \[ \varphi \begin{pmatrix} x \\ y \end{pmatrix} = \begin{pmatrix} x \\ y \end{pmatrix} - \frac{1}{\Delta(x,y)} \begin{pmatrix} (y+1)e^y & -x+4y \\ -(x+1)e^x & 2x+y \end{pmatrix} \begin{pmatrix} x^2 + xy - 2y^2 - 4 \\ xe^x + ye^y \end{pmatrix} \]
    Alors la suite des itérés $(X_n) = \begin{pmatrix} x_n \\ y_n \end{pmatrix}$ converge vers l'unique solution de $(S)$ et on a :
    \begin{center}
      \begin{whitetabularx}{|l|X|X|}
        \hline
        $n$ & $x_n$ & $y_n$ \\
        \hline
        $0$ & $-2$ & $0,2$ \\
        \hline
        $1$ & $-2,130690999$ & $0,205937784$ \\
        \hline
        $2$ & $-2,126935837$ & $0,206277868$ \\
        \hline
        $3$ & $-2,126932304$ & $0,206278156$ \\
        \hline
      \end{whitetabularx}
    \end{center}
  \end{example}

  \annexessection

  \reference[I-P]{389}

  \begin{figure}[h]
    \begin{center}
      \begin{tikzpicture}
        \coordinate (A) at (0:3);
        \coordinate (B) at (72:3);
        \coordinate (C) at (2*72:3);
        \coordinate (D) at (3*72:3);
        \coordinate (E) at (4*72:3);
        \coordinate (F) at (A);
        \foreach \i in {0,...,10} {
          \draw(A) node {$\bullet$};
          \draw(B) node {$\bullet$};
          \draw(C) node {$\bullet$};
          \draw(D) node {$\bullet$};
          \draw(E) node {$\bullet$};
          \draw[fill=cyan!60, fill opacity=0.2](A) -- (B) -- (C) -- (D) -- (E) -- (A);
          \coordinate (A) at ($(A)!0.5!(B)$);
          \coordinate (B) at ($(B)!0.5!(C)$);
          \coordinate (C) at ($(C)!0.5!(D)$);
          \coordinate (D) at ($(D)!0.5!(E)$);
          \coordinate (E) at ($(E)!0.5!(F)$);
          \coordinate (F) at (A);
        }
      \end{tikzpicture}
    \end{center}
    \caption{La suite de polygones.}
  \end{figure}
  %</content>
\end{document}
