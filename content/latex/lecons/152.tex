\documentclass[12pt, a4paper]{report}

% LuaLaTeX :

\RequirePackage{iftex}
\RequireLuaTeX

% Packages :

\usepackage[french]{babel}
%\usepackage[utf8]{inputenc}
%\usepackage[T1]{fontenc}
\usepackage[pdfencoding=auto, pdfauthor={Hugo Delaunay}, pdfsubject={Mathématiques}, pdfcreator={agreg.skyost.eu}]{hyperref}
\usepackage{amsmath}
\usepackage{amsthm}
%\usepackage{amssymb}
\usepackage{stmaryrd}
\usepackage{tikz}
\usepackage{tkz-euclide}
\usepackage{fourier-otf}
\usepackage{fontspec}
\usepackage{titlesec}
\usepackage{fancyhdr}
\usepackage{catchfilebetweentags}
\usepackage[french, capitalise, noabbrev]{cleveref}
\usepackage[fit, breakall]{truncate}
\usepackage[top=2.5cm, right=2cm, bottom=2.5cm, left=2cm]{geometry}
\usepackage{enumerate}
\usepackage{tocloft}
\usepackage{microtype}
%\usepackage{mdframed}
%\usepackage{thmtools}
\usepackage{xcolor}
\usepackage{tabularx}
\usepackage{aligned-overset}
\usepackage[subpreambles=true]{standalone}
\usepackage{environ}
\usepackage[normalem]{ulem}
\usepackage{marginnote}
\usepackage{etoolbox}
\usepackage{setspace}
\usepackage[bibstyle=reading, citestyle=draft]{biblatex}
\usepackage{xpatch}
\usepackage[many, breakable]{tcolorbox}
\usepackage[backgroundcolor=white, bordercolor=white, textsize=small]{todonotes}

% Bibliographie :

\newcommand{\overridebibliographypath}[1]{\providecommand{\bibliographypath}{#1}}
\overridebibliographypath{../bibliography.bib}
\addbibresource{\bibliographypath}
\defbibheading{bibliography}[\bibname]{%
	\newpage
	\section*{#1}%
}
\renewbibmacro*{entryhead:full}{\printfield{labeltitle}}
\DeclareFieldFormat{url}{\newline\footnotesize\url{#1}}
\AtEndDocument{\printbibliography}

% Police :

\setmathfont{Erewhon Math}

% Tikz :

\usetikzlibrary{calc}

% Longueurs :

\setlength{\parindent}{0pt}
\setlength{\headheight}{15pt}
\setlength{\fboxsep}{0pt}
\titlespacing*{\chapter}{0pt}{-20pt}{10pt}
\setlength{\marginparwidth}{1.5cm}
\setstretch{1.1}

% Métadonnées :

\author{agreg.skyost.eu}
\date{\today}

% Titres :

\setcounter{secnumdepth}{3}

\renewcommand{\thechapter}{\Roman{chapter}}
\renewcommand{\thesubsection}{\Roman{subsection}}
\renewcommand{\thesubsubsection}{\arabic{subsubsection}}
\renewcommand{\theparagraph}{\alph{paragraph}}

\titleformat{\chapter}{\huge\bfseries}{\thechapter}{20pt}{\huge\bfseries}
\titleformat*{\section}{\LARGE\bfseries}
\titleformat{\subsection}{\Large\bfseries}{\thesubsection \, - \,}{0pt}{\Large\bfseries}
\titleformat{\subsubsection}{\large\bfseries}{\thesubsubsection. \,}{0pt}{\large\bfseries}
\titleformat{\paragraph}{\bfseries}{\theparagraph. \,}{0pt}{\bfseries}

\setcounter{secnumdepth}{4}

% Table des matières :

\renewcommand{\cftsecleader}{\cftdotfill{\cftdotsep}}
\addtolength{\cftsecnumwidth}{10pt}

% Redéfinition des commandes :

\renewcommand*\thesection{\arabic{section}}
\renewcommand{\ker}{\mathrm{Ker}}

% Nouvelles commandes :

\newcommand{\website}{https://agreg.skyost.eu}

\newcommand{\tr}[1]{\mathstrut ^t #1}
\newcommand{\im}{\mathrm{Im}}
\newcommand{\rang}{\operatorname{rang}}
\newcommand{\trace}{\operatorname{trace}}
\newcommand{\id}{\operatorname{id}}
\newcommand{\stab}{\operatorname{Stab}}

\providecommand{\newpar}{\\[\medskipamount]}

\providecommand{\lesson}[3]{%
	\title{#3}%
	\hypersetup{pdftitle={#3}}%
	\setcounter{section}{\numexpr #2 - 1}%
	\section{#3}%
	\fancyhead[R]{\truncate{0.73\textwidth}{#2 : #3}}%
}

\providecommand{\development}[3]{%
	\title{#3}%
	\hypersetup{pdftitle={#3}}%
	\section*{#3}%
	\fancyhead[R]{\truncate{0.73\textwidth}{#3}}%
}

\providecommand{\summary}[1]{%
	\textit{#1}%
	\medskip%
}

\tikzset{notestyleraw/.append style={inner sep=0pt, rounded corners=0pt, align=center}}

%\newcommand{\booklink}[1]{\website/bibliographie\##1}
\newcommand{\citelink}[2]{\hyperlink{cite.\therefsection @#1}{#2}}
\newcommand{\previousreference}{}
\providecommand{\reference}[2][]{%
	\notblank{#1}{\renewcommand{\previousreference}{#1}}{}%
	\todo[noline]{%
		\protect\vspace{16pt}%
		\protect\par%
		\protect\notblank{#1}{\cite{[\previousreference]}\\}{}%
		\protect\citelink{\previousreference}{p. #2}%
	}%
}

\definecolor{devcolor}{HTML}{00695c}
\newcommand{\dev}[1]{%
	\reversemarginpar%
	\todo[noline]{
		\protect\vspace{16pt}%
		\protect\par%
		\bfseries\color{devcolor}\href{\website/developpements/#1}{DEV}
	}%
	\normalmarginpar%
}

% En-têtes :

\pagestyle{fancy}
\fancyhead[L]{\truncate{0.23\textwidth}{\thepage}}
\fancyfoot[C]{\scriptsize \href{\website}{\texttt{agreg.skyost.eu}}}

% Couleurs :

\definecolor{property}{HTML}{fffde7}
\definecolor{proposition}{HTML}{fff8e1}
\definecolor{lemma}{HTML}{fff3e0}
\definecolor{theorem}{HTML}{fce4f2}
\definecolor{corollary}{HTML}{ffebee}
\definecolor{definition}{HTML}{ede7f6}
\definecolor{notation}{HTML}{f3e5f5}
\definecolor{example}{HTML}{e0f7fa}
\definecolor{cexample}{HTML}{efebe9}
\definecolor{application}{HTML}{e0f2f1}
\definecolor{remark}{HTML}{e8f5e9}
\definecolor{proof}{HTML}{e1f5fe}

% Théorèmes :

\theoremstyle{definition}
\newtheorem{theorem}{Théorème}

\newtheorem{property}[theorem]{Propriété}
\newtheorem{proposition}[theorem]{Proposition}
\newtheorem{lemma}[theorem]{Lemme}
\newtheorem{corollary}[theorem]{Corollaire}

\newtheorem{definition}[theorem]{Définition}
\newtheorem{notation}[theorem]{Notation}

\newtheorem{example}[theorem]{Exemple}
\newtheorem{cexample}[theorem]{Contre-exemple}
\newtheorem{application}[theorem]{Application}

\theoremstyle{remark}
\newtheorem{remark}[theorem]{Remarque}

\counterwithin*{theorem}{section}

\newcommand{\applystyletotheorem}[1]{
	\tcolorboxenvironment{#1}{
		enhanced,
		breakable,
		colback=#1!98!white,
		boxrule=0pt,
		boxsep=0pt,
		left=8pt,
		right=8pt,
		top=8pt,
		bottom=8pt,
		sharp corners,
		after=\par,
	}
}

\applystyletotheorem{property}
\applystyletotheorem{proposition}
\applystyletotheorem{lemma}
\applystyletotheorem{theorem}
\applystyletotheorem{corollary}
\applystyletotheorem{definition}
\applystyletotheorem{notation}
\applystyletotheorem{example}
\applystyletotheorem{cexample}
\applystyletotheorem{application}
\applystyletotheorem{remark}
\applystyletotheorem{proof}

% Environnements :

\NewEnviron{whitetabularx}[1]{%
	\renewcommand{\arraystretch}{2.5}
	\colorbox{white}{%
		\begin{tabularx}{\textwidth}{#1}%
			\BODY%
		\end{tabularx}%
	}%
}

% Maths :

\DeclareFontEncoding{FMS}{}{}
\DeclareFontSubstitution{FMS}{futm}{m}{n}
\DeclareFontEncoding{FMX}{}{}
\DeclareFontSubstitution{FMX}{futm}{m}{n}
\DeclareSymbolFont{fouriersymbols}{FMS}{futm}{m}{n}
\DeclareSymbolFont{fourierlargesymbols}{FMX}{futm}{m}{n}
\DeclareMathDelimiter{\VERT}{\mathord}{fouriersymbols}{152}{fourierlargesymbols}{147}


% Bibliographie :

\addbibresource{\bibliographypath}%
\defbibheading{bibliography}[\bibname]{%
	\newpage
	\section*{#1}%
}
\renewbibmacro*{entryhead:full}{\printfield{labeltitle}}%
\DeclareFieldFormat{url}{\newline\footnotesize\url{#1}}%

\AtEndDocument{\printbibliography}

\begin{document}
	%<*content>
	\lesson{algebra}{152}{Endomorphismes diagonalisables en dimension finie.}

	Soit $E$ un espace vectoriel sur un corps $\mathbb{K}$ de dimension finie $n$. Soit $u \in \mathcal{L}(E)$ un endomorphisme de $E$.

	\subsection{Spectre d'un endomorphisme}

	\subsubsection{Valeurs propres, vecteurs propres}

	\reference[GOU21]{171}

	\begin{definition}
		Soit $\lambda \in \mathbb{K}$.
		\begin{itemize}
			\item On dit que $\lambda$ est \textbf{valeur propre} de $u$ si $u - \lambda \operatorname{id}_E$ est non injective.
			\item Un vecteur $x \neq 0$ tel que $u(x) = \lambda x$ est un \textbf{vecteur propre} de $u$ associé à la valeur propre $\lambda$.
			\item $E_\lambda = \ker(u - \lambda \operatorname{id}_E)$ est le \textbf{sous-espace propre} associé à la valeur propre $\lambda$.
			\item L'ensemble des valeurs propres de $u$ est appelé \textbf{spectre} de $u$. On le note $\operatorname{Sp}(u)$.
		\end{itemize}
	\end{definition}

	\begin{remark}
		\begin{itemize}
			\item $0$ est valeur propre de $u$ si et seulement si $\ker(f) \neq \{ 0 \}$.
			\item On peut définir de la même manière les mêmes notions pour une matrice de $\mathcal{M}_n(\mathbb{K})$ (une valeur est propre pour une matrice si et seulement si elle l'est pour l'endomorphisme associé). On reprendra les mêmes notations.
			\item Les sous-espaces $E_\lambda$ sont stables par $u$ pour toute valeur propre $\lambda$.
		\end{itemize}
	\end{remark}

	\begin{example}
		$\begin{pmatrix} 1 \\ 1 \\ 1 \end{pmatrix}$ est vecteur propre de $\begin{pmatrix} 0 & 2 & -1 \\ 3 & -2 & 0 \\ -2 & 2 & 1 \end{pmatrix}$ associé à la valeur propre $1$.
	\end{example}

	\begin{theorem}
		Soient $\lambda_1, \dots, \lambda_k$ des valeurs propres de $u$, distinctes deux à deux. Alors les sous-espaces propres $E_{\lambda_1}, \dots, E_{\lambda_k}$ sont en somme directe.
	\end{theorem}

	\reference[ROM21]{604}

	\begin{theorem}
		Soit $P \in \mathbb{K}[X]$. Pour tout valeur propre $\lambda$ de $u$, $P(\lambda)$ est une valeur propre de $P(u)$. Si le corps $\mathbb{K}$ est algébriquement clos, on a alors
		\[ \operatorname{Sp}(P(u)) = \{ P(\lambda) \mid \lambda \in \operatorname{Sp}(u) \} \]
	\end{theorem}

	\begin{cexample}
		Pour $A = \begin{pmatrix} 0 & -1 \\ 1 & 0 \end{pmatrix} \in \mathcal{M}_2(\mathbb{R})$ et $P = X^2$, on a $A^2 = -I_2$ et $\operatorname{Sp}(A) = \emptyset$.
	\end{cexample}

	\subsubsection{Polynôme caractéristique}

	\reference{644}

	\begin{proposition}
		En notant $\chi_u = \det(X \operatorname{id}_E - u)$,
		\[ \operatorname{Sp}(u) = \{ \lambda \in \mathbb{K} \mid \chi_u(\lambda) = 0 \} \]
	\end{proposition}

	\begin{definition}
		Le polynôme $\chi_u$ précédent est appelé \textbf{polynôme caractéristique} de $u$.
	\end{definition}

	\begin{remark}
		On peut définir la même notion pour une matrice $A \in \mathcal{M}_n(\mathbb{K})$, ces deux notions coïncidant bien si $A$ est la matrice de $u$ dans une base quelconque de $E$.
	\end{remark}

	\begin{example}
		Pour $A = \begin{pmatrix} a & b \\ c & d \end{pmatrix} \in \mathcal{M}_2(\mathbb{K})$, on a $\chi_A = X^2 - \trace(A)X + \det(A)$.
	\end{example}

	\begin{proposition}
		Soit $\lambda$ une valeur propre de $u$ de multiplicité $\alpha$ en tant que racine de $\chi_u$. Alors,
		\[ \dim(E_\lambda) \in \llbracket 1, \alpha \rrbracket \]
	\end{proposition}

	\reference[GOU21]{172}

	\begin{proposition}
		\begin{enumerate}[label=(\roman*)]
			\item Le polynôme caractéristique est un invariant de similitude.
			\item Soit $A \in \mathcal{M}_n(\mathbb{K})$. On note $\chi_A = \sum_{k=0}^n a_k X^k$. Alors, $a_0 = \det(A)$ et $a_{n-1} = \trace(A)$ (à un signe près).
		\end{enumerate}
	\end{proposition}

	\subsubsection{Polynôme minimal}

	\reference[ROM21]{604}

	\begin{lemma}
		\begin{enumerate}[label=(\roman*)]
			\item $\mathrm{Ann}(u) = \{ P \in \mathbb{K}[X] \mid P(u) = 0 \}$ est un sous-ensemble de $\mathbb{K}[u]$ non réduit au polynôme nul.
			\item $\mathrm{Ann}(u)$ est le noyau de $P \mapsto P(u)$ : c'est un idéal de $\mathbb{K}[u]$.
			\item Il existe un unique polynôme unitaire engendrant cet idéal.
		\end{enumerate}
	\end{lemma}

	\begin{definition}
		On appelle \textbf{idéal annulateur} de $u$ l'idéal $\mathrm{Ann}(u)$. Le polynôme unitaire générateur est noté $\pi_u$ et est appelé \textbf{polynôme minimal} de $u$.
	\end{definition}

	\begin{remark}
		\begin{itemize}
			\item $\pi_u$ est le polynôme unitaire de plus petit degré annulant $u$.
			\item Si $A \in \mathcal{M}_n(\mathbb{K})$ est la matrice de $u$ dans une base de $E$, on a $\mathrm{Ann}(u) = \mathrm{Ann}(A)$ et $\pi_u = \pi_A$.
		\end{itemize}
	\end{remark}

	\begin{example}
		Un endomorphisme est nilpotent d'indice $q$ si et seulement si son polynôme minimal est $X^q$.
	\end{example}

	\begin{proposition}
		Soit $F$ un sous-espace vectoriel de $E$ stable par $u$. Alors, le polynôme minimal de l'endomorphisme $u_{|F} : F \rightarrow F$ divise $\pi_u$.
	\end{proposition}

	\begin{proposition}
		\begin{enumerate}[label=(\roman*)]
			\item Les valeurs propres de $u$ sont racines de tout polynôme annulateur.
			\item Les valeurs propres de $u$ sont exactement les racines de $\pi_u$.
		\end{enumerate}
	\end{proposition}

	\reference[GOU21]{186}

	\begin{remark}
		$\pi_u$ et $\chi_u$ partagent dont les mêmes racines.
	\end{remark}

	\reference[ROM21]{607}

	\begin{theorem}[Cayley-Hamilton]
		\[ \pi_u \mid \chi_u \]
	\end{theorem}

	\begin{corollary}
		\[ \dim(\mathbb{K}[u]) \leq n \]
	\end{corollary}

	\subsection{Diagonalisabilité}

	\subsubsection{Définition}

	\reference{683}

	\begin{definition}
		\begin{itemize}
			\item On dit que $u$ est \textbf{diagonalisable} s'il existe une base de $E$ dans laquelle la matrice de $u$ est diagonale.
			\item On dit qu'une matrice $A \in \mathcal{M}_n(\mathbb{K})$ est \textbf{diagonalisable} si elle est semblable à une matrice diagonale.
		\end{itemize}
	\end{definition}

	\begin{remark}
		$u$ est diagonalisable si et seulement si sa matrice dans n'importe quelle base de $E$ l'est.
	\end{remark}

	\reference[BMP]{166}

	\begin{example}
		\begin{itemize}
			\item Les projecteurs (ie. les endomorphismes $p \in \mathcal{L}(E)$ tels que $p^2 = p$) sont toujours diagonalisables, à valeurs propres dans $\{ 0, 1 \}$.
			\item Les symétries (ie. les endomorphismes $s \in \mathcal{L}(E)$ tels que $s^2 = \operatorname{id}_E$) sont toujours diagonalisables, à valeurs propres dans $\{ \pm 1 \}$. Par exemple, l'endomorphisme de transposition $A \mapsto \tr{A}$ est diagonalisable.
		\end{itemize}
	\end{example}

	\subsubsection{Critères}

	\reference[ROM21]{683}

	\begin{proposition}
		Si $u$ a $n$ valeurs propres distinctes dans $\mathbb{K}$, alors il est diagonalisable.
	\end{proposition}

	\reference{609}

	\begin{theorem}[Lemme des noyaux]
		Soit $P = P_1 \dots P_k \in \mathbb{K}[X]$ où les polynômes $P_1, \dots, P_k$ sont premiers entre eux deux à deux. Alors,
		\[ \ker(P(u)) = \bigoplus_{i=1}^k \ker(P_i(u)) \]
	\end{theorem}

	\reference{683}

	\begin{theorem}
		Soit $\operatorname{Sp}(u) = \{ \lambda_1, \dots, \lambda_p \}$. Les assertions suivantes sont équivalentes :
		\begin{enumerate}[label=(\roman*)]
			\item $u$ est diagonalisable sur $\mathbb{K}$.
			\item $E = \bigoplus_{k=1}^p E_{\lambda_k}$.
			\item $\sum_{k=1}^p \dim(E_{\lambda_k}) = n$.
			\item $\chi_n$ est scindé sur $\mathbb{K}$ et pour tout $k \in \llbracket 1, p \rrbracket$, la dimension de $E_{\lambda_k}$ est égale à la multiplicité de $\lambda_k$ dans $\chi_u$.
			\item $\exists P \in \mathrm{Ann}(u)$ scindé à racines simples.
			\item $\pi_u$ est scindé à racines simples.
		\end{enumerate}
	\end{theorem}

	\reference[GOU21]{177}

	\begin{example}
		$\begin{pmatrix} 0 & 2 & -1 \\ 3 & -2 & 0 \\ -2 & 2 & 1 \end{pmatrix}$ est diagonalisable, semblable à $\begin{pmatrix} 1 & 0 & 0 \\ 0 & 2 & 0 \\ 0 & 0 & -4 \end{pmatrix}$.
	\end{example}

	\reference{188}

	\begin{corollary}
		Sur $\mathbb{K} = \mathbb{F}_q$, $u$ est diagonalisable si et seulement si $u^q = u$.
	\end{corollary}

	\reference{176}

	\begin{theorem}[Diagonalisation simultanée]
		Soit $(u_i)_{i \in I}$ une famille d'endomorphismes de $E$ diagonalisables. Il existe une base commune de diagonalisation dans $E$ pour $(u_i)_{i \in I}$ si et seulement si ces endomorphismes commutent deux-à-deux.
	\end{theorem}

	\begin{remark}
		La réciproque est vraie.
	\end{remark}

	\subsubsection{Exemples d'endomorphismes diagonalisables dans un espace euclidien ou hermitien}

	On se place dans le cas où $\mathbb{K} = \mathbb{R}$ ou $\mathbb{C}$.
	Si $\mathbb{K} = \mathbb{R}$, on munit $E$ d'un produit scalaire $\langle ., . \rangle$. Si $\mathbb{K} = \mathbb{C}$, on munit $E$ d'un produit scalaire hermitien $\langle ., . \rangle$.

	\paragraph{Endomorphismes autoadjoints}

	\reference[GOU21]{255}

	\begin{lemma}
		Il existe un unique $u^* \in \mathcal{L}(E)$ tel que
		\[ \forall x, y \in E, \, \langle u(x), y \rangle = \langle x, u^*(y) \rangle \]
	\end{lemma}

	\begin{definition}
		L'endomorphisme $u^*$ précédent est \textbf{l'adjoint} de $u$. On dit que $u$ est \textbf{autoadjoint} si $u = u^*$.
	\end{definition}

	\begin{proposition}
		Soit $v \in \mathcal{L}(E)$. Alors $v = u^*$ si et seulement si la matrice de $v$ dans une base orthonormée $\mathcal{B}$ de $E$ est la transposée (transconjuguée dans le cas hermitien) de la matrice de $u$ dans $\mathcal{B}$.
	\end{proposition}

	\begin{theorem}
		Tout endomorphisme autoadjoint se diagonalise dans une base orthonormée, ses valeurs propres étant réelles.
	\end{theorem}

	\reference[C-G]{376}

	\begin{lemma}
		\[ \forall A \in \mathcal{S}_n^{++}(\mathbb{R}) \, \exists! B \in \mathcal{S}_n^{++}(\mathbb{R}) \text{ telle que } B^2 = A \]
	\end{lemma}

	\dev{decomposition-polaire}

	\begin{application}[Décomposition polaire]
		L'application
		\[ \mu :
		\begin{array}{ccc}
			\mathcal{O}_n(\mathbb{R}) \times \mathcal{S}_n^{++}(\mathbb{R}) &\rightarrow& \mathrm{GL}_n(\mathbb{R}) \\
			(O, S) &\mapsto& OS
		\end{array}
		\]
		est un homéomorphisme.
	\end{application}

	\paragraph{Endomorphismes normaux}

	On suppose dans toute cette sous-section que $\mathbb{K} = \mathbb{C}$.

	\reference[GRI]{286}

	\begin{definition}
		$u$ est dit \textbf{normal} s'il est tel que $u \circ u^* = u^* \circ u$.
	\end{definition}

	\begin{proposition}
		On suppose $u$ normal. Soit $\lambda \in \mathbb{C}$ une valeur propre de $u$. Alors :
		\begin{enumerate}[label=(\roman*)]
			\item $E_\lambda^\perp = \{ x \in E^\lambda \mid \forall y \in E^\lambda, \, \langle x, y \rangle = 0 \}$ est stable par $u$.
			\item $u_{| E_\lambda^\perp}$ est normal.
		\end{enumerate}
	\end{proposition}

	\begin{corollary}
		On suppose $u$ normal. Alors $u$ est diagonalisable dans une base orthonormée.
	\end{corollary}

	\subsubsection{Topologie}

	\reference[BMP]{179}

	\begin{proposition}
		L'ensemble $\mathcal{D}_n(\mathbb{C})$ des matrices diagonalisables à coefficients complexes est dense dans $\mathcal{M}_n(\mathbb{C})$.
	\end{proposition}

	\begin{application}
		L'application qui à une matrice $M \in \mathcal{M}_n(\mathbb{C})$ associe la partie diagonalisable de sa décomposition de Dunford $M = D + N$ n'est pas continue.
	\end{application}

	\reference{217}

	\begin{application}
		\[ \forall U \in \mathcal{M}_n(\mathbb{C}) \, \chi_U(U) = 0 \]
	\end{application}

	\subsection{Applications}

	\subsubsection{Réduction}

	\reference[GOU21]{203}
	\dev{decomposition-de-dunford}

	\begin{theorem}[Décomposition de Dunford]
		On suppose que $\pi_u$ est scindé sur $\mathbb{K}$. Alors il existe un unique couple d'endomorphismes $(d, n)$ tels que :
		\begin{itemize}
			\item $d$ est diagonalisable et $n$ est nilpotent.
			\item $u = d + n$.
			\item $d n = n d$.
		\end{itemize}
	\end{theorem}

	\begin{corollary}
		Si $u$ vérifie les hypothèse précédentes, pour tout $k \in \mathbb{N}$, $u^k = (d + n)^k = \sum_{i=0}^m \binom{k}{i} d^i n^{k-i}$, avec $m = \min(k, l)$ où $l$ désigne l'indice de nilpotence de $n$.
	\end{corollary}

	\begin{remark}
		On peut montrer de plus que $d$ et $n$ sont des polynômes en $u$.
	\end{remark}
	
	\newpage

	\subsubsection{Calcul d'exponentielles}

	\reference[ROM21]{761}

	\begin{lemma}
		\begin{enumerate}[label=(\roman*)]
			\item La série entière $\sum \frac{z^k}{k!}$ a un rayon de convergence infini.
			\item $\sum \frac{A^k}{k!}$ est convergente pour toute matrice $A \in \mathcal{M}_n(\mathbb{K})$.
		\end{enumerate}
	\end{lemma}

	\begin{definition}
		Soit $A \in \mathcal{M}_n(\mathbb{K})$. On définit \textbf{l'exponentielle} de $A$ par
		\[ \sum_{k=0}^{+\infty} \frac{A^k}{k!} \]
		on la note aussi $\exp(A)$ ou $e^A$.
	\end{definition}

	\begin{theorem}
		Soit $A \in \mathcal{M}_n(\mathbb{K})$.
		\begin{enumerate}[label=(\roman*)]
			\item Si $A = \operatorname{Diag}(\lambda_1, \dots, \lambda_n)$, alors $\exp(A) = \operatorname{Diag}(e^\lambda_1, \dots, e^\lambda_n)$.
			\item Si $B = PAP^{-1}$ pour $P \in \mathrm{GL}_n(\mathbb{K})$, alors $e^B = P^{-1} e^A P$.
			\item $\det(e^A) = e^{\trace(A)}$.
			\item $t \mapsto e^{tA}$ est de classe $\mathcal{C}^\infty$, de dérivée $t \mapsto e^{tA}A$.
		\end{enumerate}
	\end{theorem}

	\begin{proposition}
		Soient $A, B \in \mathcal{M}_n(\mathbb{K})$ qui commutent. Alors,
		\[ e^A e^B = e^{A+B} = e^B e^A \]
	\end{proposition}

	\begin{example}
		Soit $A \in \mathcal{M}_n(\mathbb{K})$ qui admet une décomposition de Dunford $A = D+N$ où $D$ est diagonalisable et $N$ est nilpotente d'indice $q$. Alors,
		\begin{itemize}
			\item $e^A = e^D e^N = e^D \sum_{k=0}^{q-1} \frac{N^k}{k!}$.
			\item La décomposition de Dunford de $e^A$ est $e^A = e^D + e^D(e^N - I_n)$ avec $e^D$ diagonalisable et $e^D(e^N - I_n)$ nilpotente.
		\end{itemize}
	\end{example}

	\begin{application}
	  Soit $A \in \mathcal{M}_n(\mathbb{K})$ dont le polynôme caractéristique est scindé sur $\mathbb{K}$. Alors $A$ est diagonalisable si et seulement si $e^A$ l'est.
	\end{application}

	\reference[GOU20]{380}

	\begin{application}
		Une équation différentielle linéaire homogène $(H) : Y' = AY$ (où $A$ est constante en $t$) a ses solutions maximales définies sur $\mathbb{R}$ et le problème de Cauchy
		\[ \begin{cases} Y' = AY \\ Y(0) = y_0 \end{cases} \]
		a pour (unique) solution $t \mapsto e^{tA} y_0$.
	\end{application}
	%</content>
\end{document}
