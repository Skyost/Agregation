\documentclass[12pt, a4paper]{report}

% LuaLaTeX :

\RequirePackage{iftex}
\RequireLuaTeX

% Packages :

\usepackage[french]{babel}
%\usepackage[utf8]{inputenc}
%\usepackage[T1]{fontenc}
\usepackage[pdfencoding=auto, pdfauthor={Hugo Delaunay}, pdfsubject={Mathématiques}, pdfcreator={agreg.skyost.eu}]{hyperref}
\usepackage{amsmath}
\usepackage{amsthm}
%\usepackage{amssymb}
\usepackage{stmaryrd}
\usepackage{tikz}
\usepackage{tkz-euclide}
\usepackage{fourier-otf}
\usepackage{fontspec}
\usepackage{titlesec}
\usepackage{fancyhdr}
\usepackage{catchfilebetweentags}
\usepackage[french, capitalise, noabbrev]{cleveref}
\usepackage[fit, breakall]{truncate}
\usepackage[top=2.5cm, right=2cm, bottom=2.5cm, left=2cm]{geometry}
\usepackage{enumerate}
\usepackage{tocloft}
\usepackage{microtype}
%\usepackage{mdframed}
%\usepackage{thmtools}
\usepackage{xcolor}
\usepackage{tabularx}
\usepackage{aligned-overset}
\usepackage[subpreambles=true]{standalone}
\usepackage{environ}
\usepackage[normalem]{ulem}
\usepackage{marginnote}
\usepackage{etoolbox}
\usepackage{setspace}
\usepackage[bibstyle=reading, citestyle=draft]{biblatex}
\usepackage{xpatch}
\usepackage[many, breakable]{tcolorbox}
\usepackage[backgroundcolor=white, bordercolor=white, textsize=small]{todonotes}

% Bibliographie :

\newcommand{\overridebibliographypath}[1]{\providecommand{\bibliographypath}{#1}}
\overridebibliographypath{../bibliography.bib}
\addbibresource{\bibliographypath}
\defbibheading{bibliography}[\bibname]{%
	\newpage
	\section*{#1}%
}
\renewbibmacro*{entryhead:full}{\printfield{labeltitle}}
\DeclareFieldFormat{url}{\newline\footnotesize\url{#1}}
\AtEndDocument{\printbibliography}

% Police :

\setmathfont{Erewhon Math}

% Tikz :

\usetikzlibrary{calc}

% Longueurs :

\setlength{\parindent}{0pt}
\setlength{\headheight}{15pt}
\setlength{\fboxsep}{0pt}
\titlespacing*{\chapter}{0pt}{-20pt}{10pt}
\setlength{\marginparwidth}{1.5cm}
\setstretch{1.1}

% Métadonnées :

\author{agreg.skyost.eu}
\date{\today}

% Titres :

\setcounter{secnumdepth}{3}

\renewcommand{\thechapter}{\Roman{chapter}}
\renewcommand{\thesubsection}{\Roman{subsection}}
\renewcommand{\thesubsubsection}{\arabic{subsubsection}}
\renewcommand{\theparagraph}{\alph{paragraph}}

\titleformat{\chapter}{\huge\bfseries}{\thechapter}{20pt}{\huge\bfseries}
\titleformat*{\section}{\LARGE\bfseries}
\titleformat{\subsection}{\Large\bfseries}{\thesubsection \, - \,}{0pt}{\Large\bfseries}
\titleformat{\subsubsection}{\large\bfseries}{\thesubsubsection. \,}{0pt}{\large\bfseries}
\titleformat{\paragraph}{\bfseries}{\theparagraph. \,}{0pt}{\bfseries}

\setcounter{secnumdepth}{4}

% Table des matières :

\renewcommand{\cftsecleader}{\cftdotfill{\cftdotsep}}
\addtolength{\cftsecnumwidth}{10pt}

% Redéfinition des commandes :

\renewcommand*\thesection{\arabic{section}}
\renewcommand{\ker}{\mathrm{Ker}}

% Nouvelles commandes :

\newcommand{\website}{https://agreg.skyost.eu}

\newcommand{\tr}[1]{\mathstrut ^t #1}
\newcommand{\im}{\mathrm{Im}}
\newcommand{\rang}{\operatorname{rang}}
\newcommand{\trace}{\operatorname{trace}}
\newcommand{\id}{\operatorname{id}}
\newcommand{\stab}{\operatorname{Stab}}

\providecommand{\newpar}{\\[\medskipamount]}

\providecommand{\lesson}[3]{%
	\title{#3}%
	\hypersetup{pdftitle={#3}}%
	\setcounter{section}{\numexpr #2 - 1}%
	\section{#3}%
	\fancyhead[R]{\truncate{0.73\textwidth}{#2 : #3}}%
}

\providecommand{\development}[3]{%
	\title{#3}%
	\hypersetup{pdftitle={#3}}%
	\section*{#3}%
	\fancyhead[R]{\truncate{0.73\textwidth}{#3}}%
}

\providecommand{\summary}[1]{%
	\textit{#1}%
	\medskip%
}

\tikzset{notestyleraw/.append style={inner sep=0pt, rounded corners=0pt, align=center}}

%\newcommand{\booklink}[1]{\website/bibliographie\##1}
\newcommand{\citelink}[2]{\hyperlink{cite.\therefsection @#1}{#2}}
\newcommand{\previousreference}{}
\providecommand{\reference}[2][]{%
	\notblank{#1}{\renewcommand{\previousreference}{#1}}{}%
	\todo[noline]{%
		\protect\vspace{16pt}%
		\protect\par%
		\protect\notblank{#1}{\cite{[\previousreference]}\\}{}%
		\protect\citelink{\previousreference}{p. #2}%
	}%
}

\definecolor{devcolor}{HTML}{00695c}
\newcommand{\dev}[1]{%
	\reversemarginpar%
	\todo[noline]{
		\protect\vspace{16pt}%
		\protect\par%
		\bfseries\color{devcolor}\href{\website/developpements/#1}{DEV}
	}%
	\normalmarginpar%
}

% En-têtes :

\pagestyle{fancy}
\fancyhead[L]{\truncate{0.23\textwidth}{\thepage}}
\fancyfoot[C]{\scriptsize \href{\website}{\texttt{agreg.skyost.eu}}}

% Couleurs :

\definecolor{property}{HTML}{fffde7}
\definecolor{proposition}{HTML}{fff8e1}
\definecolor{lemma}{HTML}{fff3e0}
\definecolor{theorem}{HTML}{fce4f2}
\definecolor{corollary}{HTML}{ffebee}
\definecolor{definition}{HTML}{ede7f6}
\definecolor{notation}{HTML}{f3e5f5}
\definecolor{example}{HTML}{e0f7fa}
\definecolor{cexample}{HTML}{efebe9}
\definecolor{application}{HTML}{e0f2f1}
\definecolor{remark}{HTML}{e8f5e9}
\definecolor{proof}{HTML}{e1f5fe}

% Théorèmes :

\theoremstyle{definition}
\newtheorem{theorem}{Théorème}

\newtheorem{property}[theorem]{Propriété}
\newtheorem{proposition}[theorem]{Proposition}
\newtheorem{lemma}[theorem]{Lemme}
\newtheorem{corollary}[theorem]{Corollaire}

\newtheorem{definition}[theorem]{Définition}
\newtheorem{notation}[theorem]{Notation}

\newtheorem{example}[theorem]{Exemple}
\newtheorem{cexample}[theorem]{Contre-exemple}
\newtheorem{application}[theorem]{Application}

\theoremstyle{remark}
\newtheorem{remark}[theorem]{Remarque}

\counterwithin*{theorem}{section}

\newcommand{\applystyletotheorem}[1]{
	\tcolorboxenvironment{#1}{
		enhanced,
		breakable,
		colback=#1!98!white,
		boxrule=0pt,
		boxsep=0pt,
		left=8pt,
		right=8pt,
		top=8pt,
		bottom=8pt,
		sharp corners,
		after=\par,
	}
}

\applystyletotheorem{property}
\applystyletotheorem{proposition}
\applystyletotheorem{lemma}
\applystyletotheorem{theorem}
\applystyletotheorem{corollary}
\applystyletotheorem{definition}
\applystyletotheorem{notation}
\applystyletotheorem{example}
\applystyletotheorem{cexample}
\applystyletotheorem{application}
\applystyletotheorem{remark}
\applystyletotheorem{proof}

% Environnements :

\NewEnviron{whitetabularx}[1]{%
	\renewcommand{\arraystretch}{2.5}
	\colorbox{white}{%
		\begin{tabularx}{\textwidth}{#1}%
			\BODY%
		\end{tabularx}%
	}%
}

% Maths :

\DeclareFontEncoding{FMS}{}{}
\DeclareFontSubstitution{FMS}{futm}{m}{n}
\DeclareFontEncoding{FMX}{}{}
\DeclareFontSubstitution{FMX}{futm}{m}{n}
\DeclareSymbolFont{fouriersymbols}{FMS}{futm}{m}{n}
\DeclareSymbolFont{fourierlargesymbols}{FMX}{futm}{m}{n}
\DeclareMathDelimiter{\VERT}{\mathord}{fouriersymbols}{152}{fourierlargesymbols}{147}


% Bibliographie :

\addbibresource{\bibliographypath}%
\defbibheading{bibliography}[\bibname]{%
	\newpage
	\section*{#1}%
}
\renewbibmacro*{entryhead:full}{\printfield{labeltitle}}%
\DeclareFieldFormat{url}{\newline\footnotesize\url{#1}}%

\AtEndDocument{\printbibliography}

\begin{document}
	%<*content>
	\lesson{analysis}{205}{Espaces complets. Exemples et applications.}

	\subsection{Complétude}

	\subsubsection{Complétude dans un espace métrique}

	\reference[GOU20]{20}

	Soit $(E,d)$ un espace métrique.

	\begin{definition}
		On dit qu'une suite $(x_n)$ d'éléments de $E$ est \textbf{de Cauchy} si
		\[ \forall \epsilon 0, \exists N \in \mathbb{N} \text{ tel que } \forall p > q \geq N, d(u_p, u_q) < \epsilon \]
	\end{definition}

	\begin{proposition}
		\begin{enumerate}[label=(\roman*)]
			\item Une suite convergente est de Cauchy.
			\item Une suite de Cauchy est bornée.
			\item Une suite de Cauchy qui possède une valeur d'adhérence $\ell$ converge vers $\ell$.
		\end{enumerate}
	\end{proposition}

	\reference[HAU]{312}

	\begin{cexample}
		La série $\sum \frac{1}{n}$ est une suite de Cauchy de $\mathbb{Q}$ non convergente dans $\mathbb{Q}$.
	\end{cexample}

	\reference[GOU20]{20}

	\begin{remark}
		La notion de suite de Cauchy n'est pas topologique : elle ne peut pas être définie à partir des ouverts de $E$. Cependant, si une suite est de Cauchy pour une certaine distance, alors elle l'est pour toute autre distance équivalente.
	\end{remark}

	\begin{definition}
		$E$ est \textbf{complet} si toute suite de Cauchy de $E$ converge dans $E$.
	\end{definition}

	\begin{example}
		$\forall n \in \mathbb{N}^*$, $\mathbb{R}^n$ est complet mais $\mathbb{Q}$ ne l'est pas.
	\end{example}

	\begin{proposition}
		\begin{enumerate}[label=(\roman*)]
			\item Toute partie complète d'un espace métrique est fermée.
			\item Toute partie fermée d'un espace complet est complète.
		\end{enumerate}
	\end{proposition}

	\begin{proposition}
		Soient $E_1, \dots, E_n$ des espaces métriques. Alors $E_1 \times \dots \times E_n$ est complet si et seulement si $\forall i \in \llbracket 1, n \rrbracket$, $E_i$ est complet.
	\end{proposition}

	\begin{proposition}[Fermés emboîtés]
		$E$ est complet si et seulement si toute suite décroissante de fermés non-vides de $E$ dont le diamètre converge vers $0$ converge vers un singleton.
	\end{proposition}

	\begin{proposition}[Critère de Cauchy pour les fonctions]
		Soit $(F, d')$ un espace métrique complet. Soient $f : A \rightarrow F$ où $A \subseteq E$ et $a \in \overline{A}$. Alors $f$ admet une limite quand $x$ tend vers $a$ si et seulement si
		\[ \forall \epsilon > 0, \, \exists \eta > 0 \text{ tel que } \forall x, y \in A, \, d(a,x) < \eta \text{ et } d(a,y) < \eta \implies d'(f(x), f(y)) < \epsilon \]
	\end{proposition}

	\reference{25}

	\begin{theorem}[Complété d'un espace métrique]
		Il existe un espace métrique complet $\widehat{E}$ et $i : E \rightarrow \widehat{E}$ une isométrie telle que $i(E)$ est dense dans $\widehat{E}$. De plus, $\widehat{E}$ est unique à isométrie bijective près.
	\end{theorem}

	\begin{example}
		$\mathbb{R}$ est le complété de $\mathbb{Q}$.
	\end{example}

	\subsubsection{Complétude dans un espace vectoriel normé}

	\reference{20}

	\begin{definition}
		Un espace vectoriel normé complet est un \textbf{espace de Banach}.
	\end{definition}

	\reference{52}

	\begin{proposition}
		Un espace vectoriel normé $E$ est complet si et seulement si toute série absolument convergence de $E$ est convergente dans $E$.
	\end{proposition}

	\begin{proposition}
		Un espace vectoriel de dimension finie est complet.
	\end{proposition}

	\reference[C-G]{379}

	\begin{application}
		L'exponentielle d'une matrice est un polynôme en la matrice.
	\end{application}

	\subsubsection{Exemples et contre-exemples classiques}

	\reference[DAN]{45}

	\begin{cexample}
		L'espace des fonctions polynômiales définies sur $[-1,1]$ et muni de la norme $\Vert . \Vert_{\infty}$ n'est pas complet.
	\end{cexample}

	\reference[GOU20]{21}

	\begin{example}
		Soient $X$ un ensemble et $E$ un espace de Banach. Alors, $(\mathcal{B}(X,E), \Vert . \Vert_\infty)$ est un espace de Banach.
	\end{example}

	\reference{8}

	\begin{example}
		Si $E$ est un espace vectoriel normé et $F$ est un espace de Banach, $(\mathcal{L}(E, F), \VERT . \VERT)$ est un espace de Banach.
	\end{example}

	\reference[LI]{7}

	\begin{definition}
		\begin{itemize}
			\item Pour $p \in [1, +\infty[$, on note $\mathcal{L}_p(X, \mathcal{A}, \mu))$ (où $\mathcal{L}_p$ en l'absence d'ambiguïté) l'espace des applications $f$ mesurables de $(X, \mathcal{A}, \mu)$ dans $(\mathbb{R}, \mathcal{B}(R))$ telles que
			\[ \int_X \vert f(x) \vert^p \, \mathrm{d}\mu(x) < +\infty \]
			on note alors $\Vert f \Vert_p = \left(\int_X \vert f(x) \vert^p \, \mathrm{d}\mu(x)\right)^{\frac{1}{p}}$.
			\item On note de même $\mathcal{L}_\infty$ l'espace des applications mesurables de $(X, \mathcal{A}, \mu)$ dans $(\mathbb{R}, \mathcal{B}(R))$ de sup-essentiel borné. On note alors $\Vert f \Vert_\infty$ pour $f \in \mathcal{L}_\infty$.
		\end{itemize}
	\end{definition}

	\begin{remark}
		En reprenant les notations précédentes, on a $\forall f \in \mathcal{L}_p$, $\Vert f \Vert_p = 0 \iff f = 0 \text{ pp.}$.
	\end{remark}

	\begin{theorem}[Inégalité de Minkowski]
		\[ \forall f, g \in \mathcal{L}_p, \, \Vert f + g \Vert_p \leq \Vert f \Vert_p + \Vert g \Vert_p \]
	\end{theorem}

	\begin{theorem}
		On définit pour tout $p \in [1, +\infty]$,
		\[ L_p = \mathcal{L}_p / V \]
		où $V = \{ v \in \mathcal{L}_p \mid v = 0 \text{ pp.} \}$. Muni de $\Vert . \Vert_p$, $L_p$ est un espace vectoriel normé.
	\end{theorem}

	\begin{theorem}[Riesz-Fischer]
		Pour tout $p \in [1, +\infty]$, $L_p$ est complet pour la norme $\Vert . \Vert_p$.
	\end{theorem}

	\subsection{Espaces de Hilbert}

	\subsubsection{Généralités}

	\reference{31}

	\begin{definition}
		Un espace vectoriel $H$ sur le corps $\mathbb{K}$ est un \textbf{espace de Hilbert} s'il est muni d'un produit scalaire $\langle . , . \rangle$ et est complet pour la norme associée $\Vert . \Vert = \sqrt{\langle . , . \rangle}$.
	\end{definition}

	\begin{example}
		Tout espace euclidien ou hermitien est un espace de Hilbert.
	\end{example}

	\begin{example}
		$L_2(\mu)$ muni de $\langle . , . \rangle : (f,g) \mapsto \int f \overline{g} \, \mathrm{d}\mu$ est un espace de Hilbert.
	\end{example}

	Pour toute la suite, on fixe $H$ un espace de Hilbert de norme $\Vert . \Vert$ et on note $\langle ., . \rangle$ le produit scalaire associé.

	\begin{lemma}[Identité du parallélogramme]
		\[ \forall x, y \in H, \, \Vert x + y \Vert^2 + \Vert x - y \Vert^2 = 2(\Vert x \Vert^2 \Vert y \Vert^2) \]
		et cette identité caractérise les normes issues d'un produit scalaire.
	\end{lemma}

	\dev{projection-sur-un-convexe-ferme}

	\begin{theorem}[Projection sur un convexe fermé]
		Soit $C \subseteq H$ un convexe fermé non-vide. Alors :
		\[ \forall x \in H, \exists! y \in C \text{ tel que } d(x, C) = \inf_{z \in C} \Vert x - z \Vert = d(x, y) \]
		On peut donc noter $y = P_C(x)$, le \textbf{projeté orthogonal de $x$ sur $C$}. Il s'agit de l'unique point de $C$ vérifiant
		\[ \forall z \in C, \, \langle x - P_C(x), z - P_C(x) \rangle \leq 0 \]
	\end{theorem}

	\begin{theorem}
		Si $F$ est un sous espace vectoriel fermé dans $H$, alors $P_F$ est une application linéaire continue. De plus, pour tout $x \in H$, $P_F(x)$ est l'unique point $y \in F$ tel que $x-y \in F^\perp$.
	\end{theorem}

	\begin{corollary}
		Soit $F$ un sous-espace vectoriel de $H$. Alors,
		\[ \overline{F} = H \iff F^\perp = 0 \]
	\end{corollary}

	\begin{theorem}[Représentation de Riesz]
		\[ \forall \varphi \in H', \, \exists! y \in H, \text{ tel que } \forall x \in H, \, \varphi(x) = \langle x, y \rangle \]
		et de plus, $\VERT \varphi \VERT = \Vert y \Vert$.
	\end{theorem}

	\begin{corollary}
		\[ \forall T \in H', \, \exists! U \in H' \text{ tel que } \forall x, y \in H, \, \langle T(x), y \rangle = \langle x, U(y) \rangle \]
		On note alors $U = T^*$ : c'est \textbf{l'adjoint} de $T$. On a alors $\VERT T \VERT = \VERT T^* \VERT$.
	\end{corollary}

	\reference{65}

	\begin{example}[Opérateur de Voltera]
		On définit $T$ sur $H = L_2([0,1])$ par :
		\[
		T : \begin{array}{ccc}
			H &\rightarrow& H \\
			f &\mapsto& x \mapsto \int_{0}^{x} f(t) \, \mathrm{d}t
		\end{array}
		\]
		$T$ est une application linéaire continue et son adjoint $T^*$ est défini par :
		\[ T^* : g \mapsto \left(x \mapsto \int_x^1 g(t) \, \mathrm{d}t \right) \]
	\end{example}

	\reference[Z-Q]{216}
	\dev{dual-de-lp}

	\begin{theorem}
		L'application
		\[
		\varphi :
		\begin{array}{ll}
			L_q &\rightarrow (L_p)' \\
			g &\mapsto \left( \varphi_g : f \mapsto \int_X f g \, \mathrm{d}\mu \right)
		\end{array}
		\qquad \text{ où } \frac{1}{p} + \frac{1}{q} = 1
		\]
		est une isométrie linéaire surjective. C'est donc un isomorphisme isométrique.
	\end{theorem}

	\subsubsection{Bases hilbertiennes}

	\reference[LI]{43}

	\begin{definition}
		On dit que $(e_n) \in H^{\mathbb{N}}$ est une \textbf{base hilbertienne} de $H$ si
		\begin{itemize}
			\item $(e_n)$ est orthonormale.
			\item $(e_n)$ est totale.
		\end{itemize}
	\end{definition}

	\begin{example}
		$(t \mapsto e^{2\pi int})_{n \in \mathbb{Z}}$ est une base hilbertienne de $L_2([0,1])$.
	\end{example}

	\begin{theorem}
		Soit $(e_n)_{n \in \mathbb{N}}$ une base hilbertienne de $H$. Alors :
		\[ \forall x \in H, \, x = \sum_{n=0}^{+\infty} \langle x, e_n \rangle e_n \]
		On a de plus, pour tout $x, y \in H$, les formules de Parseval :
		\begin{itemize}
			\item $\Vert x \Vert^2 = \sum_{n=0}^{+\infty} \vert \langle x, e_n \rangle \vert^2$.
			\item $\langle x, y \rangle = \sum_{n=0}^{+\infty} \langle x, e_n \rangle \overline{\langle y, e_n \rangle}$.
		\end{itemize}
	\end{theorem}

	\reference[GOU20]{272}

	\begin{application}
		\[ \sum_{n = 1}^{+\infty} \frac{1}{n^4} = \frac{\pi^4}{90} \]
	\end{application}

	\subsection{Applications}

	\subsubsection{Point fixe}

	\reference{21}

	\begin{theorem}[Point fixe de Banach]
		Soient $(E,d)$ un espace métrique complet et $f : E \rightarrow E$ une application contractant (ie. $\exists k \in ]0,1[ \text{ tel que } \forall x, y \in E, \, d(f(x), f(y)) \leq d(x, y)$). Alors,
		\[ \exists! x \in E \text{ tel que } f(x) = x \]
		De plus la suite des itérés définie par $x_0 \in E$ et $\forall n \in \mathbb{N}, x_{n+1} = f(x_n)$ converge vers $x$.
	\end{theorem}

	\reference[GOU20]{374}
	\dev{theoreme-de-cauchy-lipschitz-local}

	\begin{application}[Théorème de Cauchy-Lipschitz local]
		Soit $E$ un espace de Banach sur $\mathbb{R}$ ou $\mathbb{C}$. Soient $I$ un intervalle de $\mathbb{R}$ et $\Omega$ un ouvert de $E$. Soit $F : I \times \Omega \rightarrow E$ une fonction continue et localement lipschitzienne en la seconde variable. Alors, pour tout $(t_0, y_0) \in I \times \Omega$, le problème de Cauchy
		\[ \begin{cases} y'=F(t,y) \\ y(t_0) = y_0 \end{cases} \tag{$C$} \]
		admet une unique solution maximale.
	\end{application}

	\subsubsection{Prolongement}

	\reference[DAN]{47}

	\begin{theorem}[Prolongement des applications uniformément continues]
		Soient $(E,d_E)$ et $(F,d_F)$ des espaces métriques. On suppose $F$ complet. Soient $A \subseteq E$ dense et $f : A \rightarrow F$ une application uniformément continue. Alors, il existe une unique application $\widehat{f} : E \rightarrow F$ uniformément continue et telle que $\widehat{f}_{|A} = f$.
	\end{theorem}

	\begin{corollary}
		Soient $(E,d_E)$ et $(F,d_F)$ des espaces métriques. On suppose $F$ complet. Soient $A \subseteq E$ dense et $f : A \rightarrow F$ une application $k$-lipschitzienne. Alors, il existe une unique application $\widehat{f} : E \rightarrow F$ $k$-lipschitzienne et telle que $\widehat{f}_{|A} = f$.
	\end{corollary}

	\begin{example}
		Une application dérivable sur un intervalle $]a,b[$ et de dérivée bornée est prolongeable par une application lipschitzienne sur $[a,b]$.
	\end{example}

	\reference[BMP]{106}

	\begin{application}[Théorème de Hahn-Banach analytique]
		Soient $H$ un espace de Hilbert et $F$ un sous-espace vectoriel de $H$. Soit $f \in F'$. Alors, il existe $\widehat{f} \in H'$ telle que $\widehat{f}_{|F} = f$ et $\VERT \widehat{f} \VERT_{H} = \VERT f \VERT_{F}$.
	\end{application}

	\reference[LI]{94}

	\begin{application}[Transformation de Fourier-Plancherel]
		La transformation de Fourier $\mathcal{F}$ définie sur $L_1(\mathbb{R}) \, \cap \, L_2(\mathbb{R})$ se prolonge de manière unique en un isomorphisme d'espaces de Hilbert de $L_2(\mathbb{R})$ sur lui-même.
	\end{application}

	\subsubsection{Théorème de Baire}

	\reference[LI]{111}

	\begin{theorem}[Baire]
		On suppose $E$ complet. Alors toute intersection d'ouvert denses est encore dense dans $E$.
	\end{theorem}

	\reference[GOU20]{419}

	\begin{application}
		Un espace vectoriel normé à base dénombrable n'est pas complet.
	\end{application}

	\reference[LI]{112}

	\begin{application}[Théorème de Banach-Steinhaus]
		Soient $(E, \Vert . \Vert_E)$ et $(F, \Vert . \Vert_F)$ deux espaces de Banach et $(T_i)_{i \in I}$ des applications linéaires continues telles que
		\[ \forall x \in E, \, \sup_{i \in I} \Vert T_i(x) \Vert_F < +\infty \]
		alors,
		\[ \sup_{i \in I} \VERT T_i \VERT < +\infty \]
	\end{application}

	\begin{application}[Théorème du graphe fermé]
		Soient $E$ et $F$ deux espaces de Banach et $T \in L(E,F)$. Si le graphe de $T$ :
		\[ \{ (x, T(x)) \mid x \in E \} \subseteq E \times F \]
		est fermé dans $E \times F$, alors $T$ est continue.
	\end{application}

	\begin{application}[Théorème de l'application ouverte]
		Soient $E$ et $F$ deux espaces de Banach et $T \in \mathcal{L}(E,F)$ surjective. Alors,
		\[ \exists c > 0, \, T\left(B_E(0,1)\right) \supseteq B_F(0,c) \]
	\end{application}

	\begin{corollary}[Théorème des isomorphismes de Banach]
		Soient $E$ et $F$ deux espaces de Banach et $T \in \mathcal{L}(E,F)$ bijective. Alors $T^{-1}$ est continue.
	\end{corollary}

	\begin{corollary}
		On suppose que $E$ est de Banach. Soient $E_1$ et $E_2$ deux supplémentaires algébriques fermés dans $E$. Alors les projections associées sur $E_1$ et $E_2$ sont continues.
	\end{corollary}
	%</content>
\end{document}
