\documentclass[12pt, a4paper]{report}

% LuaLaTeX :

\RequirePackage{iftex}
\RequireLuaTeX

% Packages :

\usepackage[french]{babel}
%\usepackage[utf8]{inputenc}
%\usepackage[T1]{fontenc}
\usepackage[pdfencoding=auto, pdfauthor={Hugo Delaunay}, pdfsubject={Mathématiques}, pdfcreator={agreg.skyost.eu}]{hyperref}
\usepackage{amsmath}
\usepackage{amsthm}
%\usepackage{amssymb}
\usepackage{stmaryrd}
\usepackage{tikz}
\usepackage{tkz-euclide}
\usepackage{fourier-otf}
\usepackage{fontspec}
\usepackage{titlesec}
\usepackage{fancyhdr}
\usepackage{catchfilebetweentags}
\usepackage[french, capitalise, noabbrev]{cleveref}
\usepackage[fit, breakall]{truncate}
\usepackage[top=2.5cm, right=2cm, bottom=2.5cm, left=2cm]{geometry}
\usepackage{enumerate}
\usepackage{tocloft}
\usepackage{microtype}
%\usepackage{mdframed}
%\usepackage{thmtools}
\usepackage{xcolor}
\usepackage{tabularx}
\usepackage{aligned-overset}
\usepackage[subpreambles=true]{standalone}
\usepackage{environ}
\usepackage[normalem]{ulem}
\usepackage{marginnote}
\usepackage{etoolbox}
\usepackage{setspace}
\usepackage[bibstyle=reading, citestyle=draft]{biblatex}
\usepackage{xpatch}
\usepackage[many, breakable]{tcolorbox}
\usepackage[backgroundcolor=white, bordercolor=white, textsize=small]{todonotes}

% Bibliographie :

\newcommand{\overridebibliographypath}[1]{\providecommand{\bibliographypath}{#1}}
\overridebibliographypath{../bibliography.bib}
\addbibresource{\bibliographypath}
\defbibheading{bibliography}[\bibname]{%
	\newpage
	\section*{#1}%
}
\renewbibmacro*{entryhead:full}{\printfield{labeltitle}}
\DeclareFieldFormat{url}{\newline\footnotesize\url{#1}}
\AtEndDocument{\printbibliography}

% Police :

\setmathfont{Erewhon Math}

% Tikz :

\usetikzlibrary{calc}

% Longueurs :

\setlength{\parindent}{0pt}
\setlength{\headheight}{15pt}
\setlength{\fboxsep}{0pt}
\titlespacing*{\chapter}{0pt}{-20pt}{10pt}
\setlength{\marginparwidth}{1.5cm}
\setstretch{1.1}

% Métadonnées :

\author{agreg.skyost.eu}
\date{\today}

% Titres :

\setcounter{secnumdepth}{3}

\renewcommand{\thechapter}{\Roman{chapter}}
\renewcommand{\thesubsection}{\Roman{subsection}}
\renewcommand{\thesubsubsection}{\arabic{subsubsection}}
\renewcommand{\theparagraph}{\alph{paragraph}}

\titleformat{\chapter}{\huge\bfseries}{\thechapter}{20pt}{\huge\bfseries}
\titleformat*{\section}{\LARGE\bfseries}
\titleformat{\subsection}{\Large\bfseries}{\thesubsection \, - \,}{0pt}{\Large\bfseries}
\titleformat{\subsubsection}{\large\bfseries}{\thesubsubsection. \,}{0pt}{\large\bfseries}
\titleformat{\paragraph}{\bfseries}{\theparagraph. \,}{0pt}{\bfseries}

\setcounter{secnumdepth}{4}

% Table des matières :

\renewcommand{\cftsecleader}{\cftdotfill{\cftdotsep}}
\addtolength{\cftsecnumwidth}{10pt}

% Redéfinition des commandes :

\renewcommand*\thesection{\arabic{section}}
\renewcommand{\ker}{\mathrm{Ker}}

% Nouvelles commandes :

\newcommand{\website}{https://agreg.skyost.eu}

\newcommand{\tr}[1]{\mathstrut ^t #1}
\newcommand{\im}{\mathrm{Im}}
\newcommand{\rang}{\operatorname{rang}}
\newcommand{\trace}{\operatorname{trace}}
\newcommand{\id}{\operatorname{id}}
\newcommand{\stab}{\operatorname{Stab}}

\providecommand{\newpar}{\\[\medskipamount]}

\providecommand{\lesson}[3]{%
	\title{#3}%
	\hypersetup{pdftitle={#3}}%
	\setcounter{section}{\numexpr #2 - 1}%
	\section{#3}%
	\fancyhead[R]{\truncate{0.73\textwidth}{#2 : #3}}%
}

\providecommand{\development}[3]{%
	\title{#3}%
	\hypersetup{pdftitle={#3}}%
	\section*{#3}%
	\fancyhead[R]{\truncate{0.73\textwidth}{#3}}%
}

\providecommand{\summary}[1]{%
	\textit{#1}%
	\medskip%
}

\tikzset{notestyleraw/.append style={inner sep=0pt, rounded corners=0pt, align=center}}

%\newcommand{\booklink}[1]{\website/bibliographie\##1}
\newcommand{\citelink}[2]{\hyperlink{cite.\therefsection @#1}{#2}}
\newcommand{\previousreference}{}
\providecommand{\reference}[2][]{%
	\notblank{#1}{\renewcommand{\previousreference}{#1}}{}%
	\todo[noline]{%
		\protect\vspace{16pt}%
		\protect\par%
		\protect\notblank{#1}{\cite{[\previousreference]}\\}{}%
		\protect\citelink{\previousreference}{p. #2}%
	}%
}

\definecolor{devcolor}{HTML}{00695c}
\newcommand{\dev}[1]{%
	\reversemarginpar%
	\todo[noline]{
		\protect\vspace{16pt}%
		\protect\par%
		\bfseries\color{devcolor}\href{\website/developpements/#1}{DEV}
	}%
	\normalmarginpar%
}

% En-têtes :

\pagestyle{fancy}
\fancyhead[L]{\truncate{0.23\textwidth}{\thepage}}
\fancyfoot[C]{\scriptsize \href{\website}{\texttt{agreg.skyost.eu}}}

% Couleurs :

\definecolor{property}{HTML}{fffde7}
\definecolor{proposition}{HTML}{fff8e1}
\definecolor{lemma}{HTML}{fff3e0}
\definecolor{theorem}{HTML}{fce4f2}
\definecolor{corollary}{HTML}{ffebee}
\definecolor{definition}{HTML}{ede7f6}
\definecolor{notation}{HTML}{f3e5f5}
\definecolor{example}{HTML}{e0f7fa}
\definecolor{cexample}{HTML}{efebe9}
\definecolor{application}{HTML}{e0f2f1}
\definecolor{remark}{HTML}{e8f5e9}
\definecolor{proof}{HTML}{e1f5fe}

% Théorèmes :

\theoremstyle{definition}
\newtheorem{theorem}{Théorème}

\newtheorem{property}[theorem]{Propriété}
\newtheorem{proposition}[theorem]{Proposition}
\newtheorem{lemma}[theorem]{Lemme}
\newtheorem{corollary}[theorem]{Corollaire}

\newtheorem{definition}[theorem]{Définition}
\newtheorem{notation}[theorem]{Notation}

\newtheorem{example}[theorem]{Exemple}
\newtheorem{cexample}[theorem]{Contre-exemple}
\newtheorem{application}[theorem]{Application}

\theoremstyle{remark}
\newtheorem{remark}[theorem]{Remarque}

\counterwithin*{theorem}{section}

\newcommand{\applystyletotheorem}[1]{
	\tcolorboxenvironment{#1}{
		enhanced,
		breakable,
		colback=#1!98!white,
		boxrule=0pt,
		boxsep=0pt,
		left=8pt,
		right=8pt,
		top=8pt,
		bottom=8pt,
		sharp corners,
		after=\par,
	}
}

\applystyletotheorem{property}
\applystyletotheorem{proposition}
\applystyletotheorem{lemma}
\applystyletotheorem{theorem}
\applystyletotheorem{corollary}
\applystyletotheorem{definition}
\applystyletotheorem{notation}
\applystyletotheorem{example}
\applystyletotheorem{cexample}
\applystyletotheorem{application}
\applystyletotheorem{remark}
\applystyletotheorem{proof}

% Environnements :

\NewEnviron{whitetabularx}[1]{%
	\renewcommand{\arraystretch}{2.5}
	\colorbox{white}{%
		\begin{tabularx}{\textwidth}{#1}%
			\BODY%
		\end{tabularx}%
	}%
}

% Maths :

\DeclareFontEncoding{FMS}{}{}
\DeclareFontSubstitution{FMS}{futm}{m}{n}
\DeclareFontEncoding{FMX}{}{}
\DeclareFontSubstitution{FMX}{futm}{m}{n}
\DeclareSymbolFont{fouriersymbols}{FMS}{futm}{m}{n}
\DeclareSymbolFont{fourierlargesymbols}{FMX}{futm}{m}{n}
\DeclareMathDelimiter{\VERT}{\mathord}{fouriersymbols}{152}{fourierlargesymbols}{147}


% Bibliographie :

\addbibresource{\bibliographypath}%
\defbibheading{bibliography}[\bibname]{%
	\newpage
	\section*{#1}%
}
\renewbibmacro*{entryhead:full}{\printfield{labeltitle}}%
\DeclareFieldFormat{url}{\newline\footnotesize\url{#1}}%

\AtEndDocument{\printbibliography}

\begin{document}
  %<*content>
  \lesson{analysis}{245}{Fonctions holomorphes et méromorphes sur un ouvert de \texorpdfstring{$\mathbb{C}$}{C}. Exemples et applications.}

  Soit $\Omega \subseteq \mathbb{C}$ un ouvert. Soit $f : \Omega \rightarrow \mathbb{C}$.

  \subsection{Dérivabilité au sens complexe}

  \reference[QUE]{76}

  \begin{definition}
    On dit que $f$ est \textbf{holomorphe} en $a \in \Omega$ s'il existe un complexe $f'(a)$ tel que
    \[ f'(a) = \lim_{\substack{h \rightarrow 0 \\ h \neq 0}} \frac{f(a+h) - f(a)}{h} \]
    On dit que $f$ est holomorphe sur $\Omega$ si elle l'est en tout point de $\Omega$ et on note $f'$ la fonction $f' : z \mapsto f'(z)$ ainsi que $\mathcal{H}(\Omega)$ l'ensemble des fonctions holomorphes sur $\Omega$.
  \end{definition}

  \begin{example}
    \begin{itemize}
      \item $z \mapsto z^2$ est holomorphe sur $\mathbb{C}$, de dérivée $z \mapsto 2z$.
      \item $z \mapsto \overline{z}$ n'est holomorphe en aucun point de $\mathbb{C}$.
    \end{itemize}
  \end{example}

  \begin{proposition}
    \begin{enumerate}[label=(\roman*)]
      \item $\mathcal{H}(\Omega)$ est une algèbre sur $\mathbb{C}$ avec pour tout $g, h \in \mathcal{H}(\Omega)$ et $\lambda \in \mathbb{C}$ :
      \begin{itemize}
        \item $(g+h)' = g'+h'$.
        \item $(\lambda g)' = \lambda g'$.
        \item $(gh)' = g'h + gh'$.
        \item $\left( \frac{g}{h} \right)' = \frac{g'h - gh'}{g^2}$ quand $g$ ne s'annule pas sur $\Omega$.
      \end{itemize}
      \item Pour tout $g \in \mathcal{H}(\Omega)$, $h \in \mathcal{H}(\Omega_1)$ où $g(\Omega) \subseteq \Omega_1$
      \[ h \circ g \in \mathcal{H}(\Omega) \text{ et } (h \circ g)' = (h' \circ g) g' \]
      \item Soit $g \in \mathcal{H}(\Omega)$ holomorphe bijective d'inverse $h$. On suppose $h$ continue en $b = g(a)$ et $g'(a) \neq 0$. Alors $h$ est holomorphe en $b$ et
      \[ h'(b) = \frac{1}{g'(a)} \]
    \end{enumerate}
  \end{proposition}

  \reference[BMP]{57}

  \begin{theorem}[Conditions de Cauchy-Riemann]
    On pose $u = \operatorname{Re}(f)$ et $v = \operatorname{Im}(f)$. On suppose $f$ $\mathbb{R}$-différentiable en $a \in \Omega$. Alors, les propositions suivantes sont équivalentes :
    \begin{enumerate}[label=(\roman*)]
      \item $f$ est holomorphe en $a$.
      \item $\mathrm{d}f_a$ est $\mathbb{C}$-linéaire.
      \item $\frac{\partial f}{\partial y} (a) = i \frac{\partial f}{\partial x} (a)$.
      \item $\frac{\partial u}{\partial x} (a) = \frac{\partial v}{\partial y} (a)$ et $\frac{\partial u}{\partial y} (a) = -\frac{\partial v}{\partial x} (a)$.
    \end{enumerate}
  \end{theorem}

  \reference[QUE]{115}

  \begin{example}
    $z \mapsto \operatorname{Re}(z)$ et $z \mapsto \operatorname{Im}(z)$ ne sont holomorphes en aucun point de $\mathbb{C}$.
  \end{example}

  \reference[BMP]{69}

  \begin{theorem}[Weierstrass]
    Une suite de fonctions holomorphes qui converge uniformément sur tout compact de $\Omega$ a une limite holomorphe sur $\Omega$. De plus, la suite des dérivées $k$-ième converge uniformément sur tout compact vers la dérivée $k$-ième de la limite pour tout $k \in \mathbb{N}$.
  \end{theorem}

  \subsection{Séries entières et analycité}

  \subsubsection{Généralités sur les séries entières}

  \reference[GOU20]{247}

  \begin{definition}
    On appelle \textbf{série entière} toute série de fonctions de la forme $\sum a_n z^n$ où $z$ est une variable complexe et où $(a_n)$ est une suite complexe.
  \end{definition}

  \begin{lemma}[Abel]
    Soient $\sum a_n z^n$ une série entière et $z_0 \in \mathbb{C}$ tels que $(a_n z_0^n)$ soit bornée. Alors :
    \begin{enumerate}[label=(\roman*)]
      \item $\forall z \in \mathbb{C}$ tel que $|z| < |z_0|$, $\sum a_n z^n$ converge absolument.
      \item $\forall r \in ]0,z_0[, \, \sum a_n z^n$ converge normalement dans $\overline{D}(0, r) = \{ z \in \mathbb{C} \mid |z| \leq r \}$.
    \end{enumerate}
  \end{lemma}

  \begin{definition}
    En reprenant les notations précédentes, le nombre
    \[ R = \sup \{ r \geq 0 \mid (|a_n|r^n) \text{ est bornée} \} \]
    est le \textbf{rayon de convergence} de $\sum a_n z^n$.
  \end{definition}

  \reference{255}

  \begin{example}
    \begin{itemize}
      \item $\sum n^2 z^n$ a un rayon de convergence égal à $1$.
      \item $\sum \frac{z^n}{n!}$ a un rayon de convergence infini. On note $z \mapsto e^z$ la fonction somme.
    \end{itemize}
  \end{example}

  \reference[QUE]{57}

  \begin{proposition}
    Soit $\sum a_n z^n$ une série entière de rayon de convergence $r \neq 0$. Alors $S \in \mathcal{H}(D(0, r))$ et,
    \[ S'(z) = \sum_{n=0}^{+\infty} n a_n z^{n-1} \]
    pour tout $z \in D(0, r)$.
    \newpar
    Plus précisément, pour tout $k \in \mathbb{N}$, $S$ est $k$ fois dérivable avec
    \[ S^{(k)}(z) = \sum_{n=k}^{+\infty} n (n-1) \dots (n-k+1) a_n z^{n-k} \]
  \end{proposition}

  \subsubsection{Analycité}

  \reference{57}

  \begin{definition}
    On dit que $f$ est \textbf{analytique} sur $\Omega$ si, pour tout $a \in \Omega$, il existe $r > 0$ et une série entière $\sum a_n z^n$ de rayon de convergence $\geq r$, tels que
    \[ D(a, r) \subseteq \Omega \text{ et } \forall z \in D(a, r), \, f(z) = \sum_{n=0}^{+\infty} a_n z^n \]
    ie. $f$ est développable en série entière en tout point de $\Omega$. On note $\mathcal{A}(\Omega)$ l'ensemble des fonctions analytiques sur $\Omega$.
  \end{definition}

  \begin{proposition}
    Soit $\sum a_n z^n$ une série entière de rayon de convergence $r \neq 0$. Alors $S \in \mathcal{A}(D(0, r))$ et, si $\vert z - a \vert \leq r - \vert a \vert$ :
    \[ f(z) = \sum_{k=0}^{+\infty} \frac{S^{(k)}(a)}{k!} (z-a)^k \]
    (où $S^{(k)}$ désigne la $k$-ième dérivée complexe de $S$).
  \end{proposition}

  \reference{85}

  \begin{proposition}
    $\mathcal{A}(\Omega) \subseteq \mathcal{H}(\Omega)$.
  \end{proposition}

  \reference{78}

  \begin{proposition}
    Si $f = P/Q$ est une fraction rationnelle, alors $f$ est développable en série entière au voisinage de chaque point qui n'est pas un pôle de $f$ (cf. \cref{245-2}).
  \end{proposition}

  \reference[BMP]{53}

  \begin{theorem}[Zéros isolés]
    \label{245-1}
    On suppose $\Omega$ connexe et $f \in \mathcal{A}(\Omega)$. Si $f$ n'est pas identiquement nulle sur $\Omega$, alors l'ensemble des zéros de $f$ n'admet pas de point d'accumulation dans $\Omega$.
  \end{theorem}

  \reference{73}

  \begin{corollary}
    $\mathcal{A}(\Omega)$ est une algèbre intègre.
  \end{corollary}

  \reference{53}

  \begin{remark}[Prolongement analytique]
    Reformulé de manière équivalente au \cref{245-1}, si deux fonctions analytiques coïncident sur un sous-ensemble de $\Omega$ qui possède un point d'accumulation dans $\Omega$, alors elles sont égales sur $\Omega$.
  \end{remark}

  \reference{77}

  \begin{example}
    Il existe une unique fonction $g$ holomorphe sur $\mathbb{C}$ telle que
    \[ \forall n \in \mathbb{N}^*, \, g\left( \frac{1}{n} \right) = \frac{1}{n} \]
    et c'est la fonction identité.
  \end{example}

  \begin{cexample}
    Il existe au moins deux fonctions $g$ holomorphes sur $\Omega = \{ z \in \mathbb{C} \mid \operatorname{Re}(z) > 0 \}$ telles que
    \[ \forall n \in \mathbb{N}^*, \, g\left( \frac{1}{n} \right) = 0 \]
  \end{cexample}

  \subsection{Holomorphie et intégration}

  \subsubsection{Intégration sur une courbe}

  \reference[QUE]{85}

  \begin{definition}
    \begin{itemize}
      \item Un \textbf{chemin} est une application $\gamma : [a,b] \rightarrow \mathbb{C}$ (où $[a,b]$ est un segment de $\mathbb{R}$) continue.
      \item Si $\gamma(a) = \gamma(b)$, on dit que $\gamma$ est \textbf{fermé}.
      \item Si $\gamma$ est un chemin $\mathcal{C}^1$ par morceaux, on dit que $\gamma$ est une \textbf{courbe}.
      \item On appelle $\gamma^* = \gamma([a,b])$ l'\textbf{image} de $\gamma$.
    \end{itemize}
  \end{definition}

  \begin{example}
    Soient $\omega \in \mathbb{C}$ et $r \in \mathbb{R}^+_*$. Alors,
    \[
      \gamma :
      \begin{array}{ccc}
        [0,2\pi] &\rightarrow& \mathbb{C} \\
        t &\mapsto& \omega + re^{it}
      \end{array}
    \]
    est une courbe fermée (c'est la paramétrisation du cercle de centre $\omega$ et de rayon $r$).
  \end{example}

  \begin{definition}
    Soit $\gamma : [a,b] \rightarrow \Omega$ une courbe. L'\textbf{intégrale curviligne} le long de $\gamma$ est
    \[ \int_\gamma f(z) \, \mathrm{d}z = \int_b^a f(\gamma(t)) \gamma'(t) \, \mathrm{d}t \]
  \end{definition}

  \begin{proposition}
    Soit $\gamma : [a,b] \rightarrow \Omega$ une courbe de longueur $L(\gamma) = \int_{a}^{b} \vert \gamma'(t) \vert \, \mathrm{d}t$, alors,
    \[ \left\vert \int_\gamma f(z) \, \mathrm{d}z \right\vert \leq \sup_{z \in \gamma^*} \vert f(z) \vert \times L(\gamma) \]
  \end{proposition}

  \begin{proposition}
    Soit $\gamma : [a,b] \rightarrow \Omega$ une courbe. On suppose $\gamma^* \subseteq \Omega$, $f$ holomorphe sur $\Omega$ telle que $f'$ est continue sur $\gamma^*$. Alors,
    \[ \int_\gamma f'(z) \, \mathrm{d}z = f(\gamma(b)) - f(\gamma(a)) \]
  \end{proposition}

  \subsubsection{Théorie de Cauchy et lien avec l'analycité}

  \begin{definition}
    Soit $\gamma : [a,b] \rightarrow \Omega$ une courbe telle que $\omega \notin \gamma^*$. \textbf{L'indice} de $\omega$ par rapport à $\gamma$, noté $I(\omega, \gamma)$, est défini par
    \[ I(\omega, \gamma) = \frac{1}{2i\pi} \int_\gamma \frac{1}{z-a} \, \mathrm{d}z = \frac{1}{2i\pi} \int_b^a \frac{1}{\gamma(t)-a} \gamma'(t) \, \mathrm{d}t \]
  \end{definition}

  \begin{remark}
    En reprenant les notations précédentes, $I(\omega, \gamma)$ compte le nombre de tours orientés que $\gamma$ fait autour de $\omega$. En particulier :
    \begin{enumerate}[label=(\roman*)]
      \item On a toujours $I(\omega, \gamma) \in \mathbb{Z}$.
      \item On note $\gamma^* = \gamma([a,b])$ l'image de $\gamma$. $I(\omega, \gamma)$ est nulle sur la composante connexe non bornée de $\mathbb{C} \setminus \gamma^*$.
    \end{enumerate}
  \end{remark}

  \reference{134}

  \begin{theorem}[Cauchy homologique]
    Soit $\Gamma$ un cycle homologue à zéro dans $\Omega$ (ie. tel que $z \notin \Omega \implies I(a, \Gamma) = 0$). On suppose $f \in \mathcal{H}(\Omega)$. Alors,
    \[ \int_\Gamma f(z) \, \mathrm{d}z = 0 \]
  \end{theorem}

  \begin{corollary}[Formule intégrale de Cauchy]
    Soit $\Gamma$ un cycle homologue à zéro dans $\Omega$. On suppose $f \in \mathcal{H}(\Omega)$. Alors,
    \[ z_0 \in \Omega \setminus \Gamma^* \implies \frac{1}{2i\pi} \int_\Gamma \frac{f(z)}{z-z_0} \, \mathrm{d}z = I(z_0, \gamma) f(z_0) \]
  \end{corollary}

  \reference{85}
  \reference[BMP]{64}

  \begin{corollary}
    On a $\mathcal{H}(\Omega) \subseteq \mathcal{A}(\Omega)$. De plus, si $a \in \Omega$ et que l'on pose $d = d(a, \mathbb{C} \setminus \Omega)$, on a
    \[ f(a + h) = \sum_{n=0}^{+\infty} a_n h^n \text{ pour } \vert h \vert < d \text{ avec } a_n = \frac{f^{(n)}(a)}{n!} = \frac{1}{2i\pi} \int_{C^+(a,d)} \frac{f(z)}{(z-a)^{n+1}} \, \mathrm{d}z \]
  \end{corollary}

  \subsubsection{Conséquences}

  \reference[QUE]{102}

  \begin{proposition}[Inégalités de Cauchy]
    On suppose $f$ holomorphe au voisinage du disque $\overline{D}(a,R)$. On note $c_n$ les coefficients du développement en série entière de $f$ en $a$. Alors,
    \[ \forall n \in \mathbb{N}, \, \forall r \in [0,R], \, \vert c_n \vert \leq \frac{M(r)}{r^n} \]
    où $M(r) = \sup_{\vert z-a \vert = r} \vert f(z) \vert$.
  \end{proposition}

  \begin{corollary}[Théorème de Liouville]
    On suppose $f$ holomorphe sur $\mathbb{C}$ tout entier. Si $f$ est bornée, alors $f$ est constante.
  \end{corollary}

  \reference{107}

  \begin{theorem}[Principe du maximum]
    On suppose $\Omega$ borné et $f$ holomorphe dans $\Omega$ et continue dans $\overline{\Omega}$. On note $M$ le $\sup$ de $f$ sur la frontière (compacte) de $\Omega$. Alors,
    \[ \forall z \in \Omega, \, \vert f(z) \vert \leq M \]
  \end{theorem}

  \subsubsection{Holomorphie d'une intégrale à paramètre}

  \reference{101}

  \begin{theorem}[Holomorphie sous le signe intégral]
    On suppose :
    \begin{enumerate}[label=(\roman*)]
      \item $\forall z \in \Omega$, $x \mapsto f(z,x) \in L_1(X)$.
      \item pp. en $x \in X$, $z \mapsto f(z,x)$ est holomorphe dans $\Omega$. On notera $\frac{\partial f}{\partial z}$ cette dérivée définie presque partout.
      \item $\forall K \subseteq \Omega$ compact, $\exists g_K \in L_1(X)$ positive telle que
      \[ \left| f(x,z) \right| \leq g_K(x) \quad \forall z \in K, \text{pp. en } x \]
    \end{enumerate}
    Alors $F$ est holomorphe dans $\Omega$ avec
    \[ \forall z \in \Omega, \, F'(z) = \int_X \frac{\partial f}{\partial z}(z, t) \, \mathrm{d}\mu(z) \]
  \end{theorem}

  \reference{115}

  \begin{application}
    Soit $f \in L_1(\mathbb{R})$ ainsi que sa transformée de Fourier $\widehat{f} : x \mapsto \int_{\mathbb{R}} f(t) e^{-ixt} \, \mathrm{d}t$. Alors $f = 0$.
  \end{application}

  \reference[BMP]{83}

  \begin{application}
    $F : z \mapsto \int_{\mathbb{R}} e^{zx} e^{-x^2} \, \mathrm{d}x$ définit une fonction holomorphe sur $\mathbb{C}$ qui coïncide avec la transformée de Fourier de $f : x \mapsto e^{-x^2}$ sur $\mathbb{R}$. On trouve en particulier,
    \[ \forall t \in \mathbb{R}, \, \widehat{f}(t) = F(it) = \sqrt{\pi} e^{-\frac{t^2}{4}} \]
  \end{application}

  \reference{110}

  \begin{notation}
    Soient $I$ un intervalle de $\mathbb{R}$ et $\rho : I \rightarrow \mathbb{R}$ une fonction poids. On note :
    \begin{itemize}
      \item $\forall n \in \mathbb{N}$, $g_n : x \mapsto x^n$.
      \item $L_2(I, \rho)$ l'espace des fonctions de carré intégrable pour la mesure de densité $\rho$ par rapport à la mesure de Lebesgue.
    \end{itemize}
  \end{notation}

  \reference{140}

  \begin{lemma}
    On suppose que $\forall n \in \mathbb{N}$, $g_n \in L_1(I, \rho)$ et on considère $(P_n)$ la famille des polynômes orthogonaux associée à $\rho$ sur $I$. Alors $\forall n \in \mathbb{N}$, $g_n \in L_2(I, \rho)$. En particulier, l'algorithme de Gram-Schmidt a bien du sens et $(P_n)$ est bien définie.
  \end{lemma}

  \dev{densite-des-polynomes-orthogonaux}

  \begin{application}
    Soient $I$ un intervalle de $\mathbb{R}$ et $\rho$ une fonction poids. On considère $(P_n)$ la famille des polynômes orthogonaux associée à $\rho$ sur $I$.
    \newpar
    On suppose qu'il existe $a > 0$ tel que
    \[ \int_I e^{a \vert x \vert} \rho(x) \, \mathrm{d}x < +\infty \]
    alors $(P_n)$ est une base hilbertienne de $L_2(I, \rho)$ pour la norme $\Vert . \Vert_2$.
  \end{application}

  \subsection{Méromorphie}

  \subsubsection{Singularités}

  \reference[QUE]{165}

  \begin{definition}
    \label{245-2}
    Soit $a \in \Omega$. On suppose $f \in \mathcal{H}(\Omega \setminus \{ a \})$.
    \begin{itemize}
      \item On dit que $a$ est une \textbf{singularité effaçable} pour $f$ s'il existe $g \in \mathcal{H}(\Omega)$ tel que $f(z) = g(z)$ pour tout $z \in \Omega \setminus \{ a \}$.
      \item On dit que $a$ est un \textbf{pôle} d'ordre $m$ s'il existe des scalaires $c_{-1}, \dots, c_{-m}$ avec $c_{-m} \neq 0$ tels que $z \mapsto f(z) - \sum_{k=1}^m \frac{c_{-k}}{(z-a)^{k}}$ ait une singularité effaçable en $a$.
      \item $\sum_{k=1}^m \frac{c_{-k}}{(z-a)^{k}}$ est la \textbf{partie principale} de $f$ en $a$ et $c_{-1}$ est le \textbf{résidu} de $f$ en $a$ noté $\operatorname{Res}(f,a)$.
    \end{itemize}
  \end{definition}

  \begin{example}
    \begin{itemize}
      \item $z \mapsto \frac{\sin(z)}{z}$ a une singularité effaçable en $0$.
      \item $z \mapsto \frac{e^z}{z}$ a un pôle d'ordre $1$ (simple) en $0$ avec partie principale égale à $\frac{1}{z}$ et $\operatorname{Res}(f, 0) = 1$.
    \end{itemize}
  \end{example}

  \begin{definition}
    On dit que $f$ est \textbf{méromorphe} sur $\Omega$ s'il existe $A \subseteq \Omega$ tel que :
    \begin{itemize}
      \item $A$ n'a que des points isolés dans $\Omega$ (en particulier, $A$ est au plus dénombrable et $\Omega \setminus A$ est ouvert).
      \item $f \in \mathcal{H}(\Omega \setminus A)$.
      \item $f$ a un pôle en chaque point de $a$.
    \end{itemize}
  \end{definition}

  \begin{example}
    $z \mapsto \frac{1}{\sin(z)}$ est méromorphe dans $\mathbb{C}$ et en reprenant les notations précédentes, $A = \{ k\pi \mid k \in \mathbb{Z} \}$.
  \end{example}

  \reference[BMP]{82}

  \begin{example}
    La fonction $\Gamma$ définie par
    \[
      \Gamma :
      \begin{array}{ccc}
        \{ z \in \mathbb{C} \mid \operatorname{Re}(z) > 0 \} &\rightarrow& \mathbb{C} \\
        z &\mapsto& \int_{0}^{+\infty} e^{-t} t^{z-1} \, \mathrm{d}t
      \end{array}
    \]
    se prolonge en une fonction méromorphe sur $\mathbb{C} \setminus \mathbb{N}$.
  \end{example}

  \reference[QUE]{168}

  \begin{proposition}
    On suppose $f = \frac{g}{h}$ où $g$ et $h$ sont holomorphes en un voisinage de $a \in \Omega$ avec $a$ un zéro simple de $h$ et $g(a) \neq 0$. Alors, $a$ est un pôle simple de $f$ de résidu
    \[ \operatorname{Res}(f, a) = \frac{g(a)}{h'(a)} \]
  \end{proposition}

  \begin{example}
    Le résidu de $z \mapsto \frac{z^2}{(z+1)(z-1)^2}$ en $1$ est égal à $\frac{3}{4}$.
  \end{example}

  \subsubsection{Théorème des résidus}

  \begin{theorem}[des résidus]
    On suppose $f$ méromorphe sur $\Omega$ et on note $A$ l'ensemble de ses pôles. Soit $\gamma$ une courbe homologue à zéro dans $\Omega$ et ne rencontrant pas $A$. Alors,
    \[ \int_\gamma f(z) \, \mathrm{d}z = 2i\pi \sum_{a \in A} I(a, \gamma) \operatorname{Res}(f, a) \]
  \end{theorem}

  \reference{173}

  \begin{example}
    \[ \int_{0}^{2\pi} \frac{1}{3 + 2\cos(t)} \, \mathrm{d}t = \frac{2\pi}{\sqrt{5}} \]
  \end{example}

  \begin{example}[Intégrale de Dirichlet]
    \[ \int_{0}^{+\infty} \frac{\sin(x)}{x} \, \mathrm{d}x = \frac{\pi}{2} \]
  \end{example}

  \reference[AMR08]{156}
  \dev{transformee-de-fourier-d-une-gaussienne}

  \begin{example}[Transformée de Fourier d'une gaussienne]
    On définit $\forall a \in \mathbb{R}^+_*$,
    \[ \gamma_a :
    \begin{array}{ccc}
      \mathbb{R} &\rightarrow& \mathbb{R} \\
      x &\mapsto& e^{-ax^2}
    \end{array}
    \]
    Alors,
    \[ \forall \xi \in \mathbb{R}, \, \widehat{\gamma_a}(\xi) = \sqrt{\frac{\pi}{a}} e^{\frac{- \xi^2}{4a}} \]
  \end{example}

  \reference[QUE]{171}

  \begin{application}[Théorème de Kronecker]
    On suppose $f$ holomorphe sur $\Omega$ et non identiquement nulle dans $\Omega$. Soit $\gamma$ une courbe homologue à zéro dans $\Omega$ et qui ne rencontre pas l'ensemble des zéros de $f$. Alors, le nombre $Z = Z(f)$ des zéros de $f$ à l'intérieur de $\gamma$ comptés avec multiplicités vérifie
    \[ Z = \frac{1}{2i\pi} \int_\gamma \frac{f'(z)}{f(z)} \, \mathrm{d}z \]
  \end{application}

  \begin{application}[Théorème de Rouché]
    Soient $\gamma$ un cycle homologue à zéro dans $\Omega$ et $g, h \in \mathcal{H}(\Omega)$. On suppose
    \[ z \in \gamma^* \implies \vert g(z) \vert \leq \vert f(z) \vert \]
    Alors,
    \[ Z(g) = Z(g + h) \]
  \end{application}

  \reference[BMP]{67}

  \begin{example}
    $z \mapsto z^8 - 5z^3 + z - 2$ a trois zéros dans $D(0,1)$.
  \end{example}
  %</content>
\end{document}
