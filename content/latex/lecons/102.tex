\documentclass[12pt, a4paper]{report}

% LuaLaTeX :

\RequirePackage{iftex}
\RequireLuaTeX

% Packages :

\usepackage[french]{babel}
%\usepackage[utf8]{inputenc}
%\usepackage[T1]{fontenc}
\usepackage[pdfencoding=auto, pdfauthor={Hugo Delaunay}, pdfsubject={Mathématiques}, pdfcreator={agreg.skyost.eu}]{hyperref}
\usepackage{amsmath}
\usepackage{amsthm}
%\usepackage{amssymb}
\usepackage{stmaryrd}
\usepackage{tikz}
\usepackage{tkz-euclide}
\usepackage{fourier-otf}
\usepackage{fontspec}
\usepackage{titlesec}
\usepackage{fancyhdr}
\usepackage{catchfilebetweentags}
\usepackage[french, capitalise, noabbrev]{cleveref}
\usepackage[fit, breakall]{truncate}
\usepackage[top=2.5cm, right=2cm, bottom=2.5cm, left=2cm]{geometry}
\usepackage{enumerate}
\usepackage{tocloft}
\usepackage{microtype}
%\usepackage{mdframed}
%\usepackage{thmtools}
\usepackage{xcolor}
\usepackage{tabularx}
\usepackage{aligned-overset}
\usepackage[subpreambles=true]{standalone}
\usepackage{environ}
\usepackage[normalem]{ulem}
\usepackage{marginnote}
\usepackage{etoolbox}
\usepackage{setspace}
\usepackage[bibstyle=reading, citestyle=draft]{biblatex}
\usepackage{xpatch}
\usepackage[many, breakable]{tcolorbox}
\usepackage[backgroundcolor=white, bordercolor=white, textsize=small]{todonotes}

% Bibliographie :

\newcommand{\overridebibliographypath}[1]{\providecommand{\bibliographypath}{#1}}
\overridebibliographypath{../bibliography.bib}
\addbibresource{\bibliographypath}
\defbibheading{bibliography}[\bibname]{%
	\newpage
	\section*{#1}%
}
\renewbibmacro*{entryhead:full}{\printfield{labeltitle}}
\DeclareFieldFormat{url}{\newline\footnotesize\url{#1}}
\AtEndDocument{\printbibliography}

% Police :

\setmathfont{Erewhon Math}

% Tikz :

\usetikzlibrary{calc}

% Longueurs :

\setlength{\parindent}{0pt}
\setlength{\headheight}{15pt}
\setlength{\fboxsep}{0pt}
\titlespacing*{\chapter}{0pt}{-20pt}{10pt}
\setlength{\marginparwidth}{1.5cm}
\setstretch{1.1}

% Métadonnées :

\author{agreg.skyost.eu}
\date{\today}

% Titres :

\setcounter{secnumdepth}{3}

\renewcommand{\thechapter}{\Roman{chapter}}
\renewcommand{\thesubsection}{\Roman{subsection}}
\renewcommand{\thesubsubsection}{\arabic{subsubsection}}
\renewcommand{\theparagraph}{\alph{paragraph}}

\titleformat{\chapter}{\huge\bfseries}{\thechapter}{20pt}{\huge\bfseries}
\titleformat*{\section}{\LARGE\bfseries}
\titleformat{\subsection}{\Large\bfseries}{\thesubsection \, - \,}{0pt}{\Large\bfseries}
\titleformat{\subsubsection}{\large\bfseries}{\thesubsubsection. \,}{0pt}{\large\bfseries}
\titleformat{\paragraph}{\bfseries}{\theparagraph. \,}{0pt}{\bfseries}

\setcounter{secnumdepth}{4}

% Table des matières :

\renewcommand{\cftsecleader}{\cftdotfill{\cftdotsep}}
\addtolength{\cftsecnumwidth}{10pt}

% Redéfinition des commandes :

\renewcommand*\thesection{\arabic{section}}
\renewcommand{\ker}{\mathrm{Ker}}

% Nouvelles commandes :

\newcommand{\website}{https://agreg.skyost.eu}

\newcommand{\tr}[1]{\mathstrut ^t #1}
\newcommand{\im}{\mathrm{Im}}
\newcommand{\rang}{\operatorname{rang}}
\newcommand{\trace}{\operatorname{trace}}
\newcommand{\id}{\operatorname{id}}
\newcommand{\stab}{\operatorname{Stab}}

\providecommand{\newpar}{\\[\medskipamount]}

\providecommand{\lesson}[3]{%
	\title{#3}%
	\hypersetup{pdftitle={#3}}%
	\setcounter{section}{\numexpr #2 - 1}%
	\section{#3}%
	\fancyhead[R]{\truncate{0.73\textwidth}{#2 : #3}}%
}

\providecommand{\development}[3]{%
	\title{#3}%
	\hypersetup{pdftitle={#3}}%
	\section*{#3}%
	\fancyhead[R]{\truncate{0.73\textwidth}{#3}}%
}

\providecommand{\summary}[1]{%
	\textit{#1}%
	\medskip%
}

\tikzset{notestyleraw/.append style={inner sep=0pt, rounded corners=0pt, align=center}}

%\newcommand{\booklink}[1]{\website/bibliographie\##1}
\newcommand{\citelink}[2]{\hyperlink{cite.\therefsection @#1}{#2}}
\newcommand{\previousreference}{}
\providecommand{\reference}[2][]{%
	\notblank{#1}{\renewcommand{\previousreference}{#1}}{}%
	\todo[noline]{%
		\protect\vspace{16pt}%
		\protect\par%
		\protect\notblank{#1}{\cite{[\previousreference]}\\}{}%
		\protect\citelink{\previousreference}{p. #2}%
	}%
}

\definecolor{devcolor}{HTML}{00695c}
\newcommand{\dev}[1]{%
	\reversemarginpar%
	\todo[noline]{
		\protect\vspace{16pt}%
		\protect\par%
		\bfseries\color{devcolor}\href{\website/developpements/#1}{DEV}
	}%
	\normalmarginpar%
}

% En-têtes :

\pagestyle{fancy}
\fancyhead[L]{\truncate{0.23\textwidth}{\thepage}}
\fancyfoot[C]{\scriptsize \href{\website}{\texttt{agreg.skyost.eu}}}

% Couleurs :

\definecolor{property}{HTML}{fffde7}
\definecolor{proposition}{HTML}{fff8e1}
\definecolor{lemma}{HTML}{fff3e0}
\definecolor{theorem}{HTML}{fce4f2}
\definecolor{corollary}{HTML}{ffebee}
\definecolor{definition}{HTML}{ede7f6}
\definecolor{notation}{HTML}{f3e5f5}
\definecolor{example}{HTML}{e0f7fa}
\definecolor{cexample}{HTML}{efebe9}
\definecolor{application}{HTML}{e0f2f1}
\definecolor{remark}{HTML}{e8f5e9}
\definecolor{proof}{HTML}{e1f5fe}

% Théorèmes :

\theoremstyle{definition}
\newtheorem{theorem}{Théorème}

\newtheorem{property}[theorem]{Propriété}
\newtheorem{proposition}[theorem]{Proposition}
\newtheorem{lemma}[theorem]{Lemme}
\newtheorem{corollary}[theorem]{Corollaire}

\newtheorem{definition}[theorem]{Définition}
\newtheorem{notation}[theorem]{Notation}

\newtheorem{example}[theorem]{Exemple}
\newtheorem{cexample}[theorem]{Contre-exemple}
\newtheorem{application}[theorem]{Application}

\theoremstyle{remark}
\newtheorem{remark}[theorem]{Remarque}

\counterwithin*{theorem}{section}

\newcommand{\applystyletotheorem}[1]{
	\tcolorboxenvironment{#1}{
		enhanced,
		breakable,
		colback=#1!98!white,
		boxrule=0pt,
		boxsep=0pt,
		left=8pt,
		right=8pt,
		top=8pt,
		bottom=8pt,
		sharp corners,
		after=\par,
	}
}

\applystyletotheorem{property}
\applystyletotheorem{proposition}
\applystyletotheorem{lemma}
\applystyletotheorem{theorem}
\applystyletotheorem{corollary}
\applystyletotheorem{definition}
\applystyletotheorem{notation}
\applystyletotheorem{example}
\applystyletotheorem{cexample}
\applystyletotheorem{application}
\applystyletotheorem{remark}
\applystyletotheorem{proof}

% Environnements :

\NewEnviron{whitetabularx}[1]{%
	\renewcommand{\arraystretch}{2.5}
	\colorbox{white}{%
		\begin{tabularx}{\textwidth}{#1}%
			\BODY%
		\end{tabularx}%
	}%
}

% Maths :

\DeclareFontEncoding{FMS}{}{}
\DeclareFontSubstitution{FMS}{futm}{m}{n}
\DeclareFontEncoding{FMX}{}{}
\DeclareFontSubstitution{FMX}{futm}{m}{n}
\DeclareSymbolFont{fouriersymbols}{FMS}{futm}{m}{n}
\DeclareSymbolFont{fourierlargesymbols}{FMX}{futm}{m}{n}
\DeclareMathDelimiter{\VERT}{\mathord}{fouriersymbols}{152}{fourierlargesymbols}{147}


% Bibliographie :

\addbibresource{\bibliographypath}%
\defbibheading{bibliography}[\bibname]{%
	\newpage
	\section*{#1}%
}
\renewbibmacro*{entryhead:full}{\printfield{labeltitle}}%
\DeclareFieldFormat{url}{\newline\footnotesize\url{#1}}%

\AtEndDocument{\printbibliography}

\begin{document}
  %<*content>
  \lesson{algebra}{102}{Groupe des nombres complexes de module \texorpdfstring{$1$}{1}. Racines de l'unité. Applications.}

  \subsection{Nombres complexes de module \texorpdfstring{$1$}{1}}

  \subsubsection{Le groupe \texorpdfstring{$\mathbb{U}$}{U}}

  \reference[ROM21]{36}

  \begin{definition}
    On définit
    \[ \mathbb{U} = \{ z \in \mathbb{C} \mid \vert z \vert = 1 \} \]
    le groupe abélien des nombres complexes de module $1$.
  \end{definition}

  \begin{proposition}
    L'application
    \[ \exp(i\theta) \mapsto
      \begin{pmatrix}
        \cos(\theta) & \sin(\theta) \\
        -\sin(\theta) & \cos(\theta)
      \end{pmatrix}
    \]
    (où $\exp$ est définie dans la sous-section suivante) définit un isomorphisme de $\mathbb{U}$ dans $\mathrm{SO}_2(\mathbb{R})$.
  \end{proposition}

  \reference[FGN3]{51}

  \begin{proposition}
    Un sous-groupe additif de $\mathbb{R}$ est soit dense dans $\mathbb{R}$, soit de la forme $n\mathbb{Z}$.
  \end{proposition}

  \begin{corollary}
    Un sous-groupe de $\mathbb{U}$ est soit fini, soit dense dans $\mathbb{U}$.
  \end{corollary}

  \begin{corollary}
    Soit $\theta \notin 2\pi\mathbb{Q}$. $\{ e^{in\theta} \mid n \in \mathbb{N} \}$ est dense dans $\mathbb{U}$.
  \end{corollary}

  \begin{application}
    $\{ \sin(n) \mid n \in \mathbb{N} \}$ est dense dans $[-1,1]$.
  \end{application}

  \reference[GOU20]{44}

  \begin{proposition}
    $\mathbb{U}$ est un sous-groupe compact et connexe de $\mathbb{C}^*$.
  \end{proposition}

  \begin{application}
    Soit $f : \mathbb{U} \rightarrow \mathbb{R}$ continue. Alors il existe deux points diamétralement opposés de $\mathbb{U}$ qui ont la même image par $f$.
  \end{application}

  \subsubsection{L'exponentielle complexe}

  \reference[QUE]{4}

  \begin{definition}
    On définit la fonction \textbf{exponentielle complexe} pour tout $z \in \mathbb{C}$ par
    \[ \sum_{n=0}^{+\infty} \frac{z^n}{n!} \]
    on note cette somme $e^z$ ou parfois $\exp(z)$.
  \end{definition}

  \begin{remark}
    Cette somme est bien définie pour tout $z \in \mathbb{C}$ d'après le critère de d'Alembert.
  \end{remark}

  \begin{proposition}
    \begin{enumerate}[label=(\roman*)]
      \item $\forall z, z' \in \mathbb{C}, \, e^{z+z'} = e^z e^{z'}$.
      \item $\exp$ est holomorphe sur $\mathbb{C}$, de dérivée elle-même.
      \item $\exp$ ne s'annule jamais.
    \end{enumerate}
  \end{proposition}

  \begin{proposition}
    La fonction $\varphi : t \mapsto e^{it}$ est un morphisme surjectif de $\mathbb{R}$ sur $\mathbb{U}$.
  \end{proposition}

  \begin{proposition}
    En reprenant les notations précédentes, $\ker(\varphi)$ est un sous-groupe fermé de $\mathbb{R}$, de la forme $\ker(\varphi) = a\mathbb{Z}$. On note $a = 2\pi$.
  \end{proposition}

  \subsubsection{Trigonométrie}

  \begin{definition}
    Les fonctions $\sin$ et $\cos$ sont définies sur $\mathbb{R}$ par
    \begin{itemize}
      \item $\cos(t) = \operatorname{Re}(e^{it}) = \frac{e^{it} + e^{-it}}{2} = \sum_{n=0}^{+\infty} (-1)^n \frac{t^{2n}}{(2n)!}$.
      \item $\sin(t) = \operatorname{Im}(e^{it}) = \frac{e^{it} - e^{-it}}{2i} = \sum_{n=0}^{+\infty} (-1)^n \frac{t^{2n+1}}{(2n+1)!}$.
    \end{itemize}
  \end{definition}

  \begin{proposition}
    Ces fonctions sont réelles, $2\pi$-périodiques, et admettent un développement en série entière de rayon de convergence infini. On peut en particulier les prolonger sur le plan complexe entier.
  \end{proposition}

  \reference[R-R]{259}

  \begin{proposition}
    Tout nombre complexe $z \in \mathbb{C}$ peut s'écrire de la manière suivante :
    \[ z = \vert z \vert e^{i\theta} = \cos(\theta) + i\sin(\theta) \]
  \end{proposition}

  \begin{proposition}[Formule de Moivre]
    \[ \forall n \in \mathbb{N}, \, \forall \theta \in \mathbb{R}, \, (\cos(\theta) + i\sin(\theta))^n = \cos(n\theta) + i\sin(n\theta) \]
  \end{proposition}

  \reference[GOU20]{271}

  \begin{application}[Calcul du noyau de Dirichlet]
    \[ \forall n \in \mathbb{N}^*, \, \forall x \in \mathbb{R} \setminus 2\pi\mathbb{Z}, \, \sum_{k=-n}^n \frac{\sin\left( \frac{(2n+1)x}{2} \right)}{\sin\left( \frac{x}{2} \right)} \]
  \end{application}

  \subsection{Le groupe des racines de l'unité}

  Soit $n \in \mathbb{N}^*$.

  \subsubsection{Racines \texorpdfstring{$n$}{n}-ièmes de l'unité}

  \reference[R-R]{259}

  \begin{definition}
    Étant donnés $\alpha \in \mathbb{C}$, on appelle :
    \begin{itemize}
      \item \textbf{Racine $n$-ième de $\alpha$} tout nombre $z \in \mathbb{C}$ tel que $z^n = \alpha$.
      \item \textbf{Racine $n$-ième de l'unité} toute racine $n$-ième de $1$. On note $\mu_n$ cet ensemble.
    \end{itemize}
  \end{definition}

  \begin{example}
    Les racines cubiques de l'unité sont $1$, $j = -\frac{1}{2} +i\frac{\sqrt{3}}{2}$ et $\overline{j}$.
  \end{example}

  \begin{proposition}
    Pour tout $n \in \mathbb{N}^*$, il y a $n$ racines $n$-ièmes de l'unité, données par
    \[ e^{\frac{2ik\pi}{n}} = \cos \left( \frac{2ik\pi}{n} \right) + i \sin \left( \frac{2ik\pi}{n} \right) \]
    où $k$ parcourt les entiers de $0$ à $n-1$.
  \end{proposition}

  \begin{corollary}
    Pour tout $n \in \mathbb{N}^*$,
    \[ X^n - 1 = \prod_{k=0}^{n-1} (X - e^{\frac{2ik\pi}{n}}) \]
  \end{corollary}

  \begin{corollary}
    Tout nombre complexe non nul $\alpha$ écrit $\alpha = re^{i\theta}$ admet exactement $n$ racines $n$-ièmes données par
    \[ \sqrt[n]{r} e^{i\frac{\theta}{n}} e^{\frac{2ik\pi}{n}} \]
    où $k$ parcourt les entiers de $0$ à $n-1$.
  \end{corollary}

  \reference[GOZ]{67}

  \begin{proposition}
    $\mu_n$ est un groupe, et l'application
    \[
    \begin{array}{ccc}
      \mathbb{Z}/n\mathbb{Z} &\rightarrow& \mu_n \\
      k &\mapsto& e^{\frac{2ik\pi}{n}}
    \end{array}
    \]
    est un isomorphisme.
  \end{proposition}

  \reference[ROM21]{36}

  \begin{proposition}
    $\mathbb{C}^*$ admet exactement un sous-groupe d'ordre $n$ : $\mu_n$.
  \end{proposition}

  \subsubsection{Générateurs et polynômes cyclotomiques}

  \reference[GOZ]{67}

  \begin{definition}
    L'ensemble des générateurs de $\mu_n$, noté $\mu_n^*$, est formé des \textbf{racines primitives $n$-ièmes de l'unité}.
  \end{definition}

  \begin{proposition}
    \begin{enumerate}[label=(\roman*)]
      \item $\mu_n^* = \{ e^{\frac{2ik\pi}{n}} \mid k \in \llbracket 0, n-1 \rrbracket, \, \operatorname{pgcd}(k, m) = 1 \}$.
      \item $\vert \mu_n^* \vert = \varphi(n)$, où $\varphi$ désigne l'indicatrice d'Euler.
    \end{enumerate}
  \end{proposition}

  \begin{definition}
    On appelle \textbf{$n$-ième polynôme cyclotomique} le polynôme
    \[ \Phi_n = \prod_{\xi \in \mu_n^*} (X - \xi) \]
  \end{definition}

  \begin{theorem}
    \begin{enumerate}[label=(\roman*)]
      \item $X^n - 1 = \prod_{d \mid n} \Phi_d$.
      \item $\Phi_n \in \mathbb{Z}[X]$.
      \item $\Phi_n$ est irréductible sur $\mathbb{Q}$.
    \end{enumerate}
  \end{theorem}

  \begin{corollary}
    Le polynôme minimal sur $\mathbb{Q}$ de tout élément $\xi$ de $\mu_n^*$ est $\Phi_n$. En particulier,
    \[ [\mathbb{Q}(\xi):\mathbb{Q}]=\varphi(m) \]
  \end{corollary}

  \begin{application}[Théorème de Wedderburn]
    Tout corps fini est commutatif.
  \end{application}

  \reference[GOU21]{99}
  \dev{theoreme-de-dirichlet-faible}

  \begin{application}[Dirichlet faible]
    Pour tout entier $n$, il existe une infinité de nombres premiers congrus à $1$ modulo $n$.
  \end{application}

  \subsection{Applications en algèbre}

  \subsubsection{Une application géométrique}

  \reference{153}

  \begin{proposition}[Déterminant circulant]
    Soient $n \in \mathbb{N}^*$ et $a_1, \dots, a_n \in \mathbb{C}$. On pose $\omega = e^{\frac{2i\pi}{n}}$. Alors
    \[ \begin{vmatrix} a_0 & a_1 & \dots & a_{n-1} \\ a_{n-1} & a_0 & \dots & a_{n-2}\\ \vdots & \vdots & \ddots & \vdots \\ a_1 & a_2 & \dots & a_0 \end{vmatrix} = \prod_{j=0}^{n-1} P(\omega^j) \]
    où $P = \sum_{k=0}^{n-1} a_k X^k$.
  \end{proposition}

  \reference[I-P]{389}

  \begin{application}[Suite de polygones]
    Soit $P_0$ un polygone dont les sommets sont $\{ z_{0,1}, \dots, z_{0,n} \}$. On définit la suite de polygones $(P_k)$ par récurrence en disant que, pour tout $k \in \mathbb{N}^*$, les sommets de $P_{k+1}$ sont les milieux des arêtes de $P_k$.
    \newpar
    Alors la suite $(P_k)$ converge vers l'isobarycentre de $P_0$.
  \end{application}

  \subsubsection{Racines de polynômes}

  \reference{279}
  \dev{theoreme-de-kronecker}

  \begin{theorem}[Kronecker]
    Soit $P \in \mathbb{Z}[X]$ unitaire tel que toutes ses racines complexes appartiennent au disque unité épointé en l'origine (que l'on note $D$). Alors toutes ses racines sont des racines de l'unité.
  \end{theorem}

  \begin{corollary}
    Soit $P \in \mathbb{Z}[X]$ unitaire et irréductible sur $\mathbb{Q}$ tel que toutes ses racines complexes soient de module inférieur ou égal à $1$. Alors $P = X$ ou $P$ est un polynôme cyclotomique.
  \end{corollary}

  \subsubsection{Dual d'un groupe}

  Soit $G$ un groupe fini de cardinal $n$.

  \reference[PEY]{2}

  \begin{definition}
    Un \textbf{caractère} est un morphisme de $G$ dans $\mathbb{C}^*$. On note $\widehat{G}$ l'ensemble des caractères, qu'on appelle \textbf{dual} de $G$.
  \end{definition}

  \begin{proposition}
    $\widehat{G}$ est un groupe pour la multiplication.
  \end{proposition}

  \begin{proposition}
    \begin{enumerate}[label=(\roman*)]
      \item $\widehat{G}$ est constitué des morphismes de $G$ dans $\mu_n$.
      \item $\forall g \in G$, $\vert \chi(g) \vert = 1$.
      \item $\forall g \in G$, $\chi(g^{-1}) = \chi(g)^{-1} = \overline{\chi(g)}$.
    \end{enumerate}
  \end{proposition}

  \begin{proposition}
    Si $G = \langle g_0 \rangle$, en notant $\omega$ une racine primitive $n$-ième de l'unité, les éléments de $\widehat{G}$ sont de la forme $g_0^k \mapsto (\omega^j)^k$ pour $j \in \llbracket 0, n-1 \rrbracket$.
  \end{proposition}

  \begin{corollary}
    Si $G$ est cyclique, $G \cong \widehat{G}$.
  \end{corollary}

  \subsubsection{Transformée de Fourier discrète}

  \reference{64}

  Soit $N \in \mathbb{N}^*$.

  \begin{notation}
    Soit $f$ un vecteur de $\mathbb{C}^{N}$. On note $f[k]$ sa $k$-ième composante pour tout $k \in \llbracket 1, N \rrbracket$.
  \end{notation}

  \begin{definition}
    Soit $f$ un vecteur de $\mathbb{C}^{N}$.
    La \textbf{transformée de Fourier discrète} de $f$ est
    \[ \widehat{f} = \sum_{n=0}^{N-1} f[n] \omega_N^{-nk} \]
    pour $k \in \llbracket 0, N-1 \rrbracket$ où l'on a noté $\omega_N = e^{\frac{2i\pi}{N}}$ une racine primitive $N$-ième de l'unité.
    On note
    \[
    \mathcal{F} :
    \begin{array}{ccc}
      \mathbb{C}^{N} &\rightarrow& \mathbb{C}^{N} \\
      f &\mapsto& \widehat{f}
    \end{array}
    \]
  \end{definition}

  \begin{proposition}[Transformée de Fourier inverse]
    \[ \forall n \in \llbracket 0, N-1 \rrbracket, f[n] = \frac{1}{N} \sum_{k=0}^{N-1} \widehat{f}[k] \omega_N^{nk} \]
  \end{proposition}

  \begin{corollary}
    Soit $f$ un vecteur de $\mathbb{C}^{N}$. En notant $f_1$ le vecteur défini par
    \[ f_1[0] = \frac{1}{N} f[0] \text{ et } \forall n \in \llbracket 1, \dots, N-1 \rrbracket, f_1[n] = \frac{1}{N} f[N-n] \]
    on a
    \[ \mathcal{F}^{-1}(f) = \mathcal{F}(f_1) \]
  \end{corollary}

  \annexessection

  \reference[I-P]{389}

  \begin{figure}[h]
    \begin{center}
      \begin{tikzpicture}
        \coordinate (A) at (0:3);
        \coordinate (B) at (72:3);
        \coordinate (C) at (2*72:3);
        \coordinate (D) at (3*72:3);
        \coordinate (E) at (4*72:3);
        \coordinate (F) at (A);
        \foreach \i in {0,...,10} {
          \draw(A) node {$\bullet$};
          \draw(B) node {$\bullet$};
          \draw(C) node {$\bullet$};
          \draw(D) node {$\bullet$};
          \draw(E) node {$\bullet$};
          \draw[fill=cyan!60, fill opacity=0.2](A) -- (B) -- (C) -- (D) -- (E) -- (A);
          \coordinate (A) at ($(A)!0.5!(B)$);
          \coordinate (B) at ($(B)!0.5!(C)$);
          \coordinate (C) at ($(C)!0.5!(D)$);
          \coordinate (D) at ($(D)!0.5!(E)$);
          \coordinate (E) at ($(E)!0.5!(F)$);
          \coordinate (F) at (A);
        }
      \end{tikzpicture}
    \end{center}
    \caption{La suite de polygones.}
  \end{figure}
  %</content>
\end{document}
