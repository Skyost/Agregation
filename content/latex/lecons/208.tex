\documentclass[12pt, a4paper]{report}

% LuaLaTeX :

\RequirePackage{iftex}
\RequireLuaTeX

% Packages :

\usepackage[french]{babel}
%\usepackage[utf8]{inputenc}
%\usepackage[T1]{fontenc}
\usepackage[pdfencoding=auto, pdfauthor={Hugo Delaunay}, pdfsubject={Mathématiques}, pdfcreator={agreg.skyost.eu}]{hyperref}
\usepackage{amsmath}
\usepackage{amsthm}
%\usepackage{amssymb}
\usepackage{stmaryrd}
\usepackage{tikz}
\usepackage{tkz-euclide}
\usepackage{fourier-otf}
\usepackage{fontspec}
\usepackage{titlesec}
\usepackage{fancyhdr}
\usepackage{catchfilebetweentags}
\usepackage[french, capitalise, noabbrev]{cleveref}
\usepackage[fit, breakall]{truncate}
\usepackage[top=2.5cm, right=2cm, bottom=2.5cm, left=2cm]{geometry}
\usepackage{enumerate}
\usepackage{tocloft}
\usepackage{microtype}
%\usepackage{mdframed}
%\usepackage{thmtools}
\usepackage{xcolor}
\usepackage{tabularx}
\usepackage{aligned-overset}
\usepackage[subpreambles=true]{standalone}
\usepackage{environ}
\usepackage[normalem]{ulem}
\usepackage{marginnote}
\usepackage{etoolbox}
\usepackage{setspace}
\usepackage[bibstyle=reading, citestyle=draft]{biblatex}
\usepackage{xpatch}
\usepackage[many, breakable]{tcolorbox}
\usepackage[backgroundcolor=white, bordercolor=white, textsize=small]{todonotes}

% Bibliographie :

\newcommand{\overridebibliographypath}[1]{\providecommand{\bibliographypath}{#1}}
\overridebibliographypath{../bibliography.bib}
\addbibresource{\bibliographypath}
\defbibheading{bibliography}[\bibname]{%
	\newpage
	\section*{#1}%
}
\renewbibmacro*{entryhead:full}{\printfield{labeltitle}}
\DeclareFieldFormat{url}{\newline\footnotesize\url{#1}}
\AtEndDocument{\printbibliography}

% Police :

\setmathfont{Erewhon Math}

% Tikz :

\usetikzlibrary{calc}

% Longueurs :

\setlength{\parindent}{0pt}
\setlength{\headheight}{15pt}
\setlength{\fboxsep}{0pt}
\titlespacing*{\chapter}{0pt}{-20pt}{10pt}
\setlength{\marginparwidth}{1.5cm}
\setstretch{1.1}

% Métadonnées :

\author{agreg.skyost.eu}
\date{\today}

% Titres :

\setcounter{secnumdepth}{3}

\renewcommand{\thechapter}{\Roman{chapter}}
\renewcommand{\thesubsection}{\Roman{subsection}}
\renewcommand{\thesubsubsection}{\arabic{subsubsection}}
\renewcommand{\theparagraph}{\alph{paragraph}}

\titleformat{\chapter}{\huge\bfseries}{\thechapter}{20pt}{\huge\bfseries}
\titleformat*{\section}{\LARGE\bfseries}
\titleformat{\subsection}{\Large\bfseries}{\thesubsection \, - \,}{0pt}{\Large\bfseries}
\titleformat{\subsubsection}{\large\bfseries}{\thesubsubsection. \,}{0pt}{\large\bfseries}
\titleformat{\paragraph}{\bfseries}{\theparagraph. \,}{0pt}{\bfseries}

\setcounter{secnumdepth}{4}

% Table des matières :

\renewcommand{\cftsecleader}{\cftdotfill{\cftdotsep}}
\addtolength{\cftsecnumwidth}{10pt}

% Redéfinition des commandes :

\renewcommand*\thesection{\arabic{section}}
\renewcommand{\ker}{\mathrm{Ker}}

% Nouvelles commandes :

\newcommand{\website}{https://agreg.skyost.eu}

\newcommand{\tr}[1]{\mathstrut ^t #1}
\newcommand{\im}{\mathrm{Im}}
\newcommand{\rang}{\operatorname{rang}}
\newcommand{\trace}{\operatorname{trace}}
\newcommand{\id}{\operatorname{id}}
\newcommand{\stab}{\operatorname{Stab}}

\providecommand{\newpar}{\\[\medskipamount]}

\providecommand{\lesson}[3]{%
	\title{#3}%
	\hypersetup{pdftitle={#3}}%
	\setcounter{section}{\numexpr #2 - 1}%
	\section{#3}%
	\fancyhead[R]{\truncate{0.73\textwidth}{#2 : #3}}%
}

\providecommand{\development}[3]{%
	\title{#3}%
	\hypersetup{pdftitle={#3}}%
	\section*{#3}%
	\fancyhead[R]{\truncate{0.73\textwidth}{#3}}%
}

\providecommand{\summary}[1]{%
	\textit{#1}%
	\medskip%
}

\tikzset{notestyleraw/.append style={inner sep=0pt, rounded corners=0pt, align=center}}

%\newcommand{\booklink}[1]{\website/bibliographie\##1}
\newcommand{\citelink}[2]{\hyperlink{cite.\therefsection @#1}{#2}}
\newcommand{\previousreference}{}
\providecommand{\reference}[2][]{%
	\notblank{#1}{\renewcommand{\previousreference}{#1}}{}%
	\todo[noline]{%
		\protect\vspace{16pt}%
		\protect\par%
		\protect\notblank{#1}{\cite{[\previousreference]}\\}{}%
		\protect\citelink{\previousreference}{p. #2}%
	}%
}

\definecolor{devcolor}{HTML}{00695c}
\newcommand{\dev}[1]{%
	\reversemarginpar%
	\todo[noline]{
		\protect\vspace{16pt}%
		\protect\par%
		\bfseries\color{devcolor}\href{\website/developpements/#1}{DEV}
	}%
	\normalmarginpar%
}

% En-têtes :

\pagestyle{fancy}
\fancyhead[L]{\truncate{0.23\textwidth}{\thepage}}
\fancyfoot[C]{\scriptsize \href{\website}{\texttt{agreg.skyost.eu}}}

% Couleurs :

\definecolor{property}{HTML}{fffde7}
\definecolor{proposition}{HTML}{fff8e1}
\definecolor{lemma}{HTML}{fff3e0}
\definecolor{theorem}{HTML}{fce4f2}
\definecolor{corollary}{HTML}{ffebee}
\definecolor{definition}{HTML}{ede7f6}
\definecolor{notation}{HTML}{f3e5f5}
\definecolor{example}{HTML}{e0f7fa}
\definecolor{cexample}{HTML}{efebe9}
\definecolor{application}{HTML}{e0f2f1}
\definecolor{remark}{HTML}{e8f5e9}
\definecolor{proof}{HTML}{e1f5fe}

% Théorèmes :

\theoremstyle{definition}
\newtheorem{theorem}{Théorème}

\newtheorem{property}[theorem]{Propriété}
\newtheorem{proposition}[theorem]{Proposition}
\newtheorem{lemma}[theorem]{Lemme}
\newtheorem{corollary}[theorem]{Corollaire}

\newtheorem{definition}[theorem]{Définition}
\newtheorem{notation}[theorem]{Notation}

\newtheorem{example}[theorem]{Exemple}
\newtheorem{cexample}[theorem]{Contre-exemple}
\newtheorem{application}[theorem]{Application}

\theoremstyle{remark}
\newtheorem{remark}[theorem]{Remarque}

\counterwithin*{theorem}{section}

\newcommand{\applystyletotheorem}[1]{
	\tcolorboxenvironment{#1}{
		enhanced,
		breakable,
		colback=#1!98!white,
		boxrule=0pt,
		boxsep=0pt,
		left=8pt,
		right=8pt,
		top=8pt,
		bottom=8pt,
		sharp corners,
		after=\par,
	}
}

\applystyletotheorem{property}
\applystyletotheorem{proposition}
\applystyletotheorem{lemma}
\applystyletotheorem{theorem}
\applystyletotheorem{corollary}
\applystyletotheorem{definition}
\applystyletotheorem{notation}
\applystyletotheorem{example}
\applystyletotheorem{cexample}
\applystyletotheorem{application}
\applystyletotheorem{remark}
\applystyletotheorem{proof}

% Environnements :

\NewEnviron{whitetabularx}[1]{%
	\renewcommand{\arraystretch}{2.5}
	\colorbox{white}{%
		\begin{tabularx}{\textwidth}{#1}%
			\BODY%
		\end{tabularx}%
	}%
}

% Maths :

\DeclareFontEncoding{FMS}{}{}
\DeclareFontSubstitution{FMS}{futm}{m}{n}
\DeclareFontEncoding{FMX}{}{}
\DeclareFontSubstitution{FMX}{futm}{m}{n}
\DeclareSymbolFont{fouriersymbols}{FMS}{futm}{m}{n}
\DeclareSymbolFont{fourierlargesymbols}{FMX}{futm}{m}{n}
\DeclareMathDelimiter{\VERT}{\mathord}{fouriersymbols}{152}{fourierlargesymbols}{147}


% Bibliographie :

\addbibresource{\bibliographypath}%
\defbibheading{bibliography}[\bibname]{%
	\newpage
	\section*{#1}%
}
\renewbibmacro*{entryhead:full}{\printfield{labeltitle}}%
\DeclareFieldFormat{url}{\newline\footnotesize\url{#1}}%

\AtEndDocument{\printbibliography}

\begin{document}
	%<*content>
	\lesson{analysis}{208}{Espaces vectoriels normés, applications linéaires continues. Exemples.}

	Dans toute la suite, $\mathbb{K}$ désignera le corps $\mathbb{R}$ ou $\mathbb{C}$ et $E$ un espace vectoriel sur $\mathbb{K}$.
	
	\subsection{Définitions et premières propriétés}
	
	\subsubsection{Espace vectoriel normé}
	
	\reference[GOU20]{7}
	
	\begin{definition}
		Une \textbf{norme} sur $E$ est une application $\Vert . \Vert : E \rightarrow \mathbb{R}^+$ telle que :
		\begin{enumerate}[(i)]
			\item $\Vert x \Vert = 0 \iff x = 0$ (séparabilité).
			\item $\forall \lambda \in \mathbb{K}, \, \forall x \in E, \, \Vert \lambda x \Vert = \vert \lambda \vert \Vert x \Vert$ (homogénéité).
			\item $\forall x, y \in E, \, \Vert x + y \Vert \leq \Vert x \Vert + \Vert y \Vert$ (inégalité triangulaire).
		\end{enumerate}
	\end{definition}
	
	\begin{example}
		\label{208-1}
		\begin{itemize}
			\item $x \mapsto \vert x \vert$ est une norme sur $\mathbb{R}$, $z \mapsto \vert z \vert$ est une norme sur $\mathbb{C}$.
			\item $\forall \alpha \geq 1, \Vert . \Vert_\alpha : (x_1, \dots, x_n) \mapsto \left(\sum_{i=1}^n \vert x_i \vert^\alpha\right)^{\frac{1}{\alpha}}$ est une norme sur $\mathbb{R}^n$.
		\end{itemize}
	\end{example}
	
	\reference{47}
	
	\begin{definition}
		$E$ est dit \textbf{normé} s'il est muni d'une norme $\Vert . \Vert$.
	\end{definition}
	
	Dans toute la suite, $E$ désignera un espace vectoriel normé muni d'une norme $\Vert . \Vert$.
	
	\begin{definition}
		Deux normes $\Vert . \Vert_1$ et $\Vert . \Vert_2$ sur $E$ sont dites \textbf{équivalentes} si
		\[ \exists a, b > 0 \text{ tels que } \forall x \in E, \, a \Vert x \Vert_1 \leq \Vert x \Vert_2 \leq b \Vert x \Vert_1 \]
	\end{definition}
	
	\begin{remark}
		Deux normes équivalentes définissent des distances équivalentes. Sur un plan topologique et lorsqu'on travaille avec des suites de Cauchy, il est indifférent de prendre l'une ou l'autre de ces normes.
	\end{remark}
	
	\subsubsection{Quelques exemples}
	
	\begin{example}
		Comme mentionné précédemment, $\mathbb{R}^n$ et $\mathbb{C}^n$ sont des espaces vectoriels normés (munis de $\Vert . \Vert_\alpha$ définie à l'\cref{208-1}).
	\end{example}
	
	\reference{8}
	
	\begin{example}
		L'ensemble $\mathcal{B}(X,E)$ des applications bornées d'un ensemble $X$ dans $E$ est un espace vectoriel normé muni de la norme $\Vert . \Vert_\infty : f \mapsto \sup_{x \in X} \vert f(x) \vert$.
	\end{example}
	
	\reference{53}
	
	\begin{example}
		\begin{itemize}
			\item $\ell_1(\mathbb{R}) = \{ (u_n) \in \mathbb{R}^n \mid \sum_{n=0}^{+\infty} \vert u_n \vert < +\infty \}$ est un espace vectoriel normé muni de la norme $\Vert (u_n) \Vert_1 = \sum_{n=0}^{+\infty} \vert u_n \vert$.
			\item $\ell_\infty(\mathbb{R}) = \{ (u_n) \in \mathbb{R}^n \mid (u_n) \text{ est bornée} \}$ est un espace vectoriel normé muni de la norme $\Vert (u_n) \Vert_\infty = \sup_{n \in \mathbb{N}} \vert u_n \vert$.
		\end{itemize}
	\end{example}
	
	\subsubsection{Applications linéaires continues}
	
	Soit $(F, \Vert . \Vert_F)$ un espace vectoriel normé sur $\mathbb{K}$. $\Vert . \Vert_E$ désigne la norme sur $E$.
	
	\begin{notation}
		On note $L(E,F)$ l'ensemble des applications linéaires de $E$ dans $F$ et $\mathcal{L}(E,F)$ l'ensemble des applications linéaires continues de $E$ dans $F$. Si $E = F$, on note $L(E,F) = L(E)$ et $\mathcal{L}(E,F) = \mathcal{L}(E)$.
	\end{notation}
	
	\begin{theorem}
		Soit $f \in L(E,F)$. Les assertions suivantes sont équivalentes.
		\begin{enumerate}[(i)]
			\item $f \in \mathcal{L}(E,F)$.
			\item $f$ est continue en $0$.
			\item $f$ est bornée sur $\overline{B}(0,1) \subseteq E$.
			\item $f$ est bornée sur $S(0,1) \subseteq E$.
			\item Il existe $M \geq 0$ tel que $\Vert f(x) \Vert_F \leq M \Vert x \Vert_E$.
			\item $f$ est lipschitzienne.
			\item $f$ est uniformément continue sur $E$.
		\end{enumerate}
	\end{theorem}
	
	\begin{corollary}
		L'application $\VERT . \VERT : f \mapsto \sup_{\Vert x \Vert_E = 1} \Vert f(x) \Vert_F = \sup_{x \neq 0} \frac{\Vert f(x) \Vert_F}{\Vert x \Vert_E}$ est correctement définie sur $\mathcal{L}(E,F)$ et définit une norme sur cet espace.
	\end{corollary}
	
	\begin{remark}
		Le réel $\VERT f \VERT$ du corollaire précédent est le plus petit réel positif $M$ tel que $\Vert f(x) \Vert_F \leq M \Vert x \Vert_E$ pour tout $x \in E$. En particulier,
		\[ \forall x \in E, \, \Vert f(x) \Vert_F \leq \VERT f \VERT \Vert x \Vert_E \]
	\end{remark}
	
	\begin{proposition}
		Soient $(G, \Vert . \Vert_G)$ un espace vectoriel normé, $f \in \mathcal{L}(E,F)$ et $g \in \mathcal{L}(F,G)$. Alors, $\VERT g \circ f \VERT \leq \VERT g \VERT \VERT f \VERT$.
	\end{proposition}
	
	\begin{proposition}
		Si $f \in \mathcal{L}(E,F)$ est inversible, $\VERT f \VERT^{-1} \leq \VERT f^{-1} \VERT$. 
	\end{proposition}
	
	\begin{proposition}
		Une forme linéaire sur $E$ (ie. un élément de $L(E, \mathbb{K}) = E^*$) est continue (ie. est un élément de $\mathcal{L}(E, \mathbb{K}) = E'$) si et seulement si son noyau est fermé.
	\end{proposition}
	
	\reference[LI]{19}
	
	\begin{example}
		L'application
		\[
		\delta_0 : \begin{array}{ccc}
			\mathcal{C}([0,1], \mathbb{K}) &\rightarrow& \mathbb{K} \\
			f &\mapsto& f(0)
		\end{array}
		\]
		est continue pour $\Vert . \Vert_\infty$ mais pas pour $\Vert . \Vert_1$ (où $\Vert . \Vert_1 = \int_{[0,1]} \vert . \vert \mathrm{d}\mu$ et $\Vert . \Vert_\infty = \sup_{[0,1]}$).
	\end{example}
	
	\subsection{Étude en dimension finie}
	
	On se place ici dans le cas où $E$ est de dimension finie.
	
	\begin{theorem}
		Dans un espace vectoriel normé de dimension finie, toutes les normes sont équivalentes.
	\end{theorem}
	
	\begin{corollary}
		Toute application continue d'un espace vectoriel normé de dimension finie dans un espace vectoriel normé (quelconque) est continue.
	\end{corollary}
	
	\begin{corollary}
		Tout sous-espace vectoriel d'un espace vectoriel normé de dimension finie est fermé.
	\end{corollary}
	
	\begin{corollary}
		Les parties compactes d'un espace vectoriel normé de dimension finie sont les parties fermées et bornées.
	\end{corollary}
	
	\begin{cexample}
		Munir $\mathbb{R}[X]$ de la norme $\Vert . \Vert_\infty \mapsto \sum_{i} a_i X^i \mapsto \sup_i \vert a_i \vert$ rend l'opérateur de dérivation $P \mapsto P'$ non continu.
	\end{cexample}
	
	\reference{56}
	
	\begin{theorem}[Riesz]
		La boule unité fermée d'un espace vectoriel normé est compacte si et seulement s'il est dimension finie.
	\end{theorem}
	
	\subsection{Complétude}
	
	\subsubsection{Espaces de Banach}
	
	\reference[LI]{20}
	
	\begin{definition}
		Un espace vectoriel normé complet (ie. dans lequel toute suite de Cauchy converge) est un \textbf{espace de Banach}.
	\end{definition}
	
	\reference{50}
	
	\begin{example}
		Tout espace vectoriel normé de dimension finie est complet.
	\end{example}
	
	\begin{example}
		Soit $F$ un espace de Banach. Alors $\mathcal{L}(E,F)$ est un espace de Banach.
	\end{example}
	
	\reference{21}
	
	\begin{example}
		Soient $X$ un ensemble. On suppose que $E$ un espace de Banach. Alors $\mathcal{B}(X,E)$ est un espace de Banach.
	\end{example}
	
	\reference[LI]{10}
	
	\begin{example}
		Pour tout compact $K$ de $\mathbb{R}$, $(\mathcal{C}(K, \mathbb{K}), \Vert . \Vert_\infty)$ est complet. Mais pas $(\mathcal{C}([0,1], \mathbb{K}), \Vert . \Vert_1)$.
	\end{example}
	
	\begin{theorem}[Riesz-Fischer]
		$\forall p \in [1, +\infty[, \, L_p$ est un espace de Banach.
	\end{theorem}
	
	\reference[GOU20]{52}
	
	\begin{proposition}
		$E$ est de Banach si et seulement si toute série de $E$ absolument convergente est convergente.
	\end{proposition}
	
	\reference[LI]{111}
	
	\begin{theorem}[Baire]
		On suppose $E$ complet. Alors toute intersection d'ouvert denses est encore dense dans $E$.
	\end{theorem}
	
	\reference[GOU20]{419}
	
	\begin{application}
		Un espace vectoriel normé à base dénombrable n'est pas complet.
	\end{application}
	
	\reference[LI]{112}
	
	\begin{application}[Théorème de Banach-Steinhaus]
		Soient $(E, \Vert . \Vert_E)$ et $(F, \Vert . \Vert_F)$ deux espaces de Banach et $(T_i)_{i \in I}$ des applications linéaires continues telles que
		\[ \forall x \in E, \, \sup_{i \in I} \Vert T_i(x) \Vert_F < +\infty \]
		alors,
		\[ \sup_{i \in I} \VERT T_i \VERT < +\infty \]
	\end{application}
	
	\begin{application}[Théorème du graphe fermé]
		Soient $E$ et $F$ deux espaces de Banach et $T \in L(E,F)$. Si le graphe de $T$ :
		\[ \{ (x, T(x)) \mid x \in E \} \subseteq E \times F \]
		est fermé dans $E \times F$, alors $T$ est continue.
	\end{application}
	
		\begin{application}[Théorème de l'application ouverte]
		Soient $E$ et $F$ deux espaces de Banach et $T \in \mathcal{L}(E,F)$ surjective. Alors,
		\[ \exists c > 0, \, T\left(B_E(0,1)\right) \supseteq B_F(0,c) \]
	\end{application}
	
	\begin{corollary}[Théorème des isomorphismes de Banach]
		Soient $E$ et $F$ deux espaces de Banach et $T \in \mathcal{L}(E,F)$ bijective. Alors $T^{-1}$ est continue.
	\end{corollary}
	
	\begin{corollary}
		On suppose que $E$ est de Banach. Soient $E_1$ et $E_2$ deux supplémentaires algébriques fermés dans $E$. Alors les projections associées sur $E_1$ et $E_2$ sont continues.
	\end{corollary}
	
	\subsubsection{Espaces de Hilbert}
	
	\paragraph{Généralités}
	
	\reference[LI]{31}
	
	\begin{definition}
		Un espace vectoriel $H$ sur le corps $\mathbb{K}$ est un \textbf{espace de Hilbert} s'il est muni d'un produit scalaire $\langle . , . \rangle$ et est complet pour la norme associée $\Vert . \Vert = \sqrt{\langle . , . \rangle}$.
	\end{definition}
	
	\begin{example}
		Tout espace euclidien / hermitien est de Hilbert.
	\end{example}
	
	\begin{example}
		$L_2(\mu)$ muni de $\langle . , . \rangle : (f,g) \mapsto \int f \overline{g} \, \mathrm{d}\mu$ est un espace de Hilbert.
	\end{example}
	
	Pour toute la suite, on fixe $H$ un espace de Hilbert de norme $\Vert . \Vert$ et dont on note $\langle ., . \rangle$ le produit scalaire associé.

	\begin{lemma}[Identité du parallélogramme]
		Soient $x, y \in H$. Alors :
		\[ \Vert x + y \Vert^2 + \Vert x - y \Vert^2 = 2(\Vert x \Vert^2 + \Vert y \Vert^2) \]
	\end{lemma}
	
	\dev{projection-sur-un-convexe-ferme}
	
	\begin{theorem}[Projection sur un convexe fermé]
		Soit $C \subset H$ un convexe fermé non-vide. Alors :
		\[ \forall x \in H, \exists! y \in C \text{ tel que } d(x, C) = \inf_{z \in C} \Vert x - z \Vert = d(x, y) \]
		On peut donc noter $y = P_C(x)$, le \textbf{projeté orthogonal de $x$ sur $C$}. Il s'agit de l'unique point de $C$ vérifiant
		\[ \forall z \in C, \, \langle x - P_C(x), z - P_C(x) \rangle \leq 0 \]
		\begin{center}
			\begin{tikzpicture}
				\def\a{3}
				\def\b{1.5}
				\def\angle{20}
				
				\coordinate (O) at (0, 0);
				\coordinate (X) at (5.6, 2.54);
				\coordinate (Z) at (1.8, -0.8);
				\coordinate (P) at (\angle:{\a} and {\b});
				\coordinate (F1) at ({-sqrt(\a*\a-\b*\b)}, 0);
				\coordinate (F2) at ({+sqrt(\a*\a-\b*\b)}, 0);
				\tkzDefLine[bisector out](F1,P,F2) \tkzGetPoint{K}
				\coordinate (A) at ($(K)!+0.5!(P)$);
				\coordinate (B) at ($(P)!-0.5!(K)$);
				
				\draw[fill=blue!30, fill opacity=0.3] (O) ellipse ({\a} and {\b});
				
				\tkzMarkRightAngle[color=red](X,P,K);
				\tkzMarkAngle[->,size=0.8,color=cyan,mark=none](Z,P,X);
				\tkzLabelAngle[pos=0,shift={(2,0)}](Z,P,X){\color{cyan} Angle obtus};
				
				\draw(O) node {$C$};
				\draw(P) node{$\bullet$} node[left]{$P_C(x)$};
				\draw(X) node{$\bullet$} node[above right]{$x$};
				\draw(Z) node{$\bullet$} node[above left]{$z$};
				
				\draw[dashed] (A) -- (B);
				\draw[dashed] ($(A)!(X)!(B)$) -- (X);
				\draw[dashed] (Z) -- (P);
			\end{tikzpicture}
		\end{center}
	\end{theorem}
	
	\begin{theorem}
		Si $F$ est un sous espace vectoriel fermé dans $H$, alors $P_F$ est une application linéaire continue. De plus, pour tout $x \in H$, $P_F(x)$ est l'unique point $y \in F$ tel que $x-y \in F^\perp$.
	\end{theorem}
	
	\begin{corollary}
		$F$ est un sous espace vectoriel dense dans $H$ si et seulement si $F^\perp = \{ 0 \}$.
	\end{corollary}
	
	\begin{theorem}[Représentation de Riesz]
		\[ \forall \varphi \in \mathcal{L}(H, \mathbb{K}), \, \exists! y \in H \text{ tel que } \forall x \in H, \, \varphi(x) = \langle x, y \rangle \]
	\end{theorem}
	
	\begin{corollary}
		\[ \forall T \in H', \, \exists! U \in H' \text{ tel que } \forall x, y \in H, \, \langle T(x), y \rangle = \langle x, U(y) \rangle \]
		On note alors $U = T^*$ : c'est \textbf{l'adjoint} de $T$. On a alors $\VERT T \VERT = \VERT T^* \VERT$.
	\end{corollary}
	
	\reference{65}
	
	\begin{example}[Opérateur de Voltera]
		On définit $T$ sur $H = L_2([0,1])$ par :
		\[
		T : \begin{array}{ccc}
			H &\rightarrow& H \\
			f &\mapsto& x \mapsto \int_{0}^{x} f(t) \, \mathrm{d}t
		\end{array}
		\]
		$T$ est une application linéaire continue et son adjoint $T^*$ est défini par :
		\[ T^* : g \mapsto \left(x \mapsto \int_x^1 g(t) \, \mathrm{d}t \right) \]
	\end{example}
	
	\reference[Z-Q]{216}
	\dev{dual-de-lp}
	
	\begin{theorem}
		L'application
		\[
		\varphi :
		\begin{array}{ll}
			L_q &\rightarrow (L_p)' \\
			g &\mapsto \left( \varphi_g : f \mapsto \int_X f g \, \mathrm{d}\mu \right)
		\end{array}
		\qquad \text{ où } \frac{1}{p} + \frac{1}{q} = 1
		\]
		est une isométrie linéaire surjective. C'est donc un isomorphisme isométrique.
	\end{theorem}
	
	\paragraph{Bases hilbertiennes}
	
	\reference{43}
	
	\begin{definition}
		On dit que $(e_n) \in H^{\mathbb{N}}$ est une \textbf{base hilbertienne} de $H$ si
		\begin{itemize}
			\item $(e_n)$ est orthonormale.
			\item $(e_n)$ est totale.
		\end{itemize}
	\end{definition}
	
	\begin{example}
		$(t \mapsto e^{2\pi int})_{n \in \mathbb{Z}}$ est une base hilbertienne de $L_2([0,1])$.
	\end{example}
	
	\begin{theorem}
		Soit $(e_n)_{n \in \mathbb{N}}$ une base hilbertienne de $H$. Alors :
		\[ \forall x \in H, \, x = \sum_{n=0}^{+\infty} \langle x, e_n \rangle e_n \]
		On a de plus, pour tout $x, y \in H$, les formules de Parseval :
		\begin{itemize}
			\item $\Vert x \Vert^2 = \sum_{n=0}^{+\infty} \vert \langle x, e_n \rangle \vert^2$.
			\item $\langle x, y \rangle = \sum_{n=0}^{+\infty} \langle x, e_n \rangle \overline{\langle y, e_n \rangle}$.
		\end{itemize}
	\end{theorem}
	%</content>
\end{document}
