\documentclass[12pt, a4paper]{report}

% LuaLaTeX :

\RequirePackage{iftex}
\RequireLuaTeX

% Packages :

\usepackage[french]{babel}
%\usepackage[utf8]{inputenc}
%\usepackage[T1]{fontenc}
\usepackage[pdfencoding=auto, pdfauthor={Hugo Delaunay}, pdfsubject={Mathématiques}, pdfcreator={agreg.skyost.eu}]{hyperref}
\usepackage{amsmath}
\usepackage{amsthm}
%\usepackage{amssymb}
\usepackage{stmaryrd}
\usepackage{tikz}
\usepackage{tkz-euclide}
\usepackage{fourier-otf}
\usepackage{fontspec}
\usepackage{titlesec}
\usepackage{fancyhdr}
\usepackage{catchfilebetweentags}
\usepackage[french, capitalise, noabbrev]{cleveref}
\usepackage[fit, breakall]{truncate}
\usepackage[top=2.5cm, right=2cm, bottom=2.5cm, left=2cm]{geometry}
\usepackage{enumerate}
\usepackage{tocloft}
\usepackage{microtype}
%\usepackage{mdframed}
%\usepackage{thmtools}
\usepackage{xcolor}
\usepackage{tabularx}
\usepackage{aligned-overset}
\usepackage[subpreambles=true]{standalone}
\usepackage{environ}
\usepackage[normalem]{ulem}
\usepackage{marginnote}
\usepackage{etoolbox}
\usepackage{setspace}
\usepackage[bibstyle=reading, citestyle=draft]{biblatex}
\usepackage{xpatch}
\usepackage[many, breakable]{tcolorbox}
\usepackage[backgroundcolor=white, bordercolor=white, textsize=small]{todonotes}

% Bibliographie :

\newcommand{\overridebibliographypath}[1]{\providecommand{\bibliographypath}{#1}}
\overridebibliographypath{../bibliography.bib}
\addbibresource{\bibliographypath}
\defbibheading{bibliography}[\bibname]{%
	\newpage
	\section*{#1}%
}
\renewbibmacro*{entryhead:full}{\printfield{labeltitle}}
\DeclareFieldFormat{url}{\newline\footnotesize\url{#1}}
\AtEndDocument{\printbibliography}

% Police :

\setmathfont{Erewhon Math}

% Tikz :

\usetikzlibrary{calc}

% Longueurs :

\setlength{\parindent}{0pt}
\setlength{\headheight}{15pt}
\setlength{\fboxsep}{0pt}
\titlespacing*{\chapter}{0pt}{-20pt}{10pt}
\setlength{\marginparwidth}{1.5cm}
\setstretch{1.1}

% Métadonnées :

\author{agreg.skyost.eu}
\date{\today}

% Titres :

\setcounter{secnumdepth}{3}

\renewcommand{\thechapter}{\Roman{chapter}}
\renewcommand{\thesubsection}{\Roman{subsection}}
\renewcommand{\thesubsubsection}{\arabic{subsubsection}}
\renewcommand{\theparagraph}{\alph{paragraph}}

\titleformat{\chapter}{\huge\bfseries}{\thechapter}{20pt}{\huge\bfseries}
\titleformat*{\section}{\LARGE\bfseries}
\titleformat{\subsection}{\Large\bfseries}{\thesubsection \, - \,}{0pt}{\Large\bfseries}
\titleformat{\subsubsection}{\large\bfseries}{\thesubsubsection. \,}{0pt}{\large\bfseries}
\titleformat{\paragraph}{\bfseries}{\theparagraph. \,}{0pt}{\bfseries}

\setcounter{secnumdepth}{4}

% Table des matières :

\renewcommand{\cftsecleader}{\cftdotfill{\cftdotsep}}
\addtolength{\cftsecnumwidth}{10pt}

% Redéfinition des commandes :

\renewcommand*\thesection{\arabic{section}}
\renewcommand{\ker}{\mathrm{Ker}}

% Nouvelles commandes :

\newcommand{\website}{https://agreg.skyost.eu}

\newcommand{\tr}[1]{\mathstrut ^t #1}
\newcommand{\im}{\mathrm{Im}}
\newcommand{\rang}{\operatorname{rang}}
\newcommand{\trace}{\operatorname{trace}}
\newcommand{\id}{\operatorname{id}}
\newcommand{\stab}{\operatorname{Stab}}

\providecommand{\newpar}{\\[\medskipamount]}

\providecommand{\lesson}[3]{%
	\title{#3}%
	\hypersetup{pdftitle={#3}}%
	\setcounter{section}{\numexpr #2 - 1}%
	\section{#3}%
	\fancyhead[R]{\truncate{0.73\textwidth}{#2 : #3}}%
}

\providecommand{\development}[3]{%
	\title{#3}%
	\hypersetup{pdftitle={#3}}%
	\section*{#3}%
	\fancyhead[R]{\truncate{0.73\textwidth}{#3}}%
}

\providecommand{\summary}[1]{%
	\textit{#1}%
	\medskip%
}

\tikzset{notestyleraw/.append style={inner sep=0pt, rounded corners=0pt, align=center}}

%\newcommand{\booklink}[1]{\website/bibliographie\##1}
\newcommand{\citelink}[2]{\hyperlink{cite.\therefsection @#1}{#2}}
\newcommand{\previousreference}{}
\providecommand{\reference}[2][]{%
	\notblank{#1}{\renewcommand{\previousreference}{#1}}{}%
	\todo[noline]{%
		\protect\vspace{16pt}%
		\protect\par%
		\protect\notblank{#1}{\cite{[\previousreference]}\\}{}%
		\protect\citelink{\previousreference}{p. #2}%
	}%
}

\definecolor{devcolor}{HTML}{00695c}
\newcommand{\dev}[1]{%
	\reversemarginpar%
	\todo[noline]{
		\protect\vspace{16pt}%
		\protect\par%
		\bfseries\color{devcolor}\href{\website/developpements/#1}{DEV}
	}%
	\normalmarginpar%
}

% En-têtes :

\pagestyle{fancy}
\fancyhead[L]{\truncate{0.23\textwidth}{\thepage}}
\fancyfoot[C]{\scriptsize \href{\website}{\texttt{agreg.skyost.eu}}}

% Couleurs :

\definecolor{property}{HTML}{fffde7}
\definecolor{proposition}{HTML}{fff8e1}
\definecolor{lemma}{HTML}{fff3e0}
\definecolor{theorem}{HTML}{fce4f2}
\definecolor{corollary}{HTML}{ffebee}
\definecolor{definition}{HTML}{ede7f6}
\definecolor{notation}{HTML}{f3e5f5}
\definecolor{example}{HTML}{e0f7fa}
\definecolor{cexample}{HTML}{efebe9}
\definecolor{application}{HTML}{e0f2f1}
\definecolor{remark}{HTML}{e8f5e9}
\definecolor{proof}{HTML}{e1f5fe}

% Théorèmes :

\theoremstyle{definition}
\newtheorem{theorem}{Théorème}

\newtheorem{property}[theorem]{Propriété}
\newtheorem{proposition}[theorem]{Proposition}
\newtheorem{lemma}[theorem]{Lemme}
\newtheorem{corollary}[theorem]{Corollaire}

\newtheorem{definition}[theorem]{Définition}
\newtheorem{notation}[theorem]{Notation}

\newtheorem{example}[theorem]{Exemple}
\newtheorem{cexample}[theorem]{Contre-exemple}
\newtheorem{application}[theorem]{Application}

\theoremstyle{remark}
\newtheorem{remark}[theorem]{Remarque}

\counterwithin*{theorem}{section}

\newcommand{\applystyletotheorem}[1]{
	\tcolorboxenvironment{#1}{
		enhanced,
		breakable,
		colback=#1!98!white,
		boxrule=0pt,
		boxsep=0pt,
		left=8pt,
		right=8pt,
		top=8pt,
		bottom=8pt,
		sharp corners,
		after=\par,
	}
}

\applystyletotheorem{property}
\applystyletotheorem{proposition}
\applystyletotheorem{lemma}
\applystyletotheorem{theorem}
\applystyletotheorem{corollary}
\applystyletotheorem{definition}
\applystyletotheorem{notation}
\applystyletotheorem{example}
\applystyletotheorem{cexample}
\applystyletotheorem{application}
\applystyletotheorem{remark}
\applystyletotheorem{proof}

% Environnements :

\NewEnviron{whitetabularx}[1]{%
	\renewcommand{\arraystretch}{2.5}
	\colorbox{white}{%
		\begin{tabularx}{\textwidth}{#1}%
			\BODY%
		\end{tabularx}%
	}%
}

% Maths :

\DeclareFontEncoding{FMS}{}{}
\DeclareFontSubstitution{FMS}{futm}{m}{n}
\DeclareFontEncoding{FMX}{}{}
\DeclareFontSubstitution{FMX}{futm}{m}{n}
\DeclareSymbolFont{fouriersymbols}{FMS}{futm}{m}{n}
\DeclareSymbolFont{fourierlargesymbols}{FMX}{futm}{m}{n}
\DeclareMathDelimiter{\VERT}{\mathord}{fouriersymbols}{152}{fourierlargesymbols}{147}


% Bibliographie :

\addbibresource{\bibliographypath}%
\defbibheading{bibliography}[\bibname]{%
	\newpage
	\section*{#1}%
}
\renewbibmacro*{entryhead:full}{\printfield{labeltitle}}%
\DeclareFieldFormat{url}{\newline\footnotesize\url{#1}}%

\AtEndDocument{\printbibliography}

\begin{document}
	%<*content>
	\lesson{analysis}{229}{Fonctions monotones. Fonctions convexes. Exemples et applications.}

	\subsection{Fonctions monotones}

	\subsubsection{Définition et première propriétés}

	\reference[R-R]{31}

	\begin{definition}
		Soient $X$ une partie de $\mathbb{R}$ et $f : X \rightarrow \mathbb{R}$.
		\begin{itemize}
			\item On dit que $f$ est \textbf{croissante} si $\forall x, y \in X, \, x \leq y \implies f(x) \leq f(y)$.
			\item On dit que $f$ est \textbf{décroissante} si $\forall x, y \in X, \, x \leq y \implies f(x) \geq f(y)$.
			\item On dit que $f$ est \textbf{monotone} si $f$ est croissante ou décroissante.
		\end{itemize}
	\end{definition}

	\begin{remark}
		Les définitions de $f$ \textbf{strictement croissante} et $f$ \textbf{strictement décroissante} s'obtiennent en remplaçant les inégalités larges par des inégalités strictes dans la définition précédente.
	\end{remark}

	Par conséquent, $f$ est décroissante si et seulement si $-f$ est croissante. Pour cette raison, nous pouvons nous limiter à l'étude des fonctions croissantes.

	\begin{example}
		$x \mapsto \lfloor x \rfloor$ est une fonction monotone.
	\end{example}

	\reference[D-L]{405}

	\begin{proposition}
		L'ensemble des fonctions croissantes est stable par addition, par multiplication par un scalaire positif et par composition.
	\end{proposition}

	\reference[ROM19]{205}

	\begin{proposition}
		Soient $I \subseteq \mathbb{R}$ un intervalle non réduit à un point et $f : I \rightarrow \mathbb{R}$. On suppose $f$ dérivable sur $\mathring{I}$. Alors $f$ est croissante si et seulement si $f'(x) \geq 0$ pour tout $x \in I$.
	\end{proposition}

	\subsubsection{Régularité}

	\reference{162}

	Soit $I \subseteq \mathbb{R}$ un intervalle non réduit à un point.

	\begin{definition}
		On dit que $f : I \rightarrow \mathbb{R}$ a pour \textbf{limite à gauche} (resp. \textbf{à droite}) $\ell$ en $\alpha \in \overline{I}$ si :
		\[ \forall \epsilon > 0, \, \exists \eta > 0, \, \text{tel que } \forall x \in I \, \cap \, ]\alpha-\eta, \alpha[, \, |f(x) - \ell| < \epsilon \]
		(resp. $\forall \epsilon > 0, \, \exists \eta > 0, \, \text{tel que } \forall x \in I \, \cap \, ]\alpha, \alpha+\eta[, \, |f(x) - \ell| < \epsilon$).
	\end{definition}

	\begin{theorem}
		On suppose que $I$ est un intervalle ouvert. Si $f : I \rightarrow \mathbb{R}$ est une fonction monotone, elle admet alors une limite à gauche et à droite en tout point. Dans le cas où $f$ est croissante, on a
		\[ \forall x \in I, \quad f(x^-) = \sup_{\substack{t \in I \\ t < x}} f(t) \leq f(x) \leq f(x^+) = \inf_{\substack{t \in I \\ t > x}} f(t) \]
	\end{theorem}

	\begin{definition}
		Si $\alpha \in \mathring{I}$, et si $f : I \rightarrow \mathbb{R}$ est discontinue en $\alpha$ avec des limites à gauche et à droite en ce point, on dit que $f$ a une \textbf{discontinuité de première espèce} en $\alpha$.
	\end{definition}

	\begin{proposition}
		Une fonction monotone de $I$ dans $\mathbb{R}$ ne peut avoir que des discontinuités de première espèce.
	\end{proposition}

	\begin{theorem}
		On suppose que $I$ est un intervalle ouvert. Si $f : I \rightarrow \mathbb{R}$ est une fonction monotone, alors l'ensemble des points de discontinuités de $f$ est dénombrable.
	\end{theorem}

	\begin{example}
		La fonction $f$ définie sur $[0,1]$ par $f(0) = 0$ et $f(x) = \frac{1}{\lfloor \frac{1}{x} \rfloor}$ est croissante avec une infinité de points de discontinuité.
	\end{example}

	\reference{175}

	\begin{proposition}
		Si $f : I \rightarrow \mathbb{R}$ est une fonction monotone telle que $f(I)$ est un intervalle, elle est alors continue sur $I$.
	\end{proposition}

	\begin{theorem}[Bijection]
		Si $f : I \rightarrow \mathbb{R}$ est une application continue et strictement monotone, alors :
		\begin{enumerate}[(i)]
			\item $f(I)$ est un intervalle.
			\item $f^{-1}$ est continue.
			\item $f^{-1}$ est strictement monotone de même sens de variation que $f$.
		\end{enumerate}
	\end{theorem}

	\begin{example}
		La fonction $\exp : x \mapsto e^x$ est une bijection de $\mathbb{R}$ dans $\mathbb{R}^{+}_{*}$ qui admet donc une bijection réciproque $\ln$ qui est strictement croissante.
	\end{example}

	\begin{proposition}
		Soit $f : I \rightarrow \mathbb{R}$. Cette fonction $f$ est injective si et seulement si elle est strictement monotone.
	\end{proposition}

	\reference[D-L]{405}

	\begin{theorem}[Lebesgue]
		Une application monotone est dérivable presque partout.
	\end{theorem}

	\subsubsection{Suites et séries}

	\reference[GOU20]{238}

	\begin{lemma}
		Une limite simple d'une suite de fonctions croissantes est croissante.
	\end{lemma}

	\begin{theorem}[Second théorème de Dini]
		Soit $(f_n)$ une suite de fonctions croissantes réelles continues définies sur un segment $I$ de $\mathbb{R}$. Si $(f_n)$ converge simplement vers une fonction continue sur $I$, alors la convergence est uniforme.
	\end{theorem}

	\reference{212}

	\begin{proposition}[Comparaison série - intégrale]
		Soit $f : \mathbb{R}^+ \rightarrow \mathbb{R}^+$ une fonction positive, continue par morceaux et décroissante sur $\mathbb{R}^+$. Alors la suite $(U_n)$ définie par
		\[ \forall n \in \mathbb{N}, \, \sum_{k=0}^n f(k) - \int_0^n f(t) \, \mathrm{d}t \]
		est convergente. En particulier, la série $\sum f(n)$ et l'intégrale $\int_0^{+\infty} f(t) \, \mathrm{d}t$ sont de même nature.
	\end{proposition}

	\begin{application}[Développement asymptotique de la série harmonique]
		\[ \sum_{k=1}^n \frac{1}{k} = \ln(n) + \gamma + o (1) \]
		où $\gamma$ désigne la constante d'Euler.
	\end{application}

	\subsection{Fonctions convexes}

	Soit $I$ une partie d'un espace vectoriel normé $(E, \Vert . \Vert)$ non réduite à un point.

	\subsubsection{Définitions}

	\reference[ROM19]{225}

	\begin{definition}
		\label{229-2}
		\begin{itemize}
			\item $I$ est \textbf{convexe} si $\forall a, b \in I, \, [a,b] \subseteq I$.
			\item Une fonction $f : I \rightarrow \mathbb{R}$ est \textbf{convexe} si
			\[ \forall x, y \in I, \, \forall t \in [0,1], \, f((1-t)x + ty) \leq (1-t)f(x) + tf(y) \]
			\item Une fonction $f : I \rightarrow \mathbb{R}$ est \textbf{concave} si $-f$ est convexe.
		\end{itemize}
	\end{definition}

	\begin{remark}
		Les définitions de $f$ \textbf{strictement convexe} et $f$ \textbf{strictement concave} s'obtiennent en remplaçant les inégalités larges par des inégalités strictes dans la définition précédente.
	\end{remark}

	\begin{example}
		\begin{itemize}
			\item $x \mapsto \Vert x \Vert$ est convexe sur $E$.
			\item $\exp$ est convexe sur $\mathbb{R}$.
		\end{itemize}
	\end{example}

	\begin{proposition}
		Une fonction $f : I \rightarrow \mathbb{R}$ est convexe si et seulement si son épigraphe est convexe dans $E \times \mathbb{R}$.
	\end{proposition}

	\begin{theorem}
		Une fonction $f : I \rightarrow \mathbb{R}$ est convexe si et seulement si $\forall x, y \in I$, $t : \mapsto f((1-t)x + ty)$ est convexe sur $[0,1]$.
	\end{theorem}

	\begin{proposition}
		\begin{itemize}
			\item Une combinaison linéaire à coefficients positifs de fonctions convexes est convexe.
			\item La composée $\varphi \circ g$ d'une fonction convexe croissante $\varphi : J \rightarrow \mathbb{R}$ avec une fonction fonction convexe $g : I \rightarrow J$ est croissante.
			\item Une limite simple d'une suite de fonctions convexes est convexe.
		\end{itemize}
	\end{proposition}

	Ce dernier théorème justifie que l'étude des fonctions convexes se ramène à l'étude des fonctions convexes sur un intervalle réelle. À partir de maintenant, on supposera donc que $I$ est un intervalle réel non réduit à un point.

	\subsubsection{Propriétés sur \texorpdfstring{$\mathbb{R}$}{R}}

	\reference[GOU20]{95}

	\begin{remark}
		Dans le cadre réel, la \cref{229-2} revient à dire que les cordes $[(a, f(a)), (b, f(b))]$ sont au-dessus du graphe de $f$ pour tout $a, b \in I$ avec $a < b$.
	\end{remark}

	\begin{proposition}
		Une fonction $f : I \rightarrow \mathbb{R}$ est convexe si et seulement si $\forall x_0 \in I$, l'application
		\[
		\begin{array}{ccc}
			I \setminus \{ x_0 \} &\rightarrow& \mathbb{R} \\
			x &\mapsto& \frac{f(x) - f(x_0)}{x - x_0}
		\end{array}
		\]
		est croissante.
	\end{proposition}

	\begin{corollary}[Inégalité des trois pentes]
		Soient fonction $f : I \rightarrow \mathbb{R}$ convexe et $a, b, c \in I$ tels que $a < b < c$. Alors,
		\[ \frac{f(b) - f(a)}{b-a} < \frac{f(c) - f(a)}{c-a} < \frac{f(c) - f(b)}{c-b} \]
	\end{corollary}

	\reference{71}

	\begin{definition}
		On dit que $f : I \rightarrow \mathbb{R}$ est \textbf{dérivable à gauche} (resp. \textbf{à droite}) en $\alpha \in I$ si la limite
		\[ \lim_{\substack{t \rightarrow a^{-} \\ t \in I}} \frac{f(t) - f(a)}{t-a} \]
		(resp. $\lim_{\substack{t \rightarrow a^{+} \\ t \in I}} \frac{f(t) - f(a)}{t-a}$) existe.
	\end{definition}

	\reference{96}

	\begin{proposition}
		Une fonction $f : I \rightarrow \mathbb{R}$ convexe possède en tout point de $\mathring{I}$ une dérivée à droite et une dérivée à gauche. Elle est donc continue sur $\mathring{I}$. De plus les applications dérivées à gauche $f'_g$ et à droite $f'_d$ sont croissantes avec $f'_g(x) \leq f'_d(x)$ pour tout $x \in \mathring{I}$.
	\end{proposition}

	\begin{theorem}
		Soit $f : I \rightarrow \mathbb{R}$ une fonction dérivable sur $I$. Alors, les assertions suivantes sont équivalentes :
		\begin{enumerate}[(i)]
			\item $f$ est convexe.
			\item $f'$ est croissante.
			\item La courbe représentative de $f$ est au-dessus de ses tangentes.
		\end{enumerate}
	\end{theorem}

	\begin{proposition}
		Une fonction $f : I \rightarrow \mathbb{R}$ deux fois dérivable est convexe si et seulement si $f''(x) \geq 0$ pour tout $x \in I$.
	\end{proposition}

	\subsubsection{Fonctions log-convexes}

	\reference[ROM19]{228}

	\begin{definition}
		On dit qu'une fonction $f : I \rightarrow \mathbb{R}^+_*$ est \textbf{log-convexe} si $\ln \circ f$ est convexe sur $I$.
	\end{definition}

	\begin{proposition}
		Une fonction log-convexe est convexe.
	\end{proposition}

	\begin{cexample}
		$x \mapsto x$ est convexe mais non log-convexe.
	\end{cexample}

	\begin{theorem}
		Pour une fonction $f : I \rightarrow \mathbb{R}^+_*$, les assertions suivantes sont équivalentes :
		\begin{enumerate}[(i)]
			\item $f$ est log-convexe.
			\item $\forall \alpha > 0, \, x \mapsto \alpha^x f(x)$ est convexe.
			\item $\forall x, y \in I, \, \forall t \in [0,1], \, f((1-t)x + ty) \leq (f(x))^{1-t} (f(y))^t$.
			\item $\forall \alpha > 0, \, f^\alpha$ est convexe.
		\end{enumerate}
	\end{theorem}

	\reference{364}
	\dev{caracterisation-reelle-de-gamma}

	\begin{lemma}
		\label{229-1}
		La fonction $\Gamma$ définie pour tout $x > 0$ par $\Gamma(x) = \int_0^{+\infty} t^{x-1} e^{-t} \, \mathrm{d}t$ vérifie :
		\begin{enumerate}[(i)]
			\item $\forall x \in \mathbb{R}^+_*$, $\Gamma(x+1) = x\Gamma(x)$.
			\item $\Gamma(1) = 1$.
			\item $\Gamma$ est log-convexe sur $\mathbb{R}^+_*$.
		\end{enumerate}
	\end{lemma}

	\reference[RUD]{94}

	\begin{theorem}[Bohr-Mollerup]
		Soit $f : \mathbb{R}^+_* \rightarrow \mathbb{R}^+$ vérifiant les points $(i)$, $(ii)$ et $(iii)$ du \cref{229-1}. Alors $f = \Gamma$.
	\end{theorem}

	\begin{remark}
		À la fin de la preuve, on obtient une formule due à Gauss :
		\[ \forall x \in ]0, 1], \Gamma(x) = \lim_{n \rightarrow +\infty} \frac{n^x n!}{(x+n) \dots (x+1)x} \]
		que l'on peut aisément étendre à $\mathbb{R}^+_*$ entier.
	\end{remark}

	\subsection{Applications}

	\subsubsection{Inégalités}

	\reference[GOU20]{97}

	\begin{proposition}[Inégalité de Hölder]
		Soient $p, q > 0$ tels que $\frac{1}{p} + \frac{1}{q} = 1$. Alors,
		\[ \forall a_1, \dots, a_n, b_1, \dots, b_n \geq 0, \, \sum_{i=1}^n a_i b_i \leq \left( \sum_{i=1}^n a_i^p \right)^{\frac{1}{p}} \left( \sum_{i=1}^n b_i^q \right)^{\frac{1}{q}} \]
	\end{proposition}

	\begin{proposition}[Inégalité de Minkowski] fsqg
		Soit $p \geq 1$. Alors,
		\[ \forall x_1, \dots, x_n, y_1, \dots, y_n \geq 0, \, \left( \sum_{i=1}^n |x_i + y_i|^p \right)^{\frac{1}{p}} \leq \left( \sum_{i=1}^n x_i^p \right)^{\frac{1}{p}} \left( \sum_{i=1}^n y_i^p \right)^{\frac{1}{p}} \]
	\end{proposition}

	\reference[ROM19]{241}

	\begin{proposition}[Inégalité de Jensen]
		Si $f : \mathbb{R} \rightarrow \mathbb{R}$ est convexe, alors pour toute fonction $u$ continue sur un intervalle $[a, b]$, on a :
		\[ f \left( \frac{1}{b-a} \int_a^b u(t) \, \mathrm{d}t \right) \leq \frac{1}{b-a} \int_a^b f \circ u (t) \, \mathrm{d}t \]
	\end{proposition}

	\begin{proposition}[Comparaison des moyennes harmonique, géométrique et arithmétique]
		Pour toute suite finie $x = (x_i)$ de $n$ réels strictement positifs, on a :
		\[ \frac{n}{\sum_{i=1}^n \frac{1}{x_i}} \leq \left( \prod_{i=1}^n x_i \right)^{\frac{1}{n}} \leq \frac{1}{n} \sum_{i=1}^n x_i \]
	\end{proposition}

	\subsubsection{Recherche d'extrema}

	\reference{234}

	\begin{proposition}
		Une fonction $f : \mathbb{R} \rightarrow \mathbb{R}$ est constante si et seulement si elle est convexe et majorée.
	\end{proposition}

	\begin{cexample}
		La fonction $f$ définie sur $\mathbb{R}^+$ par $f(x) = \frac{1}{1+x}$ est convexe, majorée, mais non constante.
	\end{cexample}

	\begin{proposition}
		Si $f : I \rightarrow \mathbb{R}$ est convexe et est dérivable en un point $\alpha \in \mathring{I}$ tel que $f'(\alpha) = 0$, alors $f$ admet un minimum global en $\alpha$.
	\end{proposition}

	\begin{proposition}
		Si $f : I \rightarrow \mathbb{R}$ est convexe et admet un minimum local, alors ce minimum est global.
	\end{proposition}

	\subsubsection{Méthode de Newton}

	\reference[ROU]{142}
	\dev{methode-de-newton}

	\begin{theorem}
		Soit $f : [c, d] \rightarrow \mathbb{R}$ une fonction de classe $\mathcal{C}^2$ strictement croissante sur $[c, d]$. On considère la fonction
		\[ \varphi :
		\begin{array}{ccc}
			[c, d] &\rightarrow& \mathbb{R} \\
			x &\mapsto& x - \frac{f(x)}{f'(x)}
		\end{array}
		\]
		(qui est bien définie car $f' > 0$). Alors :
		\begin{enumerate}[(i)]
			\item $\exists! a \in [c, d]$ tel que $f(a) = 0$.
			\item $\exists \alpha > 0$ tel que $I = [a - \alpha, a + \alpha]$ est stable par $\varphi$.
			\item La suite $(x_n)$ des itérés (définie par récurrence par $x_{n+1} = \varphi(x_n)$ pour tout $n \geq 0$) converge quadratiquement vers $a$ pour tout $x_0 \in I$.
		\end{enumerate}
	\end{theorem}

	\begin{corollary}
		En reprenant les hypothèses et notations du théorème précédent, et en supposant de plus $f$ strictement convexe sur $[c, d]$, le résultat du théorème est vrai sur $I = [a, d]$. De plus :
		\begin{enumerate}[(i)]
			\item $(x_n)$ est strictement décroissante (ou constante).
			\item $x_{n+1} - a \sim \frac{f''(a)}{f'(a)} (x_n - a)^2$ pour $x_0 > a$.
		\end{enumerate}
	\end{corollary}
	%</content>
\end{document}
