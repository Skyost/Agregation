\documentclass[12pt, a4paper]{report}

% LuaLaTeX :

\RequirePackage{iftex}
\RequireLuaTeX

% Packages :

\usepackage[french]{babel}
%\usepackage[utf8]{inputenc}
%\usepackage[T1]{fontenc}
\usepackage[pdfencoding=auto, pdfauthor={Hugo Delaunay}, pdfsubject={Mathématiques}, pdfcreator={agreg.skyost.eu}]{hyperref}
\usepackage{amsmath}
\usepackage{amsthm}
%\usepackage{amssymb}
\usepackage{stmaryrd}
\usepackage{tikz}
\usepackage{tkz-euclide}
\usepackage{fourier-otf}
\usepackage{fontspec}
\usepackage{titlesec}
\usepackage{fancyhdr}
\usepackage{catchfilebetweentags}
\usepackage[french, capitalise, noabbrev]{cleveref}
\usepackage[fit, breakall]{truncate}
\usepackage[top=2.5cm, right=2cm, bottom=2.5cm, left=2cm]{geometry}
\usepackage{enumerate}
\usepackage{tocloft}
\usepackage{microtype}
%\usepackage{mdframed}
%\usepackage{thmtools}
\usepackage{xcolor}
\usepackage{tabularx}
\usepackage{aligned-overset}
\usepackage[subpreambles=true]{standalone}
\usepackage{environ}
\usepackage[normalem]{ulem}
\usepackage{marginnote}
\usepackage{etoolbox}
\usepackage{setspace}
\usepackage[bibstyle=reading, citestyle=draft]{biblatex}
\usepackage{xpatch}
\usepackage[many, breakable]{tcolorbox}
\usepackage[backgroundcolor=white, bordercolor=white, textsize=small]{todonotes}

% Bibliographie :

\newcommand{\overridebibliographypath}[1]{\providecommand{\bibliographypath}{#1}}
\overridebibliographypath{../bibliography.bib}
\addbibresource{\bibliographypath}
\defbibheading{bibliography}[\bibname]{%
	\newpage
	\section*{#1}%
}
\renewbibmacro*{entryhead:full}{\printfield{labeltitle}}
\DeclareFieldFormat{url}{\newline\footnotesize\url{#1}}
\AtEndDocument{\printbibliography}

% Police :

\setmathfont{Erewhon Math}

% Tikz :

\usetikzlibrary{calc}

% Longueurs :

\setlength{\parindent}{0pt}
\setlength{\headheight}{15pt}
\setlength{\fboxsep}{0pt}
\titlespacing*{\chapter}{0pt}{-20pt}{10pt}
\setlength{\marginparwidth}{1.5cm}
\setstretch{1.1}

% Métadonnées :

\author{agreg.skyost.eu}
\date{\today}

% Titres :

\setcounter{secnumdepth}{3}

\renewcommand{\thechapter}{\Roman{chapter}}
\renewcommand{\thesubsection}{\Roman{subsection}}
\renewcommand{\thesubsubsection}{\arabic{subsubsection}}
\renewcommand{\theparagraph}{\alph{paragraph}}

\titleformat{\chapter}{\huge\bfseries}{\thechapter}{20pt}{\huge\bfseries}
\titleformat*{\section}{\LARGE\bfseries}
\titleformat{\subsection}{\Large\bfseries}{\thesubsection \, - \,}{0pt}{\Large\bfseries}
\titleformat{\subsubsection}{\large\bfseries}{\thesubsubsection. \,}{0pt}{\large\bfseries}
\titleformat{\paragraph}{\bfseries}{\theparagraph. \,}{0pt}{\bfseries}

\setcounter{secnumdepth}{4}

% Table des matières :

\renewcommand{\cftsecleader}{\cftdotfill{\cftdotsep}}
\addtolength{\cftsecnumwidth}{10pt}

% Redéfinition des commandes :

\renewcommand*\thesection{\arabic{section}}
\renewcommand{\ker}{\mathrm{Ker}}

% Nouvelles commandes :

\newcommand{\website}{https://agreg.skyost.eu}

\newcommand{\tr}[1]{\mathstrut ^t #1}
\newcommand{\im}{\mathrm{Im}}
\newcommand{\rang}{\operatorname{rang}}
\newcommand{\trace}{\operatorname{trace}}
\newcommand{\id}{\operatorname{id}}
\newcommand{\stab}{\operatorname{Stab}}

\providecommand{\newpar}{\\[\medskipamount]}

\providecommand{\lesson}[3]{%
	\title{#3}%
	\hypersetup{pdftitle={#3}}%
	\setcounter{section}{\numexpr #2 - 1}%
	\section{#3}%
	\fancyhead[R]{\truncate{0.73\textwidth}{#2 : #3}}%
}

\providecommand{\development}[3]{%
	\title{#3}%
	\hypersetup{pdftitle={#3}}%
	\section*{#3}%
	\fancyhead[R]{\truncate{0.73\textwidth}{#3}}%
}

\providecommand{\summary}[1]{%
	\textit{#1}%
	\medskip%
}

\tikzset{notestyleraw/.append style={inner sep=0pt, rounded corners=0pt, align=center}}

%\newcommand{\booklink}[1]{\website/bibliographie\##1}
\newcommand{\citelink}[2]{\hyperlink{cite.\therefsection @#1}{#2}}
\newcommand{\previousreference}{}
\providecommand{\reference}[2][]{%
	\notblank{#1}{\renewcommand{\previousreference}{#1}}{}%
	\todo[noline]{%
		\protect\vspace{16pt}%
		\protect\par%
		\protect\notblank{#1}{\cite{[\previousreference]}\\}{}%
		\protect\citelink{\previousreference}{p. #2}%
	}%
}

\definecolor{devcolor}{HTML}{00695c}
\newcommand{\dev}[1]{%
	\reversemarginpar%
	\todo[noline]{
		\protect\vspace{16pt}%
		\protect\par%
		\bfseries\color{devcolor}\href{\website/developpements/#1}{DEV}
	}%
	\normalmarginpar%
}

% En-têtes :

\pagestyle{fancy}
\fancyhead[L]{\truncate{0.23\textwidth}{\thepage}}
\fancyfoot[C]{\scriptsize \href{\website}{\texttt{agreg.skyost.eu}}}

% Couleurs :

\definecolor{property}{HTML}{fffde7}
\definecolor{proposition}{HTML}{fff8e1}
\definecolor{lemma}{HTML}{fff3e0}
\definecolor{theorem}{HTML}{fce4f2}
\definecolor{corollary}{HTML}{ffebee}
\definecolor{definition}{HTML}{ede7f6}
\definecolor{notation}{HTML}{f3e5f5}
\definecolor{example}{HTML}{e0f7fa}
\definecolor{cexample}{HTML}{efebe9}
\definecolor{application}{HTML}{e0f2f1}
\definecolor{remark}{HTML}{e8f5e9}
\definecolor{proof}{HTML}{e1f5fe}

% Théorèmes :

\theoremstyle{definition}
\newtheorem{theorem}{Théorème}

\newtheorem{property}[theorem]{Propriété}
\newtheorem{proposition}[theorem]{Proposition}
\newtheorem{lemma}[theorem]{Lemme}
\newtheorem{corollary}[theorem]{Corollaire}

\newtheorem{definition}[theorem]{Définition}
\newtheorem{notation}[theorem]{Notation}

\newtheorem{example}[theorem]{Exemple}
\newtheorem{cexample}[theorem]{Contre-exemple}
\newtheorem{application}[theorem]{Application}

\theoremstyle{remark}
\newtheorem{remark}[theorem]{Remarque}

\counterwithin*{theorem}{section}

\newcommand{\applystyletotheorem}[1]{
	\tcolorboxenvironment{#1}{
		enhanced,
		breakable,
		colback=#1!98!white,
		boxrule=0pt,
		boxsep=0pt,
		left=8pt,
		right=8pt,
		top=8pt,
		bottom=8pt,
		sharp corners,
		after=\par,
	}
}

\applystyletotheorem{property}
\applystyletotheorem{proposition}
\applystyletotheorem{lemma}
\applystyletotheorem{theorem}
\applystyletotheorem{corollary}
\applystyletotheorem{definition}
\applystyletotheorem{notation}
\applystyletotheorem{example}
\applystyletotheorem{cexample}
\applystyletotheorem{application}
\applystyletotheorem{remark}
\applystyletotheorem{proof}

% Environnements :

\NewEnviron{whitetabularx}[1]{%
	\renewcommand{\arraystretch}{2.5}
	\colorbox{white}{%
		\begin{tabularx}{\textwidth}{#1}%
			\BODY%
		\end{tabularx}%
	}%
}

% Maths :

\DeclareFontEncoding{FMS}{}{}
\DeclareFontSubstitution{FMS}{futm}{m}{n}
\DeclareFontEncoding{FMX}{}{}
\DeclareFontSubstitution{FMX}{futm}{m}{n}
\DeclareSymbolFont{fouriersymbols}{FMS}{futm}{m}{n}
\DeclareSymbolFont{fourierlargesymbols}{FMX}{futm}{m}{n}
\DeclareMathDelimiter{\VERT}{\mathord}{fouriersymbols}{152}{fourierlargesymbols}{147}


% Bibliographie :

\addbibresource{\bibliographypath}%
\defbibheading{bibliography}[\bibname]{%
	\newpage
	\section*{#1}%
}
\renewbibmacro*{entryhead:full}{\printfield{labeltitle}}%
\DeclareFieldFormat{url}{\newline\footnotesize\url{#1}}%

\AtEndDocument{\printbibliography}

\begin{document}
  %<*content>
  \lesson{analysis}{244}{Exemples d'études et d'applications de fonctions usuelles et spéciales.}

  \subsection{La fonction exponentielle}

  \subsubsection{Dans le champ complexe}

  \reference[QUE]{4}

  \begin{definition}
    \label{244-1}
    On définit la fonction \textbf{exponentielle complexe} pour tout $z \in \mathbb{C}$ par
    \[ \sum_{n=0}^{+\infty} \frac{z^n}{n!} \]
    on note cette somme $e^z$ ou parfois $\exp(z)$.
  \end{definition}

  \begin{remark}
    Cette somme est bien définie pour tout $z \in \mathbb{C}$ d'après le critère de d'Alembert.
  \end{remark}

  \begin{proposition}
    \begin{enumerate}[label=(\roman*)]
      \item $\forall z, z' \in \mathbb{C}, \, e^{z+z'} = e^z e^{z'}$.
      \item $\exp$ est holomorphe sur $\mathbb{C}$, de dérivée elle-même.
      \item $\exp$ ne s'annule jamais.
      \item $\vert \exp(z) \vert = \exp(\operatorname{Re}(z))$ pour tout $z \in \mathbb{C}$.
    \end{enumerate}
  \end{proposition}

  \begin{proposition}
    La fonction $\varphi : t \mapsto e^{it}$ est un morphisme surjectif de $\mathbb{R}$ sur $\mathbb{U}$.
  \end{proposition}

  \begin{proposition}
    En reprenant les notations précédentes, $\ker(\varphi)$ est un sous-groupe fermé de $\mathbb{R}$, de la forme $\ker(\varphi) = a\mathbb{Z}$. On note $a = 2\pi$.
  \end{proposition}

  \reference[R-R]{259}

  \begin{application}
    Pour tout $n \in \mathbb{N}^*$, il y a $n$ racines $n$-ièmes de l'unité, données par
    \[ e^{\frac{2ik\pi}{n}} = \cos \left( \frac{2ik\pi}{n} \right) + i \sin \left( \frac{2ik\pi}{n} \right) \]
    où $k$ parcourt les entiers de $0$ à $n-1$.
  \end{application}

  \begin{corollary}
    Tout nombre complexe non nul $\alpha$ écrit $\alpha = re^{i\theta}$ admet exactement $n$ racines $n$-ièmes données par
    \[ \sqrt[n]{r} e^{i\frac{\theta}{n}} e^{\frac{2ik\pi}{n}} \]
    où $k$ parcourt les entiers de $0$ à $n-1$.
  \end{corollary}

  \subsubsection{Dans le champ réel}

  \reference[D-L]{528}

  \begin{definition}
    On a plusieurs définitions (équivalentes) de la fonction exponentielle réelle.
    \begin{itemize}
      \item \textbf{Vision ``moderne'' :} Soit $x \in \mathbb{R}$. $\exp(x) = \sum_{n=0}^{+\infty} \frac{x^k}{k!}$ (restriction de la série entière de la \cref{244-1}).
      \item \textbf{Vision ``pédagogique'' :} $\exp$ est l'unique solution au problème de Cauchy
      \[
        \begin{cases}
          y' = y \\
          y(0) = 1
        \end{cases}
      \]
      \item \textbf{Vision ``historique'' :} Soit $x \in \mathbb{R}$. $\exp(x) = \lim_{n \rightarrow +\infty} \left( 1 + \frac{x}{n} \right)^n$.
    \end{itemize}
  \end{definition}

  \reference[QUE]{6}

  \begin{theorem}
    \begin{enumerate}[label=(\roman*)]
      \item $\exp$ est une bijection croissante de $\mathbb{R}$ sur $\mathbb{R}_*^+$.
      \item $\lim_{x \rightarrow -\infty} \exp(x) = 0$ et $\lim_{x \rightarrow +\infty} \exp(x) = +\infty$.
      \item $x < 0 \iff \exp(x) < 1$.
    \end{enumerate}
  \end{theorem}

  \subsubsection{Fonctions circulaires}

  \begin{definition}
    On définit les fonctions $\sin$ et $\cos$ sur $\mathbb{R}$ par
    \[ \forall t \in \mathbb{R}, \, \sin(t) = \operatorname{Im}(\exp(it)) \text{ et } \sin(t) = \operatorname{Re}(\exp(it)) \]
  \end{definition}

  \reference[DAN]{352}

  \begin{proposition}
    Soit $t \in \mathbb{R}$.
    \begin{enumerate}[label=(\roman*)]
      \item $\sin(t) = \frac{e^{it} - e^{-it}}{2i} = \sum_{n=0}^{+\infty} \frac{t^{2n+1}}{(2n+1)!}$.
      \item $\cos(t) = \frac{e^{it} + e^{-it}}{2} = \sum_{n=0}^{+\infty} \frac{t^{2n}}{(2n)!}$.
      \item Ces fonctions sont réelles, $2\pi$-périodiques et admettent un développement en série entière de rayon de convergence infini. Ceci permet de les prolonger de manière unique sur tout le plan complexe.
      \item $\sin$ et $\cos$ sont dérivables avec $\cos' = -\sin$ et $\sin' = \cos$.
      \item $\cos$ est paire, $\sin$ est impaire.
    \end{enumerate}
  \end{proposition}

  \reference[ROM21]{36}

  \begin{proposition}
    L'application
    \[ \exp(i\theta) \mapsto
    \begin{pmatrix}
      \cos(\theta) & \sin(\theta) \\
      -\sin(\theta) & \cos(\theta)
    \end{pmatrix}
    \]
    définit un isomorphisme de $\mathbb{U}$ dans $\mathrm{SO}_2(\mathbb{R})$.
  \end{proposition}

  \subsubsection{Polynômes trigonométriques}

  \reference[GOU20]{268}

  \begin{definition}
    \begin{itemize}
      \item On appelle \textbf{polynôme trigonométrique} de degré inférieur à $N \in \mathbb{N}$ toute fonction de la forme $x \mapsto \sum_{n=-N}^{N} c_n e^{inx}$ avec $\forall n \in \llbracket -N, N \rrbracket$, $c_n \in \mathbb{C}$.
      \item On appelle \textbf{série trigonométrique} une série de fonctions de la variable réelle $x$ et de la forme $c_n + \sum_{n \in \mathbb{N}^*} (c_n e^{inx} + c_{-n} e^{-inx})$, notée $\sum_{n \in \mathbb{Z}} c_n e^{inx}$.
    \end{itemize}
  \end{definition}

  \reference[AMR08]{184}

  \begin{example}
    \begin{itemize}
      \item Pour tout $N \in \mathbb{N}$, la fonction $D_N = \sum_{n=-N}^{N} e_N$ est appelée \textbf{noyau de Dirichlet} d'ordre $N$.
      \item Pour tout $N \in \mathbb{N}$, la fonction $K_N = \frac{1}{N} \sum_{j=0}^{N-1} D_j$ est appelé \textbf{noyau de Fejér} d'ordre $N$.
    \end{itemize}
  \end{example}

  \reference{190}

  \begin{theorem}[Fejér]
    Soit $f : \mathbb{R} \rightarrow \mathbb{C}$ une fonction $2\pi$-périodique.
    \begin{enumerate}[label=(\roman*)]
      \item Si $f$ est continue, alors $\Vert \sigma_N(f) \Vert_\infty \leq \Vert f \Vert_\infty$ et $(\sigma_N(f))$ converge uniformément vers $f$.
      \item Si $f \in L_p^{2\pi}$ pour $p \in [1,+\infty[$, alors $\Vert \sigma_N(f) \Vert_p \leq \Vert f \Vert_p$ et $(\sigma_N(f))$ converge vers $f$ pour $\Vert . \Vert_p$.
    \end{enumerate}
  \end{theorem}

  \begin{corollary}
    L'espace des polynômes trigonométriques $\{ \sum_{n=-N}^N c_n e_n \mid (c_n) \in \mathbb{C}^{\mathbb{N}}, \, N \in \mathbb{N} \}$ est dense dans l'espace des fonction continues $2\pi$-périodiques pour $\Vert . \Vert_\infty$ et est dense dans $L_p^{2\pi}$ pour $\Vert . \Vert_p$ avec $p \in [1,+\infty[$.
  \end{corollary}

  \reference[GOU20]{271}

  \begin{theorem}[Dirichlet]
    Soient $f : \mathbb{R} \rightarrow \mathbb{C}$ $2\pi$-périodique, continue par morceaux sur $\mathbb{R}$ et $t_0 \in \mathbb{R}$ tels que la fonction
    \[ h \mapsto \frac{f(t_0 + h) + f(t_0 - h) - f(t_0^+) - f(t_0^-)}{h} \]
    est bornée au voisinage de $0$. Alors,
    \[ S_N(f)(t_0) \longrightarrow_{N \rightarrow +\infty} \frac{f(t_0^+) + f(t_0^-)}{2} \]
  \end{theorem}

  \begin{cexample}
    Soit $f : \mathbb{R} \rightarrow \mathbb{R}$ paire, $2\pi$-périodique telle que :
    \[ \forall x \in [0, \pi], f(x) = \sum_{p=1}^{+\infty} \frac{1}{p^2} \sin \left( (2^{p^3} + 1) \frac{x}{2} \right)
    \]
    Alors $f$ est bien définie et continue sur $\mathbb{R}$. Cependant, sa série de Fourier diverge en $0$.
  \end{cexample}

  \begin{corollary}
    Soient $f : \mathbb{R} \rightarrow \mathbb{C}$ $2\pi$-périodique, $\mathcal{C}^1$ par morceaux sur $\mathbb{R}$. Alors,
    \[ \forall x \in \mathbb{R}, \, S_N(f)(x) \longrightarrow_{N \rightarrow +\infty} \frac{f(x^+) + f(x^-)}{2} \]
    En particulier, si $f$ est continue en $x$, la série de Fourier de $f$ converge vers $f(x)$.
  \end{corollary}

  \begin{example}
    On considère $f : x \mapsto 1 - \frac{x^2}{\pi^2}$ sur $[-\pi, \pi]$. Alors,
    \[ \forall x \in [-\pi, \pi], \, f(x) = \frac{2}{3} - \frac{4}{\pi^2} \sum_{n=1}^{+\infty} (-1)^n \frac{\cos(nx)}{n^2} \]
  \end{example}

  \reference{284}

  \begin{theorem}[Formule sommatoire de Poisson]
    Soit $f : \mathbb{R} \rightarrow \mathbb{C}$ une fonction de classe $\mathcal{C}^1$ telle que $f(x) = O \left( \frac{1}{x^2} \right)$ et $f'(x) = O \left( \frac{1}{x^2} \right)$ quand $|x| \longrightarrow +\infty$. Alors :
    \[ \forall x \in \mathbb{R}, \, \sum_{n \in \mathbb{Z}} f(x+n) = \sum_{n \in \mathbb{Z}} \widehat{f}(2 \pi n) e^{2 i \pi n x} \]
  \end{theorem}

  \begin{application}[Identité de Jacobi]
    \[ \forall s > 0, \, \sum_{n=-\infty}^{+\infty} e^{-\pi n^2 s} = \frac{1}{\sqrt{s}} \sum_{n=-\infty}^{+\infty} e^{-\frac{\pi n^2}{s}} \]
  \end{application}

  \subsection{Logarithmes}

  \subsubsection{Logarithme dans le champ réel}

  \reference[DAN]{346}

  \begin{proposition}
    $\exp$ réalise une bijection strictement croissante de $\mathbb{R}$ sur $\mathbb{R}_*^+$.
  \end{proposition}

  \begin{definition}
    La bijection réciproque de $\exp : \mathbb{R} \rightarrow \mathbb{R}_*^+$ est appelée \textbf{logarithme népérien} et est notée $\ln$.
  \end{definition}

  \begin{theorem}
    \begin{enumerate}[label=(\roman*)]
      \item $\forall x \in \mathbb{R}_*^+$, $\ln(x) = \int_1^x \frac{1}{x} \, \mathrm{d}x$.
      \item $\forall x, y \in \mathbb{R}_*^+$, $\ln(xy) = \ln(x) + \ln(y)$.
    \end{enumerate}
  \end{theorem}

  \begin{remark}
    La fonction $\ln$ permet de définir la mise à la puissance par un réel :
    \[ \forall t \in \mathbb{R}^+_*, \, \forall \alpha \in \mathbb{R}, \, t^\alpha = e^{\alpha \ln(t)} \]
  \end{remark}

  \subsubsection{Logarithmes dans le champ complexe}

  \reference[QUE]{81}

  \begin{theorem}
    Soient $\alpha \in \mathbb{R}$ et $\Omega_\alpha = \mathbb{C} \setminus \mathbb{R}^* e^{i\alpha}$. Alors, il existe une fonction $L_\alpha$ holomorphe sur $\Omega_\alpha$. Elle vérifie :
    \begin{enumerate}[label=(\roman*)]
      \item $e^{L_\alpha(z)} = z$ pour tout $z \in \Omega_\alpha$.
      \item $L_\alpha(z) = \ln(\vert z \vert) + i\theta_\alpha(z)$ avec $\theta_\alpha \in ]\alpha, \alpha + 2\pi[$.
      \item $L_\alpha$ est dérivable dans $\Omega_\alpha$ avec $L'(z) = \frac{1}{z}$ pour tout $z \in \Omega_\alpha$.
    \end{enumerate}
  \end{theorem}

  \begin{definition}
    La fonction $L_\alpha$ précédente est appelée \textbf{détermination d'ordre $\alpha$} (ou \textbf{détermination principale} si $\alpha = -\pi$) du logarithme.
  \end{definition}

  \begin{theorem}
    On pose $D = D(0,1)$ et on définit $\ell : D \rightarrow \mathbb{C}$ par $\ell : z \mapsto \sum_{n=1}^{+\infty} (-1)^{n+1} \frac{z^n}{n}$. Alors :
    \begin{enumerate}[label=(\roman*)]
      \item $1 + z = \exp(\ell(z))$ pour tout $z \in D$.
      \item $\ell(z) = L_{-\pi}(1+z)$ pour tout $z \in D$.
    \end{enumerate}
  \end{theorem}

  \subsection{La fonction \texorpdfstring{$\Gamma$}{Gamma} d'Euler}

  \subsubsection{Définition}

  \reference[GOU20]{162}

  \begin{definition}
    On pose
    \[ \forall x > 0, \, \Gamma(x) = \int_0^{+\infty} e^{-t} t^{x-1} \, \mathrm{d}t \]
  \end{definition}

  \begin{proposition}
    \begin{enumerate}[label=(\roman*)]
      \item $\Gamma$ est $\mathcal{C}^\infty$ sur $]0, +\infty[$ et pour tout $n \in \mathbb{N}^*$, on a
      \[ \forall x \in \mathbb{R}_*^+, \, \Gamma^{(n)}(x) = \int_{0}^{+\infty} (\ln(t))^n e^{-t} t^{x-1} \, \mathrm{d}t \]
      \item $\Gamma \left( \frac{1}{2} \right) = \sqrt{\pi}$.
      \item $\forall x > 0, \, \Gamma(x+1) = x \Gamma(x)$ et en particulier, $\forall n \in \mathbb{N}, \, \Gamma(n) = n!$.
    \end{enumerate}
  \end{proposition}

  \reference[ROM19-1]{364}

  \begin{lemma}
    \label{244-2}
    La fonction $\Gamma$ définie pour tout $x > 0$ par $\Gamma(x) = \int_0^{+\infty} t^{x-1} e^{-t} \, \mathrm{d}t$ vérifie :
    \begin{enumerate}[label=(\roman*)]
      \item \label{244-3} $\forall x \in \mathbb{R}^+_*$, $\Gamma(x+1) = x\Gamma(x)$.
      \item \label{244-4} $\Gamma(1) = 1$.
      \item \label{244-5} $\Gamma$ est log-convexe sur $\mathbb{R}^+_*$.
    \end{enumerate}
  \end{lemma}

  \dev{caracterisation-reelle-de-gamma}

  \begin{theorem}[Bohr-Mollerup]
    Soit $f : \mathbb{R}^+_* \rightarrow \mathbb{R}^+$ vérifiant le \cref{244-3}, \cref{244-4} et \cref{244-5} du \cref{244-2}. Alors $f = \Gamma$.
  \end{theorem}

  \begin{remark}
    À la fin de la preuve, on obtient une formule due à Gauss :
    \[ \forall x \in ]0, 1], \Gamma(x) = \lim_{n \rightarrow +\infty} \frac{n^x n!}{(x+n) \dots (x+1)x} \]
    que l'on peut aisément étendre à $\mathbb{R}^+_*$ entier.
  \end{remark}

  \reference[G-K]{180}

  \begin{lemma}
    Soient $X$ et $Y$ deux variables aléatoires indépendantes telles que $X \sim \Gamma(a, \gamma)$ et $Y \sim \Gamma(b, \gamma)$. Alors $Z = X + Y \sim \Gamma(a+b, \gamma)$.
  \end{lemma}

  \reference{556}
  \dev{formule-de-stirling}

  \begin{application}[Formule de Stirling]
    \[ n! \sim \sqrt{2n\pi} \left(\frac{n}{e} \right)^n \]
  \end{application}

  \subsubsection{Prolongement complexe}

  \reference[Z-Q]{314}

  On suppose ici que $E$ est un ouvert $\Omega$ de $\mathbb{C}$.

  \begin{theorem}[Holomorphie sous le signe intégral]
    On suppose :
    \begin{enumerate}[label=(\roman*)]
      \item $\forall z \in \Omega$, $x \mapsto f(z,x) \in L_1(X)$.
      \item pp. en $x \in X$, $z \mapsto f(z,x)$ est holomorphe dans $\Omega$. On notera $\frac{\partial f}{\partial z}$ cette dérivée définie presque partout.
      \item $\forall K \subseteq \Omega$ compact, $\exists g_K \in L_1(X)$ positive telle que
      \[ \left| f(x,z) \right| \leq g_K(x) \quad \forall z \in K, \text{pp. en } x \]
    \end{enumerate}
    Alors $F$ est holomorphe dans $\Omega$ avec
    \[ \forall z \in \Omega, \, F'(z) = \int_X \frac{\partial f}{\partial z}(z, t) \, \mathrm{d}\mu(z) \]
  \end{theorem}

  \reference{318}

  \begin{example}
    La fonction $\Gamma$ est holomorphe dans l'ouvert $\{ z \in \mathbb{C} \mid \operatorname{Re}(z) > 0 \}$.
  \end{example}

  \begin{theorem}
    On peut prolonger $\Gamma$ en une fonction holomorphe non nulle sur $\mathbb{C} \setminus -\mathbb{N}$.
  \end{theorem}

  \reference[QUE]{255}

  \begin{theorem}[Formule des compléments]
    \[ \forall \mathbb{C} \setminus \mathbb{Z}, \, \Gamma(z) \Gamma(1-z) = \frac{\pi}{\sin(\pi z)} \]
  \end{theorem}

  \subsection{La fonction \texorpdfstring{$\zeta$}{Zeta} de Riemann}

  \subsubsection{Définition}

  \reference[GOU20]{302}

  \begin{definition}
    \label{244-6}
    Pour tout $s > 1$, on pose
    \[ \zeta(s) = \sum_{n=1}^{+\infty} \frac{1}{n^s} \]
  \end{definition}

  \begin{proposition}
    $\zeta$ définit une fonction de classe $\mathcal{C}^\infty$ sur $]1, +\infty[$ et,
    \[ \forall p \in \mathbb{N}^*, \, \forall s \in ]1, +\infty[, \, \zeta^{(p)}(s) = \sum_{n=1}^{+\infty} \frac{\ln(n)^p}{n^s} \]
  \end{proposition}

  \begin{proposition}
    \[ \lim_{s \rightarrow +\infty} \zeta(s) = 1 \text{ et } \zeta(s) \sim_{1^+} \frac{1}{s-1} + \gamma + o(1) \]
    où $\gamma$ désigne la constante d'Euler.
  \end{proposition}

  \reference[G-K]{108}

  \begin{proposition}
    \[ \forall s > 1, \, \zeta(s) \Gamma(s) = \int_0^{+\infty} t^{s-1} \frac{e^{-t}}{1 - e^{-t}} \, \mathrm{d}t \]
  \end{proposition}

  \subsubsection{Prolongement complexe}

  \reference[Z-Q]{20}

  \begin{proposition}
    On prolonge la définition de $\zeta$ donnée à la \cref{244-6} en posant
    \[
      \zeta :
      \begin{array}{ccc}
        \{ s \in \mathbb{C} \mid \operatorname{Re}(s) > 0 \} &\mapsto& \mathbb{C} \\
        s &\mapsto& \sum_{n=1}^{+\infty} \frac{1}{n^s}
      \end{array}
    \]
  \end{proposition}

  \begin{proposition}
    $\zeta$ est holomorphe sur $\{ s \in \mathbb{C} \mid \operatorname{Re}(s) > 1 \}$.
  \end{proposition}

  \reference{28}

  \begin{theorem}
    Il existe une fonction $\widetilde{\zeta}$, holomorphe dans $\mathbb{C} \setminus \{ 1 \}$ telle que :
    \begin{enumerate}[label=(\roman*)]
      \item Pour tout $s \in \mathbb{C} \setminus \{ 1 \}$, $\widetilde{\zeta}(s) = \frac{1}{s-1} + \eta(s)$ avec $\eta$ holomorphe dans $\mathbb{C}$.
      \item Pour tout $s \in \mathbb{C}$ tel que $\operatorname{Re}(s) > 1$, $\widetilde{\zeta}(s) = \zeta(s)$.
      \item En posant $I(s) = \pi^{\frac{s}{2}} \Gamma \left( \frac{s}{2} \right) \zeta(s)$, on a $I(s) = I(1-s)$.
    \end{enumerate}
  \end{theorem}
  %</content>
\end{document}
