\documentclass[12pt, a4paper]{report}

% LuaLaTeX :

\RequirePackage{iftex}
\RequireLuaTeX

% Packages :

\usepackage[french]{babel}
%\usepackage[utf8]{inputenc}
%\usepackage[T1]{fontenc}
\usepackage[pdfencoding=auto, pdfauthor={Hugo Delaunay}, pdfsubject={Mathématiques}, pdfcreator={agreg.skyost.eu}]{hyperref}
\usepackage{amsmath}
\usepackage{amsthm}
%\usepackage{amssymb}
\usepackage{stmaryrd}
\usepackage{tikz}
\usepackage{tkz-euclide}
\usepackage{fourier-otf}
\usepackage{fontspec}
\usepackage{titlesec}
\usepackage{fancyhdr}
\usepackage{catchfilebetweentags}
\usepackage[french, capitalise, noabbrev]{cleveref}
\usepackage[fit, breakall]{truncate}
\usepackage[top=2.5cm, right=2cm, bottom=2.5cm, left=2cm]{geometry}
\usepackage{enumerate}
\usepackage{tocloft}
\usepackage{microtype}
%\usepackage{mdframed}
%\usepackage{thmtools}
\usepackage{xcolor}
\usepackage{tabularx}
\usepackage{aligned-overset}
\usepackage[subpreambles=true]{standalone}
\usepackage{environ}
\usepackage[normalem]{ulem}
\usepackage{marginnote}
\usepackage{etoolbox}
\usepackage{setspace}
\usepackage[bibstyle=reading, citestyle=draft]{biblatex}
\usepackage{xpatch}
\usepackage[many, breakable]{tcolorbox}
\usepackage[backgroundcolor=white, bordercolor=white, textsize=small]{todonotes}

% Bibliographie :

\newcommand{\overridebibliographypath}[1]{\providecommand{\bibliographypath}{#1}}
\overridebibliographypath{../bibliography.bib}
\addbibresource{\bibliographypath}
\defbibheading{bibliography}[\bibname]{%
	\newpage
	\section*{#1}%
}
\renewbibmacro*{entryhead:full}{\printfield{labeltitle}}
\DeclareFieldFormat{url}{\newline\footnotesize\url{#1}}
\AtEndDocument{\printbibliography}

% Police :

\setmathfont{Erewhon Math}

% Tikz :

\usetikzlibrary{calc}

% Longueurs :

\setlength{\parindent}{0pt}
\setlength{\headheight}{15pt}
\setlength{\fboxsep}{0pt}
\titlespacing*{\chapter}{0pt}{-20pt}{10pt}
\setlength{\marginparwidth}{1.5cm}
\setstretch{1.1}

% Métadonnées :

\author{agreg.skyost.eu}
\date{\today}

% Titres :

\setcounter{secnumdepth}{3}

\renewcommand{\thechapter}{\Roman{chapter}}
\renewcommand{\thesubsection}{\Roman{subsection}}
\renewcommand{\thesubsubsection}{\arabic{subsubsection}}
\renewcommand{\theparagraph}{\alph{paragraph}}

\titleformat{\chapter}{\huge\bfseries}{\thechapter}{20pt}{\huge\bfseries}
\titleformat*{\section}{\LARGE\bfseries}
\titleformat{\subsection}{\Large\bfseries}{\thesubsection \, - \,}{0pt}{\Large\bfseries}
\titleformat{\subsubsection}{\large\bfseries}{\thesubsubsection. \,}{0pt}{\large\bfseries}
\titleformat{\paragraph}{\bfseries}{\theparagraph. \,}{0pt}{\bfseries}

\setcounter{secnumdepth}{4}

% Table des matières :

\renewcommand{\cftsecleader}{\cftdotfill{\cftdotsep}}
\addtolength{\cftsecnumwidth}{10pt}

% Redéfinition des commandes :

\renewcommand*\thesection{\arabic{section}}
\renewcommand{\ker}{\mathrm{Ker}}

% Nouvelles commandes :

\newcommand{\website}{https://agreg.skyost.eu}

\newcommand{\tr}[1]{\mathstrut ^t #1}
\newcommand{\im}{\mathrm{Im}}
\newcommand{\rang}{\operatorname{rang}}
\newcommand{\trace}{\operatorname{trace}}
\newcommand{\id}{\operatorname{id}}
\newcommand{\stab}{\operatorname{Stab}}

\providecommand{\newpar}{\\[\medskipamount]}

\providecommand{\lesson}[3]{%
	\title{#3}%
	\hypersetup{pdftitle={#3}}%
	\setcounter{section}{\numexpr #2 - 1}%
	\section{#3}%
	\fancyhead[R]{\truncate{0.73\textwidth}{#2 : #3}}%
}

\providecommand{\development}[3]{%
	\title{#3}%
	\hypersetup{pdftitle={#3}}%
	\section*{#3}%
	\fancyhead[R]{\truncate{0.73\textwidth}{#3}}%
}

\providecommand{\summary}[1]{%
	\textit{#1}%
	\medskip%
}

\tikzset{notestyleraw/.append style={inner sep=0pt, rounded corners=0pt, align=center}}

%\newcommand{\booklink}[1]{\website/bibliographie\##1}
\newcommand{\citelink}[2]{\hyperlink{cite.\therefsection @#1}{#2}}
\newcommand{\previousreference}{}
\providecommand{\reference}[2][]{%
	\notblank{#1}{\renewcommand{\previousreference}{#1}}{}%
	\todo[noline]{%
		\protect\vspace{16pt}%
		\protect\par%
		\protect\notblank{#1}{\cite{[\previousreference]}\\}{}%
		\protect\citelink{\previousreference}{p. #2}%
	}%
}

\definecolor{devcolor}{HTML}{00695c}
\newcommand{\dev}[1]{%
	\reversemarginpar%
	\todo[noline]{
		\protect\vspace{16pt}%
		\protect\par%
		\bfseries\color{devcolor}\href{\website/developpements/#1}{DEV}
	}%
	\normalmarginpar%
}

% En-têtes :

\pagestyle{fancy}
\fancyhead[L]{\truncate{0.23\textwidth}{\thepage}}
\fancyfoot[C]{\scriptsize \href{\website}{\texttt{agreg.skyost.eu}}}

% Couleurs :

\definecolor{property}{HTML}{fffde7}
\definecolor{proposition}{HTML}{fff8e1}
\definecolor{lemma}{HTML}{fff3e0}
\definecolor{theorem}{HTML}{fce4f2}
\definecolor{corollary}{HTML}{ffebee}
\definecolor{definition}{HTML}{ede7f6}
\definecolor{notation}{HTML}{f3e5f5}
\definecolor{example}{HTML}{e0f7fa}
\definecolor{cexample}{HTML}{efebe9}
\definecolor{application}{HTML}{e0f2f1}
\definecolor{remark}{HTML}{e8f5e9}
\definecolor{proof}{HTML}{e1f5fe}

% Théorèmes :

\theoremstyle{definition}
\newtheorem{theorem}{Théorème}

\newtheorem{property}[theorem]{Propriété}
\newtheorem{proposition}[theorem]{Proposition}
\newtheorem{lemma}[theorem]{Lemme}
\newtheorem{corollary}[theorem]{Corollaire}

\newtheorem{definition}[theorem]{Définition}
\newtheorem{notation}[theorem]{Notation}

\newtheorem{example}[theorem]{Exemple}
\newtheorem{cexample}[theorem]{Contre-exemple}
\newtheorem{application}[theorem]{Application}

\theoremstyle{remark}
\newtheorem{remark}[theorem]{Remarque}

\counterwithin*{theorem}{section}

\newcommand{\applystyletotheorem}[1]{
	\tcolorboxenvironment{#1}{
		enhanced,
		breakable,
		colback=#1!98!white,
		boxrule=0pt,
		boxsep=0pt,
		left=8pt,
		right=8pt,
		top=8pt,
		bottom=8pt,
		sharp corners,
		after=\par,
	}
}

\applystyletotheorem{property}
\applystyletotheorem{proposition}
\applystyletotheorem{lemma}
\applystyletotheorem{theorem}
\applystyletotheorem{corollary}
\applystyletotheorem{definition}
\applystyletotheorem{notation}
\applystyletotheorem{example}
\applystyletotheorem{cexample}
\applystyletotheorem{application}
\applystyletotheorem{remark}
\applystyletotheorem{proof}

% Environnements :

\NewEnviron{whitetabularx}[1]{%
	\renewcommand{\arraystretch}{2.5}
	\colorbox{white}{%
		\begin{tabularx}{\textwidth}{#1}%
			\BODY%
		\end{tabularx}%
	}%
}

% Maths :

\DeclareFontEncoding{FMS}{}{}
\DeclareFontSubstitution{FMS}{futm}{m}{n}
\DeclareFontEncoding{FMX}{}{}
\DeclareFontSubstitution{FMX}{futm}{m}{n}
\DeclareSymbolFont{fouriersymbols}{FMS}{futm}{m}{n}
\DeclareSymbolFont{fourierlargesymbols}{FMX}{futm}{m}{n}
\DeclareMathDelimiter{\VERT}{\mathord}{fouriersymbols}{152}{fourierlargesymbols}{147}


% Bibliographie :

\addbibresource{\bibliographypath}%
\defbibheading{bibliography}[\bibname]{%
	\newpage
	\section*{#1}%
}
\renewbibmacro*{entryhead:full}{\printfield{labeltitle}}%
\DeclareFieldFormat{url}{\newline\footnotesize\url{#1}}%

\AtEndDocument{\printbibliography}

\begin{document}
	%<*content>
	\lesson{analysis}{264}{Variables aléatoires discrètes. Exemples et applications.}

	Soient $(\Omega, \mathcal{A}, \mathbb{P})$ un espace probabilisé et $X : \Omega \rightarrow \mathbb{R}$ une variable aléatoire réelle. On munit $\mathbb{R}$ de sa tribu borélienne $\mathcal{B}(\mathbb{R})$.

	\subsection{Généralités}

	\subsubsection{Définitions}

	\reference[G-K]{335}

	\begin{definition}
		\begin{itemize}
			\item On dit qu'une loi $\mu$ est \textbf{discrète} s'il existe un ensemble $D$ fini tel que $\mu(D) = 1$.
			\item On dit que la variable aléatoire $X$ est discrète si sa loi $\mathbb{P}_X$ est discrète.
		\end{itemize}
	\end{definition}

	\reference[GOU21]{335}

	\begin{remark}
		Cela revient à dire que $X(\Omega)$ est fini ou est dénombrable.
	\end{remark}

	\begin{example}
		On pose $\Omega = \{ (\omega_n) \in \mathbb{R}^n \mid \omega_n \in \{ 0,1 \} \, \forall n \in \mathbb{N} \}$ et $X : (\omega_n) \mapsto \inf \{ n \in \mathbb{N} \mid \omega_n = 0 \}$. Alors $X$ est une variable aléatoire discrète, à valeurs dans $\mathbb{N} \cup \{ +\infty \}$.
	\end{example}

	\reference[G-K]{131}

	\begin{proposition}
		Si $X$ est une variable aléatoire discrète à valeurs dans un ensemble dénombrable $D$, alors :
		\begin{enumerate}[(i)]
			\item $\forall A \in \mathcal{B}(\mathbb{R}), \mathbb{P}_X(A) = \sum_{i \in D \cap A} \mathbb{P}(X=i)$.
			\item $\mathbb{P}_X = \sum_{i \in D} \mathbb{P}(X=i) \delta_i$ où les $\delta_i$ sont des masses de Dirac (voir \cref{264-1}).
		\end{enumerate}
	\end{proposition}

	\begin{remark}
		Si $D$ est un ensemble fini ou dénombrable et $(p_i)_{i \in D}$ est une famille de réels positifs de somme égale à $1$, alors en posant $\Omega = D$, $\mathcal{A} = \mathcal{P}(D)$, $X : \omega \mapsto \omega$ et $\mathbb{P} = \sum_{i \in D} \mathbb{P}(X = i) \delta_i$, on a construit une variable aléatoire discrète $X$ sur $(\Omega, \mathcal{A}, \mathbb{P})$.
	\end{remark}

	\subsubsection{Lois discrètes usuelles}

	\reference{137}

	\begin{definition}
		Si $A \subseteq \Omega$, l'application $\mathbb{1}_A$, appelée \textbf{indicatrice} de $A$ est définie sur $\Omega$ par
		\[
			\mathbb{1}_A :
			\begin{array}{ccc}
				\Omega &\rightarrow& \{ 0; 1 \} \\
				x &\mapsto& \begin{cases}
					1 \text{ si } x \in A \\
					0 \text{ sinon}
				\end{cases}
			\end{array}
		\]
	\end{definition}

	\begin{example}[Mesure de Dirac]
		\label{264-1}
		Si $x \in \Omega$, on pose $\delta_x : A \mapsto \mathbb{1}_A(x)$. C'est une loi discrète sur $\mathcal{P}(\Omega)$.
	\end{example}

	\begin{example}[Loi uniforme]
		Soit $E \subseteq \Omega$. On appelle loi uniforme sur $E$ la loi discrète définie sur $\mathcal{P}(\Omega)$ par
		\[
		\begin{array}{ccc}
			\mathcal{P}(\Omega) &\rightarrow& \{ 0; 1 \} \\
			A &\mapsto& \frac{\vert A \, \cap \, E \vert}{\vert E \vert}
		\end{array}
		\]
	\end{example}

	\begin{remark}
		Il s'agit du nombre de cas favorables sur le nombre de cas possibles. Ainsi, $X$ suit la loi uniforme sur $E$ si on a $\forall x \in E, \, \mathbb{P}(X=x) = \frac{1}{\vert E \vert}$ et $\forall x \notin E, \, \mathbb{P}(X=x) = 0$.
		\newpar
		C'est, par exemple, la loi suivie par une variable aléatoire représentant le lancer d'un dé non truqué avec $E = \llbracket 1, 6 \rrbracket$.
	\end{remark}

	\begin{example}[Loi de Bernoulli]
		$X$ suit une loi de Bernoulli de paramètre $p \in [0,1]$, notée $\mathcal{B}(p)$, si $\mathbb{P}(X=1) = p$ et $\mathbb{P}(X=0)=1-p$. Dans ce cas, $X$ est bien une loi discrète et on a
		\[ \mathbb{P}_X = (1-p) \delta_0 + p \delta_1 \]
	\end{example}

	\begin{example}[Loi binomiale]
		$X$ suit une loi de binomiale de paramètres $n \in \mathbb{N}$ et $p \in [0,1]$, notée $\mathcal{B}(n, p)$, si $X$ est la somme de $n$ variables aléatoires indépendantes qui suivent des lois de Bernoulli de paramètre $p$. Dans ce cas, $X$ est bien une loi discrète et on a
		\[ \forall k \in \mathbb{N}, \, \mathbb{P}(X = k) = \binom{n}{k} p^k (1-p)^{n-k} \]
	\end{example}

	\begin{remark}
		Il s'agit du nombre de succès pour $n$ tentatives.
		\newpar
		C'est, par exemple, la loi suivie par une variable aléatoire représentant le nombre de ``Pile'' obtenus lors d'un lancer de pièce équilibrée.
	\end{remark}

	\begin{example}[Loi géométrique]
		$X$ suit une loi géométrique de paramètre $p \in ]0,1]$, notée $\mathcal{G}(p)$, si l'on a
		\[ \forall k \in \mathbb{N}^{*}, \, \mathbb{P}(X = k) = p(1-p)^{k-1} \]
	\end{example}

	\begin{remark}
		Il s'agit d'une succession de $k-1$ échecs consécutifs suivie d'un succès.
		\newpar
		C'est, par exemple, la loi suivie par une variable aléatoire représentant le nombre de lancers effectués avant d'obtenir ``Pile'' lors d'un lancer de pièce équilibrée.
	\end{remark}

	\begin{example}[Loi de Poisson]
		$X$ suit une loi de Poisson de paramètre $\lambda > 0$, notée $\mathcal{P}(\lambda)$, si l'on a
		\[ \forall k \in \mathbb{N}^{*}, \, \mathbb{P}(X = k) = e^{-\lambda} \frac{\lambda^k}{k!} \]
	\end{example}

	\reference{298}

	\begin{remark}
		Cette loi est une bonne modélisation pour le nombre de fois où un événement rare survient (par exemple, un tremblement de terre).
	\end{remark}

	\subsection{Propriétés spécifiques aux variables aléatoires discrètes}

	\subsubsection{Indépendance}

	\reference{128}

	\begin{definition}
		On dit que des variables aléatoires $X_1, \dots X_n$, sont \textbf{indépendantes} si
		\[ \mathbb{P}_{(X_1, \dots, X_n)} = \bigotimes_{i=1}^n \mathbb{P}_{X_i} \]
	\end{definition}

	\reference{238}

	\begin{example}
		Si $X_1$ et $X_2$ sont des variables aléatoires indépendantes suivant des lois de Poisson de paramètres respectifs $\lambda$ et $\mu$, alors $X_1 + X_2$ suit une loi de Poisson de paramètre $\lambda + \mu$.
	\end{example}

	\begin{cexample}
		Soient $X_1$ et $X_2$ deux variables aléatoires indépendantes telles que
		\[ \forall i \in \llbracket 1, 2 \rrbracket, \, \mathbb{P}(X_i = 1) = \mathbb{P}(X_i = -1) = \frac{1}{2} \]
		On pose $X_3 = X_1 X_2$. Alors, $X_2$ et $X_3$ sont indépendantes, $X_1$ et $X_3$ aussi, mais $X_1$, $X_2$ et $X_3$ ne le sont pas.
	\end{cexample}

	\reference[GOU21]{337}

	\begin{proposition}
		Des variables aléatoires discrètes $X_1, \dots, X_n$ sont indépendantes si et seulement si
		\[ \forall j \in \llbracket, 1, n \rrbracket, \, \forall x_j \in X_j(\Omega), \, \mathbb{P}(X_1 = x_1, \dots, X_n = x_n) = \prod_{j = 1}^n \mathbb{P}(X = x_i) \]
	\end{proposition}

	\begin{proposition}
		Soient $X_1, \dots, X_n$ des variables aléatoires discrètes définies sur $(\Omega, \mathcal{A}, \mathbb{P})$, $f : X_1(\Omega) \times \dots \times X_m(\Omega) \rightarrow F$ et $g : X_{m+1}(\Omega) \times \dots \times X_n(\Omega) \rightarrow F'$ deux fonctions. Si $X_1, \dots, X_n$ sont indépendantes, alors il en est de même de $f(X_1, \dots, X_m)$ et $g(X_{m+1}, \dots, X_n)$.
	\end{proposition}

	\subsubsection{Espérance}

	\reference{159}

	\begin{definition}
		\begin{itemize}
			\item On note $\mathcal{L}_1(\Omega, \mathcal{A}, \mathbb{P})$ (ou simplement $\mathcal{L}_1(\Omega)$ voire $\mathcal{L}_1$ s'il n'y a pas d'ambiguïté) l'espace des variables aléatoires intégrables sur $(\Omega, \mathcal{A}, \mathbb{P})$.
			\item Si $X \in \mathcal{L}_1$, on peut définir son \textbf{espérance}
			\[ \mathbb{E}(X) = \int_\Omega X(\omega) \, \mathrm{d}\mathbb{P}(\omega) \]
		\end{itemize}
	\end{definition}

	\reference{164}

	\begin{theorem}[Transfert]
		Si $X$ est une variable aléatoire dont la loi $\mathbb{P}_X$ admet une densité $f$ par rapport à $\mathbb{P}$ et si $g$ est une fonction mesurable, alors
		\[ g(X) \in \mathcal{L}_1 \iff \int_{\mathbb{R}} \vert g(x) \vert f(x) \, \mathrm{d}\mathbb{P}(x) < +\infty \]
		et dans ce cas,
		\[ \mathbb{E}(g(X)) = \int_{\mathbb{R}} g(x) f(x) \, \mathrm{d}\mathbb{P}(x) \]
	\end{theorem}

	\begin{corollary}
		Soit $g$ une fonction mesurable. Si $X$ est une variable aléatoire discrète telle que $X(\Omega) = D$, alors
		\[ g(X) \in \mathcal{L}_1 \iff \sum_{i \in D} \vert g(i) \vert \mathbb{P}(X = i) < +\infty \]
		et dans ce cas,
		\[ \mathbb{E}(g(X)) = \sum_{i \in D} g(i) \mathbb{P}(X = i) \]
	\end{corollary}

	\begin{remark}
		En reprenant les notations précédentes, et avec $g : x \mapsto x$, on a
		\[ X \in \mathcal{L}_1 \iff \sum_{i \in D} \vert i \vert \mathbb{P}(X = i) < +\infty \]
		et dans ce cas,
		\[ \mathbb{E}(X) = \sum_{i \in D} i \mathbb{P}(X = i) \]
	\end{remark}

	\reference{187}

	\begin{example}
		\begin{itemize}
			\item $\mathbb{E}(\mathbb{1}_A) = \mathbb{P}(A)$.
			\item $X \sim \mathcal{B}(n, p) \implies \mathbb{E}(X) = np$.
			\item $X \sim \mathcal{G}(p) \implies \mathbb{E}(X) = \frac{1}{p}$.
			\item $X \sim \mathcal{P}(\lambda) \implies \mathbb{E}(X) = \lambda$.
		\end{itemize}
	\end{example}

	\reference{171}

	\begin{proposition}
		Si $X$ est à valeurs dans $(\mathbb{N}, \mathcal{P}(\mathbb{N}))$, alors $\mathbb{E}(X) = \sum_{k=0}^{+\infty} \mathbb{P}(X > k)$.
	\end{proposition}

	\subsubsection{Fonctions génératrices}

	On suppose dans cette sous-section que $X$ est à valeurs dans $(\mathbb{N}, \mathcal{P}(\mathbb{N}))$.

	\reference{235}

	\begin{definition}
		On appelle \textbf{fonction génératrice} de $X$ la fonction
		\[
			G_X :
			\begin{array}{ccc}
				[-1,1] &\rightarrow& \mathbb{R} \\
				z &\mapsto& \sum_{k=0}^{+\infty} \mathbb{P}(X=k) z^k
			\end{array}
		\]
	\end{definition}

	\begin{example}
		\begin{itemize}
			\item $X \sim \mathcal{B}(p) \implies \forall s \in [-1,1], \, G_X(s) = (1-p) + ps$.
			\item $X \sim \mathcal{B}(n, p) \implies \forall s \in [-1,1], \, G_X(s) = ((1-p) + ps)^n$.
			\item $X \sim \mathcal{G}(p) \implies \forall s \in [-1,1], \, G_X(s) = \frac{ps}{1-(1-p)s}$.
			\item $X \sim \mathcal{P}(\lambda) \implies \forall s \in [-1,1], \, G_X(s) = e^{-\lambda (1-s)}$.
		\end{itemize}
	\end{example}

	\begin{theorem}
		Soient $X_1$ et $X_2$ deux variables aléatoires indépendantes et $\mathcal{L}_1$. Alors,
		\[ \mathbb{E}(X_1 X_2) = \mathbb{E}(X_1) \mathbb{E}(X_2) \]
	\end{theorem}

	\begin{corollary}
		Soient $X_1$ et $X_2$ deux variables aléatoires indépendantes et à valeurs dans $\mathbb{N}$. Alors,
		\[ G_{X_1 X_2} = G_{X_1} + G_{X_2} \]
	\end{corollary}

	\begin{theorem}
		Sur $[0,1]$, la fonction $G_X$ est infiniment dérivable et ses dérivées sont toutes positives, avec
		\[ G_X^{(n)}(s) = \mathbb{E}(X(X-1) \dots (X-n+1)s^{X-n}) \]
		En particulier,
		\[ \mathbb{P}(X=n) = \frac{G_X^{(n)}(0)}{n!} \]
		ce qui montre que la fonction génératrice caractérise la loi.
	\end{theorem}

	\reference[GOU21]{346}

	\begin{example}
		Si $X_1 \sim \mathcal{B}(n, p)$ et $X_2 \sim \mathcal{B}(m, p)$ sont indépendantes, alors $X_1 + X_2 \sim \mathcal{B}(n + m, p)$.
	\end{example}

	\reference[G-K]{238}

	\begin{theorem}
		$X \in \mathcal{L}_1$ si et seulement si $G_X$ admet une dérivée à gauche en $1$. Dans ce cas, $G'_X(1) = \mathbb{E}(X)$.
	\end{theorem}

	\subsection{Application en analyse réelle}

	\reference{171}

	\begin{definition}
		On dit que $X$ \textbf{admet un moment d'ordre $2$} si elle est de carré intégrable, ie. $X^2 \in \mathcal{L}_1$. On note $\mathcal{L}_1(\Omega, \mathcal{A}, \mathbb{P})$ (ou simplement $\mathcal{L}_1(\Omega)$ voire $\mathcal{L}_1$ s'il n'y a pas d'ambiguïté) l'espace des variables aléatoires de carré intégrable.
	\end{definition}

	\begin{proposition}
		\[ X_1, X_2 \in \mathcal{L}_2 \implies X_1 X_2 \in \mathcal{L}_1 \]
		En particulier, $X_1 \in \mathcal{L}_2 \implies X_1 \in \mathcal{L}_1$.
	\end{proposition}

	\begin{definition}
		Soient $X_1$ et $X_2$ deux variables aléatoires admettant chacune un moment d'ordre $2$.
		\begin{itemize}
			\item On appelle \textbf{covariance} du couple $(X_1, X_2)$ le réel
			\[ \operatorname{Covar}(X_1, X_2) = \mathbb{E}((X_1 - \mathbb{E}(X_1))(X_2 - \mathbb{E}(X_2))) \]
			\item On appelle \textbf{variance} de $X_1$ le réel positif
			\[ \operatorname{Var}(X_1) = \operatorname{Covar}(X_1, X_1) = \mathbb{E}(X_1 - \mathbb{E}(X_1))^2 = \mathbb{E}(X_1^2) - (\mathbb{E}(X_1))^2  \]
		\end{itemize}
	\end{definition}

	\reference[GOU21]{346}

	\begin{proposition}
		Si $X$ est à valeurs dans $\mathbb{N}$, alors $X \in \mathcal{L}_2$ si et seulement si $G_X \in \mathcal{C}^2([0,1])$, et dans ce cas,
		\[ \operatorname{Var}(X) = G''_X(1) + G'_X(1) - G'_X(1)^2 \]
	\end{proposition}

	\reference[G-K]{186}

	\begin{example}
		\begin{itemize}
			\item $\operatorname{Var}(\mathbb{1}_A) = \mathbb{P}(A)$.
			\item $X \sim \mathcal{B}(n, p) \implies \operatorname{Var}(X) = np (1-p)$.
			\item $X \sim \mathcal{G}(p) \implies \operatorname{Var}(X) = \frac{1-p}{p^2}$.
			\item $X \sim \mathcal{P}(\lambda) \implies \operatorname{Var}(X) = \lambda$.
		\end{itemize}
	\end{example}

	\reference{177}

	\begin{proposition}[Inégalité de Bienaymé-Tchebychev]
		On suppose $X \in \mathcal{L}_2$. Alors,
		\[ \forall a > 0, \, \mathbb{P}(\vert X - \mathbb{E}(X) \vert \geq a) \leq \frac{\operatorname{Var}(X)}{a^2} \]
	\end{proposition}

	\reference{195}
	\dev{theoreme-de-weierstrass-par-les-probabilites}

	\begin{theorem}[Bernstein]
		Soit $f : [0,1] \rightarrow \mathbb{R}$ continue. On note
		\[ \forall n \in \mathbb{N}^*, \, B_n(f) : x \mapsto \sum_{k=0}^n \binom{n}{k} f \left( \frac{k}{n} \right) x^k (1-x)^{n-k} \]
		le $n$-ième polynôme de Bernstein associé à $f$. Alors le suite de fonctions $(B_n(f))$ converge uniformément vers $f$.
	\end{theorem}

	\begin{theorem}[Weierstrass]
		Toute fonction continue $f : [a,b] \rightarrow \mathbb{R}$ (avec $a, b \in \mathbb{R}$ tels que $a \leq b$) est limite uniforme de fonctions polynômiales sur $[a, b]$.
	\end{theorem}

	\subsection{Théorèmes limites et d'approximations}

	\subsubsection{Théorèmes limites}

	\reference[Z-Q]{536}

	\begin{theorem}[Lévy]
		Soient $(X_n)$ une suite de variables aléatoires réelles et $X$ une variable aléatoire réelle. Alors :
		\[ X_n \overset{(d)}{\longrightarrow} X \iff \phi_{X_n} \text{ converge simplement vers } \phi_X \]
		où $\phi_Y$ désigne la fonction caractéristique d'une variable aléatoire réelle $Y$.
	\end{theorem}

	\reference[G-K]{307}
	\dev{theoreme-central-limite}

	\begin{theorem}[Central limite]
		Soit $(X_n)$ une suite de variables aléatoires réelles indépendantes de même loi admettant un moment d'ordre $2$. On note $m$ l'espérance et $\sigma^2$ la variance commune à ces variables. On pose $S_n = X_1 + \dots + X_n - nm$. Alors,
		\[ \left ( \frac{S_n}{\sqrt{n}} \right) \overset{(d)}{\longrightarrow} \mathcal{N}(0, \sigma^2) \]
	\end{theorem}

	\reference{270}

	\begin{theorem}[Loi faible des grands nombres]
		Soit $(X_n)$ une suite de variables aléatoires deux à deux indépendantes de même loi et $\mathcal{L}_1$. On pose $M_n = \frac{X_1 + \dots + X_n}{n}$. Alors,
		\[ M_n \overset{(p)}{\longrightarrow} \mathbb{E}(X_1) \]
	\end{theorem}

	\reference[Z-Q]{526}

	\begin{theorem}[Loi forte des grands nombres]
		Soit $(X_n)$ une suite de variables aléatoires mutuellement indépendantes de même loi. On pose $M_n = \frac{X_1 + \dots + X_n}{n}$. Alors,
		\[ X_1 \in \mathcal{L}_1 \iff M_n \overset{(ps.)}{\longrightarrow} \ell \in \mathbb{R} \]
		Dans ce cas, on a $\ell = \mathbb{E}(X_1)$.
	\end{theorem}

	\subsubsection{Approximation d'une loi normale}

	\reference[G-K]{308}

	\begin{theorem}[Moivre-Laplace]
		Soit $(X_n)$ une suite de variables aléatoires indépendantes de même loi $\mathcal{B}(p)$. Alors,
		\[ \frac{\sum_{k=1}^{n} X_k - np}{\sqrt{n}} \overset{(d)}{\longrightarrow} \mathcal{N}(0, p(1-p)) \]
	\end{theorem}

	\subsubsection{Approximation d'une loi de Poisson}

	\reference{297}

	\begin{theorem}
		Soit, pour $n \geq 1$, une variable aléatoire $X_n$ suivant la loi binomiale de paramètres $n$ et $p_n$. On suppose que $\lim_{n \rightarrow +\infty} n p_n = \lambda > 0$.
		Alors,
		\[ X_n \overset{(d)}{\longrightarrow} \mathcal{P}(\lambda) \]
	\end{theorem}

	\begin{remark}
		En pratique, pour $n \geq 30$ et $np \leq 10$, on a une ``bonne'' approximation de $\mathcal{P}(\lambda)$.
	\end{remark}

	\reference[GOU21]{343}

	\begin{example}
		Si chaque seconde, il y a une probabilité $p = \frac{1}{600}$ qu'un client entre dans un magasin, le nombre de clients qui entrent sut un intervalle d'une heure suit approximativement une loi de Poisson de paramètre $\lambda = 3600p = 6$.
		\newpar
		Pour cette raison, on appelle parfois cette loi la \textit{loi des événements rares}.
	\end{example}

	\reference[G-K]{297}

	\begin{application}[Nombre de dérangements]
		Soit $\sigma_n$ une permutation aléatoire suivant la loi uniforme sur $S_n$. Si on note $D_n$ le nombre de points fixes de $\sigma_n$, on a
		\[ \mathbb{P}(D_n = k) = \frac{1}{k!} \frac{d_{n-k}}{(n-k)!} \]
		où $d_n$ est le nombre de permutations de $S_n$ sans point fixe. En particulier, comme $d_n \sim \frac{1}{e} n!$, on a
		\[ D_n \overset{(d)}{\longrightarrow} \mathcal{P}(1) \]
	\end{application}
	%</content>
\end{document}
