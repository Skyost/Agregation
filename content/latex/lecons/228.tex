\documentclass[12pt, a4paper]{report}

% LuaLaTeX :

\RequirePackage{iftex}
\RequireLuaTeX

% Packages :

\usepackage[french]{babel}
%\usepackage[utf8]{inputenc}
%\usepackage[T1]{fontenc}
\usepackage[pdfencoding=auto, pdfauthor={Hugo Delaunay}, pdfsubject={Mathématiques}, pdfcreator={agreg.skyost.eu}]{hyperref}
\usepackage{amsmath}
\usepackage{amsthm}
%\usepackage{amssymb}
\usepackage{stmaryrd}
\usepackage{tikz}
\usepackage{tkz-euclide}
\usepackage{fourier-otf}
\usepackage{fontspec}
\usepackage{titlesec}
\usepackage{fancyhdr}
\usepackage{catchfilebetweentags}
\usepackage[french, capitalise, noabbrev]{cleveref}
\usepackage[fit, breakall]{truncate}
\usepackage[top=2.5cm, right=2cm, bottom=2.5cm, left=2cm]{geometry}
\usepackage{enumerate}
\usepackage{tocloft}
\usepackage{microtype}
%\usepackage{mdframed}
%\usepackage{thmtools}
\usepackage{xcolor}
\usepackage{tabularx}
\usepackage{aligned-overset}
\usepackage[subpreambles=true]{standalone}
\usepackage{environ}
\usepackage[normalem]{ulem}
\usepackage{marginnote}
\usepackage{etoolbox}
\usepackage{setspace}
\usepackage[bibstyle=reading, citestyle=draft]{biblatex}
\usepackage{xpatch}
\usepackage[many, breakable]{tcolorbox}
\usepackage[backgroundcolor=white, bordercolor=white, textsize=small]{todonotes}

% Bibliographie :

\newcommand{\overridebibliographypath}[1]{\providecommand{\bibliographypath}{#1}}
\overridebibliographypath{../bibliography.bib}
\addbibresource{\bibliographypath}
\defbibheading{bibliography}[\bibname]{%
	\newpage
	\section*{#1}%
}
\renewbibmacro*{entryhead:full}{\printfield{labeltitle}}
\DeclareFieldFormat{url}{\newline\footnotesize\url{#1}}
\AtEndDocument{\printbibliography}

% Police :

\setmathfont{Erewhon Math}

% Tikz :

\usetikzlibrary{calc}

% Longueurs :

\setlength{\parindent}{0pt}
\setlength{\headheight}{15pt}
\setlength{\fboxsep}{0pt}
\titlespacing*{\chapter}{0pt}{-20pt}{10pt}
\setlength{\marginparwidth}{1.5cm}
\setstretch{1.1}

% Métadonnées :

\author{agreg.skyost.eu}
\date{\today}

% Titres :

\setcounter{secnumdepth}{3}

\renewcommand{\thechapter}{\Roman{chapter}}
\renewcommand{\thesubsection}{\Roman{subsection}}
\renewcommand{\thesubsubsection}{\arabic{subsubsection}}
\renewcommand{\theparagraph}{\alph{paragraph}}

\titleformat{\chapter}{\huge\bfseries}{\thechapter}{20pt}{\huge\bfseries}
\titleformat*{\section}{\LARGE\bfseries}
\titleformat{\subsection}{\Large\bfseries}{\thesubsection \, - \,}{0pt}{\Large\bfseries}
\titleformat{\subsubsection}{\large\bfseries}{\thesubsubsection. \,}{0pt}{\large\bfseries}
\titleformat{\paragraph}{\bfseries}{\theparagraph. \,}{0pt}{\bfseries}

\setcounter{secnumdepth}{4}

% Table des matières :

\renewcommand{\cftsecleader}{\cftdotfill{\cftdotsep}}
\addtolength{\cftsecnumwidth}{10pt}

% Redéfinition des commandes :

\renewcommand*\thesection{\arabic{section}}
\renewcommand{\ker}{\mathrm{Ker}}

% Nouvelles commandes :

\newcommand{\website}{https://agreg.skyost.eu}

\newcommand{\tr}[1]{\mathstrut ^t #1}
\newcommand{\im}{\mathrm{Im}}
\newcommand{\rang}{\operatorname{rang}}
\newcommand{\trace}{\operatorname{trace}}
\newcommand{\id}{\operatorname{id}}
\newcommand{\stab}{\operatorname{Stab}}

\providecommand{\newpar}{\\[\medskipamount]}

\providecommand{\lesson}[3]{%
	\title{#3}%
	\hypersetup{pdftitle={#3}}%
	\setcounter{section}{\numexpr #2 - 1}%
	\section{#3}%
	\fancyhead[R]{\truncate{0.73\textwidth}{#2 : #3}}%
}

\providecommand{\development}[3]{%
	\title{#3}%
	\hypersetup{pdftitle={#3}}%
	\section*{#3}%
	\fancyhead[R]{\truncate{0.73\textwidth}{#3}}%
}

\providecommand{\summary}[1]{%
	\textit{#1}%
	\medskip%
}

\tikzset{notestyleraw/.append style={inner sep=0pt, rounded corners=0pt, align=center}}

%\newcommand{\booklink}[1]{\website/bibliographie\##1}
\newcommand{\citelink}[2]{\hyperlink{cite.\therefsection @#1}{#2}}
\newcommand{\previousreference}{}
\providecommand{\reference}[2][]{%
	\notblank{#1}{\renewcommand{\previousreference}{#1}}{}%
	\todo[noline]{%
		\protect\vspace{16pt}%
		\protect\par%
		\protect\notblank{#1}{\cite{[\previousreference]}\\}{}%
		\protect\citelink{\previousreference}{p. #2}%
	}%
}

\definecolor{devcolor}{HTML}{00695c}
\newcommand{\dev}[1]{%
	\reversemarginpar%
	\todo[noline]{
		\protect\vspace{16pt}%
		\protect\par%
		\bfseries\color{devcolor}\href{\website/developpements/#1}{DEV}
	}%
	\normalmarginpar%
}

% En-têtes :

\pagestyle{fancy}
\fancyhead[L]{\truncate{0.23\textwidth}{\thepage}}
\fancyfoot[C]{\scriptsize \href{\website}{\texttt{agreg.skyost.eu}}}

% Couleurs :

\definecolor{property}{HTML}{fffde7}
\definecolor{proposition}{HTML}{fff8e1}
\definecolor{lemma}{HTML}{fff3e0}
\definecolor{theorem}{HTML}{fce4f2}
\definecolor{corollary}{HTML}{ffebee}
\definecolor{definition}{HTML}{ede7f6}
\definecolor{notation}{HTML}{f3e5f5}
\definecolor{example}{HTML}{e0f7fa}
\definecolor{cexample}{HTML}{efebe9}
\definecolor{application}{HTML}{e0f2f1}
\definecolor{remark}{HTML}{e8f5e9}
\definecolor{proof}{HTML}{e1f5fe}

% Théorèmes :

\theoremstyle{definition}
\newtheorem{theorem}{Théorème}

\newtheorem{property}[theorem]{Propriété}
\newtheorem{proposition}[theorem]{Proposition}
\newtheorem{lemma}[theorem]{Lemme}
\newtheorem{corollary}[theorem]{Corollaire}

\newtheorem{definition}[theorem]{Définition}
\newtheorem{notation}[theorem]{Notation}

\newtheorem{example}[theorem]{Exemple}
\newtheorem{cexample}[theorem]{Contre-exemple}
\newtheorem{application}[theorem]{Application}

\theoremstyle{remark}
\newtheorem{remark}[theorem]{Remarque}

\counterwithin*{theorem}{section}

\newcommand{\applystyletotheorem}[1]{
	\tcolorboxenvironment{#1}{
		enhanced,
		breakable,
		colback=#1!98!white,
		boxrule=0pt,
		boxsep=0pt,
		left=8pt,
		right=8pt,
		top=8pt,
		bottom=8pt,
		sharp corners,
		after=\par,
	}
}

\applystyletotheorem{property}
\applystyletotheorem{proposition}
\applystyletotheorem{lemma}
\applystyletotheorem{theorem}
\applystyletotheorem{corollary}
\applystyletotheorem{definition}
\applystyletotheorem{notation}
\applystyletotheorem{example}
\applystyletotheorem{cexample}
\applystyletotheorem{application}
\applystyletotheorem{remark}
\applystyletotheorem{proof}

% Environnements :

\NewEnviron{whitetabularx}[1]{%
	\renewcommand{\arraystretch}{2.5}
	\colorbox{white}{%
		\begin{tabularx}{\textwidth}{#1}%
			\BODY%
		\end{tabularx}%
	}%
}

% Maths :

\DeclareFontEncoding{FMS}{}{}
\DeclareFontSubstitution{FMS}{futm}{m}{n}
\DeclareFontEncoding{FMX}{}{}
\DeclareFontSubstitution{FMX}{futm}{m}{n}
\DeclareSymbolFont{fouriersymbols}{FMS}{futm}{m}{n}
\DeclareSymbolFont{fourierlargesymbols}{FMX}{futm}{m}{n}
\DeclareMathDelimiter{\VERT}{\mathord}{fouriersymbols}{152}{fourierlargesymbols}{147}


% Bibliographie :

\addbibresource{\bibliographypath}%
\defbibheading{bibliography}[\bibname]{%
	\newpage
	\section*{#1}%
}
\renewbibmacro*{entryhead:full}{\printfield{labeltitle}}%
\DeclareFieldFormat{url}{\newline\footnotesize\url{#1}}%

\AtEndDocument{\printbibliography}

\begin{document}
	%<*content>
	\lesson{analysis}{228}{Continuité, dérivabilité des fonctions réelles d'une variable réelle. Exemples et applications.}
	
	Soient $I$ un intervalle de $\mathbb{R}$ non réduit à un point et $f : I \rightarrow \mathbb{R}$ une fonction.

	\subsection{Continuité et dérivabilité}
	
	\subsubsection{Continuité}
	
	\reference[ROM19]{163}
	
	\begin{definition}
		\begin{itemize}
			\item $f$ est \textbf{continue au point $a \in I$} si
			\[ \forall \epsilon > 0, \, \exists \eta > 0, \, \forall x \in I, \, \vert x - a \vert < \eta \implies \vert f(x) - f(a) \vert < \epsilon \]
			\item $f$ est \textbf{continue sur $I$} si $f$ est continue en tout point de $I$.
		\end{itemize}
	\end{definition}
	
	\begin{example}
		Pour tout entier $n$, $x \mapsto x^n$ est continue sur $\mathbb{R}$.
	\end{example}
	
	\begin{theorem}[Caractérisations séquentielle et topologique de la continuité]
		\begin{enumerate}[(i)]
			\item $f$ est continue en $a \in I$ si et seulement si toute suite de points de $I$ qui converge vers $a$ est transformée par $f$ en une suite convergente vers $f(a)$.
			\item $f$ est continue en $a \in I$ si et seulement si l'image réciproque par $f$ de tout ouvert (resp. fermé) de $\mathbb{R}$ est un ouvert (resp. fermé) de $I$.
		\end{enumerate}
	\end{theorem}
	
	\begin{example}
		La fonction $x \mapsto \cos \left( \frac{1}{x} \right)$ définie sur $\mathbb{R}_*$ n'est pas continue en $0$.
	\end{example}
	
	\begin{proposition}
		Si $f$ et $g$ sont deux fonctions définies sur $I$ à valeurs réelles et continues en $a \in I$, alors $\vert f \vert$, $f + g$, $fg$, $\min(f,g)$ et $\max(f,g)$ sont continues en $a$.
	\end{proposition}
	
	\subsubsection{Uniforme continuité}
	
	\reference[GOU20]{12}
	
	\begin{definition}
		$f$ est \textbf{uniformément continue sur $I$} si
		\[ \forall \epsilon > 0, \, \exists \eta > 0, \, \forall x, y \in I, \, \vert x - y \vert < \eta \implies \vert f(x) - f(y) < \epsilon \]
	\end{definition}
	
	\begin{remark}
		En particulier, une fonction uniformément continue sur un intervalle est continue sur ce même intervalle.
	\end{remark}
	
	\begin{example}
		Une fonction lipschitzienne sur $I$ est uniformément continue sur $I$.
	\end{example}
	
	\begin{cexample}
		La fonction $x \mapsto \frac{1}{x}$ définie sur $]0,1]$ est continue mais n'est pas uniformément continue.
	\end{cexample}
	
	\reference{18}
	
	\begin{proposition}
		On se place dans le cas où $I = \mathbb{R}^+$ et on suppose $f$ uniformément continue sur $\mathbb{R}^+$. Alors,
		\[ \exists \alpha, \beta \in \mathbb{R}^+_* \text{ tels que } \forall x \in \mathbb{R}^+, \vert f(x) \vert \leq \alpha x + \beta \]
	\end{proposition}
	
	\reference{24}
	
	\begin{theorem}[Prolongement des applications uniformément continues]
		Soit $J \subseteq I$ dense dans $I$ et soit $g : J \rightarrow \mathbb{R}$ uniformément continue sur $J$. Alors,
		\[ \exists! h : I \rightarrow \mathbb{R} \text{ uniformément continue et telle que } h_{|J} = g \]
	\end{theorem}
	
	\subsubsection{Dérivabilité}
	
	\reference{71}
	
	\begin{definition}
		On dit que $f$ est \textbf{dérivable en $a \in I$} si
		\[ \lim_{\substack{t \rightarrow a \\ t \neq a}} \frac{f(t) - f(a)}{t-a} \]
		existe. Lorsque cette limite existe, elle est notée $f'(a)$.
	\end{definition}
	
	\begin{remark}
		\begin{itemize}
			\item De même, $f$ est \textbf{dérivable à gauche (resp. à droite) en $a \in I$} si $\lim_{\substack{t \rightarrow a \\ t < a \\ t \in I}} \frac{f(t) - f(a)}{t-a}$ existe (resp. $\lim_{\substack{t \rightarrow a \\ t > a \\ t \in I}} \frac{f(t) - f(a)}{t-a}$ existe). On la note alors $f'_g(a)$ (resp. $f'_d(a)$).
			\item $f$ est dérivable en $a \in I$ si et seulement si $f$ est dérivable à gauche, à droite et $f'_g(a) = f'_d(a)$.
			\item $f$ est dérivable en $a \in I$ si et seulement si, quand $x$ tend vers $a$,
			\[ \exists \ell \in \mathbb{R}, \, f(x) = f(a) + (x-a)\ell + o(x-a) \]
		\end{itemize}
	\end{remark}
	
	\begin{proposition}
		Si $f$ est dérivable en $a \in I$, alors $f$ est continue en $a$.
	\end{proposition}
	
	\reference{86}
	
	\begin{cexample}
		On note $\Delta$ la fonction définie sur $\mathbb{R}$ $1$-périodique et telle que la restriction à $\left[ -\frac{1}{2}, \frac{1}{2} \right]$ vérifie $\Delta(x) = \vert x \vert$. Alors,
		\[ f : x \mapsto \sum_{p=0}^{+\infty} \frac{1}{2^p} \Delta (2^p x) \]
		est bien définie, continue sur $\mathbb{R}$, mais dérivable en aucun point de $\mathbb{R}$.
	\end{cexample}
	
	\reference{72}
	
	\begin{remark}
		La fonction dérivée $f' : x \mapsto f'(x)$ n'est pas forcément continue là où elle est définie.
	\end{remark}
	
	\begin{example}
		La fonction $x \mapsto \begin{cases}
			x^2 \sin \left(\frac{1}{x}\right) \text{ si x } \neq 0 \\
			0 \text{ sinon}
		\end{cases}$ définie sur $\mathbb{R}$ est dérivable, de dérivée $x \mapsto \begin{cases}
		2x \sin \left(\frac{1}{x}\right) - \cos \left(\frac{1}{x}\right) \text{ si x } \neq 0 \\
		0 \text{ sinon}
		\end{cases}$ définie sur $\mathbb{R}$ mais non continue en $0$.
	\end{example}
	
	\begin{proposition}
		Si $f$ et $g$ sont deux fonctions définies sur $I$ à valeurs réelles et dérivables en $a \in I$. Alors :
		\begin{enumerate}[(i)]
			\item $\forall \lambda, \mu \in \mathbb{R}, \, \lambda f + \mu g$ est dérivable en $a$ et $(\lambda f + \mu g)'(a) = \lambda f'(a) + \mu g'(a)$.
			\item $fg$ est dérivable en $a$ et $(fg)'(a) = f'(a) g(a) + f(a) g'(a)$.
			\item Si $g(a) \neq 0$, alors $\frac{f}{g}$ est dérivable en $a$ et $\left(\frac{f}{g}\right)'(a) = \frac{f'(a) g(a) - f(a) g'(a)}{g(a)^2}$.
 		\end{enumerate}
	\end{proposition}
	
	\begin{definition}
		On dit que $f$ est \textbf{de classe $\mathcal{C}^n$ sur $I$} si $\forall k \in \llbracket 0, n \rrbracket$, $f^{(k)}$ (la dérivée $k$-ième de $f$) existe et continue.
	\end{definition}
	
	\begin{proposition}[Formule de Leibniz]
		Soit $a \in I$. Si $f$ et $g$ sont deux fonctions définies sur $I$ à valeurs réelles et qui admettent une dérivée $n$-ième en $a$,
		\[ (fg)^{(n)}(a) = \sum_{k=0}^n \binom{n}{k} f^{(k)}(a) g^{(n-k)}(a) \]
	\end{proposition}
	
	\begin{proposition}
		Soit $J$ un intervalle de $\mathbb{R}$. Si $f : I \rightarrow \mathbb{R}$ et $g : J \rightarrow I$ sont deux fonctions, alors, en supposant $f$ dérivable en $a$ et $g$ dérivable en $f(a)$, $(f \circ g)$ est dérivable en $a$ et,
		\[ (f \circ g)'(a) = g'(a) (f' \circ g)(a) \]
	\end{proposition}
	
	\begin{corollary}
		Soient $J$ un intervalle de $\mathbb{R}$ et $h : I \rightarrow J$ une bijection dérivable en $a \in I$. Alors, $h^{-1}$ est dérivable en $b = h(a)$ si et seulement si $h'(a) \neq 0$, et on a,
		\[ (f^{-1})'(b) = \frac{1}{f'(a)} = \frac{1}{f'(f^{-1}(b))} \]
	\end{corollary}
	
	\begin{corollary}
		La composée de deux applications de classe $\mathcal{C}^n$ est de classe $\mathcal{C}^n$.
	\end{corollary}

	\subsection{Fonctions particulières qui sont dérivables ou continues}
	
	\subsubsection{Fonctions convexes}
	
	\reference[ROM19]{225}
	
	\begin{definition}
		$f$ est \textbf{convexe} si
		\[ \forall x, y \in I, \, \forall t \in [0,1], \, f((1-t)x + ty) \leq (1-t)f(x) + tf(y) \]
	\end{definition}
	
	\begin{example}
		\begin{itemize}
			\item $x \mapsto \vert x \vert$ est convexe sur $\mathbb{R}$.
			\item $\exp$ est convexe sur $\mathbb{R}$.
		\end{itemize}
	\end{example}

	\reference[GOU20]{96}
	
	\begin{proposition}
		Si $f$ est convexe, elle possède en tout point de $\mathring{I}$ une dérivée à droite et une dérivée à gauche. Elle est donc continue sur $\mathring{I}$. De plus les applications dérivées à gauche $f'_g$ et à droite $f'_d$ sont croissantes avec $f'_g(x) \leq f'_d(x)$ pour tout $x \in \mathring{I}$.
	\end{proposition}
	
	\begin{proposition}
		On suppose $f$ deux fois dérivable. Alors, $f$ est convexe si et seulement si $f''(x) \geq 0$ pour tout $x \in I$.
	\end{proposition}
	
	\subsubsection{Fonction monotones}
	
	\reference[R-R]{31}
	
	\begin{definition}
		\begin{itemize}
			\item On dit que $f$ est \textbf{croissante} si $\forall x, y \in I, \, x \leq y \implies f(x) \leq f(y)$.
			\item On dit que $f$ est \textbf{décroissante} si $\forall x, y \in I, \, x \leq y \implies f(x) \geq f(y)$.
			\item On dit que $f$ est \textbf{monotone} si $f$ est croissante ou décroissante.
		\end{itemize}
	\end{definition}
	
	\reference[ROM19]{163}
	
	\begin{definition}
		Si $a \in \mathring{I}$, et si $f$ est discontinue en $a$ avec des limites à gauche et à droite en ce point, on dit que $f$ a une \textbf{discontinuité de première espèce} en $a$.
	\end{definition}
	
	\begin{proposition}
		Une fonction monotone de $I$ dans $\mathbb{R}$ ne peut avoir que des discontinuités de première espèce.
	\end{proposition}
	
	\begin{theorem}
		On suppose que $I$ est un intervalle ouvert. Si $f$ est une fonction monotone, alors l'ensemble des points de discontinuités de $f$ est dénombrable.
	\end{theorem}
	
	\begin{example}
		La fonction $f$ définie sur $[0,1]$ par $f(0) = 0$ et $f(x) = \frac{1}{\lfloor \frac{1}{x} \rfloor}$ est croissante avec une infinité de points de discontinuité.
	\end{example}
	
	\reference{175}
	
	\begin{proposition}
		Si $f$ est une fonction monotone telle que $f(I)$ est un intervalle, elle est alors continue sur $I$.
	\end{proposition}
	
	\begin{theorem}[Bijection]
		Si $f$ est une application continue et strictement monotone sur $I$, alors :
		\begin{enumerate}[(i)]
			\item $f(I)$ est un intervalle.
			\item $f^{-1}$ est continue.
			\item $f^{-1}$ est strictement monotone de même sens de variation que $f$.
		\end{enumerate}
	\end{theorem}
	
	\begin{example}
		La fonction $\exp : x \mapsto e^x$ est une bijection de $\mathbb{R}$ dans $\mathbb{R}^{+}_{*}$ qui admet donc une bijection réciproque $\ln$ qui est strictement croissante.
	\end{example}
	
	\reference[D-L]{405}
	
	\begin{theorem}[Lebesgue]
		Une application monotone est dérivable presque partout.
	\end{theorem}
	
	\subsection{Propriétés importantes}
	
	\subsubsection{Théorème des valeurs intermédiaires}
	
	\reference[GOU20]{41}
	
	\begin{theorem}[Des valeurs intermédiaires]
		On suppose $f$ continue sur $I$. Alors $f(I)$ est un intervalle.
	\end{theorem}
	
	\begin{remark}
		Une autre manière d'écrire ce résultat est la suivante. Si $f(a) \leq f(b)$ (resp. $f(a) \geq f(b)$) avec $a < b$, alors pour tout $f(a) \leq \gamma \leq f(b)$ (resp. $f(b) \leq \gamma \leq f(a)$), il existe $c \in [a,b]$ tel que $f(c) = \gamma$.
	\end{remark}
	
	\begin{corollary}
		L'image d'un segment de $\mathbb{R}$ par $f$ est un segment de $\mathbb{R}$.
	\end{corollary}
	
	\subsubsection{Théorème de Rolle}
	
	Dans cette partie, $I$ désigne un segment $[a,b]$ de $\mathbb{R}$ non réduit à un point.
	
	\begin{theorem}[Rolle]
		On suppose $f$ continue sur $[a,b]$, dérivable sur $]a,b[$ et telle que $f(a) = f(b)$. Alors,
		\[ \exists c \in ]a,b[ \text{ tel que } f'(c) = 0 \]
	\end{theorem}
	
	\begin{theorem}[Des accroissements finis]
		On suppose $f$ continue sur $[a,b]$ et dérivable sur $]a,b[$. Alors,
		\[ \exists c \in ]a,b[ \text{ tel que } f'(c) = \frac{f(b) - f(a)}{b-a} \]
	\end{theorem}
	
	\begin{corollary}
		On suppose $f$ continue sur $[a,b]$ et dérivable sur $]a,b[$. Alors, $f$ est croissante si et seulement si $f'(x) \geq 0$ pour tout $x \in ]a,b[$, avec égalité si et seulement si $f$ est constante.
	\end{corollary}
	
	\begin{corollary}
		On suppose $f$ continue sur $[a,b[$ et dérivable sur $]a,b[$ et telle que $\ell = \lim_{\substack{t \rightarrow a \\ t > a} f'(t)}$ existe. Alors, $f$ est dérivable en $a$ et $f'(a) = \ell$.
	\end{corollary}
	
	\reference{80}
	
	\begin{theorem}[Darboux]
		On suppose $f$ dérivable sur $I$. Alors $f'(I)$ est un intervalle.
	\end{theorem}
	
	\subsubsection{Formules de Taylor}
	
	\reference{75}
	
	Dans cette partie, $I$ désigne encore un segment $[a,b]$ de $\mathbb{R}$ non réduit à un point.
	
	\begin{theorem}[Formule de Taylor-Lagrange]
		On suppose $f$ de classe $\mathcal{C}^n$ sur $[a,b]$ telle que $f^{(n+1)}$ existe sur $]a,b$. Alors,
		\[ \exists c \in ]a,b[ \text{ tel que } f(b) =  \sum_{k=0}^{n} \frac{f^{(k)} (a)}{n!} (b-a)^n + \frac{f^{(n+1)}(c)}{(n+1)!} (b-a)^{n+1} \]
	\end{theorem}
	
	\begin{application}
		\begin{itemize}
			\item $\forall x \in \mathbb{R}^+, \, x - \frac{x^2}{2} \leq \ln(1+x) \leq x - \frac{x^2}{2} + \frac{x^3}{3}$.
			\item $\forall x \in \mathbb{R}^+, \, x - \frac{x^3}{6} \leq \sin(x) \leq x - \frac{x^3}{6} + \frac{x^5}{120}$.
			\item $\forall x \in \mathbb{R}, \, 1 - \frac{x^2}{2} \leq \cos(x) \leq 1 - \frac{x^2}{2} + \frac{x^4}{24}$.
		\end{itemize}
	\end{application}
	
	\subsubsection{Continuité sur un compact}
	
	\reference{31}
	
	\begin{proposition}
		Une fonction continue sur un compact est bornée et atteint ses bornes.
	\end{proposition}
	
	\begin{theorem}[Heine]
		Une fonction continue sur un compact y est uniformément continue.
	\end{theorem}
	
	\reference{242}
	
	\begin{theorem}[Bernstein]
		On suppose $I = [0,1]$ et $f$ continue sur $[0,1]$. On note
		\[ \forall n \in \mathbb{N}^*, \, \forall x \in [0,1], \, B_n(f)(x) = \sum_{k=0}^n f \left(\frac{k}{n}\right) b_n^k(x) \text{ avec } b_n^k(x) = \binom{n}{k} x^k (1-x)^{n-k} \]
	\end{theorem}
	
	\begin{theorem}[Weierstrass]
		Toute fonction continue sur un compact est limite uniforme de fonctions polynômiales.
	\end{theorem}
	
	\subsection{Régularité des fonctions limites}
	
	\subsubsection{Suites et séries de fonctions}
	
	\begin{proposition}
		Si une suite de fonctions est continue en un point $a$ et converge uniformément vers une fonction limite, alors celle-ci est continue en $a$.
	\end{proposition}
	
	\begin{cexample}
		La suite de fonctions $(g_n)$ définie pour tout $n \in \mathbb{N}$ et pour tout $x \in [0,1]$ par $g_n(x) = x^n$ converge vers une fonction non continue.
	\end{cexample}
	
	\begin{proposition}
		On suppose que $I$ est un segment $[a,b]$ de $\mathbb{R}$ non réduit à un point. Soit $(f_n)$ une suite de fonctions de $I$ dans $\mathbb{R}$. On suppose que :
		\begin{enumerate}[(i)]
			\item Il existe $x_0 \in [a,b]$ tel que $(f_n(x_0))$ converge.
			\item La suite $(f'_n)$ converge uniformément sur $[a,b]$ vers une fonction $g$.
		\end{enumerate}
		Alors $(f_n)$ converge uniformément vers une fonction $f$ de classe $\mathcal{C}^1$ sur $[a,b]$ et telle que $f'=g$.
	\end{proposition}
	
	\reference{302}
	
	\begin{example}
		La fonction $\zeta : s \mapsto \sum_{n=1}^{+\infty} \frac{1}{n^s}$ est $\mathcal{C}^\infty$ sur $]1,+\infty[$.
	\end{example}
	
	\subsubsection{Fonctions définies par une intégrale}
	
	\reference[Z-Q]{306}
	
	Soient $(X, \mathcal{A}, \mu)$ un espace mesuré et $g : E \times X \rightarrow \mathbb{C}$ où $(E, d)$ est un espace métrique. On pose $G : t \mapsto \int_X g(t, x) \, \mathrm{d}\mu(x)$.
	
	\begin{theorem}[Continuité sous le signe intégral]
		On suppose :
		\begin{enumerate}[(i)]
			\item $\forall t \in E$, $x \mapsto g(t,x)$ est mesurable.
			\item pp. en $x \in X$, $t \mapsto g(t,x)$ est continue en $t_0 \in E$.
			\item $\exists h \in L_1(X)$ positive telle que
			\[ |g(t,x)| \leq h(x) \quad \forall t \in E, \text{pp. en } x \in X \]
		\end{enumerate}
		Alors $G$ est continue en $t_0$.
	\end{theorem}
	
	\begin{theorem}[Dérivation sous le signe intégral]
		On suppose :
		\begin{enumerate}[(i)]
			\item $\forall t \in I$, $x \mapsto g(t,x) \in L_1(X)$.
			\item pp. en $x \in X$, $t \mapsto g(t,x)$ est dérivable sur $I$. On notera $\frac{\partial g}{\partial t}$ cette dérivée définie presque partout.
			\item $\forall K \subseteq I$ compact, $\exists h_K \in L_1(X)$ positive telle que
			\[ \left| \frac{\partial g}{\partial t}(x,t) \right| \leq h_K(x) \quad \forall t \in I, \text{pp. en } x \]
		\end{enumerate}
		Alors $\forall t \in I$, $x \mapsto \frac{\partial g}{\partial t}(x, t) \in L_1(X)$ et $G$ est dérivable sur $I$ avec
		\[ \forall t \in I, \, G'(t) = \int_X \frac{\partial g}{\partial t}(x, t) \, \mathrm{d}\mu(x) \]
	\end{theorem}
	
	\reference[G-K]{107}
	
	\begin{application}[Intégrale de Dirichlet]
		On pose $\forall x \geq 0$,
		\[ F(x) = \int_0^{+\infty} \frac{\sin(t)}{t} e^{-xt} \, \mathrm{d}t \]
		alors :
		\begin{enumerate}[(i)]
			\item $F$ est bien définie et est continue sur $\mathbb{R}^+$.
			\item $F$ est dérivable sur $\mathbb{R}^+_*$ et $\forall x \in \mathbb{R}^+_*$, $F'(x) = -\frac{1}{1+x^2}$.
			\item $F(0) = \int_0^{+\infty} \frac{\sin(t)}{t} \, \mathrm{d}t = \frac{\pi}{2}$.
		\end{enumerate}
	\end{application}
	
	\subsection*{Annexes}
	
	\reference[GOU20]{73}
	
	\begin{figure}[h]
		\begin{center}
			\begin{whitetabularx}{|X|X|}
				\hline
				\textbf{Valeur de $f(x)$} & \textbf{Valeur de $f'(x)$} \\
				\hline
				$x^r$ ($r \in \mathbb{R}$) & $rx^{r-1}$ \\
				\hline
				$\ln(x)$ & $\frac{1}{x}$ \\
				\hline
				$\sin(x)$ & $\cos(x)$ \\
				\hline
				$\cos(x)$ & $-\sin(x)$ \\
				\hline
				$\tan(x)$ & $\frac{1}{\cos(x)^2}$ \\
				\hline
				$e^x$ & $e^x$ \\
				\hline
				$\arctan(x)$ & $\frac{1}{1+x^2}$ \\
				\hline
				$\arcsin(x)$ & $\frac{1}{\sqrt{1-x^2}}$ \\
				\hline
				$\arccos(x)$ & $-\frac{1}{\sqrt{1-x^2}}$ \\
				\hline
			\end{whitetabularx}
		\end{center}
		\caption{Dérivées de fonctions usuelles}
	\end{figure}
	%</content>
\end{document}
