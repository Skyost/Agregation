\documentclass[12pt, a4paper]{report}

% LuaLaTeX :

\RequirePackage{iftex}
\RequireLuaTeX

% Packages :

\usepackage[french]{babel}
%\usepackage[utf8]{inputenc}
%\usepackage[T1]{fontenc}
\usepackage[pdfencoding=auto, pdfauthor={Hugo Delaunay}, pdfsubject={Mathématiques}, pdfcreator={agreg.skyost.eu}]{hyperref}
\usepackage{amsmath}
\usepackage{amsthm}
%\usepackage{amssymb}
\usepackage{stmaryrd}
\usepackage{tikz}
\usepackage{tkz-euclide}
\usepackage{fourier-otf}
\usepackage{fontspec}
\usepackage{titlesec}
\usepackage{fancyhdr}
\usepackage{catchfilebetweentags}
\usepackage[french, capitalise, noabbrev]{cleveref}
\usepackage[fit, breakall]{truncate}
\usepackage[top=2.5cm, right=2cm, bottom=2.5cm, left=2cm]{geometry}
\usepackage{enumerate}
\usepackage{tocloft}
\usepackage{microtype}
%\usepackage{mdframed}
%\usepackage{thmtools}
\usepackage{xcolor}
\usepackage{tabularx}
\usepackage{aligned-overset}
\usepackage[subpreambles=true]{standalone}
\usepackage{environ}
\usepackage[normalem]{ulem}
\usepackage{marginnote}
\usepackage{etoolbox}
\usepackage{setspace}
\usepackage[bibstyle=reading, citestyle=draft]{biblatex}
\usepackage{xpatch}
\usepackage[many, breakable]{tcolorbox}
\usepackage[backgroundcolor=white, bordercolor=white, textsize=small]{todonotes}

% Bibliographie :

\newcommand{\overridebibliographypath}[1]{\providecommand{\bibliographypath}{#1}}
\overridebibliographypath{../bibliography.bib}
\addbibresource{\bibliographypath}
\defbibheading{bibliography}[\bibname]{%
	\newpage
	\section*{#1}%
}
\renewbibmacro*{entryhead:full}{\printfield{labeltitle}}
\DeclareFieldFormat{url}{\newline\footnotesize\url{#1}}
\AtEndDocument{\printbibliography}

% Police :

\setmathfont{Erewhon Math}

% Tikz :

\usetikzlibrary{calc}

% Longueurs :

\setlength{\parindent}{0pt}
\setlength{\headheight}{15pt}
\setlength{\fboxsep}{0pt}
\titlespacing*{\chapter}{0pt}{-20pt}{10pt}
\setlength{\marginparwidth}{1.5cm}
\setstretch{1.1}

% Métadonnées :

\author{agreg.skyost.eu}
\date{\today}

% Titres :

\setcounter{secnumdepth}{3}

\renewcommand{\thechapter}{\Roman{chapter}}
\renewcommand{\thesubsection}{\Roman{subsection}}
\renewcommand{\thesubsubsection}{\arabic{subsubsection}}
\renewcommand{\theparagraph}{\alph{paragraph}}

\titleformat{\chapter}{\huge\bfseries}{\thechapter}{20pt}{\huge\bfseries}
\titleformat*{\section}{\LARGE\bfseries}
\titleformat{\subsection}{\Large\bfseries}{\thesubsection \, - \,}{0pt}{\Large\bfseries}
\titleformat{\subsubsection}{\large\bfseries}{\thesubsubsection. \,}{0pt}{\large\bfseries}
\titleformat{\paragraph}{\bfseries}{\theparagraph. \,}{0pt}{\bfseries}

\setcounter{secnumdepth}{4}

% Table des matières :

\renewcommand{\cftsecleader}{\cftdotfill{\cftdotsep}}
\addtolength{\cftsecnumwidth}{10pt}

% Redéfinition des commandes :

\renewcommand*\thesection{\arabic{section}}
\renewcommand{\ker}{\mathrm{Ker}}

% Nouvelles commandes :

\newcommand{\website}{https://agreg.skyost.eu}

\newcommand{\tr}[1]{\mathstrut ^t #1}
\newcommand{\im}{\mathrm{Im}}
\newcommand{\rang}{\operatorname{rang}}
\newcommand{\trace}{\operatorname{trace}}
\newcommand{\id}{\operatorname{id}}
\newcommand{\stab}{\operatorname{Stab}}

\providecommand{\newpar}{\\[\medskipamount]}

\providecommand{\lesson}[3]{%
	\title{#3}%
	\hypersetup{pdftitle={#3}}%
	\setcounter{section}{\numexpr #2 - 1}%
	\section{#3}%
	\fancyhead[R]{\truncate{0.73\textwidth}{#2 : #3}}%
}

\providecommand{\development}[3]{%
	\title{#3}%
	\hypersetup{pdftitle={#3}}%
	\section*{#3}%
	\fancyhead[R]{\truncate{0.73\textwidth}{#3}}%
}

\providecommand{\summary}[1]{%
	\textit{#1}%
	\medskip%
}

\tikzset{notestyleraw/.append style={inner sep=0pt, rounded corners=0pt, align=center}}

%\newcommand{\booklink}[1]{\website/bibliographie\##1}
\newcommand{\citelink}[2]{\hyperlink{cite.\therefsection @#1}{#2}}
\newcommand{\previousreference}{}
\providecommand{\reference}[2][]{%
	\notblank{#1}{\renewcommand{\previousreference}{#1}}{}%
	\todo[noline]{%
		\protect\vspace{16pt}%
		\protect\par%
		\protect\notblank{#1}{\cite{[\previousreference]}\\}{}%
		\protect\citelink{\previousreference}{p. #2}%
	}%
}

\definecolor{devcolor}{HTML}{00695c}
\newcommand{\dev}[1]{%
	\reversemarginpar%
	\todo[noline]{
		\protect\vspace{16pt}%
		\protect\par%
		\bfseries\color{devcolor}\href{\website/developpements/#1}{DEV}
	}%
	\normalmarginpar%
}

% En-têtes :

\pagestyle{fancy}
\fancyhead[L]{\truncate{0.23\textwidth}{\thepage}}
\fancyfoot[C]{\scriptsize \href{\website}{\texttt{agreg.skyost.eu}}}

% Couleurs :

\definecolor{property}{HTML}{fffde7}
\definecolor{proposition}{HTML}{fff8e1}
\definecolor{lemma}{HTML}{fff3e0}
\definecolor{theorem}{HTML}{fce4f2}
\definecolor{corollary}{HTML}{ffebee}
\definecolor{definition}{HTML}{ede7f6}
\definecolor{notation}{HTML}{f3e5f5}
\definecolor{example}{HTML}{e0f7fa}
\definecolor{cexample}{HTML}{efebe9}
\definecolor{application}{HTML}{e0f2f1}
\definecolor{remark}{HTML}{e8f5e9}
\definecolor{proof}{HTML}{e1f5fe}

% Théorèmes :

\theoremstyle{definition}
\newtheorem{theorem}{Théorème}

\newtheorem{property}[theorem]{Propriété}
\newtheorem{proposition}[theorem]{Proposition}
\newtheorem{lemma}[theorem]{Lemme}
\newtheorem{corollary}[theorem]{Corollaire}

\newtheorem{definition}[theorem]{Définition}
\newtheorem{notation}[theorem]{Notation}

\newtheorem{example}[theorem]{Exemple}
\newtheorem{cexample}[theorem]{Contre-exemple}
\newtheorem{application}[theorem]{Application}

\theoremstyle{remark}
\newtheorem{remark}[theorem]{Remarque}

\counterwithin*{theorem}{section}

\newcommand{\applystyletotheorem}[1]{
	\tcolorboxenvironment{#1}{
		enhanced,
		breakable,
		colback=#1!98!white,
		boxrule=0pt,
		boxsep=0pt,
		left=8pt,
		right=8pt,
		top=8pt,
		bottom=8pt,
		sharp corners,
		after=\par,
	}
}

\applystyletotheorem{property}
\applystyletotheorem{proposition}
\applystyletotheorem{lemma}
\applystyletotheorem{theorem}
\applystyletotheorem{corollary}
\applystyletotheorem{definition}
\applystyletotheorem{notation}
\applystyletotheorem{example}
\applystyletotheorem{cexample}
\applystyletotheorem{application}
\applystyletotheorem{remark}
\applystyletotheorem{proof}

% Environnements :

\NewEnviron{whitetabularx}[1]{%
	\renewcommand{\arraystretch}{2.5}
	\colorbox{white}{%
		\begin{tabularx}{\textwidth}{#1}%
			\BODY%
		\end{tabularx}%
	}%
}

% Maths :

\DeclareFontEncoding{FMS}{}{}
\DeclareFontSubstitution{FMS}{futm}{m}{n}
\DeclareFontEncoding{FMX}{}{}
\DeclareFontSubstitution{FMX}{futm}{m}{n}
\DeclareSymbolFont{fouriersymbols}{FMS}{futm}{m}{n}
\DeclareSymbolFont{fourierlargesymbols}{FMX}{futm}{m}{n}
\DeclareMathDelimiter{\VERT}{\mathord}{fouriersymbols}{152}{fourierlargesymbols}{147}


% Bibliographie :

\addbibresource{\bibliographypath}%
\defbibheading{bibliography}[\bibname]{%
	\newpage
	\section*{#1}%
}
\renewbibmacro*{entryhead:full}{\printfield{labeltitle}}%
\DeclareFieldFormat{url}{\newline\footnotesize\url{#1}}%

\AtEndDocument{\printbibliography}

\begin{document}
  %<*content>
  \lesson{analysis}{203}{Utilisation de la notion de compacité.}

  \subsection{Diverses caractérisations de la compacité}

  \subsubsection{Caractérisation topologique}

  \reference[GOU20]{27}

  \begin{definition}
    Un espace métrique $(E, d)$ est \textbf{compact} s'il vérifie la propriété de Borel-Lebesgue :
    \begin{center}
      \textit{De toute recouvrement de $E$ par des ouverts de $E$, on peut en extraire un sous-recouvrement fini.}
    \end{center}
  \end{definition}

  \begin{example}
    Tout espace métrique fini est compact.
  \end{example}

  \begin{proposition}
    Un espace métrique $(E, d)$ est compact si de toute famille de fermés de $E$ d'intersection vide, on peut extraire une sous-famille d'intersection vide.
  \end{proposition}

  \begin{proposition}
    \begin{enumerate}[label=(\roman*)]
      \item Une réunion finie de parties compactes est compacte.
      \item Une intersection quelconque de parties compactes est compacte.
    \end{enumerate}
  \end{proposition}

  \subsubsection{Caractérisation séquentielle}

  \reference[DAN]{51}

  Soit $(E,d)$ un espace métrique.

  \begin{theorem}[Bolzano-Weierstrass]
    $(E,d)$ est compact si toute suite de $E$ admet une sous-suite convergente dans $E$.
  \end{theorem}

  \begin{example}
    Tout segment $[a,b]$ de $\mathbb{R}$ est compact, mais $\mathbb{R}$ n'est pas compact.
  \end{example}

  \begin{proposition}
    \begin{enumerate}[label=(\roman*)]
      \item Un espace métrique compact est complet.
      \item Un espace métrique compact est borné.
    \end{enumerate}
  \end{proposition}

  \begin{proposition}
    Soit $A \subseteq E$.
    \begin{enumerate}[label=(\roman*)]
      \item Si $A$ est compacte, alors $A$ est une partie fermée bornée de $E$.
      \item Si $E$ est compact et $A$ est fermée, alors $A$ est compacte.
    \end{enumerate}
  \end{proposition}

  \begin{proposition}
    Un produit d'espaces métriques compacts est compact pour la distance produit.
  \end{proposition}

  \reference[I-P]{116}

  \begin{application}
    Soit $(E, d)$ un espace métrique compact. Soit $(u_n)$ une suite de $E$ telle que $d(u_n,u_{n-1}) \longrightarrow 0$. Alors l'ensemble $\Gamma$ des valeurs d'adhérence de $(u_n)$ est connexe.
  \end{application}

  \begin{corollary}[Lemme de la grenouille]
    Soient $f : [0, 1] \rightarrow [0, 1]$ continue et $(x_n)$ une suite de $[0, 1]$ telle que
    \[ \begin{cases} x_0 \in [0, 1] \\ x_{n+1} = f(x_n) \end{cases} \]
    Alors $(x_n)$ converge si et seulement si $\lim_{n \rightarrow +\infty } x_{n+1} - x_n = 0$.
  \end{corollary}

  \subsubsection{Caractérisation dans un espace vectoriel normé de dimension finie}

  \reference[LI]{15}
  \dev{equivalence-des-normes-en-dimension-finie-et-theoreme-de-riesz}

  \begin{theorem}
    En dimension finie, toutes les normes sont équivalentes.
  \end{theorem}

  \begin{corollary}
    Les parties compactes d'un espace vectoriel normé de dimension finie sont les parties fermées bornées.
  \end{corollary}

  \begin{corollary}
    \begin{enumerate}[label=(\roman*)]
      \item Tout espace vectoriel de dimension finie est complet.
      \item Tout espace vectoriel de dimension finie dans un espace vectoriel normé est fermé dans cet espace.
      \item Si $E$ est un espace vectoriel normé, alors toute application linéaire $T : E \rightarrow F$ (où $F$ désigne un espace vectoriel normé arbitraire) est continue.
    \end{enumerate}
  \end{corollary}

  \reference[C-G]{407}

  \begin{application}
    L'exponentielle d'une matrice est un polynôme en la matrice.
  \end{application}

  \reference[LI]{17}

  \begin{theorem}[Riesz]
    La boule unité fermée d'un espace vectoriel normé est compacte si et seulement s'il est dimension finie.
  \end{theorem}

  \subsection{Utilisation en analyse}

  \subsubsection{Continuité et compacité}

  \reference[DAN]{55}

  \begin{proposition}
    \label{203-1}
    Soient $(E, d_E)$, $(F, d_F)$ deux espaces métriques et $f : E \rightarrow F$ une application continue. Si $E$ est compact, alors $f(E)$ est compact.
  \end{proposition}

  \begin{corollary}
    Toute application définie et continue sur un espace métrique compact à valeurs dans un espace métrique est bornée.
  \end{corollary}

  \begin{proposition}
    Sous les hypothèses et notations de la \cref{203-1}, en supposant de plus $f$ injective, alors $f$ réalise un homéomorphisme entre $E$ et $f(E)$.
  \end{proposition}

  \begin{theorem}[des bornes]
    Toute fonction réelle continue sur un espace métrique compact est bornée et atteint ses bornes.
  \end{theorem}

  \begin{corollary}[Théorème des valeurs intermédiaires]
    L'image d'un segment $[a,b]$ de $\mathbb{R}$ par une fonction réelle continue est un segment $[c,d]$ de $\mathbb{R}$.
  \end{corollary}

  \reference[GOU20]{73}

  \begin{application}[Théorème de Rolle]
    Soit $f$ une fonction réelle continue sur un intervalle $[a,b]$, dérivable sur $]a,b[$ et telle que $f(a) = f(b)$. Alors,
    \[ \exists c \in ]a,b[ \text{ tel que } f'(c) = 0 \]
  \end{application}

  \reference[ROU]{171}

  \begin{application}[Point fixe dans un compact]
    Soit $(E,d)$ un espace métrique compact et $f : E \rightarrow E$ telle que
    \[ \forall x, y \in E, \, x \neq y \implies d(f(x), f(y)) < d(x,y) \]
    alors $f$ admet un unique point fixe et pour tout $x_0 \in E$, la suite des itérés
    \[ x_{n+1} = f(x_n) \]
    converge vers ce point fixe.
  \end{application}

  \begin{example}
    $\sin$ admet un unique point fixe sur $[0,1]$.
  \end{example}

  \reference[DAN]{58}

  \begin{application}[Théorème de d'Alembert-Gauss]
    Tout polynôme non constant de $\mathbb{C}$ admet une racine dans $\mathbb{C}$.
  \end{application}

  \begin{theorem}[Heine]
    Une application continue à valeurs dans un espace métrique définie sur un espace métrique compact est uniformément continue.
  \end{theorem}

  \reference[GOU20]{238}

  \begin{theorem}[Théorèmes de Dini]
    \begin{enumerate}[label=(\roman*)]
      \item Soit $(f_n)$ une suite \textit{croissante} de fonctions réelles \textit{continues} définies sur un segment $I$ de $\mathbb{R}$. Si $(f_n)$ converge simplement vers une fonction \textit{continue} sur $I$, alors la convergence est uniforme.
      \item Soit $(f_n)$ une suite de \textit{fonctions croissantes} réelles \textit{continues} définies sur un segment $I$ de $\mathbb{R}$. Si $(f_n)$ converge simplement vers une fonction \textit{continue} sur $I$, alors la convergence est uniforme.
    \end{enumerate}
  \end{theorem}

  \subsubsection{Approximation de fonctions}

  \reference{304}
  \dev{theoreme-de-weierstrass-par-la-convolution}

  \begin{theorem}[Weierstrass]
    Toute fonction continue $f : [a,b] \rightarrow \mathbb{R}$ (avec $a, b \in \mathbb{R}$ tels que $a \leq b$) est limite uniforme de fonctions polynômiales sur $[a, b]$.
  \end{theorem}

  On a une version plus générale de ce théorème.

  \reference[LI]{46}

  \begin{theorem}[Stone-Weierstrass]
    Soit $K$ un espace compact et $\mathcal{A}$ une sous-algèbre de l'algèbre de Banach réelle $\mathcal{C}(K, \mathbb{R})$. On suppose de plus que :
    \begin{enumerate}[label=(\roman*)]
      \item $\mathcal{A}$ sépare les points de $K$ (ie. $\forall x \in K, \exists f \in A \text{ telle que } f(x) \neq f(y)$).
      \item $\mathcal{A}$ contient les constantes.
    \end{enumerate}
    Alors $\mathcal{A}$ est dense dans $\mathcal{C}(K, \mathbb{R})$.
  \end{theorem}

  \begin{remark}
    Il existe aussi une version ``complexe'' de ce théorème, où il faut supposer de plus que $\mathcal{A}$ est stable par conjugaison.
  \end{remark}

  \begin{example}
    La suite de polynômes réels $(r_n)$ définie par récurrence par
    \[ r_0 = 0 \text{ et } \forall n \in \mathbb{N}, r_{n+1} : t \mapsto r_n(t) + \frac{1}{2} (t - r_n(t)^2) \]
    converge vers $\sqrt{.}$ sur $[0,1]$.
  \end{example}

  \subsubsection{Étude d'équations différentielles}

  \reference[GOU20]{375}

  \begin{theorem}[Arzelà-Peano]
    Soit $F$ une fonction continue sur un ouvert $U$ de $\mathbb{R} \times \mathbb{R}^n$ à valeurs dans $\mathbb{R}^n$. On considère l'équation différentielle
    \[ y' = F(t,y) \]
    Pour tout couple $(y_0, t_0)$ de $U$, le problème de Cauchy admet une solution $y$ définie sur un intervalle ouvert contenant $t_0$.
  \end{theorem}

  \begin{example}
    L'équation différentielle
    \[
    y' =
    \begin{cases}
      0 &\text{ si } y < 0 \\
      \sqrt{y} &\text{ si } y \geq 0
    \end{cases}
    \]
    admet des solutions.
  \end{example}

  \reference{400}

  \begin{theorem}[Lemme de sortie de tout compact]
    Soient $]a,b[$ un intervalle ouvert de $\mathbb{R}$, $O$ un ouvert de $\mathbb{R}^n$ et $F : ]a,b[ \times O \rightarrow \mathbb{R}^n$ une fonction continue et localement lipschitzienne en la seconde variable. Soit $\varphi : ]\alpha,\beta[ \rightarrow \mathbb{R}^n$ une solution maximale de $y' = F(t,y)$.
    \newpar
    Alors, si $\beta < b$ (resp. si $a < \alpha$), pour tout compact $K \subseteq O$, il existe un voisinage $V$ de $\beta$ (resp. de $\alpha$) dans $]\alpha,\beta[$ tel que $\varphi(t) \notin K$ pour tout $t \in V$.
  \end{theorem}

  \subsubsection{Recherche d'extrema}

  \reference[ROU]{409}

  \begin{proposition}
    Le maximum de
    \[
    f :
    \begin{array}{ccc}
      \mathbb{R}^n \times \dots \times \mathbb{R}^n &\rightarrow& \mathbb{R} \\
      (v_1, \dots, v_n) &\mapsto& \det(v_1, \dots, v_n)
    \end{array}
    \]
    est atteint sur le cercle unité de $\mathbb{R}^n$.
  \end{proposition}

  \begin{corollary}[Inégalité de Hadamard]
    \[ \forall v_1, \dots, v_n \in \mathbb{R}^n, \, \vert \det(v_1, \dots, v_n) \vert \leq \Vert v_1 \Vert \dots \Vert v_n \Vert \]
    où $\Vert . \Vert$ désigne la norme associée au produit scalaire usuel sur $\mathbb{R}^n$. On a égalité si et seulement si un des $v_i$ est nul.
  \end{corollary}

  \begin{remark}
    Géométriquement, cette inégalité exprime que les parallélépipèdes de volume maximum sont rectangles.
  \end{remark}

  \subsubsection{Convexité et compacité}

  \reference[LI]{159}

  \begin{theorem}[Hahn-Banach géométrique]
    Soit $E$ un espace vectoriel normé. Soient $C$ et $K$ deux parties non vides de $E$ disjointes et telles que $C$ soit convexe et fermée, et $K$ soit convexe et compacte.
    Alors, il existe une forme linéaire continue $\varphi in E'$ telle que :
    \[ \sup_{x \in C} \operatorname{Re}(\varphi(x)) < \inf_{x \in K} \operatorname{Re}(\varphi(x)) \]
  \end{theorem}

  \begin{corollary}[Théorème de Minkowski]
    Toute partie convexe et fermée d'un espace vectoriel normé réel est égale à
    l'intersection des demi-espaces fermés qui le contiennent.
  \end{corollary}

  \reference[BMP]{133}

  \begin{corollary}
    Soit $H$ un espace de Hilbert sur $\mathbb{R}$ et soit $D$ une partie de $H$. Alors l'enveloppe convexe fermée de $D$ est égale à l'intersection des demi-espaces de la forme
    \[ \{ y \in H \mid f(y) \leq \alpha \} \]
    qui contiennent $D$, où $f \in H'$ et $\alpha \in \mathbb{R}$.
  \end{corollary}

  \subsection{Utilisation en algèbre}

  \reference[C-G]{62}

  \begin{proposition}
    \begin{enumerate}[label=(\roman*)]
      \item $\mathrm{SO}_n(\mathbb{R})$ est compact (et connexe).
      \item $\mathcal{O}_n(\mathbb{R})$ est compact (non-connexe).
    \end{enumerate}
  \end{proposition}

  \reference{376}

  \begin{application}[Décomposition polaire]
    L'application
    \[ \mu :
    \begin{array}{ccc}
      \mathcal{O}_n(\mathbb{R}) \times \mathcal{S}_n^{++}(\mathbb{R}) &\rightarrow& \mathrm{GL}_n(\mathbb{R}) \\
      (O, S) &\mapsto& OS
    \end{array}
    \]
    est un homéomorphisme.
  \end{application}

  \begin{corollary}
    Tout sous-groupe compact de $\mathrm{GL}_n(\mathbb{R})$ qui contient $\mathcal{O}_n(\mathbb{R})$ est $\mathcal{O}_n(\mathbb{R})$.
  \end{corollary}

  \reference{401}

  \begin{corollary}
    $\mathrm{GL}_n(\mathbb{R})^+$ est connexe.
  \end{corollary}
  %</content>
\end{document}
