\documentclass[12pt, a4paper]{report}

% LuaLaTeX :

\RequirePackage{iftex}
\RequireLuaTeX

% Packages :

\usepackage[french]{babel}
%\usepackage[utf8]{inputenc}
%\usepackage[T1]{fontenc}
\usepackage[pdfencoding=auto, pdfauthor={Hugo Delaunay}, pdfsubject={Mathématiques}, pdfcreator={agreg.skyost.eu}]{hyperref}
\usepackage{amsmath}
\usepackage{amsthm}
%\usepackage{amssymb}
\usepackage{stmaryrd}
\usepackage{tikz}
\usepackage{tkz-euclide}
\usepackage{fourier-otf}
\usepackage{fontspec}
\usepackage{titlesec}
\usepackage{fancyhdr}
\usepackage{catchfilebetweentags}
\usepackage[french, capitalise, noabbrev]{cleveref}
\usepackage[fit, breakall]{truncate}
\usepackage[top=2.5cm, right=2cm, bottom=2.5cm, left=2cm]{geometry}
\usepackage{enumerate}
\usepackage{tocloft}
\usepackage{microtype}
%\usepackage{mdframed}
%\usepackage{thmtools}
\usepackage{xcolor}
\usepackage{tabularx}
\usepackage{aligned-overset}
\usepackage[subpreambles=true]{standalone}
\usepackage{environ}
\usepackage[normalem]{ulem}
\usepackage{marginnote}
\usepackage{etoolbox}
\usepackage{setspace}
\usepackage[bibstyle=reading, citestyle=draft]{biblatex}
\usepackage{xpatch}
\usepackage[many, breakable]{tcolorbox}
\usepackage[backgroundcolor=white, bordercolor=white, textsize=small]{todonotes}

% Bibliographie :

\newcommand{\overridebibliographypath}[1]{\providecommand{\bibliographypath}{#1}}
\overridebibliographypath{../bibliography.bib}
\addbibresource{\bibliographypath}
\defbibheading{bibliography}[\bibname]{%
	\newpage
	\section*{#1}%
}
\renewbibmacro*{entryhead:full}{\printfield{labeltitle}}
\DeclareFieldFormat{url}{\newline\footnotesize\url{#1}}
\AtEndDocument{\printbibliography}

% Police :

\setmathfont{Erewhon Math}

% Tikz :

\usetikzlibrary{calc}

% Longueurs :

\setlength{\parindent}{0pt}
\setlength{\headheight}{15pt}
\setlength{\fboxsep}{0pt}
\titlespacing*{\chapter}{0pt}{-20pt}{10pt}
\setlength{\marginparwidth}{1.5cm}
\setstretch{1.1}

% Métadonnées :

\author{agreg.skyost.eu}
\date{\today}

% Titres :

\setcounter{secnumdepth}{3}

\renewcommand{\thechapter}{\Roman{chapter}}
\renewcommand{\thesubsection}{\Roman{subsection}}
\renewcommand{\thesubsubsection}{\arabic{subsubsection}}
\renewcommand{\theparagraph}{\alph{paragraph}}

\titleformat{\chapter}{\huge\bfseries}{\thechapter}{20pt}{\huge\bfseries}
\titleformat*{\section}{\LARGE\bfseries}
\titleformat{\subsection}{\Large\bfseries}{\thesubsection \, - \,}{0pt}{\Large\bfseries}
\titleformat{\subsubsection}{\large\bfseries}{\thesubsubsection. \,}{0pt}{\large\bfseries}
\titleformat{\paragraph}{\bfseries}{\theparagraph. \,}{0pt}{\bfseries}

\setcounter{secnumdepth}{4}

% Table des matières :

\renewcommand{\cftsecleader}{\cftdotfill{\cftdotsep}}
\addtolength{\cftsecnumwidth}{10pt}

% Redéfinition des commandes :

\renewcommand*\thesection{\arabic{section}}
\renewcommand{\ker}{\mathrm{Ker}}

% Nouvelles commandes :

\newcommand{\website}{https://agreg.skyost.eu}

\newcommand{\tr}[1]{\mathstrut ^t #1}
\newcommand{\im}{\mathrm{Im}}
\newcommand{\rang}{\operatorname{rang}}
\newcommand{\trace}{\operatorname{trace}}
\newcommand{\id}{\operatorname{id}}
\newcommand{\stab}{\operatorname{Stab}}

\providecommand{\newpar}{\\[\medskipamount]}

\providecommand{\lesson}[3]{%
	\title{#3}%
	\hypersetup{pdftitle={#3}}%
	\setcounter{section}{\numexpr #2 - 1}%
	\section{#3}%
	\fancyhead[R]{\truncate{0.73\textwidth}{#2 : #3}}%
}

\providecommand{\development}[3]{%
	\title{#3}%
	\hypersetup{pdftitle={#3}}%
	\section*{#3}%
	\fancyhead[R]{\truncate{0.73\textwidth}{#3}}%
}

\providecommand{\summary}[1]{%
	\textit{#1}%
	\medskip%
}

\tikzset{notestyleraw/.append style={inner sep=0pt, rounded corners=0pt, align=center}}

%\newcommand{\booklink}[1]{\website/bibliographie\##1}
\newcommand{\citelink}[2]{\hyperlink{cite.\therefsection @#1}{#2}}
\newcommand{\previousreference}{}
\providecommand{\reference}[2][]{%
	\notblank{#1}{\renewcommand{\previousreference}{#1}}{}%
	\todo[noline]{%
		\protect\vspace{16pt}%
		\protect\par%
		\protect\notblank{#1}{\cite{[\previousreference]}\\}{}%
		\protect\citelink{\previousreference}{p. #2}%
	}%
}

\definecolor{devcolor}{HTML}{00695c}
\newcommand{\dev}[1]{%
	\reversemarginpar%
	\todo[noline]{
		\protect\vspace{16pt}%
		\protect\par%
		\bfseries\color{devcolor}\href{\website/developpements/#1}{DEV}
	}%
	\normalmarginpar%
}

% En-têtes :

\pagestyle{fancy}
\fancyhead[L]{\truncate{0.23\textwidth}{\thepage}}
\fancyfoot[C]{\scriptsize \href{\website}{\texttt{agreg.skyost.eu}}}

% Couleurs :

\definecolor{property}{HTML}{fffde7}
\definecolor{proposition}{HTML}{fff8e1}
\definecolor{lemma}{HTML}{fff3e0}
\definecolor{theorem}{HTML}{fce4f2}
\definecolor{corollary}{HTML}{ffebee}
\definecolor{definition}{HTML}{ede7f6}
\definecolor{notation}{HTML}{f3e5f5}
\definecolor{example}{HTML}{e0f7fa}
\definecolor{cexample}{HTML}{efebe9}
\definecolor{application}{HTML}{e0f2f1}
\definecolor{remark}{HTML}{e8f5e9}
\definecolor{proof}{HTML}{e1f5fe}

% Théorèmes :

\theoremstyle{definition}
\newtheorem{theorem}{Théorème}

\newtheorem{property}[theorem]{Propriété}
\newtheorem{proposition}[theorem]{Proposition}
\newtheorem{lemma}[theorem]{Lemme}
\newtheorem{corollary}[theorem]{Corollaire}

\newtheorem{definition}[theorem]{Définition}
\newtheorem{notation}[theorem]{Notation}

\newtheorem{example}[theorem]{Exemple}
\newtheorem{cexample}[theorem]{Contre-exemple}
\newtheorem{application}[theorem]{Application}

\theoremstyle{remark}
\newtheorem{remark}[theorem]{Remarque}

\counterwithin*{theorem}{section}

\newcommand{\applystyletotheorem}[1]{
	\tcolorboxenvironment{#1}{
		enhanced,
		breakable,
		colback=#1!98!white,
		boxrule=0pt,
		boxsep=0pt,
		left=8pt,
		right=8pt,
		top=8pt,
		bottom=8pt,
		sharp corners,
		after=\par,
	}
}

\applystyletotheorem{property}
\applystyletotheorem{proposition}
\applystyletotheorem{lemma}
\applystyletotheorem{theorem}
\applystyletotheorem{corollary}
\applystyletotheorem{definition}
\applystyletotheorem{notation}
\applystyletotheorem{example}
\applystyletotheorem{cexample}
\applystyletotheorem{application}
\applystyletotheorem{remark}
\applystyletotheorem{proof}

% Environnements :

\NewEnviron{whitetabularx}[1]{%
	\renewcommand{\arraystretch}{2.5}
	\colorbox{white}{%
		\begin{tabularx}{\textwidth}{#1}%
			\BODY%
		\end{tabularx}%
	}%
}

% Maths :

\DeclareFontEncoding{FMS}{}{}
\DeclareFontSubstitution{FMS}{futm}{m}{n}
\DeclareFontEncoding{FMX}{}{}
\DeclareFontSubstitution{FMX}{futm}{m}{n}
\DeclareSymbolFont{fouriersymbols}{FMS}{futm}{m}{n}
\DeclareSymbolFont{fourierlargesymbols}{FMX}{futm}{m}{n}
\DeclareMathDelimiter{\VERT}{\mathord}{fouriersymbols}{152}{fourierlargesymbols}{147}


% Bibliographie :

\addbibresource{\bibliographypath}%
\defbibheading{bibliography}[\bibname]{%
	\newpage
	\section*{#1}%
}
\renewbibmacro*{entryhead:full}{\printfield{labeltitle}}%
\DeclareFieldFormat{url}{\newline\footnotesize\url{#1}}%

\AtEndDocument{\printbibliography}

\begin{document}
	%<*content>
	\lesson{analysis}{250}{Transformation de Fourier. Applications.}

	\subsection{Transformation de Fourier dans \texorpdfstring{$L_1(\mathbb{R}^d)$}{L₁(Rᵈ)}}

	\subsubsection{Définitions et premières propriétés}
	
	\reference[AMR08]{109}
	
	\begin{definition}
		Soit $f : \mathbb{R}^d \rightarrow \mathbb{C}$ une fonction mesurable. On définit, lorsque cela a un sens, sa \textbf{transformée de Fourier}, notée $\widehat{f}$ par
		\[
		\widehat{f} :
		\begin{array}{ccc}
			\mathbb{R}^d &\rightarrow& \mathbb{C} \\
			\xi &\mapsto& \int_{\mathbb{R}^d} f(x) e^{-i\langle x, \xi \rangle} \, \mathrm{d}x
		\end{array}
		\]
	\end{definition}
	
	\begin{example}[Densité de Poisson]
		On pose $\forall x \in \mathbb{R}$, $p(x) = \frac{1}{2} e^{-|x|}$. Alors $p \in L_1(\mathbb{R})$ et, $\forall \xi \in \mathbb{R}$, $\widehat{p}(\xi) = \frac{1}{1+\xi^2}$.
	\end{example}
	
	\reference{156}
	
	\begin{example}[Transformée de Fourier d'une gaussienne]
		\label{250-1}
		On définit $\forall a \in \mathbb{R}^+_*$,
		\[ \gamma_a :
		\begin{array}{ccc}
			\mathbb{R} &\rightarrow& \mathbb{R} \\
			x &\mapsto& e^{-ax^2}
		\end{array}
		\]
		Alors,
		\[ \forall \xi \in \mathbb{R}, \, \widehat{\gamma_a}(\xi) = \sqrt{\frac{\pi}{a}} e^{\frac{- \xi^2}{4a}} \]
	\end{example}
	
	\reference{109}
	
	\begin{lemma}[Riemann-Lebesgue]
		Soit $f \in L_1(\mathbb{R}^d)$, $\widehat{f}$ existe et
		\[ \lim_{\Vert \xi \Vert \rightarrow +\infty} \widehat{f}(\xi) \]
	\end{lemma}
	
	\begin{remark}
		La transformée de Fourier d'une fonction intégrable n'est pas forcément intégrable.
	\end{remark}
	
	\begin{theorem}
		$\forall f \in L_1(\mathbb{R}^d)$, $\widehat{f}$ est continue, bornée par $\Vert f \Vert_1$. Donc la \textbf{transformation de Fourier}
		\[
		\mathcal{F} :
		\begin{array}{ccc}
			L_1(\mathbb{R}^d) &\rightarrow& \mathcal{C}_0(\mathbb{R}^d) \\
			f &\mapsto& \widehat{f}
		\end{array}
		\]
		est bien définie.
	\end{theorem}
	
	\begin{corollary}
		La transformation de Fourier $\mathcal{F} : L_1(\mathbb{R}^d) \rightarrow \mathcal{C}_0(\mathbb{R}^d)$ est une application linéaire continue.
	\end{corollary}
	
	\begin{proposition}
		Soit $f \in L_1(\mathbb{R}^d)$. Alors :
		\begin{enumerate}[(i)]
			\item $(\mathcal{F}(x \mapsto f(-x))) = (\xi \mapsto \mathcal{F}(f)(-\xi))$.
			\item $(\mathcal{F}(\overline{f})) = (\xi \mapsto \overline{\mathcal{F}(f)(-\xi)})$.
			\item Pour tout $\lambda \in \mathbb{R}_*$, et $\xi \in \mathbb{R}^d$,
			\[ (\mathcal{F}(x \mapsto f(\lambda x))) = (\frac{1}{\vert \lambda \vert^d} \mathcal{F}(f) \left( \frac{\xi}{\lambda} \right)) \]
			\item Pour tout $a \in \mathbb{R}^d$,
			\[ (\mathcal{F}(x \mapsto f(x - a))) = (e^{-i \langle a, \xi \rangle} \mathcal{F}(f)) \text{ et } (\mathcal{F}(x \mapsto e^{-i \langle a, \xi \rangle} f(x))) = (\xi \mapsto \mathcal{F}(f)(\xi - a)) \]
		\end{enumerate}
	\end{proposition}
	
	\reference{120}
	
	\begin{proposition}
		Soit $f \in L_1(\mathbb{R}^d)$.
		\begin{enumerate}[(i)]
			\item On suppose $f \in \mathcal{C}^1(\mathbb{R}^d)$ et $\frac{\partial f}{\partial x_j} \in L_1(\mathbb{R}^d)$. Alors,
			\[ \forall \xi = (\xi_1, \dots, \xi_d) \in \mathbb{R}^d, \, \widehat{\frac{\partial f}{\partial x_j}}(\xi) = i \xi_j \xi \widehat{f}(\xi) \]
			\item On suppose $y_j f \in L_1(\mathbb{R}^d)$. Alors, la $j$-ième dérivée partielle de $\widehat{f}$ existe, et,
			\[ \forall \xi \in \mathbb{R}^d, \, \frac{\partial \widehat{f}}{\partial x_j}(\xi) = -i \widehat{(y_j f)}(\xi) \]
		\end{enumerate}
	\end{proposition}
	
	\reference[GOU20]{169}
	
	\begin{application}
		On considère $f : x \mapsto e^{- \alpha x^2}$ pour $\alpha > 0$. Alors, $f$ vérifie
		\[ \forall x \in \mathbb{R}, \, \widehat{f}(\xi) = \frac{1}{i \xi} f(\xi) \]
		ce qui permet de retrouver l'\cref{250-1}.
	\end{application}
	
	\subsubsection{Convolution}
	
	\reference[AMR08]{75}
	
	\begin{definition}
		Soient $f$ et $g$ deux fonctions de $\mathbb{R}^d$ dans $\mathbb{R}$. On dit que \textbf{la convolée} (ou \textbf{le produit de convolution}) de $f$ et $g$ en $x \in \mathbb{R}$ \textbf{existe} si la fonction
		\[
		\begin{array}{ccc}
			\mathbb{R} &\rightarrow& \mathbb{C} \\
			t &\mapsto& f(x-t)g(t)
		\end{array}
		\]
		est intégrable sur $\mathbb{R}^d$ pour la mesure de Lebesgue. On pose alors :
		\[ (f * g)(x) = \int_{\mathbb{R}^d} f(x-t)g(t) \, \mathrm{d}t \]
	\end{definition}
	
	\begin{example}
		Soient $a < b \in \mathbb{R}^+_*$. Alors $\mathbb{1}_{[-a, a]} * \mathbb{1}_{[-b,b]}$ existe pour tout $x \in \mathbb{R}$ et
		\[ \left( \mathbb{1}_{[-a, a]} * \mathbb{1}_{[-b,b]} \right)(x) =
		\begin{cases}
			2a &\text{si } 0 \leq \vert x \vert \leq b-a \\
			b+a-\vert x \vert &\text{si } b-a \leq \vert x \vert \leq b+a \\
			0 &\text{sinon}
		\end{cases}
		\]
	\end{example}
	
	\begin{proposition}
		Dans $L_1(\mathbb{R}^d)$, dès qu'il a un sens, le produit de convolution de deux fonctions est commutatif, bilinéaire et associatif.
	\end{proposition}
	
	\begin{theorem}[Convolution dans $L_1(\mathbb{R}^d)$]
		Soient $f, g \in L_1(\mathbb{R}^d)$. Alors :
		\begin{enumerate}[(i)]
			\item pp. en $x \in \mathbb{R}^d$, $t \mapsto f(x-t)g(t)$ est intégrable sur $\mathbb{R}^d$.
			\item $f * g$ est intégrable sur $\mathbb{R}^d$.
			\item $\Vert f * g \Vert_1 \leq \Vert f \Vert_1 \Vert g \Vert_1$.
			\item L'espace vectoriel normé $(L_1(\mathbb{R}^d), \Vert . \Vert_1)$ muni de $*$ est une algèbre de Banach commutative.
		\end{enumerate}
	\end{theorem}
	
	\reference{114}
	
	\begin{proposition}
		\[ \forall f, g \in L_1(\mathbb{R}^d), \, \widehat{f * g} = \widehat{f} \widehat{g} \]
		ie. $\mathcal{F} : (L_1(\mathbb{R}^d), +, *, \cdot) \rightarrow (\mathcal{C}_0(\mathbb{R}^d), +, \times, \cdot)$ est un morphisme d'algèbres.
	\end{proposition}
	
	\begin{corollary}
		L'algèbre $(L_1(\mathbb{R}^d), +, *, \cdot)$ n'a pas d'élément unité.
	\end{corollary}
	
	\begin{application}
		\[ f * f = f \iff f = 0 \]
	\end{application}
	
	\begin{theorem}[Formule de dualité]
		\[ \forall f, g \in L_1(\mathbb{R}^d), \int_{\mathbb{R}^d} f(t) \widehat{g}(t) \, \mathrm{d}t = \int_{\mathbb{R}^d} \widehat{f}(t) g(t) \, \mathrm{d}t \]
	\end{theorem}
	
	\subsubsection{Inversion}
	
	\begin{theorem}[Formule d'inversion de Fourier]
		Si $f \in L_1(\mathbb{R}^d)$ est telle que $\widehat{f} \in L_1(\mathbb{R}^d)$, alors
		\[ \widehat{\widehat{f}}(x) = (2\pi)^d f(x) \text{ pp. en } x \in \mathbb{R}^d \]
	\end{theorem}
	
	\begin{example}
		Une solution de l'équation intégrale d'inconnue $y$ :
		\[ \int_{\mathbb{R}} \frac{y(t)}{(x-t)^2 + a^2} = \frac{1}{x^2 + b^2} \]
		est $x \mapsto \frac{a(b-a)}{b \pi (x^2 + (b-a)^2)}$ pour $0 < a < b$.
	\end{example}
	
	\begin{corollary}
		La transformation de Fourier $\mathcal{F} : L_1(\mathbb{R}^d) \rightarrow \mathcal{C}_0(\mathbb{R}^d)$ est une application injective.
	\end{corollary}
	
	\begin{proposition}
		Soient $g \in L_1(\mathbb{R}^d)$ et $f \in L_1(\mathbb{R}^d)$ telle que $\widehat{f} \in L_1(\mathbb{R}^d)$, alors
		\[ \widehat{fg} = \frac{1}{(2\pi)^d} \widehat{f} * \widehat{g} \]
	\end{proposition}
	
	\subsection{Transformation de Fourier dans d'autres espaces}
	
	\subsubsection{Dans \texorpdfstring{$L_2(\mathbb{R}^d)$}{L₂(Rᵈ)}}
	
	\reference{122}
	
	\begin{theorem}[Plancherel-Parseval]
		\[ \forall f \in L_1(\mathbb{R}^d) \, \cap \, L_2(\mathbb{R}^d), \, \Vert \widehat{f} \Vert^2_2 = (2 \pi)^d \Vert f \Vert^2_2 \]
	\end{theorem}
	
	\begin{remark}
		En termes de produit scalaire, la formule précédente s'écrit
		\[ \forall f, g \in L_2(\mathbb{R}^d), \, \int_{\mathbb{R}^d} \widehat{f}(\xi) \overline{\widehat{g}(\xi)} \, \mathrm{d}\xi = (2 \pi)^d \int_{\mathbb{R}^d} f(x) \overline{g(x)} \, \mathrm{d}x \]
	\end{remark}
	
	\begin{theorem}
		Soit $f \in L_2(\mathbb{R}^d)$. Alors :
		\begin{enumerate}[(i)]
			\item Il existe une suite $(f_n)$ de $L_1(\mathbb{R}^d) \, \cap \, L_2(\mathbb{R}^d)$ qui converge vers $f$ dans $L_2(\mathbb{R}^d)$.
			\item Pour une telle suite $(f_n)$, la suite $(\widehat{f_n})$ converge dans $L_2(\mathbb{R}^d)$ vers une limite $\widetilde{f}$ indépendante de la suite choisie.
		\end{enumerate}
	\end{theorem}
	
	\begin{definition}
		La limite $\widetilde{f}$ est la \textbf{transformée de Fourier} de $f$ dans $L_2(\mathbb{R}^d)$.
	\end{definition}
	
	\begin{proposition}
		Les transformations de Fourier $L_1(\mathbb{R}^d)$ et $L_2(\mathbb{R}^d)$ coïncident sur $L_1(\mathbb{R}^d) \, \cap \, L_2(\mathbb{R}^d)$.
	\end{proposition}
	
	\begin{remark}
		On a prolongé $\mathcal{F}$ à $L_2(\mathbb{R}^d)$, mais il faut prendre garde au fait que $\mathcal{F}$ désigne deux applications distinctes : $\mathcal{F} : L_1(\mathbb{R}^d) \rightarrow \mathcal{C}_0(\mathbb{R}^d)$ et $\mathcal{F} : L_2(\mathbb{R}^d) \rightarrow L_2(\mathbb{R}^d)$, ces deux applications ne coïncidant que sur $L_1(\mathbb{R}^d) \, \cap \, L_2(\mathbb{R}^d)$.
	\end{remark}
	
	\begin{proposition}
		Soit $f \in L_2(\mathbb{R})$. On a les relations suivantes :
		\[ \lim_{A \rightarrow +\infty} \Vert \varphi_A - f \Vert_2 = 0 \text{ et } \lim_{A \rightarrow +\infty} \Vert \psi_A - f \Vert_2 = 0 \]
		où
		\[ \varphi_A(\xi) = \int_{-A}^{A} f(x) e^{-ix\xi} \, \mathrm{d}x \text{ et } \psi_A(\xi) = \frac{1}{2 \pi} \int_{-A}^{A} \widehat{f}(\xi) e^{-ix\xi} \, \mathrm{d}\xi \]
	\end{proposition}
	
	\begin{corollary}
		Lorsque $f \in L_2(\mathbb{R})$ et $\widehat{f} \in L_1(\mathbb{R})$, on a
		\[ \text{pp. en } x \in \mathbb{R}, \, f(x) = \frac{1}{2 \pi} \int_{-\infty}^{+\infty} \widehat{f}(\xi) e^{-ix\xi} \, \mathrm{d}\xi \]
	\end{corollary}
	
	\begin{theorem}[Formule d'inversion de Fourier-Plancherel]
		\textbf{L'opérateur de Fourier-Plancherel}
		\[
		\mathcal{P} :
		\begin{array}{ccc}
			L_2(\mathbb{R}^d) &\rightarrow& L_2(\mathbb{R}^d) \\
			f &\mapsto& \frac{1}{(\sqrt{2 \pi})^d} \mathcal{F}(f)
		\end{array}
		\]
		est un automorphisme d'inverse $\mathcal{P}^{-1} = \overline{\mathcal{P}}$.
	\end{theorem}
	
	\reference[D-L]{451}
	
	\begin{example}
		On pose $f = \mathbb{1}_{[-a, a]}$ et on a $\forall \xi \neq 0, \, \widehat{f}(\xi) = \frac{2 \sin(a\xi)}{\xi}$. Or, $\widehat{f} \in L_2(\mathbb{R}) \setminus L_1(\mathbb{R})$. On peut calculer sa transformée de Fourier dans $L_2(\mathbb{R})$ :
		\[ \forall x \in \mathbb{R}, \, \widehat{\widehat{f}}(x) = \widehat{(\widehat{f})}(x) = f(-x) = \mathbb{1}_{[-a, a]}(x) \]
	\end{example}
	
	\subsubsection{Dans \texorpdfstring{$\mathcal{S}(\mathbb{R}^d)$}{S(Rᵈ)}}
	
	\reference[AMR08]{133}
	
	\begin{definition}
		Une fonction $f : \mathbb{R}^d \rightarrow \mathbb{C}$ est dite \textbf{à décroissance rapide} si
		\[ \forall \alpha \in \mathbb{N}^d, \, \lim_{\Vert x \Vert \rightarrow +\infty} \vert x^\alpha f(x) \vert = 0 \]
		où $(x_1, \dots, x_d)^{(\alpha_1, \dots, \alpha_d)} = x_1^{\alpha_1} \dots x_d^{\alpha_d}$.
	\end{definition}
	
	\begin{example}
		$x \mapsto e^{-\vert x \vert}$ est à décroissance rapide sur $\mathbb{R}$.
	\end{example}
	
	\begin{definition}
		On appelle \textbf{classe de Schwartz}, noté $\mathcal{S}(\mathbb{R}^d)$, l'espaces des fonctions de $f : \mathbb{R}^d \rightarrow \mathbb{C}$ telles que :
		\begin{itemize}
			\item $f \in \mathcal{C}^\infty(\mathbb{R}^d)$.
			\item $f$ est à décroissance rapide ainsi que toutes ses dérivées.
		\end{itemize}
	\end{definition}
	
	\begin{proposition}
		$\mathcal{S}(\mathbb{R}^d)$ est un espace vectoriel stable par dérivation, par multiplication par un polynôme, par produit, par conjugaison et par translation.
	\end{proposition}
	
	\begin{theorem}
		\begin{enumerate}[(i)]
			\item $\mathcal{S}(\mathbb{R}^d) \subseteq L_1(\mathbb{R}^d)$.
			\item $\mathcal{F}(\mathcal{S}(\mathbb{R}^d)) \subseteq \mathcal{S}(\mathbb{R}^d)$.
		\end{enumerate}
	\end{theorem}
	
	\begin{theorem}
		$\mathcal{F} : \mathcal{S}(\mathbb{R}^d) \rightarrow \mathcal{S}(\mathbb{R}^d)$ est un automorphisme bicontinu de $\mathcal{S}(\mathbb{R}^d)$ dont l'inverse est donné par
		\[ \mathcal{F}^{-1} = \frac{1}{(2 \pi)^d} \overline{F} \]
	\end{theorem}
	
	\subsection{Applications}
	
	\subsubsection{Séries de fonctions}
	
	\reference[GOU20]{284}
	\dev{formule-sommatoire-de-poisson}
	
	\begin{theorem}[Formule sommatoire de Poisson]
		Soit $f : \mathbb{R} \rightarrow \mathbb{C}$ une fonction de classe $\mathcal{C}^1$ telle que $f(x) = O \left( \frac{1}{x^2} \right)$ et $f'(x) = O \left( \frac{1}{x^2} \right)$ quand $|x| \longrightarrow +\infty$. Alors :
		\[ \forall x \in \mathbb{R}, \, \sum_{n \in \mathbb{Z}} f(x+n) = \sum_{n \in \mathbb{Z}} \widehat{f}(2 \pi n) e^{2 i \pi n x} \]
	\end{theorem}
	
	\begin{application}[Fonction $\theta$]
		\[ \forall s > 0, \, \sum_{n=-\infty}^{+\infty} e^{-\pi n^2 s} = \frac{1}{\sqrt{s}} \sum_{n=-\infty}^{+\infty} e^{-\frac{\pi n^2}{s}} \]
	\end{application}
	
	\newpage
	\subsubsection{Bases hilbertiennes}
	
	\reference[BMP]{110}
	
	Soit $I$ un intervalle de $\mathbb{R}$. On pose $\forall n \in \mathbb{N}$, $g_n : x \mapsto x^n$.
	
	\begin{definition}
		On appelle \textbf{fonction poids} une fonction $\rho : I \rightarrow \mathbb{R}$ mesurable, positive et telle que $\forall n \in \mathbb{N}, \rho g_n \in L_1(I)$.
	\end{definition}
	
	Soit $\rho : I \rightarrow \mathbb{R}$ une fonction poids.
	
	\begin{notation}
		On note $L_2(I, \rho)$ l'espace des fonctions de carré intégrable pour la mesure de densité $\rho$ par rapport à la mesure de Lebesgue.
	\end{notation}
	
	\begin{proposition}
		Muni de
		\[ \langle ., . \rangle : (f,g) \mapsto \int_I f(x) \overline{g(x)} \rho(x) \, \mathrm{d}x \]
		$L_2(I, \rho)$ est un espace de Hilbert.
	\end{proposition}
	
	\begin{theorem}
		Il existe une unique famille $(P_n)$ de polynômes unitaires orthogonaux deux-à-deux telle que $\deg(P_n) = n$ pour tout entier $n$. C'est la famille de \textbf{polynômes orthogonaux} associée à $\rho$ sur $I$.
	\end{theorem}
	
	\begin{example}[Polynômes de Hermite]
		Si $\forall x \in I, \, \rho(x) = e^{-x^2}$, alors
		\[ \forall n \in \mathbb{N}, \, \forall x \in I, \, P_n(x) = \frac{(-1)^n}{2^n} e^{x^2} \frac{\partial}{\partial x^n} \left( e^{-x^2} \right) \] 
	\end{example}
	
	\reference{140}
	
	\begin{lemma}
		On suppose que $\forall n \in \mathbb{N}$, $g_n \in L_1(I, \rho)$ et on considère $(P_n)$ la famille des polynômes orthogonaux associée à $\rho$ sur $I$. Alors $\forall n \in \mathbb{N}$, $g_n \in L_2(I, \rho)$. En particulier, l'algorithme de Gram-Schmidt a bien du sens et $(P_n)$ est bien définie.
	\end{lemma}
	
	\dev{densite-des-polynomes-orthogonaux}
	
	\begin{application}
		On considère $(P_n)$ la famille des polynômes orthogonaux associée à $\rho$ sur $I$ et on suppose qu'il existe $a > 0$ tel que
		\[ \int_I e^{a \vert x \vert} \rho(x) \, \mathrm{d}x < +\infty \]
		alors $(P_n)$ est une base hilbertienne de $L_2(I, \rho)$ pour la norme $\Vert . \Vert_2$.
	\end{application}
	
	\begin{cexample}
		On considère, sur $I = \mathbb{R}^+_*$, la fonction poids $\rho : x \mapsto x^{-\ln(x)}$. Alors, la famille des $g_n$ n'est pas totale. La famille des polynômes orthogonaux associée à ce poids particulier n'est donc pas totale non plus : ce n'est pas une base hilbertienne.
	\end{cexample}
	
	\subsubsection{En probabilités}
	
	\reference[G-K]{239}
	
	Soit $X : (\Omega, \mathcal{A}, \mathbb{P}) \rightarrow (\mathbb{R}^d, \mathcal{B}(\mathbb{R}^d))$ un vecteur aléatoire.
	
	\begin{definition}
		On appelle \textbf{fonction caractéristique} de $X$, notée $\phi_X$, la transformée de Fourier de la loi $\mathbb{P}_X$ (définie à un signe près) :
		\[ \phi_X : t \mapsto \mathbb{E}(e^{i \langle t, x \rangle}) \]
	\end{definition}
	
	\reference{165}
	
	\begin{remark}
		Si $X$ est un vecteur aléatoire réel admettant $f$ pour densité, alors
		\[ \forall t \in \mathbb{R}^d, \, \phi_X(t) = \int_{\mathbb{R}^d} e^{i \langle t, x \rangle} f(x) \, \mathrm{d}\mathbb{P}(x) \]
	\end{remark}
	
	\reference{239}
	
	\begin{theorem}
		Soient $X$ et $Y$ deux vecteurs aléatoires réels. Alors,
		\[ \phi_X = \phi_Y \iff \mathbb{P}_X = \mathbb{P}_Y \]
	\end{theorem}
	
	\begin{example}
		\begin{itemize}
			\item $X \sim \mathcal{U}([-1, 1]) \iff \forall t \in \mathbb{R}, \, \phi_X(t) = \begin{cases}
				\frac{\sin(t)}{t} \text{ si } t \neq 0 \\
				1 \text{ sinon}
			\end{cases}$
			\item $X \sim \mathcal{E}(\lambda) \iff \forall t \in \mathbb{R}, \, \phi_X(t) = \frac{1}{1-it}$.
			\item $X(\Omega) \subseteq \mathbb{N} \implies \forall t \in \mathbb{R}, \, \phi_X(t) = G_X(e^{it})$ où $G_X$ est la fonction génératrice de $X$.
		\end{itemize}
	\end{example}
	
	\begin{theorem}
		Soit $N \in \mathbb{N}^*$, alors dans pour une variable aléatoire réelle,
		\[ \mathbb{E}(\vert X \vert^N) < +\infty \implies \phi_X \in \mathcal{C}^n(\mathbb{R}) \]
	\end{theorem}
	
	\begin{corollary}
		On se place dans le cadre du théorème précédent. On a :
		\[ \forall k \in \llbracket 0, N \rrbracket, \, \phi_X^{(k)} (0) = i^k \mathbb{E}(X^k) \]
	\end{corollary}
	
	\begin{theorem}
		Si $X$ et $Y$ sont deux vecteurs aléatoires réels indépendants :
		\begin{enumerate}[(i)]
			\item $\phi_{X+Y} = \phi_X \phi_Y$.
			\item $\forall s, t \in \mathbb{R}^d, \, \phi_{(X,Y)}(s,t) = \phi_X(s) \phi_Y(t)$.
		\end{enumerate}
	\end{theorem}
	%</content>
\end{document}
