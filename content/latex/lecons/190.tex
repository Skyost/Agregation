\documentclass[12pt, a4paper]{report}

% LuaLaTeX :

\RequirePackage{iftex}
\RequireLuaTeX

% Packages :

\usepackage[french]{babel}
%\usepackage[utf8]{inputenc}
%\usepackage[T1]{fontenc}
\usepackage[pdfencoding=auto, pdfauthor={Hugo Delaunay}, pdfsubject={Mathématiques}, pdfcreator={agreg.skyost.eu}]{hyperref}
\usepackage{amsmath}
\usepackage{amsthm}
%\usepackage{amssymb}
\usepackage{stmaryrd}
\usepackage{tikz}
\usepackage{tkz-euclide}
\usepackage{fourier-otf}
\usepackage{fontspec}
\usepackage{titlesec}
\usepackage{fancyhdr}
\usepackage{catchfilebetweentags}
\usepackage[french, capitalise, noabbrev]{cleveref}
\usepackage[fit, breakall]{truncate}
\usepackage[top=2.5cm, right=2cm, bottom=2.5cm, left=2cm]{geometry}
\usepackage{enumerate}
\usepackage{tocloft}
\usepackage{microtype}
%\usepackage{mdframed}
%\usepackage{thmtools}
\usepackage{xcolor}
\usepackage{tabularx}
\usepackage{aligned-overset}
\usepackage[subpreambles=true]{standalone}
\usepackage{environ}
\usepackage[normalem]{ulem}
\usepackage{marginnote}
\usepackage{etoolbox}
\usepackage{setspace}
\usepackage[bibstyle=reading, citestyle=draft]{biblatex}
\usepackage{xpatch}
\usepackage[many, breakable]{tcolorbox}
\usepackage[backgroundcolor=white, bordercolor=white, textsize=small]{todonotes}

% Bibliographie :

\newcommand{\overridebibliographypath}[1]{\providecommand{\bibliographypath}{#1}}
\overridebibliographypath{../bibliography.bib}
\addbibresource{\bibliographypath}
\defbibheading{bibliography}[\bibname]{%
	\newpage
	\section*{#1}%
}
\renewbibmacro*{entryhead:full}{\printfield{labeltitle}}
\DeclareFieldFormat{url}{\newline\footnotesize\url{#1}}
\AtEndDocument{\printbibliography}

% Police :

\setmathfont{Erewhon Math}

% Tikz :

\usetikzlibrary{calc}

% Longueurs :

\setlength{\parindent}{0pt}
\setlength{\headheight}{15pt}
\setlength{\fboxsep}{0pt}
\titlespacing*{\chapter}{0pt}{-20pt}{10pt}
\setlength{\marginparwidth}{1.5cm}
\setstretch{1.1}

% Métadonnées :

\author{agreg.skyost.eu}
\date{\today}

% Titres :

\setcounter{secnumdepth}{3}

\renewcommand{\thechapter}{\Roman{chapter}}
\renewcommand{\thesubsection}{\Roman{subsection}}
\renewcommand{\thesubsubsection}{\arabic{subsubsection}}
\renewcommand{\theparagraph}{\alph{paragraph}}

\titleformat{\chapter}{\huge\bfseries}{\thechapter}{20pt}{\huge\bfseries}
\titleformat*{\section}{\LARGE\bfseries}
\titleformat{\subsection}{\Large\bfseries}{\thesubsection \, - \,}{0pt}{\Large\bfseries}
\titleformat{\subsubsection}{\large\bfseries}{\thesubsubsection. \,}{0pt}{\large\bfseries}
\titleformat{\paragraph}{\bfseries}{\theparagraph. \,}{0pt}{\bfseries}

\setcounter{secnumdepth}{4}

% Table des matières :

\renewcommand{\cftsecleader}{\cftdotfill{\cftdotsep}}
\addtolength{\cftsecnumwidth}{10pt}

% Redéfinition des commandes :

\renewcommand*\thesection{\arabic{section}}
\renewcommand{\ker}{\mathrm{Ker}}

% Nouvelles commandes :

\newcommand{\website}{https://agreg.skyost.eu}

\newcommand{\tr}[1]{\mathstrut ^t #1}
\newcommand{\im}{\mathrm{Im}}
\newcommand{\rang}{\operatorname{rang}}
\newcommand{\trace}{\operatorname{trace}}
\newcommand{\id}{\operatorname{id}}
\newcommand{\stab}{\operatorname{Stab}}

\providecommand{\newpar}{\\[\medskipamount]}

\providecommand{\lesson}[3]{%
	\title{#3}%
	\hypersetup{pdftitle={#3}}%
	\setcounter{section}{\numexpr #2 - 1}%
	\section{#3}%
	\fancyhead[R]{\truncate{0.73\textwidth}{#2 : #3}}%
}

\providecommand{\development}[3]{%
	\title{#3}%
	\hypersetup{pdftitle={#3}}%
	\section*{#3}%
	\fancyhead[R]{\truncate{0.73\textwidth}{#3}}%
}

\providecommand{\summary}[1]{%
	\textit{#1}%
	\medskip%
}

\tikzset{notestyleraw/.append style={inner sep=0pt, rounded corners=0pt, align=center}}

%\newcommand{\booklink}[1]{\website/bibliographie\##1}
\newcommand{\citelink}[2]{\hyperlink{cite.\therefsection @#1}{#2}}
\newcommand{\previousreference}{}
\providecommand{\reference}[2][]{%
	\notblank{#1}{\renewcommand{\previousreference}{#1}}{}%
	\todo[noline]{%
		\protect\vspace{16pt}%
		\protect\par%
		\protect\notblank{#1}{\cite{[\previousreference]}\\}{}%
		\protect\citelink{\previousreference}{p. #2}%
	}%
}

\definecolor{devcolor}{HTML}{00695c}
\newcommand{\dev}[1]{%
	\reversemarginpar%
	\todo[noline]{
		\protect\vspace{16pt}%
		\protect\par%
		\bfseries\color{devcolor}\href{\website/developpements/#1}{DEV}
	}%
	\normalmarginpar%
}

% En-têtes :

\pagestyle{fancy}
\fancyhead[L]{\truncate{0.23\textwidth}{\thepage}}
\fancyfoot[C]{\scriptsize \href{\website}{\texttt{agreg.skyost.eu}}}

% Couleurs :

\definecolor{property}{HTML}{fffde7}
\definecolor{proposition}{HTML}{fff8e1}
\definecolor{lemma}{HTML}{fff3e0}
\definecolor{theorem}{HTML}{fce4f2}
\definecolor{corollary}{HTML}{ffebee}
\definecolor{definition}{HTML}{ede7f6}
\definecolor{notation}{HTML}{f3e5f5}
\definecolor{example}{HTML}{e0f7fa}
\definecolor{cexample}{HTML}{efebe9}
\definecolor{application}{HTML}{e0f2f1}
\definecolor{remark}{HTML}{e8f5e9}
\definecolor{proof}{HTML}{e1f5fe}

% Théorèmes :

\theoremstyle{definition}
\newtheorem{theorem}{Théorème}

\newtheorem{property}[theorem]{Propriété}
\newtheorem{proposition}[theorem]{Proposition}
\newtheorem{lemma}[theorem]{Lemme}
\newtheorem{corollary}[theorem]{Corollaire}

\newtheorem{definition}[theorem]{Définition}
\newtheorem{notation}[theorem]{Notation}

\newtheorem{example}[theorem]{Exemple}
\newtheorem{cexample}[theorem]{Contre-exemple}
\newtheorem{application}[theorem]{Application}

\theoremstyle{remark}
\newtheorem{remark}[theorem]{Remarque}

\counterwithin*{theorem}{section}

\newcommand{\applystyletotheorem}[1]{
	\tcolorboxenvironment{#1}{
		enhanced,
		breakable,
		colback=#1!98!white,
		boxrule=0pt,
		boxsep=0pt,
		left=8pt,
		right=8pt,
		top=8pt,
		bottom=8pt,
		sharp corners,
		after=\par,
	}
}

\applystyletotheorem{property}
\applystyletotheorem{proposition}
\applystyletotheorem{lemma}
\applystyletotheorem{theorem}
\applystyletotheorem{corollary}
\applystyletotheorem{definition}
\applystyletotheorem{notation}
\applystyletotheorem{example}
\applystyletotheorem{cexample}
\applystyletotheorem{application}
\applystyletotheorem{remark}
\applystyletotheorem{proof}

% Environnements :

\NewEnviron{whitetabularx}[1]{%
	\renewcommand{\arraystretch}{2.5}
	\colorbox{white}{%
		\begin{tabularx}{\textwidth}{#1}%
			\BODY%
		\end{tabularx}%
	}%
}

% Maths :

\DeclareFontEncoding{FMS}{}{}
\DeclareFontSubstitution{FMS}{futm}{m}{n}
\DeclareFontEncoding{FMX}{}{}
\DeclareFontSubstitution{FMX}{futm}{m}{n}
\DeclareSymbolFont{fouriersymbols}{FMS}{futm}{m}{n}
\DeclareSymbolFont{fourierlargesymbols}{FMX}{futm}{m}{n}
\DeclareMathDelimiter{\VERT}{\mathord}{fouriersymbols}{152}{fourierlargesymbols}{147}


% Bibliographie :

\addbibresource{\bibliographypath}%
\defbibheading{bibliography}[\bibname]{%
	\newpage
	\section*{#1}%
}
\renewbibmacro*{entryhead:full}{\printfield{labeltitle}}%
\DeclareFieldFormat{url}{\newline\footnotesize\url{#1}}%

\AtEndDocument{\printbibliography}

\begin{document}
  %<*content>
  \lesson{algebra}{190}{Méthodes combinatoires, problèmes de dénombrement.}

  \subsection{Dénombrement}
  
  \subsubsection{Principes de base}
  
  \reference[GOU21]{299}
  
  \begin{definition}
    On dit qu'un ensemble $E$ est \textbf{fini} s'il est vide ou s'il existe $n \in \mathbb{N}^*$ tel qu'il existe une bijection de $\llbracket 1, n \rrbracket$ dans $E$. Dans ce cas, l'entier $n$ ne dépend pas de la bijection, on l'appelle \textbf{cardinal} de $E$. Il est noté $\vert E \vert$. Si $E$ est vide, on pose $\vert E \vert = 0$.
  \end{definition}
  
  \begin{proposition}
    Soient $E$ et $F$ deux ensembles.
    \begin{enumerate}[label=(\roman*)]
      \item Si $E$ est fini et s'il existe une injection de $E$ vers $F$, alors $E$ est fini et $\vert E \vert \leq \vert F \vert$.
      \item Si $E$ est fini et s'il existe une surjection de $E$ vers $F$, alors $F$ est fini et $\vert E \vert \geq \vert F \vert$.
      \item Si $E$ et s'il existe une bijection de $E$ vers $F$, alors $F$ est fini et $\vert E \vert = \vert F \vert$.
    \end{enumerate}
  \end{proposition}
  
  \begin{corollary}
    Soit $B$ un ensemble fini et $A \subseteq B$. Alors $A$ est fini et $\vert A \vert \leq \vert B \vert$. Si $\vert A \vert = \vert B \vert$, alors $A = B$. 
  \end{corollary}
  
  \begin{corollary}[Principe des tiroirs]
    Soient $E$ et $F$ deux ensembles finis avec $\vert E \vert > \vert F \vert$. Si $\varphi$ est une application de $E$ vers $F$, alors il existe $y \in F$ ayant au moins deux antécédents par $\varphi$ dans $E$.
  \end{corollary}
  
  \begin{remark}[Interprétation]
    Si on doit ranger $n+1$ chaussettes dans $n$ tiroirs, alors un des tiroirs (au moins) contiendra deux chaussettes ou plus.
  \end{remark}
  
  \begin{proposition}
    Soient $A$ et $B$ deux ensembles finis. Alors,
    \begin{enumerate}[label=(\roman*)]
      \item $\vert A \, \cup \, B \vert = \vert A \vert + \vert B \vert - \vert A \, \cap \, B \vert$.
      \item $\vert A \setminus B \vert = \vert A \vert - \vert A \, \cap \, B \vert$.
    \end{enumerate}
  \end{proposition}
  
  \begin{proposition}[Formule du crible de Poincaré]
    Soient $A_1, \dots, A_n$ des ensembles finis. Alors,
    \[ \left\vert \bigcup_{i=1}^n A_i \right\vert = \sum_{k=1}^n (-1)^{k-1} \sum_{1 \leq i_1 < \dots < i_k \leq n} \vert A_{i_1} \, \cap \, \dots \, \cap \, A_{i_k} \]
  \end{proposition}
  
  \reference[G-K]{401}
  
  \begin{example}
    Pour $n = 3$, on a
    \[ \vert A_1 \, \cap \, A_2 \, \cap \, A_3 \vert = \vert A_1 \vert + \vert A_2 \vert + \vert A_3 \vert - \vert A_1 \, \cap \, A_2 \vert - \vert A_2 \, \cap \, A_3 \vert - - \vert A_1 \, \cap \, A_3 \vert + \vert A_1 \, \cup \, A_2 \, \cup \, A_3 \vert \]
  \end{example}
  
  \begin{lemma}[des bergers]
    Soient $A$ et $B$ deux ensembles. On suppose $A$ fini. Soit $\varphi : A \rightarrow B$ surjective telle que tout élément de $B$ admet exactement $a$ antécédents par $\varphi$. Alors,
    \[ \vert A \vert = \frac{\vert B \vert}{a} \]
  \end{lemma}
  
  \subsubsection{Combinatoire}
  
  \paragraph{Listes}
  
  \reference[GOU21]{301}
  
  \begin{proposition}
    Soient $n$ ensembles finis $E_1, \dots, E_n$. Le produit cartésien $E_1 \times \dots \times E_n$ est un ensemble fini et vérifie $\vert E_1 \times \dots \times E_n \vert = \vert E_1 \vert \times \dots \times \vert E_n \vert$. En particulier, pour un ensemble $E$ fini, on a $\vert E^n \vert = \vert E \vert^n$.
  \end{proposition}
  
  \begin{definition}
    Soit $E$ un ensemble et $p \in \mathbb{N}^*$. On appelle \textbf{$p$-liste} (ou \textbf{$p$-uplet}) de $E$, tout élément $(x_1, \dots, x_p)$ de $E^p$.
  \end{definition}
  
  \begin{remark}
    \begin{itemize}
      \item Si $E$ est fini, il y a $\vert E \vert^p$ $p$-listes de $E$.
      \item Dans une liste, l'ordre des éléments importe.
    \end{itemize}
  \end{remark}
  
  \begin{example}
    Dans un jeu de $52$ cartes, le nombre de façons de tirer $10$ cartes avec remise est $52^{10}$.
  \end{example}
  
  \paragraph{Arrangements}
  
  \begin{definition}
    Soit $E$ un ensemble fini de cardinal $n$. Soit $p$ un entier inférieur à $n$. On appelle \textbf{$p$-arrangement} de $E$ toute $p$-liste de $E$ d'éléments distincts.
  \end{definition}
  
  \begin{proposition}
    En reprenant les notations précédentes, le nombre de $p$-arrangements de $E$ est
    \[ A_n^p = n(n-1) \dots (n-p+1) = \frac{n!}{(n-p)!} \]
  \end{proposition}
  
  \begin{remark}
    \begin{itemize}
      \item Si $p = n$, on trouve que le nombre de $n$-arrangements est $n!$.
      \item Dans les arrangements, l'ordre des éléments importe, mais ceux-ci sont distincts.
    \end{itemize}
  \end{remark}
  
  \begin{example}
    Dans un jeu de $52$ cartes, le nombre de façons de tirer $10$ cartes sans remise est $A^{10}_{52} = 52 \times \dots \times 43$.
  \end{example}
  
  \begin{application}[Nombre d'applications entre deux ensembles finis]
    Soient $E$ et $F$ deux ensembles finis.
    \begin{enumerate}[label=(\roman*)]
      \item L'ensemble des applications de $E$ vers $F$, noté $F^E$ est fini, de cardinal $\vert F \vert^{\vert E \vert}$.
      \item Lorsque $\vert E \vert \leq \vert F \vert$, l'ensemble des applications injectives de $E$ dans $F$ est fini, de cardinal $A_n^p$.
      \item L'ensemble des bijections de $E$ vers $E$ appelées permutations de $E$, noté $\mathcal{S}(E)$, est fini et de cardinal $\vert E \vert!$.
    \end{enumerate}
  \end{application}
  
  \begin{corollary}
    Soit $E$ un ensemble fini. Le nombre total de parties de $E$ est $\vert \mathcal{P}(E) \vert = 2^{\vert E \vert}$.
  \end{corollary}
  
  \paragraph{Combinaisons}
  
  \begin{definition}
    Soit $E$ un ensemble fini de cardinal $n$. Soit $p \in \mathbb{N}$. On appelle \textbf{$p$-combinaison} de $E$ toute partie de $E$ de cardinal $p$. Ce nombre ne dépend que de $n$ et de $p$, on le note $\binom{n}{p}$.
  \end{definition}
  
  \begin{proposition}
    Soient $n, p \in \mathbb{N}$. Alors,
    \[
      \binom{n}{p} =
      \begin{cases}
        \frac{n!}{p! (n-p)!} &\text{ si } p \leq n \\
        0 &\text{ sinon}
      \end{cases}
    \]
  \end{proposition}
  
  \begin{remark}
    Dans les combinaisons, l'ordre des éléments n'importe pas, mais ceux-ci sont distincts.
  \end{remark}
  
  \begin{example}
    Dans un jeu de $52$ cartes, le nombre de façons de tirer $10$ cartes simultanément est $\binom{52}{10}$.
  \end{example}
  
  \begin{definition}
    Soit $E$ un ensemble fini de cardinal $n$. Soit $p$ un entier inférieur à $n$. On appelle \textbf{$p$-combinaison avec répétition} les $p$-listes dans lesquelles ont autorise les répétions, mais dans lesquelles l'ordre ne compte pas.
  \end{definition}
  
  \begin{proposition}
    En reprenant les notations précédentes, il y a $\binom{n+p-1}{p}$ $p$-combinaisons avec répétition.
  \end{proposition}
  
  \begin{proposition}
    Soit $n \in \mathbb{N}$. 
    \begin{enumerate}[label=(\roman*)]
      \item On a :
      \[ \sum_{k = 0}^n \binom{n}{k} = 2^n \]
      \item Soient $a$ et $b$ deux éléments d'une algèbre qui commutent. Alors,
      \[ (a+b)^n \sum_{k = 0}^n \binom{n}{k} a^k b^{n-k} \]
    \end{enumerate}
  \end{proposition}
  
  \reference{311}
  
  \begin{application}
    Soit $(F_n)$ la suite de Fibonacci définie par $F_0 = 0$, $F_1 = 1$ et $\forall n \geq 2$, $F_n = F_{n-1} + F_{n-2}$. Alors,
    \[ \forall n \in \mathbb{N}, \, F_n = \frac{1}{\sqrt{5}} \left ( \left ( \frac{1 + \sqrt{5}}{2} \right)^n - \left ( \frac{1 - \sqrt{5}}{2} \right)^n \right) = \sum_{k=0}^{\left\lfloor \frac{n}{2} \right\rfloor} \binom{n-k}{k} \]
  \end{application}
  
  \subsection{En théorie des groupes}
  
  Soit $G$ un groupe fini.
  
  \subsubsection{Actions de groupes}
  
  \reference[ULM21]{71}
  
  Soit $X$ un ensemble fini. On considère une action $\cdot$ de $G$ sur $X$.
  
  \begin{proposition}
    Soit $x \in X$. Alors :
    \begin{itemize}
      \item $|G \cdot x| = (G : \stab_G(x))$.
      \item $|G| = |\stab_G(x)| |G \cdot x|$.
      \item $|G \cdot x| = \frac{|G|}{|\stab_G(x)|}$
    \end{itemize}
  \end{proposition}
  
  \begin{theorem}[Formule des classes]
    Soit $\Omega$ un système de représentants d'orbites de l'action de $G$ sur $X$. Alors,
    \[ |X| = \sum_{\omega \in \Omega} |G \cdot \omega| = \sum_{\omega \in \Omega} (G : \stab_G(\omega)) = \sum_{\omega \in \Omega} \frac{|G|}{|\stab_G(\omega)|} \]
  \end{theorem}
  
  \begin{definition}
    On définit :
    \begin{itemize}
      \item $X^G = \{ x \in X \mid \forall g \in G, \, g \cdot x = x \}$ l'ensemble des points de $X$ laissés fixes par tous les éléments de $G$.
      \item $X^g = \{ x \in X \mid g \cdot x = x \}$ l'ensemble des points de $X$ laissés fixes par $g \in G$.
    \end{itemize}
  \end{definition}
  
  \begin{theorem}[Formule de Burnside]
    Le nombre $r$ d'orbites de $X$ sous l'action de $G$ est donné par
    \[ r = \frac{1}{|G|} \sum_{g \in G} |X^g| \]
  \end{theorem}
  
  \reference[I-P]{121}
  
  \begin{application}
    Deux colorations des faces d'un cube sont les mêmes si on peut passer de l'une à l'autre par une isométrie du dodécaèdre. Alors, le nombre de colorations distinctes d'un cube avec $c$ couleurs est
    \[ \frac{c^2}{24} (c^4 + 3^2 + 12c + 8) \]
  \end{application}
  
  \subsubsection{\texorpdfstring{$p$}{p}-groupes}
  
  \reference[ROM21]{22}
  
  \begin{definition}
    On dit que $G$ est un \textbf{$p$-groupe} s'il est d'ordre une puissance d'un nombre premier $p$.
  \end{definition}
  
  \begin{proposition}
    Soit $p$ un nombre premier. Si $G$ est un $p$-groupe opérant sur un ensemble $X$, alors,
    \[ |X^G| \equiv |X| \mod p \]
    où $X^G$ désigne l'ensemble des points fixes de $X$ sous l'action de $G$.
  \end{proposition}
  
  \begin{corollary}
    On note $G \cdot h_1, \dots, G \cdot h_r$ les classes de conjugaison de $G$. Alors,
    \begin{align*}
      \vert G \vert &= \vert Z(G) \vert + \sum_{\substack{i=1 \\ \vert G \cdot h_i \vert = 2}}^{r} \vert G \cdot h_i \vert \\
      &= \vert Z(G) \vert + \sum_{\substack{i=1 \\ \vert G \cdot h_i \vert = 2}}^{r} \frac{\vert G \vert}{\vert \stab_G(h_i) \vert}
    \end{align*}
  \end{corollary}
  
  \begin{corollary}
    Soit $p$ un nombre premier. Le centre d'un $p$-groupe non trivial est non trivial.
  \end{corollary}
  
  \begin{corollary}
    Soit $p$ un nombre premier. Un groupe d'ordre $p^2$ est toujours abélien.
  \end{corollary}
  
  \begin{application}[Théorème de Cauchy]
    On suppose $G$ non trivial et fini. Soit $p$ un premier divisant l'ordre de $G$. Alors il existe un élément d'ordre $p$ dans $G$.
  \end{application}
  
  \dev{theoreme-de-sylow}
  \reference[GOU21]{44}
  
  \begin{application}[Premier théorème de Sylow]
    On suppose $G$ fini d'ordre $n p^\alpha$ avec $n, \alpha \in \mathbb{N}$ et $p$ premier tel que $p \nmid n$. Alors, il existe un sous-groupe de $G$ d’ordre $p^\alpha$.
  \end{application}
  
  \subsection{En théorie des corps finis}
  
  Soit $q = p^n$ avec $p$ premier et $n \geq 2$.
  
  \subsubsection{Polynômes irréductibles}
  
  \reference[GOZ]{87}
  
  \begin{theorem}
    \[ \mathbb{F}_q = \mathbb{F}_p[X] / (P) \]
    où $P \in \mathbb{F}_p[X]$ est un polynôme irréductible de degré $n$ sur $\mathbb{F}_p$.
  \end{theorem}
  
  \begin{corollary}
    \begin{enumerate}[label=(\roman*)]
      \item Il existe des polynômes irréductibles de tout degré dans $\mathbb{F}_p[X]$.
      \item Si $P \in \mathbb{F}_p[X]$ est un polynôme irréductible sur $\mathbb{F}_p$ de degré $n$, alors $P$ divise $X^q - X$. En particulier, il est scindé sur $\mathbb{F}_q$. Donc son corps de rupture $\mathbb{F}_q = \mathbb{F}_p[X] / (P)$ est aussi son corps de décomposition.
    \end{enumerate}
  \end{corollary}
  
  \begin{theorem}
    Pour tout $j \in \mathbb{N}^*$, on note $I(p,q)$ l'ensemble des polynômes irréductibles unitaires de degré $j$ sur $\mathbb{F}_p$. Alors,
    \[ X^q - X = \prod_{d \mid n} \prod_{Q \in I(p,q)} Q \]
  \end{theorem}
  
  \begin{corollary}
    \[ q = \sum_{d \mid n} d \vert I(p,d) \vert \]
  \end{corollary}
  
  \begin{definition}
    On définit la \textbf{fonction de Möbius}, notée $\mu$, par
    \[
    \mu :
    \begin{array}{ccc}
      \mathbb{Z} &\rightarrow& \mathbb{Z} \\
      n &\mapsto& \begin{cases}
        1 &\text{si } n = 1 \\
        (-1)^k &\text{si } n = p_1 \dots p_k \text{ avec } p_1, \dots, p_k \text{ premiers distincts} \\
        0 &\text{sinon}
      \end{cases}
    \end{array}
    \]
  \end{definition}
  
  \begin{theorem}[Formule d'inversion de Möbius]
    Soient $f$ et $g$ des fonctions de $\mathbb{N}^*$ dans $\mathbb{C}$ telles que $\forall n \in \mathbb{N}^*, \, f(n) = \sum_{d \mid n} g(d)$. Alors,
    \[ \forall n \in \mathbb{N}^*, \, g(n) = \sum_{d \mid n} \mu(d) f \left( \frac{n}{d} \right) \]
  \end{theorem}
  
  \begin{corollary}
    \[ \forall n \in \mathbb{N}^*, \, \vert I(p,q) \vert = \frac{1}{n} \sum_{d \mid n} \mu(d) p^{\frac{n}{d}} = \frac{1}{n} \sum_{d \mid n} \mu \left( \frac{n}{d} \right) p^{d} \]
  \end{corollary}
  
  \subsubsection{Carrés dans les corps finis}
  
  \reference{93}
  
  \begin{proposition}
    On note $\mathbb{F}_q^2 = \{ x^2 \mid x \in \mathbb{F}_q \}$ et $\mathbb{F}_q^{*2} = \mathbb{F}_q^2 \, \cap \, \mathbb{F}_q^*$. Alors $\mathbb{F}_q^{*2}$ est un sous-groupe de $\mathbb{F}_q^*$.
  \end{proposition}
  
  \begin{proposition}
    \begin{enumerate}[label=(\roman*)]
      \item Si $p = 2$, $\mathbb{F}_q^2 = \mathbb{F}_q$, donc $\mathbb{F}_q^{*2} = \mathbb{F}_q^*$.
      \item Si $p > 2$, alors :
      \begin{itemize}
        \item $\mathbb{F}_q^{*2}$ est le noyau de l'endomorphisme de $\mathbb{F}_q^*$ défini par $x \mapsto x^{\frac{q-1}{2}}$.
        \item $\mathbb{F}_q^{*2}$ est un sous-groupe d'indice $2$ de $\mathbb{F}_q^*$.
        \item $\vert \mathbb{F}_q^{*2} \vert = \frac{q-1}{2}$ et $\vert \mathbb{F}_q^2 \vert = \frac{q+1}{2}$.
        \item $(-1) \in \mathbb{F}_q^{*2} \iff q \equiv 1 \mod 4$.
      \end{itemize}
    \end{enumerate}
  \end{proposition}
  
  \subsubsection{Groupe linéaire sur un corps fini}
  
  \reference[PER]{119}
  
  Soit $V$ un espace vectoriel de dimension finie $n$ sur un corps $\mathbb{K}$.
  
  \begin{definition}
    \begin{itemize}
      \item Le \textbf{groupe linéaire} de $V$, $\mathrm{GL}(V)$ est le groupe des applications linéaires de $V$ dans lui-même qui sont inversibles.
      \item Le \textbf{groupe spécial linéaire} de $V$, $\mathrm{SL}(V)$ est le sous-groupe de $\mathrm{GL}(V)$ constitué des applications de déterminant $1$.
      \item Les quotients de ces groupes par leur centre sont respectivement notés $\mathrm{PGL}(V)$ et $\mathrm{PSL}(V)$.
    \end{itemize}
  \end{definition}
  
  \reference{124}
  
  \begin{proposition}
    On se place dans le cas où $\mathbb{K} = \mathbb{F}_q$. Alors, les groupes précédents sont finis, et :
    \begin{enumerate}[label=(\roman*)]
      \item $|\mathrm{GL}(V)| = q^{\frac{n(n-1)}{2}}((q^n-1) \dots (q-1))$.
      \item $|\mathrm{PGL}(V)| = |\mathrm{SL}(V)| = \frac{|\mathrm{GL}(V)|}{q-1}$.
      \item $|\mathrm{PSL}(V)| = |\mathrm{SL}(V)| = \frac{|\mathrm{GL}(V)|}{(q-1)\operatorname{pgcd}(n,q-1)}$.
    \end{enumerate}
  \end{proposition}
  
  \subsection{En analyse}
  
  \subsubsection{Probabilités sur un ensemble fini}
  
  \reference[G-K]{137}
  
  Soit $(\Omega, \mathcal{A}, \mathbb{P})$ un espace probabilisé.
  
  \begin{definition}
    Soit $E \subseteq \Omega$ fini. On appelle loi uniforme sur $E$ la loi discrète définie sur $\mathcal{P}(\Omega)$ par
    \[
    \begin{array}{ccc}
      \mathcal{P}(\Omega) &\rightarrow& \llbracket 0, 1 \rrbracket \\
      A &\mapsto& \frac{\vert A \, \cap \, E \vert}{\vert E \vert}
    \end{array}
    \]
  \end{definition}
  
  \begin{remark}
    Il s'agit du nombre de cas favorables sur le nombre de cas possibles. Ainsi, $X$ suit la loi uniforme sur $E$ si on a $\forall x \in E, \, \mathbb{P}(X=x) = \frac{1}{\vert E \vert}$ et $\forall x \notin E, \, \mathbb{P}(X=x) = 0$.
    \newpar
    C'est, par exemple, la loi suivie par une variable aléatoire représentant le lancer d'un dé non truqué avec $E = \llbracket 1, 6 \rrbracket$.
  \end{remark}
  
  \begin{definition}
    Une variable aléatoire $X$ suit une \textbf{loi de Bernoulli} de paramètre $p \in [0,1]$, notée $\mathcal{B}(p)$, si $\mathbb{P}(X=1) = p$ et $\mathbb{P}(X=0)=1-p$.
  \end{definition}
  
  \begin{proposition}
    En reprenant les notations précédentes, $X$ est une loi discrète et on a
    \[ \mathbb{P}_X = (1-p) \delta_0 + p \delta_1 \]
  \end{proposition}
  
  \begin{definition}
    Une variable aléatoire $X$ suit une \textbf{loi de binomiale} de paramètres $n \in \mathbb{N}$ et $p \in [0,1]$, notée $\mathcal{B}(n, p)$, si $X$ est la somme de $n$ variables aléatoires indépendantes qui suivent des lois de Bernoulli de paramètre $p$.
  \end{definition}
  
  \begin{proposition}
    En reprenant les notations précédentes, $X$ est une loi discrète et on a
    \[ \forall k \in \mathbb{N}, \, \mathbb{P}(X = k) = \binom{n}{k} p^k (1-p)^{n-k} \]
  \end{proposition}
  
  \begin{remark}
    Il s'agit du nombre de succès pour $n$ tentatives.
    \newpar
    C'est, par exemple, la loi suivie par une variable aléatoire représentant le nombre de ``Pile'' obtenus lors d'un lancer de pièce équilibrée.
  \end{remark}
  
  \subsubsection{Utilisation des séries pour dénombrer}
  
  \reference[GOU21]{312}
  
  \begin{theorem}[Dérangements]
    Soit $n \in \mathbb{N}^*$. On note $\mathcal{D}_n$ l'ensemble des permutations de $\llbracket 1, n \rrbracket$ sans point fixe. Alors,
    \[ \vert \mathcal{D}_n \vert = n! \sum_{k=0}^{n} \frac{(-1)^k}{k!} = \left\lfloor \frac{n!}{e} + \frac{1}{2} \right\rfloor \]
  \end{theorem}
  
  \begin{example}
    $n$ personnes laissent leur chapeau à un vestiaire. En repartant, chaque personne prend un chapeau au hasard. La probabilité que personne ne reprenne son propre chapeau est d'environ $\frac{1}{e}$.
  \end{example}
  
  \reference{314}
  \dev{nombres-de-bell}
  
  \begin{theorem}[Nombres de Bell]
    Pour tout $n \in \mathbb{N}^*$, on note $B_n$ le nombre de partitions de $\llbracket 1, n \rrbracket$. Par convention on pose $B_0 = 1$. Alors,
    \[ \forall k \in \mathbb{N}^*, \, B_k = \frac{1}{e} \sum_{n=0}^{+\infty} \frac{n^k}{n!} \]
  \end{theorem}
  %</content>
\end{document}
