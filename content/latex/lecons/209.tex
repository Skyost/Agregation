\documentclass[12pt, a4paper]{report}

% LuaLaTeX :

\RequirePackage{iftex}
\RequireLuaTeX

% Packages :

\usepackage[french]{babel}
%\usepackage[utf8]{inputenc}
%\usepackage[T1]{fontenc}
\usepackage[pdfencoding=auto, pdfauthor={Hugo Delaunay}, pdfsubject={Mathématiques}, pdfcreator={agreg.skyost.eu}]{hyperref}
\usepackage{amsmath}
\usepackage{amsthm}
%\usepackage{amssymb}
\usepackage{stmaryrd}
\usepackage{tikz}
\usepackage{tkz-euclide}
\usepackage{fourier-otf}
\usepackage{fontspec}
\usepackage{titlesec}
\usepackage{fancyhdr}
\usepackage{catchfilebetweentags}
\usepackage[french, capitalise, noabbrev]{cleveref}
\usepackage[fit, breakall]{truncate}
\usepackage[top=2.5cm, right=2cm, bottom=2.5cm, left=2cm]{geometry}
\usepackage{enumerate}
\usepackage{tocloft}
\usepackage{microtype}
%\usepackage{mdframed}
%\usepackage{thmtools}
\usepackage{xcolor}
\usepackage{tabularx}
\usepackage{aligned-overset}
\usepackage[subpreambles=true]{standalone}
\usepackage{environ}
\usepackage[normalem]{ulem}
\usepackage{marginnote}
\usepackage{etoolbox}
\usepackage{setspace}
\usepackage[bibstyle=reading, citestyle=draft]{biblatex}
\usepackage{xpatch}
\usepackage[many, breakable]{tcolorbox}
\usepackage[backgroundcolor=white, bordercolor=white, textsize=small]{todonotes}

% Bibliographie :

\newcommand{\overridebibliographypath}[1]{\providecommand{\bibliographypath}{#1}}
\overridebibliographypath{../bibliography.bib}
\addbibresource{\bibliographypath}
\defbibheading{bibliography}[\bibname]{%
	\newpage
	\section*{#1}%
}
\renewbibmacro*{entryhead:full}{\printfield{labeltitle}}
\DeclareFieldFormat{url}{\newline\footnotesize\url{#1}}
\AtEndDocument{\printbibliography}

% Police :

\setmathfont{Erewhon Math}

% Tikz :

\usetikzlibrary{calc}

% Longueurs :

\setlength{\parindent}{0pt}
\setlength{\headheight}{15pt}
\setlength{\fboxsep}{0pt}
\titlespacing*{\chapter}{0pt}{-20pt}{10pt}
\setlength{\marginparwidth}{1.5cm}
\setstretch{1.1}

% Métadonnées :

\author{agreg.skyost.eu}
\date{\today}

% Titres :

\setcounter{secnumdepth}{3}

\renewcommand{\thechapter}{\Roman{chapter}}
\renewcommand{\thesubsection}{\Roman{subsection}}
\renewcommand{\thesubsubsection}{\arabic{subsubsection}}
\renewcommand{\theparagraph}{\alph{paragraph}}

\titleformat{\chapter}{\huge\bfseries}{\thechapter}{20pt}{\huge\bfseries}
\titleformat*{\section}{\LARGE\bfseries}
\titleformat{\subsection}{\Large\bfseries}{\thesubsection \, - \,}{0pt}{\Large\bfseries}
\titleformat{\subsubsection}{\large\bfseries}{\thesubsubsection. \,}{0pt}{\large\bfseries}
\titleformat{\paragraph}{\bfseries}{\theparagraph. \,}{0pt}{\bfseries}

\setcounter{secnumdepth}{4}

% Table des matières :

\renewcommand{\cftsecleader}{\cftdotfill{\cftdotsep}}
\addtolength{\cftsecnumwidth}{10pt}

% Redéfinition des commandes :

\renewcommand*\thesection{\arabic{section}}
\renewcommand{\ker}{\mathrm{Ker}}

% Nouvelles commandes :

\newcommand{\website}{https://agreg.skyost.eu}

\newcommand{\tr}[1]{\mathstrut ^t #1}
\newcommand{\im}{\mathrm{Im}}
\newcommand{\rang}{\operatorname{rang}}
\newcommand{\trace}{\operatorname{trace}}
\newcommand{\id}{\operatorname{id}}
\newcommand{\stab}{\operatorname{Stab}}

\providecommand{\newpar}{\\[\medskipamount]}

\providecommand{\lesson}[3]{%
	\title{#3}%
	\hypersetup{pdftitle={#3}}%
	\setcounter{section}{\numexpr #2 - 1}%
	\section{#3}%
	\fancyhead[R]{\truncate{0.73\textwidth}{#2 : #3}}%
}

\providecommand{\development}[3]{%
	\title{#3}%
	\hypersetup{pdftitle={#3}}%
	\section*{#3}%
	\fancyhead[R]{\truncate{0.73\textwidth}{#3}}%
}

\providecommand{\summary}[1]{%
	\textit{#1}%
	\medskip%
}

\tikzset{notestyleraw/.append style={inner sep=0pt, rounded corners=0pt, align=center}}

%\newcommand{\booklink}[1]{\website/bibliographie\##1}
\newcommand{\citelink}[2]{\hyperlink{cite.\therefsection @#1}{#2}}
\newcommand{\previousreference}{}
\providecommand{\reference}[2][]{%
	\notblank{#1}{\renewcommand{\previousreference}{#1}}{}%
	\todo[noline]{%
		\protect\vspace{16pt}%
		\protect\par%
		\protect\notblank{#1}{\cite{[\previousreference]}\\}{}%
		\protect\citelink{\previousreference}{p. #2}%
	}%
}

\definecolor{devcolor}{HTML}{00695c}
\newcommand{\dev}[1]{%
	\reversemarginpar%
	\todo[noline]{
		\protect\vspace{16pt}%
		\protect\par%
		\bfseries\color{devcolor}\href{\website/developpements/#1}{DEV}
	}%
	\normalmarginpar%
}

% En-têtes :

\pagestyle{fancy}
\fancyhead[L]{\truncate{0.23\textwidth}{\thepage}}
\fancyfoot[C]{\scriptsize \href{\website}{\texttt{agreg.skyost.eu}}}

% Couleurs :

\definecolor{property}{HTML}{fffde7}
\definecolor{proposition}{HTML}{fff8e1}
\definecolor{lemma}{HTML}{fff3e0}
\definecolor{theorem}{HTML}{fce4f2}
\definecolor{corollary}{HTML}{ffebee}
\definecolor{definition}{HTML}{ede7f6}
\definecolor{notation}{HTML}{f3e5f5}
\definecolor{example}{HTML}{e0f7fa}
\definecolor{cexample}{HTML}{efebe9}
\definecolor{application}{HTML}{e0f2f1}
\definecolor{remark}{HTML}{e8f5e9}
\definecolor{proof}{HTML}{e1f5fe}

% Théorèmes :

\theoremstyle{definition}
\newtheorem{theorem}{Théorème}

\newtheorem{property}[theorem]{Propriété}
\newtheorem{proposition}[theorem]{Proposition}
\newtheorem{lemma}[theorem]{Lemme}
\newtheorem{corollary}[theorem]{Corollaire}

\newtheorem{definition}[theorem]{Définition}
\newtheorem{notation}[theorem]{Notation}

\newtheorem{example}[theorem]{Exemple}
\newtheorem{cexample}[theorem]{Contre-exemple}
\newtheorem{application}[theorem]{Application}

\theoremstyle{remark}
\newtheorem{remark}[theorem]{Remarque}

\counterwithin*{theorem}{section}

\newcommand{\applystyletotheorem}[1]{
	\tcolorboxenvironment{#1}{
		enhanced,
		breakable,
		colback=#1!98!white,
		boxrule=0pt,
		boxsep=0pt,
		left=8pt,
		right=8pt,
		top=8pt,
		bottom=8pt,
		sharp corners,
		after=\par,
	}
}

\applystyletotheorem{property}
\applystyletotheorem{proposition}
\applystyletotheorem{lemma}
\applystyletotheorem{theorem}
\applystyletotheorem{corollary}
\applystyletotheorem{definition}
\applystyletotheorem{notation}
\applystyletotheorem{example}
\applystyletotheorem{cexample}
\applystyletotheorem{application}
\applystyletotheorem{remark}
\applystyletotheorem{proof}

% Environnements :

\NewEnviron{whitetabularx}[1]{%
	\renewcommand{\arraystretch}{2.5}
	\colorbox{white}{%
		\begin{tabularx}{\textwidth}{#1}%
			\BODY%
		\end{tabularx}%
	}%
}

% Maths :

\DeclareFontEncoding{FMS}{}{}
\DeclareFontSubstitution{FMS}{futm}{m}{n}
\DeclareFontEncoding{FMX}{}{}
\DeclareFontSubstitution{FMX}{futm}{m}{n}
\DeclareSymbolFont{fouriersymbols}{FMS}{futm}{m}{n}
\DeclareSymbolFont{fourierlargesymbols}{FMX}{futm}{m}{n}
\DeclareMathDelimiter{\VERT}{\mathord}{fouriersymbols}{152}{fourierlargesymbols}{147}


% Bibliographie :

\addbibresource{\bibliographypath}%
\defbibheading{bibliography}[\bibname]{%
	\newpage
	\section*{#1}%
}
\renewbibmacro*{entryhead:full}{\printfield{labeltitle}}%
\DeclareFieldFormat{url}{\newline\footnotesize\url{#1}}%

\AtEndDocument{\printbibliography}

\begin{document}
	%<*content>
	\lesson{analysis}{209}{Approximation d'une fonction par des fonctions régulières. Exemples d'applications.}

	Dans toute la suite, $\mathbb{K}$ désignera le corps $\mathbb{R}$ ou $\mathbb{C}$.

	\subsection{Approximation par des polynômes}

	\subsubsection{Approximation locale}

	\reference[GOU20]{75}

	\begin{theorem}[Formule de Taylor-Lagrange]
		\label{209-1}
		Soit $f$ une fonction réelle de classe $\mathcal{C}^n$ sur un intervalle $[a,b]$ telle que $f^{(n+1)}$ existe sur un intervalle $]a,b[$. Alors,
		\[ \exists c \in ]a,b[ \text{ tel que } f(b) = \underbrace{\sum_{k=0}^{n} \frac{f^{(k)} (a)}{n!} (b-a)^n}_{= T_n(b)} + \frac{f^{(n+1)}(c)}{(n+1)!} (b-a)^{n+1} \]
	\end{theorem}

	\begin{application}
		\begin{itemize}
			\item $\forall x \in \mathbb{R}^+, \, x - \frac{x^2}{2} \leq \ln(1+x) \leq x - \frac{x^2}{2} + \frac{x^3}{3}$.
			\item $\forall x \in \mathbb{R}^+, \, x - \frac{x^3}{6} \leq \sin(x) \leq x - \frac{x^3}{6} + \frac{x^5}{120}$.
			\item $\forall x \in \mathbb{R}, \, 1 - \frac{x^2}{2} \leq \cos(x) \leq 1 - \frac{x^2}{2} + \frac{x^4}{24}$.
		\end{itemize}
	\end{application}

	\begin{proposition}
		En reprenant les notations du \cref{209-1}, on a
		\[ \Vert \exp - T_n \Vert_\infty \longrightarrow 0 \]
		sur $[a,b]$.
	\end{proposition}

	\subsubsection{Approximation sur un compact}

	\reference{238}

	\begin{theorem}[Théorèmes de Dini]
		\begin{enumerate}[label=(\roman*)]
			\item Soit $(f_n)$ une suite \textit{croissante} de fonctions réelles \textit{continues} définies sur un segment $I$ de $\mathbb{R}$. Si $(f_n)$ converge simplement vers une fonction \textit{continue} sur $I$, alors la convergence est uniforme.
			\item Soit $(f_n)$ une suite de \textit{fonctions croissantes} réelles \textit{continues} définies sur un segment $I$ de $\mathbb{R}$. Si $(f_n)$ converge simplement vers une fonction \textit{continue} sur $I$, alors la convergence est uniforme.
		\end{enumerate}
	\end{theorem}

	\reference{242}

	\begin{theorem}[Bernstein]
		Soit $f : [0,1] \rightarrow \mathbb{C}$ continue. On note
		\[ B_n(f) : x \mapsto \sum_{k=0}^{n} f\left(\frac{k}{n}\right) \binom{n}{k} x^k (1-x)^{n-k} \]
		Alors,
		\[ \Vert B_n(f) - f \Vert_\infty \longrightarrow_{n \rightarrow +\infty} 0 \]
	\end{theorem}

	\reference{304}
	\dev{theoreme-de-weierstrass-par-la-convolution}

	\begin{corollary}[Weierstrass]
		Toute fonction continue $f : [a,b] \rightarrow \mathbb{R}$ (avec $a, b \in \mathbb{R}$ tels que $a \leq b$) est limite uniforme de fonctions polynômiales sur $[a, b]$.
	\end{corollary}

	On a une version plus générale de ce théorème.

	\reference[LI]{46}

	\begin{theorem}[Stone-Weierstrass]
		Soit $K$ un espace compact et $\mathcal{A}$ une sous-algèbre de l'algèbre de Banach réelle $\mathcal{C}(K, \mathbb{R})$. On suppose de plus que :
		\begin{enumerate}[label=(\roman*)]
			\item $\mathcal{A}$ sépare les points de $K$ (ie. $\forall x \in K, \exists f \in A \text{ telle que } f(x) \neq f(y)$).
			\item $\mathcal{A}$ contient les constantes.
		\end{enumerate}
		Alors $\mathcal{A}$ est dense dans $\mathcal{C}(K, \mathbb{R})$.
	\end{theorem}

	\begin{remark}
		Il existe aussi une version ``complexe'' de ce théorème, où il faut supposer de plus que $\mathcal{A}$ est stable par conjugaison.
	\end{remark}

	\begin{example}
		La suite de polynômes réels $(r_n)$ définie par récurrence par
		\[ r_0 = 0 \text{ et } \forall n \in \mathbb{N}, r_{n+1} : t \mapsto r_n(t) + \frac{1}{2} (t - r_n(t)^2) \]
		converge vers $\sqrt{.}$.
	\end{example}

	\subsubsection{Interpolation}

	\reference[DEM]{21}

	Soit $f$ une fonction réelle continue sur un intervalle $[a,b]$. On se donne $n+1$ points $x_0, \dots, x_n \in [a,b]$ distincts deux-à-deux.

	\begin{definition}
		Pour $i \in \llbracket 0, n \rrbracket$, on définit le $i$-ième \textbf{polynôme de Lagrange} associé à $x_1, \dots, x_n$ par
		\[ \ell_i : x \mapsto \prod_{\substack{j=0\\j \neq i}}^n \frac{x-x_j}{x_i-x_j} \]
	\end{definition}

	\begin{theorem}
		Il existe une unique fonction polynômiale $p_n$ de degré $n$ telle que $\forall i \in \llbracket 0, n \rrbracket, \, p_n(x_i) = f(x_i)$ :
		\[ p_n = \sum_{i=0}^n f(x_i) \ell_i \]
	\end{theorem}

	\begin{theorem}
		On note $\pi_{n+1} : x \mapsto \prod_{j=0}^{n} (x-x_j)$ et on suppose $f$ $n+1$ fois dérivable $[a,b]$. Alors, pour tout $x \in [a,b]$, il existe un réel $\xi_x \in ]\min(x,x_i),\max(x,x_i)[$ tel que
		\[ f(x)-p_n(x) = \frac{\pi_{n+1}(x)}{(n+1)!} f^{(n+1)}(\xi_x) \]
	\end{theorem}

	\begin{corollary}
		\[ \Vert f - p_n \Vert_\infty \leq \frac{1}{(n+1)!} \Vert \pi_{n+1} \Vert_\infty \Vert f^{(n+1)} \Vert_\infty \]
	\end{corollary}

	\reference[DAN]{506}

	\begin{application}[Calculs approchés d'intégrales]
		On note $I(f) = \int_a^b f(t) \, \mathrm{d}t$. L'objectif est d'approximer $I(f)$ par une expression $P(f)$ et de majorer l'erreur d'approximation $E(f) = \vert I(f) - P(f) \vert$.
		\begin{enumerate}[label=(\roman*)]
			\item \underline{Méthode des rectangles.} On suppose $f$ continue. Avec $P(f) = (b-a)f(a)$, on a $E(f) \leq \frac{(b-a)^2}{2} \Vert f' \Vert_\infty$.
			\item \underline{Méthode du point milieu.} On suppose $f$ de classe $\mathcal{C}^2$. Avec $P(f) = (b-a)f \left( \frac{a+b}{2} \right)$, on a $E(f) \leq \frac{(b-a)^3}{24} \Vert f'' \Vert_\infty$.
			\item \underline{Méthode des trapèzes.} On suppose $f$ de classe $\mathcal{C}^2$. Avec $P(f) = \frac{b-a}{2} (f(a) + f(b))$, on a $E(f) \leq \frac{(b-a)^3}{12} \Vert f'' \Vert_\infty$.
			\item \underline{Méthode de Simpson.} On suppose $f$ de classe $\mathcal{C}^4$. Avec $P(f) = \frac{b-a}{6} \left(f(a) + f(b) + 4f \left( \frac{a+b}{2} \right)\right)$, on a $E(f) \leq \frac{(b-a)^3}{2880} \Vert f^{(4)} \Vert_\infty$.
		\end{enumerate}
	\end{application}

	\subsection{Approximation dans les espaces de Lebesgue}

	\subsubsection{Convolution}

	\reference[AMR08]{75}

	\begin{definition}
		Soient $f$ et $g$ deux fonctions de $\mathbb{R}^d$ dans $\mathbb{R}$. On dit que \textbf{la convolée} (ou \textbf{le produit de convolution}) de $f$ et $g$ en $x \in \mathbb{R}$ \textbf{existe} si la fonction
		\[
		\begin{array}{ccc}
			\mathbb{R} &\rightarrow& \mathbb{C} \\
			t &\mapsto& f(x-t)g(t)
		\end{array}
		\]
		est intégrable sur $\mathbb{R}^d$ pour la mesure de Lebesgue. On pose alors :
		\[ (f * g)(x) = \int_{\mathbb{R}^d} f(x-t)g(t) \, \mathrm{d}t \]
	\end{definition}

	\begin{proposition}
		Dans $L_1(\mathbb{R}^d)$, le produit de convolution est commutatif, bilinéaire et associatif.
	\end{proposition}

	\begin{theorem}
		Soient $p, q > 0$ et $f \in L_p(\mathbb{R}^d)$ et $g \in L_q(\mathbb{R}^d)$.
		\begin{enumerate}[label=(\roman*)]
			\item Si $p, q \in [1, +\infty]$ tels que $\frac{1}{p} + \frac{1}{q} = 1$, alors $(f * g)(x)$ existe \underline{pour tout} $x \in \mathbb{R}^d$ et est uniformément continue. On a, $\Vert f * g \Vert_\infty \leq \Vert f \Vert_p \Vert g \Vert_q$ et, si $p \neq 1, +\infty$, $f * g \in \mathcal{C}_0(\mathbb{R})$.
			\item Si $p = 1$ et $q = +\infty$, alors $(f * g)(x)$ existe \underline{pour tout} $x \in \mathbb{R}^d$ et $f * g \in \mathcal{C}_b(\mathbb{R})$.
			\item Si $p = 1$ et $q \in [1, +\infty[$, alors $(f * g)(x)$ existe \underline{pp.} en $x \in \mathbb{R}^d$ et $f * g \in L_q(\mathbb{R})$ telle que $\Vert f * g \Vert_q \leq \Vert f \Vert_1 \Vert g \Vert_q$.
			\item Si $p = 1$ et $q = 1$, alors $(f * g)(x)$ existe \underline{pp.} en $x \in \mathbb{R}^d$ et $f * g \in L_1(\mathbb{R})$ telle que $\Vert f * g \Vert_1 \leq \Vert f \Vert_1 \Vert g \Vert_1$.
		\end{enumerate}
	\end{theorem}

	\begin{example}
		Soient $a < b \in \mathbb{R}^+_*$. Alors $\mathbb{1}_{[-a, a]} * \mathbb{1}_{[-b,b]}$ existe pour tout $x \in \mathbb{R}$ et
		\[ \left( \mathbb{1}_{[-a, a]} * \mathbb{1}_{[-b,b]} \right)(x) =
		\begin{cases}
			2a &\text{si } 0 \leq \vert x \vert \leq b-a \\
			b+a-\vert x \vert &\text{si } b-a \leq \vert x \vert \leq b+a \\
			0 &\text{sinon}
		\end{cases}
		\]
	\end{example}

	\reference{85}

	\begin{proposition}
		$L_1(\mathbb{R}^d)$ est une algèbre de Banach pour le produit de convolution.
	\end{proposition}

	\begin{remark}
		Cette algèbre n'a pas d'élément neutre. Afin de pallier à ce manque, nous allons voir la notion d'approximation de l'identité dans la sous-section suivante.
	\end{remark}

	\subsubsection{Densité}

	\reference[B-P]{306}

	\begin{definition}
		On appelle \textbf{approximation de l'identité} toute suite $(\rho_n)$ de fonctions mesurables de $L_1(\mathbb{R}^d)$ telles que
		\begin{enumerate}[label=(\roman*)]
			\item $\forall n \in \mathbb{N}, \, \int_{\mathbb{R}^d} \rho_n \, \mathrm{d}\lambda_d = 1$.
			\item $\sup_{n \geq 1} \Vert \rho_n \Vert < +\infty$.
			\item $\forall \epsilon > 0, \, \lim_{n \rightarrow +\infty} \int_{\mathbb{R} \setminus B(0, \epsilon)} \rho_n(x) \, \mathrm{d}x = 0$.
		\end{enumerate}
	\end{definition}

	\begin{remark}
		Dans la définition précédente, (ii) implique (i) lorsque les fonctions $\rho_n$ sont positives. Plutôt que des suites, on pourra considérer les familles indexées par $\mathbb{R}_*^+$.
	\end{remark}

	\begin{example}
		\begin{itemize}
			\item Noyau de Laplace sur $\mathbb{R}$ :
			\[ \forall t > 0, \, \rho_t(x) = \frac{1}{2t}e^{-\frac{|x|}{t}} \]
			\item Noyau de Cauchy sur $\mathbb{R}$ :
			\[ \forall t > 0, \, \rho_t(x) = \frac{t}{\pi (t^2 + x^2)} \]
			\item Noyau de Gauss sur $\mathbb{R}$ :
			\[ \forall t > 0, \, \rho_t(x) = \frac{1}{\sqrt{2\pi} t}e^{-\frac{|x|^2}{2t^2}} \]
		\end{itemize}
	\end{example}

	\reference{307}

	\begin{theorem}
		Soit $(\rho_n)$ une approximation de l'identité. Soient $p \in [1, +\infty[$ et $f \in L_p(\mathbb{R}^d)$, alors :
		\[ \forall n \geq 1, \, f * \rho_n \in L_p(\mathbb{R}^d) \quad \text{ et } \quad \Vert f * \rho_n - f \Vert_p \longrightarrow 0 \]
	\end{theorem}

	\begin{theorem}
		Soient $(\rho_n)$ une approximation de l'identité et $f \in L_\infty(\mathbb{R}^d)$. Alors :
		\begin{itemize}
			\item Si $f$ est continue en $x_0 \in \mathbb{R}^d$, alors $(f * \rho_n)(x_0) \longrightarrow_{n \rightarrow +\infty} f(x_0)$.
			\item Si $f$ est uniformément continue sur $\mathbb{R}^d$, alors $\Vert f * \rho_n - f \Vert_\infty \longrightarrow_{n \rightarrow +\infty} 0$.
			\item Si $f$ est continue sur un compact $K$, alors $\sup_{x \in K} |(f * \rho_n)(x) - f(x)| \longrightarrow_{n \rightarrow +\infty} 0$.
		\end{itemize}
	\end{theorem}

	\begin{definition}
		On qualifie de \textbf{régularisante} toute suite $(\alpha_n)$ d'approximations de l'identité telle que $\forall n \in \mathbb{N}, \, \alpha_n \in \mathcal{C}^\infty_K(\mathbb{R}^d)$.
	\end{definition}

	\reference{274}

	\begin{example}
		Soit $\alpha \in \mathcal{C}^\infty_K(\mathbb{R}^d)$ une densité de probabilité. Alors la suite $(\alpha_n)$ définie pour tout $n \in \mathbb{N}$ par $\alpha_n : x \mapsto n \alpha(nx)$ est régularisante.
	\end{example}

	\reference[AMR08]{96}

	\begin{application}
		\begin{enumerate}[label=(\roman*)]
			\item $\mathcal{C}^\infty_K(\mathbb{R}^d)$ est dense dans $\mathcal{C}_K(\mathbb{R}^d)$ pour $\Vert . \Vert_\infty$.
			\item $\mathcal{C}^\infty_K(\mathbb{R}^d)$ est dense dans $L_p(\mathbb{R}^d)$ pour $\Vert . \Vert_p$ avec $p \in [1, +\infty[$.
		\end{enumerate}
	\end{application}

	\subsection{Approximations de fonctions périodiques}

	\subsubsection{Séries de Fourier}

	\reference[Z-Q]{73}

	\begin{notation}
		\begin{itemize}
			\item Pour tout $p \in [1, +\infty]$, on note $L_p^{2\pi}$ l'espace des fonctions $f : \mathbb{R} \rightarrow \mathbb{C}$, $2\pi$-périodiques et mesurables, telles que $\Vert f \Vert_p < +\infty$.
			\item Pour tout $n \in \mathbb{Z}$, on note $e_n$ la fonction $2\pi$-périodique définie pour tout $t \in \mathbb{R}$ par $e_n(t) = e^{int}$.
		\end{itemize}
	\end{notation}

	\begin{proposition}
		$L_2^{2\pi}$ est un espace de Hilbert pour le produit scalaire
		\[ \langle ., . \rangle : (f, g) \mapsto \frac{1}{2 \pi} \int_0^{2\pi} f(t) \overline{g(t)} \, \mathrm{d}t \]
	\end{proposition}

	\reference[GOU20]{268}

	\begin{definition}
		Soit $f \in L_1^{2\pi}$. On appelle :
		\begin{itemize}
			\item \textbf{Coefficients de Fourier complexes}, les complexes définis par
			\[ \forall n \in \mathbb{Z}, \, c_n(f) = \frac{1}{2 \pi} \int_0^{2\pi} f(t) e^{-int} \, \mathrm{d}t = \langle f, e_n \rangle \]
			\item \textbf{Coefficients de Fourier réels}, les complexes définis par
			\[ \forall n \in \mathbb{N}, \, a_n(f) = \frac{1}{\pi} \int_0^{2\pi} f(t) \cos(nt) \, \mathrm{d}t \text{ et } \forall n \in \mathbb{N}^*, \, b_n(f) = \frac{1}{\pi} \int_0^{2\pi} f(t) \sin(nt) \, \mathrm{d}t \]
		\end{itemize}
	\end{definition}

	\subsubsection{Approximation hilbertienne}

	\reference{109}

	\begin{theorem}
		\label{246-2}
		Soit $H$ un espace de Hilbert et $(\epsilon_n)_{n \in I}$ une famille orthonormée dénombrable de $H$. Les propriétés suivantes sont équivalentes :
		\begin{enumerate}[label=(\roman*)]
			\item La famille orthonormée $(\epsilon_n)_{n \in I}$ est une base hilbertienne de $H$.
			\item $\forall x \in H, \, x = \sum_{n=0}^{+\infty} \langle x, \epsilon_n \rangle \epsilon_n$.
			\item \label{246-3} $\forall x \in H, \, \Vert x \Vert_2 = \sum_{n=0}^{+\infty} \vert \langle x, \epsilon_n \rangle \vert^2$.
		\end{enumerate}
	\end{theorem}

	\begin{remark}
		L'égalité du \cref{246-2} \cref{246-3} est appelée \textbf{égalité de Parseval}.
	\end{remark}

	\reference{123}

	\begin{theorem}
		La famille $(e_n)_{n \in \mathbb{Z}}$ est une base hilbertienne de $L_2^{2 \pi}$.
	\end{theorem}

	\begin{corollary}
		\label{246-4}
		\[ \forall f \in L_2^{2 \pi}, \, f = \sum_{n = -\infty}^{+\infty} c_n(f) e_n \]
	\end{corollary}

	\reference[GOU20]{272}

	\begin{example}
		\label{246-5}
		On considère $f : x \mapsto 1 - \frac{x^2}{\pi^2}$ sur $[-\pi, \pi]$. Alors,
		\[ \frac{\pi^4}{90} = \Vert f \Vert_2 = \sum_{n=0}^{+\infty} \frac{1}{n^4} \]
	\end{example}

	\reference[BMP]{124}

	\begin{remark}
		L'égalité du \cref{246-4} est valable dans $L_2^{2\pi}$, elle signifie donc que
		\[ \left\Vert \sum_{n = -N}^{N} c_n(f) e_n - f \right\Vert_2 \longrightarrow_{N \rightarrow +\infty} 0 \]
	\end{remark}

	\subsubsection{Approximation au sens de Cesàro}

	\reference[GOU20]{269}

	\begin{definition}
		Soit $f \in L_1^{2\pi}$. On appelle \textbf{série de Fourier} associée à $f$ la série $(S_N(f))$ définie par
		\[ \forall N \in \mathbb{N}, \, S_N(f) = \sum_{n=-N}^{N} c_n(f) e_n \overset{(*)}{=} \frac{a_0(f)}{2} + \sum_{n = 1}^N (a_n(f) \cos(nx) + b_n(f) \sin(nx)) \]
	\end{definition}

	\begin{remark}
		L'égalité $(*)$ de la définition précédente est justifiée car,
		\[ \forall n \in \mathbb{N}^*, \, \forall x \in \mathbb{R}, \, c_n(f) e^{inx} + c_{-n}(f) e^{-inx} = a_n(f) \cos(nx) + b_n(f) \sin(nx) \]
	\end{remark}

	\reference[AMR08]{184}

	\begin{definition}
		Pour tout $N \in \mathbb{N}$, la fonction $D_N = \sum_{n=-N}^{N} e_N$ est appelé \textbf{noyau de Dirichlet} d'ordre $N$.
	\end{definition}

	\begin{proposition}
		Soit $N \in \mathbb{N}$.
		\begin{enumerate}[label=(\roman*)]
			\item $D_N$ est une fonction paire, $2\pi$-périodique, et de norme $1$.
			\item \[ \forall x \in \mathbb{R} \setminus 2 \pi \mathbb{Z}, \, D_N(x) = \frac{\sin \left(\left( N + \frac{1}{2} \right) x \right)}{\sin \left( \frac{x}{2} \right)} \]
			\item Pour tout $f \in L_1^{2 \pi}, \, S_N(f) = f * D_N$.
		\end{enumerate}
	\end{proposition}

	\begin{definition}
		Pour tout $N \in \mathbb{N}$, la fonction $K_N = \frac{1}{N} \sum_{j=0}^{N-1} D_j$ est appelé \textbf{noyau de Fejér} d'ordre $N$.
	\end{definition}

	\begin{notation}
		Pour tout $N \in \mathbb{N}^*$, on note $\sigma_N = \frac{1}{N} \sum_{k=0}^{N-1} S_n(f)$ la somme de Cesàro d'ordre $N$ de la série de Fourier d'une fonction $f \in L_1^{2 \pi}$.
	\end{notation}

	\begin{proposition}
		Soient $N \in \mathbb{N}^*$ et $f \in L_1^{2 \pi}$.
		\begin{enumerate}[label=(\roman*)]
			\item $K_N$ est une fonction positive et de norme $1$.
			\item \[ \forall x \in \mathbb{R} \setminus 2 \pi \mathbb{Z}, \, K_N(x) = \frac{1}{N} \left(\frac{\sin \left( \frac{Nx}{2} \right)}{\sin \left( \frac{x}{2} \right)}\right)^2 \]
			\item $K_N = \sum_{n=-N}^{N} \left(1 - \frac{\vert n \vert}{N}\right) e_n$.
			\item $\sigma_N(f) = f * K_N$.
		\end{enumerate}
	\end{proposition}

	\reference{190}
	\dev{theoreme-de-fejer}

	\begin{theorem}[Fejér]
		Soit $f : \mathbb{R} \rightarrow \mathbb{C}$ une fonction $2\pi$-périodique.
		\begin{enumerate}[label=(\roman*)]
			\item Si $f$ est continue, alors $\Vert \sigma_N(f) \Vert_\infty \leq \Vert f \Vert_\infty$ et $(\sigma_N(f))$ converge uniformément vers $f$.
			\item Si $f \in L_p^{2\pi}$ pour $p \in [1,+\infty[$, alors $\Vert \sigma_N(f) \Vert_p \leq \Vert f \Vert_p$ et $(\sigma_N(f))$ converge vers $f$ pour $\Vert . \Vert_p$.
		\end{enumerate}
	\end{theorem}

	\begin{corollary}
		L'espace des polynômes trigonométriques $\{ \sum_{n=-N}^N c_n e_n \mid (c_n) \in \mathbb{C}^{\mathbb{N}}, \, N \in \mathbb{N} \}$ est dense dans l'espace des fonction continues $2\pi$-périodiques pour $\Vert . \Vert_\infty$ et est dense dans $L_p^{2\pi}$ pour $\Vert . \Vert_p$ avec $p \in [1,+\infty[$.
	\end{corollary}
	%</content>
\end{document}
