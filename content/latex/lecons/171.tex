\documentclass[12pt, a4paper]{report}

% LuaLaTeX :

\RequirePackage{iftex}
\RequireLuaTeX

% Packages :

\usepackage[french]{babel}
%\usepackage[utf8]{inputenc}
%\usepackage[T1]{fontenc}
\usepackage[pdfencoding=auto, pdfauthor={Hugo Delaunay}, pdfsubject={Mathématiques}, pdfcreator={agreg.skyost.eu}]{hyperref}
\usepackage{amsmath}
\usepackage{amsthm}
%\usepackage{amssymb}
\usepackage{stmaryrd}
\usepackage{tikz}
\usepackage{tkz-euclide}
\usepackage{fourier-otf}
\usepackage{fontspec}
\usepackage{titlesec}
\usepackage{fancyhdr}
\usepackage{catchfilebetweentags}
\usepackage[french, capitalise, noabbrev]{cleveref}
\usepackage[fit, breakall]{truncate}
\usepackage[top=2.5cm, right=2cm, bottom=2.5cm, left=2cm]{geometry}
\usepackage{enumerate}
\usepackage{tocloft}
\usepackage{microtype}
%\usepackage{mdframed}
%\usepackage{thmtools}
\usepackage{xcolor}
\usepackage{tabularx}
\usepackage{aligned-overset}
\usepackage[subpreambles=true]{standalone}
\usepackage{environ}
\usepackage[normalem]{ulem}
\usepackage{marginnote}
\usepackage{etoolbox}
\usepackage{setspace}
\usepackage[bibstyle=reading, citestyle=draft]{biblatex}
\usepackage{xpatch}
\usepackage[many, breakable]{tcolorbox}
\usepackage[backgroundcolor=white, bordercolor=white, textsize=small]{todonotes}

% Bibliographie :

\newcommand{\overridebibliographypath}[1]{\providecommand{\bibliographypath}{#1}}
\overridebibliographypath{../bibliography.bib}
\addbibresource{\bibliographypath}
\defbibheading{bibliography}[\bibname]{%
	\newpage
	\section*{#1}%
}
\renewbibmacro*{entryhead:full}{\printfield{labeltitle}}
\DeclareFieldFormat{url}{\newline\footnotesize\url{#1}}
\AtEndDocument{\printbibliography}

% Police :

\setmathfont{Erewhon Math}

% Tikz :

\usetikzlibrary{calc}

% Longueurs :

\setlength{\parindent}{0pt}
\setlength{\headheight}{15pt}
\setlength{\fboxsep}{0pt}
\titlespacing*{\chapter}{0pt}{-20pt}{10pt}
\setlength{\marginparwidth}{1.5cm}
\setstretch{1.1}

% Métadonnées :

\author{agreg.skyost.eu}
\date{\today}

% Titres :

\setcounter{secnumdepth}{3}

\renewcommand{\thechapter}{\Roman{chapter}}
\renewcommand{\thesubsection}{\Roman{subsection}}
\renewcommand{\thesubsubsection}{\arabic{subsubsection}}
\renewcommand{\theparagraph}{\alph{paragraph}}

\titleformat{\chapter}{\huge\bfseries}{\thechapter}{20pt}{\huge\bfseries}
\titleformat*{\section}{\LARGE\bfseries}
\titleformat{\subsection}{\Large\bfseries}{\thesubsection \, - \,}{0pt}{\Large\bfseries}
\titleformat{\subsubsection}{\large\bfseries}{\thesubsubsection. \,}{0pt}{\large\bfseries}
\titleformat{\paragraph}{\bfseries}{\theparagraph. \,}{0pt}{\bfseries}

\setcounter{secnumdepth}{4}

% Table des matières :

\renewcommand{\cftsecleader}{\cftdotfill{\cftdotsep}}
\addtolength{\cftsecnumwidth}{10pt}

% Redéfinition des commandes :

\renewcommand*\thesection{\arabic{section}}
\renewcommand{\ker}{\mathrm{Ker}}

% Nouvelles commandes :

\newcommand{\website}{https://agreg.skyost.eu}

\newcommand{\tr}[1]{\mathstrut ^t #1}
\newcommand{\im}{\mathrm{Im}}
\newcommand{\rang}{\operatorname{rang}}
\newcommand{\trace}{\operatorname{trace}}
\newcommand{\id}{\operatorname{id}}
\newcommand{\stab}{\operatorname{Stab}}

\providecommand{\newpar}{\\[\medskipamount]}

\providecommand{\lesson}[3]{%
	\title{#3}%
	\hypersetup{pdftitle={#3}}%
	\setcounter{section}{\numexpr #2 - 1}%
	\section{#3}%
	\fancyhead[R]{\truncate{0.73\textwidth}{#2 : #3}}%
}

\providecommand{\development}[3]{%
	\title{#3}%
	\hypersetup{pdftitle={#3}}%
	\section*{#3}%
	\fancyhead[R]{\truncate{0.73\textwidth}{#3}}%
}

\providecommand{\summary}[1]{%
	\textit{#1}%
	\medskip%
}

\tikzset{notestyleraw/.append style={inner sep=0pt, rounded corners=0pt, align=center}}

%\newcommand{\booklink}[1]{\website/bibliographie\##1}
\newcommand{\citelink}[2]{\hyperlink{cite.\therefsection @#1}{#2}}
\newcommand{\previousreference}{}
\providecommand{\reference}[2][]{%
	\notblank{#1}{\renewcommand{\previousreference}{#1}}{}%
	\todo[noline]{%
		\protect\vspace{16pt}%
		\protect\par%
		\protect\notblank{#1}{\cite{[\previousreference]}\\}{}%
		\protect\citelink{\previousreference}{p. #2}%
	}%
}

\definecolor{devcolor}{HTML}{00695c}
\newcommand{\dev}[1]{%
	\reversemarginpar%
	\todo[noline]{
		\protect\vspace{16pt}%
		\protect\par%
		\bfseries\color{devcolor}\href{\website/developpements/#1}{DEV}
	}%
	\normalmarginpar%
}

% En-têtes :

\pagestyle{fancy}
\fancyhead[L]{\truncate{0.23\textwidth}{\thepage}}
\fancyfoot[C]{\scriptsize \href{\website}{\texttt{agreg.skyost.eu}}}

% Couleurs :

\definecolor{property}{HTML}{fffde7}
\definecolor{proposition}{HTML}{fff8e1}
\definecolor{lemma}{HTML}{fff3e0}
\definecolor{theorem}{HTML}{fce4f2}
\definecolor{corollary}{HTML}{ffebee}
\definecolor{definition}{HTML}{ede7f6}
\definecolor{notation}{HTML}{f3e5f5}
\definecolor{example}{HTML}{e0f7fa}
\definecolor{cexample}{HTML}{efebe9}
\definecolor{application}{HTML}{e0f2f1}
\definecolor{remark}{HTML}{e8f5e9}
\definecolor{proof}{HTML}{e1f5fe}

% Théorèmes :

\theoremstyle{definition}
\newtheorem{theorem}{Théorème}

\newtheorem{property}[theorem]{Propriété}
\newtheorem{proposition}[theorem]{Proposition}
\newtheorem{lemma}[theorem]{Lemme}
\newtheorem{corollary}[theorem]{Corollaire}

\newtheorem{definition}[theorem]{Définition}
\newtheorem{notation}[theorem]{Notation}

\newtheorem{example}[theorem]{Exemple}
\newtheorem{cexample}[theorem]{Contre-exemple}
\newtheorem{application}[theorem]{Application}

\theoremstyle{remark}
\newtheorem{remark}[theorem]{Remarque}

\counterwithin*{theorem}{section}

\newcommand{\applystyletotheorem}[1]{
	\tcolorboxenvironment{#1}{
		enhanced,
		breakable,
		colback=#1!98!white,
		boxrule=0pt,
		boxsep=0pt,
		left=8pt,
		right=8pt,
		top=8pt,
		bottom=8pt,
		sharp corners,
		after=\par,
	}
}

\applystyletotheorem{property}
\applystyletotheorem{proposition}
\applystyletotheorem{lemma}
\applystyletotheorem{theorem}
\applystyletotheorem{corollary}
\applystyletotheorem{definition}
\applystyletotheorem{notation}
\applystyletotheorem{example}
\applystyletotheorem{cexample}
\applystyletotheorem{application}
\applystyletotheorem{remark}
\applystyletotheorem{proof}

% Environnements :

\NewEnviron{whitetabularx}[1]{%
	\renewcommand{\arraystretch}{2.5}
	\colorbox{white}{%
		\begin{tabularx}{\textwidth}{#1}%
			\BODY%
		\end{tabularx}%
	}%
}

% Maths :

\DeclareFontEncoding{FMS}{}{}
\DeclareFontSubstitution{FMS}{futm}{m}{n}
\DeclareFontEncoding{FMX}{}{}
\DeclareFontSubstitution{FMX}{futm}{m}{n}
\DeclareSymbolFont{fouriersymbols}{FMS}{futm}{m}{n}
\DeclareSymbolFont{fourierlargesymbols}{FMX}{futm}{m}{n}
\DeclareMathDelimiter{\VERT}{\mathord}{fouriersymbols}{152}{fourierlargesymbols}{147}


% Bibliographie :

\addbibresource{\bibliographypath}%
\defbibheading{bibliography}[\bibname]{%
	\newpage
	\section*{#1}%
}
\renewbibmacro*{entryhead:full}{\printfield{labeltitle}}%
\DeclareFieldFormat{url}{\newline\footnotesize\url{#1}}%

\AtEndDocument{\printbibliography}

\begin{document}
  %<*content>
  \lesson{algebra}{171}{Formes quadratiques réelles. Coniques. Exemples et applications.}

  Soit $E$ un espace vectoriel sur $\mathbb{R}$ de dimension finie $n$.

  \subsection{Formes quadratiques réelles}

  \subsubsection{Définitions}

  \reference[GOU21]{239}

  \begin{definition}
    Soit $\varphi : E \times E \rightarrow \mathbb{K}$ une application.
    \begin{itemize}
      \item On dit que $\varphi$ est une \textbf{forme bilinéaire} sur $E$ si pour tout $x \in E$, $y \mapsto \varphi(x, y)$ et pour tout $y \in E$, $x \mapsto \varphi(x, y)$ sont linéaires.
      \item Si de plus $\varphi(x, y) = \varphi(y, x)$ pour tout $x, y \in E$, on dit que $\varphi$ est \textbf{symétrique}.
    \end{itemize}
  \end{definition}

  \begin{definition}
    On appelle \textbf{forme quadratique} sur $E$ toute application $q$ de la forme
    \[ q : x \mapsto \varphi(x, x) \]
    où $\varphi$ est une forme bilinéaire sur $E$.
  \end{definition}

  \begin{example}
    \label{171-1}
    Sur $\mathbb{R}^3$, $(x, y, z) \mapsto 3x^2 + y^2 + 2xy - 3xz$ définit une forme quadratique.
  \end{example}

  \begin{proposition}
    Soit $q$ une forme quadratique sur $E$. Il existe une unique forme bilinéaire symétrique $\varphi$ telle que $q(x) = \varphi(x, x)$ pour tout $x \in E$. $\varphi$ est la \textbf{forme polaire} de $q$, et on a
    \[ \forall x, y \in E, \, \varphi(x, y) = \frac{1}{2} (q(x+y) - q(x) - q(y)) = \frac{1}{4} (q(x+y) - q(x-y))  \]
  \end{proposition}

  \reference{248}

  \begin{example}
    Sur $\mathcal{M}_n(\mathbb{R})$, $A \mapsto \trace(A)^2$ est une forme quadratique, dont la forme polaire est $(A, B) \mapsto \trace(A)\trace(B)$.
  \end{example}

  \subsubsection{Représentation matricielle}

  \reference{229}

  \begin{definition}
    Soient $q$ une forme quadratique sur $E$ et $\mathcal{B} = (e_1, \dots, e_n)$ une base de $E$. On appelle \textbf{matrice} de $q$ dans $\mathcal{B}$ la matrice $\operatorname{Mat}(q, \mathcal{B})$ définie par
    \[ \operatorname{Mat}(q, \mathcal{B}) = (\varphi(e_i, e_j))_{i, j \in \llbracket 1, n \rrbracket} \]
    où $\varphi$ est la forme polaire de $q$. Le \textbf{rang} de $q$ désigne le rang de cette matrice.
  \end{definition}

  \begin{example}
    La matrice de la forme quadratique de l'\cref{171-1} est
    \[
    \begin{pmatrix}
      3 & 1 & -\frac{3}{2} \\
      1 & 1 & 0 \\
      -\frac{3}{2} & 0 & 0
    \end{pmatrix}
    \]
  \end{example}

  \begin{proposition}
    Soient $\mathcal{B}$ et $\mathcal{B}'$ deux bases de $E$ dont on note $P$ la matrice de passage entre ces bases. Soit $q$ une forme quadratique sur $E$. Alors,
    \[ \operatorname{Mat}(q, \mathcal{B}) = \tr{P} \operatorname{Mat}(q, \mathcal{B}') P \]
  \end{proposition}

  \begin{remark}
    En particulier, en reprenant les notations précédentes, $\operatorname{Mat}(q, \mathcal{B})$ et $\operatorname{Mat}(q, \mathcal{B}')$ sont équivalentes : le rang de $q$ est bien défini et ne dépend pas de la base considérée.
  \end{remark}

  \subsection{Orthogonalité et isotropie}

  Soit $q$ une forme quadratique sur $E$ de forme polaire $\varphi$.

  \begin{definition}
    \begin{itemize}
      \item On appelle \textbf{cône isotrope} de $q$ l'ensemble
      \[ C_q = \{ x \in E \mid q(x) = 0 \} \]
      \item $q$ est dite \textbf{définie} si $C_q = \{ 0 \}$.
      \item Les vecteurs de $C_q$ sont dits \textbf{isotropes} pour $q$.
    \end{itemize}
  \end{definition}

  \reference[GRI]{303}

  \begin{example}
    La forme quadratique définie sur $\mathbb{R}^3$ par $(x, y, z) \mapsto 4x^2 + 3y^2 + 5xy - 3xz + 8yz$ n'est pas définie car $(0,0,1)$ est un vecteur isotrope non nul.
  \end{example}

  \reference[GOU21]{242}

  \begin{definition}
    \begin{itemize}
      \item Deux vecteurs $x, y \in E$ sont dits \textbf{$q$-orthogonaux} si $\varphi(x, y) = 0$. On note cela $x \perp y$.
      \item Si $A \subseteq E$, on appelle \textbf{orthogonal} de $A$ l'ensemble $A^\perp = \{ y \in E \mid \forall x \in A, \, x \perp y \}$.
    \end{itemize}
  \end{definition}

  \begin{proposition}
    \begin{enumerate}[label=(\roman*)]
      \item Si $A \subseteq E$, $A^\perp = (\operatorname{Vect}(A))^\perp$.
      \item Si $A \subseteq E$, $A \subseteq A^{\perp\perp}$.
      \item Si $A \subseteq B \subseteq E$, $B^{\perp} \subseteq A^{\perp}$.
    \end{enumerate}
  \end{proposition}

  \begin{definition}
    \begin{itemize}
      \item On appelle \textbf{noyau} de $q$ le sous-espace vectoriel
      \[ \ker(q) = E^{\perp} \]
      \item On dit que $q$ est \textbf{non-dégénérée} si $\ker(q) = \{ 0 \}$ et \textbf{dégénérée} si $\ker(q) \neq \{ 0 \}$.
    \end{itemize}
  \end{definition}

  \begin{proposition}
    On a $\ker(q) \subseteq C_q$. En particulier, si $q$ est définie, alors $q$ est non dégénérée.
  \end{proposition}

  \begin{example}
    Sur $\mathbb{R}^2$, $(x, y) \mapsto x^2 - y^2$ est une forme quadratique non dégénérée mais non définie non plus.
  \end{example}

  \begin{proposition}
    Soit $F$ un sous-espace vectoriel de $E$.
    \begin{enumerate}[label=(\roman*)]
      \item $\dim(E) = \dim(F) + \dim(F^\perp) - \dim(F \, \cap \, \ker(q))$.
      \item $F^{\perp\perp} = F + \ker(q)$.
      \item Si la restriction de $q$ à $F$ $q_{|F}$ est définie, alors $E = F \oplus F^{\perp}$.
      \item Si $q$ est définie, $F = F^{\perp\perp}$.
    \end{enumerate}
  \end{proposition}

  \begin{proposition}
    Soit $A$ la matrice de $q$ dans une base $\mathcal{B}$. Alors,
    \[ \ker(A) = \ker(q) \]
  \end{proposition}

  \reference[GRI]{296}

  \begin{corollary}
    $q$ est non dégénérée si et seulement si $\det(\operatorname{Mat}(q, \mathcal{B})) \neq 0$ pour une base quelconque $\mathcal{B}$ de $E$.
  \end{corollary}

  \begin{example}
    Sur $\mathbb{R}^4$, $(x, y) \mapsto x^2 + y^2 + z^2 - t^2$ est non dégénérée (car de déterminant $-1$).
  \end{example}

  \subsection{Classification}

  \subsubsection{Bases orthogonales}

  \reference[GOU21]{243}

  \begin{definition}
    Une base de $E$ est dite \textbf{$q$-orthogonale} si ses vecteurs sont deux à deux $q$-orthogonaux.
  \end{definition}

  \begin{remark}
    Si $(e_1, \dots, e_n)$ est une base $q$-orthogonale, alors
    \[ \forall (x_1, \dots, x_n) \in \mathbb{K}^n, \, q \left( \sum_{i=1}^n x_i e_i \right) = \sum_{i=1}^n x_i^2 q(e_i) \]
  \end{remark}

  \begin{theorem}
    Il existe une base $q$-orthogonale de $E$.
  \end{theorem}

  \begin{remark}
    Si $\mathcal{B} = (e_1, \dots, e_n)$ est une base $q$-orthogonale, en posant $\lambda_i = q(e_i)$ pour tout $i \in \llbracket 1, n \rrbracket$, on a
    \[ \forall x \in E, \, q(x) = q \left( \sum_{i=1}^n e_i^*(x) e_i \right) = \sum_{i=1}^n \lambda_i (e_i^*(x))^2 \]
    où $(e_1^*, \dots, e_n^*)$ est la base duale de $\mathcal{B}$.
  \end{remark}

  \subsubsection{Algorithme de Gauss}

  \begin{theorem}[Méthode de Gauss]
    On écrit
    \[ q(x_1, \dots, x_n) = \sum_{i=1}^n a_{i,i} x_i^2 + \sum_{1 \leq i<j \leq n} a_{i,j} x_ix_j \]
    et on cherche à écrire $q$ comme combinaison linéaire de carrés de formes linéaires indépendantes. On a deux cas :
    \begin{enumerate}[label=(\roman*)]
      \item \underline{Il existe $i \in \llbracket 1, n \rrbracket$ tel que $a_{i,i} \neq 0$.} On peut suppose $i = 1$, on pose alors $a = a_{1,1}$. On réécrit $q$ sous la forme :
      \begin{align*}
        q(x_1, \dots, x_n) &= ax_1^2 + x_1B(x_2, \dots, x_n) + C(x_2, \dots, x_n) \\
        &= a\left( x_1 + \frac{B(x_2, \dots, x_n)}{2a} \right)^2 + \left( C(x_2, \dots, x_n) - \frac{B(x_2, \dots, x_n)^2}{4a} \right)
      \end{align*}
      où $B$ est une forme linéaire et $C$ une forme quadratique. On itère alors le procédé avec $C - \frac{B^2}{4a}$.
      \item \underline{Sinon.} Si $q = 0$, c'est terminé. Sinon, il existe un $a_{i,j}$ non nul. On peut suppose $(i,j) = (1,2)$, on pose alors $a = a_{1,2}$. On réécrit $q$ sous la forme :
      \[ q(x_1, \dots, x_n) = ax_1x_2 + x_1B(x_3, \dots, x_n) + x_2C(x_3, \dots, x_n) + D(x_3, \dots, x_n) \]
      où $B$ et $C$ sont des formes linéaires et $D$ une forme quadratique. En utilisant une identité remarquable :
      \begin{align*}
        q &= a\left( x_1 + \frac{C}{a} \right) \left( x_2 + \frac{B}{a} \right) + \left( D - \frac{BA}{a} \right) \\
        &= \frac{a}{4} \left( \left( x_1 + x_2 + \frac{B+C}{a} \right)^2 - \left( x_1 - x_2 + \frac{C-B}{a} \right)^2 \right) + \left( D - \frac{BC}{a} \right)
      \end{align*}
      On itère alors le procédé avec $D - \frac{BC}{a}$.
    \end{enumerate}
  \end{theorem}

  \begin{example}
    \label{171-2}
    Sur $\mathbb{R}^3$,
    \begin{align*}
      q(x,y,z) &= x^2 - 2y^2 + xz + yz \\
      &= \left( x + \frac{z}{2} \right)^2 - \frac{z^2}{4} - 2y^2 + yz \\
      &= \left( x + \frac{z}{2} \right)^2 - 2 \left( y - \frac{z}{4} \right)^2 + \frac{z^2}{8} - \frac{z^2}{4} \\
      &= \left( x + \frac{z}{2} \right)^2 - 2 \left( y - \frac{z}{4} \right)^2 - \frac{z^2}{8}
    \end{align*}
  \end{example}

  \subsubsection{Signature}

  \begin{definition}
    $q$ est dite \textbf{positive} (resp. \textbf{négative}) si pour tout $x \in E$, $q(x) \geq 0$ (resp. $q(x) \leq 0$).
  \end{definition}

  \dev{loi-d-inertie-de-sylvester}

  \begin{theorem}[Loi d'inertie de Sylvester]
    \[ \exists p, q \in \mathbb{N} \text{ et } \exists f_1, \dots, f_{p+q} \in E^* \text{ tels que } q = \sum_{i=1}^p |f_i|^2 - \sum_{i=p+1}^q |f_i|^2 \]
    où les formes linéaires $f_i$ sont linéairement indépendantes et où $p + q \leq n$. De plus, ces entiers ne dépendent que de $q$ et pas de la décomposition choisie.
    \newpar
    Le couple $(p,q)$ est la \textbf{signature} de $q$ et le rang $q$ est égal à $p+q$.
  \end{theorem}

  \begin{remark}
    En reprenant les notations précédentes, il existe donc une base $\mathcal{B}$ telle que
    \[
    \operatorname{Mat}(q, \mathcal{B}) =
    \begin{pmatrix}
      I_p & 0 & 0 \\
      0 & -I_q & 0 \\
      0 & 0 & 0
    \end{pmatrix}
    \]
    où $r$ est le rang de $q$ et $I_r$ la matrice identité de taille $r$.
  \end{remark}

  \reference[GRI]{310}

  \begin{corollary}
    On note $\operatorname{sign}(q)$ la signature de $q$.
    \begin{enumerate}[label=(\roman*)]
      \item $q$ est définie positive si et seulement si $\operatorname{sign}(q) = (n, 0)$ si et seulement s'il existe des bases $q$-orthonormées.
      \item $q$ est définie négative si et seulement si $\operatorname{sign}(q) = (0, n)$.
      \item $q$ est non dégénérée si et seulement si $\operatorname{sign}(q) = (p, n-p)$.
    \end{enumerate}
  \end{corollary}

  \reference[GOU21]{247}

  \begin{example}
    En reprenant l'\cref{171-2}, on a $\operatorname{sign}(q) = (1,2)$ : $q$ est de rang $3$.
  \end{example}

  \begin{proposition}
    Si $q$ est définie, alors ou bien $q$ est positive, ou bien $q$ est négative.
  \end{proposition}

  \subsection{Applications}

  \subsubsection{Coniques}

  \reference[GRI]{427}

  On suppose $E = \mathbb{R}^2$ et muni d'un produit scalaire $\langle ., . \rangle$.

  \paragraph{Aspect algébrique}

  \begin{definition}
    On appelle \textbf{conique} un ensemble
    \[ \mathcal{C} = \{ v \in E \mid q(v) + \varphi(v) = k, \, k \in \mathbb{R} \} \]
    où $q$ est une forme quadratique non nulle et $\varphi$ une forme linéaire sur $E$.
  \end{definition}

  On gardera les notations de cette définition pour la suite.

  \begin{remark}
    \begin{itemize}
      \item En changeant éventuellement le signe des deux membres de l'équation, on peut supposer que a signature de $q$ est $(2,0)$, $(1,1)$ ou $(1,0)$.
      \item Si $(e_1, e_2)$ est la base de $E$, avec $v = xe_1 + ye_2$, on trouve que l'équation d'une conique est du type
      \[ \alpha x^2 + 2 \beta xy + \gamma y^2 + \lambda x + \mu y = k, \, k \in \mathbb{R} \]
    \end{itemize}
  \end{remark}

  \begin{proposition}
    Il existe une base orthogonale $(v_1, v_2)$ pour $q$ et $\langle ., . \rangle$. Dans cette base, l'équation de la conique est du type
    \[ ax^2 + by^2 - 2rx - 2sy = k, \, k \in \mathbb{R} \tag{$E$} \]
  \end{proposition}

  \begin{definition}
    En reprenant les notations précédentes, les directions définies par $v_1$ et $v_2$ sont appelés \textbf{directions principales} de la conique.
  \end{definition}

  \begin{theorem}[Classification des coniques]
    \begin{enumerate}[label=(\roman*)]
      \item \underline{Si $q$ est non dégénérée :} On peut réécrire l'équation $(E)$ de manière équivalente sous la forme
      \[ ax^2 + by^2 = h \]
      avec $a, b, h \in \mathbb{R}$.
      \begin{itemize}
        \item \underline{Si $\operatorname{sign}(q) = (2,0)$ :} si $h = 0$, $\mathcal{C}$ se réduit à un point ; si $h < 0$, $\mathcal{C} = \emptyset$. Supposons que $h > 0$, alors $\mathcal{C}$ est une ellipse, de centre $\left( \frac{r}{a}, \frac{s}{b} \right)$.
        \item \underline{Si $\operatorname{sign}(q) = (1,1)$ :} si $h \neq 0$, $\mathcal{C}$ est une hyperbole. Si $h = 0$, $\mathcal{C}$ se réduit aux deux droites d'équation $y = \pm \sqrt{\left\vert \frac{a}{b} \right\vert}x$.
      \end{itemize}
      \item \underline{Si $q$ est dégénérée :} On a $ab = 0$ et $\operatorname{sign(q)} = (1,0)$ ; on peut réécrire l'équation $(E)$ de manière équivalente sous la forme
      \[ a \left( x - \frac{r}{a} \right)^2 - 2sy = h \]
      avec $h \in \mathbb{R}$.
      \begin{itemize}
        \item \underline{Si $s \neq 0$ :} $\mathcal{C}$ est une parabole.
        \item \underline{Si $s = 0$ :} si $h = 0$, $\mathcal{C}$ se réduit à la droite $x = 0$ ; si $h < 0$, $\mathcal{C} = \emptyset$. Supposons que $h > 0$, alors $\mathcal{C}$ est constituée des deux droites parallèles d'équation $x = \pm h$.
      \end{itemize}
    \end{enumerate}
  \end{theorem}

  \paragraph{Aspect géométrique}

  \reference[ROM21]{494}

  \begin{proposition}
    En se plaçant dans le plan affine $\mathbb{R}^2$, plongé dans $\mathbb{R}^3$, une conique est l'intersection d'un cône et d'un plan.
  \end{proposition}

  \subsubsection{En analyse}

  Soit $U \subseteq \mathbb{R}^n$ un ouvert.

  \paragraph{Optimisation}

  Soit $f : U \rightarrow \mathbb{R}$ de classe $\mathcal{C}^2$ sur $U$.

  \reference[GOU20]{336}

  \begin{theorem}
    On suppose $\mathrm{d}f_a = 0$ ($a$ est un \textbf{point critique} de $f$). Alors :
    \begin{enumerate}[label=(\roman*)]
      \item Si $f$ admet un minimum (resp. maximum) relatif en $a$, $\operatorname{Hess}(f)_a$ est positive (resp. négative).
      \item Si $\operatorname{Hess}(f)_a$ définit une forme quadratique définie positive (resp. définie négative), $f$ admet un minimum (resp. maximum) relatif en $a$.
    \end{enumerate}
  \end{theorem}

  \begin{example}
    On suppose $\mathrm{d}f_a = 0$. On pose $(r,s,t) = \left(  \frac{\partial^2}{\partial x_i \partial x_j} f \right)_{i+j=2}$. Alors :
    \begin{enumerate}[label=(\roman*)]
      \item Si $rt-s^2 > 0$ (resp. $< 0$), $f$ admet une minimum (resp. maximum) relatif en $a$.
      \item Si $rt-s^2 < 0$ (resp. $< 0$), $f$ n'a pas d'extremum en $a$.
      \item Si $rt-s^2 = 0$, on ne peut rien conclure.
    \end{enumerate}
  \end{example}

  \begin{example}
    La fonction $(x,y) \mapsto x^4 + y^2 - 2(x-y)^2$ a trois points critiques qui sont des minimum locaux : $(0,0)$, $(\sqrt{2},-\sqrt{2})$ et $(-\sqrt{2},\sqrt{2})$.
  \end{example}

  \begin{cexample}
    $x \mapsto x^3$ a sa hessienne positive en $0$, mais n'a pas d'extremum en $0$.
  \end{cexample}

  \paragraph{Homéomorphismes}

  \reference[ROU]{209}

  \begin{lemma}
    Soit $A_0 \in \mathcal{S}_n(\mathbb{R})$ inversible. Alors il existe un voisinage $V$ de $A_0$ dans $\mathcal{S}_n(\mathbb{R})$ et une application $\psi : V \rightarrow \mathrm{GL}_n(\mathbb{R})$ de classe $\mathcal{C}^1$ telle que
    \[ \forall A \in V, \, A = \tr \psi(A) A_0 \psi(A) \]
  \end{lemma}

  \reference{354}

  \begin{lemma}[Morse]
    Soit $f : U \rightarrow \mathbb{R}$ une fonction de classe $\mathcal{C}^3$ (où $U$ désigne un ouvert de $\mathbb{R}^n$ contenant l'origine). On suppose :
    \begin{itemize}
      \item $\mathrm{d} f_0 = 0$.
      \item La matrice symétrique $\mathrm{Hess} (f)_0$ est inversible.
      \item La signature de $\mathrm{Hess}(f)_0$ est $(p, n-p)$.
    \end{itemize}
    Alors il existe un difféomorphisme $\phi = (\phi_1, \dots, \phi_n)$ de classe $\mathcal{C}^1$ entre deux voisinage de l'origine de $\mathbb{R}^n$ $V \subseteq U$ et $W$ tel que $\varphi(0) = 0$ et
    \[ \forall x \in U, \, f(x) - f(0) = \sum_{k=1}^p \phi_k^2(x) - \sum_{k=p+1}^n \phi_k^2(x) \]
  \end{lemma}

  \reference{341}

  \begin{application}
    Soit $S$ la surface d'équation $z = f(x, y)$ où $f$ est de classe $\mathcal{C}^3$ au voisinage de l'origine. On suppose la forme quadratique $\mathrm{d}^2 f_0$ non dégénérée. Alors, en notant $P$ le plan tangent à $S$ en $0$ :
    \begin{enumerate}[label=(\roman*)]
      \item Si $\mathrm{d}^2 f_0$ est de signature $(2, 0)$, alors $S$ est au-dessus de $P$ au voisinage de $0$.
      \item Si $\mathrm{d}^2 f_0$ est de signature $(0, 2)$, alors $S$ est en-dessous de $P$ au voisinage de $0$.
      \item Si $\mathrm{d}^2 f_0$ est de signature $(1, 1)$, alors $S$ traverse $P$ selon une courbe admettant un point double en $(0, f(0))$.
    \end{enumerate}
  \end{application}

  \subsubsection{Racines de polynômes}

  \reference[C-G]{356}

  Soit $P \in \mathbb{R}[X]$ un polynôme de degré $n$.

  \begin{notation}
    On note :
    \begin{itemize}
      \item $x_1, \dots, x_t$ les racines complexes de $P$ de multiplicités respectives $m_1, \dots, m_t$.
      \item $s_0 = n \text{ et } \forall k \geq 1, \, s_k = \sum_{i=1}^t m_i x_i^k$.
    \end{itemize}
  \end{notation}

  \begin{proposition}
    $\sigma = \sum_{i, j \in \llbracket 0, n-1 \rrbracket} s_{i+j} X_i X_j$ définit une forme quadratique sur $\mathbb{C}^n$ ainsi qu'une forme quadratique $\sigma_{\mathbb{R}}$ sur $\mathbb{R}^n$.
  \end{proposition}

  \dev{formes-de-hankel}

  \begin{theorem}[Formes de Hankel]
    On note $(p,q)$ la signature de $\sigma_{\mathbb{R}}$, on a :
    \begin{itemize}
      \item $t = p + q$.
      \item Le nombre de racines réelles distinctes de $P$ est $p-q$.
    \end{itemize}
  \end{theorem}

  \annexessection

  \begin{figure}[H]
    \begin{center}
      \begin{tikzpicture}
        \draw[->] (-2.5, 0) -- (2.5, 0) node[right] {$x$};
        \draw[->] (0, -2.5) -- (0,2.5) node[above] {$y$};
        \draw[thick, teal] (0,0) ellipse (2 and 1);
      \end{tikzpicture}
    \end{center}
    \caption{Une ellipse ($\operatorname{sign}(q) = (2,0)$).}
  \end{figure}

  \begin{figure}[H]
    \begin{center}
      \begin{tikzpicture}
        \draw[->] (-2.5, 0) -- (2.5, 0) node[right] {$x$};
        \draw[->] (0, -2.5) -- (0,2.5) node[above] {$y$};
        \draw[dashed] (-2.50277783, -2.00347121) -- (2.50277783, 2.00347121);
        \draw[dashed] (-2.50277783, 2.00347121) -- (2.50277783, -2.00347121);
        \draw[teal,thick,smooth,variable=\t,domain=-1.1:1.1] plot({-1.5*cosh(\t)},{1.5*sinh(\t)});
        \draw[teal,thick,smooth,variable=\t,domain=-1.1:1.1] plot({ 1.5*cosh(\t)},{1.5*sinh(\t)});
      \end{tikzpicture}
    \end{center}
    \caption{Une hyperbole ($\operatorname{sign}(q) = (1,1)$).}
  \end{figure}

  \begin{figure}[H]
    \begin{center}
      \begin{tikzpicture}
        \draw[->] (-2.5, 0) -- (2.5, 0) node[right] {$x$};
        \draw[->] (0, -2.5) -- (0,2.5) node[above] {$y$};
        \draw[teal,thick] (-2.50277783, -2.00347121) -- (2.50277783, 2.00347121);
        \draw[teal,thick] (-2.50277783, 2.00347121) -- (2.50277783, -2.00347121);
      \end{tikzpicture}
    \end{center}
    \caption{Une hyperbole dégénérée en deux droites sécantes ($\operatorname{sign}(q) = (1,1)$).}
  \end{figure}

  \begin{figure}[H]
    \begin{center}
      \begin{tikzpicture}
        \draw[->] (-2.5, 0) -- (2.5, 0) node[right] {$x$};
        \draw[->] (0, -2.5) -- (0,2.5) node[above] {$y$};
        \draw[teal,thick,smooth,variable=\x,domain=-2.5:2.5] plot({\x},{0.4*\x*\x});
      \end{tikzpicture}
    \end{center}
    \caption{Une parabole ($\operatorname{sign}(q) = (1,0)$).}
  \end{figure}

  \begin{figure}[H]
    \begin{center}
      \begin{tikzpicture}
        \draw[->] (-2.5, 0) -- (2.5, 0) node[right] {$x$};
        \draw[->] (0, -2.5) -- (0,2.5) node[above] {$y$};
        \draw[teal,thick] (-1,-2.5) -- (-1,2.5);
        \draw[teal,thick] (1,-2.5) -- (1,2.5);
      \end{tikzpicture}
    \end{center}
    \caption{Une parabole dégénérée en deux droites parallèles ($\operatorname{sign}(q) = (1,0)$).}
  \end{figure}
  %</content>
\end{document}
