\documentclass[12pt, a4paper]{report}

% LuaLaTeX :

\RequirePackage{iftex}
\RequireLuaTeX

% Packages :

\usepackage[french]{babel}
%\usepackage[utf8]{inputenc}
%\usepackage[T1]{fontenc}
\usepackage[pdfencoding=auto, pdfauthor={Hugo Delaunay}, pdfsubject={Mathématiques}, pdfcreator={agreg.skyost.eu}]{hyperref}
\usepackage{amsmath}
\usepackage{amsthm}
%\usepackage{amssymb}
\usepackage{stmaryrd}
\usepackage{tikz}
\usepackage{tkz-euclide}
\usepackage{fourier-otf}
\usepackage{fontspec}
\usepackage{titlesec}
\usepackage{fancyhdr}
\usepackage{catchfilebetweentags}
\usepackage[french, capitalise, noabbrev]{cleveref}
\usepackage[fit, breakall]{truncate}
\usepackage[top=2.5cm, right=2cm, bottom=2.5cm, left=2cm]{geometry}
\usepackage{enumerate}
\usepackage{tocloft}
\usepackage{microtype}
%\usepackage{mdframed}
%\usepackage{thmtools}
\usepackage{xcolor}
\usepackage{tabularx}
\usepackage{aligned-overset}
\usepackage[subpreambles=true]{standalone}
\usepackage{environ}
\usepackage[normalem]{ulem}
\usepackage{marginnote}
\usepackage{etoolbox}
\usepackage{setspace}
\usepackage[bibstyle=reading, citestyle=draft]{biblatex}
\usepackage{xpatch}
\usepackage[many, breakable]{tcolorbox}
\usepackage[backgroundcolor=white, bordercolor=white, textsize=small]{todonotes}

% Bibliographie :

\newcommand{\overridebibliographypath}[1]{\providecommand{\bibliographypath}{#1}}
\overridebibliographypath{../bibliography.bib}
\addbibresource{\bibliographypath}
\defbibheading{bibliography}[\bibname]{%
	\newpage
	\section*{#1}%
}
\renewbibmacro*{entryhead:full}{\printfield{labeltitle}}
\DeclareFieldFormat{url}{\newline\footnotesize\url{#1}}
\AtEndDocument{\printbibliography}

% Police :

\setmathfont{Erewhon Math}

% Tikz :

\usetikzlibrary{calc}

% Longueurs :

\setlength{\parindent}{0pt}
\setlength{\headheight}{15pt}
\setlength{\fboxsep}{0pt}
\titlespacing*{\chapter}{0pt}{-20pt}{10pt}
\setlength{\marginparwidth}{1.5cm}
\setstretch{1.1}

% Métadonnées :

\author{agreg.skyost.eu}
\date{\today}

% Titres :

\setcounter{secnumdepth}{3}

\renewcommand{\thechapter}{\Roman{chapter}}
\renewcommand{\thesubsection}{\Roman{subsection}}
\renewcommand{\thesubsubsection}{\arabic{subsubsection}}
\renewcommand{\theparagraph}{\alph{paragraph}}

\titleformat{\chapter}{\huge\bfseries}{\thechapter}{20pt}{\huge\bfseries}
\titleformat*{\section}{\LARGE\bfseries}
\titleformat{\subsection}{\Large\bfseries}{\thesubsection \, - \,}{0pt}{\Large\bfseries}
\titleformat{\subsubsection}{\large\bfseries}{\thesubsubsection. \,}{0pt}{\large\bfseries}
\titleformat{\paragraph}{\bfseries}{\theparagraph. \,}{0pt}{\bfseries}

\setcounter{secnumdepth}{4}

% Table des matières :

\renewcommand{\cftsecleader}{\cftdotfill{\cftdotsep}}
\addtolength{\cftsecnumwidth}{10pt}

% Redéfinition des commandes :

\renewcommand*\thesection{\arabic{section}}
\renewcommand{\ker}{\mathrm{Ker}}

% Nouvelles commandes :

\newcommand{\website}{https://agreg.skyost.eu}

\newcommand{\tr}[1]{\mathstrut ^t #1}
\newcommand{\im}{\mathrm{Im}}
\newcommand{\rang}{\operatorname{rang}}
\newcommand{\trace}{\operatorname{trace}}
\newcommand{\id}{\operatorname{id}}
\newcommand{\stab}{\operatorname{Stab}}

\providecommand{\newpar}{\\[\medskipamount]}

\providecommand{\lesson}[3]{%
	\title{#3}%
	\hypersetup{pdftitle={#3}}%
	\setcounter{section}{\numexpr #2 - 1}%
	\section{#3}%
	\fancyhead[R]{\truncate{0.73\textwidth}{#2 : #3}}%
}

\providecommand{\development}[3]{%
	\title{#3}%
	\hypersetup{pdftitle={#3}}%
	\section*{#3}%
	\fancyhead[R]{\truncate{0.73\textwidth}{#3}}%
}

\providecommand{\summary}[1]{%
	\textit{#1}%
	\medskip%
}

\tikzset{notestyleraw/.append style={inner sep=0pt, rounded corners=0pt, align=center}}

%\newcommand{\booklink}[1]{\website/bibliographie\##1}
\newcommand{\citelink}[2]{\hyperlink{cite.\therefsection @#1}{#2}}
\newcommand{\previousreference}{}
\providecommand{\reference}[2][]{%
	\notblank{#1}{\renewcommand{\previousreference}{#1}}{}%
	\todo[noline]{%
		\protect\vspace{16pt}%
		\protect\par%
		\protect\notblank{#1}{\cite{[\previousreference]}\\}{}%
		\protect\citelink{\previousreference}{p. #2}%
	}%
}

\definecolor{devcolor}{HTML}{00695c}
\newcommand{\dev}[1]{%
	\reversemarginpar%
	\todo[noline]{
		\protect\vspace{16pt}%
		\protect\par%
		\bfseries\color{devcolor}\href{\website/developpements/#1}{DEV}
	}%
	\normalmarginpar%
}

% En-têtes :

\pagestyle{fancy}
\fancyhead[L]{\truncate{0.23\textwidth}{\thepage}}
\fancyfoot[C]{\scriptsize \href{\website}{\texttt{agreg.skyost.eu}}}

% Couleurs :

\definecolor{property}{HTML}{fffde7}
\definecolor{proposition}{HTML}{fff8e1}
\definecolor{lemma}{HTML}{fff3e0}
\definecolor{theorem}{HTML}{fce4f2}
\definecolor{corollary}{HTML}{ffebee}
\definecolor{definition}{HTML}{ede7f6}
\definecolor{notation}{HTML}{f3e5f5}
\definecolor{example}{HTML}{e0f7fa}
\definecolor{cexample}{HTML}{efebe9}
\definecolor{application}{HTML}{e0f2f1}
\definecolor{remark}{HTML}{e8f5e9}
\definecolor{proof}{HTML}{e1f5fe}

% Théorèmes :

\theoremstyle{definition}
\newtheorem{theorem}{Théorème}

\newtheorem{property}[theorem]{Propriété}
\newtheorem{proposition}[theorem]{Proposition}
\newtheorem{lemma}[theorem]{Lemme}
\newtheorem{corollary}[theorem]{Corollaire}

\newtheorem{definition}[theorem]{Définition}
\newtheorem{notation}[theorem]{Notation}

\newtheorem{example}[theorem]{Exemple}
\newtheorem{cexample}[theorem]{Contre-exemple}
\newtheorem{application}[theorem]{Application}

\theoremstyle{remark}
\newtheorem{remark}[theorem]{Remarque}

\counterwithin*{theorem}{section}

\newcommand{\applystyletotheorem}[1]{
	\tcolorboxenvironment{#1}{
		enhanced,
		breakable,
		colback=#1!98!white,
		boxrule=0pt,
		boxsep=0pt,
		left=8pt,
		right=8pt,
		top=8pt,
		bottom=8pt,
		sharp corners,
		after=\par,
	}
}

\applystyletotheorem{property}
\applystyletotheorem{proposition}
\applystyletotheorem{lemma}
\applystyletotheorem{theorem}
\applystyletotheorem{corollary}
\applystyletotheorem{definition}
\applystyletotheorem{notation}
\applystyletotheorem{example}
\applystyletotheorem{cexample}
\applystyletotheorem{application}
\applystyletotheorem{remark}
\applystyletotheorem{proof}

% Environnements :

\NewEnviron{whitetabularx}[1]{%
	\renewcommand{\arraystretch}{2.5}
	\colorbox{white}{%
		\begin{tabularx}{\textwidth}{#1}%
			\BODY%
		\end{tabularx}%
	}%
}

% Maths :

\DeclareFontEncoding{FMS}{}{}
\DeclareFontSubstitution{FMS}{futm}{m}{n}
\DeclareFontEncoding{FMX}{}{}
\DeclareFontSubstitution{FMX}{futm}{m}{n}
\DeclareSymbolFont{fouriersymbols}{FMS}{futm}{m}{n}
\DeclareSymbolFont{fourierlargesymbols}{FMX}{futm}{m}{n}
\DeclareMathDelimiter{\VERT}{\mathord}{fouriersymbols}{152}{fourierlargesymbols}{147}


% Bibliographie :

\addbibresource{\bibliographypath}%
\defbibheading{bibliography}[\bibname]{%
	\newpage
	\section*{#1}%
}
\renewbibmacro*{entryhead:full}{\printfield{labeltitle}}%
\DeclareFieldFormat{url}{\newline\footnotesize\url{#1}}%

\AtEndDocument{\printbibliography}

\begin{document}
	%<*content>
	\lesson{analysis}{266}{Utilisation de la notion d'indépendance en probabilités.}

	Soit $(\Omega, \mathcal{A}, \mathbb{P})$ un espace probabilisé.

	\subsection{Indépendance en probabilités}
	
	\subsubsection{Indépendance d'événements}
	
	\reference[G-K]{52}
	
	\begin{definition}
		On dit que deux événements $A$ et $B$ sont \textbf{indépendants} (sous $\mathbb{P}$) si
		\[ \mathbb{P}(A \, \cap \, B) = \mathbb{P}(A) \mathbb{P}(B) \]
	\end{definition}
	
	\begin{definition}
		On dit que les événements d'une famille $(A_i)_{i \in I}$ sont \textbf{mutuellement indépendants} si
		\[ \forall J \subseteq I, \, J \text{ fini}, \, \mathbb{P}\left( \bigcap_{j \in J} A_j \right) = \prod_{j \in J} \mathbb{P}(A_j) \] 
	\end{definition}
	
	\reference[DAN]{425}
	
	\begin{proposition}
		Soient $A, B$ deux événements. Alors,
		\[ A \text{ et } B \text{ sont indépendants} \iff \mathbb{P}(A \setminus B) = \mathbb{P}(A) \iff \mathbb{P}(B \setminus A) = \mathbb{P}(B) \]
	\end{proposition}
	
	\begin{proposition}
		Soient $A_1, \dots, A_n$ des événements mutuellement indépendants. Alors, pour tout $k \in \llbracket 1, n \rrbracket$, $A_1^c, \dots, A_k^c, A_{k+1}, \dots, A_n$ sont mutuellement indépendants.
	\end{proposition}
	
	\begin{example}
		On considère deux gênes $a$ et $b$ tels que la redondance de l'un d'entre eux entraîne l'acquisition d'un caractère d'un caractère $\mathcal{C}$. Anselme et Colette possèdent chacun la combinaison $ab$ et attendant un enfant : elles lui transmettront chacun et indépendamment soit le gêne $a$, soit le gêne $b$ avec la même probabilité de $\frac{1}{2}$. On considère les événements :
		\begin{itemize}
			\item $A$ : Colette transmet le gêne $a$.
			\item $B$ : Anselme transmet le gêne $b$.
			\item $C$ : l'enfant présent le caractère $\mathcal{C}$.
		\end{itemize}
		$A$, $B$ et $C$ sont indépendants deux à deux, mais non mutuellement indépendants.
	\end{example}
	
	\begin{application}[Indicatrice d'Euler]
		On note $\varphi$ l'indicatrice d'Euler. Alors,
		\[ \forall n \geq 2, \, \varphi(n) = n \prod_{\substack{p \text{ premier} \\ p \mid n}} \left( 1 - \frac{1}{p} \right) \]
	\end{application}
	
	\subsubsection{Indépendance de tribus}
	
	\reference[G-K]{52}
	
	\begin{definition}
		On dit que deux sous-tribus $\mathcal{A}_1$ et $\mathcal{A}_2$ de $\mathcal{A}$ sont \textbf{indépendantes} (sous $\mathbb{P}$) si
		\[ \forall A \in \mathcal{A}_1, \, \forall B \in \mathcal{A}_2, \, \mathbb{P}(A \, \cap \, B) = \mathbb{P}(A) \mathbb{P}(B) \]
	\end{definition}
	
	\begin{definition}
		On dit qu'une famille de sous-tribus $(\mathcal{A}_i)_{i \in I}$ de $\mathcal{A}$ sont \textbf{indépendantes} (sous $\mathbb{P}$) si
		\[ \forall J \subseteq I, \, J \text{ fini}, \, \forall (A_j)_{j \in J} \in \prod_{j \in J} \mathcal{A}_j, \, \mathbb{P}\left( \bigcap_{j \in J} A_j \right) = \prod_{j \in J} \mathbb{P}(A_j) \]
	\end{definition}
	
	\begin{remark}
		\begin{itemize}
			\item Pour tout $A, B \in \mathcal{A}$, $A$ est indépendante de $B$ si et seulement si $\sigma(A)$ est indépendante de $\sigma(B)$.
			\item Si deux tribus $\mathcal{A}$ et $\mathcal{B}$ sont indépendantes, toute sous tribu de $\mathcal{A}$ est indépendante de toute sous tribu de $\mathcal{B}$.
		\end{itemize}
	\end{remark}
	
	\subsubsection{Indépendance de variables aléatoires}
	
	\paragraph{Variables aléatoires indépendantes}
	
	\reference{125}
	
	\begin{definition}
		Soit $X$ une variable aléatoire réelle définie sur $(\Omega, \mathcal{A}, \mathbb{P})$. On note
		\[ \sigma(X) = \{ X^{-1}(A) \mid A \in \mathcal{B}(\mathbb{R}) \} \]
		Cette famille est la \textbf{tribu engendrée} par $X$.
	\end{definition}
	
	\begin{definition}
		On dit que deux variables aléatoires $X$ et $Y$ sont \textbf{indépendantes} si les tribus qu'elles engendrent sont indépendantes.
	\end{definition}
	
	\begin{example}
		Si $X$ et $Y$ sont deux variables aléatoires indépendantes, on a
		\[ \forall A, B \in \mathcal{B}(\mathbb{R}), \, \mathbb{P}(\{ X \in A \} \, \cap \, \{ Y \in B \}) = \mathbb{P}(X \in A) \mathbb{P}(Y \in B) \]
	\end{example}
	
	\begin{proposition}
		Si $X$ et $Y$ sont deux variables aléatoires indépendantes, alors $f(X)$ et $g(Y)$ sont indépendantes pour toutes fonctions mesurables $f$ et $g$.
	\end{proposition}
	
	\reference{136}
	
	\begin{proposition}
		Soient $X$ et $Y$ deux vecteurs aléatoires indépendants. On suppose que $X$ admet une densité $f$ et $Y$ admet une densité $g$. Alors, $(X,Y)$ admet comme densité $(x,y) \mapsto f(x)g(y)$.
	\end{proposition}
	
	\reference{175}
	
	\begin{proposition}
		Soient $X$ et $Y$ deux vecteurs aléatoires indépendants intégrables. Alors,
		\[ \mathbb{E}(XY) = \mathbb{E}(X) \mathbb{E}(Y) \]
	\end{proposition}
	
	\paragraph{Variables aléatoires non corrélées}
	
	\reference{174}
	
	\begin{definition}
		On dit que deux variables aléatoires $X$ et $Y$ sont \textbf{non corrélées} si
		\[ \operatorname{Covar}(X,Y) = \mathbb{E}(X - \mathbb{E}(X))\mathbb{E}(Y - \mathbb{E}(Y)) = 0 \]
	\end{definition}
	
	\begin{proposition}
		Soient $X$ et $Y$ deux variables aléatoires indépendantes intégrables. Alors $X$ et $Y$ ne sont pas corrélées.
	\end{proposition}
	
	\begin{cexample}
		La réciproque est fausse. Ainsi, soient $X$ et $Y$ deux variables aléatoires vérifiant
		\begin{align*}
			\mathbb{P}(\{ X = 1 \} \, \cap \, \{ Y = 1 \}) &= \mathbb{P}(\{ X = 1 \} \, \cap \, \{ Y = -1 \}) \\
			&= \mathbb{P}(\{ X = -1 \} \, \cap \, \{ Y = 0 \}) \\
			&= \frac{1}{3}
		\end{align*}
		alors, $X$ et $Y$ sont non corrélées mais pas indépendantes.
	\end{cexample}
	
	\subsection{Étude de variables aléatoires indépendantes}
	
	\subsubsection{Critères d'indépendance}
	
	\reference{128}
	
	\begin{theorem}
		Soient $X$ et $Y$ deux variables aléatoires. Alors, $X$ et $Y$ sont indépendantes si et seulement si $\mathbb{P}_{(X,Y)} = \mathbb{P}_X \otimes \mathbb{P}_Y$.
	\end{theorem}
	
	\begin{corollary}
		Soient $X$ et $Y$ deux variables aléatoires indépendantes. Alors, $\mathbb{P}_{X+Y} = \mathbb{P}_X * \mathbb{P}_Y$.
	\end{corollary}
	
	\reference{136}
	
	\begin{proposition}
		Soient $X$ et $Y$ deux variables aléatoires définies sur $(\Omega, \mathcal{A}, \mathbb{P})$. On suppose que $(X,Y)$ admet une densité $h : (x,y) \mapsto f(x) g(y)$ à variables séparées. Alors, $X$ et $Y$ sont indépendantes. De plus, $X$ et $Y$ admettent respectivement pour densité
		\[ x \mapsto \frac{f(x)}{\int_{\mathbb{R}} f(t) \, \mathrm{d}t} \text{ et } y \mapsto \frac{g(y)}{\int_{\mathbb{R}} g(t) \, \mathrm{d}t} \]
		par rapport à la mesure de Lebesgue.
	\end{proposition}
	
	\subsubsection{Sommes de variables aléatoires indépendantes}
	
	\reference{179}
	
	\begin{theorem}
		Soient $X$ et $Y$ deux variables aléatoires réelles indépendantes de densités respectives $f$ et $g$. Alors, $X + Y$ admet comme densité la fonction $f * g : x \mapsto \int_{\mathbb{R}} f(x-t) g(t) \, \mathrm{d}t$.
	\end{theorem}
	
	\begin{application}
		Soient $X$ et $Y$ deux variables aléatoires indépendantes telles que $X \sim \Gamma(a, \gamma)$ et $Y \sim \Gamma(b, \gamma)$. Alors $Z = X + Y \sim \Gamma(a+b, \gamma)$.
	\end{application}
	
	\begin{application}
		\[ \frac{\Gamma(a) \Gamma(b)}{\Gamma(a+b)} = \int_0^1 \theta^{a-1} (1-\theta)^{b-1} \, \mathrm{d}\theta \]
		où $\Gamma$ désigne la fonction $\Gamma$ d'Euler.
	\end{application}
	
	\reference{235}
	
	\begin{definition}
		Soit $X$ une variable aléatoire à valeurs dans $\mathbb{N}$. On appelle \textbf{fonction génératrice} de $X$ la fonction
		\[
		G_X :
		\begin{array}{ccc}
			[-1,1] &\rightarrow& \mathbb{R} \\
			z &\mapsto& \sum_{k=0}^{+\infty} \mathbb{P}(X=k) z^k
		\end{array}
		\]
	\end{definition}
	
	\begin{proposition}
		Soient $X$ et $Y$ deux variables aléatoires à valeurs dans $\mathbb{N}$ indépendantes. Alors,
		\[ G_{X+Y} = G_X G_Y \]
	\end{proposition}
	
	\begin{theorem}
		Sur $[0,1]$, la fonction $G_X$ est infiniment dérivable et ses dérivées sont toutes positives, avec
		\[ G_X^{(n)}(s) = \mathbb{E}(X(X-1) \dots (X-n+1)s^{X-n}) \]
		En particulier,
		\[ \mathbb{P}(X=n) = \frac{G_X^{(n)}(0)}{n!} \]
		ce qui montre que la fonction génératrice caractérise la loi.
	\end{theorem}
	
	\begin{example}
		Si $X_1 \sim \mathcal{P}(\lambda)$ et $X_2 \sim \mathcal{P}(\mu)$ sont indépendantes, alors $X_1 + X_2 \sim \mathcal{P}(\lambda + \mu)$.
	\end{example}
	
	\reference[GOU21]{346}
	
	\begin{example}
		Si $X_1 \sim \mathcal{B}(n, p)$ et $X_2 \sim \mathcal{B}(m, p)$ sont indépendantes, alors $X_1 + X_2 \sim \mathcal{B}(n + m, p)$.
	\end{example}
	
	\reference[G-K]{239}
	
	\begin{definition}
		On appelle \textbf{fonction caractéristique} de $X$ la fonction $\phi_X$ définie sur $\mathbb{R}^d$ par
		\[ \phi_X : t \mapsto \mathbb{E}\left( e^{i \langle t, X \rangle} \right) \]
	\end{definition}
	
	\begin{theorem}
		Si deux variables (ou vecteurs) aléatoires ont la même fonction caractéristique, alors elles ont même loi.
	\end{theorem}
	
	\begin{proposition}
		Si deux variables aléatoires réelles sont indépendantes, alors $\phi_{X+Y} = \phi_X \phi_Y$.
	\end{proposition}
	
	\subsection{Indépendance et théorèmes limites}
	
	\subsubsection{Lemmes de Borel-Cantelli}
	
	\reference{272}
	
	\begin{theorem}[1\ier{} lemme de Borel-Cantelli]
		Soit $(A_n)$ une suite d'événements. Si $\sum \mathbb{P}(A_n)$ converge, alors
		\[ \mathbb{P} \left( \limsup_{n \rightarrow +\infty} A_n \right) = 0 \]
	\end{theorem}
	
	\begin{remark}
		Cela signifie que presque sûrement, seul un nombre fini d'événements $A_n$ se réalisent.
	\end{remark}
	
	\begin{corollary}
		Si $\sum \mathbb{P}(\vert X_n - X \vert > \epsilon)$ converge pour tout $\epsilon > 0$, alors $X_n \overset{(ps.)}{\longrightarrow} X$.
	\end{corollary}
	
	\reference{285}
	
	\begin{example}
		Si $(X_n)$ est telle que $\forall n \geq 1$, $\mathbb{P}(X_n = n) = \mathbb{P}(X_n = \pm n) = \frac{1}{2n^2}$ et $\mathbb{P}(X_n = 0) = 1 - \frac{1}{2n^2}$, alors la suite $(S_n)$ définie pour tout $n \geq 1$ par $S_n = \sum_{k=1}^n X_k$ est constante à partir d'un certain rang.
	\end{example}
	
	\reference{273}
	
	\begin{theorem}[2\ieme{} lemme de Borel-Cantelli]
		Soit $(A_n)$ une suite d'événements indépendants. Si $\sum \mathbb{P}(A_n)$ diverge, alors
		\[ \mathbb{P} \left( \limsup_{n \rightarrow +\infty} A_n \right) = 1 \]
	\end{theorem}
	
	\begin{remark}
		Cela signifie que presque sûrement, un nombre infini d'événements $A_n$ se réalisent.
	\end{remark}
	
	\reference{286}
	
	\begin{example}
		On fait une infinité de lancers d'une pièce de monnaie équilibrée. Alors, la probabilité de l'événement ``on obtient une infinité de fois deux ``Face'' consécutifs'' est $1$.
	\end{example}
	
	\begin{corollary}[Loi du $0$-$1$ de Borel]
		Soit $(A_n)$ une suite d'événements indépendants, alors
		\[ \mathbb{P} \left( \limsup_{n \rightarrow +\infty} A_n \right) = 0 \text{ ou } 1 \]
		et elle vaut $1$ si et seulement si $\sum \mathbb{P}(A_n)$ diverge.
	\end{corollary}
	
	\subsubsection{Lois des grands nombres}
	
	\reference{270}
	
	\begin{theorem}[Loi faible des grands nombres]
		Soit $(X_n)$ une suite de variables aléatoires deux à deux indépendantes de même loi et $\mathcal{L}_1$. On pose $M_n = \frac{X_1 + \dots + X_n}{n}$. Alors,
		\[ M_n \overset{(p)}{\longrightarrow} \mathbb{E}(X_1) \]
	\end{theorem}
	
	\reference[Z-Q]{532}
	
	\begin{theorem}[Loi forte des grands nombres]
		Soit $(X_n)$ une suite de variables aléatoires mutuellement indépendantes de même loi. On pose $M_n = \frac{X_1 + \dots + X_n}{n}$. Alors,
		\[ X_1 \in \mathcal{L}_1 \iff M_n \overset{(ps.)}{\longrightarrow} \ell \in \mathbb{R} \]
		Dans ce cas, on a $\ell = \mathbb{E}(X_1)$.
	\end{theorem}
	
	\reference[G-K]{195}
	
	\begin{application}[Théorème de Bernstein]
		Soit $f : [0,1] \rightarrow \mathbb{R}$ continue. On note
		\[ \forall n \in \mathbb{N}^*, \, B_n(f) : x \mapsto \sum_{k=0}^n \binom{n}{k} f \left( \frac{k}{n} \right) x^k (1-x)^{n-k} \]
		le $n$-ième polynôme de Bernstein associé à $f$. Alors la suite de fonctions $(B_n(f))$ converge uniformément vers $f$.
	\end{application}
	
	\dev{theoreme-de-weierstrass-par-les-probabilites}
	
	\begin{corollary}[Théorème de Weierstrass]
		Toute fonction continue $f : [a,b] \rightarrow \mathbb{R}$ (avec $a, b \in \mathbb{R}$ tels que $a \leq b$) est limite uniforme de fonctions polynômiales sur $[a, b]$.
	\end{corollary}
	
	\newpage
	\subsubsection{Théorème central limite}
	
	\reference[G-K]{307}
	\dev{theoreme-central-limite}
	
	\begin{theorem}[Central limite]
		On suppose que $(X_n)$ est une suite de variables aléatoires réelles indépendantes de même loi admettant un moment d'ordre $2$. On note $m$ l'espérance et $\sigma^2$ la variance commune à ces variables. On pose $S_n = X_1 + \dots + X_n - nm$. Alors,
		\[ \left ( \frac{S_n}{\sqrt{n}} \right) \overset{(d)}{\longrightarrow} \mathcal{N}(0, \sigma^2) \]
	\end{theorem}
	
	\begin{application}[Théorème de Moivre-Laplace]
		On suppose que $(X_n)$ est une suite de variables aléatoires indépendantes de même loi $\mathcal{B}(p)$. Alors,
		\[ \frac{\sum_{k=1}^{n} X_k - np}{\sqrt{n}} \overset{(d)}{\longrightarrow} \mathcal{N}(0, p(1-p)) \]
	\end{application}
	
	\reference{556}
	
	\begin{application}[Formule de Stirling]
		\[ n! \sim \sqrt{2n\pi} \left(\frac{n}{e} \right)^n \]
	\end{application}
	
	\newpage
	\subsection*{Annexes}
	
	\reference[G-K]{137}
	\reference{236}
	
	\begin{figure}[H]
		\begin{center}
			\begin{whitetabularx}{|X|X|}
				\hline
				\textbf{Loi} & \textbf{Somme} (indépendantes de même loi) \\
				\hline
				de Bernoulli & $\sum_{k=1}^n \mathcal{B}(p) \sim \mathcal{B}(n,p)$ \\
				\hline
				Binomiale & $\sum_{k=1}^n \mathcal{B}(n_k, p) \sim \mathcal{B} \left( \sum_{k=1}^n n_k, p \right)$ \\
				\hline
				de Poisson & $\sum_{k=1}^n \mathcal{P}(\lambda_k) \sim \mathcal{P} \left( \sum_{k=1}^n \lambda_k \right)$ \\
				\hline
			\end{whitetabularx}
		\end{center}
		\caption{Sommes de variables aléatoires à lois discrètes}
	\end{figure}
	
	\reference{142}
	\reference{247}
	\reference{178}
	
	\begin{figure}[H]
		\begin{center}
			\begin{whitetabularx}{|X|X|}
				\hline
				\textbf{Loi} & \textbf{Somme} (indépendantes de même loi) \\
				\hline
				Normale & $\sum_{k=1}^n \mathcal{N}(\mu_k, \sigma_k^2) \sim \mathcal{N} \left( \sum_{k=1}^n \mu_k, \sum_{k=1}^n \sigma_k^2 \right)$ \\
				\hline
				Exponentielle & $\sum_{k=1}^n \mathcal{E}(\lambda) \sim \mathcal{E} \left( n\lambda \right)$ \\
				\hline
				Gamma & $\sum_{k=1}^n \Gamma(a_k, \gamma) \sim \Gamma \left( \sum_{k=1}^n a_k, \gamma \right)$ \\
				\hline
			\end{whitetabularx}
		\end{center}
		\caption{Sommes de variables aléatoires à densité}
	\end{figure}
	%</content>
\end{document}
