\documentclass[12pt, a4paper]{report}

% LuaLaTeX :

\RequirePackage{iftex}
\RequireLuaTeX

% Packages :

\usepackage[french]{babel}
%\usepackage[utf8]{inputenc}
%\usepackage[T1]{fontenc}
\usepackage[pdfencoding=auto, pdfauthor={Hugo Delaunay}, pdfsubject={Mathématiques}, pdfcreator={agreg.skyost.eu}]{hyperref}
\usepackage{amsmath}
\usepackage{amsthm}
%\usepackage{amssymb}
\usepackage{stmaryrd}
\usepackage{tikz}
\usepackage{tkz-euclide}
\usepackage{fourier-otf}
\usepackage{fontspec}
\usepackage{titlesec}
\usepackage{fancyhdr}
\usepackage{catchfilebetweentags}
\usepackage[french, capitalise, noabbrev]{cleveref}
\usepackage[fit, breakall]{truncate}
\usepackage[top=2.5cm, right=2cm, bottom=2.5cm, left=2cm]{geometry}
\usepackage{enumerate}
\usepackage{tocloft}
\usepackage{microtype}
%\usepackage{mdframed}
%\usepackage{thmtools}
\usepackage{xcolor}
\usepackage{tabularx}
\usepackage{aligned-overset}
\usepackage[subpreambles=true]{standalone}
\usepackage{environ}
\usepackage[normalem]{ulem}
\usepackage{marginnote}
\usepackage{etoolbox}
\usepackage{setspace}
\usepackage[bibstyle=reading, citestyle=draft]{biblatex}
\usepackage{xpatch}
\usepackage[many, breakable]{tcolorbox}
\usepackage[backgroundcolor=white, bordercolor=white, textsize=small]{todonotes}

% Bibliographie :

\newcommand{\overridebibliographypath}[1]{\providecommand{\bibliographypath}{#1}}
\overridebibliographypath{../bibliography.bib}
\addbibresource{\bibliographypath}
\defbibheading{bibliography}[\bibname]{%
	\newpage
	\section*{#1}%
}
\renewbibmacro*{entryhead:full}{\printfield{labeltitle}}
\DeclareFieldFormat{url}{\newline\footnotesize\url{#1}}
\AtEndDocument{\printbibliography}

% Police :

\setmathfont{Erewhon Math}

% Tikz :

\usetikzlibrary{calc}

% Longueurs :

\setlength{\parindent}{0pt}
\setlength{\headheight}{15pt}
\setlength{\fboxsep}{0pt}
\titlespacing*{\chapter}{0pt}{-20pt}{10pt}
\setlength{\marginparwidth}{1.5cm}
\setstretch{1.1}

% Métadonnées :

\author{agreg.skyost.eu}
\date{\today}

% Titres :

\setcounter{secnumdepth}{3}

\renewcommand{\thechapter}{\Roman{chapter}}
\renewcommand{\thesubsection}{\Roman{subsection}}
\renewcommand{\thesubsubsection}{\arabic{subsubsection}}
\renewcommand{\theparagraph}{\alph{paragraph}}

\titleformat{\chapter}{\huge\bfseries}{\thechapter}{20pt}{\huge\bfseries}
\titleformat*{\section}{\LARGE\bfseries}
\titleformat{\subsection}{\Large\bfseries}{\thesubsection \, - \,}{0pt}{\Large\bfseries}
\titleformat{\subsubsection}{\large\bfseries}{\thesubsubsection. \,}{0pt}{\large\bfseries}
\titleformat{\paragraph}{\bfseries}{\theparagraph. \,}{0pt}{\bfseries}

\setcounter{secnumdepth}{4}

% Table des matières :

\renewcommand{\cftsecleader}{\cftdotfill{\cftdotsep}}
\addtolength{\cftsecnumwidth}{10pt}

% Redéfinition des commandes :

\renewcommand*\thesection{\arabic{section}}
\renewcommand{\ker}{\mathrm{Ker}}

% Nouvelles commandes :

\newcommand{\website}{https://agreg.skyost.eu}

\newcommand{\tr}[1]{\mathstrut ^t #1}
\newcommand{\im}{\mathrm{Im}}
\newcommand{\rang}{\operatorname{rang}}
\newcommand{\trace}{\operatorname{trace}}
\newcommand{\id}{\operatorname{id}}
\newcommand{\stab}{\operatorname{Stab}}

\providecommand{\newpar}{\\[\medskipamount]}

\providecommand{\lesson}[3]{%
	\title{#3}%
	\hypersetup{pdftitle={#3}}%
	\setcounter{section}{\numexpr #2 - 1}%
	\section{#3}%
	\fancyhead[R]{\truncate{0.73\textwidth}{#2 : #3}}%
}

\providecommand{\development}[3]{%
	\title{#3}%
	\hypersetup{pdftitle={#3}}%
	\section*{#3}%
	\fancyhead[R]{\truncate{0.73\textwidth}{#3}}%
}

\providecommand{\summary}[1]{%
	\textit{#1}%
	\medskip%
}

\tikzset{notestyleraw/.append style={inner sep=0pt, rounded corners=0pt, align=center}}

%\newcommand{\booklink}[1]{\website/bibliographie\##1}
\newcommand{\citelink}[2]{\hyperlink{cite.\therefsection @#1}{#2}}
\newcommand{\previousreference}{}
\providecommand{\reference}[2][]{%
	\notblank{#1}{\renewcommand{\previousreference}{#1}}{}%
	\todo[noline]{%
		\protect\vspace{16pt}%
		\protect\par%
		\protect\notblank{#1}{\cite{[\previousreference]}\\}{}%
		\protect\citelink{\previousreference}{p. #2}%
	}%
}

\definecolor{devcolor}{HTML}{00695c}
\newcommand{\dev}[1]{%
	\reversemarginpar%
	\todo[noline]{
		\protect\vspace{16pt}%
		\protect\par%
		\bfseries\color{devcolor}\href{\website/developpements/#1}{DEV}
	}%
	\normalmarginpar%
}

% En-têtes :

\pagestyle{fancy}
\fancyhead[L]{\truncate{0.23\textwidth}{\thepage}}
\fancyfoot[C]{\scriptsize \href{\website}{\texttt{agreg.skyost.eu}}}

% Couleurs :

\definecolor{property}{HTML}{fffde7}
\definecolor{proposition}{HTML}{fff8e1}
\definecolor{lemma}{HTML}{fff3e0}
\definecolor{theorem}{HTML}{fce4f2}
\definecolor{corollary}{HTML}{ffebee}
\definecolor{definition}{HTML}{ede7f6}
\definecolor{notation}{HTML}{f3e5f5}
\definecolor{example}{HTML}{e0f7fa}
\definecolor{cexample}{HTML}{efebe9}
\definecolor{application}{HTML}{e0f2f1}
\definecolor{remark}{HTML}{e8f5e9}
\definecolor{proof}{HTML}{e1f5fe}

% Théorèmes :

\theoremstyle{definition}
\newtheorem{theorem}{Théorème}

\newtheorem{property}[theorem]{Propriété}
\newtheorem{proposition}[theorem]{Proposition}
\newtheorem{lemma}[theorem]{Lemme}
\newtheorem{corollary}[theorem]{Corollaire}

\newtheorem{definition}[theorem]{Définition}
\newtheorem{notation}[theorem]{Notation}

\newtheorem{example}[theorem]{Exemple}
\newtheorem{cexample}[theorem]{Contre-exemple}
\newtheorem{application}[theorem]{Application}

\theoremstyle{remark}
\newtheorem{remark}[theorem]{Remarque}

\counterwithin*{theorem}{section}

\newcommand{\applystyletotheorem}[1]{
	\tcolorboxenvironment{#1}{
		enhanced,
		breakable,
		colback=#1!98!white,
		boxrule=0pt,
		boxsep=0pt,
		left=8pt,
		right=8pt,
		top=8pt,
		bottom=8pt,
		sharp corners,
		after=\par,
	}
}

\applystyletotheorem{property}
\applystyletotheorem{proposition}
\applystyletotheorem{lemma}
\applystyletotheorem{theorem}
\applystyletotheorem{corollary}
\applystyletotheorem{definition}
\applystyletotheorem{notation}
\applystyletotheorem{example}
\applystyletotheorem{cexample}
\applystyletotheorem{application}
\applystyletotheorem{remark}
\applystyletotheorem{proof}

% Environnements :

\NewEnviron{whitetabularx}[1]{%
	\renewcommand{\arraystretch}{2.5}
	\colorbox{white}{%
		\begin{tabularx}{\textwidth}{#1}%
			\BODY%
		\end{tabularx}%
	}%
}

% Maths :

\DeclareFontEncoding{FMS}{}{}
\DeclareFontSubstitution{FMS}{futm}{m}{n}
\DeclareFontEncoding{FMX}{}{}
\DeclareFontSubstitution{FMX}{futm}{m}{n}
\DeclareSymbolFont{fouriersymbols}{FMS}{futm}{m}{n}
\DeclareSymbolFont{fourierlargesymbols}{FMX}{futm}{m}{n}
\DeclareMathDelimiter{\VERT}{\mathord}{fouriersymbols}{152}{fourierlargesymbols}{147}


% Bibliographie :

\addbibresource{\bibliographypath}%
\defbibheading{bibliography}[\bibname]{%
	\newpage
	\section*{#1}%
}
\renewbibmacro*{entryhead:full}{\printfield{labeltitle}}%
\DeclareFieldFormat{url}{\newline\footnotesize\url{#1}}%

\AtEndDocument{\printbibliography}

\begin{document}
	%<*content>
	\lesson{algebra}{101}{Groupe opérant sur un ensemble. Exemples et applications.}

	Soit $G$ un groupe.

	\subsection{Actions de groupe}

	Soit $X \neq \emptyset$ un ensemble.

	\subsubsection{Cas général}

	\reference[ULM21]{29}

	\begin{definition}
		On appelle \textbf{action} (à gauche) de $G$ sur $X$ toute application
		\[
		\begin{array}{ccc}
			G \times X &\rightarrow& X \\
			(g, x) &\mapsto& g \cdot x
		\end{array}
		\]
		satisfaisant les conditions suivantes :
		\begin{enumerate}[(i)]
			\item $\forall g, h \in G$, $\forall x \in X$, $g \cdot (h \cdot x) = (gh) \cdot x$.
			\item $\forall x \in X$, $e_G \cdot x = x$.
		\end{enumerate}
	\end{definition}

	\begin{remark}
		On peut de même définir une action à droite de $G$ sur $X$.
	\end{remark}

	\begin{example}
		\begin{itemize}
			\item Le groupe $S_X$ des bijections de $X$ dans $X$ opère naturellement sur $X$ par la relation $\sigma \cdot x = \sigma(x)$ pour tout $\sigma \in S_x$ et pour tout $x \in X$.
			\item Pour un espace vectoriel $V$, le groupe $\mathrm{GL}(V)$ opère sur $V$.
		\end{itemize}
	\end{example}

	On supposera par la suite que $G$ agit sur $X$ à gauche via l'action $\cdot$.

	\begin{theorem}
		On a une correspondance bijective entre les actions de $G$ sur $X$ et les morphismes de $G$ dans $S_X$. En effet, si $\cdot$ désigne une action de $G$ sur $X$, on peut y faire correspondre le morphisme
		\[ \varphi :
		\begin{array}{ccc}
			G &\rightarrow& S_X \\
			g &\mapsto& (x \mapsto g \cdot x)
		\end{array}
		\]
	\end{theorem}

	\begin{definition}
		On définit pour tout $x \in X$ :
		\begin{itemize}
			\item $G \cdot x = \{ g \cdot x \mid g \in G \} \subseteq X$ l'\textbf{orbite de $x$}.
			\item $\stab_G(x) = \{ g \in G \mid g \cdot x = x \} < G$ le \textbf{stabilisateur de $x$}.
		\end{itemize}
		On dit que l'action de $G$ sur $X$ est :
		\begin{itemize}
			\item \textbf{Libre} si $\stab_G(x) = \{ e_G \}$ pour tout $x \in X$.
			\item \textbf{Transitive} si $G$ n'admet qu'une seule orbite.
		\end{itemize}
	\end{definition}

	\begin{example}
		L'action du groupe diédral $\mathcal{D}_3$ sur les sommets d'un triangle équilatéral est transitive mais n'est pas libre.
	\end{example}

	\begin{proposition}
		\label{101-1}
		La relation $\sim$ définie sur $X$ par
		\[ x \sim y \iff x \in G \cdot y \]
		est une relation d'équivalence dont les classes d'équivalence sont les orbites des éléments de $X$ sous l'action de $G$.
	\end{proposition}

	\reference[PER]{57}

	\begin{application}
		Toute permutation $\sigma \in S_n$ s'écrit comme produit
		\[ \sigma = \gamma_1 \dots \gamma_m \]
		de cycles $\gamma_i$ de longueur $\geq 2$ dont les supports sont deux-à-deux disjoints. Cette décomposition est unique à l'ordre près.
	\end{application}

	\reference[ULM21]{33}

	\begin{definition}
		Une action  $\varphi : G \rightarrow S_X$ une action de $G$ sur $X$ est dite \textbf{fidèle} si $\ker(\varphi) = \{ e_G \}$.
	\end{definition}

	\begin{proposition}
		Soit $\varphi : G \rightarrow S_X$ une action de $G$ sur $X$. Alors,
		\[ \ker(\varphi) = \bigcap_{x \in X} \stab_G(x) \]
	\end{proposition}

	\begin{corollary}
		Une action libre est fidèle.
	\end{corollary}

	\reference{71}

	\begin{proposition}
		Soit $x \in X$. L'application
		\[ f :
		\begin{array}{ccc}
			 G/\stab_G(x) &\rightarrow& G \cdot x \\
			 g \stab_G(x) &\mapsto& g \cdot x
		\end{array}
		\]
		est une bijection.
	\end{proposition}

	\begin{remark}
		Attention cependant, $G/\stab_G(x)$ n'est pas un groupe en général.
	\end{remark}

	\subsubsection{Cas fini}

	On suppose ici que $G$ et $X$ sont finis.

	\begin{proposition}
		Soit $x \in X$. Alors :
		\begin{itemize}
			\item $|G \cdot x| = (G : \stab_G(x))$.
			\item $|G| = |\stab_G(x)| |G \cdot x|$.
			\item $|G \cdot x| = \frac{|G|}{|\stab_G(x)|}$
		\end{itemize}
	\end{proposition}

	\begin{theorem}[Formule des classes]
		Soit $\Omega$ un système de représentants associé à la relation $\sim$ de la \cref{101-1}. Alors,
		\[ |X| = \sum_{\omega \in \Omega} |G \cdot \omega| = \sum_{\omega \in \Omega} (G : \stab_G(\omega)) = \sum_{\omega \in \Omega} \frac{|G|}{|\stab_G(\omega)|} \]
	\end{theorem}

	\begin{definition}
		On définit :
		\begin{itemize}
			\item $X^G = \{ x \in X \mid \forall g \in G, \, g \cdot x = x \}$ l'ensemble des points de $X$ laissés fixes par tous les éléments de $G$.
			\item $X^g = \{ x \in X \mid g \cdot x = x \}$ l'ensemble des points de $X$ laissés fixes par $g \in G$.
		\end{itemize}
	\end{definition}

	\begin{theorem}[Formule de Burnside]
		Le nombre $r$ d'orbites de $X$ sous l'action de $G$ est donné par
		\[ r = \frac{1}{|G|} \sum_{g \in G} |X^g| \]
	\end{theorem}

	\begin{corollary}
		Soit $p$ un nombre premier. Si $G$ est un $p$-groupe (ie. l'ordre de $G$ est une puissance de $p$), alors,
		\[ |X^G| \equiv |X| \mod p \]
	\end{corollary}

	\begin{corollary}
		Soit $p$ un nombre premier. Le centre d'un $p$-groupe non trivial est non trivial.
	\end{corollary}

	\begin{corollary}
		Soit $p$ un nombre premier. Un groupe d'ordre $p^2$ est toujours abélien.
	\end{corollary}

	\begin{application}[Théorème de Cauchy]
  	On suppose $G$ non trivial et fini. Soit $p$ un premier divisant l'ordre de $G$. Alors il existe un élément d'ordre $p$ dans $G$.
  \end{application}

	\newpage
	\subsection{Action d'un groupe sur un groupe}

	\subsubsection{Action par translation}

	\reference{34}

	\begin{proposition}
		$G$ agit sur lui-même par translation (à gauche) via l'action
		\[ (g, h) \mapsto g \cdot h = gh \]
		De plus, cette action est fidèle et transitive.
	\end{proposition}

	\begin{application}[Théorème de Cayley]
		Tout groupe fini d'ordre $n$ est isomorphe à un sous-groupe de $S_n$.
	\end{application}

	\begin{proposition}
		Soit $H < G$. Alors $G$ agit sur $G/H$ via l'action
		\[ (g, hH) \mapsto g \cdot hH = (gh)H \]
		De plus, cette action est transitive.
	\end{proposition}

	\begin{proposition}
		Soit $H < G$. Soit $\varphi : G \rightarrow S_{G/H}$ le morphisme de l'action par translation de $G$ sur $G/H$. Alors,
		\[ \ker(\varphi) = \bigcap_{g \in G} gHg^{-1} \]
	\end{proposition}

	\reference[PER]{17}

	\begin{application}
		On suppose que $G$ est de cardinal infini et que $G$ possède un sous-groupe d'indice fini distinct de $G$. Alors $G$ n'est pas simple.
	\end{application}

	\subsubsection{Action par conjugaison}

	\reference[ULM21]{36}

	\begin{proposition}
		$G$ agit sur lui-même par conjugaison via l'action
		\[ (g, h) \mapsto g \cdot h = ghg^{-1} \]
	\end{proposition}

	\begin{definition}
		\begin{itemize}
			\item L'orbite de $g \in G$ sous l'action par conjugaison de $G$ sur lui-même s'appelle la \textbf{classe de conjugaison de $g$}.
			\item Le stabilisateur de $g \in G$ sous l'action par conjugaison de $G$ sur lui-même s'appelle le \textbf{centralisateur de $g$}.
			\item Deux éléments de $G$ qui appartiennent à la même classe de conjugaison sont dits \textbf{conjugués}.
		\end{itemize}
	\end{definition}

	\reference[PER]{15}

	\begin{example}
		\begin{itemize}
			\item Si $\sigma = \begin{pmatrix} a_1 & \dots & a_p \end{pmatrix} \in S_n$ est un $p$-cycle, et si $\tau \in S_n$, alors
			\[ \tau \sigma \tau^{-1} = \begin{pmatrix} \tau(a_1) & \dots & \tau(a_p) \end{pmatrix} \]
			\item Par conséquent, dans $S_n$, les $p$-cycles sont conjugués.
			\item Pour $n \geq 5$, les $3$-cycles sont conjugués dans $A_n$.
		\end{itemize}
	\end{example}

	\reference[ULM21]{36}

	\begin{proposition}
		Soit $g \in G$. Alors $g$ appartient au centre de $G$ (noté $Z(G)$) si et seulement si sa classe de conjugaison est réduite à un seul élément.
	\end{proposition}

	\begin{corollary}
		$Z(G)$ est l'union des classes de conjugaison de taille $1$.
	\end{corollary}

	\reference[GOU21]{24}

	\begin{proposition}
		Soit $\Omega$ un système de représentants associé à la relation $\sim$ de la \cref{101-1} pour l'action par conjugaison. On note $\Omega' = Z(G) \setminus \Omega$. Alors,
		\[ |G| = |Z(G)| + \sum_{\omega \in \Omega'} \frac{|G|}{|\stab_G(\omega)|} \]
	\end{proposition}

	\reference{100}
	\dev{theoreme-de-wedderburn}

	\begin{application}[Théorème de Wedderburn]
		Tout corps fini est commutatif.
	\end{application}

	\reference[ULM21]{38}

	\begin{proposition}
		\label{101-2}
		$G$ agit sur ses sous-groupes par conjugaison via l'action
		\[ (g, H) \mapsto g \cdot H = gHg^{-1} \]
	\end{proposition}

	\begin{proposition}
		Soit $H < G$. Alors $H$ est distingué dans $G$ si et seulement si $H$ est un point fixe pour l'action de la \cref{101-2}.
	\end{proposition}

	\subsection{Action d'un groupe sur un espace vectoriel}

	\subsubsection{Action par conjugaison sur les espaces de matrices}

	Soit $E$ un espace vectoriel de dimension finie $n$ sur un corps $\mathbb{K}$.

	\reference[D-L]{137}

	\begin{theorem}
		Soit $f \in \mathcal{L}(E)$. Soient $\mathcal{B}_1$ et $\mathcal{B}_2$ deux bases de $E$. Alors,
		\[ \operatorname{Mat}(f, \mathcal{B}_1) = P \operatorname{Mat}(f, \mathcal{B}_2) P^{-1} \]
		où $P \in \mathrm{GL}_n(\mathbb{K})$ est la matrice de passage de $\mathcal{B}_1$ vers $\mathcal{B}_2$.
	\end{theorem}

	C'est ce théorème qui justifie que l'on va étudier l'action par conjugaison de $\mathrm{GL}_n(\mathbb{K})$ sur $\mathcal{M}_n(\mathbb{K})$.

	\begin{definition}
		Deux matrices qui sont dans le même orbite pour cette action sont dites \textbf{semblables}.
	\end{definition}

	\begin{theorem}
		Soient $A$ et $B$ deux matrices semblables. Alors :
		\begin{itemize}
			\item $\trace(A) = \trace(B)$.
			\item $\det(A) = \det(B)$.
			\item $\rang(A) = \rang(B)$.
			\item $\chi_A = \chi_B$.
			\item $\pi_A = \pi_B$.
		\end{itemize}
	\end{theorem}

	\begin{cexample}
		Les matrices $\begin{pmatrix} 0 & 0 \\ 0 & 0\end{pmatrix}$ et $\begin{pmatrix} 0 & 1 \\ 0 & 0\end{pmatrix}$ ont la même trace, le même déterminant, le même polynôme caractéristique, mais ne sont pas semblables.
	\end{cexample}

	\reference[GOU21]{397}

	\begin{notation}
		Soient $f \in \mathcal{L}(E)$ et $x \in E$. On note $P_{f,x}$ le polynôme unitaire engendrant l'idéal $\{ P \in \mathbb{K}[X] \mid P(f)(x) = 0 \}$ et $E_{f,x} = \{ P(f)(x) \mid P \in \mathbb{K}[X] \}$.
	\end{notation}

	\begin{lemma}
		Soit $f \in \mathcal{L}(E)$.
		\begin{enumerate}[(i)]
			\item Si $k = \deg(\pi_f)$, alors $\mathbb{K}[f]$ est un sous-espace vectoriel de $\mathcal{L}(E)$ de dimension $k$, dont une base est $(f^i)_{i \in \llbracket 0, k-1 \rrbracket}$.
			\item Soit $x \in E$. Si $l = \deg(P_{f,x})$, alors $E_x$ est un sous-espace vectoriel de $E$ de dimension $l$, dont une base est $(f^i(x))_{i \in \llbracket 0, l-1 \rrbracket}$.
		\end{enumerate}
	\end{lemma}

	\begin{lemma}
		Soit $f \in \mathcal{L}(E)$. Il existe $x \in E$ tel que $P_{f,x} = \pi_f$.
	\end{lemma}

	\dev{invariants-de-similitude}

	\begin{theorem}[Frobenius]
		Soit $f \in \mathcal{L}(E)$. Il existe des sous-espaces vectoriels $F_1, \dots, F_r$ de $E$ tous stables par $f$ tels que :
		\begin{enumerate}[(i)]
			\item $E = \bigoplus_{i = 1}^r F_i$.
			\item $\forall i \in \llbracket 1, r \rrbracket$, la restriction $f_i = f_{|F_i}$ est un endomorphisme cyclique de $F_i$.
			\item Si $P_i = \pi_{f_i}$ est le polynôme minimal de $f_i$, on a $P_{i+1} \mid P_i$ $\forall i \in \llbracket 1, r-1 \rrbracket$.
		\end{enumerate}
		La suite $(P_i)_{i \in \llbracket 1, r \rrbracket}$ ne dépend que de $f$ et non du choix de la décomposition (elle est donc unique). On l'appelle \textbf{suite des invariants de $f$}.
	\end{theorem}

	\begin{corollary}
		Deux endomorphismes sont semblables si et seulement s'ils ont les mêmes invariants de similitude.
	\end{corollary}

	\subsubsection{Représentations linéaires et caractères}

	On suppose ici que $G$ est fini.

	\reference[SER]{15}

	\begin{definition}
		Une \textbf{représentation linéaire de $G$} est un morphisme $\rho$ de $G$ dans $\mathrm{GL}(V)$ où $V$ désigne un espace vectoriel de dimension finie sur le corps $\mathbb{C}$.
	\end{definition}

	\reference[ULM21]{143}

	\begin{remark}
		Ceci correspond à la donnée d'une action de $G$ sur $V$ :
		\[ (g, v) \mapsto g \cdot v = \rho(g)(v) \]
	\end{remark}

	\reference[SER]{15}

	\begin{example}
		\begin{itemize}
			\item La \textbf{représentation triviale} de $G$ est donnée par $g \mapsto 1$.
			\item Soit $V$ un espace vectoriel de dimension $|G|$ ayant une base $(e_g)_{g \in G}$ indexée par les éléments de $G$. La \textbf{représentation régulière} $\rho_r$ de $G$ est donnée par $g \mapsto \rho_r(g)$ où $\rho_r(g)$ est l'application linéaire de $V$ dans $V$ qui transforme $e_h$ en $e_{gh}$.
		\end{itemize}
	\end{example}

	\begin{definition}
		Soit $\rho : G \rightarrow \mathrm{GL}(V)$ une représentation linéaire de $G$ et soit $W$ un sous-espace vectoriel de $V$ stable par $\rho(g)$ pour tout $g \in G$. Alors $\rho$ induit une représentation linéaire de $G$ par restriction à $W$ notée $\rho_W$. On dit alors que $\rho_W$ est une \textbf{sous-représentation} de $\rho$.
	\end{definition}

	\reference{19}

	\begin{definition}
		Soit $\rho : G \rightarrow \mathrm{GL}(V)$ une représentation linéaire de $G$. On dit $\rho$ est \textbf{irréductible} si elle admet uniquement deux sous-représentations : $\rho$ elle-même et $\rho_{\{ 0 \}}$.
	\end{definition}

	\begin{definition}
		Soit $\rho : G \rightarrow \mathrm{GL}(V)$ une représentation linéaire de $G$. On suppose $V = W \oplus W_0$ avec $W$ et $W_0$ stables par $\rho(g)$ pour tout $g \in G$. On dit alors que $\rho$ est \textbf{somme directe} de $\rho_W$ et de $\rho_{W_0}$.
	\end{definition}

	\begin{theorem}[Maschke]
		Toute représentation linéaire de $G$ est somme directe de représentations irréductibles.
	\end{theorem}

	\reference{23}

	\begin{definition}
		Soit $\rho : G \rightarrow \mathrm{GL}(V)$ une représentation linéaire de $G$. On associe à $\rho$ son \textbf{caractère} $\chi_\rho : g \mapsto \trace(\rho(g))$. On dit que $\chi_{\rho}$ est \textbf{irréductible} si $\rho$ l'est.
	\end{definition}

	\reference[ULM21]{151}

	\begin{example}
		\[ \forall g \in G, \, \chi_{\rho_r}(g) = \begin{cases} |G| &\text{si } g = e_G \\ 0 &\text{sinon} \end{cases} \]
	\end{example}

	\reference[SER]{32}

	\begin{theorem}
		\begin{itemize}
			\item Les caractères forment une base orthonormale de l'espace des fonctions centrales pour le produit scalaire $\langle ., . \rangle$ défini par
			\[ \forall \chi, \Psi \text{ caractères}, \, \langle \chi, \Psi \rangle = \frac{1}{|G|} \sum_{g \in G} \chi(g) \overline{\Psi(g)} \]
			\item Un caractère $\chi$ est irréductible si et seulement si $\langle \chi, \chi \rangle = 1$.
			\item Il y a autant de caractères irréductibles que de classes de conjugaison.
		\end{itemize}
	\end{theorem}

	\begin{theorem}
		Si $\rho_1 : G \rightarrow \mathrm{GL}(V_1), \dots, \rho_s : G \rightarrow \mathrm{GL}(V_s)$ sont les représentations linéaires irréductibles de $G$,
		\[ \forall i \in \llbracket 1, s \rrbracket, \, |G| = \sum_{i=1}^n \dim(V_i)^2 = \sum_{i=1}^n \chi_{\rho_i}(e_G)^2 = \langle \chi_{\rho_r}, \chi_{\rho_i} \rangle \]
	\end{theorem}

	\reference[I-P]{73}

	\begin{application}
		La table des caractères de $S_4$ est :

		\begin{center}
			\begin{whitetabularx}{|l|X|X|X|X|X|}
				\hline
				& $\id$ & $\begin{pmatrix} 1 & 2 \end{pmatrix}$ & $\begin{pmatrix} 1 & 2 & 3 \end{pmatrix}$ & $\begin{pmatrix} 1 & 2 & 3 & 4 \end{pmatrix}$ & $\begin{pmatrix} 1 & 2 \end{pmatrix} \begin{pmatrix} 3 & 4 \end{pmatrix}$ \\
				\hline
				$\chi_1$ & $1$ & $1$ & $1$ & $1$ & $1$ \\
				\hline
				$\chi_2$ & $1$ & $-1$ & $1$ & $-1$ & $1$ \\
				\hline
				$\chi_3$ & $2$ & $0$ & $-1$ & $0$ & $2$ \\
				\hline
				$\chi_4$ & $3$ & $1$ & $0$ & $-1$ & $-1$ \\
				\hline
				$\chi_5$ & $3$ & $-1$ & $0$ & $1$ & $-1$ \\
				\hline
			\end{whitetabularx}
		\end{center}
	\end{application}
	%</content>
\end{document}
