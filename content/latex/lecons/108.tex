\documentclass[12pt, a4paper]{report}

% LuaLaTeX :

\RequirePackage{iftex}
\RequireLuaTeX

% Packages :

\usepackage[french]{babel}
%\usepackage[utf8]{inputenc}
%\usepackage[T1]{fontenc}
\usepackage[pdfencoding=auto, pdfauthor={Hugo Delaunay}, pdfsubject={Mathématiques}, pdfcreator={agreg.skyost.eu}]{hyperref}
\usepackage{amsmath}
\usepackage{amsthm}
%\usepackage{amssymb}
\usepackage{stmaryrd}
\usepackage{tikz}
\usepackage{tkz-euclide}
\usepackage{fourier-otf}
\usepackage{fontspec}
\usepackage{titlesec}
\usepackage{fancyhdr}
\usepackage{catchfilebetweentags}
\usepackage[french, capitalise, noabbrev]{cleveref}
\usepackage[fit, breakall]{truncate}
\usepackage[top=2.5cm, right=2cm, bottom=2.5cm, left=2cm]{geometry}
\usepackage{enumerate}
\usepackage{tocloft}
\usepackage{microtype}
%\usepackage{mdframed}
%\usepackage{thmtools}
\usepackage{xcolor}
\usepackage{tabularx}
\usepackage{aligned-overset}
\usepackage[subpreambles=true]{standalone}
\usepackage{environ}
\usepackage[normalem]{ulem}
\usepackage{marginnote}
\usepackage{etoolbox}
\usepackage{setspace}
\usepackage[bibstyle=reading, citestyle=draft]{biblatex}
\usepackage{xpatch}
\usepackage[many, breakable]{tcolorbox}
\usepackage[backgroundcolor=white, bordercolor=white, textsize=small]{todonotes}

% Bibliographie :

\newcommand{\overridebibliographypath}[1]{\providecommand{\bibliographypath}{#1}}
\overridebibliographypath{../bibliography.bib}
\addbibresource{\bibliographypath}
\defbibheading{bibliography}[\bibname]{%
	\newpage
	\section*{#1}%
}
\renewbibmacro*{entryhead:full}{\printfield{labeltitle}}
\DeclareFieldFormat{url}{\newline\footnotesize\url{#1}}
\AtEndDocument{\printbibliography}

% Police :

\setmathfont{Erewhon Math}

% Tikz :

\usetikzlibrary{calc}

% Longueurs :

\setlength{\parindent}{0pt}
\setlength{\headheight}{15pt}
\setlength{\fboxsep}{0pt}
\titlespacing*{\chapter}{0pt}{-20pt}{10pt}
\setlength{\marginparwidth}{1.5cm}
\setstretch{1.1}

% Métadonnées :

\author{agreg.skyost.eu}
\date{\today}

% Titres :

\setcounter{secnumdepth}{3}

\renewcommand{\thechapter}{\Roman{chapter}}
\renewcommand{\thesubsection}{\Roman{subsection}}
\renewcommand{\thesubsubsection}{\arabic{subsubsection}}
\renewcommand{\theparagraph}{\alph{paragraph}}

\titleformat{\chapter}{\huge\bfseries}{\thechapter}{20pt}{\huge\bfseries}
\titleformat*{\section}{\LARGE\bfseries}
\titleformat{\subsection}{\Large\bfseries}{\thesubsection \, - \,}{0pt}{\Large\bfseries}
\titleformat{\subsubsection}{\large\bfseries}{\thesubsubsection. \,}{0pt}{\large\bfseries}
\titleformat{\paragraph}{\bfseries}{\theparagraph. \,}{0pt}{\bfseries}

\setcounter{secnumdepth}{4}

% Table des matières :

\renewcommand{\cftsecleader}{\cftdotfill{\cftdotsep}}
\addtolength{\cftsecnumwidth}{10pt}

% Redéfinition des commandes :

\renewcommand*\thesection{\arabic{section}}
\renewcommand{\ker}{\mathrm{Ker}}

% Nouvelles commandes :

\newcommand{\website}{https://agreg.skyost.eu}

\newcommand{\tr}[1]{\mathstrut ^t #1}
\newcommand{\im}{\mathrm{Im}}
\newcommand{\rang}{\operatorname{rang}}
\newcommand{\trace}{\operatorname{trace}}
\newcommand{\id}{\operatorname{id}}
\newcommand{\stab}{\operatorname{Stab}}

\providecommand{\newpar}{\\[\medskipamount]}

\providecommand{\lesson}[3]{%
	\title{#3}%
	\hypersetup{pdftitle={#3}}%
	\setcounter{section}{\numexpr #2 - 1}%
	\section{#3}%
	\fancyhead[R]{\truncate{0.73\textwidth}{#2 : #3}}%
}

\providecommand{\development}[3]{%
	\title{#3}%
	\hypersetup{pdftitle={#3}}%
	\section*{#3}%
	\fancyhead[R]{\truncate{0.73\textwidth}{#3}}%
}

\providecommand{\summary}[1]{%
	\textit{#1}%
	\medskip%
}

\tikzset{notestyleraw/.append style={inner sep=0pt, rounded corners=0pt, align=center}}

%\newcommand{\booklink}[1]{\website/bibliographie\##1}
\newcommand{\citelink}[2]{\hyperlink{cite.\therefsection @#1}{#2}}
\newcommand{\previousreference}{}
\providecommand{\reference}[2][]{%
	\notblank{#1}{\renewcommand{\previousreference}{#1}}{}%
	\todo[noline]{%
		\protect\vspace{16pt}%
		\protect\par%
		\protect\notblank{#1}{\cite{[\previousreference]}\\}{}%
		\protect\citelink{\previousreference}{p. #2}%
	}%
}

\definecolor{devcolor}{HTML}{00695c}
\newcommand{\dev}[1]{%
	\reversemarginpar%
	\todo[noline]{
		\protect\vspace{16pt}%
		\protect\par%
		\bfseries\color{devcolor}\href{\website/developpements/#1}{DEV}
	}%
	\normalmarginpar%
}

% En-têtes :

\pagestyle{fancy}
\fancyhead[L]{\truncate{0.23\textwidth}{\thepage}}
\fancyfoot[C]{\scriptsize \href{\website}{\texttt{agreg.skyost.eu}}}

% Couleurs :

\definecolor{property}{HTML}{fffde7}
\definecolor{proposition}{HTML}{fff8e1}
\definecolor{lemma}{HTML}{fff3e0}
\definecolor{theorem}{HTML}{fce4f2}
\definecolor{corollary}{HTML}{ffebee}
\definecolor{definition}{HTML}{ede7f6}
\definecolor{notation}{HTML}{f3e5f5}
\definecolor{example}{HTML}{e0f7fa}
\definecolor{cexample}{HTML}{efebe9}
\definecolor{application}{HTML}{e0f2f1}
\definecolor{remark}{HTML}{e8f5e9}
\definecolor{proof}{HTML}{e1f5fe}

% Théorèmes :

\theoremstyle{definition}
\newtheorem{theorem}{Théorème}

\newtheorem{property}[theorem]{Propriété}
\newtheorem{proposition}[theorem]{Proposition}
\newtheorem{lemma}[theorem]{Lemme}
\newtheorem{corollary}[theorem]{Corollaire}

\newtheorem{definition}[theorem]{Définition}
\newtheorem{notation}[theorem]{Notation}

\newtheorem{example}[theorem]{Exemple}
\newtheorem{cexample}[theorem]{Contre-exemple}
\newtheorem{application}[theorem]{Application}

\theoremstyle{remark}
\newtheorem{remark}[theorem]{Remarque}

\counterwithin*{theorem}{section}

\newcommand{\applystyletotheorem}[1]{
	\tcolorboxenvironment{#1}{
		enhanced,
		breakable,
		colback=#1!98!white,
		boxrule=0pt,
		boxsep=0pt,
		left=8pt,
		right=8pt,
		top=8pt,
		bottom=8pt,
		sharp corners,
		after=\par,
	}
}

\applystyletotheorem{property}
\applystyletotheorem{proposition}
\applystyletotheorem{lemma}
\applystyletotheorem{theorem}
\applystyletotheorem{corollary}
\applystyletotheorem{definition}
\applystyletotheorem{notation}
\applystyletotheorem{example}
\applystyletotheorem{cexample}
\applystyletotheorem{application}
\applystyletotheorem{remark}
\applystyletotheorem{proof}

% Environnements :

\NewEnviron{whitetabularx}[1]{%
	\renewcommand{\arraystretch}{2.5}
	\colorbox{white}{%
		\begin{tabularx}{\textwidth}{#1}%
			\BODY%
		\end{tabularx}%
	}%
}

% Maths :

\DeclareFontEncoding{FMS}{}{}
\DeclareFontSubstitution{FMS}{futm}{m}{n}
\DeclareFontEncoding{FMX}{}{}
\DeclareFontSubstitution{FMX}{futm}{m}{n}
\DeclareSymbolFont{fouriersymbols}{FMS}{futm}{m}{n}
\DeclareSymbolFont{fourierlargesymbols}{FMX}{futm}{m}{n}
\DeclareMathDelimiter{\VERT}{\mathord}{fouriersymbols}{152}{fourierlargesymbols}{147}


% Bibliographie :

\addbibresource{\bibliographypath}%
\defbibheading{bibliography}[\bibname]{%
	\newpage
	\section*{#1}%
}
\renewbibmacro*{entryhead:full}{\printfield{labeltitle}}%
\DeclareFieldFormat{url}{\newline\footnotesize\url{#1}}%

\AtEndDocument{\printbibliography}

\begin{document}
	%<*content>
	\lesson{algebra}{108}{Exemples de parties génératrices d'un groupe. Applications.}

	\subsection{Généralités}

	Soit $G$ un groupe.

	\subsubsection{Définitions}

	\reference[ROM21]{10}

	\begin{lemma}
		Une intersection (quelconque) de sous-groupes de $G$ est un sous-groupe de $G$.
	\end{lemma}

	\begin{definition}
		Soit $X \subseteq G$. On appelle \textbf{sous-groupe engendré} par $X$, le plus petit sous-groupe (pour l'inclusion) de $G$ contenant $X$. C'est l'intersection des sous-groupes de $G$ contenant $X$. On le note $\langle X \rangle$ ou $\langle x_1, \dots, x_n$ si $X = \{ x_1, \dots, x_n \}$.
	\end{definition}

	\begin{proposition}
		Soit $X \subseteq G$. On pose $X^{-1} = \{ x^{-1} \mid x \in X \}$. Alors,
		\[ \langle X \rangle = \{ x_1 \dots x_n \mid (x_1 \dots x_n) \in X \, \cup \, X^{-1}, \, n \in \mathbb{N}^* \} \]
	\end{proposition}

	\begin{definition}
		Une \textbf{partie génératrice} de $G$ est un sous-ensemble $X \subseteq G$ tel que $G = \langle X \rangle$.
	\end{definition}

	\begin{example}
		Soit $D_G$ l'ensemble des commutateurs de $G$ (ie. éléments de la forme $ghg^{-1}$ pour $g,h \in G$). On pose $D(G) = \langle D_G \rangle$ : $D(G)$ est le groupe dérivé de $G$, c'est le plus grand sous-groupe tel que $G/D(G)$ est abélien.
	\end{example}

	\subsubsection{Groupes monogènes}

	\begin{definition}
		On dit que $G$ est \textbf{monogène} s'il existe $g \in G$ tel que $G = \langle g \rangle$, et on dit que $G$ est cyclique s'il est monogène et fini.
	\end{definition}

	\begin{example}
		\begin{enumerate}[label=(\roman*)]
			\item $\mathbb{Z}$ est monogène, l'ensemble de ses générateurs est $\mathbb{Z}^\times = \{ \pm 1 \}$.
			\item $\mathbb{Z}/n\mathbb{Z}$, l'ensemble de ses générateurs est $(\mathbb{Z}/n\mathbb{Z})^\times = \{ k \in \mathbb{Z}/n\mathbb{Z} \mid \operatorname{pgcd}(k,n) = 1 \}$.
		\end{enumerate}
	\end{example}

	\begin{theorem}
		\begin{enumerate}[label=(\roman*)]
			\item Si $G$ est monogène infini, alors $G \cong \mathbb{Z}$.
			\item Si $G$ est cyclique d'ordre $n$, alors $G \cong \mathbb{Z}/n\mathbb{Z}$.
		\end{enumerate}
	\end{theorem}

	\begin{corollary}
		Si $G = \langle g \rangle$ est cyclique d'ordre $n$, alors l'ensemble de ses générateurs est $\{ g^k \mid \operatorname{pgcd}(k,n) = 1 \}$.
	\end{corollary}

	\reference{6}

	\begin{definition}
		L'ordre d'un élément $g \in G$ est le cardinal de l'ensemble $\langle g \rangle$.
	\end{definition}

	\begin{remark}
		$g \in G$ est d'ordre $n$ si et seulement si $g^n = e_G$ et $g^k \neq e_G$ pour tout $k \in \llbracket 1,n-1 \rrbracket$.
	\end{remark}

	\reference{14}

	\begin{proposition}
		Un groupe de cardinal premier est cyclique.
	\end{proposition}

	\begin{theorem}
		On suppose $G = \langle g \rangle$ cyclique d'ordre $n$.
		\begin{enumerate}[label=(\roman*)]
			\item Les sous-groupes de $G$ sont cycliques d'ordre divisant $n$.
			\item Pour tout diviseur $d$ de $n$, il existe un unique sous-groupe d'ordre $d$ : $\langle g^{\frac{n}{d}} \rangle$.
		\end{enumerate}
	\end{theorem}

	\begin{remark}
		Le résultat précédent est en fait caractéristique des groupes cycliques.
	\end{remark}

	\subsubsection{Structure des groupes abéliens de type fini}

	\reference[ULM21]{105}

	On suppose dans cette sous-section que $G$ est abélien.

	\begin{definition}
		$G$ est dit de \textbf{type fini} s'il existe une partie génératrice finie de $G$.
	\end{definition}

	\reference{112}

	\begin{theorem}[Kronecker]
		Soit $G$ un groupe abélien d'ordre $n \geq 2$. Il existe $r \in \mathbb{N}$ et une suite d'entiers $n_1 \geq 2$, $n_2$ multiple de $n_1$, \dots, $n_k$ multiple de $n_{k-1}$ telle que $G$ est isomorphe au groupe produit
		\[ \prod_{i=1}^k \mathbb{Z}/n_i\mathbb{Z} \times Z^r \]
	\end{theorem}

	\begin{example}
		Si $G = \mathbb{Z}/5\mathbb{Z} \times \mathbb{Z}/5\mathbb{Z} \times \mathbb{Z}/90\mathbb{Z}$. Alors,
		\begin{align*}
			G &\cong \mathbb{Z}/5\mathbb{Z} \times \mathbb{Z}/5\mathbb{Z} \times (\mathbb{Z}/2\mathbb{Z} \times \mathbb{Z}/3^2\mathbb{Z} \times \mathbb{Z}/5\mathbb{Z}) \\
			&\cong \mathbb{Z}/2\mathbb{Z} \times \mathbb{Z}/3^2\mathbb{Z} \times (\mathbb{Z}/5\mathbb{Z} \times \mathbb{Z}/5\mathbb{Z} \times \mathbb{Z}/5\mathbb{Z})
		\end{align*}
	\end{example}

	\newpage
	\subsection{Exemples de parties génératrices}

	\subsubsection{Groupe symétrique}

	\reference[ROM21]{37}

	\begin{definition}
		Soit $E$ un ensemble. On appelle \textbf{groupe des permutations} de $E$ le groupe des bijections de $E$ dans lui-même. On le note $S(E)$.
	\end{definition}

	\begin{notation}
		Si $E = \llbracket 1, n \rrbracket$, on note $S(E) = S_n$, le groupe symétrique à $n$ éléments.
	\end{notation}

	\begin{notation}
		Soit $\sigma \in S_n$. On note :
		\[
		\sigma =
		\begin{pmatrix}
			1 & 2 & \dots & n \\
			\sigma(1) & \sigma(2) & \dots & \sigma(n)
		\end{pmatrix}
		\]
		pour signifier que $\sigma$ est la bijection $\sigma : k \mapsto \sigma(k)$.
	\end{notation}

	\begin{definition}
		Soient $l \leq n$ et $i_1, \dots, i_l \in \llbracket 1, n \rrbracket$ des éléments distincts. La permutation $\gamma \in S_n$ définie par
		\[
		\gamma(j) =
		\begin{cases}
			j &\text{si } j \not \{ i_1, \dots, i_l \} \\
			i_{k+1} &\text{si } j = i_k \text{ avec } k<l \\
			i_1 &\text{si } j=i_l
		\end{cases}
		\]
		et notée $\begin{pmatrix} i_1 & \dots & i_l \end{pmatrix}$ est appelée \textbf{cycle} de longueur $l$ et de \textbf{support} $\{ i_1, \dots, i_l \}$. Un cycle de longueur $2$ est une \textbf{transposition}.
	\end{definition}

	\reference{44}

	\begin{proposition}
		\begin{enumerate}[label=(\roman*)]
			\item $S_n$ est engendré par les transpositions. On peut même se limiter aux transpositions de la forme $\begin{pmatrix} 1 & k \end{pmatrix}$ ou encore  $\begin{pmatrix} k & k+1 \end{pmatrix}$ (pour $k \leq n$).
			\item $S_n$ est engendré par $\begin{pmatrix} 1 & 2 \end{pmatrix}$ et $\begin{pmatrix} 1 & \dots & n \end{pmatrix}$.
		\end{enumerate}
	\end{proposition}

	\reference{48}

	\begin{definition}
		\begin{itemize}
			\item Soit $\sigma \in S_n$. On appelle \textbf{signature} de $\sigma$, notée $\epsilon(\sigma)$ l'entier $\epsilon(\sigma) = \prod_{i \neq j} \frac{\sigma(i) - \sigma(j)}{i-j}$.
			\item $\sigma \mapsto \epsilon(\sigma)$ est un morphisme de $S_n$ dans $\{ \pm 1 \}$, on note $A_n$ son noyau.
		\end{itemize}
	\end{definition}

	\reference[PER]{15}

	\begin{lemma}
		Les $3$-cycles sont conjugués dans $A_n$ pour $n \geq 5$.
	\end{lemma}

	\reference[ROM21]{49}

	\begin{lemma}
		Le produit de deux transpositions est un produit de $3$-cycles.
	\end{lemma}

	\begin{proposition}
		$A_n$ est engendré par les $3$-cycles pour $n \geq 3$.
	\end{proposition}

	\reference[PER]{28}
	\dev{simplicite-du-groupe-alterne}

	\begin{theorem}
		$A_n$ est simple pour $n \geq 5$.
	\end{theorem}

	\begin{corollary}
		Le groupe dérivé de $A_n$ est $A_n$ pour $n \geq 5$, et le groupe dérivé de $S_n$ est $A_n$ pour $n \geq 2$.
	\end{corollary}

	\subsubsection{Groupe diédral}

	\reference[ULM21]{8}

	\begin{definition}
		Pour un entier $n \geq 1$, le \textbf{groupe diédral} $D_n$ est le sous-groupe, de $\mathrm{GL}_2(\mathbb{R})$ engendré par la symétrie axiale $s$ et la rotation d'angle $\theta = \frac{2\pi}{n}$ définies respectivement par les matrices
		\[
		S =
		\begin{pmatrix}
			1 & 0 \\
			0 & -1
		\end{pmatrix}
		\text{ et }
		R =
		\begin{pmatrix}
			\cos(\theta) & -\sin(\theta) \\
			\sin(\theta) & \cos(\theta)
		\end{pmatrix}
		\]
	\end{definition}

	\begin{example}
		$D_1 = \{ \operatorname{id}, s \}$.
	\end{example}

	\begin{proposition}
		\begin{enumerate}[label=(\roman*)]
			\item $D_n$ est un groupe d'ordre $2n$.
			\item $r^n = s^2 = \operatorname{id}$ et $sr = r^{-1}s$.
		\end{enumerate}
	\end{proposition}

	\reference{28}

	\begin{proposition}
		Un groupe non cyclique d'ordre $4$ est isomorphe à $D_2$.
	\end{proposition}

	\reference{65}

	\begin{example}
		$S_2$ est isomorphe à $D_2$.
	\end{example}

	\reference{28}

	\begin{proposition}
		Un groupe fini d'ordre $2p$ avec $p$ premier est soit cyclique, soit isomorphe à $D_p$.
	\end{proposition}

	\begin{example}
		$S_3$ est isomorphe à $D_3$.
	\end{example}

	\reference{47}

	\begin{proposition}
		Les sous-groupes de $D_n$ sont soit cyclique, soit isomorphes à un $D_m$ où $m \mid n$.
	\end{proposition}

	\subsection{Applications en algèbre linéaire}

	\subsubsection{Groupe linéaire}

	\reference[PER]{97}

	\begin{proposition}
		Soit $u \in \mathrm{GL}(E) \setminus \{ \operatorname{id}_E \}$. Soit $H$ un hyperplan de $E$ tel que $u_{|H} = \operatorname{id}_H$. Les assertions suivantes sont équivalentes :
		\begin{enumerate}[label=(\roman*)]
			\item $\det(u) = 1$.
			\item $u$ n'est pas diagonalisable.
			\item $\im(u - \operatorname{id}_E) \subseteq H$.
			\item Le morphisme induit $\overline{u} : E/H \rightarrow E/H$ est l'identité de $E/H$.
			\item En notant $H = \ker(f)$ (où $f$ désigne une forme linéaire sur $E$), il existe $a \in H \setminus \{ 0 \}$ tel que
			\[ u = \operatorname{id}_E + f \cdot a \]
			\item Dans une base adaptée, la matrice de $u$ s'écrit
			\[
			\begin{pmatrix}
				I_{n-2} & 0 & 0 \\
				0 & 1 & 1 \\
				0 & 0 & 1
			\end{pmatrix}
			\]
		\end{enumerate}
	\end{proposition}

	\begin{definition}
		En reprenant les notations précédentes, on dit que $u$ est une \textbf{transvection} d'hyperplan $H$ et de droite $\operatorname{Vect}(a)$.
	\end{definition}

	\begin{proposition}
		Soient $u \in \mathrm{GL}(E)$ et $\tau$ une transvection d'hyperplan $H$ et de droite $D$. Alors, $u \tau u^{-1}$ est une transvection d'hyperplan $u(H)$ et de droite $u(D)$.
	\end{proposition}

	\begin{theorem}
		Si $n \geq 2$, les transvections engendrent $\mathrm{SL}(E)$.
	\end{theorem}

	\begin{proposition}
		Soit $u \in \mathrm{GL}(E)$. Soit $H$ un hyperplan de $E$ tel que $u_{|H} = \operatorname{id}_H$. Les assertions suivantes sont équivalentes :
		\begin{enumerate}[label=(\roman*)]
			\item $\det(u) = \lambda \neq 1$.
			\item $u$ admet une valeur propre $\lambda \neq 1$.
			\item $\im(u - \operatorname{id}_E) \not\subseteq H$.
			\item Dans une base adaptée, la matrice de $u$ s'écrit
			\[
			\begin{pmatrix}
				I_{n-1} & 0 \\
				0 & \lambda
			\end{pmatrix}
			\]
			avec $\lambda \neq 1$.
		\end{enumerate}
	\end{proposition}

	\begin{theorem}
		Si $n \geq 2$, les transvections et les dilatations engendrent $\mathrm{GL}(E)$.
	\end{theorem}

	\reference[I-P]{203}

	\begin{notation}
		Soit $a \in \mathbb{F}_p$. On note $\left( \frac{a}{p} \right)$ le symbole de Legendre de $a$ modulo $p$.
	\end{notation}

	\begin{lemma}
		Soient $p \geq 3$ un nombre premier et $V$ un espace vectoriel sur $\mathbb{F}_p$ de dimension finie. Les dilatations engendrent $\mathrm{GL}(V)$.
	\end{lemma}

	\dev{theoreme-de-frobenius-zolotarev}

	\begin{application}[Théorème de Frobenius-Zolotarev]
		Soient $p \geq 3$ un nombre premier et $V$ un espace vectoriel sur $\mathbb{F}_p$ de dimension finie.
		\[ \forall u \in \mathrm{GL}(V), \, \epsilon(u) = \left( \frac{\det(u)}{p} \right) \]
		où $u$ est vu comme une permutation des éléments de $V$.
	\end{application}

	\subsubsection{Groupe orthogonal}

	\reference[PER]{123}

	Soit $E$ un espace vectoriel réel de dimension $n$. Soit $\varphi$ une forme bilinéaire, symétrique, non dégénérée sur $E$. On note $q$ la forme quadratique associée.

	\begin{definition}
		\begin{itemize}
			\item On appelle \textbf{isométries} de $E$ (relativement à $q$), les endomorphismes $u \in \mathrm{GL}(E)$ qui vérifient :
			\[ \forall x, y \in E, \, q(x,y) = q(u(x), u(y)) \]
			\item L'ensemble des isométries de $E$ forme un groupe, appelé \textbf{groupe orthogonal} de $E$, et noté $\mathcal{O}_q(E)$.
			\item Le sous-groupe des isométries de $E$ de déterminant $1$ est appelé \textbf{groupe spécial orthogonal} de $E$, et est noté $\mathrm{SO}_q(E)$.
		\end{itemize}

	\end{definition}

	\begin{definition}
		Soit $u \in \mathrm{SO}_q(E)$ tel que $u^2 = \operatorname{id}_E$.
		\begin{itemize}
			\item On dit que $u$ est une \textbf{réflexion} si $\dim(\ker(u+\operatorname{id}_E)) = 1$ (ie. $u$ est une symétrie par rapport à un hyperplan).
			\item On dit que $u$ est un \textbf{retournement} si $\dim(\ker(u+\operatorname{id}_E)) = 2$ (ie. $u$ est une symétrie par rapport à un plan).
		\end{itemize}
	\end{definition}

	On suppose désormais de plus que $\varphi$ est définie positive (ie. $\varphi$ est un produit scalaire).

	\begin{theorem}
		On suppose $n \geq 3$. Alors :
		\begin{enumerate}[label=(\roman*)]
			\item $\mathcal{O}_q(E)$ est engendré par les réflexions.
			\item $\mathrm{SO}_q(E)$ est engendré par les retournements.
		\end{enumerate}
	\end{theorem}

	\begin{application}
		On suppose $n \geq 3$. Alors :
		\begin{enumerate}[label=(\roman*)]
			\item $D(\mathcal{O}_q(E)) = \mathrm{SO}_q(E)$.
			\item $D(\mathrm{SO}_q(E)) = \mathrm{SO}_q(E)$.
		\end{enumerate}
	\end{application}
	%</content>
\end{document}
