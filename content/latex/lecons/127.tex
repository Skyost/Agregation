\documentclass[12pt, a4paper]{report}

% LuaLaTeX :

\RequirePackage{iftex}
\RequireLuaTeX

% Packages :

\usepackage[french]{babel}
%\usepackage[utf8]{inputenc}
%\usepackage[T1]{fontenc}
\usepackage[pdfencoding=auto, pdfauthor={Hugo Delaunay}, pdfsubject={Mathématiques}, pdfcreator={agreg.skyost.eu}]{hyperref}
\usepackage{amsmath}
\usepackage{amsthm}
%\usepackage{amssymb}
\usepackage{stmaryrd}
\usepackage{tikz}
\usepackage{tkz-euclide}
\usepackage{fourier-otf}
\usepackage{fontspec}
\usepackage{titlesec}
\usepackage{fancyhdr}
\usepackage{catchfilebetweentags}
\usepackage[french, capitalise, noabbrev]{cleveref}
\usepackage[fit, breakall]{truncate}
\usepackage[top=2.5cm, right=2cm, bottom=2.5cm, left=2cm]{geometry}
\usepackage{enumerate}
\usepackage{tocloft}
\usepackage{microtype}
%\usepackage{mdframed}
%\usepackage{thmtools}
\usepackage{xcolor}
\usepackage{tabularx}
\usepackage{aligned-overset}
\usepackage[subpreambles=true]{standalone}
\usepackage{environ}
\usepackage[normalem]{ulem}
\usepackage{marginnote}
\usepackage{etoolbox}
\usepackage{setspace}
\usepackage[bibstyle=reading, citestyle=draft]{biblatex}
\usepackage{xpatch}
\usepackage[many, breakable]{tcolorbox}
\usepackage[backgroundcolor=white, bordercolor=white, textsize=small]{todonotes}

% Bibliographie :

\newcommand{\overridebibliographypath}[1]{\providecommand{\bibliographypath}{#1}}
\overridebibliographypath{../bibliography.bib}
\addbibresource{\bibliographypath}
\defbibheading{bibliography}[\bibname]{%
	\newpage
	\section*{#1}%
}
\renewbibmacro*{entryhead:full}{\printfield{labeltitle}}
\DeclareFieldFormat{url}{\newline\footnotesize\url{#1}}
\AtEndDocument{\printbibliography}

% Police :

\setmathfont{Erewhon Math}

% Tikz :

\usetikzlibrary{calc}

% Longueurs :

\setlength{\parindent}{0pt}
\setlength{\headheight}{15pt}
\setlength{\fboxsep}{0pt}
\titlespacing*{\chapter}{0pt}{-20pt}{10pt}
\setlength{\marginparwidth}{1.5cm}
\setstretch{1.1}

% Métadonnées :

\author{agreg.skyost.eu}
\date{\today}

% Titres :

\setcounter{secnumdepth}{3}

\renewcommand{\thechapter}{\Roman{chapter}}
\renewcommand{\thesubsection}{\Roman{subsection}}
\renewcommand{\thesubsubsection}{\arabic{subsubsection}}
\renewcommand{\theparagraph}{\alph{paragraph}}

\titleformat{\chapter}{\huge\bfseries}{\thechapter}{20pt}{\huge\bfseries}
\titleformat*{\section}{\LARGE\bfseries}
\titleformat{\subsection}{\Large\bfseries}{\thesubsection \, - \,}{0pt}{\Large\bfseries}
\titleformat{\subsubsection}{\large\bfseries}{\thesubsubsection. \,}{0pt}{\large\bfseries}
\titleformat{\paragraph}{\bfseries}{\theparagraph. \,}{0pt}{\bfseries}

\setcounter{secnumdepth}{4}

% Table des matières :

\renewcommand{\cftsecleader}{\cftdotfill{\cftdotsep}}
\addtolength{\cftsecnumwidth}{10pt}

% Redéfinition des commandes :

\renewcommand*\thesection{\arabic{section}}
\renewcommand{\ker}{\mathrm{Ker}}

% Nouvelles commandes :

\newcommand{\website}{https://agreg.skyost.eu}

\newcommand{\tr}[1]{\mathstrut ^t #1}
\newcommand{\im}{\mathrm{Im}}
\newcommand{\rang}{\operatorname{rang}}
\newcommand{\trace}{\operatorname{trace}}
\newcommand{\id}{\operatorname{id}}
\newcommand{\stab}{\operatorname{Stab}}

\providecommand{\newpar}{\\[\medskipamount]}

\providecommand{\lesson}[3]{%
	\title{#3}%
	\hypersetup{pdftitle={#3}}%
	\setcounter{section}{\numexpr #2 - 1}%
	\section{#3}%
	\fancyhead[R]{\truncate{0.73\textwidth}{#2 : #3}}%
}

\providecommand{\development}[3]{%
	\title{#3}%
	\hypersetup{pdftitle={#3}}%
	\section*{#3}%
	\fancyhead[R]{\truncate{0.73\textwidth}{#3}}%
}

\providecommand{\summary}[1]{%
	\textit{#1}%
	\medskip%
}

\tikzset{notestyleraw/.append style={inner sep=0pt, rounded corners=0pt, align=center}}

%\newcommand{\booklink}[1]{\website/bibliographie\##1}
\newcommand{\citelink}[2]{\hyperlink{cite.\therefsection @#1}{#2}}
\newcommand{\previousreference}{}
\providecommand{\reference}[2][]{%
	\notblank{#1}{\renewcommand{\previousreference}{#1}}{}%
	\todo[noline]{%
		\protect\vspace{16pt}%
		\protect\par%
		\protect\notblank{#1}{\cite{[\previousreference]}\\}{}%
		\protect\citelink{\previousreference}{p. #2}%
	}%
}

\definecolor{devcolor}{HTML}{00695c}
\newcommand{\dev}[1]{%
	\reversemarginpar%
	\todo[noline]{
		\protect\vspace{16pt}%
		\protect\par%
		\bfseries\color{devcolor}\href{\website/developpements/#1}{DEV}
	}%
	\normalmarginpar%
}

% En-têtes :

\pagestyle{fancy}
\fancyhead[L]{\truncate{0.23\textwidth}{\thepage}}
\fancyfoot[C]{\scriptsize \href{\website}{\texttt{agreg.skyost.eu}}}

% Couleurs :

\definecolor{property}{HTML}{fffde7}
\definecolor{proposition}{HTML}{fff8e1}
\definecolor{lemma}{HTML}{fff3e0}
\definecolor{theorem}{HTML}{fce4f2}
\definecolor{corollary}{HTML}{ffebee}
\definecolor{definition}{HTML}{ede7f6}
\definecolor{notation}{HTML}{f3e5f5}
\definecolor{example}{HTML}{e0f7fa}
\definecolor{cexample}{HTML}{efebe9}
\definecolor{application}{HTML}{e0f2f1}
\definecolor{remark}{HTML}{e8f5e9}
\definecolor{proof}{HTML}{e1f5fe}

% Théorèmes :

\theoremstyle{definition}
\newtheorem{theorem}{Théorème}

\newtheorem{property}[theorem]{Propriété}
\newtheorem{proposition}[theorem]{Proposition}
\newtheorem{lemma}[theorem]{Lemme}
\newtheorem{corollary}[theorem]{Corollaire}

\newtheorem{definition}[theorem]{Définition}
\newtheorem{notation}[theorem]{Notation}

\newtheorem{example}[theorem]{Exemple}
\newtheorem{cexample}[theorem]{Contre-exemple}
\newtheorem{application}[theorem]{Application}

\theoremstyle{remark}
\newtheorem{remark}[theorem]{Remarque}

\counterwithin*{theorem}{section}

\newcommand{\applystyletotheorem}[1]{
	\tcolorboxenvironment{#1}{
		enhanced,
		breakable,
		colback=#1!98!white,
		boxrule=0pt,
		boxsep=0pt,
		left=8pt,
		right=8pt,
		top=8pt,
		bottom=8pt,
		sharp corners,
		after=\par,
	}
}

\applystyletotheorem{property}
\applystyletotheorem{proposition}
\applystyletotheorem{lemma}
\applystyletotheorem{theorem}
\applystyletotheorem{corollary}
\applystyletotheorem{definition}
\applystyletotheorem{notation}
\applystyletotheorem{example}
\applystyletotheorem{cexample}
\applystyletotheorem{application}
\applystyletotheorem{remark}
\applystyletotheorem{proof}

% Environnements :

\NewEnviron{whitetabularx}[1]{%
	\renewcommand{\arraystretch}{2.5}
	\colorbox{white}{%
		\begin{tabularx}{\textwidth}{#1}%
			\BODY%
		\end{tabularx}%
	}%
}

% Maths :

\DeclareFontEncoding{FMS}{}{}
\DeclareFontSubstitution{FMS}{futm}{m}{n}
\DeclareFontEncoding{FMX}{}{}
\DeclareFontSubstitution{FMX}{futm}{m}{n}
\DeclareSymbolFont{fouriersymbols}{FMS}{futm}{m}{n}
\DeclareSymbolFont{fourierlargesymbols}{FMX}{futm}{m}{n}
\DeclareMathDelimiter{\VERT}{\mathord}{fouriersymbols}{152}{fourierlargesymbols}{147}


% Bibliographie :

\addbibresource{\bibliographypath}%
\defbibheading{bibliography}[\bibname]{%
	\newpage
	\section*{#1}%
}
\renewbibmacro*{entryhead:full}{\printfield{labeltitle}}%
\DeclareFieldFormat{url}{\newline\footnotesize\url{#1}}%

\AtEndDocument{\printbibliography}

\begin{document}
	%<*content>
	\lesson{algebra}{127}{Exemples de nombres remarquables. Exemples d'anneaux de nombres remarquables. Applications.}

	\subsection{Nombres remarquables}

	\subsubsection{Deux exemples fondamentaux : \texorpdfstring{$e$}{e} et \texorpdfstring{$\pi$}{π}}

	\reference[QUE]{4}

	\begin{definition}
		On définit la fonction \textbf{exponentielle complexe} pour tout $z \in \mathbb{C}$ par
		\[ \sum_{n=0}^{+\infty} \frac{z^n}{n!} \]
		on note cette somme $e^z$ ou parfois $\exp(z)$.
	\end{definition}

	\begin{remark}
		Cette somme est bien définie pour tout $z \in \mathbb{C}$ d'après le critère de d'Alembert.
	\end{remark}

	\begin{proposition}
		\begin{enumerate}[label=(\roman*)]
			\item $\forall z, z' \in \mathbb{C}, \, e^{z+z'} = e^z e^{z'}$.
			\item $\exp$ est holomorphe sur $\mathbb{C}$, de dérivée elle-même.
			\item $\exp$ ne s'annule jamais.
		\end{enumerate}
	\end{proposition}

	\reference{383}

	\begin{definition}
		On définit $e$ le \textbf{nombre d'Euler} par $e = e^1 > 0$.
	\end{definition}

	\reference{7}

	\begin{proposition}
		La fonction $\varphi : t \mapsto e^{it}$ est un morphisme surjectif de $\mathbb{R}$ sur $\mathbb{U}$, le groupe des nombres complexes de module $1$.
	\end{proposition}

	\begin{proposition}
		En reprenant les notations précédentes, $\ker(\varphi)$ est un sous-groupe fermé de $\mathbb{R}$, de la forme $\ker(\varphi) = a\mathbb{Z}$. On note $a = 2\pi$.
	\end{proposition}

	\subsubsection{Nombres algébriques, transcendants}

	\reference[GOZ]{40}

	\begin{definition}
		Un nombre complexe (resp. réel) $z$ est dit \textbf{nombre algébrique complexe} (resp. \textbf{nombre algébrique réel}) s'il existe $P \in \mathbb{Z}[X] \setminus \{ 0 \}$ tel que $P(z) = 0$.
	\end{definition}

	\reference{40}

	\begin{example}
		$\pm \sqrt{2}$ et $\pm i$ sont des nombres algébriques.
	\end{example}

	\reference[QUE]{391}

	\begin{theorem}[Liouville]
		Soit $\alpha$ un nombre algébrique réel, racine d'un polynôme $P \in \mathbb{Z}[X]$ de degré supérieur ou égal à $2$. Alors,
		\[ \exists C_\alpha > 0 \text{ tel que } \forall \frac{p}{q} \in \mathbb{Q}, \, \left\Vert x - \frac{p}{q} \right\Vert \geq \frac{C}{q^d} \]
	\end{theorem}

	\begin{application}
		Le nombre de Liouville,
		\[ \sum_{n=1}^{+\infty} \frac{1}{10^{n!}} \]
		est transcendant.
	\end{application}

	\reference[GOZ]{41}

	\begin{theorem}[Hermite]
		$e$ est transcendant.
	\end{theorem}

	\begin{theorem}[Lindemann]
		$\pi$ est transcendant.
	\end{theorem}

	\reference[QUE]{385}

	\begin{application}
		Les nombres $\zeta(2k)$ sont transcendants pour $k \in \mathbb{N}^*$.
	\end{application}

	\subsection{Anneaux de nombres algébriques}

	\reference[GOZ]{40}

	\begin{notation}
		On note $\mathbb{A}$ l'ensemble des nombres algébriques complexes. $\mathbb{A} \, \cap \, \mathbb{R}$ est alors l'ensemble des nombres algébriques réels.
	\end{notation}

	\begin{theorem}
		\begin{enumerate}[label=(\roman*)]
			\item $\mathbb{A}$ est un sous-corps de $\mathbb{C}$ qui contient $\mathbb{Q}$.
			\item $\mathbb{A} \, \cap \, \mathbb{R}$ est un sous-corps de $\mathbb{R}$ qui contient $\mathbb{Q}$.
		\end{enumerate}
	\end{theorem}

	\reference{63}

	\begin{corollary}
		$\mathbb{A}$ est la clôture algébrique de $\mathbb{Q}$.
	\end{corollary}

	\begin{remark}
		Toute extension de $\mathbb{Q}$ de degré fini est alors un sous-corps de $\mathbb{A}$.
	\end{remark}

	\subsubsection{Corps de nombres quadratiques}

	\reference{33}

	\begin{proposition}
		Soit $d \in \mathbb{N}$ tel que $d \geq 2$. Les assertions suivantes sont équivalentes.
		\begin{enumerate}[label=(\roman*)]
			\item $\sqrt{d} \notin \mathbb{Q}$.
			\item $\sqrt{d} \notin \mathbb{N}$.
			\item Il existe $p$ premier tel que $v_p(d)$ (la valuation $p$-adique de $d$) est impair.
			\item $\mathbb{Q}[\sqrt{d}]$ est une extension de $\mathbb{Q}$ de degré $2$.
		\end{enumerate}
	\end{proposition}

	\begin{definition}
		On appelle \textbf{corps quadratique} toute extension de degré $2$ de $\mathbb{Q}$ dans $\mathbb{C}$.
	\end{definition}

	\begin{theorem}
		Soit $\mathbb{L}$ un corps quadratique. Alors, il existe un entier relatif $d \notin \{ 0, 1 \}$, sans facteur carré tel que
		\[ \mathbb{L} = \mathbb{Q}[\sqrt{d}] \]
		où $\sqrt{d}$ désigne un complexe dont le carré est égal à $d$.
	\end{theorem}

	\reference[ULM18]{67}

	\begin{definition}
		Soit $d$ un entier non nul qui n'est pas un carré dans $\mathbb{Z}$ et $z = x+y\sqrt{d} \in \mathbb{Q}[\sqrt{d}]$ avec $x, y \in \mathbb{Q}$. La \textbf{norme} de $z$ est
		\[ N(z) = x^2-y^2d \]
	\end{definition}

	\begin{proposition}
		Soit $d$ un entier non nul qui n'est pas un carré dans $\mathbb{Z}$. Pour $z = x+y\sqrt{d} \in \mathbb{Q}[\sqrt{d}]$ avec $x,y \in \mathbb{Q}$. Posons $\widetilde{z} = x-y\sqrt{d} \in \mathbb{Q}[\sqrt{d}]$.
		\begin{enumerate}[label=(\roman*)]
			\item L'application $z \mapsto \widetilde{z}$ est un automorphisme des anneaux $\mathbb{Q}[\sqrt{d}]$ et $\mathbb{Z}[\sqrt{d}]$. Pour tout $z \in \mathbb{Q}[\sqrt{d}]$, nous avons $\widetilde{\widetilde{z}} = z$ et $N(z) = z \widetilde{z}$. Si $z \in \mathbb{Z}[\sqrt{d}]$, alors $N(z) \in \mathbb{Z}$.
			\item $\mathbb{Q}[\sqrt{d}] = \mathbb{Q}[X]/(X^2 - d)$ est un corps.
			\item Dans $\mathbb{Q}[\sqrt{d}]$ et $\mathbb{Z}[\sqrt{d}]$, nous avons $N(z_1 z_2) = N(z_1) N(z_2)$ et $N(z) = 0 \iff z = 0$.
		\end{enumerate}
	\end{proposition}

	\begin{proposition}
		Soit $d$ un entier non nul qui n'est pas un carré dans $\mathbb{Z}$.
		\begin{enumerate}[label=(\roman*)]
			\item Les inversibles de $\mathbb{Z}[\sqrt{d}]$ avec $N(z) = \pm 1$.
			\item Tout élément non nul, non inversible possède une décomposition en irréductibles dans $\mathbb{Z}(\sqrt{d})$.
		\end{enumerate}
	\end{proposition}

	\subsubsection{Anneau des entiers de Gauss}

	\begin{definition}
		L'anneau $\mathbb{Z}[\sqrt{-1}] = \mathbb{Z}[i]$ est \textbf{l'anneau des entiers de Gauss}.
	\end{definition}

	\begin{example}
		Pour $z = x+iy \in \mathbb{Z}[i]$, nous avons $N(z) = x^2+y^2$ et donc les inversibles de $\mathbb{Z}[i]$ sont $\pm 1$ et $\pm i$.
	\end{example}

	\reference[I-P]{137}

	\begin{notation}
		On note $\Sigma$ l'ensemble des entiers qui sont somme de deux carrés.
	\end{notation}

	\begin{remark}
		$n \in \Sigma \iff \exists z \in \mathbb{Z}[i] \text{ tel que } N(z)=n$.
	\end{remark}

	\begin{lemma}
		$\mathbb{Z}[i]$ est euclidien de stathme $N$.
	\end{lemma}

	\begin{lemma}
		Soit $p$ un nombre premier. Si $p$ n'est pas irréductible dans $\mathbb{Z}[i]$, alors $p \in \Sigma$.
	\end{lemma}

	\dev{theoreme-des-deux-carres-fermat}

	\begin{theorem}[Deux carrés de Fermat]
		Soit $n \in \mathbb{N}^*$. Alors $n \in \Sigma$ si et seulement si $v_p(n)$ est pair pour tout $p$ premier tel que $p \equiv 3 \mod 4$ (où $v_p(n)$ désigne la valuation $p$-adique de $n$).
	\end{theorem}

	\subsubsection{Corps cyclotomiques}

	Soit $m$ un entier supérieur ou égal à $1$.

	\reference{67}

	\begin{definition}
		On définit
		\[ \mu_m = \{ z \in \mathbb{C}^* \mid z^m = 1 \} \]
		l'ensemble des \textbf{racines $m$-ièmes de l'unité}. C'est un groupe (cyclique) pour la multiplication dont l'ensemble des générateurs, noté $\mu_m^*$, est formé des \textbf{racines primitives $m$-ièmes de l'unité}.
	\end{definition}

	\begin{proposition}
		\begin{enumerate}[label=(\roman*)]
			\item $\mu_m^* = \{ e^{\frac{2ik\pi}{m}} \mid k \in \llbracket 0, m-1 \rrbracket, \, \operatorname{pgcd}(k, m) = 1 \}$.
			\item $\vert \mu_m^* \vert = \varphi(m)$, où $\varphi$ désigne l'indicatrice d'Euler.
		\end{enumerate}
	\end{proposition}

	\begin{proposition}
		\label{127-1}
		Le sous-corps $\mathbb{Q}(\xi)$ de $\mathbb{C}$ ne dépend pas de la racine $m$-ième primitive $\xi$ de l'unité considérée.
	\end{proposition}

	\begin{definition}
		On appelle \textbf{corps cyclotomique}, un corps de la forme de la \cref{127-1} (ie. engendré par une racine primitive de l'unité).
	\end{definition}

	\begin{definition}
		On appelle \textbf{$m$-ième polynôme cyclotomique} le polynôme
		\[ \Phi_m = \prod_{\xi \in \mu_m^*} (X - \xi) \]
	\end{definition}

	\begin{theorem}
		\begin{enumerate}[label=(\roman*)]
			\item $X^m - 1 = \prod_{d \mid m} \Phi_d$.
			\item $\Phi_m \in \mathbb{Z}[X]$.
			\item $\Phi_m$ est irréductible sur $\mathbb{Q}$.
		\end{enumerate}
	\end{theorem}

	\begin{corollary}
		Le polynôme minimal sur $\mathbb{Q}$ de tout élément $\xi$ de $\mu_m^*$ est $\Phi_m$. En particulier, le degré de $\mathbb{Q}(\xi)$ sur $\mathbb{Q}$ est $\varphi(m)$.
	\end{corollary}

	\begin{application}[Théorème de Wedderburn]
		Tout corps fini est commutatif.
	\end{application}

	\reference[GOU21]{99}

	\begin{application}[Dirichlet faible]
		Pour tout entier $n$, il existe une infinité de nombres premiers congrus à $1$ modulo $n$.
	\end{application}

	\subsection{Application à la constructibilité à la règle et au compas}

	\reference[GOZ]{47}

	On note $\mathcal{P}$ un plan affine euclidien muni d'un repère orthonormé direct $\mathcal{R} = (O, \overrightarrow{i}, \overrightarrow{j})$. On s'autorise à identifier chaque point $M \in \mathcal{P}$ avec ses coordonnées $(x,y) \in \mathbb{R}^2$ dans $\mathcal{R}$.

	\begin{definition}
		On dit qu'un point $M \in \mathcal{P}$ est \textbf{constructible} (sous-entendu \textit{à la règle et au compas}) si on peut le construire en utilisant uniquement la règle et le compas, en supposant $O$ et $I=(1,0)$ déjà construits.
	\end{definition}

	\begin{proposition}
		Soient $A$, $B$ deux points constructibles distincts.
		\begin{enumerate}[label=(\roman*)]
			\item Si $A$ est constructible, son symétrique par rapport à $O$ l'est aussi.
			\item $J = (0,1)$ est constructible.
			\item Si $C$ est un point constructible, on peut construire à la règle et au compas la perpendiculaire à $(AB)$ passant par $C$.
			\item Si $C$ est un point constructible, on peut construire à la règle et au compas la parallèle à $(AB)$ passant par $C$.
		\end{enumerate}
	\end{proposition}

	\begin{proposition}
		Soit $x \in \mathbb{R}$.
		\[ (x,0) \text{ est constructible} \iff (0,x) \text{ est constructible} \]
	\end{proposition}

	\begin{definition}
		Un nombre vérifiant la proposition précédente est dit \textbf{nombre constructible}.
	\end{definition}

	\begin{proposition}
		\begin{enumerate}[label=(\roman*)]
			\item Tout élément de $\mathbb{Q}$ est constructible.
			\item $(x,y)$ est constructible si et seulement si $x$ et $y$ le sont.
		\end{enumerate}
	\end{proposition}

	\begin{theorem}
		L'ensemble $\mathbb{E}$ des nombres constructibles est un sous-corps de $\mathbb{R}$ stable par racine carrée.
	\end{theorem}

	\begin{theorem}[Wantzel]
		Soit $t \in \mathbb{R}$. $t$ est constructible si et seulement s'il existe une suite fini $(L_0, \dots, L_p)$ de sous-corps de $\mathbb{R}$ vérifiant :
		\begin{enumerate}[label=(\roman*)]
			\item $L_0 = \mathbb{Q}$.
			\item $\forall i \in \llbracket 1, p-1 \rrbracket$, $L_i$ est une extension quadratique de $L_{i-1}$.
			\item $t \in L_p$.
		\end{enumerate}
	\end{theorem}

	\begin{corollary}
		\begin{enumerate}[label=(\roman*)]
			\item Si $x$ est constructible, le degré de l'extension $\mathbb{Q}[x]$ sur $\mathbb{Q}$ est de la forme $2^s$ pour $s \in \mathbb{N}$.
			\item Tout nombre constructible est algébrique.
		\end{enumerate}
	\end{corollary}

	\begin{cexample}
		\begin{itemize}
			\item $\sqrt[3]{2}$ est algébrique, non constructible.
			\item $\sqrt{\pi}$ est transcendant et n'est donc pas constructible.
		\end{itemize}
	\end{cexample}

	\begin{application}[Quadrature du cercle]
		Il est impossible de construire, à la règle et au compas, un carré ayant même aire qu'un disque donné.
	\end{application}

	\begin{application}[Duplication du cube]
		Il est impossible de construire, à la règle et au compas, l'arête d'un cube ayant un volume double de celui d'un cube donné.
	\end{application}
	%</content>
\end{document}
