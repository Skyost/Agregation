\documentclass[12pt, a4paper]{report}

% LuaLaTeX :

\RequirePackage{iftex}
\RequireLuaTeX

% Packages :

\usepackage[french]{babel}
%\usepackage[utf8]{inputenc}
%\usepackage[T1]{fontenc}
\usepackage[pdfencoding=auto, pdfauthor={Hugo Delaunay}, pdfsubject={Mathématiques}, pdfcreator={agreg.skyost.eu}]{hyperref}
\usepackage{amsmath}
\usepackage{amsthm}
%\usepackage{amssymb}
\usepackage{stmaryrd}
\usepackage{tikz}
\usepackage{tkz-euclide}
\usepackage{fourier-otf}
\usepackage{fontspec}
\usepackage{titlesec}
\usepackage{fancyhdr}
\usepackage{catchfilebetweentags}
\usepackage[french, capitalise, noabbrev]{cleveref}
\usepackage[fit, breakall]{truncate}
\usepackage[top=2.5cm, right=2cm, bottom=2.5cm, left=2cm]{geometry}
\usepackage{enumerate}
\usepackage{tocloft}
\usepackage{microtype}
%\usepackage{mdframed}
%\usepackage{thmtools}
\usepackage{xcolor}
\usepackage{tabularx}
\usepackage{aligned-overset}
\usepackage[subpreambles=true]{standalone}
\usepackage{environ}
\usepackage[normalem]{ulem}
\usepackage{marginnote}
\usepackage{etoolbox}
\usepackage{setspace}
\usepackage[bibstyle=reading, citestyle=draft]{biblatex}
\usepackage{xpatch}
\usepackage[many, breakable]{tcolorbox}
\usepackage[backgroundcolor=white, bordercolor=white, textsize=small]{todonotes}

% Bibliographie :

\newcommand{\overridebibliographypath}[1]{\providecommand{\bibliographypath}{#1}}
\overridebibliographypath{../bibliography.bib}
\addbibresource{\bibliographypath}
\defbibheading{bibliography}[\bibname]{%
	\newpage
	\section*{#1}%
}
\renewbibmacro*{entryhead:full}{\printfield{labeltitle}}
\DeclareFieldFormat{url}{\newline\footnotesize\url{#1}}
\AtEndDocument{\printbibliography}

% Police :

\setmathfont{Erewhon Math}

% Tikz :

\usetikzlibrary{calc}

% Longueurs :

\setlength{\parindent}{0pt}
\setlength{\headheight}{15pt}
\setlength{\fboxsep}{0pt}
\titlespacing*{\chapter}{0pt}{-20pt}{10pt}
\setlength{\marginparwidth}{1.5cm}
\setstretch{1.1}

% Métadonnées :

\author{agreg.skyost.eu}
\date{\today}

% Titres :

\setcounter{secnumdepth}{3}

\renewcommand{\thechapter}{\Roman{chapter}}
\renewcommand{\thesubsection}{\Roman{subsection}}
\renewcommand{\thesubsubsection}{\arabic{subsubsection}}
\renewcommand{\theparagraph}{\alph{paragraph}}

\titleformat{\chapter}{\huge\bfseries}{\thechapter}{20pt}{\huge\bfseries}
\titleformat*{\section}{\LARGE\bfseries}
\titleformat{\subsection}{\Large\bfseries}{\thesubsection \, - \,}{0pt}{\Large\bfseries}
\titleformat{\subsubsection}{\large\bfseries}{\thesubsubsection. \,}{0pt}{\large\bfseries}
\titleformat{\paragraph}{\bfseries}{\theparagraph. \,}{0pt}{\bfseries}

\setcounter{secnumdepth}{4}

% Table des matières :

\renewcommand{\cftsecleader}{\cftdotfill{\cftdotsep}}
\addtolength{\cftsecnumwidth}{10pt}

% Redéfinition des commandes :

\renewcommand*\thesection{\arabic{section}}
\renewcommand{\ker}{\mathrm{Ker}}

% Nouvelles commandes :

\newcommand{\website}{https://agreg.skyost.eu}

\newcommand{\tr}[1]{\mathstrut ^t #1}
\newcommand{\im}{\mathrm{Im}}
\newcommand{\rang}{\operatorname{rang}}
\newcommand{\trace}{\operatorname{trace}}
\newcommand{\id}{\operatorname{id}}
\newcommand{\stab}{\operatorname{Stab}}

\providecommand{\newpar}{\\[\medskipamount]}

\providecommand{\lesson}[3]{%
	\title{#3}%
	\hypersetup{pdftitle={#3}}%
	\setcounter{section}{\numexpr #2 - 1}%
	\section{#3}%
	\fancyhead[R]{\truncate{0.73\textwidth}{#2 : #3}}%
}

\providecommand{\development}[3]{%
	\title{#3}%
	\hypersetup{pdftitle={#3}}%
	\section*{#3}%
	\fancyhead[R]{\truncate{0.73\textwidth}{#3}}%
}

\providecommand{\summary}[1]{%
	\textit{#1}%
	\medskip%
}

\tikzset{notestyleraw/.append style={inner sep=0pt, rounded corners=0pt, align=center}}

%\newcommand{\booklink}[1]{\website/bibliographie\##1}
\newcommand{\citelink}[2]{\hyperlink{cite.\therefsection @#1}{#2}}
\newcommand{\previousreference}{}
\providecommand{\reference}[2][]{%
	\notblank{#1}{\renewcommand{\previousreference}{#1}}{}%
	\todo[noline]{%
		\protect\vspace{16pt}%
		\protect\par%
		\protect\notblank{#1}{\cite{[\previousreference]}\\}{}%
		\protect\citelink{\previousreference}{p. #2}%
	}%
}

\definecolor{devcolor}{HTML}{00695c}
\newcommand{\dev}[1]{%
	\reversemarginpar%
	\todo[noline]{
		\protect\vspace{16pt}%
		\protect\par%
		\bfseries\color{devcolor}\href{\website/developpements/#1}{DEV}
	}%
	\normalmarginpar%
}

% En-têtes :

\pagestyle{fancy}
\fancyhead[L]{\truncate{0.23\textwidth}{\thepage}}
\fancyfoot[C]{\scriptsize \href{\website}{\texttt{agreg.skyost.eu}}}

% Couleurs :

\definecolor{property}{HTML}{fffde7}
\definecolor{proposition}{HTML}{fff8e1}
\definecolor{lemma}{HTML}{fff3e0}
\definecolor{theorem}{HTML}{fce4f2}
\definecolor{corollary}{HTML}{ffebee}
\definecolor{definition}{HTML}{ede7f6}
\definecolor{notation}{HTML}{f3e5f5}
\definecolor{example}{HTML}{e0f7fa}
\definecolor{cexample}{HTML}{efebe9}
\definecolor{application}{HTML}{e0f2f1}
\definecolor{remark}{HTML}{e8f5e9}
\definecolor{proof}{HTML}{e1f5fe}

% Théorèmes :

\theoremstyle{definition}
\newtheorem{theorem}{Théorème}

\newtheorem{property}[theorem]{Propriété}
\newtheorem{proposition}[theorem]{Proposition}
\newtheorem{lemma}[theorem]{Lemme}
\newtheorem{corollary}[theorem]{Corollaire}

\newtheorem{definition}[theorem]{Définition}
\newtheorem{notation}[theorem]{Notation}

\newtheorem{example}[theorem]{Exemple}
\newtheorem{cexample}[theorem]{Contre-exemple}
\newtheorem{application}[theorem]{Application}

\theoremstyle{remark}
\newtheorem{remark}[theorem]{Remarque}

\counterwithin*{theorem}{section}

\newcommand{\applystyletotheorem}[1]{
	\tcolorboxenvironment{#1}{
		enhanced,
		breakable,
		colback=#1!98!white,
		boxrule=0pt,
		boxsep=0pt,
		left=8pt,
		right=8pt,
		top=8pt,
		bottom=8pt,
		sharp corners,
		after=\par,
	}
}

\applystyletotheorem{property}
\applystyletotheorem{proposition}
\applystyletotheorem{lemma}
\applystyletotheorem{theorem}
\applystyletotheorem{corollary}
\applystyletotheorem{definition}
\applystyletotheorem{notation}
\applystyletotheorem{example}
\applystyletotheorem{cexample}
\applystyletotheorem{application}
\applystyletotheorem{remark}
\applystyletotheorem{proof}

% Environnements :

\NewEnviron{whitetabularx}[1]{%
	\renewcommand{\arraystretch}{2.5}
	\colorbox{white}{%
		\begin{tabularx}{\textwidth}{#1}%
			\BODY%
		\end{tabularx}%
	}%
}

% Maths :

\DeclareFontEncoding{FMS}{}{}
\DeclareFontSubstitution{FMS}{futm}{m}{n}
\DeclareFontEncoding{FMX}{}{}
\DeclareFontSubstitution{FMX}{futm}{m}{n}
\DeclareSymbolFont{fouriersymbols}{FMS}{futm}{m}{n}
\DeclareSymbolFont{fourierlargesymbols}{FMX}{futm}{m}{n}
\DeclareMathDelimiter{\VERT}{\mathord}{fouriersymbols}{152}{fourierlargesymbols}{147}


% Bibliographie :

\addbibresource{\bibliographypath}%
\defbibheading{bibliography}[\bibname]{%
	\newpage
	\section*{#1}%
}
\renewbibmacro*{entryhead:full}{\printfield{labeltitle}}%
\DeclareFieldFormat{url}{\newline\footnotesize\url{#1}}%

\AtEndDocument{\printbibliography}

\begin{document}
	%<*content>
	\lesson{analysis}{246}{Séries de Fourier. Exemples et applications.}

	\subsection{Coefficients de Fourier}

	\subsubsection{Définitions}

	\reference[Z-Q]{73}

	\begin{notation}
		\begin{itemize}
			\item Pour tout $p \in [1, +\infty]$, on note $L_p^{2\pi}$ l'espace des fonctions $f : \mathbb{R} \rightarrow \mathbb{C}$, $2\pi$-périodiques et mesurables, telles que $\Vert f \Vert_p < +\infty$.
			\item Pour tout $n \in \mathbb{Z}$, on note $e_n$ la fonction $2\pi$-périodique définie pour tout $t \in \mathbb{R}$ par $e_n(t) = e^{int}$.
		\end{itemize}
	\end{notation}

	\begin{remark}
		\[ 1 \leq p < q \leq +\infty \implies L_q \subseteq L_p \text{ et } \Vert . \Vert_p \leq \Vert . \Vert_q \]
	\end{remark}

	\begin{proposition}
		$L_2^{2\pi}$ est un espace de Hilbert pour le produit scalaire
		\[ \langle ., . \rangle : (f, g) \mapsto \frac{1}{2 \pi} \int_0^{2\pi} f(t) \overline{g(t)} \, \mathrm{d}t \]
	\end{proposition}

	\reference[GOU20]{268}

	\begin{definition}
		Soit $f \in L_1^{2\pi}$. On appelle :
		\begin{itemize}
			\item \textbf{Coefficients de Fourier complexes}, les complexes définis par
			\[ \forall n \in \mathbb{Z}, \, c_n(f) = \frac{1}{2 \pi} \int_0^{2\pi} f(t) e^{-int} \, \mathrm{d}t = \langle f, e_n \rangle \]
			\item \textbf{Coefficients de Fourier réels}, les complexes définis par
			\[ \forall n \in \mathbb{N}, \, a_n(f) = \frac{1}{\pi} \int_0^{2\pi} f(t) \cos(nt) \, \mathrm{d}t \text{ et } \forall n \in \mathbb{N}^*, \, b_n(f) = \frac{1}{\pi} \int_0^{2\pi} f(t) \sin(nt) \, \mathrm{d}t \]
		\end{itemize}
	\end{definition}

	\begin{remark}
		Soit $f \in L_1^{2\pi}$.
		\begin{itemize}
			\item On utilise en général les coefficients réels lorsque $f$ est à valeurs réelles.
			\item Si $f$ est paire (resp. impaire), les coefficients $b_n(f)$ (resp. $a_n(f)$) sont nuls.
			\item $\forall n \in \mathbb{N}, a_n(f) = c_n(f) + c_{-n}(f)$ et $\forall n \in \mathbb{N}^*, b_n(f) = i(c_n(f) + c_{-n}(f))$.
			\item On pourrait plus généralement définir les coefficients de Fourier d'une fonction $T$-périodique pour toute période $T > 0$.
		\end{itemize}
	\end{remark}

	\reference{273}

	\begin{example}
		On définit $\forall \alpha \in \mathbb{R} \setminus \mathbb{Z}, \, f_\alpha : t \mapsto \cos(\alpha t)$. Alors,
		\[ \forall n \in \mathbb{N}, \, a_n(f_\alpha) = (-1)^n \frac{2 \alpha \sin(\alpha \pi)}{\pi (\alpha^2 - n^2)} \text{ et } \forall n \in \mathbb{N}^*, \, b_n(f_\alpha) = 0 \]
	\end{example}

	\subsubsection{Propriétés}

	\paragraph{L'algèbre \texorpdfstring{$L_1^{2 \pi}$}{L₁²ᵖⁱ}}

	\reference[BMP]{125}

	\begin{proposition}
		Tout comme sur $L_1(\mathbb{R})$, on a un opérateur de convolution sur $L_1^{2 \pi}$ :
		\[ \forall f, g \in L_1^{2 \pi}, \, \forall x \in \mathbb{R}, \, f*g(x) = \frac{1}{2 \pi} \int_0^{2\pi} f(y) g(x - y) \, \mathrm{d}y \]
		qui munit $L_1^{2 \pi}$ d'une structure d'algèbre normée.
	\end{proposition}

	\reference[AMR08]{174}

	\begin{proposition}
		Soient $f \in L_1^{2 \pi}$, $a \in \mathbb{R}$ et $k, n \in \mathbb{Z}$.
		\begin{enumerate}[label=(\roman*)]
			\item $f * e_n = c_n(f) e_n$.
			\item $\vert c_n(f) \vert \leq \Vert f \Vert_1$.
			\item $c_{-n}(f) = c_n(x \mapsto f(-x))$.
			\item $c_n(\overline{f}) = \overline{c_{-n}(f)}$.
			\item $c_n(x \mapsto f(x-a)) = e_n(a) c_n(f)$.
			\item $c_n(e_k f) = c_{n-k}(f) e_n$.
			\item $c_n(f') = in c_n(f)$ si $f$ est continue et $\mathcal{C}^1$ par morceaux.
		\end{enumerate}
	\end{proposition}

	\reference[BMP]{126}

	\begin{lemma}[Riemann-Lebesgue]
		Soit $f \in L_1^{2 \pi}$. Alors $(c_n(f))$ tend vers $0$ lorsque $n$ tend vers $\pm \infty$.
	\end{lemma}

	\begin{theorem}
		\label{246-1}
		Soit $c_0$ l'espace des suites de complexes qui convergent vers $0$ en $-\infty$ et $+\infty$. L'application
		\[
		\mathcal{F} :
		\begin{array}{ccc}
			L_1^{2\pi} &\rightarrow& c_0 \\
			f &\mapsto& (c_n(f))_{n \in \mathbb{Z}}
		\end{array}
		\]
		est un morphisme d'algèbres de $(L_1^{2\pi}, +, *, \Vert . \Vert_1)$ dans $(c_0, +, \cdot, \Vert . \Vert_\infty)$ continu, de norme $1$.
	\end{theorem}

	\paragraph{Propriétés hilbertiennes de \texorpdfstring{$L_2^{2 \pi}$}{L₂²ᵖⁱ}}

	\reference{109}

	\begin{theorem}
		\label{246-2}
		Soit $H$ un espace de Hilbert et $(\epsilon_n)_{n \in I}$ une famille orthonormée dénombrable de $H$. Les propriétés suivantes sont équivalentes :
		\begin{enumerate}[label=(\roman*)]
			\item La famille orthonormée $(\epsilon_n)_{n \in I}$ est une base hilbertienne de $H$.
			\item $\forall x \in H, \, x = \sum_{n=0}^{+\infty} \langle x, \epsilon_n \rangle \epsilon_n$.
			\item \label{246-3} $\forall x \in H, \, \Vert x \Vert_2 = \sum_{n=0}^{+\infty} \vert \langle x, \epsilon_n \rangle \vert^2$.
		\end{enumerate}
	\end{theorem}

	\begin{remark}
		L'égalité du \cref{246-2} \cref{246-3} est appelée \textbf{égalité de Parseval}.
	\end{remark}

	\reference{123}

	\begin{theorem}
		La famille $(e_n)_{n \in \mathbb{Z}}$ est une base hilbertienne de $L_2^{2 \pi}$.
	\end{theorem}

	\begin{corollary}
		\label{246-4}
		\[ \forall f \in L_2^{2 \pi}, \, f = \sum_{n = -\infty}^{+\infty} c_n(f) e_n \]
	\end{corollary}

	\reference[GOU20]{272}

	\begin{example}
		\label{246-5}
		On considère $f : x \mapsto 1 - \frac{x^2}{\pi^2}$ sur $[-\pi, \pi]$. Alors,
		\[ \frac{\pi^4}{90} = \Vert f \Vert_2 = \sum_{n=0}^{+\infty} \frac{1}{n^4} \]
	\end{example}

	\reference[BMP]{124}

	\begin{remark}
		L'égalité du \cref{246-4} est valable dans $L_2^{2\pi}$, elle signifie donc que
		\[ \left\Vert \sum_{n = -N}^{N} c_n(f) e_n - f \right\Vert_2 \longrightarrow_{N \rightarrow +\infty} 0 \]
	\end{remark}

	\subsubsection{Séries de Fourier}

	\reference[GOU20]{269}

	\begin{definition}
		Soit $f \in L_1^{2\pi}$. On appelle \textbf{série de Fourier} associée à $f$ la série $(S_N(f))$ définie par
		\[ \forall N \in \mathbb{N}, \, S_N(f) = \sum_{n=-N}^{N} c_n(f) e_n \overset{(*)}{=} \frac{a_0(f)}{2} + \sum_{n = 1}^N (a_n(f) \cos(nx) + b_n(f) \sin(nx)) \]
	\end{definition}

	\begin{remark}
		L'égalité $(*)$ de la définition précédente est justifiée car,
		\[ \forall n \in \mathbb{N}^*, \, \forall x \in \mathbb{R}, \, c_n(f) e^{inx} + c_{-n}(f) e^{-inx} = a_n(f) \cos(nx) + b_n(f) \sin(nx) \]
	\end{remark}

	\newpage
	\subsection{Divers modes de convergence}

	\reference[AMR08]{178}

	Nous avons vu que pour $f \in L_2^{2\pi}$, il y a convergence dans $L_2^{2\pi}$ de $(S_N(f))$ vers $f$. Cette section est dédiée à l'étude d'autres modes de convergence. En particulier, nous allons nous poser plusieurs questions :

	\begin{itemize}
		\item Pour quelles fonctions $f$ y a-t-il convergence de $(S_N(f))$ ?
		\item Y a-t-il convergence vers $f$ ?
		\item De quel type de convergence s'agit-il ?
	\end{itemize}

	\subsubsection{Convergence au sens de Cesàro}

	\reference{184}

	\begin{definition}
		Pour tout $N \in \mathbb{N}$, la fonction $D_N = \sum_{n=-N}^{N} e_N$ est appelé \textbf{noyau de Dirichlet} d'ordre $N$.
	\end{definition}

	\begin{proposition}
		Soit $N \in \mathbb{N}$.
		\begin{enumerate}[label=(\roman*)]
			\item $D_N$ est une fonction paire, $2\pi$-périodique, et de norme $1$.
			\item \[ \forall x \in \mathbb{R} \setminus 2 \pi \mathbb{Z}, \, D_N(x) = \frac{\sin \left(\left( N + \frac{1}{2} \right) x \right)}{\sin \left( \frac{x}{2} \right)} \]
			\item Pour tout $f \in L_1^{2 \pi}, \, S_N(f) = f * D_N$.
		\end{enumerate}
	\end{proposition}

	\begin{definition}
		Pour tout $N \in \mathbb{N}$, la fonction $K_N = \frac{1}{N} \sum_{j=0}^{N-1} D_j$ est appelé \textbf{noyau de Fejér} d'ordre $N$.
	\end{definition}

	\begin{notation}
		Pour tout $N \in \mathbb{N}^*$, on note $\sigma_N = \frac{1}{N} \sum_{k=0}^{N-1} S_n(f)$ la somme de Cesàro d'ordre $N$ de la série de Fourier d'une fonction $f \in L_1^{2 \pi}$.
	\end{notation}

	\begin{proposition}
		Soient $N \in \mathbb{N}^*$ et $f \in L_1^{2 \pi}$.
		\begin{enumerate}[label=(\roman*)]
			\item $K_N$ est une fonction positive et de norme $1$.
			\item \[ \forall x \in \mathbb{R} \setminus 2 \pi \mathbb{Z}, \, K_N(x) = \frac{1}{N} \left(\frac{\sin \left( \frac{Nx}{2} \right)}{\sin \left( \frac{x}{2} \right)}\right)^2 \]
			\item $K_N = \sum_{n=-N}^{N} \left(1 - \frac{\vert n \vert}{N}\right) e_n$.
			\item $\sigma_N(f) = f * K_N$.
		\end{enumerate}
	\end{proposition}

	\reference{190}
	\dev{theoreme-de-fejer}

	\begin{theorem}[Fejér]
		Soit $f : \mathbb{R} \rightarrow \mathbb{C}$ une fonction $2\pi$-périodique.
		\begin{enumerate}[label=(\roman*)]
			\item Si $f$ est continue, alors $\Vert \sigma_N(f) \Vert_\infty \leq \Vert f \Vert_\infty$ et $(\sigma_N(f))$ converge uniformément vers $f$.
			\item Si $f \in L_p^{2\pi}$ pour $p \in [1,+\infty[$, alors $\Vert \sigma_N(f) \Vert_p \leq \Vert f \Vert_p$ et $(\sigma_N(f))$ converge vers $f$ pour $\Vert . \Vert_p$.
		\end{enumerate}
	\end{theorem}

	\begin{corollary}
		L'espace des polynômes trigonométriques $\{ \sum_{n=-N}^N c_n e_n \mid (c_n) \in \mathbb{C}^{\mathbb{N}}, \, N \in \mathbb{N} \}$ est dense dans l'espace des fonction continues $2\pi$-périodiques pour $\Vert . \Vert_\infty$ et est dense dans $L_p^{2\pi}$ pour $\Vert . \Vert_p$ avec $p \in [1,+\infty[$.
	\end{corollary}

	\reference[BMP]{128}

	\begin{application}
		L'application $\mathcal{F}$ du \cref{246-1} est injective.
	\end{application}

	\reference[AMR08]{192}

	\begin{application}[Théorème de Weierstrass]
		Toute fonction continue sur un intervalle compact $[a,b]$ est limite uniforme sur $[a,b]$ d'une suite de polynômes.
	\end{application}

	\subsubsection{Convergence ponctuelle}

	\reference[GOU20]{271}

	\begin{theorem}[Dirichlet]
		Soient $f : \mathbb{R} \rightarrow \mathbb{C}$ $2\pi$-périodique, continue par morceaux sur $\mathbb{R}$ et $t_0 \in \mathbb{R}$ tels que la fonction
		\[ h \mapsto \frac{f(t_0 + h) + f(t_0 - h) - f(t_0^+) - f(t_0^-)}{h} \]
		est bornée au voisinage de $0$. Alors,
		\[ S_N(f)(t_0) \longrightarrow_{N \rightarrow +\infty} \frac{f(t_0^+) + f(t_0^-)}{2} \]
	\end{theorem}

	\begin{cexample}
		Soit $f : \mathbb{R} \rightarrow \mathbb{R}$ paire, $2\pi$-périodique telle que :
		\[ \forall x \in [0, \pi], f(x) = \sum_{p=1}^{+\infty} \frac{1}{p^2} \sin \left( (2^{p^3} + 1) \frac{x}{2} \right)
		\]
		Alors $f$ est bien définie et continue sur $\mathbb{R}$. Cependant, sa série de Fourier diverge en $0$.
	\end{cexample}

	\begin{corollary}
		Soient $f : \mathbb{R} \rightarrow \mathbb{C}$ $2\pi$-périodique, $\mathcal{C}^1$ par morceaux sur $\mathbb{R}$. Alors,
		\[ \forall x \in \mathbb{R}, \, S_N(f)(x) \longrightarrow_{N \rightarrow +\infty} \frac{f(x^+) + f(x^-)}{2} \]
		En particulier, si $f$ est continue en $x$, la série de Fourier de $f$ converge vers $f(x)$.
	\end{corollary}

	\begin{example}
		\label{246-6}
		En reprenant la fonction de l'\cref{246-5},
		\[ \forall x \in [-\pi, \pi], \, f(x) = \frac{2}{3} - \frac{4}{\pi^2} \sum_{n=1}^{+\infty} (-1)^n \frac{\cos(nx)}{n^2} \]
	\end{example}

	\subsubsection{Convergence normale}

	\reference[BMP]{128}

	\begin{proposition}
		Soit $f \in L_1^{2\pi}$ et telle que sa série de Fourier converge normalement. Alors, la somme $g : x \mapsto \sum_{n=-\infty}^{+\infty} c_n(f) e_n(x)$ est une fonction continue $2\pi$-périodique presque partout égale à $f$. De plus, si $f$ est continue, l'égalité $f(x) = g(x)$ est vraie pour tout $x$.
	\end{proposition}

	\begin{proposition}
		Soit $f : \mathbb{R} \rightarrow \mathbb{C}$ $2\pi$-périodique continue et $\mathcal{C}^1$ par morceaux sur $\mathbb{R}$. Alors $(S_N(f))$ converge normalement vers $f$.
	\end{proposition}

	\reference[AMR08]{211}

	\begin{application}[Développement eulérien de la cotangente]
		\[ \forall u \in \mathbb{R} \setminus \pi \mathbb{Z}, \, \operatorname{cotan}(u) = \frac{1}{u} + \sum_{n=1}^{+\infty} \frac{2u}{u^2 - n^2 \pi^2} \]
	\end{application}

	\subsection{Applications}

	\subsubsection{Calcul de sommes, de produits et d'intégrales}

	\reference[GOU20]{272}

	\begin{application}
		En utilisant l'\cref{246-6}, avec $x = \pi$, on retrouve
		\[ \sum_{n = 1}^{+\infty} \frac{1}{n^2} = \frac{\pi^2}{6} \]
	\end{application}

	\begin{application}
		\[ \forall t \in ]-\pi, \pi[, \, \sin(t) = t \prod_{n=1}^{+\infty} \left( 1 - \frac{t^2}{n^2\pi^2} \right) \]
	\end{application}

	\reference[AMR08]{221}

	\begin{application}[Sommes de Gauss]
		\[ \forall m \in \mathbb{N}^*, \, \sum_{n=0}^{m-1} e^{\frac{2i\pi n^2}{m}} = \frac{1+i^{-m}}{1+i^{-1}} \]
	\end{application}

	\begin{application}[Intégrales de Fresnel]
		\[ \int_{-\infty}^{+\infty} \cos(2 \pi u^2) \, \mathrm{d}u = \int_{-\infty}^{+\infty} \sin(2 \pi u^2) \, \mathrm{d}u = \frac{1}{2} \]
	\end{application}

	\reference[AMR11]{325}

	\begin{application}
		Soit $a > 0$. En considérant la fonction $t \mapsto \frac{1}{\cosh(a) + \cos(t)}$, on en déduit que
		\[ \forall n \in \mathbb{N}, \, \int_{0}^{\pi} \frac{\cos(nt)}{\cosh(a) + \cos(t)} \, \mathrm{d}t = (-1)^n \frac{\pi e^{-na}}{\sinh(a)} \]
	\end{application}

	\subsubsection{Équations fonctionnelles}

	\reference[GOU20]{284}
	\dev{formule-sommatoire-de-poisson}

	\begin{theorem}[Formule sommatoire de Poisson]
		\label{formule-sommatoire-de-poisson-1}
		Soit $f : \mathbb{R} \rightarrow \mathbb{C}$ une fonction de classe $\mathcal{C}^1$ telle que $f(x) = O \left( \frac{1}{x^2} \right)$ et $f'(x) = O \left( \frac{1}{x^2} \right)$ quand $|x| \longrightarrow +\infty$. Alors :
		\[ \forall x \in \mathbb{R}, \, \sum_{n \in \mathbb{Z}} f(x+n) = \sum_{n \in \mathbb{Z}} \widehat{f}(2 \pi n) e^{2 i \pi n x} \]
		où $\widehat{f}$ désigne la transformée de Fourier de $f$.
	\end{theorem}

	\begin{application}[Identité de Jacobi]
		\[ \forall s > 0, \, \sum_{n=-\infty}^{+\infty} e^{-\pi n^2 s} = \frac{1}{\sqrt{s}} \sum_{n=-\infty}^{+\infty} e^{-\frac{\pi n^2}{s}} \]
	\end{application}

	\subsubsection{Inégalités remarquables}

	\reference[AMR08]{215}

	\begin{application}[Inégalité isopérimétrique]
		Soit $\gamma : [0,1] \rightarrow \mathbb{R}^2$ une courbe de Jordan (ie. $\gamma(0) = \gamma(1)$, $\gamma$ est injective sur $]0,1[$ et $\gamma' \neq 0$) de classe $\mathcal{C}^1$ de longueur $L$ et enfermant une surface $S$. Alors,
		\[ S \leq \frac{L^2}{4 \pi} \]
		avec égalité si et seulement si $\gamma$ définit un cercle.
	\end{application}

	\begin{application}[Inégalité de Wirtinger]
		Soit $f : [a,b] \rightarrow \mathbb{C}$ de classe $\mathcal{C}^1$ telle que $f(a) = f(b) = 0$. Alors,
		\[ \int_a^b \vert f(x) \vert^2 \, \mathrm{d}x \leq \frac{(b-a)^2}{\pi}^2 \int_a^b \vert f'(x) \vert^2 \, \mathrm{d}x \]
		De plus, la constante $\frac{(b-a)^2}{\pi}^2$ est optimale.
	\end{application}

	\reference[Z-Q]{106}

	\begin{application}[Inégalité de Bernstein]
		Soient $\lambda > 0$ et $\lambda_1, \dots, \lambda_n \in \mathbb{R}$ distincts et tels que $\max_{j \in \llbracket 1, n \rrbracket} \vert \lambda_j \vert < \lambda$. On définit
		\[ h : t \mapsto \sum_{j=1}^n a_j e^{i \lambda_j t} \text{ où } a_1, \dots, a_n \in \mathbb{C} \]
		Alors $h$ et sa dérivée $h'$ sont bornées et on a :
		\[ \Vert h' \Vert_\infty \leq \lambda \Vert h \Vert_\infty \]
	\end{application}

	\newpage
	\subsection*{Annexes}

	\begin{figure}[H]
		\begin{center}
			\begin{whitetabularx}{|X|X|}
				\hline
				\textbf{Hypothèses sur $f$} & \textbf{Convergence de sa série de Fourier $(S_N(f))$} \\
				\hline
				$f \in L_2^{2\pi}$ & Convergence pour $\Vert . \Vert_2$. \\
				\hline
				$f$ continue & Convergence uniforme au sens de Cesàro. \\
				\hline
				$f \in L_p^{2\pi}$ ($p \in L_p[1,+\infty[$) & Convergence pour $\Vert . \Vert_p$ au sens de Cesàro. \\
				\hline
				$f$ $\mathcal{C}^1$ par morceaux & Convergence ponctuelle vers une valeur moyenne. \\
				\hline
				$f$ continue et $\mathcal{C}^1$ par morceaux & Convergence normale. \\
				\hline
			\end{whitetabularx}
		\end{center}
		\caption{Convergence d'une série de Fourier selon les hypothèses sur la fonction de départ}
	\end{figure}
	%</content>
\end{document}
