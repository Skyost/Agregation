\documentclass[12pt, a4paper]{report}

% LuaLaTeX :

\RequirePackage{iftex}
\RequireLuaTeX

% Packages :

\usepackage[french]{babel}
%\usepackage[utf8]{inputenc}
%\usepackage[T1]{fontenc}
\usepackage[pdfencoding=auto, pdfauthor={Hugo Delaunay}, pdfsubject={Mathématiques}, pdfcreator={agreg.skyost.eu}]{hyperref}
\usepackage{amsmath}
\usepackage{amsthm}
%\usepackage{amssymb}
\usepackage{stmaryrd}
\usepackage{tikz}
\usepackage{tkz-euclide}
\usepackage{fourier-otf}
\usepackage{fontspec}
\usepackage{titlesec}
\usepackage{fancyhdr}
\usepackage{catchfilebetweentags}
\usepackage[french, capitalise, noabbrev]{cleveref}
\usepackage[fit, breakall]{truncate}
\usepackage[top=2.5cm, right=2cm, bottom=2.5cm, left=2cm]{geometry}
\usepackage{enumerate}
\usepackage{tocloft}
\usepackage{microtype}
%\usepackage{mdframed}
%\usepackage{thmtools}
\usepackage{xcolor}
\usepackage{tabularx}
\usepackage{aligned-overset}
\usepackage[subpreambles=true]{standalone}
\usepackage{environ}
\usepackage[normalem]{ulem}
\usepackage{marginnote}
\usepackage{etoolbox}
\usepackage{setspace}
\usepackage[bibstyle=reading, citestyle=draft]{biblatex}
\usepackage{xpatch}
\usepackage[many, breakable]{tcolorbox}
\usepackage[backgroundcolor=white, bordercolor=white, textsize=small]{todonotes}

% Bibliographie :

\newcommand{\overridebibliographypath}[1]{\providecommand{\bibliographypath}{#1}}
\overridebibliographypath{../bibliography.bib}
\addbibresource{\bibliographypath}
\defbibheading{bibliography}[\bibname]{%
	\newpage
	\section*{#1}%
}
\renewbibmacro*{entryhead:full}{\printfield{labeltitle}}
\DeclareFieldFormat{url}{\newline\footnotesize\url{#1}}
\AtEndDocument{\printbibliography}

% Police :

\setmathfont{Erewhon Math}

% Tikz :

\usetikzlibrary{calc}

% Longueurs :

\setlength{\parindent}{0pt}
\setlength{\headheight}{15pt}
\setlength{\fboxsep}{0pt}
\titlespacing*{\chapter}{0pt}{-20pt}{10pt}
\setlength{\marginparwidth}{1.5cm}
\setstretch{1.1}

% Métadonnées :

\author{agreg.skyost.eu}
\date{\today}

% Titres :

\setcounter{secnumdepth}{3}

\renewcommand{\thechapter}{\Roman{chapter}}
\renewcommand{\thesubsection}{\Roman{subsection}}
\renewcommand{\thesubsubsection}{\arabic{subsubsection}}
\renewcommand{\theparagraph}{\alph{paragraph}}

\titleformat{\chapter}{\huge\bfseries}{\thechapter}{20pt}{\huge\bfseries}
\titleformat*{\section}{\LARGE\bfseries}
\titleformat{\subsection}{\Large\bfseries}{\thesubsection \, - \,}{0pt}{\Large\bfseries}
\titleformat{\subsubsection}{\large\bfseries}{\thesubsubsection. \,}{0pt}{\large\bfseries}
\titleformat{\paragraph}{\bfseries}{\theparagraph. \,}{0pt}{\bfseries}

\setcounter{secnumdepth}{4}

% Table des matières :

\renewcommand{\cftsecleader}{\cftdotfill{\cftdotsep}}
\addtolength{\cftsecnumwidth}{10pt}

% Redéfinition des commandes :

\renewcommand*\thesection{\arabic{section}}
\renewcommand{\ker}{\mathrm{Ker}}

% Nouvelles commandes :

\newcommand{\website}{https://agreg.skyost.eu}

\newcommand{\tr}[1]{\mathstrut ^t #1}
\newcommand{\im}{\mathrm{Im}}
\newcommand{\rang}{\operatorname{rang}}
\newcommand{\trace}{\operatorname{trace}}
\newcommand{\id}{\operatorname{id}}
\newcommand{\stab}{\operatorname{Stab}}

\providecommand{\newpar}{\\[\medskipamount]}

\providecommand{\lesson}[3]{%
	\title{#3}%
	\hypersetup{pdftitle={#3}}%
	\setcounter{section}{\numexpr #2 - 1}%
	\section{#3}%
	\fancyhead[R]{\truncate{0.73\textwidth}{#2 : #3}}%
}

\providecommand{\development}[3]{%
	\title{#3}%
	\hypersetup{pdftitle={#3}}%
	\section*{#3}%
	\fancyhead[R]{\truncate{0.73\textwidth}{#3}}%
}

\providecommand{\summary}[1]{%
	\textit{#1}%
	\medskip%
}

\tikzset{notestyleraw/.append style={inner sep=0pt, rounded corners=0pt, align=center}}

%\newcommand{\booklink}[1]{\website/bibliographie\##1}
\newcommand{\citelink}[2]{\hyperlink{cite.\therefsection @#1}{#2}}
\newcommand{\previousreference}{}
\providecommand{\reference}[2][]{%
	\notblank{#1}{\renewcommand{\previousreference}{#1}}{}%
	\todo[noline]{%
		\protect\vspace{16pt}%
		\protect\par%
		\protect\notblank{#1}{\cite{[\previousreference]}\\}{}%
		\protect\citelink{\previousreference}{p. #2}%
	}%
}

\definecolor{devcolor}{HTML}{00695c}
\newcommand{\dev}[1]{%
	\reversemarginpar%
	\todo[noline]{
		\protect\vspace{16pt}%
		\protect\par%
		\bfseries\color{devcolor}\href{\website/developpements/#1}{DEV}
	}%
	\normalmarginpar%
}

% En-têtes :

\pagestyle{fancy}
\fancyhead[L]{\truncate{0.23\textwidth}{\thepage}}
\fancyfoot[C]{\scriptsize \href{\website}{\texttt{agreg.skyost.eu}}}

% Couleurs :

\definecolor{property}{HTML}{fffde7}
\definecolor{proposition}{HTML}{fff8e1}
\definecolor{lemma}{HTML}{fff3e0}
\definecolor{theorem}{HTML}{fce4f2}
\definecolor{corollary}{HTML}{ffebee}
\definecolor{definition}{HTML}{ede7f6}
\definecolor{notation}{HTML}{f3e5f5}
\definecolor{example}{HTML}{e0f7fa}
\definecolor{cexample}{HTML}{efebe9}
\definecolor{application}{HTML}{e0f2f1}
\definecolor{remark}{HTML}{e8f5e9}
\definecolor{proof}{HTML}{e1f5fe}

% Théorèmes :

\theoremstyle{definition}
\newtheorem{theorem}{Théorème}

\newtheorem{property}[theorem]{Propriété}
\newtheorem{proposition}[theorem]{Proposition}
\newtheorem{lemma}[theorem]{Lemme}
\newtheorem{corollary}[theorem]{Corollaire}

\newtheorem{definition}[theorem]{Définition}
\newtheorem{notation}[theorem]{Notation}

\newtheorem{example}[theorem]{Exemple}
\newtheorem{cexample}[theorem]{Contre-exemple}
\newtheorem{application}[theorem]{Application}

\theoremstyle{remark}
\newtheorem{remark}[theorem]{Remarque}

\counterwithin*{theorem}{section}

\newcommand{\applystyletotheorem}[1]{
	\tcolorboxenvironment{#1}{
		enhanced,
		breakable,
		colback=#1!98!white,
		boxrule=0pt,
		boxsep=0pt,
		left=8pt,
		right=8pt,
		top=8pt,
		bottom=8pt,
		sharp corners,
		after=\par,
	}
}

\applystyletotheorem{property}
\applystyletotheorem{proposition}
\applystyletotheorem{lemma}
\applystyletotheorem{theorem}
\applystyletotheorem{corollary}
\applystyletotheorem{definition}
\applystyletotheorem{notation}
\applystyletotheorem{example}
\applystyletotheorem{cexample}
\applystyletotheorem{application}
\applystyletotheorem{remark}
\applystyletotheorem{proof}

% Environnements :

\NewEnviron{whitetabularx}[1]{%
	\renewcommand{\arraystretch}{2.5}
	\colorbox{white}{%
		\begin{tabularx}{\textwidth}{#1}%
			\BODY%
		\end{tabularx}%
	}%
}

% Maths :

\DeclareFontEncoding{FMS}{}{}
\DeclareFontSubstitution{FMS}{futm}{m}{n}
\DeclareFontEncoding{FMX}{}{}
\DeclareFontSubstitution{FMX}{futm}{m}{n}
\DeclareSymbolFont{fouriersymbols}{FMS}{futm}{m}{n}
\DeclareSymbolFont{fourierlargesymbols}{FMX}{futm}{m}{n}
\DeclareMathDelimiter{\VERT}{\mathord}{fouriersymbols}{152}{fourierlargesymbols}{147}


% Bibliographie :

\addbibresource{\bibliographypath}%
\defbibheading{bibliography}[\bibname]{%
	\newpage
	\section*{#1}%
}
\renewbibmacro*{entryhead:full}{\printfield{labeltitle}}%
\DeclareFieldFormat{url}{\newline\footnotesize\url{#1}}%

\AtEndDocument{\printbibliography}

\begin{document}
	%<*content>
	\lesson{analysis}{214}{Théorème d'inversion locale, théorème des fonctions implicites. Illustrations en analyse et en géométrie.}
	
	Soient $p$, $q$ deux entiers non nuls et $U \subseteq \mathbb{R}^p$ un ouvert. Pour simplifier les notations, nous nous restreignons aux produits de $\mathbb{R}$, mais il est possible de généraliser la plupart des résultats significatifs ci-dessous à des espaces de Banach.
	
	\subsection{Théorème d'inversion locale}
	
	\subsubsection{Difféomorphisme}
	
	\reference[GOU20]{341}
	
	Pour une fonction réelle $f : \mathbb{R} \rightarrow \mathbb{R}$ de classe $\mathcal{C}^1$, on sait que si $f'(x) \neq 0$ pour tout $x \in \mathbb{R}$, alors $f$ admet un inverse global $f^{-1}$ qui vérifie
	\[ \forall x \in \mathbb{R}, \, f'(f(x)) = \frac{1}{f'(x)} \]
	L'objectif ici va être de généraliser ce résultat.
	
	\reference[ROU]{54}
	
	\begin{definition}
		Soit $f : U \rightarrow \mathbb{R}^q$. On dit que $f$ est un \textbf{difféomorphisme} de classe $\mathcal{C}^1$ de $U$ sur $V = f(u)$ si $f$ et $f^{-1}$ sont bijectives et de classe $\mathcal{C}^1$ respectivement sur $U$ et $V$.
	\end{definition}
	
	\begin{proposition}
		Soit $f : U \rightarrow \mathbb{R}^q$ un difféomorphisme. Alors :
		\begin{enumerate}[label=(\roman*)]
			\item Pour tout $x \in U$, en posant $y = f(x)$,
			\[ \mathrm{d}(f^{-1})_y \circ \mathrm{d}f_x = \operatorname{id} \]
			\item $p=q$.
		\end{enumerate}
	\end{proposition}
	
	\begin{example}
		$x \mapsto x^3$ est un homéomorphisme de $\mathbb{R}$ sur $\mathbb{R}$, de classe $\mathcal{C}^1$, mais n'est pas un difféomorphisme.
	\end{example}
	
	\subsubsection{Énoncé}
	
	\reference[GOU20]{343}
	
	\begin{theorem}[Inversion locale]
		Soit $f : U \rightarrow \mathbb{R}^q$ de classe $\mathcal{C}^1$. On suppose qu'il existe $a \in U$ tel que $\mathrm{d}f_a$ est inversible.
		\newpar
		Alors, il existe $V$ voisinage de $a$ et $W$ voisinage de $f(a)$ tels que $f_{|V}$ soit un difféomorphisme de classe $\mathcal{C}^1$ de $V$ sur $W$.
	\end{theorem}
	
	\begin{remark}
		\begin{itemize}
			\item On peut remplacer ``$\mathcal{C}^1$'' par ``$\mathcal{C}^k$'' dans le théorème.
			\item $\mathrm{d}f_a$ est inversible si et seulement si le jacobien de $f$ en $a$, $\det \operatorname{Jac}(f)_a$, est non nul.
		\end{itemize}
	\end{remark}
	
	\begin{corollary}
		Soit $f : U \rightarrow \mathbb{R}^q$ de classe $\mathcal{C}^1$. On suppose que pour tout $a \in U$, $\mathrm{d}f_a$ est inversible. Alors $f$ est une application ouverte.
	\end{corollary}
	
	\reference{347}
	
	\begin{example}
		\label{214-1}
		L'application de $\mathbb{R}^2$ dans $\mathbb{R}^2$ définie par $(x, y) \mapsto (x^2-y^2, xy)$ est un difféomorphisme de classe $\mathcal{C}^\infty$ en tout point de $\mathbb{R}^2 \setminus (0,0)$.
	\end{example}
	
	\reference[BMP]{9}
	
	\begin{application}
		Soit $\varphi : U \rightarrow \mathbb{R}^q$ un difféomorphisme de classe $\mathcal{C}^1$. Alors, $V = \varphi(U)$ est mesurable et tout fonction $f$ appartient à $L_1$ si et seulement si $\vert \det \operatorname{Jac}(\varphi)_a \vert f \circ \varphi$ appartient à $L_1$. Dans ce cas,
		\[ \int_V f(x) \, \mathrm{d}x = \int_U \vert \det \operatorname{Jac}(\varphi)_a \vert f (\varphi(y)) \, \mathrm{d}y \]
	\end{application}
	
	\reference[GOU20]{355}
	
	\begin{example}
		En passant en coordonnées polaires,
		\[ \int_{\mathbb{R}} e^{-x^2} \, \mathrm{d}x = \sqrt{\pi} \]
	\end{example}
	
	\reference[BMP]{9}
	
	\begin{application}
		Soient $A \in \mathcal{M}_n(\mathbb{R})$ et $k$ un entier. Alors, si $A$ est suffisamment proche de l'identité $I_n$, $A$ est une racine $k$-ième (ie. $\exists B \in \mathcal{M}_n(\mathbb{R})$ telle que $B^k = A$).
	\end{application}
	
	\subsubsection{Généralisation}
	
	\begin{theorem}[Inversion globale]
		Soit $f : U \rightarrow \mathbb{R}^q$ de classe $\mathcal{C}^1$. Alors, $f$ est un difféomorphisme de classe $\mathcal{C}^1$ de $U$ sur $V = f(U)$ si et seulement si $f$ est injective sur $U$ et $\mathrm{d}f_a$ est un isomorphisme pour tout $a \in U$.
	\end{theorem}
	
	\reference[GOU20]{347}
	
	\begin{example}
		L'application de l'\cref{214-1} n'est pas un difféomorphisme global.
	\end{example}
	
	\reference[ROU]{191}
	
	\begin{remark}
		Il existe une version holomorphe de ce théorème :
		\newpar
		Soient $U$ un ouvert connexe de $\mathbb{C}$ et $f : U \rightarrow \mathbb{C}$ holomorphe sur $U$. On suppose $f$ injective sur $U$. Alors, $V = f(U)$ est un ouvert (connexe) de $\mathbb{C}$ et $f$ est un difféomorphisme holomorphe de classe $\mathcal{C}^1$ de $U$ sur $V$.
		\newpar
		Remarquons que seule l'injectivité de $f$ suffit.
	\end{remark}
	
	\reference{231}
	
	\begin{theorem}[du rang constant]
		Soit $f : U \rightarrow \mathbb{R}^q$ de classe $\mathcal{C}^1$. On suppose que le rang de $\mathrm{d}f_x$ est constant égal à $r \leq p$ pour tout $x \in U$. Soit $a \in U$. Alors, il existe $V$ voisinage de $a$, $W$ voisinage de $f(a)$ et deux difféomorphismes $\phi : V \rightarrow V$ et $\psi : W \rightarrow W$ tels que
		\[ \phi \circ f \circ \psi = \pi_r \]
		où $\pi_r$ désigne la projection de $\mathbb{R}^p$ sur $\mathbb{R}^r$ : $\pi_r : (x_1, \dots, x_p) \mapsto (x_1, \dots, x_{r-1}, x_r, 0, \dots, 0)$.
	\end{theorem}
	
	\subsection{Théorème des fonctions implicites}
	
	\subsubsection{Énoncé}
	
	\reference[GOU20]{344}
	
	\begin{definition}
		Soient $E_1, \dots, E_n, F$ des espaces de Banach, $\Omega \subseteq E$ un ouvert où $E = E_1 \times \dots \times E_n$ et $a = (a_1, \dots, a_n) \in E$. Soit $f : \Omega \rightarrow F$. Alors, pour tout $i \in \llbracket 1, n \rrbracket$, $f_i : x \mapsto f(a_1, \dots, a_{i-1}, x, a_{i+1}, \dots, a_n)$ est définie sur un voisinage de $a_i$ dans $E_i$. Si elle est différentiable en $a_i$, on dit que $f$ admet une \textbf{différentielle partielle} d'indice $i$ en $a$, et on note celle-ci $\partial_i f_a$.
	\end{definition}
	
	\begin{remark}
		En reprenant les notations précédentes :
		\begin{itemize}
			\item Si pour tout $i \in \llbracket 1, n \rrbracket$, $E_i = \mathbb{R}$ et $F = E = \mathbb{R}^n$, alors $\partial_i f_a = h \frac{\partial f}{\partial x_i} (a)$.
			\item Si $f$ est différentiable en $a$, alors pour tout $i \in \llbracket 1, n \rrbracket$, $\partial_i f_a$ existe et
			\[ \forall h = (h_1, \dots, h_n) \in E, \, \mathrm{d}f_a(h) = \sum_{i=1}^{n} \partial_i f_a(h_i) \]
		\end{itemize}
	\end{remark}
	
	\begin{theorem}[des fonctions implicites]
		Soient $n, m \in \mathbb{N}^*$ tels que $n+m=p$. Soient $U \times V \subseteq \mathbb{R}^n \times \mathbb{R}^m$ où $U$ et $V$ sont des ouvertes. Soit $f : U \times V \rightarrow \mathbb{R}^q$ de classe $\mathcal{C}^1$. On suppose qu'il existe $(a,b) \in U \times V$ tel que $\partial_2 f_{(a,b)} : \mathbb{R}^n \rightarrow \mathbb{R}^q$ est un isomorphisme. Alors, il existe :
		\begin{itemize}
			\item Un voisinage ouvert $U_0$ de $a$ dans $U$.
			\item Un voisinage ouvert $W$ de $f(a,b)$.
			\item Un voisinage ouvert $\Omega$ de $(a,b)$ dans $U \times V$.
			\item Une fonction $\varphi : U_0 \times W \rightarrow V$ de classe $\mathcal{C}^1$.
		\end{itemize}
		Vérifiant :
		\[ \forall x \in U_0, \, \forall z \in W, \, \exists! y \in V \text{ tel que } f(x,y)=z \text{ avec } (x, y) \in \Omega \text{ et } y=\varphi(x,z) \]
		En particulier,
		\[ \forall (x,z) \in U_0 \times W, \, f(x, \varphi(x,z)) = z \]
	\end{theorem}
	
	\reference[BMP]{11}
	
	\begin{remark}
		Avec les notations précédentes, si $p = 2$, on peut choisir n'importe quelle variable pour obtenir
		\[ y = \varphi(x) \text{ si } \frac{\partial f}{\partial y}(a,b) \neq 0 \text{ ou } x = \varphi(y) \text{ si } \frac{\partial f}{\partial y}(a,b) \neq 0 \]
	\end{remark}
	
	\reference[ROU]{193}
	
	\begin{remark}
		La signification de ce théorème est que la surface définie implicitement par l'équation $f(x,y)=0$ peut, au moins localement, être vue comme le graphe d'une fonction $\varphi$.
	\end{remark}
	
	\begin{proposition}
		Avec les notations précédentes, la différentielle de la fonction implicite $\varphi$ est donnée par :
		\[ \mathrm{d}\varphi_x = -(\partial_2 f_{(x, \varphi(x)}))^{-1} \circ (\partial_1 f_{(x, \varphi(x))}) \]
	\end{proposition}
	
	\subsubsection{Exemples}
	
	\begin{example}
		Pour l'équation $x^2 + y^2 - 1 = 0$, on a $\partial_2 f_{(x,y)} = 2y$. On exclue les points où $y = 0$. En prenant $(0,1)$ et $(0,-1)$ pour points de départ, on a deux fonctions implicites qui correspondent aux demi-cercles supérieur et inférieur :
		\begin{itemize}
			\item $y = \varphi_1(x) = \sqrt{1-x^2}$.
			\item $y = \varphi_2(x) = -\sqrt{1-x^2}$.
		\end{itemize}
		De plus, en dérivant par rapport à $x$ : $2x + 2yy' = 0$ et, $y' = \varphi_1'(x) = \frac{-x}{y}$.
	\end{example}
	
	\reference{237}
	
	\begin{example}[Folium de Descartes]
		Soit $C = \{ (x,y) \in \mathbb{R}^2 \mid x^3 + y^3 - 3xy = 0 \}$. En tout point $(a,b) \in \mathbb{R}^2 \setminus \{ (0,0), (2^{\frac{2}{3}}, 2^{\frac{1}{3}}) \}$, $C$ peut être vu comme le graphe d'une fonction $\varphi$ telle que
		\[ \varphi'(a) = \frac{a^2-b}{a-b^2} \]
	\end{example}
	
	\reference[GOU20]{348}
	
	\begin{example}
		Soit $f : (x,y) \mapsto \sin(y) + xy^4 + x^2$. Alors, il existe $U, V$ deux voisinages ouverts de $0$ dans $\mathbb{R}$, $y = \varphi(x) \in V$ est l'unique solution de $f(x,y) = 0$. De plus, on a un développement limité de $\varphi$ :
		\[ \varphi(x) = -x^2 - \frac{x^6}{6} - x^9 - \frac{x^{10}}{40} + o(x^{11}) \]
	\end{example}
	
	\subsection{Applications}
	
	Soit $n$ un entier non nul.
	
	\subsubsection{Homéomorphismes}
	
	\reference[ROU]{209}
	
	\begin{notation}
		Si $f : \mathbb{R}^n \rightarrow \mathbb{R}$ est une application dont toutes les dérivées secondes existent, on note $\mathrm{H}(f)_a$ la hessienne de $f$ au point $a$.
	\end{notation}
	
	\begin{lemma}
		Soit $A_0 \in \mathcal{S}_n(\mathbb{R})$ inversible. Alors il existe un voisinage $V$ de $A_0$ dans $\mathcal{S}_n(\mathbb{R})$ et une application $\psi : V \rightarrow \mathrm{GL}_n(\mathbb{R})$ de classe $\mathcal{C}^1$ telle que
		\[ \forall A \in V, \, A = \tr \psi(A) A_0 \psi(A) \]
	\end{lemma}
	
	\reference{354}
	\dev{lemme-de-morse}
	
	\begin{lemma}[Morse]
		Soit $f : U \rightarrow \mathbb{R}$ une fonction de classe $\mathcal{C}^3$ (où $U$ désigne un ouvert de $\mathbb{R}^n$ contenant l'origine). On suppose :
		\begin{itemize}
			\item $\mathrm{d} f_0 = 0$.
			\item La matrice symétrique $\mathrm{H} (f)_0$ est inversible.
			\item La signature de $\mathrm{H}(f)_0$ est $(p, n-p)$.
		\end{itemize}
		Alors il existe un difféomorphisme $\phi = (\phi_1, \dots, \phi_n)$ de classe $\mathcal{C}^1$ entre deux voisinage de l'origine de $\mathbb{R}^n$ $V \subseteq U$ et $W$ tel que $\varphi(0) = 0$ et
		\[ \forall x \in U, \, f(x) - f(0) = \sum_{k=1}^p \phi_k^2(x) - \sum_{k=p+1}^n \phi_k^2(x) \]
	\end{lemma}
	
	\reference{334}
	
	\begin{example}
		On considère $f : (x,y) \mapsto x^2-y^2+\frac{y^4}{4}$. La courbe d'équation
		\[ f(x,y) = 0 \]
		est (au changement près du nom des coordonnées) une projection de l'intersection d'un cylindre et d'une sphère tangents. On a
		\[ f = u^2 - v^2 \]
		avec $u : (x,y) \mapsto x$ et $v : (x,y) \mapsto y \sqrt{1-\frac{y^2}{4}}$.
	\end{example}
	
	\reference{341}
	
	\begin{application}
		Soit $S$ la surface d'équation $z = f(x, y)$ où $f$ est de classe $\mathcal{C}^3$ au voisinage de l'origine. On suppose la forme quadratique $\mathrm{d}^2 f_0$ non dégénérée. Alors, en notant $P$ le plan tangent à $S$ en $0$ :
		\begin{enumerate}[label=(\roman*)]
			\item Si $\mathrm{d}^2 f_0$ est de signature $(2, 0)$, alors $S$ est au-dessus de $P$ au voisinage de $0$.
			\item Si $\mathrm{d}^2 f_0$ est de signature $(0, 2)$, alors $S$ est en-dessous de $P$ au voisinage de $0$.
			\item Si $\mathrm{d}^2 f_0$ est de signature $(1, 1)$, alors $S$ traverse $P$ selon une courbe admettant un point double en $(0, f(0))$.
		\end{enumerate}
	\end{application}
	
	\reference[BMP]{15}
	
	\begin{application}
		Soit $f : \mathbb{R}^n \rightarrow \mathbb{R}$ de classe $\mathcal{C}^3$ telle que $\mathrm{d}f_0 = 0$ et $\mathrm{d}^2 f_0$ est définie positive. Alors $0$ est un minimum local (strict) de $f$.
	\end{application}
	
	\subsubsection{Optimisation}
	
	\reference{337}
	
	\begin{theorem}[Extrema liés]
		\label{214-2}
		Soit $U$ un ouvert de $\mathbb{R}^n$ et soient $f, g_1, \dots, g_r : U \rightarrow \mathbb{R}$ des fonctions de classe $\mathcal{C}^1$. On note $\Gamma = \{ x \in U \mid g_1(x) = \dots = g_r(x) = 0 \}$. Si $f_{|\Gamma}$ admet un extremum relatif en $a \in \Gamma$ et si les formes linéaires $\mathrm{d}(g_1)_a, \dots, \mathrm{d}(g_r)_a$ sont linéairement indépendantes, alors il existe des uniques $\lambda_1, \dots, \lambda_r$ tels que
		\[ \mathrm{d}f_a = \lambda_1 \mathrm{d}(g_1)_a + \dots + \lambda_r \mathrm{d}(g_r)_a \]
	\end{theorem}
	
	\begin{definition}
		Les $\lambda_1, \dots, \lambda_r$ du théorème précédent sont appelés appelés \textbf{multiplicateurs de Lagrange}.
	\end{definition}
	
	\reference[BMP]{21}
	
	\begin{remark}
		La relation finale du \cref{214-2} équivaut à
		\[ \bigcap_{i=1}^n \ker(\mathrm{d}(g_i)_a) \subseteq \ker(\mathrm{d}f_a) \]
		et elle exprime que $\mathrm{d}f_a$ est nulle sur l'espace tangent à $\Gamma$ en $a$ (ie. $\nabla f_a$ est orthogonal à l'espace tangent à $\Gamma$ en $a$).
	\end{remark}
	
	\begin{cexample}
		On pose $g : (x,y) \mapsto x^3-y^2$ et on considère $f : (x, y) \mapsto x+y^2$. On cherche à minimiser $f$ sous la contrainte $g(x,y) = 0$.
		\newpar
		Alors, le minimum (global) de $f$ sous cette contrainte est atteint en $(0,0)$, la différentielle de $g$ en $(0,0)$ est nulle et la relation finale du \cref{214-2} n'est pas vraie.
	\end{cexample}
	
	\begin{application}[Théorème spectral]
		Tout endomorphisme symétrique d'un espace euclidien se diagonalise dans une base orthonormée.
	\end{application}
	
	\reference{35}
	
	\begin{application}
		\[ \mathrm{SO}_n(\mathbb{R}) = \left\{ M \in \mathcal{M}_n(\mathbb{R}) \mid \Vert M \Vert^2 = \inf_{P \in \mathrm{SL}_n(\mathbb{R})} \Vert P \Vert^2 \right\} \]
		où $\Vert . \Vert : M \mapsto \sqrt{\trace(\tr{M}M)}$ (ie. $\mathrm{SO}_n(\mathbb{R})$ est l'ensemble des matrices de $\mathrm{SL}_n(\mathbb{R})$ qui minimisent la norme euclidienne canonique de $\mathcal{M}_n(\mathbb{R})$).
	\end{application}
	
	\reference[GOU20]{339}
	
	\begin{application}[Inégalité arithmético-géométrique]
		\[ \forall (x_1, \dots, x_n) \in (\mathbb{R}^+)^n, \, \left( \prod_{i=1}^{n} x_i \right)^{\frac{1}{n}} \leq \frac{1}{n} \sum_{i=1}^n x_i \]
	\end{application}
	
	\reference[ROU]{409}
	
	\begin{application}[Inégalité d'Hadamard]
		\[ \forall (x_1, \dots, x_n) \in \mathbb{R}^n, \, \det(x_1, \dots, x_n) \leq \Vert x_1 \Vert \dots \Vert x_n \Vert \]
		avec égalité si et seulement si $(x_1, \dots, x_n)$ est une base orthogonale de $\mathbb{R}^n$.
	\end{application}
	
	\subsubsection{Régularité des racines d'un polynôme}
	
	\reference[BMP]{11}
	
	\begin{proposition}
		Soient $P_0 \in \mathbb{R}_n[X]$ et $x_0 \in \mathbb{R}$ une racine simple de $P_0$. Alors, il existe $\varphi$ une application $\mathcal{C}^\infty$ définie sur un voisinage $U$ de $P_0$ dans $\mathbb{R}_n[X]$ à valeurs dans un voisinage $V$ de $x_0$ telle que
		\[ \forall P \in U, \forall x \in V, \, x = \varphi(P) \iff P(x) = 0 \]
	\end{proposition}
	
	\begin{application}
		Soit $\mathcal{S}_{rs}$ l'ensemble des polynômes de $\mathbb{R}_n[X]$ scindés à racines simples. Alors, $\mathcal{S}_{rs}$ est un ouvert de $\mathbb{R}_n[X]$.
	\end{application}
	
	\subsubsection{Surjectivité de l'exponentielle matricielle}
	
	\reference[I-P]{396}
	
	\begin{lemma}
		\begin{enumerate}[label=(\roman*)]
			\item Soit $A \in \mathcal{M}_n(\mathbb{C})$. Alors $\exp(A) \in \mathrm{GL}_n(\mathbb{C})$.
			\item $\exp$ est différentiable en $0$ et $\mathrm{d}\exp_0 = \operatorname{id}_{\mathcal{M}_n(\mathbb{C})}$.
			\item Soit $M \in \mathrm{GL}_n(\mathbb{C})$. Alors $M^{-1} \in \mathbb{C}[M]$.
		\end{enumerate}
	\end{lemma}
	
	\dev{surjectivite-de-l-exponentielle}
	
	\begin{theorem}
		$\exp : \mathcal{M}_n(\mathbb{C}) \rightarrow \mathrm{GL}_n(\mathbb{C})$ est surjective.
	\end{theorem}
	
	\begin{application}
		$\exp(\mathcal{M}_n(\mathbb{R})) = \mathrm{GL}_n(\mathbb{R})^2$, où $\mathrm{GL}_n(\mathbb{R})^2$ désigne les carrés de $\mathrm{GL}_n(\mathbb{R})$.
	\end{application}
	
	\newpage
	\subsection*{Annexes}
	
	\reference[BMP]{10}
	
	\begin{figure}[H]
		\begin{center}
			\begin{tikzpicture}
				\draw [thick, fill=teal, fill opacity=0.05]  plot[smooth, tension=.7] coordinates {(-4,2.5) (-3,3) (-2,2.8) (-0.8,2.5) (-0.5,1.5) (0,0) (0,-2) (-1.5,-2.5) (-4,-2) (-5,-0.5) (-5,1) (-4,2.5)};
				\draw [thick, fill=teal, fill opacity=0.05]  plot[smooth, tension=.9] coordinates {(3,1) (4,2.5) (5,2.8) (6,2) (6.5,0.5) (7,0) (7.5,-1) (6,-2) (4,-2.5) (3.5,-1.5) (3,1)};
				\begin{scope}[scale=0.5,rotate=80,shift={(-6,5)}]
					\draw [thick, fill=teal, fill opacity=0.2]  plot[smooth, tension=.9] coordinates {(3,1) (4,2.5) (5,2.8) (6,2) (6.5,0.5) (7,0) (7.5,-1) (6,-2) (4,-2.5) (3.5,-1.5) (3,1)};
					\draw(4.75, 0.75) node {$\bullet$} node[above]{$a$};
					\node at (4.75, -1) {$V$};
				\end{scope}
				\begin{scope}[scale=0.5,shift={(-1,0)}]
					\draw [thick, fill=teal, fill opacity=0.2]  plot[smooth, tension=.9] coordinates {(10,0.5) (11,1.5) (12,2) (13,1.5) (13.5,0.5) (14,-0.5) (13.5,-2) (12,-2.5) (10.5,-2) (9,-1.5) (9,0.5) (10,0.5)};
					\draw(11.75, -0.5) node {$\bullet$} node[above]{$f(a)$};
					\node at (10, -0.75) {$W$};
				\end{scope}
				\draw[->] (-3,2) to [out=45,in=135] (5,2);
				\draw[<-] (-3,-0.5) to [out=-45,in=-135] (5,-0.5);
				\node at (1,4) {$f$};
				\node at (1,-2.5) {$f^{-1}_{|V}$};
				\node at (-2.5,-2) {$U$};
			\end{tikzpicture}
		\end{center}
		\caption{Inversion locale}
	\end{figure}
	
	\begin{figure}[H]
		\begin{center}
			\begin{tikzpicture}[scale=2]
				\draw[->] (-3, 0) -- (2.5, 0) node[right] {$x$};
				\draw[->] (0, -0.5) -- (0, 3.5) node[above] {$y$};
				\draw [thick, teal] (-1.25,2.02) to[out=-35,in=-140] (-1,2) to[out=45,in=115] (2,2) to[out=-180+115,in=10] (1,0.5) to[out=180+10,in=10] (-1,1) to[out=180+10,in=90] (-2,-0.5);
				\draw[<->] (-1.25,1) -- (-1.8,0.5);
				\node[align=left] at (-2.4,0.8) {$x = \psi_1(x) = \varphi^{-1}(y)$ \\ $y = \varphi_1(x)$};
				\draw[<->] (0.45,0.48) -- (1.15,0.48);
				\node[align=left] at (0.8,0.3) {$y = \varphi_2(x)$};
				\draw[<->] (2.08,1.3) -- (2.08,2);
				\node[align=left] at (2.57,1.65) {$x = \varphi_2(y)$};
			\end{tikzpicture}
		\end{center}
		\caption{Fonctions implicites}
	\end{figure}
	%</content>
\end{document}
