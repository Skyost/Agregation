\documentclass[12pt, a4paper]{report}

% LuaLaTeX :

\RequirePackage{iftex}
\RequireLuaTeX

% Packages :

\usepackage[french]{babel}
%\usepackage[utf8]{inputenc}
%\usepackage[T1]{fontenc}
\usepackage[pdfencoding=auto, pdfauthor={Hugo Delaunay}, pdfsubject={Mathématiques}, pdfcreator={agreg.skyost.eu}]{hyperref}
\usepackage{amsmath}
\usepackage{amsthm}
%\usepackage{amssymb}
\usepackage{stmaryrd}
\usepackage{tikz}
\usepackage{tkz-euclide}
\usepackage{fourier-otf}
\usepackage{fontspec}
\usepackage{titlesec}
\usepackage{fancyhdr}
\usepackage{catchfilebetweentags}
\usepackage[french, capitalise, noabbrev]{cleveref}
\usepackage[fit, breakall]{truncate}
\usepackage[top=2.5cm, right=2cm, bottom=2.5cm, left=2cm]{geometry}
\usepackage{enumerate}
\usepackage{tocloft}
\usepackage{microtype}
%\usepackage{mdframed}
%\usepackage{thmtools}
\usepackage{xcolor}
\usepackage{tabularx}
\usepackage{aligned-overset}
\usepackage[subpreambles=true]{standalone}
\usepackage{environ}
\usepackage[normalem]{ulem}
\usepackage{marginnote}
\usepackage{etoolbox}
\usepackage{setspace}
\usepackage[bibstyle=reading, citestyle=draft]{biblatex}
\usepackage{xpatch}
\usepackage[many, breakable]{tcolorbox}
\usepackage[backgroundcolor=white, bordercolor=white, textsize=small]{todonotes}

% Bibliographie :

\newcommand{\overridebibliographypath}[1]{\providecommand{\bibliographypath}{#1}}
\overridebibliographypath{../bibliography.bib}
\addbibresource{\bibliographypath}
\defbibheading{bibliography}[\bibname]{%
	\newpage
	\section*{#1}%
}
\renewbibmacro*{entryhead:full}{\printfield{labeltitle}}
\DeclareFieldFormat{url}{\newline\footnotesize\url{#1}}
\AtEndDocument{\printbibliography}

% Police :

\setmathfont{Erewhon Math}

% Tikz :

\usetikzlibrary{calc}

% Longueurs :

\setlength{\parindent}{0pt}
\setlength{\headheight}{15pt}
\setlength{\fboxsep}{0pt}
\titlespacing*{\chapter}{0pt}{-20pt}{10pt}
\setlength{\marginparwidth}{1.5cm}
\setstretch{1.1}

% Métadonnées :

\author{agreg.skyost.eu}
\date{\today}

% Titres :

\setcounter{secnumdepth}{3}

\renewcommand{\thechapter}{\Roman{chapter}}
\renewcommand{\thesubsection}{\Roman{subsection}}
\renewcommand{\thesubsubsection}{\arabic{subsubsection}}
\renewcommand{\theparagraph}{\alph{paragraph}}

\titleformat{\chapter}{\huge\bfseries}{\thechapter}{20pt}{\huge\bfseries}
\titleformat*{\section}{\LARGE\bfseries}
\titleformat{\subsection}{\Large\bfseries}{\thesubsection \, - \,}{0pt}{\Large\bfseries}
\titleformat{\subsubsection}{\large\bfseries}{\thesubsubsection. \,}{0pt}{\large\bfseries}
\titleformat{\paragraph}{\bfseries}{\theparagraph. \,}{0pt}{\bfseries}

\setcounter{secnumdepth}{4}

% Table des matières :

\renewcommand{\cftsecleader}{\cftdotfill{\cftdotsep}}
\addtolength{\cftsecnumwidth}{10pt}

% Redéfinition des commandes :

\renewcommand*\thesection{\arabic{section}}
\renewcommand{\ker}{\mathrm{Ker}}

% Nouvelles commandes :

\newcommand{\website}{https://agreg.skyost.eu}

\newcommand{\tr}[1]{\mathstrut ^t #1}
\newcommand{\im}{\mathrm{Im}}
\newcommand{\rang}{\operatorname{rang}}
\newcommand{\trace}{\operatorname{trace}}
\newcommand{\id}{\operatorname{id}}
\newcommand{\stab}{\operatorname{Stab}}

\providecommand{\newpar}{\\[\medskipamount]}

\providecommand{\lesson}[3]{%
	\title{#3}%
	\hypersetup{pdftitle={#3}}%
	\setcounter{section}{\numexpr #2 - 1}%
	\section{#3}%
	\fancyhead[R]{\truncate{0.73\textwidth}{#2 : #3}}%
}

\providecommand{\development}[3]{%
	\title{#3}%
	\hypersetup{pdftitle={#3}}%
	\section*{#3}%
	\fancyhead[R]{\truncate{0.73\textwidth}{#3}}%
}

\providecommand{\summary}[1]{%
	\textit{#1}%
	\medskip%
}

\tikzset{notestyleraw/.append style={inner sep=0pt, rounded corners=0pt, align=center}}

%\newcommand{\booklink}[1]{\website/bibliographie\##1}
\newcommand{\citelink}[2]{\hyperlink{cite.\therefsection @#1}{#2}}
\newcommand{\previousreference}{}
\providecommand{\reference}[2][]{%
	\notblank{#1}{\renewcommand{\previousreference}{#1}}{}%
	\todo[noline]{%
		\protect\vspace{16pt}%
		\protect\par%
		\protect\notblank{#1}{\cite{[\previousreference]}\\}{}%
		\protect\citelink{\previousreference}{p. #2}%
	}%
}

\definecolor{devcolor}{HTML}{00695c}
\newcommand{\dev}[1]{%
	\reversemarginpar%
	\todo[noline]{
		\protect\vspace{16pt}%
		\protect\par%
		\bfseries\color{devcolor}\href{\website/developpements/#1}{DEV}
	}%
	\normalmarginpar%
}

% En-têtes :

\pagestyle{fancy}
\fancyhead[L]{\truncate{0.23\textwidth}{\thepage}}
\fancyfoot[C]{\scriptsize \href{\website}{\texttt{agreg.skyost.eu}}}

% Couleurs :

\definecolor{property}{HTML}{fffde7}
\definecolor{proposition}{HTML}{fff8e1}
\definecolor{lemma}{HTML}{fff3e0}
\definecolor{theorem}{HTML}{fce4f2}
\definecolor{corollary}{HTML}{ffebee}
\definecolor{definition}{HTML}{ede7f6}
\definecolor{notation}{HTML}{f3e5f5}
\definecolor{example}{HTML}{e0f7fa}
\definecolor{cexample}{HTML}{efebe9}
\definecolor{application}{HTML}{e0f2f1}
\definecolor{remark}{HTML}{e8f5e9}
\definecolor{proof}{HTML}{e1f5fe}

% Théorèmes :

\theoremstyle{definition}
\newtheorem{theorem}{Théorème}

\newtheorem{property}[theorem]{Propriété}
\newtheorem{proposition}[theorem]{Proposition}
\newtheorem{lemma}[theorem]{Lemme}
\newtheorem{corollary}[theorem]{Corollaire}

\newtheorem{definition}[theorem]{Définition}
\newtheorem{notation}[theorem]{Notation}

\newtheorem{example}[theorem]{Exemple}
\newtheorem{cexample}[theorem]{Contre-exemple}
\newtheorem{application}[theorem]{Application}

\theoremstyle{remark}
\newtheorem{remark}[theorem]{Remarque}

\counterwithin*{theorem}{section}

\newcommand{\applystyletotheorem}[1]{
	\tcolorboxenvironment{#1}{
		enhanced,
		breakable,
		colback=#1!98!white,
		boxrule=0pt,
		boxsep=0pt,
		left=8pt,
		right=8pt,
		top=8pt,
		bottom=8pt,
		sharp corners,
		after=\par,
	}
}

\applystyletotheorem{property}
\applystyletotheorem{proposition}
\applystyletotheorem{lemma}
\applystyletotheorem{theorem}
\applystyletotheorem{corollary}
\applystyletotheorem{definition}
\applystyletotheorem{notation}
\applystyletotheorem{example}
\applystyletotheorem{cexample}
\applystyletotheorem{application}
\applystyletotheorem{remark}
\applystyletotheorem{proof}

% Environnements :

\NewEnviron{whitetabularx}[1]{%
	\renewcommand{\arraystretch}{2.5}
	\colorbox{white}{%
		\begin{tabularx}{\textwidth}{#1}%
			\BODY%
		\end{tabularx}%
	}%
}

% Maths :

\DeclareFontEncoding{FMS}{}{}
\DeclareFontSubstitution{FMS}{futm}{m}{n}
\DeclareFontEncoding{FMX}{}{}
\DeclareFontSubstitution{FMX}{futm}{m}{n}
\DeclareSymbolFont{fouriersymbols}{FMS}{futm}{m}{n}
\DeclareSymbolFont{fourierlargesymbols}{FMX}{futm}{m}{n}
\DeclareMathDelimiter{\VERT}{\mathord}{fouriersymbols}{152}{fourierlargesymbols}{147}


% Bibliographie :

\addbibresource{\bibliographypath}%
\defbibheading{bibliography}[\bibname]{%
	\newpage
	\section*{#1}%
}
\renewbibmacro*{entryhead:full}{\printfield{labeltitle}}%
\DeclareFieldFormat{url}{\newline\footnotesize\url{#1}}%

\AtEndDocument{\printbibliography}

\begin{document}
	%<*content>
	\lesson{algebra}{153}{Valeurs propres, vecteurs propres. Calculs exacts ou approchés d'éléments propres. Applications.}

	\subsection{Spectre d'un endomorphisme}
	
	Soit $E$ un espace vectoriel sur un corps $\mathbb{K}$ de dimension finie $n$. Soit $u \in \mathcal{L}(E)$ un endomorphisme de $E$.

	\subsubsection{Valeurs propres, vecteurs propres}

	\reference[GOU21]{171}

	\begin{definition}
		Soit $\lambda \in \mathbb{K}$.
		\begin{itemize}
			\item On dit que $\lambda$ est \textbf{valeur propre} de $u$ si $u - \lambda \operatorname{id}_E$ est non injective.
			\item Un vecteur $x \neq 0$ tel que $u(x) = \lambda x$ est un \textbf{vecteur propre} de $u$ associé à la valeur propre $\lambda$.
			\item $E_\lambda = \ker(u - \lambda \operatorname{id}_E)$ est le \textbf{sous-espace propre} associé à la valeur propre $\lambda$.
			\item L'ensemble des valeurs propres de $u$ est appelé \textbf{spectre} de $u$. On le note $\operatorname{Sp}(u)$.
		\end{itemize}
	\end{definition}

	\begin{remark}
		\begin{itemize}
			\item $0$ est valeur propre de $u$ si et seulement si $\ker(f) \neq \{ 0 \}$.
			\item On peut définir de la même manière les mêmes notions pour une matrice de $\mathcal{M}_n(\mathbb{K})$ (une valeur est propre pour une matrice si et seulement si elle l'est pour l'endomorphisme associé). On reprendra les mêmes notations.
			\item Les sous-espaces $E_\lambda$ sont stables par $u$ pour toute valeur propre $\lambda$.
		\end{itemize}
	\end{remark}

	\begin{example}
		$\begin{pmatrix} 1 \\ 1 \\ 1 \end{pmatrix}$ est vecteur propre de $\begin{pmatrix} 0 & 2 & -1 \\ 3 & -2 & 0 \\ -2 & 2 & 1 \end{pmatrix}$ associé à la valeur propre $1$.
	\end{example}

	\begin{theorem}
		Soient $\lambda_1, \dots, \lambda_k$ des valeurs propres de $u$, distinctes deux à deux. Alors les sous-espaces propres $E_{\lambda_1}, \dots, E_{\lambda_k}$ sont en somme directe.
	\end{theorem}

	\reference[ROM21]{604}

	\begin{theorem}
		Soit $P \in \mathbb{K}[X]$. Pour tout valeur propre $\lambda$ de $u$, $P(\lambda)$ est une valeur propre de $P(u)$. Si le corps $\mathbb{K}$ est algébriquement clos, on a alors
		\[ \operatorname{Sp}(P(u)) = \{ P(\lambda) \mid \lambda \in \operatorname{Sp}(u) \} \]
	\end{theorem}

	\begin{cexample}
		Pour $A = \begin{pmatrix} 0 & -1 \\ 1 & 0 \end{pmatrix} \in \mathcal{M}_2(\mathbb{R})$ et $P = X^2$, on a $A^2 = -I_2$ et $\operatorname{Sp}(A) = \emptyset$.
	\end{cexample}

	\subsubsection{Polynôme caractéristique}

	\reference{644}

	\begin{proposition}
		En notant $\chi_u = \det(X \operatorname{id}_E - u)$,
		\[ \operatorname{Sp}(u) = \{ \lambda \in \mathbb{K} \mid \chi_u(\lambda) = 0 \} \]
	\end{proposition}

	\begin{definition}
		Le polynôme $\chi_u$ précédent est appelé \textbf{polynôme caractéristique} de $u$.
	\end{definition}

	\begin{remark}
		On peut définir la même notion pour une matrice $A \in \mathcal{M}_n(\mathbb{K})$, ces deux notions coïncidant bien si $A$ est la matrice de $u$ dans une base quelconque de $E$.
	\end{remark}

	\begin{example}
		Pour $A = \begin{pmatrix} a & b \\ c & d \end{pmatrix} \in \mathcal{M}_2(\mathbb{K})$, on a $\chi_A = X^2 - \trace(A)X + \det(A)$.
	\end{example}

	\begin{proposition}
		Soit $\lambda$ une valeur propre de $u$ de multiplicité $\alpha$ en tant que racine de $\chi_u$. Alors,
		\[ \dim(E_\lambda) \in \llbracket 1, \alpha \rrbracket \]
	\end{proposition}

	\reference[GOU21]{172}

	\begin{proposition}
		\begin{enumerate}[label=(\roman*)]
			\item Le polynôme caractéristique est un invariant de similitude.
			\item Soit $A \in \mathcal{M}_n(\mathbb{K})$. On note $\chi_A = \sum_{k=0}^n a_k X^k$. Alors, $a_0 = \det(A)$ et $a_{n-1} = \trace(A)$ (à un signe près).
		\end{enumerate}
	\end{proposition}

	\subsubsection{Polynôme minimal}

	\reference[ROM21]{604}

	\begin{lemma}
		\begin{enumerate}[label=(\roman*)]
			\item $\mathrm{Ann}(u) = \{ P \in \mathbb{K}[X] \mid P(u) = 0 \}$ est un sous-ensemble de $\mathbb{K}[u]$ non réduit au polynôme nul.
			\item $\mathrm{Ann}(u)$ est le noyau de $P \mapsto P(u)$ : c'est un idéal de $\mathbb{K}[u]$.
			\item Il existe un unique polynôme unitaire engendrant cet idéal.
		\end{enumerate}
	\end{lemma}

	\begin{definition}
		On appelle \textbf{idéal annulateur} de $u$ l'idéal $\mathrm{Ann}(u)$. Le polynôme unitaire générateur est noté $\pi_u$ et est appelé \textbf{polynôme minimal} de $u$.
	\end{definition}

	\begin{remark}
		\begin{itemize}
			\item $\pi_u$ est le polynôme unitaire de plus petit degré annulant $u$.
			\item Si $A \in \mathcal{M}_n(\mathbb{K})$ est la matrice de $u$ dans une base de $E$, on a $\mathrm{Ann}(u) = \mathrm{Ann}(A)$ et $\pi_u = \pi_A$.
		\end{itemize}
	\end{remark}

	\begin{example}
		Un endomorphisme est nilpotent d'indice $q$ si et seulement si son polynôme minimal est $X^q$.
	\end{example}

	\begin{proposition}
		Soit $F$ un sous-espace vectoriel de $E$ stable par $u$. Alors, le polynôme minimal de l'endomorphisme $u_{|F} : F \rightarrow F$ divise $\pi_u$.
	\end{proposition}

	\begin{proposition}
		\begin{enumerate}[label=(\roman*)]
			\item Les valeurs propres de $u$ sont racines de tout polynôme annulateur.
			\item Les valeurs propres de $u$ sont exactement les racines de $\pi_u$.
		\end{enumerate}
	\end{proposition}

	\reference[GOU21]{186}

	\begin{remark}
		$\pi_u$ et $\chi_u$ partagent dont les mêmes racines.
	\end{remark}

	\reference[ROM21]{607}

	\begin{theorem}[Cayley-Hamilton]
		\[ \pi_u \mid \chi_u \]
	\end{theorem}

	\begin{corollary}
		\[ \dim(\mathbb{K}[u]) \leq n \]
	\end{corollary}

	\subsection{Localisation}
	
	Soit $A = (a_{i,j})_{i, j \in \llbracket 1, n \rrbracket} \in \mathcal{M}_n(\mathbb{C})$.
	
	\subsubsection{Disques de Gerschgörin}
	
	\reference{650}
	
	\begin{notation}
		On note :
		\begin{itemize}
			\item Pour tout $i \in \llbracket 1, n \rrbracket$, $L_i = \sum_{\substack{j=1 \\ j \neq i}}^n \vert a_{i,j} \vert$ et $L = \max_{i \in \llbracket 1, n \rrbracket} \{ L_i + \vert a_{i,i} \vert \}$.
			\item Pour tout $j \in \llbracket 1, n \rrbracket$, $C_j = \sum_{\substack{i=1 \\ i \neq j}}^n \vert a_{i,j} \vert$ et $C = \max_{j \in \llbracket 1, n \rrbracket} \{ C_j + \vert a_{j,j} \vert \}$.
		\end{itemize}
	\end{notation}
	
	\begin{theorem}[Gerschgörin-Hadamard]
		\label{153-1}
		Soit $\lambda \in \mathbb{C}$ une valeur propre de $A$. Alors, il existe $i \in \llbracket 1, n \rrbracket$ tel que $\vert \lambda - a_{i,i} \vert \leq L_i$.
	\end{theorem}
	
	\reference[FGN2]{189}
	
	\begin{remark}
		Ainsi,
		\[ \operatorname{Sp}(A) \subseteq \bigcup_{i=1}^n \{ z \in \mathbb{C} \mid \vert z - a_{i,i} \vert \leq L_i \} \]
		Les disques de cette réunion sont appelés disques de Gerschgörin.
	\end{remark}
	
	\reference[ROM21]{672}
	
	\begin{example}
		Soient $a, b \in \mathbb{R}^2$. On pose
		\[
			A(a,b) =
		 	\begin{pmatrix}
		 		a & b & 0 & \dots & 0 \\
		 		b & a & b & \ddots & \vdots \\
		 		0 & \ddots & \ddots & \ddots & 0 \\
		 		\vdots & \ddots & b & a & b \\
		 		0 & \dots & 0 & b & a
		 	\end{pmatrix}
		 \]
		 Alors,
		 \[ \operatorname{Sp}(A(a,b)) = \left\{ a + 2b \cos \left( \frac{k \pi}{n+1} \right) \mid k \in \llbracket 1, n \rrbracket \right\} \]
	\end{example}
	
	\begin{example}
		Soit
		\[
			A =
			\begin{pmatrix}
				1 & 0 & \dots & 0 & -1 \\
				-1 & 1 & 0 & \dots & 0 \\
				0 & \ddots & \ddots & \ddots & \vdots \\
				\vdots & \ddots & -1 & 1 & 0 \\
				0 & \dots & 0 & -1 & 1
			\end{pmatrix}
		\]
		Alors,
		\[ \operatorname{Sp}(\tr{A}A) = \left\{ 4 \sin \left( \frac{k \pi}{n} \right)^2 \mid k \in \llbracket 0, n-1 \rrbracket \right\} \]
	\end{example}
	
	\reference{651}
	
	\begin{corollary}
		Pour toute valeur propre $\lambda \in \mathbb{C}$ de $A$, on a
		\[ \lambda \leq \min(L,C) \]
	\end{corollary}
	
	\begin{corollary}
		On suppose $A$ à diagonale strictement dominante (ie. $\forall i \in \llbracket 1, n \rrbracket$, $\vert a_{i,i} \vert > \sum_{\substack{j=1 \\ j \neq i}}^n \vert a_{i,j} \vert$). Alors, $A$ est inversible.
	\end{corollary}
	
	\begin{theorem}[Ostrowski]
		Pour tout $\alpha \in [0,1]$ et toute valeur propre $\lambda \in \mathbb{C}$ de $A$, il existe $i \in \llbracket 1, n \rrbracket$ tel que
		\[ \vert \lambda - a_{i,i} \vert \leq L_i^{\alpha} C_i^{1-\alpha} \]
	\end{theorem}
	
	\begin{remark}
		C'est une généralisation du \cref{153-1} : pour $\alpha = 1$, on retrouve l'énoncé correspondant.
	\end{remark}
	
	\begin{corollary}
		Pour toute valeur propre $\lambda \in \mathbb{C}$ de $A$, il existe $i \in \llbracket 1, n \rrbracket$ tel que
		\[ \vert \lambda \vert^2 \leq (L_i + \vert a_{i,i} \vert) (C_i + \vert a_{i,i} \vert) \]
	\end{corollary}
	
	\subsubsection{Utilisation du rayon spectral}
	
	\begin{notation}
		À toute norme $\vert . \vert$ sur $\mathbb{C}^n$, on associe la norme matricielle
		\[ \VERT . \VERT : M \mapsto \sup_{x \in \mathbb{C}^n \setminus \{ 0 \}} \frac{\Vert Mx \Vert}{\Vert x \Vert} \]
	\end{notation}
	
	\begin{definition}
		Le \textbf{rayon spectral} de $A$, noté $\rho(A)$ est défini par
		\[ \rho(A) = \max_{\lambda \in \operatorname{Sp}(A)} \vert \lambda \vert \]
	\end{definition}
	
	\begin{theorem}
		On a
		\[ \VERT A \VERT_2 = \sqrt{\VERT A^* A \VERT_2} = \sqrt{\rho(A^* A)} \]
		où $\VERT . \VERT_2$ est la norme matricielle associée à la norme euclidienne sur $\mathbb{C}^n$ et $A^*$ est la transconjuguée de $A$.
	\end{theorem}
	
	\begin{theorem}
		\begin{enumerate}[label=(\roman*)]
			\item On a $\rho(A) \leq \VERT A \VERT$ pour toute norme matricielle $\VERT . \VERT$ induite par une norme vectorielle.
			\item $\rho_{\VERT . \VERT \in \mathcal{N}} \VERT A \VERT$ où $\mathcal{N}$ désigne l'ensemble de toutes les normes matricielles induites par une norme vectorielle.
		\end{enumerate}
	\end{theorem}
	
	\reference[GOU21]{203}
	\dev{decomposition-de-dunford}
	
	\begin{theorem}[Décomposition de Dunford]
		Soit $f \in \mathcal{E}$ un endomorphisme tel que son polynôme minimal $\pi_f$ soit scindé sur $\mathbb{K}$. Alors il existe un unique couple d'endomorphismes $(d, n)$ tel que :
		\begin{itemize}
			\item $f = d + n$.
			\item $d$ est diagonalisable et $n$ est nilpotent.
			\item $d \circ n = n \circ d$.
		\end{itemize}
	\end{theorem}
	
	\reference[ROM21]{660}
	
	\begin{corollary}[Théorème de Gelfand]
		Soit $\Vert . \Vert$ une norme sur $\mathcal{M}_n(\mathbb{C})$. Alors,
		\[ \rho(A) = \lim_{k \rightarrow +\infty} \Vert A^k \Vert^{\frac{1}{k}} \]
	\end{corollary}
	
	\subsubsection{Application à l'étude d'une suite de polygones}
	
	\begin{proposition}
		Les conditions suivantes sont équivalentes.
		\begin{enumerate}[label=(\roman*)]
			\item $\lim_{k \rightarrow +\infty} A^k$.
			\item Pour toute valeur initiale $x_0 \in \mathbb{C}^n$, la suite définie par récurrence pour tout $k \in \mathbb{N}$ par $x_{k+1} = A x_k$, converge vers le vecteur nul.
			\item $\rho(A) < 1$.
			\item Il existe au moins une norme matricielle $\VERT . \VERT$ induite par une norme vectorielle telle que $\VERT A \VERT < 1$.
		\end{enumerate}
	\end{proposition}
	
	\reference[GOU21]{153}
	
	\begin{lemma}[Déterminant circulant]
		\label{suite-de-polygones-1}
		Soient $n \in \mathbb{N}^*$ et $a_1, \dots, a_n \in \mathbb{C}$. On pose $\omega = e^{\frac{2i\pi}{n}}$. Alors
		\[ \begin{vmatrix} a_0 & a_1 & \dots & a_{n-1} \\ a_{n-1} & a_0 & \dots & a_{n-2}\\ \vdots & \vdots & \ddots & \vdots \\ a_1 & a_2 & \dots & a_0 \end{vmatrix} = \prod_{j=0}^{n-1} P(\omega^j) \]
		où $P = \sum_{k=0}^{n-1} a_k X^k$.
	\end{lemma}
	
	\reference[I-P]{389}
	\dev{suite-de-polygones}
	
	\begin{application}[Suite de polygones]
		Soit $P_0$ un polygone dont les sommets sont $\{ z_{0,1}, \dots, z_{0,n} \}$. On définit la suite de polygones $(P_k)$ par récurrence en disant que, pour tout $k \in \mathbb{N}^*$, les sommets de $P_{k+1}$ sont les milieux des arêtes de $P_k$.
		\newpar
		Alors la suite $(P_k)$ converge vers l'isobarycentre de $P_0$.
	\end{application}
	
	\subsection{Approximation}
	
	Soit $A = (a_{i,j})_{i, j \in \llbracket 1, n \rrbracket} \in \mathcal{M}_n(\mathbb{R})$.
	
	\reference[ROM19-2]{210}
	
	\begin{theorem}
		On suppose que la valeur propre de $A$ de module maximum est unique. On la note $\lambda_1$. Elle est alors réelle est simple, l'espace propre associé est une droite vectorielle et on a
		\[ \mathbb{R}^n = \ker(A-\lambda_1 I_n) \oplus \im(A-\lambda_1 I_n) \]
	\end{theorem}
	
	On suppose pour la suite que la valeur propre de $A$ de module maximum est unique. On la note $\lambda_1$.
	
	\begin{notation}
		On note et on définit :
		\begin{itemize}
			\item $E_1 = \ker(A-\lambda_1 I_n)$ et $F_1 = \im(A-\lambda_1 I_n)$.
			\item $x_0 = e_1 + f_1$ avec $e_1 \in E_1 \setminus \{ 0 \}$ et $f_1 \in F_1$.
			\item $\forall k \in \mathbb{N}, \, x_{k+1} = \frac{1}{\Vert A x_k \Vert} a_k$ avec $\Vert . \Vert$ norme quelconque sur $\mathbb{R}^n$.
			\item Pour tout $j \in \llbracket 1,n \rrbracket$, on note $e_{1,j}$ la $j$-ième composante du vecteur $e_1$, $x_{k,j}$ celle de $x_k$ et $(Ax_k)_j$ celle de $Ax_k$.
		\end{itemize}
	\end{notation}
	
	\begin{theorem}[Méthode la puissance itérée]
		On a :
		\begin{enumerate}[label=(\roman*)]
			\item $\lim_{k \rightarrow +\infty} \Vert A x_k \Vert = \vert \lambda_1 \vert = \rho(A)$.
			\item $\lim_{k \rightarrow +\infty} x_{2k} = v_1$ où $v_1$ est un vecteur propre non nul associé à la valeur propre $\lambda_1$.
			\item $\lim_{k \rightarrow +\infty} x_{2k+1} = \operatorname{signe}(\lambda_1) v_1$.
			\item Pour tout $j \in \llbracket 1,n \rrbracket$, tel que $e_{1,j} \neq 0$,
			\[ \lim_{k \rightarrow +\infty} \frac{(A x_k)_j}{x_{k,j}} = \lambda_1 \]
		\end{enumerate}
	\end{theorem}
	
	\begin{remark}
		\begin{itemize}
			\item Si $A$ est inversible, la méthode précédente appliquée à $A^{-1}$ permet de calculer la valeur propre de plus petit module de $A$ (quand cette dernière est unique).
			\item En notant $e_1$ un vecteur propre de $A$ associé à la valeur propre $\lambda_1$ de norme euclidienne égale à $1$, les valeurs propres de la matrice $B = A - \lambda_1 e_1 \tr{e_1}$ sont $0, \lambda_2, \dots, \lambda_n$. On pourra alors appliquer la méthode à $B$.
		\end{itemize}
	\end{remark}
	
	\newpage
	\section*{Annexes}
	
	\reference[I-P]{389}
	
	\begin{figure}[h]
		\begin{center}
			\begin{tikzpicture}
				\coordinate (A) at (0:3);
				\coordinate (B) at (72:3);
				\coordinate (C) at (2*72:3);
				\coordinate (D) at (3*72:3);
				\coordinate (E) at (4*72:3);
				\coordinate (F) at (A);
				\foreach \i in {0,...,10} {
					\draw(A) node {$\bullet$};
					\draw(B) node {$\bullet$};
					\draw(C) node {$\bullet$};
					\draw(D) node {$\bullet$};
					\draw(E) node {$\bullet$};
					\draw[fill=cyan!60, fill opacity=0.2](A) -- (B) -- (C) -- (D) -- (E) -- (A);
					\coordinate (A) at ($(A)!0.5!(B)$);
					\coordinate (B) at ($(B)!0.5!(C)$);
					\coordinate (C) at ($(C)!0.5!(D)$);
					\coordinate (D) at ($(D)!0.5!(E)$);
					\coordinate (E) at ($(E)!0.5!(F)$);
					\coordinate (F) at (A);
				}
			\end{tikzpicture}
		\end{center}
		\caption{La suite de polygones.}
	\end{figure}
	%</content>
\end{document}
