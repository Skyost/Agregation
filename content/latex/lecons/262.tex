\documentclass[12pt, a4paper]{report}

% LuaLaTeX :

\RequirePackage{iftex}
\RequireLuaTeX

% Packages :

\usepackage[french]{babel}
%\usepackage[utf8]{inputenc}
%\usepackage[T1]{fontenc}
\usepackage[pdfencoding=auto, pdfauthor={Hugo Delaunay}, pdfsubject={Mathématiques}, pdfcreator={agreg.skyost.eu}]{hyperref}
\usepackage{amsmath}
\usepackage{amsthm}
%\usepackage{amssymb}
\usepackage{stmaryrd}
\usepackage{tikz}
\usepackage{tkz-euclide}
\usepackage{fourier-otf}
\usepackage{fontspec}
\usepackage{titlesec}
\usepackage{fancyhdr}
\usepackage{catchfilebetweentags}
\usepackage[french, capitalise, noabbrev]{cleveref}
\usepackage[fit, breakall]{truncate}
\usepackage[top=2.5cm, right=2cm, bottom=2.5cm, left=2cm]{geometry}
\usepackage{enumerate}
\usepackage{tocloft}
\usepackage{microtype}
%\usepackage{mdframed}
%\usepackage{thmtools}
\usepackage{xcolor}
\usepackage{tabularx}
\usepackage{aligned-overset}
\usepackage[subpreambles=true]{standalone}
\usepackage{environ}
\usepackage[normalem]{ulem}
\usepackage{marginnote}
\usepackage{etoolbox}
\usepackage{setspace}
\usepackage[bibstyle=reading, citestyle=draft]{biblatex}
\usepackage{xpatch}
\usepackage[many, breakable]{tcolorbox}
\usepackage[backgroundcolor=white, bordercolor=white, textsize=small]{todonotes}

% Bibliographie :

\newcommand{\overridebibliographypath}[1]{\providecommand{\bibliographypath}{#1}}
\overridebibliographypath{../bibliography.bib}
\addbibresource{\bibliographypath}
\defbibheading{bibliography}[\bibname]{%
	\newpage
	\section*{#1}%
}
\renewbibmacro*{entryhead:full}{\printfield{labeltitle}}
\DeclareFieldFormat{url}{\newline\footnotesize\url{#1}}
\AtEndDocument{\printbibliography}

% Police :

\setmathfont{Erewhon Math}

% Tikz :

\usetikzlibrary{calc}

% Longueurs :

\setlength{\parindent}{0pt}
\setlength{\headheight}{15pt}
\setlength{\fboxsep}{0pt}
\titlespacing*{\chapter}{0pt}{-20pt}{10pt}
\setlength{\marginparwidth}{1.5cm}
\setstretch{1.1}

% Métadonnées :

\author{agreg.skyost.eu}
\date{\today}

% Titres :

\setcounter{secnumdepth}{3}

\renewcommand{\thechapter}{\Roman{chapter}}
\renewcommand{\thesubsection}{\Roman{subsection}}
\renewcommand{\thesubsubsection}{\arabic{subsubsection}}
\renewcommand{\theparagraph}{\alph{paragraph}}

\titleformat{\chapter}{\huge\bfseries}{\thechapter}{20pt}{\huge\bfseries}
\titleformat*{\section}{\LARGE\bfseries}
\titleformat{\subsection}{\Large\bfseries}{\thesubsection \, - \,}{0pt}{\Large\bfseries}
\titleformat{\subsubsection}{\large\bfseries}{\thesubsubsection. \,}{0pt}{\large\bfseries}
\titleformat{\paragraph}{\bfseries}{\theparagraph. \,}{0pt}{\bfseries}

\setcounter{secnumdepth}{4}

% Table des matières :

\renewcommand{\cftsecleader}{\cftdotfill{\cftdotsep}}
\addtolength{\cftsecnumwidth}{10pt}

% Redéfinition des commandes :

\renewcommand*\thesection{\arabic{section}}
\renewcommand{\ker}{\mathrm{Ker}}

% Nouvelles commandes :

\newcommand{\website}{https://agreg.skyost.eu}

\newcommand{\tr}[1]{\mathstrut ^t #1}
\newcommand{\im}{\mathrm{Im}}
\newcommand{\rang}{\operatorname{rang}}
\newcommand{\trace}{\operatorname{trace}}
\newcommand{\id}{\operatorname{id}}
\newcommand{\stab}{\operatorname{Stab}}

\providecommand{\newpar}{\\[\medskipamount]}

\providecommand{\lesson}[3]{%
	\title{#3}%
	\hypersetup{pdftitle={#3}}%
	\setcounter{section}{\numexpr #2 - 1}%
	\section{#3}%
	\fancyhead[R]{\truncate{0.73\textwidth}{#2 : #3}}%
}

\providecommand{\development}[3]{%
	\title{#3}%
	\hypersetup{pdftitle={#3}}%
	\section*{#3}%
	\fancyhead[R]{\truncate{0.73\textwidth}{#3}}%
}

\providecommand{\summary}[1]{%
	\textit{#1}%
	\medskip%
}

\tikzset{notestyleraw/.append style={inner sep=0pt, rounded corners=0pt, align=center}}

%\newcommand{\booklink}[1]{\website/bibliographie\##1}
\newcommand{\citelink}[2]{\hyperlink{cite.\therefsection @#1}{#2}}
\newcommand{\previousreference}{}
\providecommand{\reference}[2][]{%
	\notblank{#1}{\renewcommand{\previousreference}{#1}}{}%
	\todo[noline]{%
		\protect\vspace{16pt}%
		\protect\par%
		\protect\notblank{#1}{\cite{[\previousreference]}\\}{}%
		\protect\citelink{\previousreference}{p. #2}%
	}%
}

\definecolor{devcolor}{HTML}{00695c}
\newcommand{\dev}[1]{%
	\reversemarginpar%
	\todo[noline]{
		\protect\vspace{16pt}%
		\protect\par%
		\bfseries\color{devcolor}\href{\website/developpements/#1}{DEV}
	}%
	\normalmarginpar%
}

% En-têtes :

\pagestyle{fancy}
\fancyhead[L]{\truncate{0.23\textwidth}{\thepage}}
\fancyfoot[C]{\scriptsize \href{\website}{\texttt{agreg.skyost.eu}}}

% Couleurs :

\definecolor{property}{HTML}{fffde7}
\definecolor{proposition}{HTML}{fff8e1}
\definecolor{lemma}{HTML}{fff3e0}
\definecolor{theorem}{HTML}{fce4f2}
\definecolor{corollary}{HTML}{ffebee}
\definecolor{definition}{HTML}{ede7f6}
\definecolor{notation}{HTML}{f3e5f5}
\definecolor{example}{HTML}{e0f7fa}
\definecolor{cexample}{HTML}{efebe9}
\definecolor{application}{HTML}{e0f2f1}
\definecolor{remark}{HTML}{e8f5e9}
\definecolor{proof}{HTML}{e1f5fe}

% Théorèmes :

\theoremstyle{definition}
\newtheorem{theorem}{Théorème}

\newtheorem{property}[theorem]{Propriété}
\newtheorem{proposition}[theorem]{Proposition}
\newtheorem{lemma}[theorem]{Lemme}
\newtheorem{corollary}[theorem]{Corollaire}

\newtheorem{definition}[theorem]{Définition}
\newtheorem{notation}[theorem]{Notation}

\newtheorem{example}[theorem]{Exemple}
\newtheorem{cexample}[theorem]{Contre-exemple}
\newtheorem{application}[theorem]{Application}

\theoremstyle{remark}
\newtheorem{remark}[theorem]{Remarque}

\counterwithin*{theorem}{section}

\newcommand{\applystyletotheorem}[1]{
	\tcolorboxenvironment{#1}{
		enhanced,
		breakable,
		colback=#1!98!white,
		boxrule=0pt,
		boxsep=0pt,
		left=8pt,
		right=8pt,
		top=8pt,
		bottom=8pt,
		sharp corners,
		after=\par,
	}
}

\applystyletotheorem{property}
\applystyletotheorem{proposition}
\applystyletotheorem{lemma}
\applystyletotheorem{theorem}
\applystyletotheorem{corollary}
\applystyletotheorem{definition}
\applystyletotheorem{notation}
\applystyletotheorem{example}
\applystyletotheorem{cexample}
\applystyletotheorem{application}
\applystyletotheorem{remark}
\applystyletotheorem{proof}

% Environnements :

\NewEnviron{whitetabularx}[1]{%
	\renewcommand{\arraystretch}{2.5}
	\colorbox{white}{%
		\begin{tabularx}{\textwidth}{#1}%
			\BODY%
		\end{tabularx}%
	}%
}

% Maths :

\DeclareFontEncoding{FMS}{}{}
\DeclareFontSubstitution{FMS}{futm}{m}{n}
\DeclareFontEncoding{FMX}{}{}
\DeclareFontSubstitution{FMX}{futm}{m}{n}
\DeclareSymbolFont{fouriersymbols}{FMS}{futm}{m}{n}
\DeclareSymbolFont{fourierlargesymbols}{FMX}{futm}{m}{n}
\DeclareMathDelimiter{\VERT}{\mathord}{fouriersymbols}{152}{fourierlargesymbols}{147}


% Bibliographie :

\addbibresource{\bibliographypath}%
\defbibheading{bibliography}[\bibname]{%
	\newpage
	\section*{#1}%
}
\renewbibmacro*{entryhead:full}{\printfield{labeltitle}}%
\DeclareFieldFormat{url}{\newline\footnotesize\url{#1}}%

\AtEndDocument{\printbibliography}

\begin{document}
	%<*content>
	\lesson{analysis}{262}{Convergences d'une suite de variables aléatoires. Théorèmes limite. Exemples et applications.}

	Soient $(\Omega, \mathcal{A}, \mathbb{P})$ un espace probabilisé et $(X_n)$ une suite de vecteurs aléatoires à valeurs dans $(\mathbb{R}^d, \mathcal{B}(\mathbb{R}^d))$.

	\subsection{Premiers modes de convergence}

	\subsubsection{Convergence presque sûre}

	\reference[G-K]{265}

	\begin{definition}
		On dit que $(X_n)$ \textbf{converge presque sûrement} vers $X : \Omega \rightarrow \mathbb{R}^d$ si
		\[ \mathbb{P}(\{ \omega \in \Omega \mid X_n(\omega) \longrightarrow_{n \rightarrow +\infty} X(\omega) \}) = 1 \]
		On note cela $X_n \overset{(ps.)}{\longrightarrow} X$.
	\end{definition}

	\begin{remark}
		La convergence presque sûre d'une suite de vecteurs aléatoires équivaut à la convergence presque sûre de chacune des composantes. Pour cette raison, on peut se limiter à l'étude du cas $d = 1$.
	\end{remark}

	\reference{285}

	\begin{example}
		Si $(X_n)$ est telle que $\forall n \geq 1, \, \mathbb{P}(X_n = \pm \sqrt{n}) = \frac{1}{2}$, alors $\frac{1}{n^2} \sum_{k=1}^n X_k^2 \overset{(ps.)}{\longrightarrow} 0$.
	\end{example}

	\reference{265}

	\begin{proposition}
		Si $X_n \overset{(ps.)}{\longrightarrow} X$ et $Y_n \overset{(ps.)}{\longrightarrow} Y$, alors :
		\begin{enumerate}[(i)]
			\item $\forall a \in \mathbb{R}, \, aX_n \overset{(ps.)}{\longrightarrow} aX$.
			\item $X_n + Y_n \overset{(ps.)}{\longrightarrow} X + Y$.
			\item $X_nY_n \overset{(ps.)}{\longrightarrow} XY$.
		\end{enumerate}
		Plus généralement, si $\forall n \in \mathbb{N}$, $X_n$ et $X$ sont à valeurs dans $E$, alors $f(X_n) \overset{(ps.)}{\longrightarrow} f(X)$ pour toute $f$ fonction définie et continue sur $E$.
	\end{proposition}

	\reference{272}

	\begin{theorem}[1\ier{} lemme de Borel-Cantelli]
		Soit $(A_n)$ une suite d'événements. Si $\sum \mathbb{P}(A_n)$ converge, alors
		\[ \mathbb{P} \left( \limsup_{n \rightarrow +\infty} A_n \right) = 0 \]
	\end{theorem}

	\begin{remark}
		Cela signifie que presque sûrement, seul un nombre fini d'événements $A_n$ se réalisent.
	\end{remark}

	\begin{corollary}
		Si $\sum \mathbb{P}(\vert X_n - X \vert > \epsilon)$ converge pour tout $\epsilon > 0$, alors $X_n \overset{(ps.)}{\longrightarrow} X$.
	\end{corollary}

	\reference{285}

	\begin{example}
		Si $(X_n)$ est telle que $\forall n \geq 1$, $\mathbb{P}(X_n = n) = \mathbb{P}(X_n = \pm n) = \frac{1}{2n^2}$ et $\mathbb{P}(X_n = 0) = 1 - \frac{1}{2n^2}$, alors la suite $(S_n)$ définie pour tout $n \geq 1$ par $S_n = \sum_{k=1}^n X_k$ est constante à partir d'un certain rang.
	\end{example}

	\reference{273}

	\begin{theorem}[2\ieme{} lemme de Borel-Cantelli]
		Soit $(A_n)$ une suite d'événements indépendants. Si $\sum \mathbb{P}(A_n)$ diverge, alors
		\[ \mathbb{P} \left( \limsup_{n \rightarrow +\infty} A_n \right) = 1 \]
	\end{theorem}

	\begin{remark}
		Cela signifie que presque sûrement, un nombre infini d'événements $A_n$ se réalisent.
	\end{remark}

	\reference{286}

	\begin{example}
		On fait une infinité de lancers d'une pièce de monnaie équilibrée. Alors, la probabilité de l'événement ``on obtient une infinité de fois deux ``Face'' consécutifs'' est $1$.
	\end{example}

	\begin{corollary}[Loi du $0$-$1$ de Borel]
		Soit $(A_n)$ une suite d'événements indépendants, alors
		\[ \mathbb{P} \left( \limsup_{n \rightarrow +\infty} A_n \right) = 0 \text{ ou } 1 \]
		et elle vaut $1$ si et seulement si $\sum \mathbb{P}(A_n)$ diverge.
	\end{corollary}

	\subsubsection{Convergence en probabilité}

	\reference{268}

	\begin{definition}
		On dit que $(X_n)$ \textbf{converge en probabilité} vers $X : \Omega \rightarrow \mathbb{R}^d$ si
		\[ \forall \epsilon, \, \mathbb{P}(\vert X_n - X \vert \geq \epsilon) = 0 \]
		On note cela $X_n \overset{(p)}{\longrightarrow} X$.
	\end{definition}

	\reference{285}

	\begin{example}
		\label{262-1}
		On suppose que $(X_n)$ est une suite de variables aléatoires indépendantes identiquement distribuées telle que $\mathbb{P}(X_1 = 1) = p$ et $\mathbb{P}(X_1 = 0) = 1-p$. On définit la suite $(Y_n)$ par
		\[
		\forall n \geq 1, \, Y_n = \begin{cases}
			0 \text{ si } X_k = X_{k+1} \\
			1 \text{ sinon}
		\end{cases}
		\]
		et la suite $(S_n)$ par $\forall n \geq 1, \, M_n = \frac{Y_1 + \dots + Y_n}{n}$. On a $M_n - 2p(1-p) \overset{(p)}{\longrightarrow} 0$.
	\end{example}

	\reference{268}

	\begin{proposition}
		Si $X_n \overset{(p)}{\longrightarrow} X$ et $Y_n \overset{(p)}{\longrightarrow} Y$, alors :
		\begin{enumerate}[(i)]
			\item $(X_n, Y_n) \overset{(p)}{\longrightarrow} (X, Y)$.
			\item $X_n + Y_n \overset{(p)}{\longrightarrow} X + Y$.
		\end{enumerate}
	\end{proposition}

	\begin{theorem}
		La convergence presque sûre implique la convergence en probabilité.
	\end{theorem}

	\reference{285}

	\begin{cexample}
		La suite $(M_n - 2p(1-p))$ de l'\cref{262-1} ne converge pas vers $0$ presque sûrement.
	\end{cexample}

	\reference{274}

	\begin{theorem}
		Si $X_n \overset{(p)}{\longrightarrow} X$, alors il existe une sous-suite $(X_{n_k})$ de $(X_n)$ telle que $X_{n_k} \overset{(ps.)}{\longrightarrow} X$.
	\end{theorem}

	\begin{corollary}
		On suppose $X_n \overset{(p)}{\longrightarrow} X$. Si $\forall n \in \mathbb{N}$, $X_n$ et $X$ sont à valeurs dans $E$, alors $f(X_n) \overset{(p)}{\longrightarrow} f(X)$ pour toute $f$ fonction définie et continue sur $E$.
	\end{corollary}

	\subsubsection{Lois des grands nombres}

	\reference{270}

	\begin{theorem}[Loi faible des grands nombres]
		Soit $(X_n)$ une suite de variables aléatoires deux à deux indépendantes de même loi et $\mathcal{L}_1$. On pose $M_n = \frac{X_1 + \dots + X_n}{n}$. Alors,
		\[ M_n \overset{(p)}{\longrightarrow} \mathbb{E}(X_1) \]
	\end{theorem}

	\reference[Z-Q]{526}

	\begin{theorem}[Loi forte des grands nombres]
		Soit $(X_n)$ une suite de variables aléatoires mutuellement indépendantes de même loi. On pose $M_n = \frac{X_1 + \dots + X_n}{n}$. Alors,
		\[ X_1 \in \mathcal{L}_1 \iff M_n \overset{(ps.)}{\longrightarrow} \ell \in \mathbb{R} \]
		Dans ce cas, on a $\ell = \mathbb{E}(X_1)$.
	\end{theorem}

	\reference[G-K]{195}

	\begin{application}[Bernstein]
		Soit $f : [0,1] \rightarrow \mathbb{R}$ continue. On note
		\[ \forall n \in \mathbb{N}^*, \, B_n(f) : x \mapsto \sum_{k=0}^n \binom{n}{k} f \left( \frac{k}{n} \right) x^k (1-x)^{n-k} \]
		le $n$-ième polynôme de Bernstein associé à $f$. Alors le suite de fonctions $(B_n(f))$ converge uniformément vers $f$.
	\end{application}

	\begin{corollary}[Weierstrass]
		Toute fonction continue $f : [a,b] \rightarrow \mathbb{R}$ (avec $a, b \in \mathbb{R}$ tels que $a \leq b$) est limite uniforme de fonctions polynômiales sur $[a, b]$.
	\end{corollary}

	\subsection{Convergence \texorpdfstring{$L_p$}{Lp}}
	
	\reference{268}
	
	\begin{definition}
		On dit que $(X_n)$ \textbf{converge dans $L_p$} vers $X : \Omega \rightarrow \mathbb{R}^d$ si
		\[ \forall n \in \mathbb{N}, X_n \in L_p, \, X \in L_p \text{ et } \mathbb{E}(\vert X_n - X \vert^p) \]
		On note cela $X_n \overset{(L_p)}{\longrightarrow} X$.
	\end{definition}
	
	\reference[D-L]{510}
	
	\begin{proposition}
		Comme les espaces sont de mesure finie,
		\[ p \geq q \implies L_p(\Omega, \mathcal{A}, \mathbb{P}) \subseteq L_q(\Omega, \mathcal{A}, \mathbb{P}) \]
	\end{proposition}
	
	\begin{corollary}
		Pour $1 \leq p \leq q$, la convergence dans $L_q$ implique la convergence dans $L_p$ qui implique elle-même la convergence dans $L_1$.
	\end{corollary}
	
	\reference[HAU]{365}
	
	\begin{cexample}
		Si,
		\[
			\forall n \in \mathbb{N}, \, \forall \omega \in \Omega, \, X_n(\omega) = \sqrt{n} \mathbb{1}_{\left[ 0, \frac{1}{n} \right]}
		\]
		alors, $(X_n)$ converge dans $L_1$ mais pas dans $L_2$.
	\end{cexample}
	
	\reference[G-K]{65}
	
	\begin{theorem}[Convergence dominée]
		Si $X_n \overset{(ps.)}{\longrightarrow} X$ et $\exists g \in L_1$ telle que $\Vert X_n \Vert_1 \leq g$, alors $X_n \overset{(L_1)}{\longrightarrow} X$.
	\end{theorem}
	
	\reference[HAU]{365}
	
	\begin{cexample}
		On se place dans le cas où $(\Omega, \mathcal{A}, \mathbb{P}) = ([0,1[, \mathcal{B}([0,1[), \lambda_{[0,1[})$. Si $\forall n \geq 1, \, X_n = n \mathbb{1}_{\left] 0, \frac{1}{n} \right[}$, alors $(X_n)$ converge vers $0$ presque sûrement, mais pas dans $L_1$.
	\end{cexample}
	
	\reference[G-K]{265}
	
	\begin{proposition}
		Si $X_n \overset{(L_p)}{\longrightarrow} X$, alors il existe une sous-suite $(X_{n_k})$ de $(X_n)$ telle que $X_{n_k} \overset{(ps.)}{\longrightarrow} X$.
	\end{proposition}
	
	\begin{theorem}
		La convergence dans $L_p$ (pour $p \geq 1$) implique la convergence en probabilité.
	\end{theorem}
	
	\begin{example}
		La convergence en probabilité de l'\cref{262-1} est en fait une convergence dans $L_2$.
	\end{example}
	
	\reference{281}
	
	\begin{cexample}
		Soit $X$ une variable aléatoire de densité $f : x \mapsto e^{-x} \mathbb{1}_{\mathbb{R}^+}$. On pose $\forall n \geq 1, \, Y_n = X \mathbb{1}_{[0,n[}(X) + e^{2n} \mathbb{1}_{[n,+\infty[}(X)$. Alors $(Y_n)$ converge vers $X$ en probabilité, mais pas dans $L_1$.
	\end{cexample}

	\subsection{Convergence en loi}

	\subsubsection{Définition et premières propriétés}
	
	\reference{295}
	
	\begin{definition}
		On dit que $(X_n)$ \textbf{converge en loi} vers $X : \Omega \rightarrow \mathbb{R}^d$ si
		\[ \forall f \in \mathcal{C}_b(\mathbb{R}^d, \mathbb{R}), \, \mathbb{E}(f(X_n)) \longrightarrow_{n \rightarrow +\infty} \mathbb{E}(f(X)) \]
		On note cela $X_n \overset{(d)}{\longrightarrow} X$.
	\end{definition}
	
	\reference{313}
	
	\begin{example}
		Si $\forall n \geq 1$, $X_n$ suit une loi uniforme sur $\llbracket 1, n-1 \rrbracket$, alors $\frac{X_n}{n}$ converge en loi vers la loi uniforme sur $[0,1]$.
	\end{example}
	
	\reference{295}
	
	\begin{proposition}
		Si $X_n \overset{(d)}{\longrightarrow} X$ et $Y_n \overset{(d)}{\longrightarrow} Y$, alors :
		\begin{enumerate}[(i)]
			\item La limite $X$ est unique.
			\item $\langle X_n, Y_n \rangle \overset{(d)}{\longrightarrow} \langle X, Y \rangle$.
		\end{enumerate}
		Plus généralement, si $\forall n \in \mathbb{N}$, $X_n$ et $X$ sont à valeurs dans $E$, alors $f(X_n) \overset{(d)}{\longrightarrow} f(X)$ pour toute $f$ fonction définie et continue sur $E$.
	\end{proposition}
	
	\begin{theorem}[Lemme de Scheffé]
		On suppose :
		\begin{itemize}
			\item $X_n \overset{(ps.)}{\longrightarrow} X$.
			\item $\lim_{n \rightarrow +\infty} \int_\Omega X_n \, \mathrm{d}\mathbb{P} = \int_\Omega X \, \mathrm{d}\mathbb{P}$.
		\end{itemize}
		Alors, $X_n \overset{(L_1)}{\longrightarrow} X$.
	\end{theorem}
	
	\begin{corollary}
		On suppose :
		\begin{itemize}
			\item $\forall n \in \mathbb{N}$, $X_n$ admet une densité $f_n$.
			\item $(f_n)$ converge presque partout vers une fonction $f$.
			\item Il existe une variable aléatoire $X$ admettant $f$ pour densité.
		\end{itemize}
		Alors, $X_n \overset{(d)}{\longrightarrow} X$.
	\end{corollary}
	
	\begin{corollary}
		Si $X$ et $X_n$ sont des variables aléatoires à valeurs dans un ensemble dénombrable $D$ pour tout $n \in \mathbb{N}$, en supposant
		\[ \forall k \in D, \, \mathbb{P}(X_n = k) = \mathbb{P}(X = k) \]
		alors $X_n \overset{(d)}{\longrightarrow} X$.
	\end{corollary}
	
	\begin{application}
		Soit, pour $n \geq 1$, une variable aléatoire $X_n$ suivant la loi binomiale de paramètres $n$ et $p_n$. On suppose que $\lim_{n \rightarrow +\infty} n p_n = \lambda > 0$.
		Alors,
		\[ X_n \overset{(d)}{\longrightarrow} X \]
		où $X$ suit une loi de Poisson de paramètre $\lambda$.
	\end{application}
	
	\reference{302}
	
	\begin{theorem}
		En notant $F_X$ la fonction de répartition d'une variable aléatoire $X$, on a,
		\[ X_n \overset{(d)}{\longrightarrow} X \iff F_{X_n}(x) \longrightarrow_{n \rightarrow +\infty} F_X(x) \]
		en tout point $x$ où $F_X$ est continue.
	\end{theorem}
	
	\begin{theorem}
		Soit $X : \Omega \rightarrow \mathbb{R}^d$ une variable aléatoire.
		\begin{enumerate}[(i)]
			\item Si $(X_n)$ converge en probabilité vers $X$, alors $(X_n)$ converge en loi vers $X$.
			\item Si $(X_n)$ converge en loi vers une constante $a$ (ou de manière équivalente, vers une masse de Dirac $\delta_a$), alors $(X_n)$ converge en probabilité vers $a$.
		\end{enumerate}
	\end{theorem}
	
	\reference[HAU]{362}
	
	\begin{cexample}
		Si $(X_n)$ est une suite de variables aléatoires indépendantes identiquement distribuées de loi $\mathcal{B}(p)$, alors $(X_n)$ converge en loi vers $\mathcal{B}(2p(1-p))$, mais pas en probabilité.
	\end{cexample}

	\subsubsection{Théorème central limite et applications}
	
	\reference[G-K]{305}
	
	\begin{theorem}[Slutsky]
		Si $X_n \overset{(d)}{\longrightarrow} X$ et $Y_n \overset{(d)}{\longrightarrow} c$ où $c$ est un vecteur constant, alors :
		\begin{enumerate}[(i)]
			\item $X_n + Y_n \overset{(d)}{\longrightarrow} X + c$.
			\item $\langle X_n, Y_n \rangle \overset{(d)}{\longrightarrow} \langle X, c \rangle$.
		\end{enumerate}
	\end{theorem}
	
	\begin{notation}
		Si $X$ est une variable aléatoire réelle, on note $\phi_X$ sa fonction caractéristique.
	\end{notation}
	
	\reference[Z-Q]{536}
	
	\begin{theorem}[Lévy]
		On suppose que $(X_n)$ est une suite de variables aléatoires réelles et $X$ une variable aléatoire réelle. Alors :
		\[ X_n \overset{(d)}{\longrightarrow} X \iff \phi_{X_n} \text{ converge simplement vers } \phi_X \]
	\end{theorem}
	
	\reference[G-K]{307}
	\dev{theoreme-central-limite}
	
	\begin{theorem}[Central limite]
		On suppose que $(X_n)$ est une suite de variables aléatoires réelles indépendantes de même loi admettant un moment d'ordre $2$. On note $m$ l'espérance et $\sigma^2$ la variance commune à ces variables. On pose $S_n = X_1 + \dots + X_n - nm$. Alors,
		\[ \left ( \frac{S_n}{\sqrt{n}} \right) \overset{(d)}{\longrightarrow} \mathcal{N}(0, \sigma^2) \]
	\end{theorem}
	
	\begin{application}[Théorème de Moivre-Laplace]
		On suppose que $(X_n)$ est une suite de variables aléatoires indépendantes de même loi $\mathcal{B}(p)$. Alors,
		\[ \frac{\sum_{k=1}^{n} X_k - np}{\sqrt{n}} \overset{(d)}{\longrightarrow} \mathcal{N}(0, p(1-p)) \]
	\end{application}
	
	\reference{180}
	\dev{formule-de-stirling}
	
	\begin{lemma}
		Soient $X$ et $Y$ deux variables aléatoires indépendantes telles que $X \sim \Gamma(a, \gamma)$ et $Y \sim \Gamma(b, \gamma)$. Alors $Z = X + Y \sim \Gamma(a+b, \gamma)$.
	\end{lemma}
	
	\begin{application}[Formule de Stirling]
		\[ n! \sim \sqrt{2n\pi} \left(\frac{n}{e} \right)^n \]
	\end{application}
	
	\subsection*{Annexes}
	
	\begin{figure}[h]
		\begin{center}
			\begin{tikzpicture}
				\node (A) at (0,0) {Convergence presque sûre};
				\node (B) at ($(A.east)+(4,0)$) {Convergence en probabilité};
				\node (C) at ($(B.south)+(0,-1)$) {Convergence $L_1$};
				\node (D) at ($(B.east)+(4,0)$) {Convergence en loi};
				\node (E) at ($(C.south)+(0,-1)$) {Convergence $L_p$};
				\draw[-implies,double equal sign distance] (A) -- (B);
				\draw[-implies,double equal sign distance] (C) -- (B);
				\draw[-implies,double equal sign distance] (B) -- (D);
				\draw[-implies,double equal sign distance] (E) -- (C);
				\draw[-implies,double equal sign distance] (B.north west) to [out=150,in=30] (A.north east);
				\node at ($(A.north east)!0.5!(B.north west)+(0,0.3)$) [above] {\tiny Sous-suite};
				\draw[-implies,double equal sign distance] (D.north west) to [out=150,in=30] (B.north east);
				\node at ($(B.north east)!0.5!(D.north west)+(0,0.3)$) [above] {\tiny Limite constante};
				\draw[-implies,double equal sign distance] (A.south east) to [out=-30,in=-180] (C.west);
				\node at ($(A.south east)!0.5!(C.west)+(0,-0.25)$) [below left] {\tiny Domination};
			\end{tikzpicture}
		\end{center}
		\caption{Liens entre les différents modes de convergence}
	\end{figure}
	%</content>
\end{document}
