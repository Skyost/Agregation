\documentclass[12pt, a4paper]{report}

% LuaLaTeX :

\RequirePackage{iftex}
\RequireLuaTeX

% Packages :

\usepackage[french]{babel}
%\usepackage[utf8]{inputenc}
%\usepackage[T1]{fontenc}
\usepackage[pdfencoding=auto, pdfauthor={Hugo Delaunay}, pdfsubject={Mathématiques}, pdfcreator={agreg.skyost.eu}]{hyperref}
\usepackage{amsmath}
\usepackage{amsthm}
%\usepackage{amssymb}
\usepackage{stmaryrd}
\usepackage{tikz}
\usepackage{tkz-euclide}
\usepackage{fourier-otf}
\usepackage{fontspec}
\usepackage{titlesec}
\usepackage{fancyhdr}
\usepackage{catchfilebetweentags}
\usepackage[french, capitalise, noabbrev]{cleveref}
\usepackage[fit, breakall]{truncate}
\usepackage[top=2.5cm, right=2cm, bottom=2.5cm, left=2cm]{geometry}
\usepackage{enumerate}
\usepackage{tocloft}
\usepackage{microtype}
%\usepackage{mdframed}
%\usepackage{thmtools}
\usepackage{xcolor}
\usepackage{tabularx}
\usepackage{aligned-overset}
\usepackage[subpreambles=true]{standalone}
\usepackage{environ}
\usepackage[normalem]{ulem}
\usepackage{marginnote}
\usepackage{etoolbox}
\usepackage{setspace}
\usepackage[bibstyle=reading, citestyle=draft]{biblatex}
\usepackage{xpatch}
\usepackage[many, breakable]{tcolorbox}
\usepackage[backgroundcolor=white, bordercolor=white, textsize=small]{todonotes}

% Bibliographie :

\newcommand{\overridebibliographypath}[1]{\providecommand{\bibliographypath}{#1}}
\overridebibliographypath{../bibliography.bib}
\addbibresource{\bibliographypath}
\defbibheading{bibliography}[\bibname]{%
	\newpage
	\section*{#1}%
}
\renewbibmacro*{entryhead:full}{\printfield{labeltitle}}
\DeclareFieldFormat{url}{\newline\footnotesize\url{#1}}
\AtEndDocument{\printbibliography}

% Police :

\setmathfont{Erewhon Math}

% Tikz :

\usetikzlibrary{calc}

% Longueurs :

\setlength{\parindent}{0pt}
\setlength{\headheight}{15pt}
\setlength{\fboxsep}{0pt}
\titlespacing*{\chapter}{0pt}{-20pt}{10pt}
\setlength{\marginparwidth}{1.5cm}
\setstretch{1.1}

% Métadonnées :

\author{agreg.skyost.eu}
\date{\today}

% Titres :

\setcounter{secnumdepth}{3}

\renewcommand{\thechapter}{\Roman{chapter}}
\renewcommand{\thesubsection}{\Roman{subsection}}
\renewcommand{\thesubsubsection}{\arabic{subsubsection}}
\renewcommand{\theparagraph}{\alph{paragraph}}

\titleformat{\chapter}{\huge\bfseries}{\thechapter}{20pt}{\huge\bfseries}
\titleformat*{\section}{\LARGE\bfseries}
\titleformat{\subsection}{\Large\bfseries}{\thesubsection \, - \,}{0pt}{\Large\bfseries}
\titleformat{\subsubsection}{\large\bfseries}{\thesubsubsection. \,}{0pt}{\large\bfseries}
\titleformat{\paragraph}{\bfseries}{\theparagraph. \,}{0pt}{\bfseries}

\setcounter{secnumdepth}{4}

% Table des matières :

\renewcommand{\cftsecleader}{\cftdotfill{\cftdotsep}}
\addtolength{\cftsecnumwidth}{10pt}

% Redéfinition des commandes :

\renewcommand*\thesection{\arabic{section}}
\renewcommand{\ker}{\mathrm{Ker}}

% Nouvelles commandes :

\newcommand{\website}{https://agreg.skyost.eu}

\newcommand{\tr}[1]{\mathstrut ^t #1}
\newcommand{\im}{\mathrm{Im}}
\newcommand{\rang}{\operatorname{rang}}
\newcommand{\trace}{\operatorname{trace}}
\newcommand{\id}{\operatorname{id}}
\newcommand{\stab}{\operatorname{Stab}}

\providecommand{\newpar}{\\[\medskipamount]}

\providecommand{\lesson}[3]{%
	\title{#3}%
	\hypersetup{pdftitle={#3}}%
	\setcounter{section}{\numexpr #2 - 1}%
	\section{#3}%
	\fancyhead[R]{\truncate{0.73\textwidth}{#2 : #3}}%
}

\providecommand{\development}[3]{%
	\title{#3}%
	\hypersetup{pdftitle={#3}}%
	\section*{#3}%
	\fancyhead[R]{\truncate{0.73\textwidth}{#3}}%
}

\providecommand{\summary}[1]{%
	\textit{#1}%
	\medskip%
}

\tikzset{notestyleraw/.append style={inner sep=0pt, rounded corners=0pt, align=center}}

%\newcommand{\booklink}[1]{\website/bibliographie\##1}
\newcommand{\citelink}[2]{\hyperlink{cite.\therefsection @#1}{#2}}
\newcommand{\previousreference}{}
\providecommand{\reference}[2][]{%
	\notblank{#1}{\renewcommand{\previousreference}{#1}}{}%
	\todo[noline]{%
		\protect\vspace{16pt}%
		\protect\par%
		\protect\notblank{#1}{\cite{[\previousreference]}\\}{}%
		\protect\citelink{\previousreference}{p. #2}%
	}%
}

\definecolor{devcolor}{HTML}{00695c}
\newcommand{\dev}[1]{%
	\reversemarginpar%
	\todo[noline]{
		\protect\vspace{16pt}%
		\protect\par%
		\bfseries\color{devcolor}\href{\website/developpements/#1}{DEV}
	}%
	\normalmarginpar%
}

% En-têtes :

\pagestyle{fancy}
\fancyhead[L]{\truncate{0.23\textwidth}{\thepage}}
\fancyfoot[C]{\scriptsize \href{\website}{\texttt{agreg.skyost.eu}}}

% Couleurs :

\definecolor{property}{HTML}{fffde7}
\definecolor{proposition}{HTML}{fff8e1}
\definecolor{lemma}{HTML}{fff3e0}
\definecolor{theorem}{HTML}{fce4f2}
\definecolor{corollary}{HTML}{ffebee}
\definecolor{definition}{HTML}{ede7f6}
\definecolor{notation}{HTML}{f3e5f5}
\definecolor{example}{HTML}{e0f7fa}
\definecolor{cexample}{HTML}{efebe9}
\definecolor{application}{HTML}{e0f2f1}
\definecolor{remark}{HTML}{e8f5e9}
\definecolor{proof}{HTML}{e1f5fe}

% Théorèmes :

\theoremstyle{definition}
\newtheorem{theorem}{Théorème}

\newtheorem{property}[theorem]{Propriété}
\newtheorem{proposition}[theorem]{Proposition}
\newtheorem{lemma}[theorem]{Lemme}
\newtheorem{corollary}[theorem]{Corollaire}

\newtheorem{definition}[theorem]{Définition}
\newtheorem{notation}[theorem]{Notation}

\newtheorem{example}[theorem]{Exemple}
\newtheorem{cexample}[theorem]{Contre-exemple}
\newtheorem{application}[theorem]{Application}

\theoremstyle{remark}
\newtheorem{remark}[theorem]{Remarque}

\counterwithin*{theorem}{section}

\newcommand{\applystyletotheorem}[1]{
	\tcolorboxenvironment{#1}{
		enhanced,
		breakable,
		colback=#1!98!white,
		boxrule=0pt,
		boxsep=0pt,
		left=8pt,
		right=8pt,
		top=8pt,
		bottom=8pt,
		sharp corners,
		after=\par,
	}
}

\applystyletotheorem{property}
\applystyletotheorem{proposition}
\applystyletotheorem{lemma}
\applystyletotheorem{theorem}
\applystyletotheorem{corollary}
\applystyletotheorem{definition}
\applystyletotheorem{notation}
\applystyletotheorem{example}
\applystyletotheorem{cexample}
\applystyletotheorem{application}
\applystyletotheorem{remark}
\applystyletotheorem{proof}

% Environnements :

\NewEnviron{whitetabularx}[1]{%
	\renewcommand{\arraystretch}{2.5}
	\colorbox{white}{%
		\begin{tabularx}{\textwidth}{#1}%
			\BODY%
		\end{tabularx}%
	}%
}

% Maths :

\DeclareFontEncoding{FMS}{}{}
\DeclareFontSubstitution{FMS}{futm}{m}{n}
\DeclareFontEncoding{FMX}{}{}
\DeclareFontSubstitution{FMX}{futm}{m}{n}
\DeclareSymbolFont{fouriersymbols}{FMS}{futm}{m}{n}
\DeclareSymbolFont{fourierlargesymbols}{FMX}{futm}{m}{n}
\DeclareMathDelimiter{\VERT}{\mathord}{fouriersymbols}{152}{fourierlargesymbols}{147}


% Bibliographie :

\addbibresource{\bibliographypath}%
\defbibheading{bibliography}[\bibname]{%
	\newpage
	\section*{#1}%
}
\renewbibmacro*{entryhead:full}{\printfield{labeltitle}}%
\DeclareFieldFormat{url}{\newline\footnotesize\url{#1}}%

\AtEndDocument{\printbibliography}

\begin{document}
  %<*content>
  \lesson{algebra}{154}{Exemples de décompositions de matrices. Applications.}
  
  Soit $\mathbb{K} = \mathbb{R}$ ou $\mathbb{C}$. Soit $n \geq 1$.

  \subsection{Décomposition et réduction}
  
  \subsubsection{Décomposition de Dunford}
  
  \paragraph{Décomposition ``classique''}
  
  \reference[GOU21]{203}
  \dev{decomposition-de-dunford}
  
  \begin{theorem}[Décomposition de Dunford]
    Soit $A \in \mathcal{M}_n(\mathbb{K})$. On suppose que $\pi_A$ est scindé sur $\mathbb{K}$. Alors il existe un unique couple de matrices $(D, N)$ tels que :
    \begin{itemize}
      \item $D$ est diagonalisable et $N$ est nilpotente.
      \item $A = D + N$.
      \item $DN = ND$.
    \end{itemize}
  \end{theorem}
  
  \begin{corollary}
    Si $A$ vérifie les hypothèse précédentes, pour tout $k \in \mathbb{N}$, $A^k = (D + N)^k = \sum_{i=0}^m \binom{k}{i} D^i N^{k-i}$, avec $m = \min(k, l)$ où $l$ désigne l'indice de nilpotence de $N$.
  \end{corollary}
  
  \begin{remark}
    On peut montrer de plus que $D$ et $N$ sont des polynômes en $A$.
  \end{remark}
  
  \reference[C-G]{165}
  
  \begin{example}
    On a la décomposition de Dunford suivante :
    \[ \begin{pmatrix} 1 & 1 \\ 0 & 1 \end{pmatrix} = \begin{pmatrix} 1 & 0 \\ 0 & 1 \end{pmatrix} + \begin{pmatrix} 0 & 1 \\ 0 & 0 \end{pmatrix} \]
  \end{example}
  
  \begin{cexample}
    L'égalité suivante n'est pas une décomposition de Dunford :
    \[ \begin{pmatrix} 1 & 1 \\ 0 & 2 \end{pmatrix} = \begin{pmatrix} 1 & 0 \\ 0 & 2 \end{pmatrix} + \begin{pmatrix} 0 & 1 \\ 0 & 0 \end{pmatrix} \]
    car les deux matrices du membre de droite ne commutent pas.
  \end{cexample}
  
  \reference[ROM21]{761}
  
  \begin{lemma}
    \begin{enumerate}[label=(\roman*)]
      \item La série entière $\sum \frac{z^k}{k!}$ a un rayon de convergence infini.
      \item $\sum \frac{A^k}{k!}$ est convergente pour toute matrice $A \in \mathcal{M}_n(\mathbb{K})$.
    \end{enumerate}
  \end{lemma}
  
  \begin{definition}
    Soit $A \in \mathcal{M}_n(\mathbb{K})$. On définit \textbf{l'exponentielle} de $A$ par
    \[ \sum_{k=0}^{+\infty} \frac{A^k}{k!} \]
    on la note aussi $\exp(A)$ ou $e^A$.
  \end{definition}
  
  \begin{theorem}
    Soit $A \in \mathcal{M}_n(\mathbb{K})$.
    \begin{enumerate}[label=(\roman*)]
      \item Si $A = \operatorname{Diag}(\lambda_1, \dots, \lambda_n)$, alors $\exp(A) = \operatorname{Diag}(e^\lambda_1, \dots, e^\lambda_n)$.
      \item Si $B = PAP^{-1}$ pour $P \in \mathrm{GL}_n(\mathbb{K})$, alors $e^B = P^{-1} e^A P$.
      \item $\det(e^A) = e^{\trace(A)}$.
      \item $t \mapsto e^{tA}$ est de classe $\mathcal{C}^\infty$, de dérivée $t \mapsto e^{tA}A$.
    \end{enumerate}
  \end{theorem}
  
  \begin{proposition}
    Soient $A, B \in \mathcal{M}_n(\mathbb{K})$ qui commutent. Alors,
    \[ e^A e^B = e^{A+B} = e^B e^A \]
  \end{proposition}
  
  \begin{example}
    Soit $A \in \mathcal{M}_n(\mathbb{K})$ qui admet une décomposition de Dunford $A = D+N$ où $D$ est diagonalisable et $N$ est nilpotente d'indice $q$. Alors,
    \begin{itemize}
      \item $e^A = e^D e^N = e^D \sum_{k=0}^{q-1} \frac{N^k}{k!}$.
      \item La décomposition de Dunford de $e^A$ est $e^A = e^D + e^D(e^N - I_n)$ avec $e^D$ diagonalisable et $e^D(e^N - I_n)$ nilpotente.
    \end{itemize}
  \end{example}
  
  \reference[GOU20]{380}
  
  \begin{application}
    Une équation différentielle linéaire homogène $(H) : Y' = AY$ (où $A$ est constante en $t$) a ses solutions maximales définies sur $\mathbb{R}$ et le problème de Cauchy
    \[ \begin{cases} Y' = AY \\ Y(0) = y_0 \end{cases} \]
    a pour (unique) solution $t \mapsto e^{tA} y_0$.
  \end{application}
  
  \paragraph{Décomposition multiplicative}
  
  \reference[ROM21]{687}
  
  \begin{definition}
    On dit qu'une matrice $U \in \mathcal{M}_n(\mathbb{K})$ est \textbf{unipotente} si $U - I_n$ est nilpotente.
  \end{definition}
  
  \begin{theorem}[Décomposition de Dunford multiplicative]
    Soit $A \in \mathcal{M}_n(\mathbb{K})$. On suppose que $\pi_A$ est scindé sur $\mathbb{K}$. Alors il existe un unique couple de matrices $(D, U)$ tels que :
    \begin{itemize}
      \item $D$ est diagonalisable et $U$ est unipotente.
      \item $A = DU$.
      \item $DU = UD$.
    \end{itemize}
  \end{theorem}
    
  \subsubsection{Décomposition de Jordan}
  
  \reference[BMP]{171}
  
  \begin{definition}
    Un \textbf{bloc de Jordan} de taille $m$ associé à $\lambda \in \mathbb{K}$ désigne la matrice $J_m(\lambda)$ suivante :
    \[ J_m(\lambda) = \begin{pmatrix} \lambda & 1 & \\ & \ddots & \ddots & \\ & & \ddots & 1 \\ & & & \lambda \end{pmatrix} \in \mathcal{M}_m(\mathbb{K}) \]
  \end{definition}
  
  \begin{proposition}
    Soit $A \in \mathcal{M}_n(\mathbb{K})$. Les assertions suivantes sont équivalentes :
    \begin{enumerate}[label=(\roman*)]
      \item $A$ est semblable à $J_n(0)$.
      \item $A$ est nilpotente et cyclique (voir \cref{154-1}).
      \item $A$ est nilpotente d'indice de nilpotence $n$.
    \end{enumerate}
  \end{proposition}
  
  \begin{theorem}[Réduction de Jordan d'un endomorphisme nilpotent]
    On suppose que $A$ est nilpotente. Alors il existe des entiers $n_1 \geq \dots \geq n_p$ tels que $A$ est semblable à la matrice
    \[ \begin{pmatrix} J_{n_1}(0) & & \\ & \ddots & \\ & & J_{n_p}(0) \end{pmatrix} \]
    De plus, on a unicité dans cette décomposition.
  \end{theorem}
  
  \begin{remark}
    Comme l'indice de nilpotence d'un bloc de Jordan est égal à sa taille, l'indice de nilpotence de $A$ est la plus grande des tailles des blocs de Jordan de la réduite.
  \end{remark}
  
  \reference[GOU21]{209}
  
  \begin{theorem}[Réduction de Jordan d'un endomorphisme]
    Soit $A \in \mathcal{M}_n(\mathbb{K})$. On suppose que le polynôme caractéristique de $A$ est scindé sur $\mathbb{K}$ :
    \[ \chi_A = \prod_{i=1}^p (X - \lambda_i)^{\alpha_i} \text{ où les } \lambda_i \text{ sont distincts deux-à-deux} \]
    Alors il existe des entiers $n_1 \geq \dots \geq n_p$ tels que $A$ est semblable à la matrice
    \[ \begin{pmatrix} J_{n_1}(\lambda_1) & & \\ & \ddots & \\ & & J_{n_p}(\lambda_p) \end{pmatrix} \]
    De plus, on a unicité dans cette décomposition.
  \end{theorem}
  
  \begin{application}
    Soit $A \in \mathcal{M}_n(\mathbb{K})$. Alors, $A$ et $2A$ sont semblables si et seulement si $A$ est nilpotente.
  \end{application}
  
  \begin{application}
    Soit $A \in \mathcal{M}_n(\mathbb{K})$. Alors, $A$ et $\tr{A}$ sont semblables.
  \end{application}
  
  \subsubsection{Décomposition de Frobenius}
  
  \reference{397}
  
  Soient $E$ un espace vectoriel de dimension finie $n$ et $u \in \mathcal{L}(E)$.
  
  \begin{definition}
    \label{154-1}
    On dit que $u$ est \textbf{cyclique} s'il existe $x \in E$ tel que $\{ P(u)(x) \mid P \in \mathbb{K}[X] \} = E$.
  \end{definition}
  
  \begin{proposition}
    $u$ est cyclique si et seulement si $\deg(\pi_u) = n$.
  \end{proposition}
  
  \begin{definition}
    Soit $P = X^p + a_{p-1} X^{p-1} + \dots + a_0 \in \mathbb{K}[X]$. On appelle \textbf{matrice compagnon} de $P$ la matrice
    \[ \mathcal{C}(P) = \begin{pmatrix} 0 & \dots & \dots & 0 & -a_0 \\ 1 & 0 & \ddots & \vdots & -a_1 \\ 0 & 1 & \ddots & \vdots & \vdots \\ \vdots & \ddots & \ddots & 0 & -a_{p-2} \\ 0 & \dots & 0 & 1 & -a_{p-1} \end{pmatrix} \]
  \end{definition}
  
  \begin{proposition}
    $u$ est cyclique si et seulement s'il existe une base $\mathcal{B}$ de $E$ telle que $\operatorname{Mat}(u, \mathcal{B}) = \mathcal{C}(\pi_u)$.
  \end{proposition}
  
  \begin{theorem}
    Il existe $F_1, \dots, F_r$ des sous-espaces vectoriels de $E$ tous stables par $u$ tels que :
    \begin{itemize}
      \item $E = F_1 \oplus \dots \oplus F_r$.
      \item $u_i = u_{|F_i}$ est cyclique pour tout $i$.
      \item Si $P_i = \pi_{u_i}$, on a $P_{i+1} \mid P_i$ pour tout $i$.
    \end{itemize}
    La famille de polynômes $P_1, \dots, P_r$ ne dépend que de $u$ et non du choix de la décomposition. On l'appelle \textbf{suite des invariants de similitude} de $u$.
  \end{theorem}
  
  \begin{theorem}[Réduction de Frobenius]
    Si $P_1, \dots, P_r$ désigne la suite des invariants de $u$, alors il existe une base $\mathcal{B}$ de $E$ telle que :
    \[ \operatorname{Mat}(u, \mathcal{B}) = \begin{pmatrix} \mathcal{C}(P_1) & & \\ & \ddots & \\ & & \mathcal{C}(P_r) \end{pmatrix} \]
    On a d'ailleurs $P_1 = \pi_u$ et $P_1 \dots P_r = \chi_u$.
  \end{theorem}
  
  \begin{corollary}
    Deux endomorphismes de $E$ sont semblables si et seulement s'ils ont la même suite d'invariants de similitude.
  \end{corollary}
  
  \begin{application}
    Pour $n = 2$ ou $3$, deux matrices sont semblables si et seulement si elles ont mêmes polynômes minimal et caractéristique.
  \end{application}
  
  \begin{application}
    Soit $\mathbb{L}$ une extension de $\mathbb{K}$. Alors, si $A, B \in \mathcal{M}_n(\mathbb{K})$ sont semblables dans $\mathcal{M}_n(\mathbb{L})$, elles le sont aussi dans $\mathcal{M}_n(\mathbb{K})$.
  \end{application}
  
  \subsection{Décomposition et résolution de systèmes}
  
  \subsubsection{Décomposition LU}
  
  \reference[ROM21]{690}
  
  \begin{definition}
    Les \textbf{sous-matrices principales} d'une matrice $(a_{i,j})_{i,j \in \llbracket 1, n \rrbracket} \in \mathcal{M}_n(\mathbb{K})$ sont les matrices $A_k = (a_{i,j})_{i,j \in \llbracket 1, k \rrbracket} \in \mathcal{M}_k(\mathbb{K})$ où $k \in \llbracket 1, n \rrbracket$. Les \textbf{déterminants principaux} sont les déterminants des matrices $A_k$, pour $k \in \llbracket 1, n \rrbracket$.
  \end{definition}
  
  \begin{theorem}[Décomposition lower-upper]
    \label{154-2}
    Soit $A \in \mathrm{GL}_n(\mathbb{K})$. Alors, $A$ admet une décomposition
    \[ A = LU \]
    (où $L$ est une matrice triangulaire inférieure à diagonale unité et $U$ une matrice triangulaire supérieure) si et seulement si tous les déterminants principaux de $A$ sont non nuls. Dans ce cas, une telle décomposition est unique.
  \end{theorem}
  
  \begin{corollary}
    Soit $A \in \mathrm{GL}_n(\mathbb{K}) \, \cap \, \mathcal{S}_n(\mathbb{K})$. Alors, on a l'unique décomposition de $A$ :
    \[ A = LD\tr{L} \]
    où $L$ est une matrice triangulaire inférieure et $D$ une matrice diagonale.
  \end{corollary}
  
  \begin{application}[Décomposition de Cholesky]
    Soit $A \in \mathcal{M}_n(\mathbb{R})$. Alors, $A \in \mathcal{S}_n^{++}(\mathbb{R})$ si et seulement s'il existe $B \in \mathrm{GL}_n(\mathbb{R})$ triangulaire inférieure telle que $A = B\tr{B}$. De plus, une telle décomposition est unique si on impose la positivité des coefficients diagonaux de $B$.
  \end{application}
  
  \reference[GRI]{368}
  
  \begin{example}
    On a la décomposition de Cholesky :
    \[ \begin{pmatrix} 1 & 2 \\ 2 & 5 \end{pmatrix} = \begin{pmatrix} 1 & 0 \\ 2 & 1 \end{pmatrix} \begin{pmatrix} 1 & 2 \\ 0 & 1 \end{pmatrix} \]
  \end{example}
  
  \reference[C-G]{257}
  
  \begin{proposition}
    Soit $A \in \mathrm{GL}_n(\mathbb{K})$ vérifiant les hypothèses du \cref{154-2}. On définit la suite $(A_k)$ où $A_0 = A$ et $\forall k \in \mathbb{N}$, $A_{k+1}$ est la matrice obtenue à partir de $A_k$ à l'aide du pivot de Gauss sur la $(k+1)$-ième colonne. Alors, $A_{n-1}$ est la matrice $U$ de la décomposition $A = LU$ du \cref{154-2}.
  \end{proposition}
  
  \begin{remark}
    Pour résoudre un système linéaire $AX = Y$, on se ramène à $A = LU$ en $O \left( \frac{2}{3}n^3 \right)$. Puis, on résout deux systèmes triangulaires ``en cascade'' :
    \[ LX' = Y \text{ puis } UX = X' \]
    ceux-ci demandant chacun $O(2n^2)$ opérations.
  \end{remark}
  
  \begin{theorem}[Décomposition PLU]
    Soit $A \in \mathrm{GL}_n(\mathbb{K})$. Alors, il existe $P \in \mathrm{GL}_n(\mathbb{K})$, matrice de permutations, telle que $P^{-1}A$ admet une décomposition $LU$.
  \end{theorem}
  
  \subsubsection{Décomposition QR}
  
  \reference[ROM21]{692}
  
  \begin{theorem}[Décomposition QR]
    Soit $A \in \mathrm{GL}_n(\mathbb{R})$. Alors, $A$ admet une décomposition
    \[ A = QR \]
    où $Q$ est une matrice orthogonale et $R$ est une matrice triangulaire supérieure à coefficients diagonaux strictement positifs. On a unicité d'une telle décomposition.
  \end{theorem}
  
  \begin{corollary}[Théorème d'Iwasawa]
    Soit $A \in \mathrm{GL}_n(\mathbb{R})$. Alors, $A$ admet une décomposition
    \[ A = QDR \]
    où $Q$ est une matrice orthogonale, $D$ est une matrice diagonale à coefficients strictement positifs et $R$ est une matrice triangulaire supérieure à coefficients diagonaux égaux à $1$. On a unicité d'une telle décomposition.
  \end{corollary}
  
  \reference[GRI]{272}
  
  \begin{example}
    On a la factorisation QR suivante,
    \[ \begin{pmatrix} 0 & 1 & 1 \\ 1 & 0 & 1 \\ 1 & 1 & 0 \end{pmatrix} = \left( \frac{1}{\sqrt{6}} \begin{pmatrix} 0 & 2 & \sqrt{2} \\ \sqrt{3} & -1 & \sqrt{2} \\ \sqrt{3} & 1 & -\sqrt{2} \end{pmatrix} \right) \left( \frac{1}{\sqrt{6}} \begin{pmatrix} 2\sqrt{3} & \sqrt{3} & 1 \\ 0 & 3 & 1 \\ 0 & 0 & 2\sqrt{2} \end{pmatrix} \right) \]
    qui peut être obtenue via un procédé de Gram-Schmidt.
  \end{example}
  
  \reference{368}
  
  \begin{remark}
    Pour résoudre un système linéaire $AX = Y$, si l'on a trouvé une telle factorisation $A = QR$, on résout
    \[ RX = \tr{Q}Y \]
    c'est-à-dire, un seul système triangulaire (contre deux pour la factorisation LU).
  \end{remark}
  
  \subsection{Décomposition et topologie}
  
  \reference[C-G]{376}
  
  \begin{lemma}
    \[ \forall A \in \mathcal{S}_n^{++}(\mathbb{R}) \, \exists! B \in \mathcal{S}_n^{++}(\mathbb{R}) \text{ telle que } B^2 = A \]
  \end{lemma}
  
  \dev{decomposition-polaire}
  
  \begin{theorem}[Décomposition polaire]
    L'application
    \[ \mu :
    \begin{array}{ccc}
      \mathcal{O}_n(\mathbb{R}) \times \mathcal{S}_n^{++}(\mathbb{R}) &\rightarrow& \mathrm{GL}_n(\mathbb{R}) \\
      (O, S) &\mapsto& OS
    \end{array}
    \]
    est un homéomorphisme.
  \end{theorem}
  
  \begin{corollary}
    Tout sous-groupe compact de $\mathrm{GL}_n(\mathbb{R})$ qui contient $\mathcal{O}_n(\mathbb{R})$ est $\mathcal{O}_n(\mathbb{R})$.
  \end{corollary}
  
  \reference{401}
  
  \begin{corollary}
    $\mathrm{GL}_n(\mathbb{R})^+$ est connexe.
  \end{corollary}
  %</content>
\end{document}
