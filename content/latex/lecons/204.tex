\documentclass[12pt, a4paper]{report}

% LuaLaTeX :

\RequirePackage{iftex}
\RequireLuaTeX

% Packages :

\usepackage[french]{babel}
%\usepackage[utf8]{inputenc}
%\usepackage[T1]{fontenc}
\usepackage[pdfencoding=auto, pdfauthor={Hugo Delaunay}, pdfsubject={Mathématiques}, pdfcreator={agreg.skyost.eu}]{hyperref}
\usepackage{amsmath}
\usepackage{amsthm}
%\usepackage{amssymb}
\usepackage{stmaryrd}
\usepackage{tikz}
\usepackage{tkz-euclide}
\usepackage{fourier-otf}
\usepackage{fontspec}
\usepackage{titlesec}
\usepackage{fancyhdr}
\usepackage{catchfilebetweentags}
\usepackage[french, capitalise, noabbrev]{cleveref}
\usepackage[fit, breakall]{truncate}
\usepackage[top=2.5cm, right=2cm, bottom=2.5cm, left=2cm]{geometry}
\usepackage{enumerate}
\usepackage{tocloft}
\usepackage{microtype}
%\usepackage{mdframed}
%\usepackage{thmtools}
\usepackage{xcolor}
\usepackage{tabularx}
\usepackage{aligned-overset}
\usepackage[subpreambles=true]{standalone}
\usepackage{environ}
\usepackage[normalem]{ulem}
\usepackage{marginnote}
\usepackage{etoolbox}
\usepackage{setspace}
\usepackage[bibstyle=reading, citestyle=draft]{biblatex}
\usepackage{xpatch}
\usepackage[many, breakable]{tcolorbox}
\usepackage[backgroundcolor=white, bordercolor=white, textsize=small]{todonotes}

% Bibliographie :

\newcommand{\overridebibliographypath}[1]{\providecommand{\bibliographypath}{#1}}
\overridebibliographypath{../bibliography.bib}
\addbibresource{\bibliographypath}
\defbibheading{bibliography}[\bibname]{%
	\newpage
	\section*{#1}%
}
\renewbibmacro*{entryhead:full}{\printfield{labeltitle}}
\DeclareFieldFormat{url}{\newline\footnotesize\url{#1}}
\AtEndDocument{\printbibliography}

% Police :

\setmathfont{Erewhon Math}

% Tikz :

\usetikzlibrary{calc}

% Longueurs :

\setlength{\parindent}{0pt}
\setlength{\headheight}{15pt}
\setlength{\fboxsep}{0pt}
\titlespacing*{\chapter}{0pt}{-20pt}{10pt}
\setlength{\marginparwidth}{1.5cm}
\setstretch{1.1}

% Métadonnées :

\author{agreg.skyost.eu}
\date{\today}

% Titres :

\setcounter{secnumdepth}{3}

\renewcommand{\thechapter}{\Roman{chapter}}
\renewcommand{\thesubsection}{\Roman{subsection}}
\renewcommand{\thesubsubsection}{\arabic{subsubsection}}
\renewcommand{\theparagraph}{\alph{paragraph}}

\titleformat{\chapter}{\huge\bfseries}{\thechapter}{20pt}{\huge\bfseries}
\titleformat*{\section}{\LARGE\bfseries}
\titleformat{\subsection}{\Large\bfseries}{\thesubsection \, - \,}{0pt}{\Large\bfseries}
\titleformat{\subsubsection}{\large\bfseries}{\thesubsubsection. \,}{0pt}{\large\bfseries}
\titleformat{\paragraph}{\bfseries}{\theparagraph. \,}{0pt}{\bfseries}

\setcounter{secnumdepth}{4}

% Table des matières :

\renewcommand{\cftsecleader}{\cftdotfill{\cftdotsep}}
\addtolength{\cftsecnumwidth}{10pt}

% Redéfinition des commandes :

\renewcommand*\thesection{\arabic{section}}
\renewcommand{\ker}{\mathrm{Ker}}

% Nouvelles commandes :

\newcommand{\website}{https://agreg.skyost.eu}

\newcommand{\tr}[1]{\mathstrut ^t #1}
\newcommand{\im}{\mathrm{Im}}
\newcommand{\rang}{\operatorname{rang}}
\newcommand{\trace}{\operatorname{trace}}
\newcommand{\id}{\operatorname{id}}
\newcommand{\stab}{\operatorname{Stab}}

\providecommand{\newpar}{\\[\medskipamount]}

\providecommand{\lesson}[3]{%
	\title{#3}%
	\hypersetup{pdftitle={#3}}%
	\setcounter{section}{\numexpr #2 - 1}%
	\section{#3}%
	\fancyhead[R]{\truncate{0.73\textwidth}{#2 : #3}}%
}

\providecommand{\development}[3]{%
	\title{#3}%
	\hypersetup{pdftitle={#3}}%
	\section*{#3}%
	\fancyhead[R]{\truncate{0.73\textwidth}{#3}}%
}

\providecommand{\summary}[1]{%
	\textit{#1}%
	\medskip%
}

\tikzset{notestyleraw/.append style={inner sep=0pt, rounded corners=0pt, align=center}}

%\newcommand{\booklink}[1]{\website/bibliographie\##1}
\newcommand{\citelink}[2]{\hyperlink{cite.\therefsection @#1}{#2}}
\newcommand{\previousreference}{}
\providecommand{\reference}[2][]{%
	\notblank{#1}{\renewcommand{\previousreference}{#1}}{}%
	\todo[noline]{%
		\protect\vspace{16pt}%
		\protect\par%
		\protect\notblank{#1}{\cite{[\previousreference]}\\}{}%
		\protect\citelink{\previousreference}{p. #2}%
	}%
}

\definecolor{devcolor}{HTML}{00695c}
\newcommand{\dev}[1]{%
	\reversemarginpar%
	\todo[noline]{
		\protect\vspace{16pt}%
		\protect\par%
		\bfseries\color{devcolor}\href{\website/developpements/#1}{DEV}
	}%
	\normalmarginpar%
}

% En-têtes :

\pagestyle{fancy}
\fancyhead[L]{\truncate{0.23\textwidth}{\thepage}}
\fancyfoot[C]{\scriptsize \href{\website}{\texttt{agreg.skyost.eu}}}

% Couleurs :

\definecolor{property}{HTML}{fffde7}
\definecolor{proposition}{HTML}{fff8e1}
\definecolor{lemma}{HTML}{fff3e0}
\definecolor{theorem}{HTML}{fce4f2}
\definecolor{corollary}{HTML}{ffebee}
\definecolor{definition}{HTML}{ede7f6}
\definecolor{notation}{HTML}{f3e5f5}
\definecolor{example}{HTML}{e0f7fa}
\definecolor{cexample}{HTML}{efebe9}
\definecolor{application}{HTML}{e0f2f1}
\definecolor{remark}{HTML}{e8f5e9}
\definecolor{proof}{HTML}{e1f5fe}

% Théorèmes :

\theoremstyle{definition}
\newtheorem{theorem}{Théorème}

\newtheorem{property}[theorem]{Propriété}
\newtheorem{proposition}[theorem]{Proposition}
\newtheorem{lemma}[theorem]{Lemme}
\newtheorem{corollary}[theorem]{Corollaire}

\newtheorem{definition}[theorem]{Définition}
\newtheorem{notation}[theorem]{Notation}

\newtheorem{example}[theorem]{Exemple}
\newtheorem{cexample}[theorem]{Contre-exemple}
\newtheorem{application}[theorem]{Application}

\theoremstyle{remark}
\newtheorem{remark}[theorem]{Remarque}

\counterwithin*{theorem}{section}

\newcommand{\applystyletotheorem}[1]{
	\tcolorboxenvironment{#1}{
		enhanced,
		breakable,
		colback=#1!98!white,
		boxrule=0pt,
		boxsep=0pt,
		left=8pt,
		right=8pt,
		top=8pt,
		bottom=8pt,
		sharp corners,
		after=\par,
	}
}

\applystyletotheorem{property}
\applystyletotheorem{proposition}
\applystyletotheorem{lemma}
\applystyletotheorem{theorem}
\applystyletotheorem{corollary}
\applystyletotheorem{definition}
\applystyletotheorem{notation}
\applystyletotheorem{example}
\applystyletotheorem{cexample}
\applystyletotheorem{application}
\applystyletotheorem{remark}
\applystyletotheorem{proof}

% Environnements :

\NewEnviron{whitetabularx}[1]{%
	\renewcommand{\arraystretch}{2.5}
	\colorbox{white}{%
		\begin{tabularx}{\textwidth}{#1}%
			\BODY%
		\end{tabularx}%
	}%
}

% Maths :

\DeclareFontEncoding{FMS}{}{}
\DeclareFontSubstitution{FMS}{futm}{m}{n}
\DeclareFontEncoding{FMX}{}{}
\DeclareFontSubstitution{FMX}{futm}{m}{n}
\DeclareSymbolFont{fouriersymbols}{FMS}{futm}{m}{n}
\DeclareSymbolFont{fourierlargesymbols}{FMX}{futm}{m}{n}
\DeclareMathDelimiter{\VERT}{\mathord}{fouriersymbols}{152}{fourierlargesymbols}{147}


% Bibliographie :

\addbibresource{\bibliographypath}%
\defbibheading{bibliography}[\bibname]{%
	\newpage
	\section*{#1}%
}
\renewbibmacro*{entryhead:full}{\printfield{labeltitle}}%
\DeclareFieldFormat{url}{\newline\footnotesize\url{#1}}%

\AtEndDocument{\printbibliography}

\begin{document}
	%<*content>
	\lesson{analysis}{204}{Connexité. Exemples d'applications.}

	\subsection{Diverses approches de la connexité}

	Soit $(E,d)$ un espace métrique.

	\subsubsection{Une approche topologique}

	\reference[GOU21]{38}

	\begin{proposition}
		\label{204-1}
		Les assertions suivantes sont équivalentes.
		\begin{enumerate}[label=(\roman*)]
			\item Il n'existe pas de partition de $E$ en deux ouverts disjoints non vides.
			\item Il n'existe pas de partition de $E$ en deux fermés disjoints non vides.
			\item Les seules parties ouvertes de $E$ sont $\emptyset$ et $E$.
		\end{enumerate}
	\end{proposition}

	\begin{definition}
		Un espace métrique vérifiant l'une des assertions de \cref{204-1} est dit \textbf{connexe}.
	\end{definition}

	\begin{remark}
		Remarquons qu'il s'agit-là d'une définition \textit{topologique} : tous les résultats de cette sous-section sont donc valables dans le cadre plus général d'un espace topologique.
	\end{remark}

	\begin{proposition}
		Soit $A \subseteq E$. Les assertions suivantes sont équivalentes.
		\begin{enumerate}[label=(\roman*)]
			\item $A$ est connexe.
			\item Si $A \subseteq O_1 \, \cap \, O_2$ avec $O_1$, $O_2$ ouverts de $E$ tels que $A \, \cap \, O_1 \, \cap \, O_2 = \emptyset$, alors
			\[ (A \, \cap \, O_1 = \emptyset \text{ et } A \subseteq O_2) \text{ ou } (A \, \cap \, O_2 = \emptyset \text{ et } A \subseteq O_1) \]
			\item Si $A \subseteq F_1 \, \cap \, F_2$ avec $F_1$, $F_2$ fermés de $E$ tels que $A \, \cap \, F_1 \, \cap \, F_2 = \emptyset$, alors
			\[ (A \, \cap \, F_1 = \emptyset \text{ et } A \subseteq F_2) \text{ ou } (A \, \cap \, F_2 = \emptyset \text{ et } A \subseteq F_1) \]
		\end{enumerate}
	\end{proposition}

	\begin{example}
		$\mathbb{Q}$ n'est pas un connexe de $\mathbb{R}$.
	\end{example}

	\reference{350}

	\begin{proposition}
		Une partie ouverte et fermée d'un espace connexe est vide ou égale à l'espace entier.
	\end{proposition}

	\reference{39}

	\begin{proposition}
		L'image d'un connexe par une application continue est connexe.
	\end{proposition}

	\reference{44}

	\begin{application}
		Soit $f : \mathbb{U} \rightarrow \mathbb{R}$ continue. Alors il existe deux points diamétralement opposés de $\mathbb{U}$ qui ont la même image par $f$.
	\end{application}

	\reference{39}

	\begin{corollary}
		$E$ est connexe si et seulement si toute application continue de $E$ dans $\{ 0, 1 \}$ est constante.
	\end{corollary}

	\begin{proposition}
		Soit $(C_i)_{i \in I}$ une famille de parties connexes de $E$. On suppose que
		\[ \exists i_0 \in I \text{ tel que } \forall i \in I, \, C_{i_0} \, \cap \, C_i \neq \emptyset \]
		Alors, $\bigcup_{i \in I} C_i$ est connexe.
	\end{proposition}

	\begin{cexample}
		$\{ 0 \}$ et $\{ 1 \}$ sont des connexes de $\mathbb{R}$, mais pas $\{ 0 \} \, \cup \, \{ 1 \} = \{ 0, 1 \}$.
	\end{cexample}

	\begin{proposition}
		Un produit fini d'espaces métriques est connexe si et seulement si ces espaces métriques sont tous connexes.
	\end{proposition}

	\reference[I-P]{116}
	\dev{connexite-des-valeurs-d-adherence-d-une-suite-dans-un-compact}

	\begin{application}
		Soit $(E, d)$ un espace métrique compact. Soit $(u_n)$ une suite de $E$ telle que $d(u_n,u_{n-1}) \longrightarrow 0$. Alors l'ensemble $\Gamma$ des valeurs d'adhérence de $(u_n)$ est connexe.
	\end{application}

	\begin{corollary}[Lemme de la grenouille]
		Soient $f : [0, 1] \rightarrow [0, 1]$ continue et $(x_n)$ une suite de $[0, 1]$ telle que
		\[ \begin{cases} x_0 \in [0, 1] \\ x_{n+1} = f(x_n) \end{cases} \]
		Alors $(x_n)$ converge si et seulement si $\lim_{n \rightarrow +\infty } x_{n+1} - x_n = 0$.
	\end{corollary}

	\subsubsection{Une approche géométrique}

	\reference[GOU20]{42}

	\begin{definition}
		On appelle \textbf{chemin} de $E$ tout application $\gamma : [0,1] \rightarrow E$ continue. L'image $\gamma^* = \gamma([0,1])$ du chemin s'appelle un \textbf{arc}, $\gamma(0)$ \textbf{l'origine} de l'arc et $\gamma(1)$ son \textbf{extrémité}.
	\end{definition}

	\begin{definition}
		$E$ est dit \textbf{connexe par arcs} si pour tout $(a,b) \in E^2$, il existe un arc inclus dans $E$ d'origine $a$ et d'extrémité $b$.
	\end{definition}

	\begin{remark}
		Il s'agit là encore d'une définition topologique.
	\end{remark}

	\begin{theorem}
		Un espace connexe par arcs est connexe.
	\end{theorem}

	\begin{cexample}
		L'ensemble
		\[
			\Gamma = \left( \bigcup_{x \in \mathbb{Q}} (\{ x \} \times \mathbb{R}^+) \right) \, \cup \, \left( \bigcup_{x \in \mathbb{R} \setminus \mathbb{Q}} (\{ x \} \times \mathbb{R}^-_*) \right)
		\]
		est un connexe de $\mathbb{R}^2$ non connexe par arcs.
	\end{cexample}

	\begin{proposition}
		La réciproque est vraie dans un ouvert d'un espace vectoriel normé.
	\end{proposition}

	\begin{application}
		$\mathbb{R}$ et $\mathbb{R}^2$ ne sont pas homéomorphes.
	\end{application}

	\subsubsection{Une approche algébrique}

	\begin{definition}
		On définit la relation $\mathcal{R}$ suivante sur $E$ :
		\[ x \mathcal{R} y \iff \exists C \subseteq E \text{ connexe tel que } x, y \in E \]
	\end{definition}

	\begin{proposition}
		\begin{enumerate}[label=(\roman*)]
			\item $\mathcal{R}$ est une relation d'équivalence sur $E$.
			\item Si $x \in E$, sa classe d'équivalence est la réunion des connexes contenant $x$.
		\end{enumerate}
	\end{proposition}

	\begin{definition}
		Une classe d'équivalence pour la relation $\mathcal{R}$ est une \textbf{composante connexe} de $E$.
	\end{definition}

	\begin{remark}
		$E$ est la réunion disjointe de ses composantes connexes. $E$ est donc connexe s'il n'admet qu'une seule composante connexe.
	\end{remark}

	\reference[ROM21]{724}

	\begin{example}
		On se place dans le cadre où $E$ est un espace vectoriel euclidien. Alors, $\mathcal{O}(E)$ est non-connexe. Ses composantes connexes sont $\mathrm{SO}(E)$ et $\{ u \in \mathcal{O}(E) \mid \det(u) = -1 \}$.
	\end{example}

	\reference[GOU20]{41}

	\begin{proposition}
		Les composantes connexes de $E$ sont des fermés de $E$. Si elles sont en nombre fini, ce sont également des ouverts de $E$.
	\end{proposition}
	
	\newpage

	\subsection{Exemples d'applications en analyse}

	\subsubsection{En analyse réelle}

	\reference{41}

	\begin{theorem}
		Les connexes de $\mathbb{R}$ sont les intervalles.
	\end{theorem}

	\begin{theorem}[Des valeurs intermédiaires]
		Soient $I$ un intervalle de $\mathbb{R}$ et $f : I \rightarrow \mathbb{R}$ continue sur $I$. Alors $f(I)$ est un intervalle.
	\end{theorem}

	\begin{remark}
		Une autre manière d'écrire ce résultat est que si $f(a) \leq f(b)$ (resp. $f(a) \geq f(b)$) avec $a<b$, alors pour tout $\gamma \in [f(a),f(b)]$ (resp. pour tout $\gamma \in [f(b),f(a)]$), il existe $c \in [a,b]$ tel que $f(c) = \gamma$.
	\end{remark}

	\reference{47}

	\begin{corollary}[Darboux]
		Soient $I$ un intervalle de $\mathbb{R}$ et $f : I \rightarrow \mathbb{R}$ dérivable sur $I$. Alors $f'(I)$ est un intervalle.
	\end{corollary}

	\subsubsection{En calcul différentiel}

	\reference{328}

	\begin{proposition}
		Soit $U$ un ouvert connexe d'un espace vectoriel normé $E$. Soit $f : U \rightarrow F$ où $F$ est un espace vectoriel normé. Si $f$ est différentiable telle que $\forall x \in U, \, \mathrm{d}f_x = 0$, alors $f$ est constante sur $U$.
	\end{proposition}

	\reference[BMP]{80}

	\begin{example}
		Soit $f$ une fonction holomorphe sur un ouvert connexe $\Omega$ de $\mathbb{C}$ telle que la suite $(f^{(n)})$ converge uniformément sur tout compact de $\Omega$. On note $g$ la limite de la suite $(f^{(n)})$. Alors, il existe $C \in \mathbb{C}$ tel que $g = C\exp$.
	\end{example}

	\reference[GOU20]{349}

	\begin{proposition}
		Soit $U$ un ouvert connexe d'un espace vectoriel normé $E$. Soit $f : U \rightarrow E$. Si $f$ est de classe $\mathcal{C}^1$ telle que $\forall x \in U, \, \mathrm{d}f_x$ est une isométrie, alors $f$ est une isométrie affine.
	\end{proposition}

	\subsubsection{En analyse complexe}

	\reference[BMP]{53}

	Soit $\Omega \subseteq \mathbb{C}$ un ouvert. On suppose $\Omega$ connexe. Soit $f : \Omega \rightarrow \mathbb{C}$.

	\begin{theorem}[Zéros isolés]
		\label{204-2}
		Si $f$ est une fonction analytique sur $\Omega$ et si $f$ n'est pas identiquement nulle, alors l'ensemble des zéros de $f$ n'admet pas de point d'accumulation dans $\Omega$.
	\end{theorem}

	\begin{corollary}
		L'ensemble des zéros d'une fonction analytique non nulle sur $\Omega$ est au plus dénombrable.
	\end{corollary}

	\begin{remark}[Prolongement analytique]
		Reformulé de manière équivalente au \cref{204-2}, si deux fonctions analytiques coïncident sur un sous-ensemble de $\Omega$ qui possède un point d'accumulation dans $\Omega$, alors elles sont égales sur $\Omega$.
	\end{remark}

	\reference{77}

	\begin{example}
		Il existe une unique fonction $g$ holomorphe sur $\mathbb{C}$ telle que
		\[ \forall n \in \mathbb{N}^*, \, g\left( \frac{1}{n} \right) = \frac{1}{n} \]
		et c'est la fonction identité.
	\end{example}

	\begin{cexample}
		Il existe au moins deux fonctions $g$ holomorphes sur $\Omega = \{ z \in \mathbb{C} \mid \operatorname{Re}(z) > 0 \}$ telles que
		\[ \forall n \in \mathbb{N}^*, \, g\left( \frac{1}{n} \right) = 0 \]
	\end{cexample}

	\reference{83}

	\begin{application}[Transformée de Fourier d'une Gaussienne]
		On a
		\[ \forall x \in \mathbb{R}, \, \int_{\mathbb{R}} e^{-t^2} e^{-itx} \, \mathrm{d}t = \sqrt{\pi} e^{-\frac{x^2}{4}} \]
	\end{application}

	\reference{72}

	\begin{theorem}
		On suppose $\Omega$ borné et $f$ holomorphe sur $\Omega$ et continue sur $\overline{\Omega}$. On note $M$ le maximum de $f$ sur la frontière de $\Omega$. Alors,
		\begin{enumerate}[label=(\roman*)]
			\item Pour tout $z \in \Omega$, $\vert f(z) \vert \leq M$.
			\item S'il existe $z_0 \in \Omega$ tel que $\vert f(z) \vert = M$, alors $f$ est constant sur $\Omega$.
		\end{enumerate}
	\end{theorem}

	\reference{80}

	\begin{application}
		Soit $(f_n)$ une suite de fonctions holomorphes sur $\Omega$ et continues sur $\overline{\Omega}$. Si $(f_n)$ converge uniformément sur la frontière de $\Omega$, alors $(f_n)$ converge uniformément sur $\Omega$ et la limite est holomorphe.
	\end{application}

	\begin{application}
		On suppose que $D(0,1) \subseteq \Omega$ et $f$ holomorphe sur $\Omega$. On suppose de plus que $f(0) = 1$ et $\vert f(z) \vert \geq 2$ sur le cercle unité. Alors $f$ s'annule sur le cercle unité.
	\end{application}

	\subsection{Exemple d'application en algèbre}

	\reference[BMP]{213}

	\begin{proposition}
		$\mathrm{GL}_n(\mathbb{R})$ n'est pas connexe. Ses composantes connexes sont $\mathrm{GL}_n(\mathbb{R})^+$ et $\mathrm{GL}_n(\mathbb{R})^-$.
	\end{proposition}

	\begin{application}
		$\exp : \mathcal{M}_n(\mathbb{R}) \rightarrow \mathrm{GL}_n(\mathbb{R})$ n'est pas surjective.
	\end{application}

	\reference[ROM21]{770}

	\begin{proposition}
		$\mathrm{GL}_n(\mathbb{C})$ est connexe par arcs.
	\end{proposition}

	\begin{application}
		$\exp : \mathcal{M}_n(\mathbb{C}) \rightarrow \mathrm{GL}_n(\mathbb{C})$ est surjective.
	\end{application}

	\reference[I-P]{396}

    \begin{lemma}
      \begin{enumerate}[label=(\roman*)]
        \item Soit $A \in \mathcal{M}_n(\mathbb{C})$. Alors $\exp(A) \in \mathrm{GL}_n(\mathbb{C})$.
        \item $\exp$ est différentiable en $0$ et $\mathrm{d}\exp_0 = \operatorname{id}_{\mathcal{M}_n(\mathbb{C})}$.
        \item Soit $M \in \mathrm{GL}_n(\mathbb{C})$. Alors $M^{-1} \in \mathbb{C}[M]$.
      \end{enumerate}
    \end{lemma}

    \dev{surjectivite-de-l-exponentielle}

    \begin{theorem}
      $\exp : \mathcal{M}_n(\mathbb{C}) \rightarrow \mathrm{GL}_n(\mathbb{C})$ est surjective.
    \end{theorem}

    \begin{application}
      $\exp(\mathcal{M}_n(\mathbb{R})) = \mathrm{GL}_n(\mathbb{R})^2$, où $\mathrm{GL}_n(\mathbb{R})^2$ désigne les carrés de $\mathrm{GL}_n(\mathbb{R})$.
    \end{application}
	%</content>
\end{document}
