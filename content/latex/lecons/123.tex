\documentclass[12pt, a4paper]{report}

% LuaLaTeX :

\RequirePackage{iftex}
\RequireLuaTeX

% Packages :

\usepackage[french]{babel}
%\usepackage[utf8]{inputenc}
%\usepackage[T1]{fontenc}
\usepackage[pdfencoding=auto, pdfauthor={Hugo Delaunay}, pdfsubject={Mathématiques}, pdfcreator={agreg.skyost.eu}]{hyperref}
\usepackage{amsmath}
\usepackage{amsthm}
%\usepackage{amssymb}
\usepackage{stmaryrd}
\usepackage{tikz}
\usepackage{tkz-euclide}
\usepackage{fourier-otf}
\usepackage{fontspec}
\usepackage{titlesec}
\usepackage{fancyhdr}
\usepackage{catchfilebetweentags}
\usepackage[french, capitalise, noabbrev]{cleveref}
\usepackage[fit, breakall]{truncate}
\usepackage[top=2.5cm, right=2cm, bottom=2.5cm, left=2cm]{geometry}
\usepackage{enumerate}
\usepackage{tocloft}
\usepackage{microtype}
%\usepackage{mdframed}
%\usepackage{thmtools}
\usepackage{xcolor}
\usepackage{tabularx}
\usepackage{aligned-overset}
\usepackage[subpreambles=true]{standalone}
\usepackage{environ}
\usepackage[normalem]{ulem}
\usepackage{marginnote}
\usepackage{etoolbox}
\usepackage{setspace}
\usepackage[bibstyle=reading, citestyle=draft]{biblatex}
\usepackage{xpatch}
\usepackage[many, breakable]{tcolorbox}
\usepackage[backgroundcolor=white, bordercolor=white, textsize=small]{todonotes}

% Bibliographie :

\newcommand{\overridebibliographypath}[1]{\providecommand{\bibliographypath}{#1}}
\overridebibliographypath{../bibliography.bib}
\addbibresource{\bibliographypath}
\defbibheading{bibliography}[\bibname]{%
	\newpage
	\section*{#1}%
}
\renewbibmacro*{entryhead:full}{\printfield{labeltitle}}
\DeclareFieldFormat{url}{\newline\footnotesize\url{#1}}
\AtEndDocument{\printbibliography}

% Police :

\setmathfont{Erewhon Math}

% Tikz :

\usetikzlibrary{calc}

% Longueurs :

\setlength{\parindent}{0pt}
\setlength{\headheight}{15pt}
\setlength{\fboxsep}{0pt}
\titlespacing*{\chapter}{0pt}{-20pt}{10pt}
\setlength{\marginparwidth}{1.5cm}
\setstretch{1.1}

% Métadonnées :

\author{agreg.skyost.eu}
\date{\today}

% Titres :

\setcounter{secnumdepth}{3}

\renewcommand{\thechapter}{\Roman{chapter}}
\renewcommand{\thesubsection}{\Roman{subsection}}
\renewcommand{\thesubsubsection}{\arabic{subsubsection}}
\renewcommand{\theparagraph}{\alph{paragraph}}

\titleformat{\chapter}{\huge\bfseries}{\thechapter}{20pt}{\huge\bfseries}
\titleformat*{\section}{\LARGE\bfseries}
\titleformat{\subsection}{\Large\bfseries}{\thesubsection \, - \,}{0pt}{\Large\bfseries}
\titleformat{\subsubsection}{\large\bfseries}{\thesubsubsection. \,}{0pt}{\large\bfseries}
\titleformat{\paragraph}{\bfseries}{\theparagraph. \,}{0pt}{\bfseries}

\setcounter{secnumdepth}{4}

% Table des matières :

\renewcommand{\cftsecleader}{\cftdotfill{\cftdotsep}}
\addtolength{\cftsecnumwidth}{10pt}

% Redéfinition des commandes :

\renewcommand*\thesection{\arabic{section}}
\renewcommand{\ker}{\mathrm{Ker}}

% Nouvelles commandes :

\newcommand{\website}{https://agreg.skyost.eu}

\newcommand{\tr}[1]{\mathstrut ^t #1}
\newcommand{\im}{\mathrm{Im}}
\newcommand{\rang}{\operatorname{rang}}
\newcommand{\trace}{\operatorname{trace}}
\newcommand{\id}{\operatorname{id}}
\newcommand{\stab}{\operatorname{Stab}}

\providecommand{\newpar}{\\[\medskipamount]}

\providecommand{\lesson}[3]{%
	\title{#3}%
	\hypersetup{pdftitle={#3}}%
	\setcounter{section}{\numexpr #2 - 1}%
	\section{#3}%
	\fancyhead[R]{\truncate{0.73\textwidth}{#2 : #3}}%
}

\providecommand{\development}[3]{%
	\title{#3}%
	\hypersetup{pdftitle={#3}}%
	\section*{#3}%
	\fancyhead[R]{\truncate{0.73\textwidth}{#3}}%
}

\providecommand{\summary}[1]{%
	\textit{#1}%
	\medskip%
}

\tikzset{notestyleraw/.append style={inner sep=0pt, rounded corners=0pt, align=center}}

%\newcommand{\booklink}[1]{\website/bibliographie\##1}
\newcommand{\citelink}[2]{\hyperlink{cite.\therefsection @#1}{#2}}
\newcommand{\previousreference}{}
\providecommand{\reference}[2][]{%
	\notblank{#1}{\renewcommand{\previousreference}{#1}}{}%
	\todo[noline]{%
		\protect\vspace{16pt}%
		\protect\par%
		\protect\notblank{#1}{\cite{[\previousreference]}\\}{}%
		\protect\citelink{\previousreference}{p. #2}%
	}%
}

\definecolor{devcolor}{HTML}{00695c}
\newcommand{\dev}[1]{%
	\reversemarginpar%
	\todo[noline]{
		\protect\vspace{16pt}%
		\protect\par%
		\bfseries\color{devcolor}\href{\website/developpements/#1}{DEV}
	}%
	\normalmarginpar%
}

% En-têtes :

\pagestyle{fancy}
\fancyhead[L]{\truncate{0.23\textwidth}{\thepage}}
\fancyfoot[C]{\scriptsize \href{\website}{\texttt{agreg.skyost.eu}}}

% Couleurs :

\definecolor{property}{HTML}{fffde7}
\definecolor{proposition}{HTML}{fff8e1}
\definecolor{lemma}{HTML}{fff3e0}
\definecolor{theorem}{HTML}{fce4f2}
\definecolor{corollary}{HTML}{ffebee}
\definecolor{definition}{HTML}{ede7f6}
\definecolor{notation}{HTML}{f3e5f5}
\definecolor{example}{HTML}{e0f7fa}
\definecolor{cexample}{HTML}{efebe9}
\definecolor{application}{HTML}{e0f2f1}
\definecolor{remark}{HTML}{e8f5e9}
\definecolor{proof}{HTML}{e1f5fe}

% Théorèmes :

\theoremstyle{definition}
\newtheorem{theorem}{Théorème}

\newtheorem{property}[theorem]{Propriété}
\newtheorem{proposition}[theorem]{Proposition}
\newtheorem{lemma}[theorem]{Lemme}
\newtheorem{corollary}[theorem]{Corollaire}

\newtheorem{definition}[theorem]{Définition}
\newtheorem{notation}[theorem]{Notation}

\newtheorem{example}[theorem]{Exemple}
\newtheorem{cexample}[theorem]{Contre-exemple}
\newtheorem{application}[theorem]{Application}

\theoremstyle{remark}
\newtheorem{remark}[theorem]{Remarque}

\counterwithin*{theorem}{section}

\newcommand{\applystyletotheorem}[1]{
	\tcolorboxenvironment{#1}{
		enhanced,
		breakable,
		colback=#1!98!white,
		boxrule=0pt,
		boxsep=0pt,
		left=8pt,
		right=8pt,
		top=8pt,
		bottom=8pt,
		sharp corners,
		after=\par,
	}
}

\applystyletotheorem{property}
\applystyletotheorem{proposition}
\applystyletotheorem{lemma}
\applystyletotheorem{theorem}
\applystyletotheorem{corollary}
\applystyletotheorem{definition}
\applystyletotheorem{notation}
\applystyletotheorem{example}
\applystyletotheorem{cexample}
\applystyletotheorem{application}
\applystyletotheorem{remark}
\applystyletotheorem{proof}

% Environnements :

\NewEnviron{whitetabularx}[1]{%
	\renewcommand{\arraystretch}{2.5}
	\colorbox{white}{%
		\begin{tabularx}{\textwidth}{#1}%
			\BODY%
		\end{tabularx}%
	}%
}

% Maths :

\DeclareFontEncoding{FMS}{}{}
\DeclareFontSubstitution{FMS}{futm}{m}{n}
\DeclareFontEncoding{FMX}{}{}
\DeclareFontSubstitution{FMX}{futm}{m}{n}
\DeclareSymbolFont{fouriersymbols}{FMS}{futm}{m}{n}
\DeclareSymbolFont{fourierlargesymbols}{FMX}{futm}{m}{n}
\DeclareMathDelimiter{\VERT}{\mathord}{fouriersymbols}{152}{fourierlargesymbols}{147}


% Bibliographie :

\addbibresource{\bibliographypath}%
\defbibheading{bibliography}[\bibname]{%
	\newpage
	\section*{#1}%
}
\renewbibmacro*{entryhead:full}{\printfield{labeltitle}}%
\DeclareFieldFormat{url}{\newline\footnotesize\url{#1}}%

\AtEndDocument{\printbibliography}

\begin{document}
	%<*content>
	\lesson{algebra}{123}{Corps finis. Applications.}

	Soient $p$ un nombre premier, $n$ un nombre entier, et $q = p^n$.
	
	\subsection{Construction}
	
	\subsubsection{Caractéristique, sous-corps premier}
	
	\reference[GOZ]{7}
	
	\begin{definition}
		Soit $A$ un anneau. L'application
		\[
		f_A :
		\begin{array}{ccc}
			\mathbb{Z} &\rightarrow& A \\
			n &\mapsto& \underbrace{1 + \dots + 1}_{n \text{ fois}}
		\end{array}
		\]
		On note $\operatorname{car}(A)$ l'unique $n \in \mathbb{N}$ tel que $\operatorname{Ker}(f_A) = n\mathbb{Z}$ : c'est la \textbf{caractéristique} de $A$.
	\end{definition}
	
	\begin{example}
		La caractéristique de l'anneau $\mathbb{Z}/n\mathbb{Z}$ est $n$.
	\end{example}
	
	\begin{proposition}
		\begin{enumerate}[label=(\roman*)]
			\item \label{121-1} Soit $A$ un anneau intègre. Alors, $\operatorname{car}(A) = 0 \text{ ou } p$ avec $p$ premier.
			\item Soit $A$ un anneau fini. Alors, $\operatorname{car}(A) \neq 0$ et $\operatorname{car}(A) \mid |A|$.
			\item Un anneau et un quelconque de ses sous-anneaux ont la même caractéristique.
		\end{enumerate}
	\end{proposition}
	
	\begin{remark}
		\begin{itemize}
			\item Le \cref{121-1} est en particulier vrai pour un corps.
			\item Si $\operatorname{car}(A) = 0$, $A$ est infini.
		\end{itemize}
	\end{remark}
	
	\begin{definition}
		Soit $\mathbb{K}$ un corps.
		\begin{itemize}
			\item $\mathbb{K}$ est dit \textbf{premier} s'il n'a pas d'autre sous-corps que lui-même.
			\item Le \textbf{sous-corps premier} de $\mathbb{K}$ est le sous-corps de $\mathbb{K}$ engendré par $1$ (ie. l'intersection de tous les sous-corps de $\mathbb{K}$) : c'est un corps premier.
		\end{itemize}
	\end{definition}
	
	\begin{remark}
		Un corps et l'un de ses sous-corps ont le même sous-corps premier.
	\end{remark}
	
	\begin{proposition}
		Soient $\mathbb{K}$ un corps et $\mathbb{P}$ son corps premier. Alors, si $\operatorname{car}(\mathbb{K}) = 0$, $\mathbb{P} \cong \mathbb{Q}$.
	\end{proposition}
	
	\newpage
	
	\subsubsection{Construction de \texorpdfstring{$\mathbb{F}_p$}{Fp}}
	
	\reference{3}
	
	\begin{proposition}
		Les conditions suivantes sont équivalentes :
		\begin{enumerate}[label=(\roman*)]
			\item $n$ est un nombre premier.
			\item $\mathbb{Z}/n\mathbb{Z}$ est un anneau intègre.
			\item $\mathbb{Z}/n\mathbb{Z}$ est un corps.
		\end{enumerate}
	\end{proposition}
	
	\begin{notation}
		On note $\mathbb{F}_p = \mathbb{Z}/p\mathbb{Z}$.
	\end{notation}
	
	\reference{81}
	
	\begin{proposition}
		Soit $\mathbb{K}$ un corps fini.
		\begin{enumerate}[label=(\roman*)]
			\item $\operatorname{car}(\mathbb{K})$ est un nombre premier $p$.
			\item Le sous-corps premier de $\mathbb{K}$ est isomorphe à $\mathbb{F}_p$.
			\item $\vert \mathbb{K} \vert = p^m$ pour $m \geq 2$.
		\end{enumerate}
	\end{proposition}
		
	\begin{example}
		\begin{itemize}
			\item Il n'existe pas de corps fini à $6$ éléments.
			\item $\mathbb{F}_p(X)$, est un corps infini de caractéristique $p$.
		\end{itemize}
	\end{example}
	
	\reference{8}
	
	\begin{proposition}
		Tout corps fini à $p$ éléments est isomorphe à $\mathbb{F}_p$.
	\end{proposition}
	
	\subsubsection{Construction de \texorpdfstring{$\mathbb{F}_q$}{Fq}}
	
	\reference{85}
	
	\begin{proposition}
		Soit $\mathbb{K}$ un corps de caractéristique $p$. L'application
		\[
		\operatorname{Frob} :
		\begin{array}{ccc}
			\mathbb{K} &\rightarrow& \mathbb{K} \\
			x &\mapsto& x^p
		\end{array}
		\]
		est un morphisme de corps.
		\begin{enumerate}[label=(\roman*)]
			\item Si $\mathbb{K}$ est fini, c'est un automorphisme.
			\item Si $\mathbb{K} = \mathbb{F}_p$, c'est l'identité.
		\end{enumerate}
	\end{proposition}
	
	\begin{corollary}
		Dans un corps fini de caractéristique $p$, chaque élément admet exactement une racine $p$-ième.
	\end{corollary}
	
	\begin{application}[Petit théorème de Fermat]
		\[ \forall x \in \mathbb{Z}, \, x^p \equiv x \mod p \]
	\end{application}
	
	\begin{theorem}
		\begin{enumerate}[label=(\roman*)]
			\item Il existe un corps $\mathbb{K}$ à $q$ éléments : c'est le corps de décomposition de $X^q - X$ sur $\mathbb{F}_p$.
			\item $\mathbb{K}$ est unique à isomorphisme près : on le note $\mathbb{F}_q$.
		\end{enumerate}
	\end{theorem}
	
	\begin{corollary}
		Le produit des éléments de $\mathbb{F}_q^*$ vaut $-1$.
	\end{corollary}
	
	\begin{application}[Théorème de Wilson]
		Soit $n \geq 2$ un entier. Alors,
		\[ n \text{ est premier} \iff (n-1)!+1 \equiv 0 \mod n \]
	\end{application}
	
	\subsection{Propriétés}
	
	\subsubsection{Commutativité}
	
	\reference{67}
	
	\begin{definition}
		L'ensemble des générateurs de $\mu_n$, noté $\mu_n^*$, est formé des \textbf{racines primitives $n$-ièmes de l'unité}.
	\end{definition}
	
	\begin{proposition}
		\begin{enumerate}[label=(\roman*)]
			\item $\mu_n^* = \{ e^{\frac{2ik\pi}{n}} \mid k \in \llbracket 0, n-1 \rrbracket, \, \operatorname{pgcd}(k, m) = 1 \}$.
			\item $\vert \mu_n^* \vert = \varphi(n)$, où $\varphi$ désigne l'indicatrice d'Euler.
		\end{enumerate}
	\end{proposition}
	
	\begin{definition}
		On appelle \textbf{$n$-ième polynôme cyclotomique} le polynôme
		\[ \Phi_n = \prod_{\xi \in \mu_n^*} (X - \xi) \]
	\end{definition}
	
	\begin{theorem}
		\begin{enumerate}[label=(\roman*)]
			\item $X^n - 1 = \prod_{d \mid n} \Phi_d$.
			\item $\Phi_n \in \mathbb{Z}[X]$.
			\item $\Phi_n$ est irréductible sur $\mathbb{Q}$.
		\end{enumerate}
	\end{theorem}
	
	\dev{theoreme-de-wedderburn}
	
	\reference[GOU21]{100}
	
	\begin{theorem}[Wedderburn]
		Tout corps fini est commutatif.
	\end{theorem}
	
	\subsubsection{Sous-corps}
	
	\reference[ULM18]{122}
	
	\begin{theorem}
		Tout sous-corps de $\mathbb{F}_q$ est de cardinal $p^d$ avec $d \mid n$. Réciproquement, pour tout $d \mid n$, $\mathbb{F}_q$ admet un unique sous-corps de cardinal $p^d$.
	\end{theorem}
	
	\begin{example}
		Les sous-corps de $\mathbb{F}_{2^{12}}$ sont $\mathbb{F}_{2^6}$, $\mathbb{F}_{2^4}$, $\mathbb{F}_{2^3}$, $\mathbb{F}_{2^2}$ et $\mathbb{F}_{2}$.
	\end{example}
	
	\begin{corollary}
		Le polynôme $X^q - X \in \mathbb{F}_p[X]$ est produit de tous les polynômes irréductibles unitaires de $\mathbb{F}_p[X]$ dont le degré divise $n$.
	\end{corollary}
	
	\begin{corollary}
		Il existe des polynômes irréductibles de tout degré dans $\mathbb{F}_q[X]$.
	\end{corollary}
	
	\begin{corollary}
		Un corps de rupture d'un polynôme irréductible de $\mathbb{F}_q[X]$ sur $\mathbb{F}_q$ est aussi un corps de décomposition pour ce polynôme sur $\mathbb{F}_q$.
	\end{corollary}
	
	\subsubsection{Groupe multiplicatif}
	
	\reference[GOZ]{83}
	
	\begin{theorem}
		Tout sous-groupe fini du groupe multiplicatif d'un corps commutatif est cyclique.
	\end{theorem}
	
	\begin{corollary}
		Le groupe multiplicatif d'un corps fini est cyclique.
	\end{corollary}
	
	\begin{corollary}
		\[ \mathbb{F}_q^* \cong \mathbb{Z}/(q-1)\mathbb{Z} \]
	\end{corollary}
	
	\subsubsection{Groupe des automorphismes}
	
	\begin{theorem}
		Le groupe des automorphismes de $\mathbb{F}_q$ est cyclique, engendré par $\operatorname{Frob}$, et d'ordre $n$.
	\end{theorem}
	
	\begin{proposition}
		Pour chaque application $f : \mathbb{F}_q \rightarrow \mathbb{F}_q$, il existe un unique polynôme $P \in \mathbb{F}_q[X]$ de degré inférieur ou égal à $q-1$ tel que
		\[ P = \sum_{u \in \mathbb{F}_q} f(u) (1 - (X-u)^{q-1}) \]
	\end{proposition}
	
	\begin{proposition}
		Les sous-groupes additifs de $\mathbb{F}_q$ sous les sous-$\mathbb{F}_q$-espaces vectoriels. Ils sont au nombre de
		\[ \sum_{s=0}^{n} \frac{(p^n-1) (p^{n-1}-1) \dots (p^{n-s+1}-1)}{(p^s-1) (p^{s-1}-1) \dots (p-1)} \]
	\end{proposition}
	
	\subsubsection{Carrés}
	
	\reference{93}
	
	\begin{proposition}
		On note $\mathbb{F}_q^2 = \{ x^2 \mid x \in \mathbb{F}_q \}$ et $\mathbb{F}_q^{*2} = \mathbb{F}_q^2 \, \cap \, \mathbb{F}_q^*$. Alors $\mathbb{F}_q^{*2}$ est un sous-groupe de $\mathbb{F}_q^*$.
	\end{proposition}
	
	\begin{proposition}
		\begin{enumerate}[label=(\roman*)]
			\item Si $p = 2$, $\mathbb{F}_q^2 = \mathbb{F}_q$, donc $\mathbb{F}_q^{*2} = \mathbb{F}_q^*$.
			\item Si $p > 2$, alors :
			\begin{itemize}
				\item $\mathbb{F}_q^{*2}$ est le noyau de l'endomorphisme de $\mathbb{F}_q^*$ défini par $x \mapsto x^{\frac{q-1}{2}}$.
				\item $\mathbb{F}_q^{*2}$ est un sous-groupe d'indice $2$ de $\mathbb{F}_q^*$.
				\item $\vert \mathbb{F}_q^{*2} \vert = \frac{q-1}{2}$ et $\vert \mathbb{F}_q^2 \vert = \frac{q+1}{2}$.
				\item $(-1) \in \mathbb{F}_q^{*2} \iff q \equiv 1 \mod 4$.
			\end{itemize}
		\end{enumerate}
	\end{proposition}
	
	\reference{155}
	
	On suppose, pour la suite de cette sous-section, $p > 2$.
	
	\begin{definition}
		On définit le \textbf{symbole de Legendre} $\left( \frac{x}{p} \right)$ pour $x \in \mathbb{F}_p^*$ par :
		\[ \left( \frac{x}{p} \right) = \pm 1 \text{ avec } \left( \frac{x}{p} \right) = 1 \iff x \in \mathbb{F}_p^{*2} \]
	\end{definition}
	
	\begin{proposition}
		$x \mapsto \left( \frac{x}{p} \right)$ est un morphisme de groupes non constant et,
		\[ \forall x \in \mathbb{F}_p^{*2}, \, \left( \frac{x}{p} \right) = x^{\frac{p-1}{2}} \]
	\end{proposition}
	
	\begin{theorem}[Loi de réciprocité quadratique]
		Soit $q \neq p$ un premier impair. Alors,
		\[ \left( \frac{p}{q} \right) \left( \frac{q}{p} \right) = (-1)^{\frac{(p-1)(q-1)}{4}} \]
	\end{theorem}
	
	\begin{remark}
		Cela signifie qu'il est équivalent d'avoir $p$ résidu quadratique modulo $q$ ou $q$ résidu quadratique modulo $p$, sauf si $p \equiv q \equiv 3 \mod 4$ auquel cas ces propositions s'excluent mutuellement.
	\end{remark}
	
	\begin{proposition}
		\[ \left( \frac{2}{p} \right) = (-1)^{\frac{(p-1)^2}{8}} \]
	\end{proposition}
	
	\begin{example}
		\[ \left( \frac{17}{41} \right) = (-1)^{8 \times 20} \left( \frac{41}{27} \right) = \left( \frac{7}{17} \right) = (-1)^{3 \times 8} \left( \frac{17}{7} \right) = \left( \frac{3}{7} \right) = (-1)^{3} \left( \frac{7}{3} \right) = - \left( \frac{1}{3} \right) = -1 \]
	\end{example}
	
	\subsection{Application}
	
	\subsubsection{Irréductibilité de polynômes}
	
	\reference{57}
	
	\begin{theorem}
		Soit $P \in \mathbb{K}[X]$ un polynôme irréductible sur un corps $\mathbb{K}$.
		\begin{itemize}
			\item Il existe un corps de rupture de $P$.
			\item Si $\mathbb{L} = \mathbb{K}[\alpha]$ et $\mathbb{L}' = \mathbb{K}[\beta]$ sont deux corps de rupture de $P$, alors il existe un unique $\mathbb{K}$-isomorphisme $\varphi : \mathbb{L} \rightarrow \mathbb{L}'$ tel que $\varphi(\alpha) = \beta$.
			\item $\mathbb{K}[X]/(P)$ est un corps de rupture de $P$.
		\end{itemize}
	\end{theorem}
	
	\reference{10}
	
	\begin{lemma}[Gauss]
		\begin{enumerate}[label=(\roman*)]
			\item Le produit de deux polynômes primitifs est primitif (ie. dont le pgcd des coefficients est égal à $1$).
			\item $\forall P, Q \in \mathbb{Z}[X] \setminus \{ 0 \}$, $\gamma(PQ) = \gamma(P) \gamma(Q)$ (où $\gamma(P)$ est le contenu du polynôme $P$).
		\end{enumerate}
	\end{lemma}
	
	\begin{theorem}[Critère d'Eisenstein]
		Soit $P = \sum_{i=0}^n a_i X^i \in \mathbb{Z}[X]$ de degré $n \geq 1$. On suppose qu'il existe $p$ premier tel que :
		\begin{enumerate}[label=(\roman*)]
			\item $p \mid a_i$, $\forall i \in \llbracket 0, n-1 \rrbracket$.
			\item $p \nmid a_n$.
			\item $p^2 \nmid a_0$.
		\end{enumerate}
		Alors $P$ est irréductible dans $\mathbb{Q}[X]$.
	\end{theorem}
	
	\reference[PER]{67}
	
	\begin{application}
		Soit $n \in \mathbb{N}^*$. Il existe des polynômes irréductibles de degré $n$ sur $\mathbb{Z}$.
	\end{application}
	
	\reference[GOZ]{12}
	
	\begin{theorem}[Critère d'irréductibilité modulo $p$]
		Soit $P = \sum_{i=0}^n a_i X^i \in \mathbb{Z}[X]$ de degré $n \geq 1$. Soit $p$ un premier. On suppose $p \nmid a_n$.
		\newpar
		Si $\overline{P}$ est irréductible dans $(\mathbb{Z}/p\mathbb{Z})[X]$, alors $P$ est irréductible dans $\mathbb{Q}[X]$.
	\end{theorem}
	
	\begin{example}
		Le polynôme $X^3-127X^2+3608X+19$ est irréductible dans $\mathbb{Z}[X]$.
	\end{example}
	
	\newpage
	
	\subsubsection{Entiers sommes de deux carrés}
	
	\reference[I-P]{137}
	
	\begin{notation}
		On note \[ N :
		\begin{array}{ccc}
			\mathbb{Z}[i] &\rightarrow& \mathbb{N} \\
			a+ib &\mapsto& a^2 + b^2
		\end{array}
		\] et $\Sigma$ l'ensemble des entiers qui sont somme de deux carrés.
	\end{notation}
	
	\begin{remark}
		$n \in \Sigma \iff \exists z \in \mathbb{Z}[i] \text{ tel que } N(z)=n$.
	\end{remark}
	
	\begin{theorem}[Deux carrés de Fermat]
		Soit $n \in \mathbb{N}^*$. Alors $n \in \Sigma$ si et seulement si $v_p(n)$ est pair pour tout $p$ premier tel que $p \equiv 3 \mod 4$ (où $v_p(n)$ désigne la valuation $p$-adique de $n$).
	\end{theorem}
	
	\subsubsection{En algèbre linéaire}
	
	\reference[I-P]{203}
	
	\begin{lemma}
		Soient $p \geq 3$ un nombre premier et $V$ un espace vectoriel sur $\mathbb{F}_p$ de dimension finie. Les dilatations engendrent $\mathrm{GL}(V)$.
	\end{lemma}
	
	\dev{theoreme-de-frobenius-zolotarev}
	
	\begin{theorem}[Frobenius-Zolotarev]
		Soient $p \geq 3$ un nombre premier et $V$ un espace vectoriel sur $\mathbb{F}_p$ de dimension finie.
		\[ \forall u \in \mathrm{GL}(V), \, \epsilon(u) = \left( \frac{\det(u)}{p} \right) \]
		où $u$ est vu comme une permutation des éléments de $V$.
	\end{theorem}
	
	\reference[ULM21]{124}
	
	On se place pour la suite de cette sous-section dans le cadre d'un espace vectoriel $E$ de dimension $m$ sur le corps $\mathbb{F}_q$.
	
	\begin{proposition}
		Les groupes précédents sont finis, et :
		\begin{enumerate}[label=(\roman*)]
			\item $|\mathrm{GL}(E)| = q^{\frac{m(m-1)}{2}}((q^m-1) \dots (q-1))$.
			\item $|\mathrm{PGL}(E)| = |\mathrm{SL}(E)| = \frac{|\mathrm{GL}(E)|}{q-1}$.
			\item $|\mathrm{PSL}(E)| = |\mathrm{SL}(E)| = \frac{|\mathrm{GL}(E)|}{(q-1)\operatorname{pgcd}(m,q-1)}$.
		\end{enumerate}
	\end{proposition}
	
	\reference[ROM21]{157}
	
	\begin{application}
		Pour tout entier $p \in \llbracket 1, m \rrbracket$, il y a
		\[ \frac{\prod_{k=m-(p-1)}^{n} (q^k - 1)}{\prod_{k=1}^{p} (q^k - 1)} \]
		sous-espaces vectoriels de dimension $p$ dans $E$.
	\end{application}
	
	\subsubsection{Codes correcteurs}
	
	\reference[BMP]{190}
	
	\begin{definition}
		On appelle :
		\begin{itemize}
			\item \textbf{Mot} un vecteur à coefficients dans $\mathbb{F}_q$.
			\item \textbf{Code correcteur} de taille $m$ un sous-ensemble de $\mathbb{F}_q^m$.
			\item \textbf{Code linéaire} de taille $m$ et de dimension $r$ un sous-espace vectoriel de dimension $r$ de $\mathbb{F}_q^m$.
			\item \textbf{Code cyclique} de taille $m$, un code linéaire stable par décalage circulaire.
		\end{itemize}
	\end{definition}
	
	\begin{example}
		Soit un code linéaire $\mathcal{C}$ de taille $m$ et de dimension $r$. On peut décrire $\mathcal{C}$ avec une matrice $G \in \mathcal{M}_{m \times r}(\mathbb{F}_q)$, dont les colonnes forment une base de $\mathcal{C}$, de la manière suivante :
		\[ \mathcal{C} = \{ Gx \mid x \in \mathbb{F}_q^m \} \]
		$G$ est la \textbf{matrice génératrice} de $\mathcal{C}$. Le codage consiste alors à transformer un mot $m$ du message d'origine en un mot $c \in \mathcal{C}$.
	\end{example}
	
	\begin{definition}
		\begin{itemize}
			\item Le \textbf{poids} d'un mot $x \in \mathbb{F}_q^m$, noté $\omega(x)$ est le nombre de coefficients non nuls de $x$.
			\item La \textbf{distance de Hamming} entre deux mots $x, y \in \mathbb{F}_q^m$, est définie par $d_H(x,y) = \omega(x-y)$.
		\end{itemize}
	\end{definition}
	
	Cette distance permet de mesurer la qualité d'un code comme l'atteste la remarque ci-dessous.
	
	\begin{remark}
		$d_H$ est une distance, elle quantifie la notion de ``mot le plus proche''.
	\end{remark}
	
	\begin{definition}
		Un code $\mathcal{C}$ est dit $t$-correcteur si les boules de centre un mot du code et de rayon $t$ (pour $d_H$) sont disjointes : les mots de $\mathcal{C}$ sont à une distance d'au moins $2t+1$ les uns des autres.
	\end{definition}
	
	\begin{proposition}
		Soit $\mathcal{C}$ un code correcteur. On note $d$ la \textbf{distance minimale} de $\mathcal{C}$ :
		\[ d = \min_{\substack{x, y \in \mathcal{C} \\ x \neq y}} \{ d_H(x,y) \} \]
		Alors $\mathcal{C}$ est $t$-correcteur si et seulement si $d \geq 2t+1$.
	\end{proposition}
	
	\begin{example}
		On considère le code $\mathcal{C}$ de taille $7$ et de dimension $4$ sur $\mathbb{F}_2$ dont la matrice génératrice est
		\[
			G =
			\begin{pmatrix}
				1 & 0 & 0 & 0 \\
				1 & 1 & 0 & 0 \\
				0 & 1 & 1 & 0 \\
				1 & 0 & 1 & 1 \\
				0 & 1 & 0 & 1 \\
				0 & 0 & 1 & 0 \\
				0 & 0 & 0 & 1
			\end{pmatrix}
		\]
		$\mathcal{C}$ est un code linéaire, dont chacun des mots non nuls est de poids supérieur à $3$ : il est $1$-correcteur.
	\end{example}
	
	\begin{proposition}[Borne de Singleton]
		Soit $\mathcal{C}$ un code linéaire de longueur $m$, de dimension $r$ et de distance minimale $d$. Alors,
		\[ d = \min_{x \in \mathcal{C} \setminus \{ 0 \}} \{ \omega(x) \} \leq m+1-r \]
	\end{proposition}

	\newpage
	\subsection*{Annexes}
	
	\reference[ULM18]{122}
	
	\begin{figure}[H]
		\begin{center}
			\begin{tikzpicture}[scale=1.5]
				\node[draw=none](F) at (0,3) {$\mathbb{F}_{2^{12}}$};
				\node[draw=none](E) at (1,2) {$\mathbb{F}_{2^{4}}$};
				\node[draw=none](D) at (-1,2) {$\mathbb{F}_{2^{6}}$};
				\node[draw=none](C) at (-1,1) {$\mathbb{F}_{2^{3}}$};
				\node[draw=none](B) at (0,1) {$\mathbb{F}_{2^{2}}$};
				\node[draw=none](A) at (0,0) {$\mathbb{F}_2$};
				\draw (A.north) -- (B.south);
				\draw (A.north west) -- (C.south east);
				\draw (C.north) -- (D.south);
				\draw (B.north west) -- (D.south east);
				\draw (B.north east) -- (E.south west);
				\draw (D.north east) -- (F.south west);
				\draw (E.north west) -- (F.south east);
			\end{tikzpicture}
		\end{center}
		\caption{Sous-corps de $\mathbb{F}_{2^{12}}$}
	\end{figure}
	%</content>
\end{document}
