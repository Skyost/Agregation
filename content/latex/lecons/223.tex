\documentclass[12pt, a4paper]{report}

% LuaLaTeX :

\RequirePackage{iftex}
\RequireLuaTeX

% Packages :

\usepackage[french]{babel}
%\usepackage[utf8]{inputenc}
%\usepackage[T1]{fontenc}
\usepackage[pdfencoding=auto, pdfauthor={Hugo Delaunay}, pdfsubject={Mathématiques}, pdfcreator={agreg.skyost.eu}]{hyperref}
\usepackage{amsmath}
\usepackage{amsthm}
%\usepackage{amssymb}
\usepackage{stmaryrd}
\usepackage{tikz}
\usepackage{tkz-euclide}
\usepackage{fourier-otf}
\usepackage{fontspec}
\usepackage{titlesec}
\usepackage{fancyhdr}
\usepackage{catchfilebetweentags}
\usepackage[french, capitalise, noabbrev]{cleveref}
\usepackage[fit, breakall]{truncate}
\usepackage[top=2.5cm, right=2cm, bottom=2.5cm, left=2cm]{geometry}
\usepackage{enumerate}
\usepackage{tocloft}
\usepackage{microtype}
%\usepackage{mdframed}
%\usepackage{thmtools}
\usepackage{xcolor}
\usepackage{tabularx}
\usepackage{aligned-overset}
\usepackage[subpreambles=true]{standalone}
\usepackage{environ}
\usepackage[normalem]{ulem}
\usepackage{marginnote}
\usepackage{etoolbox}
\usepackage{setspace}
\usepackage[bibstyle=reading, citestyle=draft]{biblatex}
\usepackage{xpatch}
\usepackage[many, breakable]{tcolorbox}
\usepackage[backgroundcolor=white, bordercolor=white, textsize=small]{todonotes}

% Bibliographie :

\newcommand{\overridebibliographypath}[1]{\providecommand{\bibliographypath}{#1}}
\overridebibliographypath{../bibliography.bib}
\addbibresource{\bibliographypath}
\defbibheading{bibliography}[\bibname]{%
	\newpage
	\section*{#1}%
}
\renewbibmacro*{entryhead:full}{\printfield{labeltitle}}
\DeclareFieldFormat{url}{\newline\footnotesize\url{#1}}
\AtEndDocument{\printbibliography}

% Police :

\setmathfont{Erewhon Math}

% Tikz :

\usetikzlibrary{calc}

% Longueurs :

\setlength{\parindent}{0pt}
\setlength{\headheight}{15pt}
\setlength{\fboxsep}{0pt}
\titlespacing*{\chapter}{0pt}{-20pt}{10pt}
\setlength{\marginparwidth}{1.5cm}
\setstretch{1.1}

% Métadonnées :

\author{agreg.skyost.eu}
\date{\today}

% Titres :

\setcounter{secnumdepth}{3}

\renewcommand{\thechapter}{\Roman{chapter}}
\renewcommand{\thesubsection}{\Roman{subsection}}
\renewcommand{\thesubsubsection}{\arabic{subsubsection}}
\renewcommand{\theparagraph}{\alph{paragraph}}

\titleformat{\chapter}{\huge\bfseries}{\thechapter}{20pt}{\huge\bfseries}
\titleformat*{\section}{\LARGE\bfseries}
\titleformat{\subsection}{\Large\bfseries}{\thesubsection \, - \,}{0pt}{\Large\bfseries}
\titleformat{\subsubsection}{\large\bfseries}{\thesubsubsection. \,}{0pt}{\large\bfseries}
\titleformat{\paragraph}{\bfseries}{\theparagraph. \,}{0pt}{\bfseries}

\setcounter{secnumdepth}{4}

% Table des matières :

\renewcommand{\cftsecleader}{\cftdotfill{\cftdotsep}}
\addtolength{\cftsecnumwidth}{10pt}

% Redéfinition des commandes :

\renewcommand*\thesection{\arabic{section}}
\renewcommand{\ker}{\mathrm{Ker}}

% Nouvelles commandes :

\newcommand{\website}{https://agreg.skyost.eu}

\newcommand{\tr}[1]{\mathstrut ^t #1}
\newcommand{\im}{\mathrm{Im}}
\newcommand{\rang}{\operatorname{rang}}
\newcommand{\trace}{\operatorname{trace}}
\newcommand{\id}{\operatorname{id}}
\newcommand{\stab}{\operatorname{Stab}}

\providecommand{\newpar}{\\[\medskipamount]}

\providecommand{\lesson}[3]{%
	\title{#3}%
	\hypersetup{pdftitle={#3}}%
	\setcounter{section}{\numexpr #2 - 1}%
	\section{#3}%
	\fancyhead[R]{\truncate{0.73\textwidth}{#2 : #3}}%
}

\providecommand{\development}[3]{%
	\title{#3}%
	\hypersetup{pdftitle={#3}}%
	\section*{#3}%
	\fancyhead[R]{\truncate{0.73\textwidth}{#3}}%
}

\providecommand{\summary}[1]{%
	\textit{#1}%
	\medskip%
}

\tikzset{notestyleraw/.append style={inner sep=0pt, rounded corners=0pt, align=center}}

%\newcommand{\booklink}[1]{\website/bibliographie\##1}
\newcommand{\citelink}[2]{\hyperlink{cite.\therefsection @#1}{#2}}
\newcommand{\previousreference}{}
\providecommand{\reference}[2][]{%
	\notblank{#1}{\renewcommand{\previousreference}{#1}}{}%
	\todo[noline]{%
		\protect\vspace{16pt}%
		\protect\par%
		\protect\notblank{#1}{\cite{[\previousreference]}\\}{}%
		\protect\citelink{\previousreference}{p. #2}%
	}%
}

\definecolor{devcolor}{HTML}{00695c}
\newcommand{\dev}[1]{%
	\reversemarginpar%
	\todo[noline]{
		\protect\vspace{16pt}%
		\protect\par%
		\bfseries\color{devcolor}\href{\website/developpements/#1}{DEV}
	}%
	\normalmarginpar%
}

% En-têtes :

\pagestyle{fancy}
\fancyhead[L]{\truncate{0.23\textwidth}{\thepage}}
\fancyfoot[C]{\scriptsize \href{\website}{\texttt{agreg.skyost.eu}}}

% Couleurs :

\definecolor{property}{HTML}{fffde7}
\definecolor{proposition}{HTML}{fff8e1}
\definecolor{lemma}{HTML}{fff3e0}
\definecolor{theorem}{HTML}{fce4f2}
\definecolor{corollary}{HTML}{ffebee}
\definecolor{definition}{HTML}{ede7f6}
\definecolor{notation}{HTML}{f3e5f5}
\definecolor{example}{HTML}{e0f7fa}
\definecolor{cexample}{HTML}{efebe9}
\definecolor{application}{HTML}{e0f2f1}
\definecolor{remark}{HTML}{e8f5e9}
\definecolor{proof}{HTML}{e1f5fe}

% Théorèmes :

\theoremstyle{definition}
\newtheorem{theorem}{Théorème}

\newtheorem{property}[theorem]{Propriété}
\newtheorem{proposition}[theorem]{Proposition}
\newtheorem{lemma}[theorem]{Lemme}
\newtheorem{corollary}[theorem]{Corollaire}

\newtheorem{definition}[theorem]{Définition}
\newtheorem{notation}[theorem]{Notation}

\newtheorem{example}[theorem]{Exemple}
\newtheorem{cexample}[theorem]{Contre-exemple}
\newtheorem{application}[theorem]{Application}

\theoremstyle{remark}
\newtheorem{remark}[theorem]{Remarque}

\counterwithin*{theorem}{section}

\newcommand{\applystyletotheorem}[1]{
	\tcolorboxenvironment{#1}{
		enhanced,
		breakable,
		colback=#1!98!white,
		boxrule=0pt,
		boxsep=0pt,
		left=8pt,
		right=8pt,
		top=8pt,
		bottom=8pt,
		sharp corners,
		after=\par,
	}
}

\applystyletotheorem{property}
\applystyletotheorem{proposition}
\applystyletotheorem{lemma}
\applystyletotheorem{theorem}
\applystyletotheorem{corollary}
\applystyletotheorem{definition}
\applystyletotheorem{notation}
\applystyletotheorem{example}
\applystyletotheorem{cexample}
\applystyletotheorem{application}
\applystyletotheorem{remark}
\applystyletotheorem{proof}

% Environnements :

\NewEnviron{whitetabularx}[1]{%
	\renewcommand{\arraystretch}{2.5}
	\colorbox{white}{%
		\begin{tabularx}{\textwidth}{#1}%
			\BODY%
		\end{tabularx}%
	}%
}

% Maths :

\DeclareFontEncoding{FMS}{}{}
\DeclareFontSubstitution{FMS}{futm}{m}{n}
\DeclareFontEncoding{FMX}{}{}
\DeclareFontSubstitution{FMX}{futm}{m}{n}
\DeclareSymbolFont{fouriersymbols}{FMS}{futm}{m}{n}
\DeclareSymbolFont{fourierlargesymbols}{FMX}{futm}{m}{n}
\DeclareMathDelimiter{\VERT}{\mathord}{fouriersymbols}{152}{fourierlargesymbols}{147}


% Bibliographie :

\addbibresource{\bibliographypath}%
\defbibheading{bibliography}[\bibname]{%
	\newpage
	\section*{#1}%
}
\renewbibmacro*{entryhead:full}{\printfield{labeltitle}}%
\DeclareFieldFormat{url}{\newline\footnotesize\url{#1}}%

\AtEndDocument{\printbibliography}

\begin{document}
	%<*content>
	\lesson{analysis}{223}{Suites numériques. Convergence, valeurs d'adhérence. Exemples et applications.}

	Dans toute la suite, $\mathbb{K}$ désignera le corps $\mathbb{R}$ ou $\mathbb{C}$.

	\subsection{Convergence des suites numériques}
	
	\subsubsection{Limite d'une suite}
	
	\reference[AMR11]{1}
	
	\begin{definition}
		Soit $E$ un ensemble non vide. On appelle \textbf{suite} à valeurs dans $E$ toute application $u: D \rightarrow E$ où $D$ est une partie de $\mathbb{N}$. Lorsque $E$ est une partie de $\mathbb{R}$ (resp. de $\mathbb{C}$), on dit que $u$ est \textbf{réelle} (resp. \textbf{complexe}). Dans ces deux cas, on parle de \textbf{suite numérique}.
	\end{definition}
	
	On fixe, pour tout le reste de la leçon, $(u_n)$ une suite numérique à coefficients dans $\mathbb{K}$.
	
	\reference{12}
	
	\begin{definition}
		\label{223-1}
		\begin{itemize}
			\item Si $\mathbb{K} = \mathbb{R}$, on dit que $(u_n)$ est \textbf{majorée} (resp. \textbf{minorée}) s'il existe $A \in \mathbb{R}$ tel que $\forall n \in \mathbb{N}$, $u_n \leq A$ (resp. $A \leq u_n$).
			\item On dit que $(u_n)$ est \textbf{bornée} s'il existe $A \in \mathbb{R}$ tel que $\forall n \in \mathbb{N}$, $\vert u_n \vert \leq A$ (resp. $A \leq u_n$). Dans le cas où $\mathbb{K} = \mathbb{R}$, cela revient à dire que $(u_n)$ est majorée et minorée.
			\item On dit que $(u_n)$ admet $\ell \in \mathbb{K}$ pour \textbf{limite} (ou \textbf{converge} / \textbf{tend} vers $\ell$) si,
			\[ \forall \epsilon > 0, \, \exists N \in \mathbb{N} \text{ tel que } \forall n \geq N, \, \vert u_n - \ell \vert < \epsilon \]
			On le note $\lim_{n \rightarrow +\infty} u_n = \ell$ ou $u_n \longrightarrow_{n \rightarrow +\infty} \ell$.
			\item On dit que $(u_n)$ est \textbf{convergente} si elle admet une limite. Sinon, on dit qu'elle est \textbf{divergente}.
		\end{itemize}
	\end{definition}
	
	\begin{example}
		Si $(u_n)$ est définie par
		\[ \forall n \geq 1, \, u_n = 1 + \frac{(-1)^n}{n} \]
		alors $(u_n)$ converge vers $1$.
	\end{example}
	
	\begin{theorem}
		On a unicité de la limite dans $\mathbb{K}$.
	\end{theorem}
	
	\begin{proposition}
		Toute suite numérique convergente est bornée.
	\end{proposition}
	
	\begin{cexample}
		$((-1)^n)$ est bornée, non convergente.
	\end{cexample}
	
	\begin{proposition}
		Soit $(v_n)$ une suite numérique bornée. On suppose $\lim{n \rightarrow +\infty} u_n = 0$. Alors $\lim_{n \rightarrow +\infty} u_n v_n = 0$.
	\end{proposition}
	
	\begin{proposition}
		On suppose $\lim{n \rightarrow +\infty} u_n = \ell_1 \in \mathbb{K}$. Soit $(v_n)$ une suite numérique qui converge vers $\ell_2 \in \mathbb{K}$. Alors :
		\begin{enumerate}[label=(\roman*)]
			\item $\lim_{n \rightarrow +\infty} u_n + v_n = \ell_1 + \ell_2$.
			\item $\lim_{n \rightarrow +\infty} \lambda u_n = \lambda \ell_1$ pour tout $\lambda \in \mathbb{K}$.
			\item $\lim_{n \rightarrow +\infty} u_n v_n = \ell_1 \ell_2$.
			\item Si $\ell_2 \neq 0$, on a $v_n \neq 0$ à partir d'un certain rang et, $\lim_{n \rightarrow +\infty} \frac{u_n}{v_n} = \frac{\ell_1}{\ell_2}$.
		\end{enumerate}
	\end{proposition}
	
	\reference{20}
	
	\begin{definition}
		On suppose $\mathbb{K} = \mathbb{R}$.
		\begin{itemize}
			\item On dit que $(u_n)$ \textbf{tend vers $+\infty$} si,
			\[ \forall A \in \mathbb{R}, \, \exists N \in \mathbb{N} \text{ tel que } \forall n \geq N, \, u_n \geq A \]
			\item On dit que $(u_n)$ \textbf{tend vers $-\infty$} si $(-u_n)$ tend vers $+\infty$.
		\end{itemize}
		On a les mêmes notations qu'à la \cref{223-1}.
	\end{definition}
	
	\begin{proposition}
		On suppose $\lim_{n \rightarrow +\infty} u_n = +\infty$.
		\begin{enumerate}[label=(\roman*)]
			\item $(u_n)$ est minorée.
			\item $(u_n)$ est strictement positive à partir d'un certain rang et $\lim_{n \rightarrow +\infty} \frac{1}{u_n} = 0$.
			\item Soit $(v_n)$ une suite numérique.
			\begin{itemize}
				\item Si $(v_n)$ est convergente ou $\lim_{n \rightarrow +\infty} v_n = +\infty$, on a $\lim_{n \rightarrow +\infty} u_n + v_n = +\infty$.
				\item Si $\lim_{n \rightarrow +\infty} v_n = +\infty$, on a $\lim_{n \rightarrow +\infty} u_n v_n = +\infty$.
			\end{itemize}
		\end{enumerate}
	\end{proposition}
	
	\reference{29}
	
	\begin{example}
		Soit $\lambda \in \mathbb{R}$. Alors,
		\[
		\lim_{n \rightarrow +\infty} \lambda n =
		\begin{cases}
			+\infty &\text{si} \lambda > 0 \\
			-\infty &\text{si} \lambda < 0 \\
			0 &\text{sinon}
		\end{cases}
		\]
	\end{example}
	
	\newpage
	\subsubsection{Convergence de suites réelles}
	
	\reference{20}
	
	Le résultat suivant justifie de se ramener au cas réel lors de l'étude de la convergence des suites numériques.
	
	\begin{proposition}
		Soient $(x_n), (y_n)$ deux suites réelles et $x, y$ deux réels. Alors,
		\[
		\lim_{n \rightarrow +\infty} x_n + iy_n = x+iy \iff
		\begin{cases}
			\lim_{n \rightarrow +\infty} x_n &= x \\
			\lim_{n \rightarrow +\infty} y_n &= y
		\end{cases}
		\]
	\end{proposition}
	
	On se place pour le restant de la sous-section dans le cas où $\mathbb{K} = \mathbb{R}$.
	
	\begin{theorem}[des gendarmes]
		Soient $(a_n)$ et $(b_n)$ deux suites réelles de même limite $\ell \in \mathbb{R}$ telles qu'à partir d'un certain rang, on ait $a_n \leq u_n \leq b_n$. Alors, $u_n \longrightarrow_{n \rightarrow +\infty} \ell$.
	\end{theorem}
	
	\begin{definition}
		$(u_n)$ est dite \textbf{croissante} (resp. \textbf{décroissante}) si pour tout entier $n$, on a $u_{n+1} \geq u_n$ (resp. $u_{n+1} \leq u_n$). Elle est dite \textbf{monotone} si elle est croissante ou décroissante.
	\end{definition}
	
	\begin{theorem}[de la limite monotone]
		Si $(u_n)$ est croissante et majorée ou décroissante et minorée, alors elle est convergente.
	\end{theorem}
	
	\begin{theorem}[Suites adjacentes]
		Si deux suites $(u_n)$ et $(v_n)$ sont adjacentes (ie. $(u_n)$ est croissante, $(v_n)$ est décroissante et la suite différence tend vers $0$), alors elles sont convergentes de même limite $\ell$ qui vérifie
		\[ \forall n \in \mathbb{N}, \, u_n \leq \ell \leq v_n \]
	\end{theorem}
	
	\begin{example}
		Les suites $\left( 1 - \frac{1}{n} \right)$ et $\left( 1 + \frac{1}{n^2} \right)$ sont adjacentes et convergent vers $1$.
	\end{example}
	
	\reference{36}
	
	\begin{corollary}[Segments emboîtés]
		Soient $(a_n)$ et $(b_n)$ deux suites réelles telles que
		\[
			\begin{cases}
				\forall n \in \mathbb{N}, \, a_n \leq b_n \\
				\forall n \in \mathbb{N}, \, [a_{n+1}, b_{n+1}] \subseteq [a_n, b_n] \\
				(b_n - a_n) \longrightarrow 0
			\end{cases}
		\]
		Alors, il existe un nombre réel unique $\ell$ tel que $\bigcap_{n \geq 0} [a_n, b_n] = \{ \ell \}$.
	\end{corollary}
	
	% TODO: critère séries alternées
	
	\reference{25}
	
	\begin{definition}
		Pour cette définition, on ne suppose pas au cas réel.
		\begin{itemize}
			\item On dit que $(u_n)$ est \textbf{négligeable} devant une suite réelle positive $(\alpha_n)$ et on note $u_n = o(\alpha_n)$ si,
			\[ \forall \epsilon > 0, \, \exists N \in \mathbb{N}, \, \text{ tel que } \forall n \geq N, \vert u_n \vert \leq \epsilon \alpha_n  \]
			\item On dit que $(u_n)$ est \textbf{équivalente} à une suite numérique $(v_n)$ et on note $u_n \sim v_n$, si $(u_n - v_n)$ est négligeable devant $(\vert u_n \vert)$.
		\end{itemize}
	\end{definition}
	
	\begin{proposition}
		En reprenant les notations précédentes,
		\begin{enumerate}[label=(\roman*)]
			\item On suppose $(\alpha_n)$ non nulle à partir d'un certain rang. $(u_n)$ est négligeable devant $\alpha_n$ si et seulement si $\frac{u_n}{\alpha_n} \longrightarrow_{n \rightarrow +\infty} 0$.
			\item On suppose $(v_n)$ non nulle à partir d'un certain rang. $(u_n)$ est équivalente à $v_n$ si et seulement si $\frac{u_n}{\alpha_n} \longrightarrow_{n \rightarrow +\infty} 1$.
			\item $\sim$ est une relation d'équivalence sur l'ensemble des suites de $\mathbb{K}$.
		\end{enumerate}
	\end{proposition}
	
	\reference{353}
	
	\begin{example}[Formule de Stirling]
		\[ n! \sim \sqrt{2n\pi} \left( \frac{n}{e} \right)^n \]
	\end{example}
	
	\reference{28}
	
	\begin{proposition}
		Deux suites convergentes équivalentes ont la même limite.
	\end{proposition}
	
	\subsubsection{Suites de Cauchy}
	
	\reference{34}
	
	\begin{definition}
		On dit que $(u_n)$ est \textbf{de Cauchy} si
		\[ \forall \epsilon > 0, \exists N \in \mathbb{N} \text{ tel que } \forall p > q \geq N, \vert u_p - u_q \vert < \epsilon \]
	\end{definition}
	
	\begin{proposition}
		\begin{enumerate}[label=(\roman*)]
			\item Une suite convergente est de Cauchy.
			\item Une suite de Cauchy est bornée.
		\end{enumerate}
	\end{proposition}
	
	\begin{theorem}
		Toute suite de Cauchy de $\mathbb{K}$ est convergente dans $\mathbb{K}$.
	\end{theorem}
	
	\reference[HAU]{312}
	
	\begin{cexample}
		La série $\sum \frac{1}{n}$ est une suite de Cauchy de $\mathbb{Q}$ non convergente dans $\mathbb{Q}$.
	\end{cexample}
	
	\newpage
	
	\subsubsection{Convergence au sens de Cesàro}
	
	\reference[AMR11]{53}
	
	\begin{definition}
		À toute suite numérique $(u_n)$ on y associe sa suite $(v_n)$ des moyennes de Cesàro où
		\[ \forall n \in \mathbb{N}, v_n = \frac{1}{n} \sum_{k=1}^{n} u_k \]
	\end{definition}
	
	\begin{theorem}
		\label{223-2}
		Si $(u_n)$ converge vers $\ell \in \mathbb{K}$, alors sa suite des moyennes de Cesàro converge vers $\ell$. On dit que $(u_n)$ converge \textbf{au sens de Cesàro}.
	\end{theorem}
	
	\begin{example}
		\begin{itemize}
			\item Soit $(v_n)$ une suite numérique dont aucun terme n'est nul, qui converge vers $\ell \neq 0$. Alors,
			\[ \frac{1}{n} \frac{1}{\sum_{k=1}^n \frac{1}{v_k}} \longrightarrow_{n \rightarrow +\infty} \frac{1}{\ell} \]
			converge vers $\frac{1}{\ell}$.
			\item Soit $(w_n)$ une suite numérique telle que $(w_{n+1} - w_n)$ converge vers $\ell \in \mathbb{K}$. Alors,
			\[ \frac{w_n}{n} \longrightarrow_{n \rightarrow +\infty} \ell \]
		\end{itemize}
	\end{example}
	
	\begin{remark}
		La réciproque du \cref{223-2} est fausse.
	\end{remark}
	
	\begin{example}
		$(-1)^n$ converge au sens de Cesàro vers $0$, mais pas au sens usuel.
	\end{example}
	
	\subsection{Valeurs d'adhérence}
	
	\subsubsection{Suites extraites}
	
	\reference{14}
	
	\begin{definition}
		On appelle \textbf{sous-suite} ou \textbf{suite extraite} de $(u_n)$, toute suite $(u_{\varphi(n)})$ où $\varphi : \mathbb{N} \rightarrow \mathbb{N}$ est strictement croissante (on dit que $\varphi$ est une \textbf{extractrice}).
	\end{definition}
	
	\begin{proposition}
		Si une suite converge vers $\ell \in \mathbb{K}$, alors toute suite extraite converge vers $\ell$.
	\end{proposition}
	
	\begin{definition}
		On appelle \textbf{valeur d'adhérence} d'une suite numérique, tout élément de $\mathbb{K}$ limite d'une de ses sous-suites convergentes.
	\end{definition}
	
	\begin{remark}
		\begin{itemize}
			\item Toute suite numérique convergente ne possède que sa limite comme valeur d'adhérence.
			\item Une suite possédant une unique valeur d'adhérence n'est pas nécessairement convergente.
		\end{itemize}
	\end{remark}
	
	\begin{example}
		$((1 - (-1)^n)n)$ ne possède que $0$ comme valeur d'adhérence, mais ne converge pas.
	\end{example}
	
	\reference{36}
	
	\begin{theorem}[Bolzano-Weierstrass]
		Toute suite numérique bornée possède au moins une sous-suite convergente.
	\end{theorem}
	
	\reference[DAN]{73}
	
	\begin{proposition}
		Une suite numérique est convergente si et seulement si elle est bornée et n'a qu'une seule valeur d'adhérence.
	\end{proposition}
	
	\reference[I-P]{116}
	\dev{connexite-des-valeurs-d-adherence-d-une-suite-dans-un-compact}
	
	\begin{application}
		Soit $(E, d)$ un espace métrique compact. Soit $(v_n)$ une suite de $E$ telle que $d(v_n,v_{n-1}) \longrightarrow 0$. Alors l'ensemble $\Gamma$ des valeurs d'adhérence de $(v_n)$ est connexe.
	\end{application}
	
	\begin{corollary}[Lemme de la grenouille]
		Soient $f : [0, 1] \rightarrow [0, 1]$ continue et $(x_n)$ une suite de $[0, 1]$ tel que
		\[ \begin{cases} x_0 \in [0, 1] \\ x_{n+1} = f(x_n) \end{cases} \]
		Alors $(x_n)$ converge si et seulement si $\lim_{n \rightarrow +\infty } x_{n+1} - x_n = 0$.
	\end{corollary}
	
	\subsubsection{Limites inférieure et supérieure}
	
	\reference{93}
	
	On se place dans le cas réel pour toute cette sous-section.
	
	\begin{lemma}
		Si $(u_n)$ n'est pas bornée, on peut extraire une sous-suite qui tend vers $\pm \infty$ : $\pm \infty$ est une valeur d'adhérence de $(u_n)$ dans $\overline{\mathbb{R}}$.
	\end{lemma}
	
	\begin{definition}
		On appelle \textbf{limite inférieure} (resp. \textbf{limite supérieure}) de $(u_n)$, notée $\limsup_{n \rightarrow +\infty} u_n$ (resp. $\liminf_{n \rightarrow +\infty} u_n$) la plus grande (resp. plus petite) de ses valeurs d'adhérence.
	\end{definition}
	
	\reference[DAN]{77}
	
	\begin{proposition}
		$(u_n)$ converge si et seulement si $\liminf_{n \rightarrow +\infty} u_n = \limsup_{n \rightarrow +\infty} u_n$.
	\end{proposition}
	
	\subsection{Suites particulières}
	
	\subsubsection{Suites récurrentes}
	
	\reference[GOU20]{200}
	
	\begin{definition}
		Soit $E \subseteq \mathbb{K}$. On dit que $(u_n)$ est \textbf{récurrente} d'ordre $h \in \mathbb{N}^*$ si on peut écrire
		\[ \forall n \geq h, \, u_{n+h} = f(u_{n-1}, \dots, u_{n-h}) \tag{$*$} \]
		où $f : E^h \rightarrow E$ et les premières valeurs $u_0, \dots, u_{h-1} \in E$ étant donnés.
	\end{definition}
	
	\reference[AMR11]{38}
	
	\begin{theorem}[Caractérisation séquentielle de la continuité]
		En reprenant les notations précédentes, une fonction $g : E \rightarrow \mathbb{K}$ est continue si et seulement si pour toute suite numérique convergente $(v_n) \in E^{\mathbb{N}}$ dont on note $\ell$ la limite, $g(v_n) \longrightarrow_{n \rightarrow +\infty} \ell$.
	\end{theorem}
	
	\begin{corollary}
		Si une suite récurrente d'ordre $1$ (dont on note $f$ la fonction) converge vers $\ell$, alors $f(\ell) = \ell$.
	\end{corollary}
	
	\begin{example}
		La suite $(u_n)$ définie par $u_0 \in \left[ -\frac{\pi}{2}, \frac{\pi}{2} \right]$ et $\forall n \geq 1, \, u_{n+1} = \sin(u_n)$ converge vers $0$.
	\end{example}
	
	\reference[ROU]{152}
	\dev{methode-de-newton}
	
	\begin{application}[Méthode de Newton]
		Soit $g : [c, d] \rightarrow \mathbb{R}$ une fonction de classe $\mathcal{C}^2$ strictement croissante sur $[c, d]$. On considère la fonction
		\[ \varphi :
		\begin{array}{ccc}
			[c, d] &\rightarrow& \mathbb{R} \\
			x &\mapsto& x - \frac{g(x)}{g'(x)}
		\end{array}
		\]
		(qui est bien définie car $g' > 0$). Alors :
		\begin{enumerate}[label=(\roman*)]
			\item $\exists! a \in [c, d]$ tel que $g(a) = 0$.
			\item $\exists \alpha > 0$ tel que $I = [a - \alpha, a + \alpha]$ est stable par $\varphi$.
			\item La suite $(x_n)$ des itérés (définie par récurrence par $x_{n+1} = \varphi(x_n)$ pour tout $n \geq 0$) converge quadratiquement vers $a$ pour tout $x_0 \in I$.
		\end{enumerate}
	\end{application}
	
	\begin{corollary}
		En reprenant les hypothèses et notations du théorème précédent, et en supposant de plus $g$ strictement convexe sur $[c, d]$, le résultat du théorème est vrai sur $I = [a, d]$. De plus :
		\begin{enumerate}[label=(\roman*)]
			\item $(x_n)$ est strictement décroissante (ou constante).
			\item $x_{n+1} - a \sim \frac{g''(a)}{g'(a)} (x_n - a)^2$ pour $x_0 > a$.
		\end{enumerate}
	\end{corollary}
	
	\begin{example}
		\begin{itemize}
			\item On fixe $y > 0$. En itérant la fonction $F : x \mapsto \frac{1}{2} \left( x + \frac{y}{x} \right)$ pour un nombre de départ compris entre $c$ et $d$ où $0 < c < d$ et $c^2 < 0 < d^2$, on peut obtenir une approximation du nombre $\sqrt{y}$.
			\item En itérant la fonction $F : x \mapsto \frac{x^2+1}{2x-1}$ pour un nombre de départ supérieur à $2$, on peut obtenir une approximation du nombre d'or $\varphi = \frac{1+\sqrt{5}}{2}$.
		\end{itemize}
	\end{example}
	
	\subsubsection{Séries numériques}
	
	\reference[GOU20]{208}
	
	\begin{definition}
		\begin{itemize}
			\item On appelle \textbf{série} de terme général $u_n$ la suite $(S_n)$ définie par
			\[ \forall n \in \mathbb{N}, \, S_n = u_0 + \dots + u_n \]
			On note cette série $\sum u_n$.
			\item $u_n$ s'appelle le \textbf{terme} d'indice $n$.
			\item $S_n$ s'appelle la \textbf{somme partielle} d'indice $n$.
		\end{itemize}
	\end{definition}
	
	\begin{definition}
		En reprenant les notations précédentes, on dit que $\sum u_n$ \textbf{converge} si la suite $(S_n)$ converge. Dans ce cas, la limite s'appelle la \textbf{somme} de la série, et on la note $\sum_{n=0}^{+\infty} u_n$.
	\end{definition}
	
	\reference[AMR11]{81}
	
	\begin{proposition}
		Si $\sum u_n$ converge, alors $\lim_{n \rightarrow +\infty} u_n = 0$.
	\end{proposition}
	
	\begin{cexample}
		La réciproque est fausse, par exemple en considérant la suite $(u_n)$ définie pour tout $n \in \mathbb{N}$ par $u_n = \ln(1 + \frac{1}{n})$, on a $\sum_{k=1}^{n} u_k = \ln(n+1) \longrightarrow_{n \rightarrow +\infty} +\infty$.
	\end{cexample}
	
	\begin{proposition}
		Muni des opérations :
		\begin{itemize}
			\item $\forall (u_n), (v_n) \in \mathbb{K}^{\mathbb{N}}, \, \sum u_n + \sum v_n = \sum (u_n + v_n)$,
			\item $\forall \lambda \in \mathbb{K}, \, \forall (u_n) \in \mathbb{K}^{\mathbb{N}}, \, \lambda \sum u_n = \sum (\lambda u_n)$,
		\end{itemize}
		l'ensemble des séries numériques est un espace vectoriel sur $\mathbb{K}$ dont l'ensemble des séries convergentes est un sous-espace vectoriel.
	\end{proposition}
	
	\reference[GOU20]{214}
	
	\begin{proposition}[Règle de d'Alembert]
		Soit $\sum u_n$ une série à termes strictement positifs telle que
		\[ \lim_{n \rightarrow +\infty} \frac{u_{n+1}}{u_n} = \lambda \in [0, +\infty] \]
		Alors :
		\begin{enumerate}[label=(\roman*)]
			\item Si $\lambda < 1$, $\sum u_n$ converge.
			\item Si $\lambda > 1$, $\sum u_n$ diverge.
		\end{enumerate}
	\end{proposition}
	
	\reference[AMR11]{94}
	
	\begin{example}
		$\sum \left( 1 - \frac{1}{n} \right)^{n^2}$ converge.
	\end{example}
	
	\reference{108}
	
	\begin{example}
		$\sum_{k=0}^{10} \frac{1}{n!}$ donne une valeur approchée de $e$ à moins de $3 \times 10^{-8}$ près par défaut.
	\end{example}
	
	\reference[GOU20]{214}
	
	\begin{proposition}[Règle de Cauchy]
		Soit $\sum u_n$ une série à termes strictement positifs telle que
		\[ \lim_{n \rightarrow +\infty} \sqrt[n]{u_n} = \lambda \in [0, +\infty] \]
		Alors :
		\begin{enumerate}[label=(\roman*)]
			\item Si $\lambda < 1$, $\sum u_n$ converge.
			\item Si $\lambda > 1$, $\sum u_n$ diverge.
		\end{enumerate}
	\end{proposition}
	
	\reference[AMR11]{112}
	
	\begin{example}
		$\sum \left( \frac{4n+1}{3n+2} \right)^{n}$ converge.
	\end{example}
	%</content>
\end{document}
