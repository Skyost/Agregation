\documentclass[12pt, a4paper]{report}

% LuaLaTeX :

\RequirePackage{iftex}
\RequireLuaTeX

% Packages :

\usepackage[french]{babel}
%\usepackage[utf8]{inputenc}
%\usepackage[T1]{fontenc}
\usepackage[pdfencoding=auto, pdfauthor={Hugo Delaunay}, pdfsubject={Mathématiques}, pdfcreator={agreg.skyost.eu}]{hyperref}
\usepackage{amsmath}
\usepackage{amsthm}
%\usepackage{amssymb}
\usepackage{stmaryrd}
\usepackage{tikz}
\usepackage{tkz-euclide}
\usepackage{fourier-otf}
\usepackage{fontspec}
\usepackage{titlesec}
\usepackage{fancyhdr}
\usepackage{catchfilebetweentags}
\usepackage[french, capitalise, noabbrev]{cleveref}
\usepackage[fit, breakall]{truncate}
\usepackage[top=2.5cm, right=2cm, bottom=2.5cm, left=2cm]{geometry}
\usepackage{enumerate}
\usepackage{tocloft}
\usepackage{microtype}
%\usepackage{mdframed}
%\usepackage{thmtools}
\usepackage{xcolor}
\usepackage{tabularx}
\usepackage{aligned-overset}
\usepackage[subpreambles=true]{standalone}
\usepackage{environ}
\usepackage[normalem]{ulem}
\usepackage{marginnote}
\usepackage{etoolbox}
\usepackage{setspace}
\usepackage[bibstyle=reading, citestyle=draft]{biblatex}
\usepackage{xpatch}
\usepackage[many, breakable]{tcolorbox}
\usepackage[backgroundcolor=white, bordercolor=white, textsize=small]{todonotes}

% Bibliographie :

\newcommand{\overridebibliographypath}[1]{\providecommand{\bibliographypath}{#1}}
\overridebibliographypath{../bibliography.bib}
\addbibresource{\bibliographypath}
\defbibheading{bibliography}[\bibname]{%
	\newpage
	\section*{#1}%
}
\renewbibmacro*{entryhead:full}{\printfield{labeltitle}}
\DeclareFieldFormat{url}{\newline\footnotesize\url{#1}}
\AtEndDocument{\printbibliography}

% Police :

\setmathfont{Erewhon Math}

% Tikz :

\usetikzlibrary{calc}

% Longueurs :

\setlength{\parindent}{0pt}
\setlength{\headheight}{15pt}
\setlength{\fboxsep}{0pt}
\titlespacing*{\chapter}{0pt}{-20pt}{10pt}
\setlength{\marginparwidth}{1.5cm}
\setstretch{1.1}

% Métadonnées :

\author{agreg.skyost.eu}
\date{\today}

% Titres :

\setcounter{secnumdepth}{3}

\renewcommand{\thechapter}{\Roman{chapter}}
\renewcommand{\thesubsection}{\Roman{subsection}}
\renewcommand{\thesubsubsection}{\arabic{subsubsection}}
\renewcommand{\theparagraph}{\alph{paragraph}}

\titleformat{\chapter}{\huge\bfseries}{\thechapter}{20pt}{\huge\bfseries}
\titleformat*{\section}{\LARGE\bfseries}
\titleformat{\subsection}{\Large\bfseries}{\thesubsection \, - \,}{0pt}{\Large\bfseries}
\titleformat{\subsubsection}{\large\bfseries}{\thesubsubsection. \,}{0pt}{\large\bfseries}
\titleformat{\paragraph}{\bfseries}{\theparagraph. \,}{0pt}{\bfseries}

\setcounter{secnumdepth}{4}

% Table des matières :

\renewcommand{\cftsecleader}{\cftdotfill{\cftdotsep}}
\addtolength{\cftsecnumwidth}{10pt}

% Redéfinition des commandes :

\renewcommand*\thesection{\arabic{section}}
\renewcommand{\ker}{\mathrm{Ker}}

% Nouvelles commandes :

\newcommand{\website}{https://agreg.skyost.eu}

\newcommand{\tr}[1]{\mathstrut ^t #1}
\newcommand{\im}{\mathrm{Im}}
\newcommand{\rang}{\operatorname{rang}}
\newcommand{\trace}{\operatorname{trace}}
\newcommand{\id}{\operatorname{id}}
\newcommand{\stab}{\operatorname{Stab}}

\providecommand{\newpar}{\\[\medskipamount]}

\providecommand{\lesson}[3]{%
	\title{#3}%
	\hypersetup{pdftitle={#3}}%
	\setcounter{section}{\numexpr #2 - 1}%
	\section{#3}%
	\fancyhead[R]{\truncate{0.73\textwidth}{#2 : #3}}%
}

\providecommand{\development}[3]{%
	\title{#3}%
	\hypersetup{pdftitle={#3}}%
	\section*{#3}%
	\fancyhead[R]{\truncate{0.73\textwidth}{#3}}%
}

\providecommand{\summary}[1]{%
	\textit{#1}%
	\medskip%
}

\tikzset{notestyleraw/.append style={inner sep=0pt, rounded corners=0pt, align=center}}

%\newcommand{\booklink}[1]{\website/bibliographie\##1}
\newcommand{\citelink}[2]{\hyperlink{cite.\therefsection @#1}{#2}}
\newcommand{\previousreference}{}
\providecommand{\reference}[2][]{%
	\notblank{#1}{\renewcommand{\previousreference}{#1}}{}%
	\todo[noline]{%
		\protect\vspace{16pt}%
		\protect\par%
		\protect\notblank{#1}{\cite{[\previousreference]}\\}{}%
		\protect\citelink{\previousreference}{p. #2}%
	}%
}

\definecolor{devcolor}{HTML}{00695c}
\newcommand{\dev}[1]{%
	\reversemarginpar%
	\todo[noline]{
		\protect\vspace{16pt}%
		\protect\par%
		\bfseries\color{devcolor}\href{\website/developpements/#1}{DEV}
	}%
	\normalmarginpar%
}

% En-têtes :

\pagestyle{fancy}
\fancyhead[L]{\truncate{0.23\textwidth}{\thepage}}
\fancyfoot[C]{\scriptsize \href{\website}{\texttt{agreg.skyost.eu}}}

% Couleurs :

\definecolor{property}{HTML}{fffde7}
\definecolor{proposition}{HTML}{fff8e1}
\definecolor{lemma}{HTML}{fff3e0}
\definecolor{theorem}{HTML}{fce4f2}
\definecolor{corollary}{HTML}{ffebee}
\definecolor{definition}{HTML}{ede7f6}
\definecolor{notation}{HTML}{f3e5f5}
\definecolor{example}{HTML}{e0f7fa}
\definecolor{cexample}{HTML}{efebe9}
\definecolor{application}{HTML}{e0f2f1}
\definecolor{remark}{HTML}{e8f5e9}
\definecolor{proof}{HTML}{e1f5fe}

% Théorèmes :

\theoremstyle{definition}
\newtheorem{theorem}{Théorème}

\newtheorem{property}[theorem]{Propriété}
\newtheorem{proposition}[theorem]{Proposition}
\newtheorem{lemma}[theorem]{Lemme}
\newtheorem{corollary}[theorem]{Corollaire}

\newtheorem{definition}[theorem]{Définition}
\newtheorem{notation}[theorem]{Notation}

\newtheorem{example}[theorem]{Exemple}
\newtheorem{cexample}[theorem]{Contre-exemple}
\newtheorem{application}[theorem]{Application}

\theoremstyle{remark}
\newtheorem{remark}[theorem]{Remarque}

\counterwithin*{theorem}{section}

\newcommand{\applystyletotheorem}[1]{
	\tcolorboxenvironment{#1}{
		enhanced,
		breakable,
		colback=#1!98!white,
		boxrule=0pt,
		boxsep=0pt,
		left=8pt,
		right=8pt,
		top=8pt,
		bottom=8pt,
		sharp corners,
		after=\par,
	}
}

\applystyletotheorem{property}
\applystyletotheorem{proposition}
\applystyletotheorem{lemma}
\applystyletotheorem{theorem}
\applystyletotheorem{corollary}
\applystyletotheorem{definition}
\applystyletotheorem{notation}
\applystyletotheorem{example}
\applystyletotheorem{cexample}
\applystyletotheorem{application}
\applystyletotheorem{remark}
\applystyletotheorem{proof}

% Environnements :

\NewEnviron{whitetabularx}[1]{%
	\renewcommand{\arraystretch}{2.5}
	\colorbox{white}{%
		\begin{tabularx}{\textwidth}{#1}%
			\BODY%
		\end{tabularx}%
	}%
}

% Maths :

\DeclareFontEncoding{FMS}{}{}
\DeclareFontSubstitution{FMS}{futm}{m}{n}
\DeclareFontEncoding{FMX}{}{}
\DeclareFontSubstitution{FMX}{futm}{m}{n}
\DeclareSymbolFont{fouriersymbols}{FMS}{futm}{m}{n}
\DeclareSymbolFont{fourierlargesymbols}{FMX}{futm}{m}{n}
\DeclareMathDelimiter{\VERT}{\mathord}{fouriersymbols}{152}{fourierlargesymbols}{147}


% Bibliographie :

\addbibresource{\bibliographypath}%
\defbibheading{bibliography}[\bibname]{%
	\newpage
	\section*{#1}%
}
\renewbibmacro*{entryhead:full}{\printfield{labeltitle}}%
\DeclareFieldFormat{url}{\newline\footnotesize\url{#1}}%

\AtEndDocument{\printbibliography}

\begin{document}
	%<*content>
	\lesson{algebra}{144}{Racines d'un polynôme. Fonctions symétriques élémentaires. Exemples et applications.}

	Soient $\mathbb{K}$ un corps commutatif et $\mathcal{A}$ une algèbre sur $\mathbb{K}$. À tout polynôme $P = \sum_{i=0}^{n} a_i X^i$ de $\mathcal{K}[X]$, on associe l'application

	\[
	\widetilde{P} :
	\begin{array}{ccc}
		\mathcal{A} &\rightarrow& \mathcal{A} \\
		x &\mapsto& \sum_{i=0}^{n} a_i x^i
	\end{array}
	\]

	L'application $P \mapsto \widetilde{P}$ est un morphisme d'algèbres. On notera abusivement $P = \widetilde{P}$ par la suite s'il n'y a pas d'ambiguïté.

	\subsection{Polynômes}

	\subsubsection{Racines}

	Soit $P \in \mathbb{K}[X]$.

	\reference[GOU21]{63}

	\begin{definition}
		Soit $\mathbb{L}$ une extension de $\mathbb{K}$ (cf. \cref{144-1}). On dit que $a \in \mathbb{L}$ est une \textbf{racine} de $P$ si $P(a) = 0$.
	\end{definition}

	\begin{proposition}
		$a \in \mathbb{K}$ est racine de $P$ si et seulement si $X - a \mid P$.
	\end{proposition}

	\begin{application}[Polynômes d'interpolation de Lagrange]
		Soient $a_1, \dots, a_n \in \mathbb{K}$ deux à deux distincts et $b_1, \dots b_n \in \mathbb{K}$. Alors
		\[ \exists! L \in \mathbb{K}[X] \text{ tel que } \forall i \in \llbracket 1, n \rrbracket, \, L(a_i) = b_i \]
	\end{application}

	\begin{definition}
		Soient $a \in \mathbb{K}$ et $h \in \mathbb{N}^*$. On dit que $a$ est \textbf{racine de $P$ d'ordre $h$} si $(X-a)^h \mid P$ mais $(X-a)^{h+1} \nmid P$.
	\end{definition}

	\begin{proposition}
		Soient $a_1, \dots, a_r \in \mathbb{K}$ des racines de $P$ distinctes deux à deux et d'ordre $h_1, \dots, h_r$. Alors, $\exists Q \in \mathbb{K}[X]$ tel que
		\[ P = (X-a_1)^{h_1} \dots (X-a_r)^{h_r} Q(X) \quad \text{ et } \quad Q(a_i) \neq 0 \forall i \in \llbracket 1, n \rrbracket \]
	\end{proposition}

	\begin{corollary}
		Si $P \in \mathbb{K}[X]$ est de degré $n \geq 1$, alors $P$ a au plus $n$ racines (comptées avec leur ordre de multiplicité).
	\end{corollary}

	\begin{cexample}
		C'est faux en général dans un anneau. Par exemple, si $P = \overline{4}X \in \mathbb{Z}/8\mathbb{Z}[X]$, alors $P$ a trois racines : $\overline{0}$, $\overline{1}$ et $\overline{4}$, mais $\deg(P) = 1$.
	\end{cexample}

	\begin{proposition}
		Si $\mathbb{K}$ est infini et $P(x) = 0$ pour tout $x \in \mathbb{K}$, alors $P = 0$.
	\end{proposition}

	\begin{cexample}
		Si $\mathbb{K} = \{ a_1, \dots, a_n \}$, le polynôme $(X-a_1) \dots (X-a_n)$ est non nul, mais son évaluation en tout élément de $\mathbb{K}$ vaut $0$.
	\end{cexample}

	\begin{definition}
		$P$ est dit \textbf{scindé sur $\mathbb{K}$} si on peut écrire
		\[ P = \lambda (X-a_1)^{h_1} \dots (X-a_r)^{h_r} \]
		avec $\lambda \in \mathbb{K}$ et pour tout $i \in \llbracket 1, n \rrbracket$, $a_i \in \mathbb{K}$ et $h_i \in \mathbb{N}^*$.
	\end{definition}

	\begin{definition}
		On appelle \textbf{polynôme dérivé} de $P$ le polynôme
		\[ P' = a_1 + 2a_2 X + \dots + na_n X^{n-1} \]
	\end{definition}

	\begin{remark}
		L'application $P \rightarrow P'$ est linéaire, et les règles de dérivation coïncident avec les règles usuelles.
	\end{remark}

	\begin{theorem}[Formule de Taylor]
		On suppose $\mathbb{K}$ de caractéristique nulle. Alors tout polynôme $F$ de degré inférieur ou égal à $n$ vérifie
		\[ \forall a \in \mathbb{K}, \, F(X) = \sum_{i=0}^{n} \frac{(X-a)^i}{i!} F^{(i)}(a) \]
	\end{theorem}

	\begin{corollary}
		On suppose $\mathbb{K}$ de caractéristique nulle et $P \neq 0$. Alors $a \in \mathbb{K}$ est racine d'ordre $h$ de $P$ si et seulement si
		\[ \forall i \in \llbracket 1, h-1 \rrbracket, P^{(i)}(a) = 0 \quad \text{ et } \quad F^{(h)}(a) \neq 0 \]
	\end{corollary}

	\begin{example}
		Le polynôme $P_n = \sum_{i=0}^{n} \frac{1}{i!} X^{i}$ n'a que des racines simples dans $\mathbb{C}$.
	\end{example}

	\begin{remark}
		C'est encore vrai en caractéristique non nulle pour $h = 1$.
	\end{remark}

	\subsubsection{Polynômes symétriques}

	Soit $A$ un anneau commutatif unitaire.

	\reference{83}

	\begin{definition}
		Soit $P \in A[X_1, \dots, X_n]$. On dit que $P$ est \textbf{symétrique} si
		\[ \forall \sigma \in S_n, \, P(X_{\sigma(1)}, \dots, X_{\sigma(n)}) = P(X_1, \dots, X_n) \]
	\end{definition}

	\begin{example}
		Dans $\mathbb{R}[X]$, le polynôme $XY + YZ + ZX$ est symétrique.
	\end{example}

	\begin{definition}
		On appelle \textbf{polynômes symétriques élémentaires} de $A[X_1, \dots, X_n]$ les polynômes noté $\Sigma_p$ où $p \in \llbracket 1, n \rrbracket$ définis par
		\[ \Sigma_p = \sum_{1 \leq i_1 < \dots < i_p \leq n} X_{i_1} \dots X_{i_p} \]
	\end{definition}

	\begin{example}
		\begin{itemize}
			\item $\Sigma_1 = X_1 + \dots + X_n$.
			\item $\Sigma_2 = \sum_{1 \leq i < j \leq n} X_i X_j$.
			\item $\Sigma_n = X_1 \dots X_n$.
		\end{itemize}
	\end{example}

	\begin{remark}
		Si $P \in A[X_1, \dots, X_n]$, alors $P(\Sigma_1(X_1, \dots, X_n), \dots, \Sigma_n(X_1, \dots, X_n))$ est symétrique. Et la réciproque est vraie.
	\end{remark}

	\begin{theorem}[Théorème fondamental des polynômes symétriques]
		Soit $P \in A[X_1, \dots, X_n]$ un polynôme symétrique. Alors,
		\[ \exists! \Phi \in A[X_1, \dots, X_n] \text{ tel que } \Phi(\Sigma_1, \dots, \Sigma_n) \]
	\end{theorem}

	\begin{example}
		$P = X^3 + Y^3 + Z^3$ s'écrit $P = \Sigma_1^3 - 3 \Sigma_1 \Sigma_2 + 3 \Sigma_3$.
	\end{example}

	\reference{64}

	\begin{application}[Relations coefficients - racines]
		Soit $P = a_0X^n + \dots + a_n \in \mathbb{K}[X]$ avec $a_0 \neq 0$ scindé sur $\mathbb{K}$, dont les racines (comptées avec leur ordre de multiplicité) sont $x_1, \dots, x_n$. Alors
		\[ \forall p \in \llbracket 1, n \rrbracket, \, \Sigma_p(x_1, \dots, x_n) = (-1)^p \frac{a_p}{a_0} \]
		En particulier,
		\begin{itemize}
			\item $\Sigma_1(x_1, \dots, x_n) = \sum_{i=1}^n x_i = -\frac{a_1}{a_0}$.
			\item $\Sigma_n(x_1, \dots, x_n) = \prod_{i=1}^n x_i = (-1)^n \frac{a_n}{a_0}$.
		\end{itemize}
	\end{application}

	\reference[I-P]{279}
	\dev{theoreme-de-kroenecker}

	\begin{application}[Théorème de Kronecker]
		Soit $P \in \mathbb{Z}[X]$ unitaire tel que toutes ses racines complexes appartiennent au disque unité épointé en l'origine (que l'on note $D$). Alors toutes ses racines sont des racines de l'unité.
	\end{application}

	\begin{corollary}
		Soit $P \in \mathbb{Z}[X]$ unitaire et irréductible sur $\mathbb{Q}$ tel que toutes ses racines complexes soient de module inférieur ou égal à $1$. Alors $P = X$ ou $P$ est un polynôme cyclotomique.
	\end{corollary}

	\reference[GOU21]{86}

	\begin{definition}
		On appelle \textbf{identités de Newton} les polynômes
		\[ S_p = \sum_{i=1}^n X_i^p \in \mathbb{R}[X] \]
	\end{definition}

	\begin{proposition}
		\begin{itemize}
			\item $\forall k \in \llbracket 1, n-1 \rrbracket, \, S_k = (-1)^{k+1} k \Sigma_k + \sum_{i=1}^{k-1} (-1)^{i+1} \Sigma_i S_{n-k+i}$.
			\item $\forall p \in \mathbb{N}, \, S_{p+n} = \sum_{i=1}^n \Sigma_i S_{p+n-i}$.
		\end{itemize}
	\end{proposition}

	\reference[C-G]{339}
	\dev{formes-de-hankel}

	\begin{application}[Formes de Hankel]
		On suppose $\mathbb{K} = \mathbb{R}$ et on note $x_1, \dots, x_t$ les racines complexes de $P$ de multiplicités respectives $m_1, \dots, m_t$. On pose
		\[ s_0 = n \text{ et } \forall k \geq 1, \, s_k = \sum_{i=1}^t m_i x_i^k \]
		Alors :
		\begin{enumerate}[(i)]
			\item $\sigma = \sum_{i, j \in \llbracket 0, n-1 \rrbracket} s_{i+j} X_i X_j$ définit forme quadratique sur $\mathbb{C}^n$ ainsi qu'une forme quadratique $\sigma_{\mathbb{R}}$ sur $\mathbb{R}^n$.
			\item Si on note $(p,q)$ la signature de $\sigma_{\mathbb{R}}$, on a :
			\begin{itemize}
				\item $t = p + q$.
				\item Le nombre de racines réelles distinctes de $P$ est $p-q$.
			\end{itemize}
		\end{enumerate}
	\end{application}

	\subsection{Adjonction de racines}
	\label{144-1}

	\reference[GOZ]{21}

	\begin{definition}
		On appelle \textbf{extension} de $\mathbb{K}$ tout corps $\mathbb{L}$ tel qu'il existe un morphisme de corps de $\mathbb{K}$ dans $\mathbb{L}$. On notera $\mathbb{L} / \mathbb{K}$ pour signifier que $\mathbb{L}$ est une extension de $\mathbb{K}$ par la suite.
	\end{definition}

	\begin{remark}
		\begin{itemize}
			\item Si $\mathbb{K}$ est un sous-corps de $\mathbb{L}$, alors $\mathbb{L}$ est une extension de $\mathbb{K}$.
			\item Un morphisme de corps est forcément injectif, donc on peut identifier $\mathbb{K}$ à son image et dire que $\mathbb{K} \subseteq \mathbb{L}$ de manière abusive.
		\end{itemize}
	\end{remark}

	\begin{example}
		$\mathbb{C}$ est une extension de $\mathbb{R}$.
	\end{example}

	L'idée dans la suite va être de chercher comment ``rajouter'' des racines à des polynômes pourtant irréductibles sur un corps.

	\reference{57}

	\begin{definition}
		Soit $P \in \mathbb{K}[X]$. On dit que $\mathbb{L}$ est un \textbf{corps de rupture} de $P$ si $\mathbb{L} = \mathbb{K}[\alpha]$ où $\alpha$ est une racine de $P$ de $\mathbb{L}$.
	\end{definition}

	\begin{example}
		\begin{itemize}
			\item Avec les notations précédentes, si $\deg(P) = 1$, $\mathbb{K}$ est un corps de rupture de $P$.
			\item $\mathbb{C}$ est un corps de rupture de $X^2+1$ sur $\mathbb{R}$.
			\item $\mathbb{F}_4$ est un corps de rupture de $X^2+X+1$ sur $\mathbb{F}_2$.
		\end{itemize}
	\end{example}

	\begin{theorem}
		Soit $P \in \mathbb{K}[X]$ un polynôme irréductible sur $\mathbb{K}$.
		\begin{itemize}
			\item Il existe un corps de rupture de $P$.
			\item Si $\mathbb{L} = \mathbb{K}[\alpha]$ et $\mathbb{L}' = \mathbb{K}[\beta]$ sont deux corps de rupture de $P$, alors il existe un unique $\mathbb{K}$-isomorphisme $\varphi : \mathbb{L} \rightarrow \mathbb{L}'$ tel que $\varphi(\alpha) = \beta$.
		\end{itemize}
	\end{theorem}

	\begin{definition}
		Soit $P \in \mathbb{K}[X]$ de degré $n \geq 1$. On dit que $\mathbb{L}$ est un \textbf{corps de décomposition} de $P$ si $\mathbb{L} = \mathbb{K}[\alpha_1, \dots, \alpha_n]$ où $\alpha_1, \dots, \alpha_n$ sont des racines de $P$ dans $\mathbb{L}$.
	\end{definition}

	\begin{example}
		\begin{itemize}
			\item $\mathbb{K}$ est un corps de décomposition de tout polynôme de degré $1$ sur $\mathbb{K}$.
			\item $\mathbb{C}$ est un corps de décomposition de $X^2+1$ sur $\mathbb{R}$.
			\item Soit $\xi \in \mu_n^*$, alors $\mathbb{Q}[\xi]$ est un corps de décomposition de $\Phi_n$ (le $n$-ième polynôme cyclotomique) sur $\mathbb{Q}$.
		\end{itemize}
	\end{example}

	\begin{theorem}
		Soit $P \in \mathbb{K}[X]$ un polynôme irréductible sur $\mathbb{K}$.
		\begin{itemize}
			\item Il existe un corps de décomposition de $P$.
			\item Deux corps de décomposition de $P$ sont $\mathbb{K}$-isomorphes.
		\end{itemize}
	\end{theorem}

	\begin{definition}
		$\mathbb{K}$ est \textbf{algébriquement clos} si tout polynôme de $\mathbb{K}[X]$ de degré supérieur ou égal à $1$ admet au moins une racine dans $\mathbb{K}$.
	\end{definition}

	\begin{example}
		\begin{itemize}
			\item $\mathbb{Q}$ n'est pas algébriquement clos.
			\item $\mathbb{R}$ non plus.
		\end{itemize}
	\end{example}

	\begin{proposition}
		Tout corps algébriquement clos est infini.
	\end{proposition}

	\begin{theorem}[D'Alembert-Gauss]
		$\mathbb{C}$ est algébriquement clos.
	\end{theorem}

	\begin{definition}
		On dit que $\mathbb{L}$ est une \textbf{clôture algébrique} de $\mathbb{K}$ si $\mathbb{L}$ est une extension de $\mathbb{K}$ algébriquement close et si
		\[ \forall x \in \mathbb{L}, \, \exists P \in \mathbb{K}[X] \text{ tel que } P(x) = 0 \]
	\end{definition}

	\begin{example}
		\begin{itemize}
			\item $\mathbb{C}$ est une clôture algébrique de $\mathbb{R}$.
			\item $\overline{\mathbb{Q}} = \{ \alpha \in \mathbb{C} \mid \exists P \in \mathbb{Q}[X] \text{ tel que } P(\alpha) = 0 \}$ est une clôture algébrique de $\mathbb{Q}$.
		\end{itemize}
	\end{example}

	\begin{theorem}[Steinitz]
		\begin{enumerate}[(i)]
			\item Il existe une clôture algébrique de $\mathbb{K}$.
			\item Deux clôtures algébriques de $\mathbb{K}$ sont $\mathbb{K}$-isomorphes.
		\end{enumerate}
	\end{theorem}

	\subsection{Application en algèbre linéaire}

	\reference[GOU21]{171}
	\reference{186}

	\begin{definition}
		Soit $A \in \mathcal{M}_n(K)$. On appelle :
		\begin{itemize}
			\item \textbf{Polynôme caractéristique} de $A$ le polynôme $\chi_A = \det(A - XI_n)$.
			\item \textbf{Polynôme minimal} de $A$ l'unique polynôme unitaire $\mu_A$ qui engendre l'idéal $\mathrm{Ann}(A) = \{ Q \in \mathbb{K}[X] \mid Q(A) = 0 \}$.
		\end{itemize}
	\end{definition}

	\reference{172}

	\begin{proposition}
		\[ \lambda \text{ est valeur propre de } A \iff \chi_A(\lambda) = 0 \iff \mu_A(\lambda) = 0 \]
	\end{proposition}

	\reference{185}

	\begin{proposition}
		\begin{itemize}
			\item $A$ est trigonalisable si et seulement si $\chi_A$ est scindé sur $\mathbb{K}$.
			\item $A$ est diagonalisable si et seulement si $\mu_A$ est scindé à racines simples sur $\mathbb{K}$.
		\end{itemize}
	\end{proposition}

	\begin{remark}
		Si $\mathbb{K} = \mathbb{F}_q$, $A$ est diagonalisable si et seulement si $A^q = A$.
	\end{remark}
	%</content>
\end{document}
