\documentclass[12pt, a4paper]{report}

% LuaLaTeX :

\RequirePackage{iftex}
\RequireLuaTeX

% Packages :

\usepackage[french]{babel}
%\usepackage[utf8]{inputenc}
%\usepackage[T1]{fontenc}
\usepackage[pdfencoding=auto, pdfauthor={Hugo Delaunay}, pdfsubject={Mathématiques}, pdfcreator={agreg.skyost.eu}]{hyperref}
\usepackage{amsmath}
\usepackage{amsthm}
%\usepackage{amssymb}
\usepackage{stmaryrd}
\usepackage{tikz}
\usepackage{tkz-euclide}
\usepackage{fourier-otf}
\usepackage{fontspec}
\usepackage{titlesec}
\usepackage{fancyhdr}
\usepackage{catchfilebetweentags}
\usepackage[french, capitalise, noabbrev]{cleveref}
\usepackage[fit, breakall]{truncate}
\usepackage[top=2.5cm, right=2cm, bottom=2.5cm, left=2cm]{geometry}
\usepackage{enumerate}
\usepackage{tocloft}
\usepackage{microtype}
%\usepackage{mdframed}
%\usepackage{thmtools}
\usepackage{xcolor}
\usepackage{tabularx}
\usepackage{aligned-overset}
\usepackage[subpreambles=true]{standalone}
\usepackage{environ}
\usepackage[normalem]{ulem}
\usepackage{marginnote}
\usepackage{etoolbox}
\usepackage{setspace}
\usepackage[bibstyle=reading, citestyle=draft]{biblatex}
\usepackage{xpatch}
\usepackage[many, breakable]{tcolorbox}
\usepackage[backgroundcolor=white, bordercolor=white, textsize=small]{todonotes}

% Bibliographie :

\newcommand{\overridebibliographypath}[1]{\providecommand{\bibliographypath}{#1}}
\overridebibliographypath{../bibliography.bib}
\addbibresource{\bibliographypath}
\defbibheading{bibliography}[\bibname]{%
	\newpage
	\section*{#1}%
}
\renewbibmacro*{entryhead:full}{\printfield{labeltitle}}
\DeclareFieldFormat{url}{\newline\footnotesize\url{#1}}
\AtEndDocument{\printbibliography}

% Police :

\setmathfont{Erewhon Math}

% Tikz :

\usetikzlibrary{calc}

% Longueurs :

\setlength{\parindent}{0pt}
\setlength{\headheight}{15pt}
\setlength{\fboxsep}{0pt}
\titlespacing*{\chapter}{0pt}{-20pt}{10pt}
\setlength{\marginparwidth}{1.5cm}
\setstretch{1.1}

% Métadonnées :

\author{agreg.skyost.eu}
\date{\today}

% Titres :

\setcounter{secnumdepth}{3}

\renewcommand{\thechapter}{\Roman{chapter}}
\renewcommand{\thesubsection}{\Roman{subsection}}
\renewcommand{\thesubsubsection}{\arabic{subsubsection}}
\renewcommand{\theparagraph}{\alph{paragraph}}

\titleformat{\chapter}{\huge\bfseries}{\thechapter}{20pt}{\huge\bfseries}
\titleformat*{\section}{\LARGE\bfseries}
\titleformat{\subsection}{\Large\bfseries}{\thesubsection \, - \,}{0pt}{\Large\bfseries}
\titleformat{\subsubsection}{\large\bfseries}{\thesubsubsection. \,}{0pt}{\large\bfseries}
\titleformat{\paragraph}{\bfseries}{\theparagraph. \,}{0pt}{\bfseries}

\setcounter{secnumdepth}{4}

% Table des matières :

\renewcommand{\cftsecleader}{\cftdotfill{\cftdotsep}}
\addtolength{\cftsecnumwidth}{10pt}

% Redéfinition des commandes :

\renewcommand*\thesection{\arabic{section}}
\renewcommand{\ker}{\mathrm{Ker}}

% Nouvelles commandes :

\newcommand{\website}{https://agreg.skyost.eu}

\newcommand{\tr}[1]{\mathstrut ^t #1}
\newcommand{\im}{\mathrm{Im}}
\newcommand{\rang}{\operatorname{rang}}
\newcommand{\trace}{\operatorname{trace}}
\newcommand{\id}{\operatorname{id}}
\newcommand{\stab}{\operatorname{Stab}}

\providecommand{\newpar}{\\[\medskipamount]}

\providecommand{\lesson}[3]{%
	\title{#3}%
	\hypersetup{pdftitle={#3}}%
	\setcounter{section}{\numexpr #2 - 1}%
	\section{#3}%
	\fancyhead[R]{\truncate{0.73\textwidth}{#2 : #3}}%
}

\providecommand{\development}[3]{%
	\title{#3}%
	\hypersetup{pdftitle={#3}}%
	\section*{#3}%
	\fancyhead[R]{\truncate{0.73\textwidth}{#3}}%
}

\providecommand{\summary}[1]{%
	\textit{#1}%
	\medskip%
}

\tikzset{notestyleraw/.append style={inner sep=0pt, rounded corners=0pt, align=center}}

%\newcommand{\booklink}[1]{\website/bibliographie\##1}
\newcommand{\citelink}[2]{\hyperlink{cite.\therefsection @#1}{#2}}
\newcommand{\previousreference}{}
\providecommand{\reference}[2][]{%
	\notblank{#1}{\renewcommand{\previousreference}{#1}}{}%
	\todo[noline]{%
		\protect\vspace{16pt}%
		\protect\par%
		\protect\notblank{#1}{\cite{[\previousreference]}\\}{}%
		\protect\citelink{\previousreference}{p. #2}%
	}%
}

\definecolor{devcolor}{HTML}{00695c}
\newcommand{\dev}[1]{%
	\reversemarginpar%
	\todo[noline]{
		\protect\vspace{16pt}%
		\protect\par%
		\bfseries\color{devcolor}\href{\website/developpements/#1}{DEV}
	}%
	\normalmarginpar%
}

% En-têtes :

\pagestyle{fancy}
\fancyhead[L]{\truncate{0.23\textwidth}{\thepage}}
\fancyfoot[C]{\scriptsize \href{\website}{\texttt{agreg.skyost.eu}}}

% Couleurs :

\definecolor{property}{HTML}{fffde7}
\definecolor{proposition}{HTML}{fff8e1}
\definecolor{lemma}{HTML}{fff3e0}
\definecolor{theorem}{HTML}{fce4f2}
\definecolor{corollary}{HTML}{ffebee}
\definecolor{definition}{HTML}{ede7f6}
\definecolor{notation}{HTML}{f3e5f5}
\definecolor{example}{HTML}{e0f7fa}
\definecolor{cexample}{HTML}{efebe9}
\definecolor{application}{HTML}{e0f2f1}
\definecolor{remark}{HTML}{e8f5e9}
\definecolor{proof}{HTML}{e1f5fe}

% Théorèmes :

\theoremstyle{definition}
\newtheorem{theorem}{Théorème}

\newtheorem{property}[theorem]{Propriété}
\newtheorem{proposition}[theorem]{Proposition}
\newtheorem{lemma}[theorem]{Lemme}
\newtheorem{corollary}[theorem]{Corollaire}

\newtheorem{definition}[theorem]{Définition}
\newtheorem{notation}[theorem]{Notation}

\newtheorem{example}[theorem]{Exemple}
\newtheorem{cexample}[theorem]{Contre-exemple}
\newtheorem{application}[theorem]{Application}

\theoremstyle{remark}
\newtheorem{remark}[theorem]{Remarque}

\counterwithin*{theorem}{section}

\newcommand{\applystyletotheorem}[1]{
	\tcolorboxenvironment{#1}{
		enhanced,
		breakable,
		colback=#1!98!white,
		boxrule=0pt,
		boxsep=0pt,
		left=8pt,
		right=8pt,
		top=8pt,
		bottom=8pt,
		sharp corners,
		after=\par,
	}
}

\applystyletotheorem{property}
\applystyletotheorem{proposition}
\applystyletotheorem{lemma}
\applystyletotheorem{theorem}
\applystyletotheorem{corollary}
\applystyletotheorem{definition}
\applystyletotheorem{notation}
\applystyletotheorem{example}
\applystyletotheorem{cexample}
\applystyletotheorem{application}
\applystyletotheorem{remark}
\applystyletotheorem{proof}

% Environnements :

\NewEnviron{whitetabularx}[1]{%
	\renewcommand{\arraystretch}{2.5}
	\colorbox{white}{%
		\begin{tabularx}{\textwidth}{#1}%
			\BODY%
		\end{tabularx}%
	}%
}

% Maths :

\DeclareFontEncoding{FMS}{}{}
\DeclareFontSubstitution{FMS}{futm}{m}{n}
\DeclareFontEncoding{FMX}{}{}
\DeclareFontSubstitution{FMX}{futm}{m}{n}
\DeclareSymbolFont{fouriersymbols}{FMS}{futm}{m}{n}
\DeclareSymbolFont{fourierlargesymbols}{FMX}{futm}{m}{n}
\DeclareMathDelimiter{\VERT}{\mathord}{fouriersymbols}{152}{fourierlargesymbols}{147}


% Bibliographie :

\addbibresource{\bibliographypath}%
\defbibheading{bibliography}[\bibname]{%
	\newpage
	\section*{#1}%
}
\renewbibmacro*{entryhead:full}{\printfield{labeltitle}}%
\DeclareFieldFormat{url}{\newline\footnotesize\url{#1}}%

\AtEndDocument{\printbibliography}

\begin{document}
	%<*content>
	\lesson{analysis}{239}{Fonctions définies par une intégrale dépendant d'un paramètre. Exemples et applications.}

	\subsection{Régularité d'une fonction définie par une intégrale à paramètre}

	Soient $(X, \mathcal{A}, \mu)$ un espace mesuré et $f : E \times X \rightarrow \mathbb{C}$ où $(E, d)$ est un espace métrique. On pose $F : t \mapsto \int_X f(t, x) \, \mathrm{d}\mu(x)$.

	\subsubsection{Continuité}

	\reference[Z-Q]{306}

	\begin{theorem}[Continuité sous le signe intégral]
		On suppose :
		\begin{enumerate}[(i)]
			\item $\forall t \in E$, $x \mapsto f(t,x)$ est mesurable.
			\item pp. en $x \in X$, $t \mapsto f(t,x)$ est continue en $t_0 \in E$.
			\item $\exists g \in L_1(X)$ positive telle que
			\[ |f(t,x)| \leq g(x) \quad \forall t \in E, \text{pp. en } x \in X \]
		\end{enumerate}
		Alors $F$ est continue en $t_0$.
	\end{theorem}

	\begin{corollary}
		On suppose :
		\begin{enumerate}[(i)]
			\item $\forall t \in E$, $x \mapsto f(t,x)$ est mesurable.
			\item pp. en $x \in X$, $t \mapsto f(t,x)$ est continue sur $E$.
			\item $\forall K \subseteq E, \, \exists g_K \in L_1(X)$ positive telle que
			\[ |f(t,x)| \leq g_K(x) \quad \forall t \in E, \text{pp. en } x \]
		\end{enumerate}
		Alors $F$ est continue sur $E$.
	\end{corollary}

	\reference{312}

	\begin{example}
		\label{239-1}
		La fonction
		\[ \Gamma :
		\begin{array}{ccc}
			\mathbb{R}^+_* &\rightarrow& \mathbb{R}^+_* \\
			t &\mapsto& \int_{0}^{+\infty} t^{x-1} e^{-t} \, \mathrm{d}t
		\end{array}
		\]
		est bien définie et continue sur $\mathbb{R}^+_*$.
	\end{example}

	\reference[G-K]{104}

	\begin{example}
		Soit $f : \mathbb{R}^+ \rightarrow \mathbb{C}$ intégrable. Alors,
		\[ \lambda \mapsto \int_0^{+\infty} e^{-\lambda t} f(t) \, \mathrm{d}t \]
		est bien définie et est continue sur $\mathbb{R}^+$.
	\end{example}

	\subsubsection{Dérivabilité}

	\reference[Z-Q]{307}

	On suppose ici que $E$ est un intervalle $I$ ouvert de $\mathbb{R}$.

	\begin{theorem}[Dérivation sous le signe intégral]
		\label{239-2}
		On suppose :
		\begin{enumerate}[(i)]
			\item $\forall t \in I$, $x \mapsto f(t,x) \in L_1(X)$.
			\item pp. en $x \in X$, $t \mapsto f(t,x)$ est dérivable sur $I$. On notera $\frac{\partial f}{\partial t}$ cette dérivée définie presque partout.
			\item $\forall K \subseteq I$ compact, $\exists g_K \in L_1(X)$ positive telle que
			\[ \left| \frac{\partial f}{\partial t}(x,t) \right| \leq g_K(x) \quad \forall t \in I, \text{pp. en } x \]
		\end{enumerate}
		Alors $\forall t \in I$, $x \mapsto \frac{\partial f}{\partial t}(x, t) \in L_1(X)$ et $F$ est dérivable sur $I$ avec
		\[ \forall t \in I, \, F'(t) = \int_X \frac{\partial f}{\partial t}(x, t) \, \mathrm{d}\mu(x) \]
	\end{theorem}

	\begin{remark}
		\begin{itemize}
			\item Si dans le \cref{239-2}, hypothèse (i), on remplace ``dérivable'' par ``$\mathcal{C}^1$'', alors la fonction $F$ est de classe $\mathcal{C}^1$.
			\item On a un résultat analogue pour les dérivées d'ordre supérieur.
		\end{itemize}
	\end{remark}

	\begin{theorem}[$k$-ième dérivée sous le signe intégral]
		On suppose :
		\begin{enumerate}[(i)]
			\item $\forall t \in I$, $x \mapsto f(t,x) \in L_1(X)$.
			\item pp. en $x \in X$, $t \mapsto f(t,x) \in \mathcal{C}^k(I)$. On notera $\left(\frac{\partial}{\partial t}\right)^j f$ la $j$-ième dérivée définie presque partout pour $j \in \llbracket 0, k \rrbracket$.
			\item $\forall j \in \llbracket 0, k \rrbracket$, $\forall K \subseteq I$ compact, $\exists g_{j,K} \in L_1(X)$ positive telle que
			\[ \left| \left(\frac{\partial}{\partial t}\right)^j f(x,t) \right| \leq g_{j,K}(x) \quad \forall t \in K, \text{pp. en } x \]
		\end{enumerate}
		Alors $\forall j \in \llbracket 0, k \rrbracket$, $\forall t \in I$, $x \mapsto \left(\frac{\partial}{\partial t}\right)^j f(x,t) \in L_1(X)$ et $F \in \mathcal{C}^k(I)$ avec
		\[ \forall j \in \llbracket 0, k \rrbracket, \, \forall t \in I, \, F^{(j)}(t) = \int_X \left(\frac{\partial}{\partial t}\right)^j f(x, t) \, \mathrm{d}\mu(x) \]
	\end{theorem}

	\reference{312}

	\begin{example}
		La fonction $\Gamma$ de l'\cref{239-1} est $\mathcal{C}^\infty$ sur $\mathbb{R}^+_*$.
	\end{example}

	\reference[B-P]{143}

	\begin{example}
		On se place dans l'espace mesuré $(\mathbb{N}, \mathcal{P}(\mathbb{N}), \operatorname{card})$ et on considère $(f_n)$ une suite de fonctions dérivables sur $I$ telle que
		\[ \forall x \in \mathbb{R}, \, \sum_{n \in \mathbb{N}} |f_n(x)| + \sup_{x \in I} |f'_n(t)| < +\infty \]
		Alors $x \mapsto \sum_{n \in \mathbb{N}} f_n(x)$ est dérivable sur $I$ de dérivée $x \mapsto \sum_{n \in \mathbb{N}} f'_n(x)$.
	\end{example}

	\reference[GOU20]{169}

	\begin{application}[Transformée de Fourier d'une Gaussienne]
		En résolvant une équation différentielle linéaire, on a
		\[ \forall \alpha > 0, \, \forall x \in \mathbb{R}, \, \int_{\mathbb{R}} e^{-\alpha t^2} e^{-itx} \, \mathrm{d}t = \sqrt{\frac{\pi}{\alpha}} e^{-\frac{x^2}{\pi \alpha}} \]
	\end{application}

	\reference[G-K]{107}
	\dev{integrale-de-dirichlet}

	\begin{application}[Intégrale de Dirichlet]
		On pose $\forall x \geq 0$,
		\[ F(x) = \int_0^{+\infty} \frac{\sin(t)}{t} e^{-xt} \, \mathrm{d}t \]
		alors :
		\begin{enumerate}[(i)]
			\item $F$ est bien définie et est continue sur $\mathbb{R}^+$.
			\item $F$ est dérivable sur $\mathbb{R}^+_*$ et $\forall x \in \mathbb{R}^+_*$, $F'(x) = -\frac{1}{1+x^2}$.
			\item $F(0) = \int_0^{+\infty} \frac{\sin(t)}{t} \, \mathrm{d}t = \frac{\pi}{2}$.
		\end{enumerate}
	\end{application}

	\subsubsection{Holomorphie}

	\reference[Z-Q]{308}

	On suppose ici que $E$ est un ouvert $\Omega$ de $\mathbb{C}$.

	\begin{theorem}[Holomorphie sous le signe intégral]
		On suppose :
		\begin{enumerate}[(i)]
			\item $\forall z \in \Omega$, $x \mapsto f(z,x) \in L_1(X)$.
			\item pp. en $x \in X$, $z \mapsto f(z,x)$ est holomorphe dans $\Omega$. On notera $\frac{\partial f}{\partial z}$ cette dérivée définie presque partout.
			\item $\forall K \subseteq \Omega$ compact, $\exists g_K \in L_1(X)$ positive telle que
			\[ \left| f(x,z) \right| \leq g_K(x) \quad \forall z \in K, \text{pp. en } x \]
		\end{enumerate}
		Alors $F$ est holomorphe dans $\Omega$ avec
		\[ \forall z \in \Omega, \, F'(z) = \int_X \frac{\partial f}{\partial z}(z, t) \, \mathrm{d}\mu(z) \]
	\end{theorem}

	\begin{example}
		La fonction $\Gamma$ de l'\cref{239-1} est holomorphe dans l'ouvert $\{ z \in \mathbb{C} \mid \operatorname{Re}(z) > 0 \}$.
	\end{example}

	\subsection{Produit de convolution}

	\subsubsection{Notion de convolée de deux fonctions}

	\reference[AMR08]{75}

	\begin{definition}
		Soient $f$ et $g$ deux fonctions de $\mathbb{R}^d$ dans $\mathbb{R}$. On dit que \textbf{la convolée} (ou \textbf{le produit de convolution}) de $f$ et $g$ en $x \in \mathbb{R}$ \textbf{existe} si la fonction
		\[
		\begin{array}{ccc}
			\mathbb{R} &\rightarrow& \mathbb{C} \\
			t &\mapsto& f(x-t)g(t)
		\end{array}
		\]
		est intégrable sur $\mathbb{R}^d$ pour la mesure de Lebesgue. On pose alors :
		\[ (f * g)(x) = \int_{\mathbb{R}^d} f(x-t)g(t) \, \mathrm{d}t \]
	\end{definition}

	\begin{proposition}
		Dans $L_1(\mathbb{R}^d)$, le produit de convolution est commutatif, bilinéaire et associatif.
	\end{proposition}

	\begin{theorem}
		Soient $p, q > 0$ et $f \in L_p(\mathbb{R}^d)$ et $g \in L_q(\mathbb{R}^d)$.
		\begin{enumerate}[(i)]
			\item Si $p, q \in [1, +\infty]$ tels que $\frac{1}{p} + \frac{1}{q} = 1$, alors $(f * g)(x)$ existe \underline{pour tout} $x \in \mathbb{R}^d$ et est uniformément continue. On a, $\Vert f * g \Vert_\infty \leq \Vert f \Vert_p \Vert g \Vert_q$ et, si $p \neq 1, +\infty$, $f * g \in \mathcal{C}_0(\mathbb{R})$.
			\item Si $p = 1$ et $q = +\infty$, alors $(f * g)(x)$ existe \underline{pour tout} $x \in \mathbb{R}^d$ et $f * g \in \mathcal{C}_b(\mathbb{R})$.
			\item Si $p = 1$ et $q \in [1, +\infty[$, alors $(f * g)(x)$ existe \underline{pp.} en $x \in \mathbb{R}^d$ et $f * g \in L_q(\mathbb{R})$ telle que $\Vert f * g \Vert_q \leq \Vert f \Vert_1 \Vert g \Vert_q$.
			\item Si $p = 1$ et $q = 1$, alors $(f * g)(x)$ existe \underline{pp.} en $x \in \mathbb{R}^d$ et $f * g \in L_1(\mathbb{R})$ telle que $\Vert f * g \Vert_1 \leq \Vert f \Vert_1 \Vert g \Vert_1$.
		\end{enumerate}
	\end{theorem}

	\begin{example}
		Soient $a < b \in \mathbb{R}^+_*$. Alors $\mathbb{1}_{[-a, a]} * \mathbb{1}_{[-b,b]}$ existe pour tout $x \in \mathbb{R}$ et
		\[ \left( \mathbb{1}_{[-a, a]} * \mathbb{1}_{[-b,b]} \right)(x) =
		\begin{cases}
			2a &\text{si } 0 \leq \vert x \vert \leq b-a \\
			b+a-\vert x \vert &\text{si } b-a \leq \vert x \vert \leq b+a \\
			0 &\text{sinon}
		\end{cases}
		\]
	\end{example}

	\reference{85}

	\begin{proposition}
		$L_1(\mathbb{R}^d)$ est une algèbre de Banach pour le produit de convolution.
	\end{proposition}

	\begin{remark}
		Cette algèbre n'a pas d'élément neutre. Afin de pallier à ce manque, nous allons voir la notion d'approximation de l'identité dans la sous-section suivante.
	\end{remark}

	\newpage
	\subsubsection{Approximation de l'identité}

	\reference[B-P]{269}

	\begin{definition}
		On appelle \textbf{approximation de l'identité} toute suite $(\rho_n)$ de fonctions mesurables de $L_1(\mathbb{R}^d)$ telles que
		\begin{enumerate}[(i)]
			\item $\forall n \in \mathbb{N}, \, \int_{\mathbb{R}^d} \rho_n \, \mathrm{d}\lambda_d = 1$.
			\item $\sup_{n \geq 1} \Vert \rho_n \Vert < +\infty$.
			\item $\forall \epsilon > 0, \, \lim_{n \rightarrow +\infty} \int_{\mathbb{R} \setminus B(0, \epsilon)} \rho_n(x) \, \mathrm{d}x = 0$.
		\end{enumerate}
	\end{definition}

	\begin{remark}
		Dans la définition précédente, (ii) implique (i) lorsque les fonctions $\rho_n$ sont positives. Plutôt que des suites, on pourra considérer les familles indexées par $\mathbb{R}_*^+$.
	\end{remark}

	\begin{example}
		\begin{itemize}
			\item Noyau de Laplace sur $\mathbb{R}$ :
			\[ \forall t > 0, \, \rho_t(x) = \frac{1}{2t}e^{-\frac{|x|}{t}} \]
			\item Noyau de Cauchy sur $\mathbb{R}$ :
			\[ \forall t > 0, \, \rho_t(x) = \frac{t}{\pi (t^2 + x^2)} \]
			\item Noyau de Gauss sur $\mathbb{R}$ :
			\[ \forall t > 0, \, \rho_t(x) = \frac{1}{\sqrt{2\pi} t}e^{-\frac{|x|^2}{2t^2}} \]
		\end{itemize}
	\end{example}

	\reference[GOU20]{304}

	\begin{application}[Théorème de Weierstrass]
		Toute fonction continue $f : [a,b] \rightarrow \mathbb{R}$ (avec $a, b \in \mathbb{R}$ tels que $a \leq b$) est limite uniforme de fonctions polynômiales sur $[a, b]$.
	\end{application}

	\reference[B-P]{270}

	\begin{theorem}
		Soit $(\rho_n)$ une approximation de l'identité. Soient $p \in [1, +\infty[$ et $f \in L_p(\mathbb{R}^d)$, alors :
		\[ \forall n \geq 1, \, f * \rho_n \in L_p(\mathbb{R}^d) \quad \text{ et } \quad \Vert f * \rho_n - f \Vert_p \longrightarrow 0 \]
	\end{theorem}

	\begin{theorem}
		Soient $(\rho_n)$ une approximation de l'identité et $f \in L_\infty(\mathbb{R}^d)$. Alors :
		\begin{itemize}
			\item Si $f$ est continue en $x_0 \in \mathbb{R}^d$, alors $(f * \rho_n)(x_0) \longrightarrow_{n \rightarrow +\infty} f(x_0)$.
			\item Si $f$ est uniformément continue sur $\mathbb{R}^d$, alors $\Vert f * \rho_n - f \Vert_\infty \longrightarrow_{n \rightarrow +\infty} 0$.
			\item Si $f$ est continue sur un compact $K$, alors $\sup_{x \in K} |(f * \rho_n)(x) - f(x)| \longrightarrow_{n \rightarrow +\infty} 0$.
		\end{itemize}
	\end{theorem}

	\begin{definition}
		On qualifie de \textbf{régularisante} toute suite $(\alpha_n)$ d'approximations de l'identité telle que $\forall n \in \mathbb{N}, \, \alpha_n \in \mathcal{C}^\infty_K(\mathbb{R}^d)$.
	\end{definition}

	\reference{274}

	\begin{example}
		Soit $\alpha \in \mathcal{C}^\infty_K(\mathbb{R}^d)$ une densité de probabilité. Alors la suite $(\alpha_n)$ définie pour tout $n \in \mathbb{N}$ par $\alpha_n : x \mapsto n \alpha(nx)$ est régularisante.
	\end{example}

	\reference[AMR08]{96}

	\begin{application}
		\begin{enumerate}[(i)]
			\item $\mathcal{C}^\infty_K(\mathbb{R}^d)$ est dense dans $\mathcal{C}_K(\mathbb{R}^d)$ pour $\Vert . \Vert_\infty$.
			\item $\mathcal{C}^\infty_K(\mathbb{R}^d)$ est dense dans $L_p(\mathbb{R}^d)$ pour $\Vert . \Vert_p$ avec $p \in [1, +\infty[$.
		\end{enumerate}
	\end{application}

	\subsection{Transformée de Fourier}

	\subsubsection{Sur \texorpdfstring{$L_1(\mathbb{R}^d)$}{Rᵈ}}

	\reference{109}

	\begin{definition}
		Soit $f : \mathbb{R}^d \rightarrow \mathbb{C}$ une fonction mesurable. On définit, lorsque cela a un sens, sa \textbf{transformée de Fourier}, notée $\widehat{f}$ par
		\[
		\widehat{f} :
		\begin{array}{ccc}
			\mathbb{R}^d &\rightarrow& \mathbb{C} \\
			\xi &\mapsto& \int_{\mathbb{R}^d} f(x) e^{-i\langle x, \xi \rangle} \, \mathrm{d}x
		\end{array}
		\]
	\end{definition}

	\begin{lemma}[Riemann-Lebesgue]
		Soit $f \in L_1(\mathbb{R}^d)$, $\widehat{f}$ existe et
		\[ \lim_{\Vert \xi \Vert \rightarrow +\infty} \widehat{f}(\xi) \]
	\end{lemma}

	\begin{theorem}
		$\forall f \in L_1(\mathbb{R}^d)$, $\widehat{f}$ est continue, bornée par $\Vert f \Vert_1$. Donc la transformation de Fourier
		\[
		\mathcal{F} :
		\begin{array}{ccc}
			L_1(\mathbb{R}^d) &\rightarrow& \mathcal{C}_0(\mathbb{R}^d) \\
			f &\mapsto& \widehat{f}
		\end{array}
		\]
		est bien définie.
	\end{theorem}

	\begin{corollary}
		La transformation de Fourier $\mathcal{F} : L_1(\mathbb{R}^d) \rightarrow \mathcal{C}_0(\mathbb{R}^d)$ est une application linéaire continue.
	\end{corollary}

	\begin{example}[Densité de Poisson]
		On pose $\forall x \in \mathbb{R}$, $p(x) = \frac{1}{2} e^{-|x|}$. Alors $p \in L_1(\mathbb{R})$ et, $\forall \xi \in \mathbb{R}$, $\widehat{p}(\xi) = \frac{1}{1+\xi^2}$.
	\end{example}

	\begin{example}
		\[
		\forall \xi \in \mathbb{R}, \, \widehat{\mathbb{1}_{[-1,1]}}(\xi) =
		\begin{cases}
			\frac{2 \sin(\xi)}{\xi} \text{ si } \xi \neq 0 \\
			2 \text{ sinon}
		\end{cases}
		\]
		Remarquons ici que la transformée de Fourier n'est pas intégrable.
	\end{example}

	\reference{114}

	\begin{proposition}
		\[ \forall f, g \in L_1(\mathbb{R}^d), \, \widehat{f * g} = \widehat{f} \widehat{g} \]
	\end{proposition}

	\begin{theorem}[Formule de dualité]
		\[ \forall f, g \in L_1(\mathbb{R}^d), \int_{\mathbb{R}^d} f(t) \widehat{g}(t) \, \mathrm{d}t = \int_{\mathbb{R}^d} \widehat{f}(t) g(t) \, \mathrm{d}t \]
	\end{theorem}

	\begin{corollary}
		La transformation de Fourier $\mathcal{F} : L_1(\mathbb{R}^d) \rightarrow \mathcal{C}_0(\mathbb{R}^d)$ est une application injective.
	\end{corollary}

	\reference[BMP]{140}
	\dev{densite-polynomes-orthogonaux}

	\begin{application}
		Soient $I$ un intervalle de $\mathbb{R}$ et $\rho$ une fonction poids. On considère $(P_n)$ la famille des polynômes orthogonaux associée à $\rho$ sur $I$. On suppose qu'il existe $a > 0$ tel que
		\[ \int_I e^{a \vert x \vert} \rho(x) \, \mathrm{d}x < +\infty \]
		alors $(P_n)$ est une base hilbertienne de $L_2(I, \rho)$ pour la norme $\Vert . \Vert_2$.
	\end{application}

	\reference[AMR08]{116}

	\begin{theorem}[Formule d'inversion de Fourier]
		Si $f \in L_1(\mathbb{R}^d)$ est telle que $\widehat{f} \in L_1(\mathbb{R}^d)$, alors
		\[ \widehat{\widehat{f}}(x) = (2\pi)^d f(x) \text{ pp. en } x \in \mathbb{R}^d \]
	\end{theorem}

	\begin{proposition}
		Soient $g \in L_1(\mathbb{R}^d)$ et $f \in L_1(\mathbb{R}^d)$ telle que $\widehat{f} \in L_1(\mathbb{R}^d)$, alors
		\[ \widehat{fg} = \frac{1}{(2\pi)^d} \widehat{f} * \widehat{g} \]
	\end{proposition}

	\subsubsection{Sur \texorpdfstring{$L_2(\mathbb{R}^d)$}{Rᵈ}}

	\begin{theorem}[Plancherel-Parseval]
		\[ \forall f \in L_1(\mathbb{R}^d), \, \Vert \widehat{f} \Vert^2_2 = (2 \pi)^d \Vert f \Vert^2_2 \]
	\end{theorem}

	\begin{theorem}
		Soit $f \in L_2(\mathbb{R}^d)$. Alors :
		\begin{enumerate}[(i)]
			\item Il existe une suite $(f_n)$ de $L_1(\mathbb{R}^d) \, \cap \, L_2(\mathbb{R}^d)$ qui converge vers $f$ dans $L_2(\mathbb{R}^d)$.
			\item Pour une telle suite $(f_n)$, la suite $(\widehat{f_n})$ converge dans $L_2(\mathbb{R}^d)$ vers une limite $\widetilde{f}$ indépendante de la suite choisie.
		\end{enumerate}
	\end{theorem}

	\begin{definition}
		La limite $\widetilde{f}$ est la \textbf{transformée de Fourier} de $f$ dans $L_2(\mathbb{R}^d)$.
	\end{definition}

	\begin{proposition}
		Les transformations de Fourier $L_1(\mathbb{R}^d)$ et $L_2(\mathbb{R}^d)$ coïncident sur $L_1(\mathbb{R}^d) \, \cap \, L_2(\mathbb{R}^d)$.
	\end{proposition}

	\subsubsection{Application en probabilités}

	\reference[G-K]{239}

	\begin{definition}
		Soit $X$ un vecteur aléatoire. On appelle \textbf{fonction caractéristique} de $X$, notée $\phi_X$, la transformée de Fourier de la loi $\mathbb{P}_X$ (définie à un signe près) :
		\[ \phi_X : t \mapsto \mathbb{E}(e^{i \langle t, x \rangle}) \]
	\end{definition}

	\begin{theorem}
		Soient $X$ et $Y$ deux vecteurs aléatoires. Alors,
		\[ \phi_X = \phi_Y \iff \mathbb{P}_X = \mathbb{P}_Y \]
	\end{theorem}

	\begin{corollary}
		Soient $X$ et $Y$ deux vecteurs aléatoires tels que $\forall a \in \mathbb{R}^d$, $\langle X, a \rangle$ et $\langle Y, a \rangle$ ont même loi. Alors, $X$ et $Y$ ont même loi.
	\end{corollary}
	%</content>
\end{document}
