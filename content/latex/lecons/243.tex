\documentclass[12pt, a4paper]{report}

% LuaLaTeX :

\RequirePackage{iftex}
\RequireLuaTeX

% Packages :

\usepackage[french]{babel}
%\usepackage[utf8]{inputenc}
%\usepackage[T1]{fontenc}
\usepackage[pdfencoding=auto, pdfauthor={Hugo Delaunay}, pdfsubject={Mathématiques}, pdfcreator={agreg.skyost.eu}]{hyperref}
\usepackage{amsmath}
\usepackage{amsthm}
%\usepackage{amssymb}
\usepackage{stmaryrd}
\usepackage{tikz}
\usepackage{tkz-euclide}
\usepackage{fourier-otf}
\usepackage{fontspec}
\usepackage{titlesec}
\usepackage{fancyhdr}
\usepackage{catchfilebetweentags}
\usepackage[french, capitalise, noabbrev]{cleveref}
\usepackage[fit, breakall]{truncate}
\usepackage[top=2.5cm, right=2cm, bottom=2.5cm, left=2cm]{geometry}
\usepackage{enumerate}
\usepackage{tocloft}
\usepackage{microtype}
%\usepackage{mdframed}
%\usepackage{thmtools}
\usepackage{xcolor}
\usepackage{tabularx}
\usepackage{aligned-overset}
\usepackage[subpreambles=true]{standalone}
\usepackage{environ}
\usepackage[normalem]{ulem}
\usepackage{marginnote}
\usepackage{etoolbox}
\usepackage{setspace}
\usepackage[bibstyle=reading, citestyle=draft]{biblatex}
\usepackage{xpatch}
\usepackage[many, breakable]{tcolorbox}
\usepackage[backgroundcolor=white, bordercolor=white, textsize=small]{todonotes}

% Bibliographie :

\newcommand{\overridebibliographypath}[1]{\providecommand{\bibliographypath}{#1}}
\overridebibliographypath{../bibliography.bib}
\addbibresource{\bibliographypath}
\defbibheading{bibliography}[\bibname]{%
	\newpage
	\section*{#1}%
}
\renewbibmacro*{entryhead:full}{\printfield{labeltitle}}
\DeclareFieldFormat{url}{\newline\footnotesize\url{#1}}
\AtEndDocument{\printbibliography}

% Police :

\setmathfont{Erewhon Math}

% Tikz :

\usetikzlibrary{calc}

% Longueurs :

\setlength{\parindent}{0pt}
\setlength{\headheight}{15pt}
\setlength{\fboxsep}{0pt}
\titlespacing*{\chapter}{0pt}{-20pt}{10pt}
\setlength{\marginparwidth}{1.5cm}
\setstretch{1.1}

% Métadonnées :

\author{agreg.skyost.eu}
\date{\today}

% Titres :

\setcounter{secnumdepth}{3}

\renewcommand{\thechapter}{\Roman{chapter}}
\renewcommand{\thesubsection}{\Roman{subsection}}
\renewcommand{\thesubsubsection}{\arabic{subsubsection}}
\renewcommand{\theparagraph}{\alph{paragraph}}

\titleformat{\chapter}{\huge\bfseries}{\thechapter}{20pt}{\huge\bfseries}
\titleformat*{\section}{\LARGE\bfseries}
\titleformat{\subsection}{\Large\bfseries}{\thesubsection \, - \,}{0pt}{\Large\bfseries}
\titleformat{\subsubsection}{\large\bfseries}{\thesubsubsection. \,}{0pt}{\large\bfseries}
\titleformat{\paragraph}{\bfseries}{\theparagraph. \,}{0pt}{\bfseries}

\setcounter{secnumdepth}{4}

% Table des matières :

\renewcommand{\cftsecleader}{\cftdotfill{\cftdotsep}}
\addtolength{\cftsecnumwidth}{10pt}

% Redéfinition des commandes :

\renewcommand*\thesection{\arabic{section}}
\renewcommand{\ker}{\mathrm{Ker}}

% Nouvelles commandes :

\newcommand{\website}{https://agreg.skyost.eu}

\newcommand{\tr}[1]{\mathstrut ^t #1}
\newcommand{\im}{\mathrm{Im}}
\newcommand{\rang}{\operatorname{rang}}
\newcommand{\trace}{\operatorname{trace}}
\newcommand{\id}{\operatorname{id}}
\newcommand{\stab}{\operatorname{Stab}}

\providecommand{\newpar}{\\[\medskipamount]}

\providecommand{\lesson}[3]{%
	\title{#3}%
	\hypersetup{pdftitle={#3}}%
	\setcounter{section}{\numexpr #2 - 1}%
	\section{#3}%
	\fancyhead[R]{\truncate{0.73\textwidth}{#2 : #3}}%
}

\providecommand{\development}[3]{%
	\title{#3}%
	\hypersetup{pdftitle={#3}}%
	\section*{#3}%
	\fancyhead[R]{\truncate{0.73\textwidth}{#3}}%
}

\providecommand{\summary}[1]{%
	\textit{#1}%
	\medskip%
}

\tikzset{notestyleraw/.append style={inner sep=0pt, rounded corners=0pt, align=center}}

%\newcommand{\booklink}[1]{\website/bibliographie\##1}
\newcommand{\citelink}[2]{\hyperlink{cite.\therefsection @#1}{#2}}
\newcommand{\previousreference}{}
\providecommand{\reference}[2][]{%
	\notblank{#1}{\renewcommand{\previousreference}{#1}}{}%
	\todo[noline]{%
		\protect\vspace{16pt}%
		\protect\par%
		\protect\notblank{#1}{\cite{[\previousreference]}\\}{}%
		\protect\citelink{\previousreference}{p. #2}%
	}%
}

\definecolor{devcolor}{HTML}{00695c}
\newcommand{\dev}[1]{%
	\reversemarginpar%
	\todo[noline]{
		\protect\vspace{16pt}%
		\protect\par%
		\bfseries\color{devcolor}\href{\website/developpements/#1}{DEV}
	}%
	\normalmarginpar%
}

% En-têtes :

\pagestyle{fancy}
\fancyhead[L]{\truncate{0.23\textwidth}{\thepage}}
\fancyfoot[C]{\scriptsize \href{\website}{\texttt{agreg.skyost.eu}}}

% Couleurs :

\definecolor{property}{HTML}{fffde7}
\definecolor{proposition}{HTML}{fff8e1}
\definecolor{lemma}{HTML}{fff3e0}
\definecolor{theorem}{HTML}{fce4f2}
\definecolor{corollary}{HTML}{ffebee}
\definecolor{definition}{HTML}{ede7f6}
\definecolor{notation}{HTML}{f3e5f5}
\definecolor{example}{HTML}{e0f7fa}
\definecolor{cexample}{HTML}{efebe9}
\definecolor{application}{HTML}{e0f2f1}
\definecolor{remark}{HTML}{e8f5e9}
\definecolor{proof}{HTML}{e1f5fe}

% Théorèmes :

\theoremstyle{definition}
\newtheorem{theorem}{Théorème}

\newtheorem{property}[theorem]{Propriété}
\newtheorem{proposition}[theorem]{Proposition}
\newtheorem{lemma}[theorem]{Lemme}
\newtheorem{corollary}[theorem]{Corollaire}

\newtheorem{definition}[theorem]{Définition}
\newtheorem{notation}[theorem]{Notation}

\newtheorem{example}[theorem]{Exemple}
\newtheorem{cexample}[theorem]{Contre-exemple}
\newtheorem{application}[theorem]{Application}

\theoremstyle{remark}
\newtheorem{remark}[theorem]{Remarque}

\counterwithin*{theorem}{section}

\newcommand{\applystyletotheorem}[1]{
	\tcolorboxenvironment{#1}{
		enhanced,
		breakable,
		colback=#1!98!white,
		boxrule=0pt,
		boxsep=0pt,
		left=8pt,
		right=8pt,
		top=8pt,
		bottom=8pt,
		sharp corners,
		after=\par,
	}
}

\applystyletotheorem{property}
\applystyletotheorem{proposition}
\applystyletotheorem{lemma}
\applystyletotheorem{theorem}
\applystyletotheorem{corollary}
\applystyletotheorem{definition}
\applystyletotheorem{notation}
\applystyletotheorem{example}
\applystyletotheorem{cexample}
\applystyletotheorem{application}
\applystyletotheorem{remark}
\applystyletotheorem{proof}

% Environnements :

\NewEnviron{whitetabularx}[1]{%
	\renewcommand{\arraystretch}{2.5}
	\colorbox{white}{%
		\begin{tabularx}{\textwidth}{#1}%
			\BODY%
		\end{tabularx}%
	}%
}

% Maths :

\DeclareFontEncoding{FMS}{}{}
\DeclareFontSubstitution{FMS}{futm}{m}{n}
\DeclareFontEncoding{FMX}{}{}
\DeclareFontSubstitution{FMX}{futm}{m}{n}
\DeclareSymbolFont{fouriersymbols}{FMS}{futm}{m}{n}
\DeclareSymbolFont{fourierlargesymbols}{FMX}{futm}{m}{n}
\DeclareMathDelimiter{\VERT}{\mathord}{fouriersymbols}{152}{fourierlargesymbols}{147}


% Bibliographie :

\addbibresource{\bibliographypath}%
\defbibheading{bibliography}[\bibname]{%
	\newpage
	\section*{#1}%
}
\renewbibmacro*{entryhead:full}{\printfield{labeltitle}}%
\DeclareFieldFormat{url}{\newline\footnotesize\url{#1}}%

\AtEndDocument{\printbibliography}

\begin{document}
	%<*content>
	\lesson{analysis}{243}{Séries entières, propriétés de la somme. Exemples et applications.}

	\subsection{Séries entières et rayons de convergence}

	\subsubsection{Définitions}

	\reference[GOU20]{247}

	\begin{definition}
		On appelle \textbf{série entière} toute série de fonctions de la forme $\sum a_n z^n$ où $z$ est une variable complexe et où $(a_n)$ est une suite complexe.
	\end{definition}

	\begin{example}
		$\sum \frac{z^n}{n!}$ est une série entière.
	\end{example}

	\begin{lemma}[Abel]
		Soient $\sum a_n z^n$ une série entière et $z_0 \in \mathbb{C}$ tels que $(a_n z_0^n)$ soit bornée. Alors :
		\begin{enumerate}[label=(\roman*)]
			\item $\forall z \in \mathbb{C}$ tel que $|z| < |z_0|$, $\sum a_n z^n$ converge absolument.
			\item $\forall r \in ]0,z_0[, \, \sum a_n z^n$ converge normalement dans $\overline{D}(0, r) = \{ z \in \mathbb{C} \mid |z| \leq r \}$.
		\end{enumerate}
	\end{lemma}

	\begin{definition}
		Soit $\sum a_n z^n$ une série entière. Le nombre
		\[ R = \sup \{ r \geq 0 \mid (|a_n|r^n) \text{ est bornée} \} \]
		est le \textbf{rayon de convergence} de $\sum a_n z^n$. On a :
		\begin{itemize}
			\item $\forall z \in \mathbb{C}$ tel que $|z| < R$, $\sum a_n z^n$ converge absolument.
			\item $\forall z \in \mathbb{C}$ tel que $|z| > R$, $\sum a_n z^n$ diverge.
			\item $\forall r \in [0,R[$, $\sum a_n z^n$ converge normalement sur $\overline{D}(0,r)$.
		\end{itemize}
		Le disque $D(0,R)$ est le \textbf{disque de convergence} de la série, le cercle $C(0,R)$ est le \textbf{cercle d'incertitude}.
	\end{definition}

	\subsubsection{Comparaison de rayons de convergence}

	\reference[AMR11]{234}

	Soient $\sum a_n z^n$ et $\sum b_n z^n$ deux séries entières dont on note $R_a$ et $R_b$ les rayons de convergence respectifs.

	\begin{proposition}
		\begin{enumerate}[label=(\roman*)]
			\item Si $\forall n \in \mathbb{N}$, on a $|a_n| \leq |b_n|$, alors $R_a \geq R_b$.
			\item Si $a_n = O(b_n)$, alors $R_a \geq R_b$.
			\item Si $a_n \sim b_n$, alors $R_a = R_b$.
		\end{enumerate}
	\end{proposition}

	\begin{example}
		La série entière $\sum e^{\cos(n)} z^n$ a un rayon de convergence égal à $1$.
	\end{example}

	\subsubsection{Calcul du rayon de convergence}

	\reference{233}

	\begin{proposition}[Règle de d'Alembert]
		Soit $\sum a_n z^n$ une série entière. Si $\lim_{n \rightarrow +\infty} \left| \frac{a_{n+1}}{a_n} \right| = \lambda$ avec $\lambda \in [0, +\infty]$, alors le rayon de convergence de $\sum a_n z^n$ est égal à $\frac{1}{\lambda}$.
	\end{proposition}

	\begin{example}
		La série entière $\sum \frac{z^n}{n!}$ a un rayon de convergence infini.
	\end{example}

	\begin{proposition}[Formule d'Hadamard]
		Le rayon de convergence d'une série entière $\sum a_n z^n$ est donné par $\frac{1}{\rho}$ où
		\[ \rho = \limsup_{n \rightarrow +\infty} |a_n|^{\frac{1}{n}} \]
	\end{proposition}

	\begin{example}
		La série entière $\sum 2^n z^{2n}$ a un rayon de convergence égal à $\frac{1}{\sqrt{2}}$.
	\end{example}

	\begin{corollary}[Règle de Cauchy]
		Soit $\sum a_n z^n$ une série entière. Si $\lim_{n \rightarrow +\infty} \left| a_n \right|^{\frac{1}{n}} = \lambda$ avec $\lambda \in [0, +\infty]$, alors le rayon de convergence de $\sum a_n z^n$ est égal à $\frac{1}{\lambda}$.
	\end{corollary}

	\begin{example}
		La série entière $\sum \frac{n}{2^n} z^n$ a un rayon de convergence égal à $2$.
	\end{example}

	\subsubsection{Étude sur le cercle d'incertitude}

	\reference{231}

	\begin{example}
		Le comportement d'une série entière peut varier sur le cercle d'incertitude suivant ses coefficients :
		\begin{itemize}
			\item $\sum z^n$ dont le rayon de convergence est égal à $1$ diverge en tout point de $C(0,1)$.
			\item $\sum \frac{1}{n^2} z^n$ dont le rayon de convergence est égal à $1$ converge en tout point de $C(0,1)$.
			\item $\sum \frac{z^n}{n} z^n$ dont le rayon de convergence est égal à $1$ converge en $1$ mais diverge en tout autre point de $C(0,1)$.
		\end{itemize}
	\end{example}

	\reference[GOU20]{263}
	\dev{theoreme-d-abel-angulaire}

	\begin{theorem}[Abel angulaire]
		\label{243-1}
		Soit $\sum a_n z^n$ une série entière de rayon de convergence supérieur ou égal à $1$ tel que $\sum a_n$ converge. On note $f$ la somme de cette série sur le disque unité $D$ de $\mathbb{C}$. On fixe $\theta_0 \in \left[ 0, \frac{\pi}{2} \right[$ et on pose $\Delta_{\theta_0} = \{ z \in D \mid \exists \rho > 0 \text{ et } \exists \theta \in [-\theta_0, \theta_0] \text{ tels que } z = 1 - \rho e^{i\theta} \}$.
		\newpar
		Alors $\lim_{\substack{z \rightarrow 1 \\ z \in \Delta_{\theta_0}}} f(z) = \sum_{n=0}^{+\infty} a_n$.
	\end{theorem}

	\begin{application}
		\[ \sum_{n=0}^{+\infty} \frac{(-1)^n}{(2n+1)} = \frac{\pi}{4} \]
	\end{application}

	\begin{application}
		\[ \sum_{n=0}^{+\infty} \frac{(-1)^{n-1}}{n} = \ln(2) \]
	\end{application}

  \begin{cexample}
    La réciproque est fausse :
    \[ \lim_{\substack{z \rightarrow 1 \\ \vert z \vert < 1}} (-1)^n z^n = \lim_{\substack{z \rightarrow 1 \\ \vert z \vert < 1}} \frac{1}{1+z} = \frac{1}{2} \]
  \end{cexample}

	\begin{theorem}[Taubérien faible]
		Soit $\sum a_n z^n$ une série entière de rayon de convergence $1$. On note $f$ la somme de cette série sur $D(0,1)$. On suppose que
		\[ \exists S \in \mathbb{C} \text{ tel que } \lim_{\substack{x \rightarrow 1 \\ x < 1}} f(x) = S \]
		Si $a_n = o \left( \frac{1}{n} \right)$, alors $\sum a_n$ converge et $\sum_{n=0}^{+\infty} a_n = S$.
	\end{theorem}

	\begin{remark}
		Ce dernier résultat est une réciproque partielle du \cref{243-1}. Il reste vrai en supposant $a_n = O \left( \frac{1}{n} \right)$ (c'est le théorème Taubérien fort).
	\end{remark}

	\subsection{Propriétés}

	\subsubsection{Opérations sur les séries entières}

	\reference[AMR11]{235}

	Soient $\sum a_n z^n$ et $\sum b_n z^n$ deux séries entières dont on note $R_a$ et $R_b$ les rayons de convergence respectifs.

	\begin{proposition}
		En multipliant $\sum a_n z^n$ par un scalaire, on ne change pas le rayon de convergence de la série initiale.
	\end{proposition}

	\begin{definition}
		On appelle \textbf{série entière somme} la série entière $\sum (a_n + b_n) z^n$.
	\end{definition}

	\begin{proposition}
		On note $R_{a+b}$ le rayon de convergence de la série somme. Alors $R_{a+b} \geq \min \{R_a, R_b\}$ avec égalité si $R_a \neq R_b$.
	\end{proposition}

	\reference[GOU20]{248}

	\begin{example}
		Les séries entières $\sum z^n$ et $\sum -z^n$ ont leur rayon de convergence égal à $1$ et la série somme un rayon de convergence infini.
	\end{example}

	\reference[AMR11]{235}

	\begin{definition}
		On appelle \textbf{produit de Cauchy} la série entière $\sum c_n z^n$ où
		\[ \forall n \in \mathbb{N}, \, c_n = \sum_{k=0}^n a_k b_{n-k} \]
	\end{definition}

	\begin{proposition}
		On note $R_{c}$ le rayon de convergence du produit de Cauchy $\sum c_n z^n$. Alors,
		\begin{enumerate}[label=(\roman*)]
			\item $R_c \geq \min \{R_a, R_b\}$.
			\item $\forall z \in D(0, \min \{R_a, R_b\}), \, \sum_{n = 0}^{+\infty} c_n z^n = (\sum_{n = 0}^{+\infty} a_n z^n) (\sum_{n = 0}^{+\infty} b_n z^n)$.
		\end{enumerate}
	\end{proposition}

	\subsubsection{Propriétés de la somme}

	\reference[AMR11]{239}

	Dans toute cette sous-partie, $\sum a_n z^n$ désigne une série entière de rayon de convergence $R > 0$. On note $S$ sa somme sur $D(0,R)$.

	\begin{proposition}
		$S$ est continue sur $D(0,R)$.
	\end{proposition}

	\begin{example}
		La série entière $\sum \frac{z^n}{n!}$ est continue sur $\mathbb{C}$.
	\end{example}

	\begin{corollary}
		$\forall p \in \mathbb{N}$, $S$ admet un développement limité à l'ordre $p$ au voisinage de l'origine, dont la partie régulière est donnée par $a_0 + a_1z + \dots + a_pz^p$.
	\end{corollary}

	\begin{proposition}
		Soit $[a, b] \in ]-R, R[$, alors
		\[ \int_{a}^{b} S(x) \, \mathrm{d}x = \sum_{n=0}^{+\infty} a_n \int_{a}^{b} x^n \, \mathrm{d}x \]
	\end{proposition}

	\begin{corollary}
		Les primitives de $S$ sont de la forme $\sum_{n=0}^{+\infty} \frac{a_n}{n+1} x^{n+1} + \alpha$ avec $\alpha \in \mathbb{C}$.
	\end{corollary}

	\begin{proposition}
		$S$ est de classe $\mathcal{C}^\infty$ sur $]-R,R[$ et
		\[ \forall k \in \mathbb{N}, \, \forall x \in ]-R,R[, \, S^{(k)}(x) = \sum_{k = 0}^{+\infty} \frac{k!}{(n-k)!} a_n x^{n-k} \]
	\end{proposition}

	\begin{remark}
		En particulier, $\forall k \in \mathbb{N}$, $a_k = \frac{S^{(k)}(0)}{k!}$.
	\end{remark}

	\begin{example}
		\[ \forall x \in ]-1, 1[, \, \sum_{n=0}^{+\infty} (n+1)x^n = \frac{1}{(1-x)^2} \]
	\end{example}

	\subsubsection{Développement en série entière}

	\reference[BMP]{46}

	\begin{definition}
		Soient $U \subseteq \mathbb{C}$ un ouvert et $f : U \rightarrow \mathbb{C}$. On dit que $f$ est \textbf{développable en série entière en $a \in U$} s'il existe $r > 0$ et $(a_n) \in \mathbb{C}^{\mathbb{N}}$ tels que $D(a, r) \subseteq U$ et
		\[ \forall z \in D(a, r), \, f(z) = \sum_{n=0}^{+\infty} a_n (z-a)^n \]
	\end{definition}

	\reference[AMR11]{241}

	\begin{example}
		Tout polynôme est développable en série entière en tout point de $\mathbb{R}$.
	\end{example}

	\begin{proposition}
		Soient $f : x \mapsto \sum_{n=0}^{+\infty} a_n x^n$ et $g : x \mapsto \sum_{n=0}^{+\infty} b_n x^n$ deux fonctions développables en séries entières en $0$. Alors :
		\begin{enumerate}[label=(\roman*)]
			\item $\forall \lambda \in \mathbb{C}$, $\lambda f + g$ est développable en série entière et son développement est
			\[ \sum_{n=0}^{+\infty} (\lambda a_n + b_n) x^n \]
			\item $fg$ est développable en série entière et son développement est le produit de Cauchy des deux séries entières.
		\end{enumerate}
	\end{proposition}

	\begin{proposition}
		Soit $f : x \mapsto \sum_{n=0}^{+\infty} a_n x^n$ une fonction développable en série entière en $0$. Alors $\exists I \subseteq \mathbb{R}$ avec $0 \in I$ tel que :
		\begin{enumerate}[label=(\roman*)]
			\item $f'$ est développable en série entière en $0$ son développement est
			\[ \sum_{n=0}^{+\infty} (n+1) a_{n+1} x^n \]
			\item $f$ est donc $\mathcal{C}^\infty$.
			\item $f$ est continue et si $F$ est une primitive de $f$ sur $I$, $F$ est développable en série entière en $0$ son développement est
			\[ F(0) + \sum_{n=0}^{+\infty} \frac{a_n}{n+1} x^{n+1} \]
		\end{enumerate}
	\end{proposition}

	\begin{example}
		Voici quelques développements en série entière usuels :
		\begin{itemize}
			\item $\forall x \in \mathbb{R}$, $e^x = \sum_{n=0}^{+\infty} \frac{x^n}{n!}$.
			\item $\forall x \in \mathbb{R}$, $\cos(x) = \sum_{n=0}^{+\infty} \frac{(-1)^n x^{2n}}{(2n)!}$ et $\sin(x) = \sum_{n=0}^{+\infty} \frac{(-1)^n x^{2n+1}}{(2n+1)!}$.
			\item $\forall \alpha > 0$, $\forall x \in ]-1,1[$, $1 + \sum_{n=0}^{+\infty} \frac{\alpha(\alpha - 1) \dots (\alpha - n + 1)}{n!} x^n$.
		\end{itemize}
	\end{example}

	\reference[BMP]{55}

	\begin{cexample}
		La fonction
		\[
		x \mapsto \begin{cases}
			e^{-\frac{1}{x^2}} \text{ si } x > 0 \\
			0 \text{ sinon}
		\end{cases}
		\]
		est $\mathcal{C}^\infty$ mais n'est pas développable en série entière en $0$.
	\end{cexample}

	\subsection{Applications}

	\subsubsection{Analyse complexe}

	\reference{46}

	\begin{definition}
		Soient $U \subseteq \mathbb{C}$ un ouvert et $f : U \rightarrow \mathbb{C}$. On dit que $f$ est \textbf{analytique sur $U$} si $f$ est développable en série entière en tout point de $U$.
	\end{definition}

	\begin{theorem}
		Soient $\sum a_n z^n$ une série entière de rayon de convergence $R > 0$ et $z_0 \in D(0,R)$. On note $f : z \mapsto \sum_{n=0}^{+\infty} a_n z^n$.
		Alors $f$ est holomorphe en $z_0$ et $f'(z_0) = \sum_{n=0}^{+\infty} n a_n z_0^{n-1}$.
	\end{theorem}

	\begin{theorem}[Zéros isolés]
		Soient $U \subseteq \mathbb{C}$ un ouvert connexe et $f : U \rightarrow \mathbb{C}$. Si $f$ est une fonction analytique si $f$ n'est pas identiquement nulle, alors l'ensemble des zéros de $f$ n'admet pas de point d'accumulation dans $U$.
	\end{theorem}

	\begin{corollary}
		Soient $U \subseteq \mathbb{C}$ un ouvert connexe et $f : U \rightarrow \mathbb{C}$. Alors $f$ admet un nombre fini de zéros dans tout compact de $U$.
	\end{corollary}

	\reference[GOU20]{250}

	\begin{corollary}
		Deux séries entières dont les sommes coïncident sur un voisinage de $0$ dans $\mathbb{R}$ sont égales.
	\end{corollary}

	\reference[BMP]{63}

	\begin{theorem}
		Soit $f$ une fonction holomorphe sur un disque ouvert de rayon $\rho$ centré en un point $a$. Alors $f$ est analytique sur ce disque.
		De plus, on a convergence normale sur tout compact du disque.
	\end{theorem}

	\subsubsection{Dénombrement}

	\reference[GOU20]{314}
	\dev{nombres-de-bell}

	\begin{application}[Nombres de Bell]
		Pour tout $n \in \mathbb{N}^*$, on note $B_n$ le nombre de partitions de $\llbracket 1, n \rrbracket$. Par convention on pose $B_0 = 1$. Alors,
		\[ \forall k \in \mathbb{N}^*, \, B_k = \frac{1}{e} \sum_{n=0}^{+\infty} \frac{n^k}{n!} \]
	\end{application}

	\reference[DAN]{336}

	\begin{application}
		Soit $n \in \mathbb{N}^*$. $\sigma \in S_n$ est un \textbf{dérangement de $S_n$} si $\forall k \in \llbracket 1, n \rrbracket$, $\sigma(k) \neq k$.
		Alors,
		\[ d_n = n! \sum_{k=0}^n \frac{(-1)^k}{k!} \]
	\end{application}

	\subsubsection{Équations différentielles}

	\reference[AMR11]{246}

	\begin{proposition}
		Pour résoudre une équation différentielle linéaire $(L)$ à l'aide des séries entières :
		\begin{enumerate}[label=(\roman*)]
			\item On suppose que $\varphi(x) = \sum_{n=0}^{+\infty} a_n x^n$ est solution de $(L)$ et on l'introduit dans $(L)$.
			\item On se ramène à $\sum_{n=0}^{+\infty} b_nx^n = 0$ où les $b_n$ dépendent des $a_n$.
			\item On trouve une relation liant les $a_n$ et on vérifie que la série $\sum_{n=0}^{+\infty} a_n x^n$ a un rayon de convergence non-nul.
		\end{enumerate}
	\end{proposition}

	\reference{273}

	\begin{example}
		Les solutions de $t^2 (1-t) y'' - t (1+t) y' + y = 0$ sont les fonctions $t \mapsto \lambda \frac{x}{1-x}$ (où $\lambda \in \mathbb{R}$).
	\end{example}

	\newpage
	\section*{Annexes}

	\reference[GOU20]{263}

	\begin{figure}[H]
		\begin{center}
			\begin{tikzpicture}
				\draw[->] (-3, 0) -- (3, 0) node[right] {$x$};
				\draw[->] (0, -3) -- (0, 3) node[above] {$y$};
				\draw (0,2) node {$\bullet$} node[above right]{$1$};
				\draw (2,0) node {$\bullet$} node[below right]{$1$};
				\draw (0,0) circle (2);
				\coordinate (A) at (130:3.5);
				\coordinate (B) at (230:3.5);
				\coordinate (C) at (2,0);
				\begin{scope}
					\path[clip] circle (2);
					\path[clip] (A) -- (B) -- (C) -- cycle;
					\draw [transparent, fill=blue!30, fill opacity=0.3] (C) circle (9);
				\end{scope}
				\begin{scope}
					\path[clip] (A) -- (180:3.5) -- (C) -- cycle;
					\draw (C) circle (1);
				\end{scope}
				\draw (0.7,0.35) node {$\theta_0$};
				\draw (C) -- (A);
				\draw (C) -- (B);
				\coordinate (S) at (210:3.5);
				\coordinate (E) at (-0.5,-0.5);
				\draw [->] (S) to [out=50] (E);
				\draw (212:3.7) node {$\Delta_{\theta_0}$};
			\end{tikzpicture}
		\end{center}
		\caption{Illustration du théorème d'Abel angulaire}
	\end{figure}
	%</content>
\end{document}
