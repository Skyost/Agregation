\documentclass[12pt, a4paper]{report}

% LuaLaTeX :

\RequirePackage{iftex}
\RequireLuaTeX

% Packages :

\usepackage[french]{babel}
%\usepackage[utf8]{inputenc}
%\usepackage[T1]{fontenc}
\usepackage[pdfencoding=auto, pdfauthor={Hugo Delaunay}, pdfsubject={Mathématiques}, pdfcreator={agreg.skyost.eu}]{hyperref}
\usepackage{amsmath}
\usepackage{amsthm}
%\usepackage{amssymb}
\usepackage{stmaryrd}
\usepackage{tikz}
\usepackage{tkz-euclide}
\usepackage{fourier-otf}
\usepackage{fontspec}
\usepackage{titlesec}
\usepackage{fancyhdr}
\usepackage{catchfilebetweentags}
\usepackage[french, capitalise, noabbrev]{cleveref}
\usepackage[fit, breakall]{truncate}
\usepackage[top=2.5cm, right=2cm, bottom=2.5cm, left=2cm]{geometry}
\usepackage{enumerate}
\usepackage{tocloft}
\usepackage{microtype}
%\usepackage{mdframed}
%\usepackage{thmtools}
\usepackage{xcolor}
\usepackage{tabularx}
\usepackage{aligned-overset}
\usepackage[subpreambles=true]{standalone}
\usepackage{environ}
\usepackage[normalem]{ulem}
\usepackage{marginnote}
\usepackage{etoolbox}
\usepackage{setspace}
\usepackage[bibstyle=reading, citestyle=draft]{biblatex}
\usepackage{xpatch}
\usepackage[many, breakable]{tcolorbox}
\usepackage[backgroundcolor=white, bordercolor=white, textsize=small]{todonotes}

% Bibliographie :

\newcommand{\overridebibliographypath}[1]{\providecommand{\bibliographypath}{#1}}
\overridebibliographypath{../bibliography.bib}
\addbibresource{\bibliographypath}
\defbibheading{bibliography}[\bibname]{%
	\newpage
	\section*{#1}%
}
\renewbibmacro*{entryhead:full}{\printfield{labeltitle}}
\DeclareFieldFormat{url}{\newline\footnotesize\url{#1}}
\AtEndDocument{\printbibliography}

% Police :

\setmathfont{Erewhon Math}

% Tikz :

\usetikzlibrary{calc}

% Longueurs :

\setlength{\parindent}{0pt}
\setlength{\headheight}{15pt}
\setlength{\fboxsep}{0pt}
\titlespacing*{\chapter}{0pt}{-20pt}{10pt}
\setlength{\marginparwidth}{1.5cm}
\setstretch{1.1}

% Métadonnées :

\author{agreg.skyost.eu}
\date{\today}

% Titres :

\setcounter{secnumdepth}{3}

\renewcommand{\thechapter}{\Roman{chapter}}
\renewcommand{\thesubsection}{\Roman{subsection}}
\renewcommand{\thesubsubsection}{\arabic{subsubsection}}
\renewcommand{\theparagraph}{\alph{paragraph}}

\titleformat{\chapter}{\huge\bfseries}{\thechapter}{20pt}{\huge\bfseries}
\titleformat*{\section}{\LARGE\bfseries}
\titleformat{\subsection}{\Large\bfseries}{\thesubsection \, - \,}{0pt}{\Large\bfseries}
\titleformat{\subsubsection}{\large\bfseries}{\thesubsubsection. \,}{0pt}{\large\bfseries}
\titleformat{\paragraph}{\bfseries}{\theparagraph. \,}{0pt}{\bfseries}

\setcounter{secnumdepth}{4}

% Table des matières :

\renewcommand{\cftsecleader}{\cftdotfill{\cftdotsep}}
\addtolength{\cftsecnumwidth}{10pt}

% Redéfinition des commandes :

\renewcommand*\thesection{\arabic{section}}
\renewcommand{\ker}{\mathrm{Ker}}

% Nouvelles commandes :

\newcommand{\website}{https://agreg.skyost.eu}

\newcommand{\tr}[1]{\mathstrut ^t #1}
\newcommand{\im}{\mathrm{Im}}
\newcommand{\rang}{\operatorname{rang}}
\newcommand{\trace}{\operatorname{trace}}
\newcommand{\id}{\operatorname{id}}
\newcommand{\stab}{\operatorname{Stab}}

\providecommand{\newpar}{\\[\medskipamount]}

\providecommand{\lesson}[3]{%
	\title{#3}%
	\hypersetup{pdftitle={#3}}%
	\setcounter{section}{\numexpr #2 - 1}%
	\section{#3}%
	\fancyhead[R]{\truncate{0.73\textwidth}{#2 : #3}}%
}

\providecommand{\development}[3]{%
	\title{#3}%
	\hypersetup{pdftitle={#3}}%
	\section*{#3}%
	\fancyhead[R]{\truncate{0.73\textwidth}{#3}}%
}

\providecommand{\summary}[1]{%
	\textit{#1}%
	\medskip%
}

\tikzset{notestyleraw/.append style={inner sep=0pt, rounded corners=0pt, align=center}}

%\newcommand{\booklink}[1]{\website/bibliographie\##1}
\newcommand{\citelink}[2]{\hyperlink{cite.\therefsection @#1}{#2}}
\newcommand{\previousreference}{}
\providecommand{\reference}[2][]{%
	\notblank{#1}{\renewcommand{\previousreference}{#1}}{}%
	\todo[noline]{%
		\protect\vspace{16pt}%
		\protect\par%
		\protect\notblank{#1}{\cite{[\previousreference]}\\}{}%
		\protect\citelink{\previousreference}{p. #2}%
	}%
}

\definecolor{devcolor}{HTML}{00695c}
\newcommand{\dev}[1]{%
	\reversemarginpar%
	\todo[noline]{
		\protect\vspace{16pt}%
		\protect\par%
		\bfseries\color{devcolor}\href{\website/developpements/#1}{DEV}
	}%
	\normalmarginpar%
}

% En-têtes :

\pagestyle{fancy}
\fancyhead[L]{\truncate{0.23\textwidth}{\thepage}}
\fancyfoot[C]{\scriptsize \href{\website}{\texttt{agreg.skyost.eu}}}

% Couleurs :

\definecolor{property}{HTML}{fffde7}
\definecolor{proposition}{HTML}{fff8e1}
\definecolor{lemma}{HTML}{fff3e0}
\definecolor{theorem}{HTML}{fce4f2}
\definecolor{corollary}{HTML}{ffebee}
\definecolor{definition}{HTML}{ede7f6}
\definecolor{notation}{HTML}{f3e5f5}
\definecolor{example}{HTML}{e0f7fa}
\definecolor{cexample}{HTML}{efebe9}
\definecolor{application}{HTML}{e0f2f1}
\definecolor{remark}{HTML}{e8f5e9}
\definecolor{proof}{HTML}{e1f5fe}

% Théorèmes :

\theoremstyle{definition}
\newtheorem{theorem}{Théorème}

\newtheorem{property}[theorem]{Propriété}
\newtheorem{proposition}[theorem]{Proposition}
\newtheorem{lemma}[theorem]{Lemme}
\newtheorem{corollary}[theorem]{Corollaire}

\newtheorem{definition}[theorem]{Définition}
\newtheorem{notation}[theorem]{Notation}

\newtheorem{example}[theorem]{Exemple}
\newtheorem{cexample}[theorem]{Contre-exemple}
\newtheorem{application}[theorem]{Application}

\theoremstyle{remark}
\newtheorem{remark}[theorem]{Remarque}

\counterwithin*{theorem}{section}

\newcommand{\applystyletotheorem}[1]{
	\tcolorboxenvironment{#1}{
		enhanced,
		breakable,
		colback=#1!98!white,
		boxrule=0pt,
		boxsep=0pt,
		left=8pt,
		right=8pt,
		top=8pt,
		bottom=8pt,
		sharp corners,
		after=\par,
	}
}

\applystyletotheorem{property}
\applystyletotheorem{proposition}
\applystyletotheorem{lemma}
\applystyletotheorem{theorem}
\applystyletotheorem{corollary}
\applystyletotheorem{definition}
\applystyletotheorem{notation}
\applystyletotheorem{example}
\applystyletotheorem{cexample}
\applystyletotheorem{application}
\applystyletotheorem{remark}
\applystyletotheorem{proof}

% Environnements :

\NewEnviron{whitetabularx}[1]{%
	\renewcommand{\arraystretch}{2.5}
	\colorbox{white}{%
		\begin{tabularx}{\textwidth}{#1}%
			\BODY%
		\end{tabularx}%
	}%
}

% Maths :

\DeclareFontEncoding{FMS}{}{}
\DeclareFontSubstitution{FMS}{futm}{m}{n}
\DeclareFontEncoding{FMX}{}{}
\DeclareFontSubstitution{FMX}{futm}{m}{n}
\DeclareSymbolFont{fouriersymbols}{FMS}{futm}{m}{n}
\DeclareSymbolFont{fourierlargesymbols}{FMX}{futm}{m}{n}
\DeclareMathDelimiter{\VERT}{\mathord}{fouriersymbols}{152}{fourierlargesymbols}{147}


% Bibliographie :

\addbibresource{\bibliographypath}%
\defbibheading{bibliography}[\bibname]{%
	\newpage
	\section*{#1}%
}
\renewbibmacro*{entryhead:full}{\printfield{labeltitle}}%
\DeclareFieldFormat{url}{\newline\footnotesize\url{#1}}%

\AtEndDocument{\printbibliography}

\begin{document}
	%<*content>
	\lesson{algebra}{156}{Endomorphismes trigonalisables. Endomorphismes nilpotents.}

	Soit $E$ un espace vectoriel de dimension finie $n$ sur un corps $\mathbb{K}$. Tout au long de la leçon, on abusera du fait que $\mathcal{L}(E) \cong \mathcal{M}_n(\mathbb{K})$ : les notions définies pour les endomorphismes sont valables pour les matrices.

	\subsection{Endomorphismes trigonalisables}

	\subsubsection{Premiers outils de réduction}

	\reference[GOU21]{171}

	\begin{definition}
		Soient $u \in \mathcal{L}(E)$ et $\lambda \in \mathbb{K}$.
		\begin{itemize}
			\item On dit que $\lambda$ est \textbf{valeur propre} de $u$ si $u - \lambda \operatorname{id}_E$ est non injective.
			\item Un vecteur $x \neq 0$ tel que $u(x) = \lambda x$ est un \textbf{vecteur propre} de $u$ associé à la valeur propre $\lambda$.
			\item L'ensemble des valeurs propres de $u$ est appelé \textbf{spectre} de $u$. On le note $\operatorname{Sp}(u)$.
		\end{itemize}
	\end{definition}

	\begin{remark}
		Soit $u \in \mathcal{L}(E)$.
		\begin{itemize}
			\item $0$ est valeur propre de $u$ si et seulement si $\ker(f) \neq \{ 0 \}$.
			\item On peut définir de la même manière les mêmes notions pour une matrice de $\mathcal{M}_n(\mathbb{K})$ (une valeur est propre pour une matrice si et seulement si elle l'est pour l'endomorphisme associé). On reprendra les mêmes notations.
		\end{itemize}
	\end{remark}

	\begin{example}
		$\begin{pmatrix} 1 \\ 1 \\ 1 \end{pmatrix}$ est vecteur propre de $\begin{pmatrix} 0 & 2 & -1 \\ 3 & -2 & 0 \\ -2 & 2 & 1 \end{pmatrix}$ associé à la valeur propre $1$.
	\end{example}

	\reference[ROM21]{644}

	\begin{proposition}
		Soit $u \in \mathcal{L}(E)$. En notant $\chi_u = \det(X \operatorname{id}_E - u)$,
		\[ \operatorname{Sp}(u) = \{ \lambda \in \mathbb{K} \mid \chi_u(\lambda) = 0 \} \]
	\end{proposition}

	\begin{definition}
		Le polynôme $\chi_u$ précédent est appelé \textbf{polynôme caractéristique} de $u$.
	\end{definition}

	\begin{remark}
		On peut définir la même notion pour une matrice $A \in \mathcal{M}_n(\mathbb{K})$, ces deux notions coïncidant bien si $A$ est la matrice de $u$ dans une base quelconque de $E$.
	\end{remark}

	\begin{example}
		Pour $A = \begin{pmatrix} a & b \\ c & d \end{pmatrix} \in \mathcal{M}_2(\mathbb{K})$, on a $\chi_A = X^2 - \trace(A)X + \det(A)$.
	\end{example}

	\reference[ROM21]{604}

	\begin{lemma}
		Soit $u \in \mathcal{L}(E)$.
		\begin{enumerate}[label=(\roman*)]
			\item $\mathrm{Ann}(u) = \{ P \in \mathbb{K}[X] \mid P(u) = 0 \}$ est un sous-ensemble de $\mathbb{K}[u]$ non réduit au polynôme nul.
			\item $\mathrm{Ann}(u)$ est le noyau de $P \mapsto P(u)$ : c'est un idéal de $\mathbb{K}[u]$.
			\item Il existe un unique polynôme unitaire engendrant cet idéal.
		\end{enumerate}
	\end{lemma}

	\begin{definition}
		On appelle \textbf{idéal annulateur} de $u$ l'idéal $\mathrm{Ann}(u)$. Le polynôme unitaire générateur est noté $\pi_u$ et est appelé \textbf{polynôme minimal} de $u$.
	\end{definition}

	\begin{remark}
		En reprenant les notations précédentes,
		\begin{itemize}
			\item $\pi_u$ est le polynôme unitaire de plus petit degré annulant $u$.
			\item Si $A \in \mathcal{M}_n(\mathbb{K})$ est la matrice de $u$ dans une base de $E$, on a $\mathrm{Ann}(u) = \mathrm{Ann}(A)$ et $\pi_u = \pi_A$.
		\end{itemize}
	\end{remark}

	\begin{example}
		Un endomorphisme est nilpotent d'indice $q$ si et seulement si son polynôme minimal est $X^q$.
	\end{example}

	\begin{proposition}
		Soit $u \in \mathcal{L}(E)$. Soit $F$ un sous-espace vectoriel de $E$ stable par $u$. Alors, le polynôme minimal de l'endomorphisme $u_{|F} : F \rightarrow F$ divise $\pi_u$.
	\end{proposition}

	\begin{proposition}
		Soit $u \in \mathcal{L}(E)$.
		\begin{enumerate}[label=(\roman*)]
			\item Les valeurs propres de $u$ sont racines de tout polynôme annulateur.
			\item Les valeurs propres de $u$ sont exactement les racines de $\pi_u$.
		\end{enumerate}
	\end{proposition}

	\reference[GOU21]{186}

	\begin{remark}
		Soit $u \in \mathcal{L}(E)$. $\pi_u$ et $\chi_u$ partagent dont les mêmes racines.
	\end{remark}

	\reference[ROM21]{607}

	\begin{theorem}[Cayley-Hamilton]
		Soit $u \in \mathcal{L}(E)$. Alors,
		\[ \pi_u \mid \chi_u \]
	\end{theorem}

	\reference{609}

	\begin{theorem}[Lemme des noyaux]
		Soit $P = P_1 \dots P_k \in \mathbb{K}[X]$ où les polynômes $P_1, \dots, P_k$ sont premiers entre eux deux à deux. Alors, pour tout endomorphisme $u$ de $E$,
		\[ \ker(P(u)) = \bigoplus_{i=1}^k \ker(P_i(u)) \]
	\end{theorem}

	\subsubsection{Trigonalisation}

	\reference{675}

	\begin{definition}
		Soit $u \in \mathcal{L}(E)$.
		\begin{itemize}
			\item On dit que $u$ est \textbf{trigonalisable} s'il existe une base de $E$ dans laquelle la matrice de $u$ est triangulaire supérieure.
			\item On dit qu'une matrice $A \in \mathcal{M}_n(\mathbb{K})$ est \textbf{trigonalisable} si elle est semblable à une matrice diagonale.
		\end{itemize}
	\end{definition}

	\begin{remark}
		Un endomorphisme $u$ de $E$ est trigonalisable si et seulement si sa matrice dans n'importe quelle base de $E$ l'est.
	\end{remark}

	\begin{example}
		Une matrice à coefficients réels ayant des valeurs propres imaginaires pures n'est pas trigonalisable dans $\mathcal{M}_n(\mathbb{R})$.
	\end{example}

	\begin{theorem}
		Un endomorphisme $u$ de $E$ est trigonalisable sur $\mathbb{K}$ si et seulement si $\chi_u$ est scindé sur $\mathbb{K}$.
	\end{theorem}

	\begin{corollary}
		Si $\mathbb{K}$ est algébriquement clos, tout endomorphisme de $u$ est trigonalisable sur $\mathbb{K}$.
	\end{corollary}

	\begin{proposition}
		Soit $u \in \mathcal{L}(E)$. Si $u$ est trigonalisable, sa trace est la somme de ses valeurs propres et son déterminant est le produit de ses valeurs propres.
	\end{proposition}

	\begin{theorem}[Trigonalisation simultanée]
		Soit $(u_i)_{i \in I}$ une famille d'endomorphismes de $E$ diagonalisables qui commutent deux-à-deux. Alors, il existe une base commune de trigonalisation.
	\end{theorem}

	\newpage

	\subsection{Endomorphismes nilpotents}

	\subsubsection{Définition, caractérisation}

	\reference[BMP]{168}

	\begin{definition}
		On note
		\[ \mathcal{N}(E) = \{ u \in \mathcal{L}(E) \mid \exists p \in \mathbb{N} \text{ tel que } u^p = 0 \} \]
		l'ensemble des éléments \textbf{nilpotents} de $\mathcal{L}(E)$.
	\end{definition}

	\begin{example}
		Dans $\mathbb{K}_n[X]$, l'opérateur de dérivation $P \mapsto P'$ est nilpotent.
	\end{example}

	\begin{definition}
		On appelle \textbf{indice de nilpotence} d'un endomorphisme $u \in \mathcal{N}(E)$ l'entier $q$ tel que
		\[ q = \inf \{ p \in \mathbb{N} \mid u^p = 0 \} \]
	\end{definition}

	\begin{proposition}
		Soit $u \in \mathcal{L}(E)$. Alors,
		\[ u \text{ est nilpotent d'indice } p \iff \pi_u = X^p \]
		En particulier, $p \leq n$.
	\end{proposition}

	\begin{theorem}
		\label{156-1}
		Soit $u \in \mathcal{L}(E)$. Les assertions suivantes sont équivalentes :
		\begin{enumerate}[label=(\roman*)]
			\item $u \in \mathcal{N}(E)$.
			\item $\chi_u = (-1)^n X^n$.
			\item Il existe $p \in \mathbb{N}$ tel que $\pi_u = X^p$. Dans ce cas, $p$ est l'indice de nilpotence de $u$.
			\item $u$ est trigonalisable avec zéros sur la diagonale.
			\item $u$ est trigonalisable et sa seule valeur propre est $0$.
			\item $0$ est la seule valeur propre de $u$ dans toute extension algébrique de $\mathbb{K}$.
			\item \underline{Si $\operatorname{car}(\mathbb{K}) = 0$ :} $u$ et $\lambda u$ sont semblables pour tout $\lambda \in \mathbb{K}^*$.
		\end{enumerate}
	\end{theorem}

	\begin{cexample}
		La matrice
		\[
			A = \begin{pmatrix}
				0 & 0 & 0 \\
				0 & 0 & -1 \\
				0 & 1 & 0
			\end{pmatrix}
		\]
		n'est pas nilpotente, alors que $\chi_A = -X(X^2 + 1)$ n'admet que $0$ comme valeur propre réelle.
	\end{cexample}

	\begin{proposition}
		Soit $u \in \mathcal{L}(E)$. On suppose $\operatorname{car}(\mathbb{K}) = 0$. Alors,
		\[ u \in \mathcal{N}(E) \iff \forall k \in \mathbb{N}, \, \trace(u^k) = 0 \]
	\end{proposition}

	\subsubsection{Cône nilpotent}

	\begin{proposition}
		$\mathcal{N}(E)$ est un cône : si $u \in \mathcal{N}(E)$, alors $\forall \lambda \in \mathbb{K}, \, \lambda u \in \mathcal{N}(E)$.
	\end{proposition}

	\begin{remark}
		$\mathcal{N}(E)$ n'est pas un sous-groupe additif de $\mathcal{L}(E)$. Par exemple,
		\[
			A =
			\begin{pmatrix}
				0 & 1 \\
				0 & 0
			\end{pmatrix}
			+
			\begin{pmatrix}
				0 & 0 \\
				1 & 0
			\end{pmatrix}
		\]
		$A$ est somme de deux matrices nilpotentes, mais est inversible donc non nilpotente. En particulier, $\mathcal{N}(E)$ n'est ni un idéal, ni un sous-espace vectoriel de $\mathcal{L}(E)$.
	\end{remark}

	\begin{proposition}
		\[ \operatorname{Vect}(\mathcal{N}(E)) = \ker(\trace) \]
	\end{proposition}

	\begin{example}
		En dimension $2$,
		\[
			\begin{pmatrix}
				a & b \\
				c & d
			\end{pmatrix}
			\text{ est nilpotente}
			\iff
			-a^2 -bc = 0
		\]
	\end{example}

	\begin{proposition}
		Soient $u, v \in \mathcal{L}(E)$ tels que $uv = vu$.
		\begin{enumerate}[label=(\roman*)]
			\item Si $u, v \in \mathcal{N}(E)$, alors $u + v \in \mathcal{N}(E)$.
			\item Si $u \in \mathcal{N}(E)$, alors $uv \in \mathcal{N}(E)$.
		\end{enumerate}
	\end{proposition}

	\subsubsection{Unipotence}

	\reference{174}

	\begin{definition}
		On note
		\[ \mathcal{U}(E) = \operatorname{id}_E + \mathcal{N}(E) \]
		l'ensemble des endomorphismes \textbf{unipotents} de $E$.
	\end{definition}

	\begin{remark}
		On dispose de caractérisations analogues au \cref{156-1} pour les endomorphismes unipotents. Par exemple, un endomorphisme $u$ de $E$ est unipotent si et seulement si $\chi_u = (1-X)^n$.
	\end{remark}

	On se place dans le cas où $\mathbb{K} = \mathbb{R}$ ou $\mathbb{C}$ pour la fin de cette sous-section.

	\reference[ROM21]{767}

	\begin{proposition}
		Soit $u \in \mathcal{N}(E)$. Alors $e^u \in \mathcal{U}(E)$ et $\ln(e^{tA}) = tA$ pour tout $t \in \mathbb{R}$.
	\end{proposition}

	\begin{theorem}
		L'exponentielle matricielle réalise une bijection de $\mathcal{N}(E)$ sur $\mathcal{U}(E)$ d'inverse le logarithme matriciel.
	\end{theorem}

	\subsubsection{Sous-espaces caractéristiques, noyaux itérés}

	\reference[GOU21]{201}

	Soit $u \in \mathcal{L}(E)$ de polynôme caractéristique scindé, de la forme
	\[ \chi_u = \prod_{i=1}^{s} (X-\lambda_i)^{\alpha_i} \]

	\begin{definition}
		Soit $i \in \llbracket 1, s \rrbracket$. On appelle \textbf{sous-espace caractéristique} de $u$ associé à la valeur propre $\lambda_i$ le sous-espace vectoriel $N_i = \ker((u-\lambda_i \operatorname{id}_E)^{\alpha_i})$.
	\end{definition}

	\begin{proposition}
		Soit $i \in \llbracket 1, s \rrbracket$.
		\begin{enumerate}[label=(\roman*)]
			\item $N_i$ est stable par $u$.
			\item $\dim(N_i) = \alpha_i$.
			\item $\chi_{u_{|N_i}} = (-X)^{\dim(N_i)} = (-X)^{\alpha_i}$.
			\item $u_{|N_i}$ est nilpotent.
		\end{enumerate}
		De plus, $E = \oplus_{i=1}^s N_i$.
	\end{proposition}

	\begin{proposition}
		Soit $v \in \mathcal{L}(E)$.
		\begin{enumerate}[label=(\roman*)]
			\item La suite de sous-espaces vectoriels $(\ker(v^n))$ est décroissante, stationnaire.
			\item La suite de sous-espaces vectoriels $(\im(v^n))$ est croissante, stationnaire.
		\end{enumerate}
	\end{proposition}

	\reference[BMP]{171}

	\begin{definition}
		Un \textbf{bloc de Jordan} de taille $m$ associé à $\lambda \in \mathbb{K}$ désigne la matrice $J_m(\lambda)$ suivante :
		\[ J_m(\lambda) = \begin{pmatrix} \lambda & 1 & \\ & \ddots & \ddots & \\ & & \ddots & 1 \\ & & & \lambda \end{pmatrix} \in \mathcal{M}_m(\mathbb{K}) \]
	\end{definition}

	\begin{application}[Réduction de Jordan d'un endomorphisme nilpotent]
		On suppose que $u$ est nilpotent. Alors il existe des entiers $n_1 \geq \dots \geq n_p$ et une base $\mathcal{B}$ de $E$ tels que :
		\[ \operatorname{Mat}(u, \mathcal{B}) = \begin{pmatrix} J_{n_1}(0) & & \\ & \ddots & \\ & & J_{n_p}(0) \end{pmatrix} \]
		De plus, on a unicité dans cette décomposition.
	\end{application}

	\subsection{Applications}

	\subsubsection{Décomposition de Dunford}

	\reference[GOU21]{203}
	\dev{decomposition-de-dunford}

	\begin{theorem}[Décomposition de Dunford]
		Soit $u \in \mathcal{L}(E)$. On suppose que $\pi_u$ est scindé sur $\mathbb{K}$. Alors il existe un unique couple d'endomorphismes $(d, n)$ tels que :
		\begin{itemize}
			\item $d$ est diagonalisable et $n$ est nilpotent.
			\item $u = d + n$.
			\item $d n = n d$.
		\end{itemize}
	\end{theorem}

	\begin{corollary}
		Si $u$ vérifie les hypothèse précédentes, pour tout $k \in \mathbb{N}$, $u^k = (d + n)^k = \sum_{i=0}^m \binom{k}{i} d^i n^{k-i}$, avec $m = \min(k, l)$ où $l$ désigne l'indice de nilpotence de $n$.
	\end{corollary}

	\begin{remark}
		On peut montrer de plus que $d$ et $n$ sont des polynômes en $u$.
	\end{remark}

	\reference[ROM21]{687}

	\begin{theorem}[Décomposition de Dunford multiplicative]
		Soit $f \in \mathcal{L}(E)$. On suppose que $\pi_f$ est scindé sur $\mathbb{K}$. Alors il existe un unique couple d'endomorphismes $(d, u)$ tels que :
		\begin{itemize}
			\item $d$ est diagonalisable et $u$ est unipotente.
			\item $f = du$.
			\item $du = ud$.
		\end{itemize}
	\end{theorem}

	\subsubsection{Invariants de similitude}

	\reference[GOU21]{397}

	Soient $E$ un espace vectoriel de dimension finie $n$ et $u \in \mathcal{L}(E)$.

	\begin{definition}
		On dit que $u$ est \textbf{cyclique} s'il existe $x \in E$ tel que $\{ P(u)(x) \mid P \in \mathbb{K}[X] \} = E$.
	\end{definition}

	\begin{proposition}
		$u$ est cyclique si et seulement si $\deg(\pi_u) = n$.
	\end{proposition}

	\begin{definition}
		Soit $P = X^p + a_{p-1} X^{p-1} + \dots + a_0 \in \mathbb{K}[X]$. On appelle \textbf{matrice compagnon} de $P$ la matrice
		\[ \mathcal{C}(P) = \begin{pmatrix} 0 & \dots & \dots & 0 & -a_0 \\ 1 & 0 & \ddots & \vdots & -a_1 \\ 0 & 1 & \ddots & \vdots & \vdots \\ \vdots & \ddots & \ddots & 0 & -a_{p-2} \\ 0 & \dots & 0 & 1 & -a_{p-1} \end{pmatrix} \]
	\end{definition}

	\begin{proposition}
		$u$ est cyclique si et seulement s'il existe une base $\mathcal{B}$ de $E$ telle que $\operatorname{Mat}(u, \mathcal{B}) = \mathcal{C}(\pi_u)$.
	\end{proposition}

	\begin{theorem}
		Il existe $F_1, \dots, F_r$ des sous-espaces vectoriels de $E$ tous stables par $u$ tels que :
		\begin{itemize}
			\item $E = F_1 \oplus \dots \oplus F_r$.
			\item $u_i = u_{|F_i}$ est cyclique pour tout $i$.
			\item Si $P_i = \pi_{u_i}$, on a $P_{i+1} \mid P_i$ pour tout $i$.
		\end{itemize}
		La famille de polynômes $P_1, \dots, P_r$ ne dépend que de $u$ et non du choix de la décomposition. On l'appelle \textbf{suite des invariants de similitude} de $u$.
	\end{theorem}

	\begin{theorem}[Réduction de Frobenius]
		Si $P_1, \dots, P_r$ désigne la suite des invariants de $u$, alors il existe une base $\mathcal{B}$ de $E$ telle que :
		\[ \operatorname{Mat}(u, \mathcal{B}) = \begin{pmatrix} \mathcal{C}(P_1) & & \\ & \ddots & \\ & & \mathcal{C}(P_r) \end{pmatrix} \]
		On a d'ailleurs $P_1 = \pi_u$ et $P_1 \dots P_r = \chi_u$.
	\end{theorem}

	\begin{corollary}
		Deux endomorphismes de $E$ sont semblables si et seulement s'ils ont la même suite d'invariants de similitude.
	\end{corollary}

	\begin{application}
		Pour $n = 2$ ou $3$, deux matrices sont semblables si et seulement si elles ont mêmes polynômes minimal et caractéristique.
	\end{application}

	\begin{application}
		Soit $\mathbb{L}$ une extension de $\mathbb{K}$. Alors, si $A, B \in \mathcal{M}_n(\mathbb{K})$ sont semblables dans $\mathcal{M}_n(\mathbb{L})$, elles le sont aussi dans $\mathcal{M}_n(\mathbb{K})$.
	\end{application}

	\subsubsection{Commutant d'une matrice}

	Soit $A \in \mathcal{M}_n(\mathbb{K})$.

	\reference{289}

	\begin{lemma}
		Si $\pi_A = \chi_A$, alors $A$ est cyclique :
		\[ \exists x \in \mathbb{K}^n \setminus \{ 0 \} \text{ tel que } (x, Ax, \dots, A^{n-1}x) \text{ est une base de } \mathbb{K}^n \]
	\end{lemma}

	\reference[FGN2]{160}

	\begin{notation}
		\begin{itemize}
			\item On note $\mathcal{T}_n(\mathbb{K})$ l'ensemble des matrices carrées triangulaires supérieures d'ordre $n$ à coefficients dans le corps $\mathbb{K}$.
			\item On note $\mathcal{C}(A)$ le commutant de $A$.
		\end{itemize}
	\end{notation}

	\begin{lemma}
		\[ \dim_{\mathbb{K}}(\mathcal{C}(A)) \geq n \]
	\end{lemma}

	\begin{lemma}
		Le rang de $A$ est invariant par extension de corps.
	\end{lemma}

	\dev{dimension-du-commutant}

	\begin{theorem}
		\[ \mathbb{K}[A] = \mathcal{C}(A) \iff \pi_A = \chi_A \]
	\end{theorem}
	%</content>
\end{document}
