\documentclass[12pt, a4paper]{report}

% LuaLaTeX :

\RequirePackage{iftex}
\RequireLuaTeX

% Packages :

\usepackage[french]{babel}
%\usepackage[utf8]{inputenc}
%\usepackage[T1]{fontenc}
\usepackage[pdfencoding=auto, pdfauthor={Hugo Delaunay}, pdfsubject={Mathématiques}, pdfcreator={agreg.skyost.eu}]{hyperref}
\usepackage{amsmath}
\usepackage{amsthm}
%\usepackage{amssymb}
\usepackage{stmaryrd}
\usepackage{tikz}
\usepackage{tkz-euclide}
\usepackage{fourier-otf}
\usepackage{fontspec}
\usepackage{titlesec}
\usepackage{fancyhdr}
\usepackage{catchfilebetweentags}
\usepackage[french, capitalise, noabbrev]{cleveref}
\usepackage[fit, breakall]{truncate}
\usepackage[top=2.5cm, right=2cm, bottom=2.5cm, left=2cm]{geometry}
\usepackage{enumerate}
\usepackage{tocloft}
\usepackage{microtype}
%\usepackage{mdframed}
%\usepackage{thmtools}
\usepackage{xcolor}
\usepackage{tabularx}
\usepackage{aligned-overset}
\usepackage[subpreambles=true]{standalone}
\usepackage{environ}
\usepackage[normalem]{ulem}
\usepackage{marginnote}
\usepackage{etoolbox}
\usepackage{setspace}
\usepackage[bibstyle=reading, citestyle=draft]{biblatex}
\usepackage{xpatch}
\usepackage[many, breakable]{tcolorbox}
\usepackage[backgroundcolor=white, bordercolor=white, textsize=small]{todonotes}

% Bibliographie :

\newcommand{\overridebibliographypath}[1]{\providecommand{\bibliographypath}{#1}}
\overridebibliographypath{../bibliography.bib}
\addbibresource{\bibliographypath}
\defbibheading{bibliography}[\bibname]{%
	\newpage
	\section*{#1}%
}
\renewbibmacro*{entryhead:full}{\printfield{labeltitle}}
\DeclareFieldFormat{url}{\newline\footnotesize\url{#1}}
\AtEndDocument{\printbibliography}

% Police :

\setmathfont{Erewhon Math}

% Tikz :

\usetikzlibrary{calc}

% Longueurs :

\setlength{\parindent}{0pt}
\setlength{\headheight}{15pt}
\setlength{\fboxsep}{0pt}
\titlespacing*{\chapter}{0pt}{-20pt}{10pt}
\setlength{\marginparwidth}{1.5cm}
\setstretch{1.1}

% Métadonnées :

\author{agreg.skyost.eu}
\date{\today}

% Titres :

\setcounter{secnumdepth}{3}

\renewcommand{\thechapter}{\Roman{chapter}}
\renewcommand{\thesubsection}{\Roman{subsection}}
\renewcommand{\thesubsubsection}{\arabic{subsubsection}}
\renewcommand{\theparagraph}{\alph{paragraph}}

\titleformat{\chapter}{\huge\bfseries}{\thechapter}{20pt}{\huge\bfseries}
\titleformat*{\section}{\LARGE\bfseries}
\titleformat{\subsection}{\Large\bfseries}{\thesubsection \, - \,}{0pt}{\Large\bfseries}
\titleformat{\subsubsection}{\large\bfseries}{\thesubsubsection. \,}{0pt}{\large\bfseries}
\titleformat{\paragraph}{\bfseries}{\theparagraph. \,}{0pt}{\bfseries}

\setcounter{secnumdepth}{4}

% Table des matières :

\renewcommand{\cftsecleader}{\cftdotfill{\cftdotsep}}
\addtolength{\cftsecnumwidth}{10pt}

% Redéfinition des commandes :

\renewcommand*\thesection{\arabic{section}}
\renewcommand{\ker}{\mathrm{Ker}}

% Nouvelles commandes :

\newcommand{\website}{https://agreg.skyost.eu}

\newcommand{\tr}[1]{\mathstrut ^t #1}
\newcommand{\im}{\mathrm{Im}}
\newcommand{\rang}{\operatorname{rang}}
\newcommand{\trace}{\operatorname{trace}}
\newcommand{\id}{\operatorname{id}}
\newcommand{\stab}{\operatorname{Stab}}

\providecommand{\newpar}{\\[\medskipamount]}

\providecommand{\lesson}[3]{%
	\title{#3}%
	\hypersetup{pdftitle={#3}}%
	\setcounter{section}{\numexpr #2 - 1}%
	\section{#3}%
	\fancyhead[R]{\truncate{0.73\textwidth}{#2 : #3}}%
}

\providecommand{\development}[3]{%
	\title{#3}%
	\hypersetup{pdftitle={#3}}%
	\section*{#3}%
	\fancyhead[R]{\truncate{0.73\textwidth}{#3}}%
}

\providecommand{\summary}[1]{%
	\textit{#1}%
	\medskip%
}

\tikzset{notestyleraw/.append style={inner sep=0pt, rounded corners=0pt, align=center}}

%\newcommand{\booklink}[1]{\website/bibliographie\##1}
\newcommand{\citelink}[2]{\hyperlink{cite.\therefsection @#1}{#2}}
\newcommand{\previousreference}{}
\providecommand{\reference}[2][]{%
	\notblank{#1}{\renewcommand{\previousreference}{#1}}{}%
	\todo[noline]{%
		\protect\vspace{16pt}%
		\protect\par%
		\protect\notblank{#1}{\cite{[\previousreference]}\\}{}%
		\protect\citelink{\previousreference}{p. #2}%
	}%
}

\definecolor{devcolor}{HTML}{00695c}
\newcommand{\dev}[1]{%
	\reversemarginpar%
	\todo[noline]{
		\protect\vspace{16pt}%
		\protect\par%
		\bfseries\color{devcolor}\href{\website/developpements/#1}{DEV}
	}%
	\normalmarginpar%
}

% En-têtes :

\pagestyle{fancy}
\fancyhead[L]{\truncate{0.23\textwidth}{\thepage}}
\fancyfoot[C]{\scriptsize \href{\website}{\texttt{agreg.skyost.eu}}}

% Couleurs :

\definecolor{property}{HTML}{fffde7}
\definecolor{proposition}{HTML}{fff8e1}
\definecolor{lemma}{HTML}{fff3e0}
\definecolor{theorem}{HTML}{fce4f2}
\definecolor{corollary}{HTML}{ffebee}
\definecolor{definition}{HTML}{ede7f6}
\definecolor{notation}{HTML}{f3e5f5}
\definecolor{example}{HTML}{e0f7fa}
\definecolor{cexample}{HTML}{efebe9}
\definecolor{application}{HTML}{e0f2f1}
\definecolor{remark}{HTML}{e8f5e9}
\definecolor{proof}{HTML}{e1f5fe}

% Théorèmes :

\theoremstyle{definition}
\newtheorem{theorem}{Théorème}

\newtheorem{property}[theorem]{Propriété}
\newtheorem{proposition}[theorem]{Proposition}
\newtheorem{lemma}[theorem]{Lemme}
\newtheorem{corollary}[theorem]{Corollaire}

\newtheorem{definition}[theorem]{Définition}
\newtheorem{notation}[theorem]{Notation}

\newtheorem{example}[theorem]{Exemple}
\newtheorem{cexample}[theorem]{Contre-exemple}
\newtheorem{application}[theorem]{Application}

\theoremstyle{remark}
\newtheorem{remark}[theorem]{Remarque}

\counterwithin*{theorem}{section}

\newcommand{\applystyletotheorem}[1]{
	\tcolorboxenvironment{#1}{
		enhanced,
		breakable,
		colback=#1!98!white,
		boxrule=0pt,
		boxsep=0pt,
		left=8pt,
		right=8pt,
		top=8pt,
		bottom=8pt,
		sharp corners,
		after=\par,
	}
}

\applystyletotheorem{property}
\applystyletotheorem{proposition}
\applystyletotheorem{lemma}
\applystyletotheorem{theorem}
\applystyletotheorem{corollary}
\applystyletotheorem{definition}
\applystyletotheorem{notation}
\applystyletotheorem{example}
\applystyletotheorem{cexample}
\applystyletotheorem{application}
\applystyletotheorem{remark}
\applystyletotheorem{proof}

% Environnements :

\NewEnviron{whitetabularx}[1]{%
	\renewcommand{\arraystretch}{2.5}
	\colorbox{white}{%
		\begin{tabularx}{\textwidth}{#1}%
			\BODY%
		\end{tabularx}%
	}%
}

% Maths :

\DeclareFontEncoding{FMS}{}{}
\DeclareFontSubstitution{FMS}{futm}{m}{n}
\DeclareFontEncoding{FMX}{}{}
\DeclareFontSubstitution{FMX}{futm}{m}{n}
\DeclareSymbolFont{fouriersymbols}{FMS}{futm}{m}{n}
\DeclareSymbolFont{fourierlargesymbols}{FMX}{futm}{m}{n}
\DeclareMathDelimiter{\VERT}{\mathord}{fouriersymbols}{152}{fourierlargesymbols}{147}


% Bibliographie :

\addbibresource{\bibliographypath}%
\defbibheading{bibliography}[\bibname]{%
	\newpage
	\section*{#1}%
}
\renewbibmacro*{entryhead:full}{\printfield{labeltitle}}%
\DeclareFieldFormat{url}{\newline\footnotesize\url{#1}}%

\AtEndDocument{\printbibliography}

\begin{document}
  %<*content>
  \development{analysis}{theoreme-de-weierstrass-par-la-convolution}{Théorème de Weierstrass (par la convolution)}

  \summary{On montre le théorème de Weierstrass par la convolution (sans forcément développer toute la théorie derrière, ce qui peut être utile dans certaines leçons).}

  \reference[GOU20]{304}

  \begin{notation}
    $\forall n \in \mathbb{N}$, on note :
    \[ a_n = \int_{-1}^1 (1-t^2)^n \, \mathrm{d}t \text{ et } p_n : t \mapsto \frac{(1-t^2)^n}{a_n} \mathbb{1}_{[-1, 1]}(t) \]
  \end{notation}

  \begin{lemma}
    \label{theoreme-de-weierstrass-par-la-convolution-1}
    La suite $(p_n)$ vérifie :
    \begin{enumerate}[label=(\roman*)]
      \item $\forall n \in \mathbb{N}$, $p_n \geq 0$.
      \item $\forall n \in \mathbb{N}$, $\int_{\mathbb{R}} p_n(t) = 1$.
      \item \label{theoreme-de-weierstrass-par-la-convolution-2} $\forall \alpha > 0$, $\lim_{n \rightarrow +\infty} \int_{|t| > \alpha} p_n(t) \, \mathrm{d}t = 0$.
    \end{enumerate}
    Autrement dit, $(p_n)$ est une \textbf{approximation positive de l'identité}.
  \end{lemma}

  \begin{proof}
    Notons tout d'abord que
    \[ \forall n \in \mathbb{N}^*, \, a_n = 2 \int_0^1 (1-t^2)^n \, \mathrm{d}t \geq 2 \int_0^1 t (1-t^2)^n \, \mathrm{d}t = \left[ - \frac{(1-t^2)^{n+1}}{n+1} \right]_0^1 = \frac{1}{n+1}  \]
    \begin{enumerate}[label=(\roman*)]
      \item $\forall n \in \mathbb{N}$, $p_n \geq 0$ car $a_n \geq 0$ et $(1-t^2)^n \geq 0$ pour tout $t \in [-1, 1]$.
      \item $\forall n \in \mathbb{N}$, $\int_{\mathbb{R}} p_n(t) \, \mathrm{d}t = \frac{1}{a_n} \int_{-1}^1 (1-t^2)^n \, \mathrm{d}t = 1$.
      \item Soit $\alpha > 0$.
      \begin{itemize}
        \item \uline{Si $\alpha < 1$ :} $\forall n \in \mathbb{N}^*$,
        \[ \int_{|t| \geq \alpha} p_n(t) \, \mathrm{d}t = \frac{2}{a_n} \int_\alpha^1 (1-t^2)^n \, \mathrm{d}t \leq \frac{2}{a_n} (1-\alpha^2)^n \leq 2(n+1)(1-\alpha^2)^n \]
        et comme $|1-\alpha^2| < 1$, on a $\int_{|t| \geq \alpha} p_n(t) \, \mathrm{d}t \longrightarrow 0$.
        \item \uline{Si $\alpha \geq 1$ :}
        \[ \int_{|t| \geq \alpha} p_n(t) \, \mathrm{d}t = 0 \]
      \end{itemize}
    \end{enumerate}
  \end{proof}

  \begin{theorem}[Weierstrass]
    Toute fonction continue $f : [a,b] \rightarrow \mathbb{R}$ (avec $a, b \in \mathbb{R}$ tels que $a \leq b$) est limite uniforme de fonctions polynômiales sur $[a, b]$.
  \end{theorem}

  \begin{proof}
    Soit $f \in \mathcal{C}_C(\mathbb{R})$ continue. Montrons que $(f * p_n)$ converge uniformément vers $f$. Soit $\epsilon > 0$. Par le théorème de Heine $f$ est uniformément continue sur son support, donc l'est aussi sur $\mathbb{R}$ entier :
    \[ \exists \eta > 0 \text{ tel que } \forall x, y \in \mathbb{R}, \, |x-y| < \eta \implies |f(x) - f(y)| < \epsilon \]
    De plus, $f$ est bornée et atteint ses bornes (donc écrire $\Vert f \Vert_\infty$ a du sens). On peut appliquer le \cref{theoreme-de-weierstrass-par-la-convolution-1} \cref{theoreme-de-weierstrass-par-la-convolution-2} :
    \[ \exists N \in \mathbb{N} \text{ tel que } \forall n \geq N, \, \int_{|t| \geq \eta} p_n(t) \, \mathrm{d}t < \epsilon \]
    Donc, toujours avec le \cref{theoreme-de-weierstrass-par-la-convolution-1}, pour tout $n \geq N$ et pour tout $x \in \mathbb{R}$,
    \begin{align*}
      |f*p_n(x) - f(x)| \overset{(ii)}&{=} \left| \int_{\mathbb{R}} f(x-t) p_n(t) \, \mathrm{d}t - f(x) \int_{\mathbb{R}} p_n(t) \, \mathrm{d}t \right| \\
      &= \left| \int_{\mathbb{R}} (f(x-t) - f(x)) p_n(t) \, \mathrm{d}t \right| \\
      &\leq \int_{\mathbb{R}} \left| (f(x-t) - f(x)) p_n(t)  \right| \, \mathrm{d}t \\
      \overset{(i)}&{=} \int_{\mathbb{R}} \left| f(x-t) - f(x) \right| p_n(t)  \, \mathrm{d}t \\
      &= \int_{|t| \geq \eta} \left| f(x-t) - f(x) \right| p_n(t)  \, \mathrm{d}t + \int_{-\eta}^\eta \left| f(x-t) - f(x) \right| p_n(t)  \, \mathrm{d}t \\
      &= 2 \Vert f \Vert_\infty \epsilon + \epsilon \int_{-\eta}^\eta p_n(t)  \, \mathrm{d}t \\
      \overset{(i)}&{\leq} 2 \Vert f \Vert_\infty \epsilon + \epsilon \int_{\mathbb{R}} p_n(t) \, \mathrm{d}t \\
      &= (2 \Vert f \Vert_\infty + 1) \epsilon
    \end{align*}
    d'où la convergence uniforme. Soit maintenant $n \in \mathbb{N}$. Supposons que $f$ est à support dans $I = \left[ -\frac{1}{2}, \frac{1}{2} \right]$ et montrons que pour tout $f * p_n$ est une fonction polynômiale.
    \[ \forall x \in I, \, (f*p_n)(x) = (p_n*f)(x) = \int_{-\frac{1}{2}}^{\frac{1}{2}} p_n(x-t) f(t) \, \mathrm{d}t \tag{$*$} \]
    Notons que $\forall x, t \in I$, $|x-t| \leq 1$, donc
    \[ p_n(x-t) = \frac{(1 - (x-t)^2)^n}{a_n} \overset{\text{développement}}{=} \sum_{k=0}^{2n} q_k(t) x^k \]
    où $\forall k \in \llbracket 0, 2n \rrbracket$, $q_k$ est une fonction polynômiale. En remplaçant dans $(*)$, on obtient :
    \[ \forall x \in I, \, (f*p_n)(x) = \sum_{k=0}^{2n} \left( \int_{-\frac{1}{2}}^{\frac{1}{2}} q_k(t) f(t) \, \mathrm{d}t \right) x^k \]
    qui est bien une fonction polynômiale sur $I$.
    \newpar
    Soient maintenant $[a,b]$ un intervalle fermé de $\mathbb{R}$ et $f : [a, b] \rightarrow \mathbb{R}$. On considère $[c, d]$ un intervalle plus grand avec $c < a$ et $b < d$ et on prolonge $f$ par :
    \begin{itemize}
      \item Une fonction affine sur $[c, a]$ qui vaut $0$ en $c$ et $f(a)$ en $a$.
      \item Une fonction affine sur $[b, d]$ qui vaut $0$ en $d$ et $f(b)$ en $b$.
    \end{itemize}
    Et on peut encore prolonger cette fonction sur $\mathbb{R}$ tout entier en une fonction $\widetilde{f}$ telle que $\widetilde{f} = 0$ pour tout $x \notin [c, d]$. On a donc $\widetilde{f} \in \mathcal{C}_C(\mathbb{R})$. Nous allons maintenant avoir besoin du changement de variable suivant :
    \[ \varphi :
    \begin{array}{ccc}
      I &\rightarrow& [c, d] \\
      x &\mapsto& (d-c)x + \frac{c+d}{2}
    \end{array}
    \]
    Comme $\widetilde{f} \circ \varphi$ est continue, à support dans $I$, on peut maintenant affirmer que $\widetilde{f} \circ \varphi$ est limite uniforme d'une suite de polynômes $(\rho_n)$. Donc $\widetilde{f}$ est limite uniforme de la suite $(\rho_n \circ \varphi^{-1})$ où $\forall n \in \mathbb{N}$, $\rho_n \circ \varphi^{-1}$ est bien une fonction polynômiale car $\varphi$ (donc $\varphi^{-1}$ aussi) est affine. A fortiori, $f = \widetilde{f}_{|[a,b]}$ est aussi limite de fonctions polynômiales sur $[a,b]$.
  \end{proof}

  La fin de la preuve me semble mieux écrite dans \cite{[I-P]}.
  %</content>
\end{document}
