\documentclass[12pt, a4paper]{report}

% LuaLaTeX :

\RequirePackage{iftex}
\RequireLuaTeX

% Packages :

\usepackage[french]{babel}
%\usepackage[utf8]{inputenc}
%\usepackage[T1]{fontenc}
\usepackage[pdfencoding=auto, pdfauthor={Hugo Delaunay}, pdfsubject={Mathématiques}, pdfcreator={agreg.skyost.eu}]{hyperref}
\usepackage{amsmath}
\usepackage{amsthm}
%\usepackage{amssymb}
\usepackage{stmaryrd}
\usepackage{tikz}
\usepackage{tkz-euclide}
\usepackage{fourier-otf}
\usepackage{fontspec}
\usepackage{titlesec}
\usepackage{fancyhdr}
\usepackage{catchfilebetweentags}
\usepackage[french, capitalise, noabbrev]{cleveref}
\usepackage[fit, breakall]{truncate}
\usepackage[top=2.5cm, right=2cm, bottom=2.5cm, left=2cm]{geometry}
\usepackage{enumerate}
\usepackage{tocloft}
\usepackage{microtype}
%\usepackage{mdframed}
%\usepackage{thmtools}
\usepackage{xcolor}
\usepackage{tabularx}
\usepackage{aligned-overset}
\usepackage[subpreambles=true]{standalone}
\usepackage{environ}
\usepackage[normalem]{ulem}
\usepackage{marginnote}
\usepackage{etoolbox}
\usepackage{setspace}
\usepackage[bibstyle=reading, citestyle=draft]{biblatex}
\usepackage{xpatch}
\usepackage[many, breakable]{tcolorbox}
\usepackage[backgroundcolor=white, bordercolor=white, textsize=small]{todonotes}

% Bibliographie :

\newcommand{\overridebibliographypath}[1]{\providecommand{\bibliographypath}{#1}}
\overridebibliographypath{../bibliography.bib}
\addbibresource{\bibliographypath}
\defbibheading{bibliography}[\bibname]{%
	\newpage
	\section*{#1}%
}
\renewbibmacro*{entryhead:full}{\printfield{labeltitle}}
\DeclareFieldFormat{url}{\newline\footnotesize\url{#1}}
\AtEndDocument{\printbibliography}

% Police :

\setmathfont{Erewhon Math}

% Tikz :

\usetikzlibrary{calc}

% Longueurs :

\setlength{\parindent}{0pt}
\setlength{\headheight}{15pt}
\setlength{\fboxsep}{0pt}
\titlespacing*{\chapter}{0pt}{-20pt}{10pt}
\setlength{\marginparwidth}{1.5cm}
\setstretch{1.1}

% Métadonnées :

\author{agreg.skyost.eu}
\date{\today}

% Titres :

\setcounter{secnumdepth}{3}

\renewcommand{\thechapter}{\Roman{chapter}}
\renewcommand{\thesubsection}{\Roman{subsection}}
\renewcommand{\thesubsubsection}{\arabic{subsubsection}}
\renewcommand{\theparagraph}{\alph{paragraph}}

\titleformat{\chapter}{\huge\bfseries}{\thechapter}{20pt}{\huge\bfseries}
\titleformat*{\section}{\LARGE\bfseries}
\titleformat{\subsection}{\Large\bfseries}{\thesubsection \, - \,}{0pt}{\Large\bfseries}
\titleformat{\subsubsection}{\large\bfseries}{\thesubsubsection. \,}{0pt}{\large\bfseries}
\titleformat{\paragraph}{\bfseries}{\theparagraph. \,}{0pt}{\bfseries}

\setcounter{secnumdepth}{4}

% Table des matières :

\renewcommand{\cftsecleader}{\cftdotfill{\cftdotsep}}
\addtolength{\cftsecnumwidth}{10pt}

% Redéfinition des commandes :

\renewcommand*\thesection{\arabic{section}}
\renewcommand{\ker}{\mathrm{Ker}}

% Nouvelles commandes :

\newcommand{\website}{https://agreg.skyost.eu}

\newcommand{\tr}[1]{\mathstrut ^t #1}
\newcommand{\im}{\mathrm{Im}}
\newcommand{\rang}{\operatorname{rang}}
\newcommand{\trace}{\operatorname{trace}}
\newcommand{\id}{\operatorname{id}}
\newcommand{\stab}{\operatorname{Stab}}

\providecommand{\newpar}{\\[\medskipamount]}

\providecommand{\lesson}[3]{%
	\title{#3}%
	\hypersetup{pdftitle={#3}}%
	\setcounter{section}{\numexpr #2 - 1}%
	\section{#3}%
	\fancyhead[R]{\truncate{0.73\textwidth}{#2 : #3}}%
}

\providecommand{\development}[3]{%
	\title{#3}%
	\hypersetup{pdftitle={#3}}%
	\section*{#3}%
	\fancyhead[R]{\truncate{0.73\textwidth}{#3}}%
}

\providecommand{\summary}[1]{%
	\textit{#1}%
	\medskip%
}

\tikzset{notestyleraw/.append style={inner sep=0pt, rounded corners=0pt, align=center}}

%\newcommand{\booklink}[1]{\website/bibliographie\##1}
\newcommand{\citelink}[2]{\hyperlink{cite.\therefsection @#1}{#2}}
\newcommand{\previousreference}{}
\providecommand{\reference}[2][]{%
	\notblank{#1}{\renewcommand{\previousreference}{#1}}{}%
	\todo[noline]{%
		\protect\vspace{16pt}%
		\protect\par%
		\protect\notblank{#1}{\cite{[\previousreference]}\\}{}%
		\protect\citelink{\previousreference}{p. #2}%
	}%
}

\definecolor{devcolor}{HTML}{00695c}
\newcommand{\dev}[1]{%
	\reversemarginpar%
	\todo[noline]{
		\protect\vspace{16pt}%
		\protect\par%
		\bfseries\color{devcolor}\href{\website/developpements/#1}{DEV}
	}%
	\normalmarginpar%
}

% En-têtes :

\pagestyle{fancy}
\fancyhead[L]{\truncate{0.23\textwidth}{\thepage}}
\fancyfoot[C]{\scriptsize \href{\website}{\texttt{agreg.skyost.eu}}}

% Couleurs :

\definecolor{property}{HTML}{fffde7}
\definecolor{proposition}{HTML}{fff8e1}
\definecolor{lemma}{HTML}{fff3e0}
\definecolor{theorem}{HTML}{fce4f2}
\definecolor{corollary}{HTML}{ffebee}
\definecolor{definition}{HTML}{ede7f6}
\definecolor{notation}{HTML}{f3e5f5}
\definecolor{example}{HTML}{e0f7fa}
\definecolor{cexample}{HTML}{efebe9}
\definecolor{application}{HTML}{e0f2f1}
\definecolor{remark}{HTML}{e8f5e9}
\definecolor{proof}{HTML}{e1f5fe}

% Théorèmes :

\theoremstyle{definition}
\newtheorem{theorem}{Théorème}

\newtheorem{property}[theorem]{Propriété}
\newtheorem{proposition}[theorem]{Proposition}
\newtheorem{lemma}[theorem]{Lemme}
\newtheorem{corollary}[theorem]{Corollaire}

\newtheorem{definition}[theorem]{Définition}
\newtheorem{notation}[theorem]{Notation}

\newtheorem{example}[theorem]{Exemple}
\newtheorem{cexample}[theorem]{Contre-exemple}
\newtheorem{application}[theorem]{Application}

\theoremstyle{remark}
\newtheorem{remark}[theorem]{Remarque}

\counterwithin*{theorem}{section}

\newcommand{\applystyletotheorem}[1]{
	\tcolorboxenvironment{#1}{
		enhanced,
		breakable,
		colback=#1!98!white,
		boxrule=0pt,
		boxsep=0pt,
		left=8pt,
		right=8pt,
		top=8pt,
		bottom=8pt,
		sharp corners,
		after=\par,
	}
}

\applystyletotheorem{property}
\applystyletotheorem{proposition}
\applystyletotheorem{lemma}
\applystyletotheorem{theorem}
\applystyletotheorem{corollary}
\applystyletotheorem{definition}
\applystyletotheorem{notation}
\applystyletotheorem{example}
\applystyletotheorem{cexample}
\applystyletotheorem{application}
\applystyletotheorem{remark}
\applystyletotheorem{proof}

% Environnements :

\NewEnviron{whitetabularx}[1]{%
	\renewcommand{\arraystretch}{2.5}
	\colorbox{white}{%
		\begin{tabularx}{\textwidth}{#1}%
			\BODY%
		\end{tabularx}%
	}%
}

% Maths :

\DeclareFontEncoding{FMS}{}{}
\DeclareFontSubstitution{FMS}{futm}{m}{n}
\DeclareFontEncoding{FMX}{}{}
\DeclareFontSubstitution{FMX}{futm}{m}{n}
\DeclareSymbolFont{fouriersymbols}{FMS}{futm}{m}{n}
\DeclareSymbolFont{fourierlargesymbols}{FMX}{futm}{m}{n}
\DeclareMathDelimiter{\VERT}{\mathord}{fouriersymbols}{152}{fourierlargesymbols}{147}


% Bibliographie :

\addbibresource{\bibliographypath}%
\defbibheading{bibliography}[\bibname]{%
	\newpage
	\section*{#1}%
}
\renewbibmacro*{entryhead:full}{\printfield{labeltitle}}%
\DeclareFieldFormat{url}{\newline\footnotesize\url{#1}}%

\AtEndDocument{\printbibliography}

\begin{document}
  %<*content>
  \development{algebra}{decomposition-de-dunford}{Décomposition de Dunford}

  \summary{On démontre l'existence et l'unicité de la décomposition de Dunford pour tout endomorphisme d'un espace vectoriel de dimension finie.}

  \reference[GOU21]{203}

  Soit $E$ un espace vectoriel de dimension finie sur un corps commutatif $\mathbb{K}$.

  \begin{theorem}[Décomposition de Dunford]
    Soit $f \in E$ un endomorphisme tel que son polynôme minimal $\pi_f$ soit scindé sur $\mathbb{K}$. Alors il existe un unique couple d'endomorphismes $(d, n)$ tel que :
    \begin{itemize}
      \item $f = d + n$.
      \item $d$ est diagonalisable et $n$ est nilpotent.
      \item $d \circ n = n \circ d$.
    \end{itemize}
  \end{theorem}

  \begin{proof}
    On écrit $\pi_f = (-1)^n \prod_{i=1}^s (X - \lambda_i)^{\alpha_i}$ et pour tout $i$, on note $N_i = \ker((f-\lambda_i \id_E)^{\alpha_i})$ le $i$-ième sous-espace caractéristique de $f$.
    \newpar
    \uline{Construction :} Comme $E = N_1 \oplus \dots \oplus N_s$, il suffit de définir $d$ et $n$ sur chaque $N_i$. On les définit pour tout $i$ et pour tout $x \in N_i$ comme tels :
    \begin{itemize}
      \item $d(x) = \lambda_i x \implies d_{|N_i} = \lambda_i \id_{N_i}$
      \item $n(x) = f(x) - \lambda_i x = f(x) - d(x) \implies n = f - d$.
    \end{itemize}
    \medskip
    \uline{Vérification :}
    \begin{itemize}
      \item Les restrictions de $d$ et $n$ à $N_i$ sont bien des endomorphismes car les espaces $N_i$ sont stables par $f$ et par $d$ (cf. définition de $d$), donc aussi par $n = f - d$.
      \item $d$ est diagonalisable et pour tout $i$, $n_{|N_i}^{\alpha_i} = 0$ (car $\forall x \in N_i, \, (f-\lambda_i)^{\alpha_i}(x) = 0$ par définition de $N_i$). On pose donc $\alpha = \max_{i} \{ \alpha_i \}$ et on a $n_{|N_i}^\alpha = 0$ pour tout $i$, donc $n^\alpha = 0$ par somme directe. Ainsi, $n$ est nilpotent.
      \item Pour tout $i$, on a $d_{|N_i} = \lambda_i \id_E$, donc $n_{|N_i} \circ d_{|N_i} = d_{|N_i} \circ n_{|N_i}$ i.e. $d$ et $n$ commutent sur chaque $N_i$ donc sur $E$ tout entier.
    \end{itemize}
    \medskip
    \uline{Unicité :} Soit $(d', n')$ un autre couple d'endomorphismes de $E$ vérifiant les hypothèses. On remarque d'abord que $d'$ et $f$ commutent (car $d'$ commute avec $d'$ et $n'$, donc avec $f = d' + n'$ aussi). Pour tout $i$, $N_i$ est stable par $d'$ (car $\forall x \in N_i, \, (f-\lambda_i \id_E)^{\alpha_i}(d'(x)) = d' \circ (f-\lambda_i \id_E)^{\alpha_i}(x) = 0$). Comme $d_{|N_i} = \lambda_i \id_{N_i}$, on en déduit que $d \circ d' = d' \circ d$ sur $N_i$. Donc c'est également vrai sur $E$ tout entier. Ainsi, $d$ et $d'$ sont diagonalisables dans une même base, donc $d - d'$ est disagonalisable.
    \newpar
    D'autre part, comme $n = f-d$, $n' = f-d'$ et que $d$ et $d'$ commutent, $n$ et $n'$ commutent. Si on choisit $p$ et $q$ tels que $n^p = n'^q = 0$, alors :
    \[ (n-n')^{p+q} = \sum_{i=0}^{p+q} \binom{p+q}{i} n^{i} (-1)^{p+q-i} n'^{p+q-i} = 0 \]
    (dans chaque terme de la somme, soit $i \geq p$, soit $p+q-i \geq q$). Donc $n - n' = d' - d$ est nilpotent. Or nous avions montré que $d' - d$ est diagonalisable, donc $d'-d = 0$. Finalement, on a $d = d'$ et $n = n'$.
  \end{proof}

  \begin{remark}
    On peut démontrer que les endomorphismes $d$ et $n$ sont des polynômes en $f$.
  \end{remark}
  %</content>
\end{document}
