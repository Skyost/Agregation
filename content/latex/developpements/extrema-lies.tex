\documentclass[12pt, a4paper]{report}

% LuaLaTeX :

\RequirePackage{iftex}
\RequireLuaTeX

% Packages :

\usepackage[french]{babel}
%\usepackage[utf8]{inputenc}
%\usepackage[T1]{fontenc}
\usepackage[pdfencoding=auto, pdfauthor={Hugo Delaunay}, pdfsubject={Mathématiques}, pdfcreator={agreg.skyost.eu}]{hyperref}
\usepackage{amsmath}
\usepackage{amsthm}
%\usepackage{amssymb}
\usepackage{stmaryrd}
\usepackage{tikz}
\usepackage{tkz-euclide}
\usepackage{fourier-otf}
\usepackage{fontspec}
\usepackage{titlesec}
\usepackage{fancyhdr}
\usepackage{catchfilebetweentags}
\usepackage[french, capitalise, noabbrev]{cleveref}
\usepackage[fit, breakall]{truncate}
\usepackage[top=2.5cm, right=2cm, bottom=2.5cm, left=2cm]{geometry}
\usepackage{enumerate}
\usepackage{tocloft}
\usepackage{microtype}
%\usepackage{mdframed}
%\usepackage{thmtools}
\usepackage{xcolor}
\usepackage{tabularx}
\usepackage{aligned-overset}
\usepackage[subpreambles=true]{standalone}
\usepackage{environ}
\usepackage[normalem]{ulem}
\usepackage{marginnote}
\usepackage{etoolbox}
\usepackage{setspace}
\usepackage[bibstyle=reading, citestyle=draft]{biblatex}
\usepackage{xpatch}
\usepackage[many, breakable]{tcolorbox}
\usepackage[backgroundcolor=white, bordercolor=white, textsize=small]{todonotes}

% Bibliographie :

\newcommand{\overridebibliographypath}[1]{\providecommand{\bibliographypath}{#1}}
\overridebibliographypath{../bibliography.bib}
\addbibresource{\bibliographypath}
\defbibheading{bibliography}[\bibname]{%
	\newpage
	\section*{#1}%
}
\renewbibmacro*{entryhead:full}{\printfield{labeltitle}}
\DeclareFieldFormat{url}{\newline\footnotesize\url{#1}}
\AtEndDocument{\printbibliography}

% Police :

\setmathfont{Erewhon Math}

% Tikz :

\usetikzlibrary{calc}

% Longueurs :

\setlength{\parindent}{0pt}
\setlength{\headheight}{15pt}
\setlength{\fboxsep}{0pt}
\titlespacing*{\chapter}{0pt}{-20pt}{10pt}
\setlength{\marginparwidth}{1.5cm}
\setstretch{1.1}

% Métadonnées :

\author{agreg.skyost.eu}
\date{\today}

% Titres :

\setcounter{secnumdepth}{3}

\renewcommand{\thechapter}{\Roman{chapter}}
\renewcommand{\thesubsection}{\Roman{subsection}}
\renewcommand{\thesubsubsection}{\arabic{subsubsection}}
\renewcommand{\theparagraph}{\alph{paragraph}}

\titleformat{\chapter}{\huge\bfseries}{\thechapter}{20pt}{\huge\bfseries}
\titleformat*{\section}{\LARGE\bfseries}
\titleformat{\subsection}{\Large\bfseries}{\thesubsection \, - \,}{0pt}{\Large\bfseries}
\titleformat{\subsubsection}{\large\bfseries}{\thesubsubsection. \,}{0pt}{\large\bfseries}
\titleformat{\paragraph}{\bfseries}{\theparagraph. \,}{0pt}{\bfseries}

\setcounter{secnumdepth}{4}

% Table des matières :

\renewcommand{\cftsecleader}{\cftdotfill{\cftdotsep}}
\addtolength{\cftsecnumwidth}{10pt}

% Redéfinition des commandes :

\renewcommand*\thesection{\arabic{section}}
\renewcommand{\ker}{\mathrm{Ker}}

% Nouvelles commandes :

\newcommand{\website}{https://agreg.skyost.eu}

\newcommand{\tr}[1]{\mathstrut ^t #1}
\newcommand{\im}{\mathrm{Im}}
\newcommand{\rang}{\operatorname{rang}}
\newcommand{\trace}{\operatorname{trace}}
\newcommand{\id}{\operatorname{id}}
\newcommand{\stab}{\operatorname{Stab}}

\providecommand{\newpar}{\\[\medskipamount]}

\providecommand{\lesson}[3]{%
	\title{#3}%
	\hypersetup{pdftitle={#3}}%
	\setcounter{section}{\numexpr #2 - 1}%
	\section{#3}%
	\fancyhead[R]{\truncate{0.73\textwidth}{#2 : #3}}%
}

\providecommand{\development}[3]{%
	\title{#3}%
	\hypersetup{pdftitle={#3}}%
	\section*{#3}%
	\fancyhead[R]{\truncate{0.73\textwidth}{#3}}%
}

\providecommand{\summary}[1]{%
	\textit{#1}%
	\medskip%
}

\tikzset{notestyleraw/.append style={inner sep=0pt, rounded corners=0pt, align=center}}

%\newcommand{\booklink}[1]{\website/bibliographie\##1}
\newcommand{\citelink}[2]{\hyperlink{cite.\therefsection @#1}{#2}}
\newcommand{\previousreference}{}
\providecommand{\reference}[2][]{%
	\notblank{#1}{\renewcommand{\previousreference}{#1}}{}%
	\todo[noline]{%
		\protect\vspace{16pt}%
		\protect\par%
		\protect\notblank{#1}{\cite{[\previousreference]}\\}{}%
		\protect\citelink{\previousreference}{p. #2}%
	}%
}

\definecolor{devcolor}{HTML}{00695c}
\newcommand{\dev}[1]{%
	\reversemarginpar%
	\todo[noline]{
		\protect\vspace{16pt}%
		\protect\par%
		\bfseries\color{devcolor}\href{\website/developpements/#1}{DEV}
	}%
	\normalmarginpar%
}

% En-têtes :

\pagestyle{fancy}
\fancyhead[L]{\truncate{0.23\textwidth}{\thepage}}
\fancyfoot[C]{\scriptsize \href{\website}{\texttt{agreg.skyost.eu}}}

% Couleurs :

\definecolor{property}{HTML}{fffde7}
\definecolor{proposition}{HTML}{fff8e1}
\definecolor{lemma}{HTML}{fff3e0}
\definecolor{theorem}{HTML}{fce4f2}
\definecolor{corollary}{HTML}{ffebee}
\definecolor{definition}{HTML}{ede7f6}
\definecolor{notation}{HTML}{f3e5f5}
\definecolor{example}{HTML}{e0f7fa}
\definecolor{cexample}{HTML}{efebe9}
\definecolor{application}{HTML}{e0f2f1}
\definecolor{remark}{HTML}{e8f5e9}
\definecolor{proof}{HTML}{e1f5fe}

% Théorèmes :

\theoremstyle{definition}
\newtheorem{theorem}{Théorème}

\newtheorem{property}[theorem]{Propriété}
\newtheorem{proposition}[theorem]{Proposition}
\newtheorem{lemma}[theorem]{Lemme}
\newtheorem{corollary}[theorem]{Corollaire}

\newtheorem{definition}[theorem]{Définition}
\newtheorem{notation}[theorem]{Notation}

\newtheorem{example}[theorem]{Exemple}
\newtheorem{cexample}[theorem]{Contre-exemple}
\newtheorem{application}[theorem]{Application}

\theoremstyle{remark}
\newtheorem{remark}[theorem]{Remarque}

\counterwithin*{theorem}{section}

\newcommand{\applystyletotheorem}[1]{
	\tcolorboxenvironment{#1}{
		enhanced,
		breakable,
		colback=#1!98!white,
		boxrule=0pt,
		boxsep=0pt,
		left=8pt,
		right=8pt,
		top=8pt,
		bottom=8pt,
		sharp corners,
		after=\par,
	}
}

\applystyletotheorem{property}
\applystyletotheorem{proposition}
\applystyletotheorem{lemma}
\applystyletotheorem{theorem}
\applystyletotheorem{corollary}
\applystyletotheorem{definition}
\applystyletotheorem{notation}
\applystyletotheorem{example}
\applystyletotheorem{cexample}
\applystyletotheorem{application}
\applystyletotheorem{remark}
\applystyletotheorem{proof}

% Environnements :

\NewEnviron{whitetabularx}[1]{%
	\renewcommand{\arraystretch}{2.5}
	\colorbox{white}{%
		\begin{tabularx}{\textwidth}{#1}%
			\BODY%
		\end{tabularx}%
	}%
}

% Maths :

\DeclareFontEncoding{FMS}{}{}
\DeclareFontSubstitution{FMS}{futm}{m}{n}
\DeclareFontEncoding{FMX}{}{}
\DeclareFontSubstitution{FMX}{futm}{m}{n}
\DeclareSymbolFont{fouriersymbols}{FMS}{futm}{m}{n}
\DeclareSymbolFont{fourierlargesymbols}{FMX}{futm}{m}{n}
\DeclareMathDelimiter{\VERT}{\mathord}{fouriersymbols}{152}{fourierlargesymbols}{147}


% Bibliographie :

\addbibresource{\bibliographypath}%
\defbibheading{bibliography}[\bibname]{%
	\newpage
	\section*{#1}%
}
\renewbibmacro*{entryhead:full}{\printfield{labeltitle}}%
\DeclareFieldFormat{url}{\newline\footnotesize\url{#1}}%

\AtEndDocument{\printbibliography}

\begin{document}
	%<*content>
	\development{algebra, analysis}{extrema-lies}{Extrema liés}

	\summary{Dans ce développement, on montre l'existence et l'unicité des multiplicateurs de Lagrange liant les différentielles de plusieurs fonctions sous certaines hypothèses.}

	\reference[GOU20]{337}

	\begin{theorem}[Extrema liés]
		Soit $U$ un ouvert de $\mathbb{R}^n$ et soient $f, g_1, \dots, g_r : U \rightarrow \mathbb{R}$ des fonctions de classe $\mathcal{C}^1$. On note $\Gamma = \{ x \in U \mid g_1(x) = \dots = g_r(x) = 0 \}$. Si $f_{|\Gamma}$ admet un extremum relatif en $a \in \Gamma$ et si les formes linéaires $\mathrm{d}(g_1)_a, \dots, \mathrm{d}(g_r)_a$ sont linéairement indépendantes, alors il existe des uniques $\lambda_1, \dots, \lambda_r$ appelés \textbf{multiplicateurs de Lagrange} tels que
		\[ \mathrm{d}f_a = \lambda_1 \mathrm{d}(g_1)_a + \dots + \lambda_r \mathrm{d}(g_r)_a \]
	\end{theorem}

	\reference{347}

	\begin{proof}
		Soit $s = n-r$. Identifions $\mathbb{R}^n$ à $\mathbb{R}^s \times \mathbb{R}^r$ et écrivons les éléments $(x, y)$ de $\mathbb{R}^n$ sous la forme $(x, y) = (x_1, \dots, x_s, y_1, \dots, y_r)$. On notera également par la suite $a = (\alpha, \beta)$ avec $\alpha \in \mathbb{R}^s$ et $\beta \in \mathbb{R}^r$. On a déjà plusieurs informations :
		\begin{itemize}
			\item Déjà, $r \leq n$, car les formes linéaires $\mathrm{d}(g_i)_a$ forment une famille libre de $(\mathbb{R}^n)^*$, qui est de dimension $n$.
			\item De plus, si $r = n$, la démonstration est triviale car $(\mathrm{d}(g_i)_a)_{i \in \llbracket 1, n \rrbracket}$ est alors une base de $(\mathbb{R}^n)^*$.
		\end{itemize}
		Pour ces raisons, nous supposerons dans la suite $r \leq n-1$ (ie. $s \geq 1$).
		\newpar
		Comme $(\mathrm{d}(g_i)_a)_{i \in \llbracket 1, r \rrbracket}$ est une famille libre, la matrice
		\[ \begin{pmatrix}
			\left( \frac{\partial g_i}{\partial x_j}(a) \right)_{\substack{i \in \llbracket 1, r \rrbracket \\ j \in \llbracket 1, s \rrbracket}} & \left( \frac{\partial g_i}{\partial y_j}(a) \right)_{\substack{i \in \llbracket 1, r \rrbracket \\ j \in \llbracket 1, r \rrbracket}}
		\end{pmatrix} \]
		est de rang $r$. On peut donc extraire une sous-matrice de taille $r \times r$ inversible. Quitte à changer le nom des variables, on peut supposer que c'est la sous-matrice de droite, ie.
		\[ \det \left( \left( \frac{\partial g_i}{\partial y_j}(a) \right)_{i, j \in \llbracket 1, r \rrbracket} \right) \neq 0 \tag{$*$} \]
		On va appliquer le théorème des fonctions implicites à la fonction $g = (g_1, \dots, g_r)$. Pour cela, on vérifie les hypothèses :
		\begin{itemize}
			\item $g$ est de classe $\mathcal{C}^1$.
			\item $g(\alpha, \beta) = 0$ car $(\alpha, \beta) = a \in \Gamma$.
			\item La différentielle partielle $\mathrm{d}_y g_a$ est inversible par $(*)$.
		\end{itemize}
		Ainsi, il existe :
		\begin{itemize}
			\item $U'$ voisinage de $\alpha$ dans $\mathbb{R}^s$.
			\item $V'$ voisinage de $\beta$ dans $\mathbb{R}^r$.
			\item $\varphi : U' \rightarrow V'$ de classe $\mathcal{C}^1$ telle que $\varphi(\alpha) = \beta$ et $\forall (x, y) \in U' \times V'$, $(x, y) \in \Gamma \iff g(x, y) = 0 \iff y = \varphi(x)$.
		\end{itemize}
		En d'autres termes, sur un voisinage de $a$, les éléments de $\Gamma$ s'écrivent $(x, \varphi(x))$. On pose maintenant $u : x \mapsto (x, \varphi(x))$ et $h = f \circ u$. Par composition, $h$ est différentiable en $\alpha$ et
		\[ 0 \overset{\alpha \text{ extremum de } h}{=} \mathrm{d}h_\alpha = \mathrm{d}(f \circ u)_\alpha = \mathrm{d}f_{u(\alpha)} \circ \mathrm{d}u_\alpha = \mathrm{d}f_a \circ \mathrm{d}u_\alpha \]
		En termes de matrices, cela donne :
		\begin{align*}
			\begin{pmatrix} 0 \\ \vdots \\ 0 \end{pmatrix} &= \begin{pmatrix} \left( \frac{\partial f}{\partial x_j}(a) \right)_{j \in \llbracket 1, s \rrbracket} & \left( \frac{\partial f}{\partial y_j}(a) \right)_{j \in \llbracket 1, r \rrbracket} \end{pmatrix} \begin{pmatrix} I_s \\ \left( \frac{\partial \varphi_i}{\partial x_j}(\alpha) \right)_{\substack{i \in \llbracket 1, r \rrbracket \\ j \in \llbracket 1, s \rrbracket}} \end{pmatrix} \\
			&= \begin{pmatrix} \frac{\partial f}{\partial x_1}(a) + \sum_{k=1}^r \frac{\partial f}{\partial y_k}(a) \frac{\partial \varphi_k}{\partial x_1}(\alpha) \\ \vdots \\ \frac{\partial f}{\partial x_s}(a) + \sum_{k=1}^r \frac{\partial f}{\partial y_k}(a) \frac{\partial \varphi_k}{\partial x_s}(\alpha) \end{pmatrix}
		\end{align*}
		On aboutit à la relation suivante :
		\[ \forall i \in \llbracket 1, s \rrbracket, \, \frac{\partial f}{\partial x_i}(a) + \sum_{k=1}^r \frac{\partial f}{\partial y_k}(a) \frac{\partial \varphi_k}{\partial x_i}(\alpha) = 0 \tag{$**$} \]
		Comme $\forall j \in \llbracket 1, r \rrbracket$, $g_j(\alpha, \varphi(\alpha)) = g_j(a) = 0$, on peut aboutir de la même manière à la relation suivante :
		\[ \forall i \in \llbracket 1, s \rrbracket, \forall j \in \llbracket 1, r \rrbracket, \, \frac{\partial g_j}{\partial x_i}(a) + \sum_{k=1}^r \frac{\partial g_j}{\partial y_k}(a) \frac{\partial \varphi_k}{\partial x_i}(\alpha) = 0 \tag{$***$} \]
		On considère maintenant la matrice $M$ suivante :
		\[ M = \begin{pmatrix}
			\left( \frac{\partial f}{\partial x_j}(a) \right)_{j \in \llbracket 1, s \rrbracket} & \left( \frac{\partial f}{\partial y_j}(a) \right)_{j \in \llbracket 1, r \rrbracket} \\
			\left( \frac{\partial g_i}{\partial x_j}(a) \right)_{\substack{i \in \llbracket 1, r \rrbracket \\ j \in \llbracket 1, s \rrbracket}} & \left( \frac{\partial g_i}{\partial y_j}(a) \right)_{\substack{i \in \llbracket 1, r \rrbracket \\ j \in \llbracket 1, r \rrbracket}}
		\end{pmatrix} \]
		Par $(**)$ et $(***)$, les $s$ premiers vecteurs colonnes de cette matrice s'expriment linéairement en fonction de ses $r$ derniers. Donc $\rang (M) \leq r$. Mais, le rang des vecteurs lignes d'une matrice est égal au rang de ses vecteurs colonnes. Donc les $r+1$ vecteurs lignes de $M$ forment une famille liée. Mais par hypothèse, les $r$ dernières lignes sont libres. Donc la première ligne est combinaison linéaire des $r$ dernières, ce qui se réécrit :
		\[ \exists \lambda_1, \dots, \lambda_r \in \mathbb{R} \text{ tels que } \mathrm{d}f_a = \lambda_1 \mathrm{d}(g_1)_a + \dots + \lambda_r \mathrm{d}(g_r)_a \]
		L'unicité est claire car $(\mathrm{d}(g_i)_a)_{i \in \llbracket 1, r \rrbracket}$ est une famille libre.
	\end{proof}

	\begin{remark}
		Attention à la rigueur et à la propreté dans cette démonstration. On peut très vite se perdre si l'on va trop vite ou si l'on ne prend pas le temps de bien écrire chaque donnée.
	\end{remark}

	\reference[BMP]{20}

	\begin{remark}
		Il paraît que le jury n'aime pas beaucoup cette démonstration. Si vous la proposez, soyez sûr de pouvoir en donner une interprétation géométrique : grâce à la condition d'indépendance des $\mathrm{d}(g_i)_a$, $\Gamma$ est une sous-variété de $\mathbb{R}^n$ autour du point $a$. D'autre part,
		\[ \mathrm{d}f_a = \lambda_1 \mathrm{d}(g_1)_a + \dots + \lambda_r \mathrm{d}(g_r)_a \iff \bigcap_{i=1}^r \ker(\mathrm{d}(g_i)_a) \subset \ker(\mathrm{d}f_a) \tag{$*$} \]
		En particulier, $\mathrm{d}f_a$ est nulle sur $\bigcap_{i=1}^r \ker(\mathrm{d}(g_i)_a)$. Or, l'espace tangent en $a$ à la sous-variété $\{ x \text{ proche de } a \mid g_1(x) = \dots = g_r(x) = 0 \}$ est justement $\{ h \in \mathbb{R}^n \mid \mathrm{d}(g_1)_a(h) = \dots = \mathrm{d}(g_r)_a(h) = 0 \}$.
		\newpar
		Bref, la condition $(*)$ exprime que $\mathrm{d}f_a$ est nulle sur le plan tangent à $\Gamma$ en $a$. Ceci équivaut aussi à ce que $\nabla f_a$ soit orthogonal à l'espace tangent à $\Gamma$ en $a$. Ainsi, la seule manière de rendre $f$ plus petit serait de ``sortir de $\Gamma$''.
	\end{remark}
	%</content>
\end{document}
