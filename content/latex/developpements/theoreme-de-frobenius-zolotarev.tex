\documentclass[12pt, a4paper]{report}

% LuaLaTeX :

\RequirePackage{iftex}
\RequireLuaTeX

% Packages :

\usepackage[french]{babel}
%\usepackage[utf8]{inputenc}
%\usepackage[T1]{fontenc}
\usepackage[pdfencoding=auto, pdfauthor={Hugo Delaunay}, pdfsubject={Mathématiques}, pdfcreator={agreg.skyost.eu}]{hyperref}
\usepackage{amsmath}
\usepackage{amsthm}
%\usepackage{amssymb}
\usepackage{stmaryrd}
\usepackage{tikz}
\usepackage{tkz-euclide}
\usepackage{fourier-otf}
\usepackage{fontspec}
\usepackage{titlesec}
\usepackage{fancyhdr}
\usepackage{catchfilebetweentags}
\usepackage[french, capitalise, noabbrev]{cleveref}
\usepackage[fit, breakall]{truncate}
\usepackage[top=2.5cm, right=2cm, bottom=2.5cm, left=2cm]{geometry}
\usepackage{enumerate}
\usepackage{tocloft}
\usepackage{microtype}
%\usepackage{mdframed}
%\usepackage{thmtools}
\usepackage{xcolor}
\usepackage{tabularx}
\usepackage{aligned-overset}
\usepackage[subpreambles=true]{standalone}
\usepackage{environ}
\usepackage[normalem]{ulem}
\usepackage{marginnote}
\usepackage{etoolbox}
\usepackage{setspace}
\usepackage[bibstyle=reading, citestyle=draft]{biblatex}
\usepackage{xpatch}
\usepackage[many, breakable]{tcolorbox}
\usepackage[backgroundcolor=white, bordercolor=white, textsize=small]{todonotes}

% Bibliographie :

\newcommand{\overridebibliographypath}[1]{\providecommand{\bibliographypath}{#1}}
\overridebibliographypath{../bibliography.bib}
\addbibresource{\bibliographypath}
\defbibheading{bibliography}[\bibname]{%
	\newpage
	\section*{#1}%
}
\renewbibmacro*{entryhead:full}{\printfield{labeltitle}}
\DeclareFieldFormat{url}{\newline\footnotesize\url{#1}}
\AtEndDocument{\printbibliography}

% Police :

\setmathfont{Erewhon Math}

% Tikz :

\usetikzlibrary{calc}

% Longueurs :

\setlength{\parindent}{0pt}
\setlength{\headheight}{15pt}
\setlength{\fboxsep}{0pt}
\titlespacing*{\chapter}{0pt}{-20pt}{10pt}
\setlength{\marginparwidth}{1.5cm}
\setstretch{1.1}

% Métadonnées :

\author{agreg.skyost.eu}
\date{\today}

% Titres :

\setcounter{secnumdepth}{3}

\renewcommand{\thechapter}{\Roman{chapter}}
\renewcommand{\thesubsection}{\Roman{subsection}}
\renewcommand{\thesubsubsection}{\arabic{subsubsection}}
\renewcommand{\theparagraph}{\alph{paragraph}}

\titleformat{\chapter}{\huge\bfseries}{\thechapter}{20pt}{\huge\bfseries}
\titleformat*{\section}{\LARGE\bfseries}
\titleformat{\subsection}{\Large\bfseries}{\thesubsection \, - \,}{0pt}{\Large\bfseries}
\titleformat{\subsubsection}{\large\bfseries}{\thesubsubsection. \,}{0pt}{\large\bfseries}
\titleformat{\paragraph}{\bfseries}{\theparagraph. \,}{0pt}{\bfseries}

\setcounter{secnumdepth}{4}

% Table des matières :

\renewcommand{\cftsecleader}{\cftdotfill{\cftdotsep}}
\addtolength{\cftsecnumwidth}{10pt}

% Redéfinition des commandes :

\renewcommand*\thesection{\arabic{section}}
\renewcommand{\ker}{\mathrm{Ker}}

% Nouvelles commandes :

\newcommand{\website}{https://agreg.skyost.eu}

\newcommand{\tr}[1]{\mathstrut ^t #1}
\newcommand{\im}{\mathrm{Im}}
\newcommand{\rang}{\operatorname{rang}}
\newcommand{\trace}{\operatorname{trace}}
\newcommand{\id}{\operatorname{id}}
\newcommand{\stab}{\operatorname{Stab}}

\providecommand{\newpar}{\\[\medskipamount]}

\providecommand{\lesson}[3]{%
	\title{#3}%
	\hypersetup{pdftitle={#3}}%
	\setcounter{section}{\numexpr #2 - 1}%
	\section{#3}%
	\fancyhead[R]{\truncate{0.73\textwidth}{#2 : #3}}%
}

\providecommand{\development}[3]{%
	\title{#3}%
	\hypersetup{pdftitle={#3}}%
	\section*{#3}%
	\fancyhead[R]{\truncate{0.73\textwidth}{#3}}%
}

\providecommand{\summary}[1]{%
	\textit{#1}%
	\medskip%
}

\tikzset{notestyleraw/.append style={inner sep=0pt, rounded corners=0pt, align=center}}

%\newcommand{\booklink}[1]{\website/bibliographie\##1}
\newcommand{\citelink}[2]{\hyperlink{cite.\therefsection @#1}{#2}}
\newcommand{\previousreference}{}
\providecommand{\reference}[2][]{%
	\notblank{#1}{\renewcommand{\previousreference}{#1}}{}%
	\todo[noline]{%
		\protect\vspace{16pt}%
		\protect\par%
		\protect\notblank{#1}{\cite{[\previousreference]}\\}{}%
		\protect\citelink{\previousreference}{p. #2}%
	}%
}

\definecolor{devcolor}{HTML}{00695c}
\newcommand{\dev}[1]{%
	\reversemarginpar%
	\todo[noline]{
		\protect\vspace{16pt}%
		\protect\par%
		\bfseries\color{devcolor}\href{\website/developpements/#1}{DEV}
	}%
	\normalmarginpar%
}

% En-têtes :

\pagestyle{fancy}
\fancyhead[L]{\truncate{0.23\textwidth}{\thepage}}
\fancyfoot[C]{\scriptsize \href{\website}{\texttt{agreg.skyost.eu}}}

% Couleurs :

\definecolor{property}{HTML}{fffde7}
\definecolor{proposition}{HTML}{fff8e1}
\definecolor{lemma}{HTML}{fff3e0}
\definecolor{theorem}{HTML}{fce4f2}
\definecolor{corollary}{HTML}{ffebee}
\definecolor{definition}{HTML}{ede7f6}
\definecolor{notation}{HTML}{f3e5f5}
\definecolor{example}{HTML}{e0f7fa}
\definecolor{cexample}{HTML}{efebe9}
\definecolor{application}{HTML}{e0f2f1}
\definecolor{remark}{HTML}{e8f5e9}
\definecolor{proof}{HTML}{e1f5fe}

% Théorèmes :

\theoremstyle{definition}
\newtheorem{theorem}{Théorème}

\newtheorem{property}[theorem]{Propriété}
\newtheorem{proposition}[theorem]{Proposition}
\newtheorem{lemma}[theorem]{Lemme}
\newtheorem{corollary}[theorem]{Corollaire}

\newtheorem{definition}[theorem]{Définition}
\newtheorem{notation}[theorem]{Notation}

\newtheorem{example}[theorem]{Exemple}
\newtheorem{cexample}[theorem]{Contre-exemple}
\newtheorem{application}[theorem]{Application}

\theoremstyle{remark}
\newtheorem{remark}[theorem]{Remarque}

\counterwithin*{theorem}{section}

\newcommand{\applystyletotheorem}[1]{
	\tcolorboxenvironment{#1}{
		enhanced,
		breakable,
		colback=#1!98!white,
		boxrule=0pt,
		boxsep=0pt,
		left=8pt,
		right=8pt,
		top=8pt,
		bottom=8pt,
		sharp corners,
		after=\par,
	}
}

\applystyletotheorem{property}
\applystyletotheorem{proposition}
\applystyletotheorem{lemma}
\applystyletotheorem{theorem}
\applystyletotheorem{corollary}
\applystyletotheorem{definition}
\applystyletotheorem{notation}
\applystyletotheorem{example}
\applystyletotheorem{cexample}
\applystyletotheorem{application}
\applystyletotheorem{remark}
\applystyletotheorem{proof}

% Environnements :

\NewEnviron{whitetabularx}[1]{%
	\renewcommand{\arraystretch}{2.5}
	\colorbox{white}{%
		\begin{tabularx}{\textwidth}{#1}%
			\BODY%
		\end{tabularx}%
	}%
}

% Maths :

\DeclareFontEncoding{FMS}{}{}
\DeclareFontSubstitution{FMS}{futm}{m}{n}
\DeclareFontEncoding{FMX}{}{}
\DeclareFontSubstitution{FMX}{futm}{m}{n}
\DeclareSymbolFont{fouriersymbols}{FMS}{futm}{m}{n}
\DeclareSymbolFont{fourierlargesymbols}{FMX}{futm}{m}{n}
\DeclareMathDelimiter{\VERT}{\mathord}{fouriersymbols}{152}{fourierlargesymbols}{147}


% Bibliographie :

\addbibresource{\bibliographypath}%
\defbibheading{bibliography}[\bibname]{%
	\newpage
	\section*{#1}%
}
\renewbibmacro*{entryhead:full}{\printfield{labeltitle}}%
\DeclareFieldFormat{url}{\newline\footnotesize\url{#1}}%

\AtEndDocument{\printbibliography}

\begin{document}
  %<*content>
  \development{algebra}{theoreme-de-frobenius-zolotarev}{Théorème de Frobenius-Zolotarev}

  \summary{Nous démontrons le théorème de Frobenius-Zolotarev qui permet de calculer la signature d'un endomorphisme d'un espace vectoriel sur un corps fini possédant au moins $3$ éléments.}

  Soient $p \geq 3$ un nombre premier et $V$ un espace vectoriel sur $\mathbb{F}_p$ de dimension finie.

  \reference[I-P]{203}

  \begin{definition}
    Soit $H$ un hyperplan de $V$ et soit $G$ une droite supplémentaire de $H$ dans $V$.
    La dilatation $u$ de base $H$, de direction $G$, et de rapport $\lambda \in \mathbb{K}^*$ est l'unique endomorphisme de $V$ défini par
    \[ \forall g \in G, \, \forall h \in H, \, u(g+h) = h + \lambda g \]
  \end{definition}

  \reference[PER]{99}

  \begin{remark}
    \label{theoreme-de-frobenius-zolotarev-1}
    On suppose connu le fait que les transvections et les dilatations engendrent $\mathrm{GL}(V)$.
  \end{remark}

  \reference{96}

  \begin{lemma}
    \label{theoreme-de-frobenius-zolotarev-2}
    Soient $u \in \mathrm{GL}(V)$ et $H$ un hyperplan de $V$ tel que $u_{|H} = \id_H$. Si $\det(u) \neq 1$, alors $u$ est une dilatation.
  \end{lemma}

  \begin{proof}
    On note $n = \dim(V)$. Comme $u_{|H} = \id_H$ et $\dim(H) = n-1$, on en déduit que $1$ est valeur propre de multiplicité $n-1$ de $u$ et que $H$ est le sous-espace propre associé :
    \[ H = E_1(u) = \ker(u - \id_V) \]
    On pose $\lambda = \det(u) \notin \{ 0, 1 \}$. $\lambda$ est valeur propre de $u$ (on peut le voir par exemple en calculant le polynôme caractéristique de $u$) de multiplicité $1$. Donc $u$ est diagonalisable, et dans une base $\mathcal{B}$ adaptée à la diagonalisation, on a :
    \[ \operatorname{Mat}(u, \mathcal{B}) =
    \begin{pmatrix}
      1 & 0 & \dots & 0 \\
      0 & \ddots & \ddots & \vdots \\
      \vdots & \ddots & 1 & 0 \\
      0 & \dots & 0 & \lambda
    \end{pmatrix}
    \]
    d'où le résultat.
  \end{proof}

  \reference[I-P]{203}

  \begin{lemma}
    \label{theoreme-de-frobenius-zolotarev-3}
    Les dilatations engendrent $\mathrm{GL}(V)$.
  \end{lemma}

  \begin{proof}
    Pour obtenir le résultat, il suffit de montrer que toute transvection est la composée de deux dilatations (cf. \cref{theoreme-de-frobenius-zolotarev-1}). Soit $u$ une transvection d'hyperplan $H$. Comme $\mathbb{F}_p$ contient au moins $3$ éléments, il existe alors $v$ une dilatation d'hyperplan $H$ et de rapport $\lambda \neq 1$.
    \newpar
    Ainsi, l'application $w = u \circ v$ est dans $\mathrm{GL}(V)$ et fixe $H$. Comme $\det(w) = \det(v) = \lambda \neq 1$, le \cref{theoreme-de-frobenius-zolotarev-2} permet de conclure que $w$ est une dilatation. Ainsi, $u = w \circ v^{-1}$ est le produit de deux dilatations $v^{-1}$ est une dilatation (toujours d'après le \cref{theoreme-de-frobenius-zolotarev-2}).
  \end{proof}

  \begin{notation}
    Soit $a \in \mathbb{F}_p$. On note $\left( \frac{a}{p} \right)$ le symbole de Legendre de $a$ modulo $p$.
  \end{notation}

  \begin{theorem}[Frobenius-Zolotarev]
    \[ \forall u \in \mathrm{GL}(V), \, \epsilon(u) = \left( \frac{\det(u)}{p} \right) \]
    où $u$ est vu comme une permutation des éléments de $V$.
  \end{theorem}

  \begin{proof}
    Le groupe multiplicatif d'un corps fini est cyclique, donc il existe $a \in \mathbb{F}_p^*$ tel que
    \[ \mathbb{F}_p^* = \langle a \rangle \]
    En conséquence, si $u$ est la dilatation de $V$ de base $H$, de direction $G$, et de rapport $\lambda \in \mathbb{F}_p^*$, alors il existe $k \in \mathbb{N}^*$ tel que $\lambda = a^k$. On en déduit que si $v$ est la dilatation de $V$ de base $H$, de direction $G$, et de rapport $a$, alors $\forall x \in V$ écrit $x = g + h$ avec $g \in G$ et $h \in H$ :
    \[ v^k(x) = v^k(g+h) = h + a^k g = h + \lambda g = u(g+h) = u(x) \]
    d'où $v^k = u$. Ainsi, toute dilatation est une puissance d'une dilatation de rapport $a$.
    \newpar
    Comme $\det$, $\left( \frac{.}{p} \right)$ et $\epsilon$ sont tous trois des morphismes de groupes, et comme les dilatations engendrent $\mathrm{GL}(V)$ (cf. \cref{theoreme-de-frobenius-zolotarev-3}), il suffit de montrer le résultat pour les dilatations de rapport $a$.
    \newpar
    Soit $u$ une dilatation de base $H$, de direction $G$, et de rapport $a$. Supposons par l'absurde que $\left( \frac{\det(u)}{p} \right) = 1$. Comme $\det(u) = a$, on a $\left( \frac{a}{p} \right) = 1$. Mais, $\mathbb{F}_p^* = \langle a \rangle$, donc $\forall x \in \mathbb{F}_p^*$, $\left( \frac{x}{p} \right) = 1$ ie. tout élément de $\mathbb{F}_p^*$ est un carré. Or, il y a $\frac{p-1}{2}$ carrés dans $\mathbb{F}_p^*$ (et $|\mathbb{F}_p^*| = p-1$, bien-sûr) : contradiction.
    \newpar
    Il ne reste qu'à montrer que $\epsilon(u) = -1$. Pour cela, on va étudier les orbites des éléments $V$ sous l'action de $u$.
    \newpar
    Soit $h \in H$. On a $u(h) = h$, donc son orbite est réduite à $\{ h \}$ qui est de cardinal $1$. Elle compte donc comme un $+$ dans le signe de $\epsilon(u)$.
    \newpar
    Soit maintenant $x \in V$ écrit $x = g + h$ avec $g \in G \setminus \{ 0 \}$ et $h \in H$ de sorte que $u^k (x) = h + a^k g$ pour tout $k \in \mathbb{N}$.
    \begin{itemize}
      \item $\mathbb{F}_p^*$ est cyclique d'ordre $p-1$, donc $a^{p-1} = 1$. Ainsi, $f^{p-1} (x) = x$.
      \item Supposons par l'absurde que $\exists 1 \leq i < j \leq p-1$ tel que $u^i(x) = u^j(x)$. On a,
      \begin{align*}
        h + a^j g = h + a^i g &\iff a^{j-i} (a^i - 1) \underbrace{g}_{\neq 0} = 0 \\
        &\implies a^{j-i} = 0 \text{ ou } a^i = 1
      \end{align*}
      ce qui est absurde dans les deux cas.
    \end{itemize}
    L'orbite de $x$ sous l'action de $u$ est donc $\{ x, \dots, u^{p-2}(x) \}$ qui est de cardinal $p-1$ (pair) et compte donc comme un $-$ dans le signe de $\epsilon(u)$.
    \newpar
    Il ne reste qu'à compter le nombre d'orbites de cardinal $p-1$. Les éléments contenus dans ces orbites forment exactement l'ensemble
    \[ \bigcup_{h \in H} \{ g + h \mid g \in G, \, g \neq 0 \} \]
    et il y en a donc
    \[ |H| \times (|G|-1) = p^{n-1}(p-1) \]
    (car $H$ est un hyperplan et $G$ est une droite). Comme ces orbites sont de cardinal $p-1$, il y a donc exactement $p^{n-1}$ orbites. Or, $p^{n-1}$ est impair, donc $\epsilon(u)$ est de signe négatif. Ainsi, $\epsilon(u) = -1$.
  \end{proof}
  %</content>
\end{document}
