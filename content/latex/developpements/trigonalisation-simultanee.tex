\documentclass[12pt, a4paper]{report}

% LuaLaTeX :

\RequirePackage{iftex}
\RequireLuaTeX

% Packages :

\usepackage[french]{babel}
%\usepackage[utf8]{inputenc}
%\usepackage[T1]{fontenc}
\usepackage[pdfencoding=auto, pdfauthor={Hugo Delaunay}, pdfsubject={Mathématiques}, pdfcreator={agreg.skyost.eu}]{hyperref}
\usepackage{amsmath}
\usepackage{amsthm}
%\usepackage{amssymb}
\usepackage{stmaryrd}
\usepackage{tikz}
\usepackage{tkz-euclide}
\usepackage{fourier-otf}
\usepackage{fontspec}
\usepackage{titlesec}
\usepackage{fancyhdr}
\usepackage{catchfilebetweentags}
\usepackage[french, capitalise, noabbrev]{cleveref}
\usepackage[fit, breakall]{truncate}
\usepackage[top=2.5cm, right=2cm, bottom=2.5cm, left=2cm]{geometry}
\usepackage{enumerate}
\usepackage{tocloft}
\usepackage{microtype}
%\usepackage{mdframed}
%\usepackage{thmtools}
\usepackage{xcolor}
\usepackage{tabularx}
\usepackage{aligned-overset}
\usepackage[subpreambles=true]{standalone}
\usepackage{environ}
\usepackage[normalem]{ulem}
\usepackage{marginnote}
\usepackage{etoolbox}
\usepackage{setspace}
\usepackage[bibstyle=reading, citestyle=draft]{biblatex}
\usepackage{xpatch}
\usepackage[many, breakable]{tcolorbox}
\usepackage[backgroundcolor=white, bordercolor=white, textsize=small]{todonotes}

% Bibliographie :

\newcommand{\overridebibliographypath}[1]{\providecommand{\bibliographypath}{#1}}
\overridebibliographypath{../bibliography.bib}
\addbibresource{\bibliographypath}
\defbibheading{bibliography}[\bibname]{%
	\newpage
	\section*{#1}%
}
\renewbibmacro*{entryhead:full}{\printfield{labeltitle}}
\DeclareFieldFormat{url}{\newline\footnotesize\url{#1}}
\AtEndDocument{\printbibliography}

% Police :

\setmathfont{Erewhon Math}

% Tikz :

\usetikzlibrary{calc}

% Longueurs :

\setlength{\parindent}{0pt}
\setlength{\headheight}{15pt}
\setlength{\fboxsep}{0pt}
\titlespacing*{\chapter}{0pt}{-20pt}{10pt}
\setlength{\marginparwidth}{1.5cm}
\setstretch{1.1}

% Métadonnées :

\author{agreg.skyost.eu}
\date{\today}

% Titres :

\setcounter{secnumdepth}{3}

\renewcommand{\thechapter}{\Roman{chapter}}
\renewcommand{\thesubsection}{\Roman{subsection}}
\renewcommand{\thesubsubsection}{\arabic{subsubsection}}
\renewcommand{\theparagraph}{\alph{paragraph}}

\titleformat{\chapter}{\huge\bfseries}{\thechapter}{20pt}{\huge\bfseries}
\titleformat*{\section}{\LARGE\bfseries}
\titleformat{\subsection}{\Large\bfseries}{\thesubsection \, - \,}{0pt}{\Large\bfseries}
\titleformat{\subsubsection}{\large\bfseries}{\thesubsubsection. \,}{0pt}{\large\bfseries}
\titleformat{\paragraph}{\bfseries}{\theparagraph. \,}{0pt}{\bfseries}

\setcounter{secnumdepth}{4}

% Table des matières :

\renewcommand{\cftsecleader}{\cftdotfill{\cftdotsep}}
\addtolength{\cftsecnumwidth}{10pt}

% Redéfinition des commandes :

\renewcommand*\thesection{\arabic{section}}
\renewcommand{\ker}{\mathrm{Ker}}

% Nouvelles commandes :

\newcommand{\website}{https://agreg.skyost.eu}

\newcommand{\tr}[1]{\mathstrut ^t #1}
\newcommand{\im}{\mathrm{Im}}
\newcommand{\rang}{\operatorname{rang}}
\newcommand{\trace}{\operatorname{trace}}
\newcommand{\id}{\operatorname{id}}
\newcommand{\stab}{\operatorname{Stab}}

\providecommand{\newpar}{\\[\medskipamount]}

\providecommand{\lesson}[3]{%
	\title{#3}%
	\hypersetup{pdftitle={#3}}%
	\setcounter{section}{\numexpr #2 - 1}%
	\section{#3}%
	\fancyhead[R]{\truncate{0.73\textwidth}{#2 : #3}}%
}

\providecommand{\development}[3]{%
	\title{#3}%
	\hypersetup{pdftitle={#3}}%
	\section*{#3}%
	\fancyhead[R]{\truncate{0.73\textwidth}{#3}}%
}

\providecommand{\summary}[1]{%
	\textit{#1}%
	\medskip%
}

\tikzset{notestyleraw/.append style={inner sep=0pt, rounded corners=0pt, align=center}}

%\newcommand{\booklink}[1]{\website/bibliographie\##1}
\newcommand{\citelink}[2]{\hyperlink{cite.\therefsection @#1}{#2}}
\newcommand{\previousreference}{}
\providecommand{\reference}[2][]{%
	\notblank{#1}{\renewcommand{\previousreference}{#1}}{}%
	\todo[noline]{%
		\protect\vspace{16pt}%
		\protect\par%
		\protect\notblank{#1}{\cite{[\previousreference]}\\}{}%
		\protect\citelink{\previousreference}{p. #2}%
	}%
}

\definecolor{devcolor}{HTML}{00695c}
\newcommand{\dev}[1]{%
	\reversemarginpar%
	\todo[noline]{
		\protect\vspace{16pt}%
		\protect\par%
		\bfseries\color{devcolor}\href{\website/developpements/#1}{DEV}
	}%
	\normalmarginpar%
}

% En-têtes :

\pagestyle{fancy}
\fancyhead[L]{\truncate{0.23\textwidth}{\thepage}}
\fancyfoot[C]{\scriptsize \href{\website}{\texttt{agreg.skyost.eu}}}

% Couleurs :

\definecolor{property}{HTML}{fffde7}
\definecolor{proposition}{HTML}{fff8e1}
\definecolor{lemma}{HTML}{fff3e0}
\definecolor{theorem}{HTML}{fce4f2}
\definecolor{corollary}{HTML}{ffebee}
\definecolor{definition}{HTML}{ede7f6}
\definecolor{notation}{HTML}{f3e5f5}
\definecolor{example}{HTML}{e0f7fa}
\definecolor{cexample}{HTML}{efebe9}
\definecolor{application}{HTML}{e0f2f1}
\definecolor{remark}{HTML}{e8f5e9}
\definecolor{proof}{HTML}{e1f5fe}

% Théorèmes :

\theoremstyle{definition}
\newtheorem{theorem}{Théorème}

\newtheorem{property}[theorem]{Propriété}
\newtheorem{proposition}[theorem]{Proposition}
\newtheorem{lemma}[theorem]{Lemme}
\newtheorem{corollary}[theorem]{Corollaire}

\newtheorem{definition}[theorem]{Définition}
\newtheorem{notation}[theorem]{Notation}

\newtheorem{example}[theorem]{Exemple}
\newtheorem{cexample}[theorem]{Contre-exemple}
\newtheorem{application}[theorem]{Application}

\theoremstyle{remark}
\newtheorem{remark}[theorem]{Remarque}

\counterwithin*{theorem}{section}

\newcommand{\applystyletotheorem}[1]{
	\tcolorboxenvironment{#1}{
		enhanced,
		breakable,
		colback=#1!98!white,
		boxrule=0pt,
		boxsep=0pt,
		left=8pt,
		right=8pt,
		top=8pt,
		bottom=8pt,
		sharp corners,
		after=\par,
	}
}

\applystyletotheorem{property}
\applystyletotheorem{proposition}
\applystyletotheorem{lemma}
\applystyletotheorem{theorem}
\applystyletotheorem{corollary}
\applystyletotheorem{definition}
\applystyletotheorem{notation}
\applystyletotheorem{example}
\applystyletotheorem{cexample}
\applystyletotheorem{application}
\applystyletotheorem{remark}
\applystyletotheorem{proof}

% Environnements :

\NewEnviron{whitetabularx}[1]{%
	\renewcommand{\arraystretch}{2.5}
	\colorbox{white}{%
		\begin{tabularx}{\textwidth}{#1}%
			\BODY%
		\end{tabularx}%
	}%
}

% Maths :

\DeclareFontEncoding{FMS}{}{}
\DeclareFontSubstitution{FMS}{futm}{m}{n}
\DeclareFontEncoding{FMX}{}{}
\DeclareFontSubstitution{FMX}{futm}{m}{n}
\DeclareSymbolFont{fouriersymbols}{FMS}{futm}{m}{n}
\DeclareSymbolFont{fourierlargesymbols}{FMX}{futm}{m}{n}
\DeclareMathDelimiter{\VERT}{\mathord}{fouriersymbols}{152}{fourierlargesymbols}{147}


% Bibliographie :

\addbibresource{\bibliographypath}%
\defbibheading{bibliography}[\bibname]{%
	\newpage
	\section*{#1}%
}
\renewbibmacro*{entryhead:full}{\printfield{labeltitle}}%
\DeclareFieldFormat{url}{\newline\footnotesize\url{#1}}%

\AtEndDocument{\printbibliography}

\begin{document}
  %<*content>
  \development{algebra}{trigonalisation-simultanee}{Trigonalisation simultanée}

  \summary{Nous montrons le théorème de trigonalisation simultanée grâce à l'utilisation des applications transposées (et donc, de la dualité).}

  \reference[GOU21]{176}

  Soit $E$ un espace vectoriel de dimension $n$ sur un corps $\mathbb{K}$.
  
  \begin{lemma}
    \label{trigonalisation-simultanee-1}
    Soit $g \in \mathcal{L}(E)$ un endomorphisme. Soit $F$ un sous-espace vectoriel de $E$ stable par $g$. Alors,
    \[ \chi_{g_{|F}} \mid \chi_{g} \]
  \end{lemma}
  
  \begin{proof}
    On note $m$ la dimension de $F$. Considérons $G$, un supplémentaire de $F$ dans $E$. Soient $\mathcal{B}_F$ et $\mathcal{B}_G$ des bases respectives de $F$ et de $G$. Alors, la matrice de $g$ dans la base de $E$ constituée de l'union disjointe de $\mathcal{B}_F$ et $\mathcal{B}_G$ est de la forme
    \[
    M =
    \begin{pmatrix}
      A & * \\
      0 & *
    \end{pmatrix}
    \]
    avec $A \in \mathcal{M}_m(\mathbb{K})$, qui est la matrice de l'endomorphisme induit $g_{|F}$. On constate clairement que $\chi_{g_{|F}} = \chi_A \mid \chi_M = \chi_g$.
  \end{proof}

  \begin{lemma}
    \label{trigonalisation-simultanee-2}
    Soit $g \in \mathcal{L}(E)$ un endomorphisme trigonalisable. Soit $F$ un sous-espace vectoriel de $E$ stable par $g$. Alors, $g_{|F}$ est trigonalisable.
  \end{lemma}

  \begin{proof}
    $g$ est trigonalisable si et seulement si son polynôme caractéristique est scindé sur $\mathbb{K}$. Dans ce cas, le polynôme caractéristique de sa restriction à $F$ l'est aussi au vu du \cref{trigonalisation-simultanee-1}.
  \end{proof}

  \begin{lemma}
    \label{trigonalisation-simultanee-3}
    Soient $f, g \in \mathcal{L}(E)$. On suppose que $f$ et $g$ sont trigonalisables et commutent. Alors, $f$ et $g$ ont un vecteur propre commun.
  \end{lemma}

  \begin{proof}
    $f$ est trigonalisable, donc $f$ admet une valeur propre $\lambda \in \mathbb{K}$ (cf. première colonne de la matrice de $f$ dans une base de trigonalisation). Le sous-espace propre $E_\lambda = \ker(f - \lambda \operatorname{id}_E)$ est alors stable par $g$ :
    \[ \forall x \in E_\lambda, \, (f - \lambda \operatorname{id}_E)(g(x)) = g((f - \lambda \operatorname{id}_E)(x)) \]
    car $f$, $g$ et $\lambda \operatorname{id}_E$ commutent. Ainsi,
    \[ \forall x \in E_\lambda, \, (f - \lambda \operatorname{id}_E)(g(x)) = 0 \]
    Par le \cref{trigonalisation-simultanee-2}, la restriction de $g$ à $E_\lambda$ est trigonalisable. Donc, $g_{|E_\lambda}$ admet un vecteur propre $x \in E_\lambda$ qui est, par construction, un vecteur propre commun à $f$ et $g$.
  \end{proof}

  \begin{theorem}[Trigonalisation simultanée]
    Soient $f, g \in \mathcal{L}(E)$. On suppose que $f$ et $g$ sont trigonalisables et commutent. Alors, il existe une base de trigonalisation commune de $f$ et $g$.
  \end{theorem}

  \begin{proof}
    On va procéder par récurrence sur $n$.
    \begin{itemize}
      \item \uline{Si $n = 1$ :} c'est évident.
      \item \uline{Supposons le résultat vrai au rang $n - 1$.} Pour tout $\varphi \in E^*$,
      \begin{align*}
        (\tr{f} \circ \tr{g})(\varphi) &= \tr{f} (\varphi \circ g) \\
        &= \varphi \circ g \circ f \\
        &= \varphi \circ f \circ g \\
        &= (\tr{g} \circ \tr{f})(\varphi)
      \end{align*}
      ie. $\tr{f} \tr{g} = \tr{f} \tr{g}$. De plus, $\tr{f}$ et $\tr{g}$ sont trigonalisables (car possèdent les mêmes polynômes caractéristiques que $f$ et $g$). Par le \cref{trigonalisation-simultanee-3} appliqué à $\tr{f}$ et $\tr{g}$, il existe un vecteur propre $\psi \in E^*$ commun à ces deux endomorphismes. Le sous-espace vectoriel $\operatorname{Vect}(\psi)$ est ainsi stable par $\tr{f}$ et $\tr{g}$. Notons
      \[ H = \operatorname{Vect}(\psi)^{\circ} = \{ x \in E \mid \psi (x) = 0 \} = \ker(\psi) \]
      c'est un hyperplan de $E$ (donc de dimension $n-1$), qui est de plus stable par $f$ et $g$. En effet, en notant $\lambda \in \mathbb{K}$ la valeur propre de $f$ associée à $\psi$, on a :
      \[ \forall x \in H, \, \psi(f(x)) = \tr{f}(\psi)(x) = \lambda \psi(x) = 0 \]
      et un même calcul montre la stabilité par $g$. D'après l'hypothèse de récurrence appliquée aux endomorphismes induits $f_{|H}$ et $g_{|H}$, on obtient une base $\mathcal{B}_H$ de $H$ de cotrigonalisation pour $f_{|H}$ et $g_{|H}$. On la complète en une base quelconque $\mathcal{B}$ de $E$, dans laquelle on obtient
      \[
        \operatorname{Mat}(f, \mathcal{B}) =
        \begin{pmatrix}
          & & & * \\
          & \operatorname{Mat}(f_{|H}, \mathcal{B}_H) & & \vdots \\
          & & & * \\
          0 & \dots & 0 & *
        \end{pmatrix}
        \text{ et }
        \operatorname{Mat}(g, \mathcal{B}) =
        \begin{pmatrix}
          & & & * \\
          & \operatorname{Mat}(g_{|H}, \mathcal{B}_H) & & \vdots \\
          & & & * \\
          0 & \dots & 0 & *
        \end{pmatrix}
      \]
      où $\operatorname{Mat}(f_{|H}, \mathcal{B}_H)$ et $\operatorname{Mat}(g_{|H}, \mathcal{B}_H)$ sont triangulaires supérieures d'ordre $n-1$.
    \end{itemize}
  \end{proof}
  %</content>
\end{document}
