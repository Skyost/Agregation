\documentclass[12pt, a4paper]{report}

% LuaLaTeX :

\RequirePackage{iftex}
\RequireLuaTeX

% Packages :

\usepackage[french]{babel}
%\usepackage[utf8]{inputenc}
%\usepackage[T1]{fontenc}
\usepackage[pdfencoding=auto, pdfauthor={Hugo Delaunay}, pdfsubject={Mathématiques}, pdfcreator={agreg.skyost.eu}]{hyperref}
\usepackage{amsmath}
\usepackage{amsthm}
%\usepackage{amssymb}
\usepackage{stmaryrd}
\usepackage{tikz}
\usepackage{tkz-euclide}
\usepackage{fourier-otf}
\usepackage{fontspec}
\usepackage{titlesec}
\usepackage{fancyhdr}
\usepackage{catchfilebetweentags}
\usepackage[french, capitalise, noabbrev]{cleveref}
\usepackage[fit, breakall]{truncate}
\usepackage[top=2.5cm, right=2cm, bottom=2.5cm, left=2cm]{geometry}
\usepackage{enumerate}
\usepackage{tocloft}
\usepackage{microtype}
%\usepackage{mdframed}
%\usepackage{thmtools}
\usepackage{xcolor}
\usepackage{tabularx}
\usepackage{aligned-overset}
\usepackage[subpreambles=true]{standalone}
\usepackage{environ}
\usepackage[normalem]{ulem}
\usepackage{marginnote}
\usepackage{etoolbox}
\usepackage{setspace}
\usepackage[bibstyle=reading, citestyle=draft]{biblatex}
\usepackage{xpatch}
\usepackage[many, breakable]{tcolorbox}
\usepackage[backgroundcolor=white, bordercolor=white, textsize=small]{todonotes}

% Bibliographie :

\newcommand{\overridebibliographypath}[1]{\providecommand{\bibliographypath}{#1}}
\overridebibliographypath{../bibliography.bib}
\addbibresource{\bibliographypath}
\defbibheading{bibliography}[\bibname]{%
	\newpage
	\section*{#1}%
}
\renewbibmacro*{entryhead:full}{\printfield{labeltitle}}
\DeclareFieldFormat{url}{\newline\footnotesize\url{#1}}
\AtEndDocument{\printbibliography}

% Police :

\setmathfont{Erewhon Math}

% Tikz :

\usetikzlibrary{calc}

% Longueurs :

\setlength{\parindent}{0pt}
\setlength{\headheight}{15pt}
\setlength{\fboxsep}{0pt}
\titlespacing*{\chapter}{0pt}{-20pt}{10pt}
\setlength{\marginparwidth}{1.5cm}
\setstretch{1.1}

% Métadonnées :

\author{agreg.skyost.eu}
\date{\today}

% Titres :

\setcounter{secnumdepth}{3}

\renewcommand{\thechapter}{\Roman{chapter}}
\renewcommand{\thesubsection}{\Roman{subsection}}
\renewcommand{\thesubsubsection}{\arabic{subsubsection}}
\renewcommand{\theparagraph}{\alph{paragraph}}

\titleformat{\chapter}{\huge\bfseries}{\thechapter}{20pt}{\huge\bfseries}
\titleformat*{\section}{\LARGE\bfseries}
\titleformat{\subsection}{\Large\bfseries}{\thesubsection \, - \,}{0pt}{\Large\bfseries}
\titleformat{\subsubsection}{\large\bfseries}{\thesubsubsection. \,}{0pt}{\large\bfseries}
\titleformat{\paragraph}{\bfseries}{\theparagraph. \,}{0pt}{\bfseries}

\setcounter{secnumdepth}{4}

% Table des matières :

\renewcommand{\cftsecleader}{\cftdotfill{\cftdotsep}}
\addtolength{\cftsecnumwidth}{10pt}

% Redéfinition des commandes :

\renewcommand*\thesection{\arabic{section}}
\renewcommand{\ker}{\mathrm{Ker}}

% Nouvelles commandes :

\newcommand{\website}{https://agreg.skyost.eu}

\newcommand{\tr}[1]{\mathstrut ^t #1}
\newcommand{\im}{\mathrm{Im}}
\newcommand{\rang}{\operatorname{rang}}
\newcommand{\trace}{\operatorname{trace}}
\newcommand{\id}{\operatorname{id}}
\newcommand{\stab}{\operatorname{Stab}}

\providecommand{\newpar}{\\[\medskipamount]}

\providecommand{\lesson}[3]{%
	\title{#3}%
	\hypersetup{pdftitle={#3}}%
	\setcounter{section}{\numexpr #2 - 1}%
	\section{#3}%
	\fancyhead[R]{\truncate{0.73\textwidth}{#2 : #3}}%
}

\providecommand{\development}[3]{%
	\title{#3}%
	\hypersetup{pdftitle={#3}}%
	\section*{#3}%
	\fancyhead[R]{\truncate{0.73\textwidth}{#3}}%
}

\providecommand{\summary}[1]{%
	\textit{#1}%
	\medskip%
}

\tikzset{notestyleraw/.append style={inner sep=0pt, rounded corners=0pt, align=center}}

%\newcommand{\booklink}[1]{\website/bibliographie\##1}
\newcommand{\citelink}[2]{\hyperlink{cite.\therefsection @#1}{#2}}
\newcommand{\previousreference}{}
\providecommand{\reference}[2][]{%
	\notblank{#1}{\renewcommand{\previousreference}{#1}}{}%
	\todo[noline]{%
		\protect\vspace{16pt}%
		\protect\par%
		\protect\notblank{#1}{\cite{[\previousreference]}\\}{}%
		\protect\citelink{\previousreference}{p. #2}%
	}%
}

\definecolor{devcolor}{HTML}{00695c}
\newcommand{\dev}[1]{%
	\reversemarginpar%
	\todo[noline]{
		\protect\vspace{16pt}%
		\protect\par%
		\bfseries\color{devcolor}\href{\website/developpements/#1}{DEV}
	}%
	\normalmarginpar%
}

% En-têtes :

\pagestyle{fancy}
\fancyhead[L]{\truncate{0.23\textwidth}{\thepage}}
\fancyfoot[C]{\scriptsize \href{\website}{\texttt{agreg.skyost.eu}}}

% Couleurs :

\definecolor{property}{HTML}{fffde7}
\definecolor{proposition}{HTML}{fff8e1}
\definecolor{lemma}{HTML}{fff3e0}
\definecolor{theorem}{HTML}{fce4f2}
\definecolor{corollary}{HTML}{ffebee}
\definecolor{definition}{HTML}{ede7f6}
\definecolor{notation}{HTML}{f3e5f5}
\definecolor{example}{HTML}{e0f7fa}
\definecolor{cexample}{HTML}{efebe9}
\definecolor{application}{HTML}{e0f2f1}
\definecolor{remark}{HTML}{e8f5e9}
\definecolor{proof}{HTML}{e1f5fe}

% Théorèmes :

\theoremstyle{definition}
\newtheorem{theorem}{Théorème}

\newtheorem{property}[theorem]{Propriété}
\newtheorem{proposition}[theorem]{Proposition}
\newtheorem{lemma}[theorem]{Lemme}
\newtheorem{corollary}[theorem]{Corollaire}

\newtheorem{definition}[theorem]{Définition}
\newtheorem{notation}[theorem]{Notation}

\newtheorem{example}[theorem]{Exemple}
\newtheorem{cexample}[theorem]{Contre-exemple}
\newtheorem{application}[theorem]{Application}

\theoremstyle{remark}
\newtheorem{remark}[theorem]{Remarque}

\counterwithin*{theorem}{section}

\newcommand{\applystyletotheorem}[1]{
	\tcolorboxenvironment{#1}{
		enhanced,
		breakable,
		colback=#1!98!white,
		boxrule=0pt,
		boxsep=0pt,
		left=8pt,
		right=8pt,
		top=8pt,
		bottom=8pt,
		sharp corners,
		after=\par,
	}
}

\applystyletotheorem{property}
\applystyletotheorem{proposition}
\applystyletotheorem{lemma}
\applystyletotheorem{theorem}
\applystyletotheorem{corollary}
\applystyletotheorem{definition}
\applystyletotheorem{notation}
\applystyletotheorem{example}
\applystyletotheorem{cexample}
\applystyletotheorem{application}
\applystyletotheorem{remark}
\applystyletotheorem{proof}

% Environnements :

\NewEnviron{whitetabularx}[1]{%
	\renewcommand{\arraystretch}{2.5}
	\colorbox{white}{%
		\begin{tabularx}{\textwidth}{#1}%
			\BODY%
		\end{tabularx}%
	}%
}

% Maths :

\DeclareFontEncoding{FMS}{}{}
\DeclareFontSubstitution{FMS}{futm}{m}{n}
\DeclareFontEncoding{FMX}{}{}
\DeclareFontSubstitution{FMX}{futm}{m}{n}
\DeclareSymbolFont{fouriersymbols}{FMS}{futm}{m}{n}
\DeclareSymbolFont{fourierlargesymbols}{FMX}{futm}{m}{n}
\DeclareMathDelimiter{\VERT}{\mathord}{fouriersymbols}{152}{fourierlargesymbols}{147}


% Bibliographie :

\addbibresource{\bibliographypath}%
\defbibheading{bibliography}[\bibname]{%
	\newpage
	\section*{#1}%
}
\renewbibmacro*{entryhead:full}{\printfield{labeltitle}}%
\DeclareFieldFormat{url}{\newline\footnotesize\url{#1}}%

\AtEndDocument{\printbibliography}

\begin{document}
  %<*content>
  \development{analysis}{equivalence-des-normes-en-dimension-finie-et-theoreme-de-riesz}{Équivalence des normes en dimension finie et théorème de Riesz}

  \summary{On montre l'équivalence des normes en dimension finie ainsi que le théorème de Riesz sur la compacité de la boule unité fermée toujours en dimension finie, qui sont deux résultats fondamentaux sur les espaces vectoriels normés.}

  \reference[I-P]{422}

  \begin{lemma}
    \label{equivalence-des-normes-en-dimension-finie-et-theoreme-de-riesz-1}
    Les compacts de $(\mathbb{R}^n, \Vert . \Vert_\infty)$ sont les fermés bornés.
  \end{lemma}

  \begin{proof}
    Soit $X$ une partie fermée bornée de $\mathbb{R}^n$. Soit $(x_n)$ une suite de $X$. On note $\forall k \in \mathbb{N}$, $\forall i \in \llbracket 1, n \rrbracket$, $x_k^i$ la $i$-ième composante du vecteur $x_k$. Comme $X$ est bornée, alors $(\Vert x_n \Vert_\infty)$ est une suite réelle bornée. Montrons par récurrence que, pour tout $k \in \llbracket 1, n \rrbracket$, il existe des extractrices $\varphi_1, \dots, \varphi_i$ telle que la suite réelle $(x^i_{\varphi_1 \circ \dots \circ \varphi_i (n)})$ converge pour tout $i \in \llbracket 1, k \rrbracket$.
    \begin{itemize}
      \item \underline{Pour $k = 1$}, c'est une réécriture du théorème de Bolzano-Weierstrass.
      \item \underline{Pour $k > 1$}, supposons avoir construit $\varphi_1, \dots, \varphi_k$ telles que $(x^i_{\varphi_1 \circ \dots \circ \varphi_k (n)})$ converge pour tout $i \in \llbracket 1, k \rrbracket$. Comme
      \[ \vert x^{k+1}_n \vert \leq \Vert x_n \Vert_\infty \]
      $(x^{k+1}_{\varphi_1 \circ \dots \circ \varphi_k (n)})$ est une suite réelle bornée. Toujours par le théorème de Bolzano-Weierstrass, il existe une extractrice $\varphi_{k+1}$ telle que $(x^{k+1}_{\varphi_1 \circ \dots \circ \varphi_{k+1}}(n))$ converge. D'où l'hérédité.
    \end{itemize}
    La propriété est en particulier pour $k = n$. En posant $\varphi = \varphi_1 \circ \dots \circ \varphi_n$, on obtient une extractrice telle que
    \[ \forall i \in \llbracket 1, n \rrbracket, \, (x^i_{\varphi(n)}) \text{ converge} \]
    et on en déduit que $(x_{\varphi(n)})$ converge vers un réel $x \in \mathbb{R}^n$. Comme $X$ est fermé, $x \in X$. $X$ est donc séquentiellement compact, donc compact.
  \end{proof}

  \begin{proposition}
    \label{equivalence-des-normes-en-dimension-finie-et-theoreme-de-riesz-2}
    Soient $(E,d_E)$, $(F,d_f)$ deux espaces métriques et $f : E \rightarrow F$ continue. Si $E$ est compact, alors $f(E)$ est compact dans $F$.
  \end{proposition}

  \begin{proof}
    Soit $(y_n)$ une suite d'éléments de $f(E)$. On pose $\forall n \in \mathbb{N}$, $x_n = f(y_n)$. $E$ est compact, donc il existe une extractrice $\varphi : \mathbb{N} \rightarrow \mathbb{N}$ telle que $x_{\varphi(n)} \longrightarrow_{n \rightarrow +\infty} x$ où $x \in E$. Par continuité,
    \[ y_{\varphi(n)} = f(x_{\varphi(n)}) \longrightarrow_{n \rightarrow +\infty} f(x) \in f(E) \]
    $f(E)$ est ainsi séquentiellement compact, donc est compact.
  \end{proof}

  \begin{theorem}
    Soit $E$ un espace vectoriel sur le corps $\mathbb{R}$ de dimension finie $n \in \mathbb{N}$. Alors, toutes les normes sur $E$ sont équivalentes.
  \end{theorem}

  \begin{proof}
    Soient $\mathcal{B} = (e_1, \dots, e_n)$ une base de $E$. On définit la norme infinie $\mathcal{N}_\infty$ associée à la base $\mathcal{B}$ pour tout $x = \sum_{i=1}^n x_i e_i \in E$ par
    \[ \mathcal{N}_\infty : x \mapsto \max_{i \in \llbracket 1, n \rrbracket} \vert x_i \vert \]
    Si $\mathcal{N}$ est une norme sur $E$, on a :
    \[ \mathcal{N}(x) \leq \underbrace{\left( \sum_{i=1}^n \mathcal{N}(e_i) \right)}_{= \alpha} \mathcal{N}_\infty(x) \]
    Donc $\mathcal{N}_\infty$ est plus fine que $\mathcal{N}$.
    \newpar
    Définissons l'isomorphisme suivant :
    \[
      f :
      \begin{array}{ccc}
        (\mathbb{R}^n, \Vert . \vert_\infty) &\rightarrow& (E, \mathcal{N}) \\
        (x_1, \dots, x_n) &\mapsto& \sum_{i=1}^n x_i e_i
      \end{array}
    \]
    La fonction $f$ vérifie
    \[ \forall x \in \mathbb{R}^n, \, \mathcal{N}(f(x)) \leq \alpha \Vert x \vert_\infty \]
    c'est une application linéaire bornée, qui est donc continue. Comme elle est bijective, l'ensemble
    \[ S_E = \{ x \in E \mid \mathcal{N}_\infty(x) = 1 \} = f(S) \]
    où $S$ désigne la sphère unité de $(\mathbb{R}^n, \Vert. \Vert_\infty)$, qui est compacte d'après le \cref{equivalence-des-normes-en-dimension-finie-et-theoreme-de-riesz-1}. D'après le \cref{equivalence-des-normes-en-dimension-finie-et-theoreme-de-riesz-2}, $S_E$ est compacte comme image d'un compact par une application continue.  L'application $\mathcal{N} : E \rightarrow \mathbb{R}$ est continue car lipschitzienne ($\forall x, y \in E, \, \vert \mathcal{N}(x) - \mathcal{N}(y) \vert \leq \mathcal{N}(x - y)$), donc est bornée et atteint ses bornes sur la sphère $S_E$. On note $x_0 \in E$ ce minimum :
    \[ \forall x \in E \text{ tel que } \mathcal{N}_\infty(x) = 1, \text{ on a } \mathcal{N}(x) \geq \underbrace{\mathcal{N}(x_0)}_{= \beta} \]
    Ainsi,
    \[ \forall x \in E, \mathcal{N} \left(\frac{x}{\mathcal{N}_\infty(x)} \right) \geq \beta \text{ ie. } \mathcal{N}(x) \geq \beta \mathcal{N}_\infty(x) \]
    Donc $\mathcal{N}$ est plus fine que $\mathcal{N}_\infty$ : les normes $\mathcal{N}$ et $\mathcal{N}_\infty$ sont équivalentes. Comme la relation d'équivalence sur les normes d'un espace vectoriel est transitive, on en déduit que toutes les normes sur $E$ sont équivalentes.
  \end{proof}

  \begin{theorem}[Riesz]
    Soit $(E, \Vert . \Vert)$ un espace vectoriel normé sur le corps $\mathbb{R}$. Alors, $E$ est de dimension finie si et seulement si sa boule unité fermée est compacte.
  \end{theorem}

  \begin{proof}
    Notons $\overline{B}$ la boule unité fermée de $E$ et supposons $E$ de dimension finie $n \in \mathbb{N}$. Comme dans la démonstration du théorème précédent, $\overline{B}$ est compacte comme image de la boule unité fermée de $\mathbb{R}^n$ par l'application continue $f$. Réciproquement, supposons $E$ de dimension finie et, par l'absurde, également que $\overline{B}$ est compacte. On a,
    \[ \overline{B} \subseteq \bigcup_{x \in E} B(x,1) \]
    où $B(x,1)$ désigne la boule ouverte centrée en $x$ de rayon $1$. Par la propriété de Borel-Lebesgue, il existe $x_1, \dots, x_n \in E$ tels que
    \[ \overline{B} \subseteq \bigcup_{i=1}^n B(x_i,1) \]
    On définit $F = \operatorname{Vect}(x_1, \dots, x_n)$. Comme $F$ est de dimension finie et $E$ de dimension infinie, on peut trouver $y \in E \setminus F$. Soit $x_0 \in F$ le projeté de $y$ sur $F$ :
    \[ d(y, F) = \Vert y - x_0 \Vert \]
    On pose
    \[ u = \frac{y - x_0}{\Vert y - x_0 \Vert} \]
    On a $u$ de norme $1$, donc $u \in \overline{B}$ et il existe $i \in \llbracket 1, n \rrbracket$ tel que $\Vert u - x_i \Vert < 1$. Or,
    \begin{align*}
      \Vert u - x_i \Vert &= \frac{\Vert y - x_0 - \Vert y - x_0 \Vert x_i \Vert}{\Vert y - x_0 \Vert} \\
      &= \frac{\Vert y - (x_0 - \Vert y - x_0 \Vert x_i) \Vert}{\Vert y - x_0 \Vert} \\
      &\geq \frac{d(y, F)}{\Vert y - x_0 \Vert} \\
      &= 1
    \end{align*}
    car $x_0 + \Vert y - x_0 \Vert x_i \in F$ : absurde.
  \end{proof}
  %</content>
\end{document}
