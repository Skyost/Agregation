\documentclass[12pt, a4paper]{report}

% LuaLaTeX :

\RequirePackage{iftex}
\RequireLuaTeX

% Packages :

\usepackage[french]{babel}
%\usepackage[utf8]{inputenc}
%\usepackage[T1]{fontenc}
\usepackage[pdfencoding=auto, pdfauthor={Hugo Delaunay}, pdfsubject={Mathématiques}, pdfcreator={agreg.skyost.eu}]{hyperref}
\usepackage{amsmath}
\usepackage{amsthm}
%\usepackage{amssymb}
\usepackage{stmaryrd}
\usepackage{tikz}
\usepackage{tkz-euclide}
\usepackage{fourier-otf}
\usepackage{fontspec}
\usepackage{titlesec}
\usepackage{fancyhdr}
\usepackage{catchfilebetweentags}
\usepackage[french, capitalise, noabbrev]{cleveref}
\usepackage[fit, breakall]{truncate}
\usepackage[top=2.5cm, right=2cm, bottom=2.5cm, left=2cm]{geometry}
\usepackage{enumerate}
\usepackage{tocloft}
\usepackage{microtype}
%\usepackage{mdframed}
%\usepackage{thmtools}
\usepackage{xcolor}
\usepackage{tabularx}
\usepackage{aligned-overset}
\usepackage[subpreambles=true]{standalone}
\usepackage{environ}
\usepackage[normalem]{ulem}
\usepackage{marginnote}
\usepackage{etoolbox}
\usepackage{setspace}
\usepackage[bibstyle=reading, citestyle=draft]{biblatex}
\usepackage{xpatch}
\usepackage[many, breakable]{tcolorbox}
\usepackage[backgroundcolor=white, bordercolor=white, textsize=small]{todonotes}

% Bibliographie :

\newcommand{\overridebibliographypath}[1]{\providecommand{\bibliographypath}{#1}}
\overridebibliographypath{../bibliography.bib}
\addbibresource{\bibliographypath}
\defbibheading{bibliography}[\bibname]{%
	\newpage
	\section*{#1}%
}
\renewbibmacro*{entryhead:full}{\printfield{labeltitle}}
\DeclareFieldFormat{url}{\newline\footnotesize\url{#1}}
\AtEndDocument{\printbibliography}

% Police :

\setmathfont{Erewhon Math}

% Tikz :

\usetikzlibrary{calc}

% Longueurs :

\setlength{\parindent}{0pt}
\setlength{\headheight}{15pt}
\setlength{\fboxsep}{0pt}
\titlespacing*{\chapter}{0pt}{-20pt}{10pt}
\setlength{\marginparwidth}{1.5cm}
\setstretch{1.1}

% Métadonnées :

\author{agreg.skyost.eu}
\date{\today}

% Titres :

\setcounter{secnumdepth}{3}

\renewcommand{\thechapter}{\Roman{chapter}}
\renewcommand{\thesubsection}{\Roman{subsection}}
\renewcommand{\thesubsubsection}{\arabic{subsubsection}}
\renewcommand{\theparagraph}{\alph{paragraph}}

\titleformat{\chapter}{\huge\bfseries}{\thechapter}{20pt}{\huge\bfseries}
\titleformat*{\section}{\LARGE\bfseries}
\titleformat{\subsection}{\Large\bfseries}{\thesubsection \, - \,}{0pt}{\Large\bfseries}
\titleformat{\subsubsection}{\large\bfseries}{\thesubsubsection. \,}{0pt}{\large\bfseries}
\titleformat{\paragraph}{\bfseries}{\theparagraph. \,}{0pt}{\bfseries}

\setcounter{secnumdepth}{4}

% Table des matières :

\renewcommand{\cftsecleader}{\cftdotfill{\cftdotsep}}
\addtolength{\cftsecnumwidth}{10pt}

% Redéfinition des commandes :

\renewcommand*\thesection{\arabic{section}}
\renewcommand{\ker}{\mathrm{Ker}}

% Nouvelles commandes :

\newcommand{\website}{https://agreg.skyost.eu}

\newcommand{\tr}[1]{\mathstrut ^t #1}
\newcommand{\im}{\mathrm{Im}}
\newcommand{\rang}{\operatorname{rang}}
\newcommand{\trace}{\operatorname{trace}}
\newcommand{\id}{\operatorname{id}}
\newcommand{\stab}{\operatorname{Stab}}

\providecommand{\newpar}{\\[\medskipamount]}

\providecommand{\lesson}[3]{%
	\title{#3}%
	\hypersetup{pdftitle={#3}}%
	\setcounter{section}{\numexpr #2 - 1}%
	\section{#3}%
	\fancyhead[R]{\truncate{0.73\textwidth}{#2 : #3}}%
}

\providecommand{\development}[3]{%
	\title{#3}%
	\hypersetup{pdftitle={#3}}%
	\section*{#3}%
	\fancyhead[R]{\truncate{0.73\textwidth}{#3}}%
}

\providecommand{\summary}[1]{%
	\textit{#1}%
	\medskip%
}

\tikzset{notestyleraw/.append style={inner sep=0pt, rounded corners=0pt, align=center}}

%\newcommand{\booklink}[1]{\website/bibliographie\##1}
\newcommand{\citelink}[2]{\hyperlink{cite.\therefsection @#1}{#2}}
\newcommand{\previousreference}{}
\providecommand{\reference}[2][]{%
	\notblank{#1}{\renewcommand{\previousreference}{#1}}{}%
	\todo[noline]{%
		\protect\vspace{16pt}%
		\protect\par%
		\protect\notblank{#1}{\cite{[\previousreference]}\\}{}%
		\protect\citelink{\previousreference}{p. #2}%
	}%
}

\definecolor{devcolor}{HTML}{00695c}
\newcommand{\dev}[1]{%
	\reversemarginpar%
	\todo[noline]{
		\protect\vspace{16pt}%
		\protect\par%
		\bfseries\color{devcolor}\href{\website/developpements/#1}{DEV}
	}%
	\normalmarginpar%
}

% En-têtes :

\pagestyle{fancy}
\fancyhead[L]{\truncate{0.23\textwidth}{\thepage}}
\fancyfoot[C]{\scriptsize \href{\website}{\texttt{agreg.skyost.eu}}}

% Couleurs :

\definecolor{property}{HTML}{fffde7}
\definecolor{proposition}{HTML}{fff8e1}
\definecolor{lemma}{HTML}{fff3e0}
\definecolor{theorem}{HTML}{fce4f2}
\definecolor{corollary}{HTML}{ffebee}
\definecolor{definition}{HTML}{ede7f6}
\definecolor{notation}{HTML}{f3e5f5}
\definecolor{example}{HTML}{e0f7fa}
\definecolor{cexample}{HTML}{efebe9}
\definecolor{application}{HTML}{e0f2f1}
\definecolor{remark}{HTML}{e8f5e9}
\definecolor{proof}{HTML}{e1f5fe}

% Théorèmes :

\theoremstyle{definition}
\newtheorem{theorem}{Théorème}

\newtheorem{property}[theorem]{Propriété}
\newtheorem{proposition}[theorem]{Proposition}
\newtheorem{lemma}[theorem]{Lemme}
\newtheorem{corollary}[theorem]{Corollaire}

\newtheorem{definition}[theorem]{Définition}
\newtheorem{notation}[theorem]{Notation}

\newtheorem{example}[theorem]{Exemple}
\newtheorem{cexample}[theorem]{Contre-exemple}
\newtheorem{application}[theorem]{Application}

\theoremstyle{remark}
\newtheorem{remark}[theorem]{Remarque}

\counterwithin*{theorem}{section}

\newcommand{\applystyletotheorem}[1]{
	\tcolorboxenvironment{#1}{
		enhanced,
		breakable,
		colback=#1!98!white,
		boxrule=0pt,
		boxsep=0pt,
		left=8pt,
		right=8pt,
		top=8pt,
		bottom=8pt,
		sharp corners,
		after=\par,
	}
}

\applystyletotheorem{property}
\applystyletotheorem{proposition}
\applystyletotheorem{lemma}
\applystyletotheorem{theorem}
\applystyletotheorem{corollary}
\applystyletotheorem{definition}
\applystyletotheorem{notation}
\applystyletotheorem{example}
\applystyletotheorem{cexample}
\applystyletotheorem{application}
\applystyletotheorem{remark}
\applystyletotheorem{proof}

% Environnements :

\NewEnviron{whitetabularx}[1]{%
	\renewcommand{\arraystretch}{2.5}
	\colorbox{white}{%
		\begin{tabularx}{\textwidth}{#1}%
			\BODY%
		\end{tabularx}%
	}%
}

% Maths :

\DeclareFontEncoding{FMS}{}{}
\DeclareFontSubstitution{FMS}{futm}{m}{n}
\DeclareFontEncoding{FMX}{}{}
\DeclareFontSubstitution{FMX}{futm}{m}{n}
\DeclareSymbolFont{fouriersymbols}{FMS}{futm}{m}{n}
\DeclareSymbolFont{fourierlargesymbols}{FMX}{futm}{m}{n}
\DeclareMathDelimiter{\VERT}{\mathord}{fouriersymbols}{152}{fourierlargesymbols}{147}


% Bibliographie :

\addbibresource{\bibliographypath}%
\defbibheading{bibliography}[\bibname]{%
	\newpage
	\section*{#1}%
}
\renewbibmacro*{entryhead:full}{\printfield{labeltitle}}%
\DeclareFieldFormat{url}{\newline\footnotesize\url{#1}}%

\AtEndDocument{\printbibliography}

\begin{document}
  %<*content>
  \development{algebra}{theoreme-de-sylow}{Premier théorème de Sylow}

  \summary{En procédant par récurrence sur le cardinal du groupe, on montre l'existence d'un sous-groupe de Sylow.}

  \reference[GOU21]{44}

  \begin{theorem}[Cauchy ``faible'']
    \label{theoreme-de-sylow-1}
    Soit $G$ un groupe abélien fini et soit $p$ un diviseur premier de l'ordre de $G$. Alors, il existe un sous-groupe de $G$ d'ordre $p$.
  \end{theorem}

  \begin{proof}
    $G$ est fini, on peut donc l'écrire
    \[ G = \langle x_1, \dots, x_n \rangle \]
    où $(x_1, \dots, x_n)$ est un système de générateurs de $G$. On définit
    \[
      \varphi :
      \begin{array}{ccc}
        \langle x_1 \rangle \times \dots \times \langle x_n \rangle &\rightarrow& G \\
        (y_1, \dots, y_n) &\mapsto& y_1 \dots y_n
      \end{array}
    \]
    Comme $G$ est abélien, $\varphi$ est clairement un morphisme de groupes. Et comme $(x_1, \dots, x_n)$  est un système de générateurs de $G$, $\varphi$ est surjectif.
    On peut appliquer le premier théorème d'isomorphisme pour obtenir
    \[ G \cong (\langle x_1 \rangle \times \dots \times \langle x_n \rangle) / \ker(\varphi) \]
    En particulier, $\vert G \vert \times \vert \ker(\varphi) \vert = \vert \langle x_1 \rangle \vert \times \dots \times \vert \langle x_n \rangle \vert$. On note, pour tout $i \in \llbracket 1, n \rrbracket$, $r_i = \vert \langle x_i \rangle \vert$. On a ainsi,
    \[ G \mid r_1 \dots r_n \implies p \mid r_1 \dots r_n \]
    par transitivité de $\mid$. Par le lemme d'Euclide, il existe $i \in \llbracket 1, n \rrbracket$ tel que $p \mid r_i$. On écrit $r_i = pq$ avec $q \in \mathbb{N}^*$, et on pose $x = x_i^q$. Alors, $x$ est d'ordre $p$ et $H = \langle x \rangle$ est un sous-groupe de $G$ d'ordre $p$.
  \end{proof}

  \begin{theorem}[Premier théorème de Sylow]
    Soit $G$ un groupe fini d'ordre $n p^\alpha$ avec $n, \alpha \in \mathbb{N}$ et $p$ premier tel que $p \nmid n$. Alors, il existe un sous-groupe de $G$ d’ordre $p^\alpha$.
  \end{theorem}

  \begin{proof}
    Posons $h = \vert G \vert$. On va procéder par récurrence forte sur $h$.
    \begin{itemize}
      \item \uline{Si $h$ = 1 :} Alors, $n = 1$ et $\alpha = 0$. La propriété est donc triviale.
      \item \uline{On suppose la propriété vraie pour les groupes d'ordre strictement inférieur à $h$.} Si $\alpha = 0$, c'est encore une fois trivial, pour les mêmes raisons qu'à l'initialisation de la propriété. Supposons donc $\alpha \geq 1$. On fait agir $G$ sur lui-même par conjugaison, via l'action :
      \[ (g,h) \mapsto ghg^{-1} \]
      Soit $\Omega$ un système de représentants associé à la relation ``être dans la même orbite''. La formule des classes donne
      \[ |G| = \sum_{\omega \in \Omega} |G \cdot \omega| = \sum_{\omega \in \Omega} (G : \stab_G(\omega)) = \sum_{\omega \in \Omega} \frac{|G|}{|\stab_G(\omega)|} \tag{$*$} \]
      Mais,
      \[ \stab_G(\omega) = G \iff \forall g \in G, \, g \omega g^{-1} = \omega \iff \omega \in Z(G) \]
      donc, en regroupant, on peut réécrire $(*)$ :
      \begin{align*}
        |G| &= \sum_{\omega \in \Omega} \frac{|G|}{|\stab_G(\omega)|} \\
        & = \sum_{\omega \in Z(G)} \frac{|G|}{|\stab_G(\omega)|} + \sum_{\omega \notin Z(G)} \frac{|G|}{|\stab_G(\omega)|} \\
        &= \vert Z(G) \vert + \sum_{\omega \notin Z(G)} \frac{|G|}{|\stab_G(\omega)|} \tag{$**$}
      \end{align*}
      On a maintenant deux cas :
      \begin{itemize}
        \item \uline{Il existe $\omega$ tel que $p^\alpha \mid |\stab_G(\omega)|$ :} Alors, $|\stab_G(\omega)|$ est un sous-groupe de $G$ d'ordre strictement inférieur à $\vert G \vert$ qui vérifie bien la propriété escomptée.
        \item \uline{Pour tout $\omega$, $p^\alpha \nmid |\stab_G(\omega)|$ :} Comme $p^\alpha \mid h$, $p \mid \frac{|G|}{|\stab_G(\omega)|}$ pour tout $\omega$. D'après $(**)$, on a
        \[ p \mid \vert Z(G) \vert \]
        $Z(G)$ étant commutatif, on peut appliquer le \cref{theoreme-de-sylow-1}. On obtient l'existence d'un sous-groupe $H$ de $Z(G)$ d'ordre $p$, qui est de plus distingué dans $G$ car inclus dans $Z(G)$. Alors,
        \[ \vert G/H \vert = \frac{\vert G \vert}{\vert H \vert} = np^{\alpha - 1} \]
        Il suffit maintenant d'appliquer l'hypothèse de récurrence à $G/H$, qui donne l'existence d'un sous-groupe $K$ de $G/H$ d'ordre $p^{\alpha - 1}$. On considère la surjection canonique
        \[ \pi_H : G \rightarrow G/H \]
        Alors, $\pi_H^{-1}(K) = \{ g \in G \mid gH \in K \}$ est un sous-groupe de $G$ d'ordre $\vert K \vert \times \vert H \vert = p^\alpha$ :
        \begin{center}
          \begin{tikzpicture}[scale=0.65]
            \draw[very thin, fill=teal!10, rotate=10] (-0.25,-1) ellipse (3.75 and 2.25);
            \draw (0.25,-2.5) circle (4.5);
            \begin{scope}[shift={(7,0)}]
              \draw[very thin, fill=teal!10, rotate=50] (0,0.4) ellipse (1.3 and 0.85);
              \foreach \a in {1,...,5} {
                \node at ({\a*72}:0.95) {\tiny $g_{\a}H$};
                \node[coordinate] (\a) at ($({\a*72}:0.95)+(0,0.2)$) {};
              }
              \draw(0,0) circle (1.5);
              \node at (0,0) {$K$};
            \end{scope}
            \foreach \center [count=\i] in {(-0.25,0),(2.25,-1),(-2,-1.75)} {
              \filldraw[teal!20] \center circle (1.1);
              \draw [very thin, ->] \center to [out=70,in=110] (\i);
            }
            \foreach \center [count=\i] in {(-0.25,0),(2.25,-1),(-2,-1.75),(2,-4.25),(-1,-5)} {
              \draw[dashed] \center circle (1.1);
              \begin{scope}[shift={(\center)}]
                \foreach \a [count=\j] in {1,...,5} {
                  \node at ({\a*72}:0.75) {\tiny $g_{\i}h_{\j}$};
                }
              \end{scope}
            }
            \node at (4.5,2) {$\pi_H$};
            \node at (0.5,-2.5) {$\pi_H^{-1}(K)$};
            \node[align=left] at (7.75,-6) {Partition de $G$ selon $\sim$ \\ où $g \sim h \iff g^{-1}h \in H$};
            \node at (9.5,-1) {$G/H$};
          \end{tikzpicture}
          % Dessin : https://agreg-maths.fr/uploads/versions/3229/Dev_Sylow_Reccurrence.pdf.
        \end{center}
        ce qu'on voulait.
      \end{itemize}
    \end{itemize}
  \end{proof}
  %</content>
\end{document}
