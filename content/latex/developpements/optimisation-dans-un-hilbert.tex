\documentclass[12pt, a4paper]{report}

% LuaLaTeX :

\RequirePackage{iftex}
\RequireLuaTeX

% Packages :

\usepackage[french]{babel}
%\usepackage[utf8]{inputenc}
%\usepackage[T1]{fontenc}
\usepackage[pdfencoding=auto, pdfauthor={Hugo Delaunay}]{hyperref}
\usepackage{amsmath}
\usepackage{amsthm}
%\usepackage{amssymb}
\usepackage{stmaryrd}
\usepackage{tikz}
\usepackage{tkz-euclide}
\usepackage{fourier-otf}
\usepackage{fontspec}
\usepackage{titlesec}
\usepackage{fancyhdr}
\usepackage{catchfilebetweentags}
\usepackage[french, capitalise, noabbrev]{cleveref}
\usepackage[fit, breakall]{truncate}
\usepackage[margin=2.5cm]{geometry}
\usepackage{enumerate}
\usepackage{tocloft}
\usepackage{microtype}
\usepackage{mdframed}
\usepackage{thmtools}
\usepackage{xcolor}
\usepackage{tabularx}
\usepackage{aligned-overset}
\usepackage[subpreambles=true]{standalone}
\usepackage{environ}
\usepackage[normalem]{ulem}
\usepackage{marginnote}
\usepackage{etoolbox}
\usepackage{setspace}
\usepackage[bibstyle=reading, citestyle=draft]{biblatex}
\usepackage{xpatch}

% Bibliographie :

\providecommand{\bibliographypath}{../bibliography.bib}
\addbibresource{\bibliographypath}
\defbibheading{bibliography}[\bibname]{%
	\newpage
	\section*{#1}%
}
\renewbibmacro*{entryhead:full}{\printfield{labeltitle}}
\DeclareFieldFormat{url}{\newline\footnotesize\url{#1}}
\AtEndDocument{\printbibliography}

% Police :

\setmathfont{Erewhon Math}

% Tikz :

\usetikzlibrary{calc}

% Longueurs :

\setlength{\parindent}{0pt}
\setlength{\headheight}{15pt}
\setlength{\fboxsep}{0pt}
\titlespacing*{\chapter}{0pt}{-20pt}{10pt}
\setlength{\marginparwidth}{1.5cm}
\setstretch{1.1}

% Métadonnées :

\author{agreg.skyost.eu}
\date{\today}

% Titres :

\setcounter{secnumdepth}{3}

\renewcommand{\thechapter}{\Roman{chapter}}
\renewcommand{\thesubsection}{\Roman{subsection}}
\renewcommand{\thesubsubsection}{\arabic{subsubsection}}

\titleformat{\chapter}{\huge\bfseries}{\thechapter}{20pt}{\huge\bfseries}
\titleformat*{\section}{\LARGE\bfseries}
\titleformat{\subsection}{\Large\bfseries}{\thesubsection \, - \,}{0pt}{\Large\bfseries}
\titleformat{\subsubsection}{\large\bfseries}{\thesubsubsection. \,}{0pt}{\large\bfseries}

% Table des matières :

\renewcommand{\cftsecleader}{\cftdotfill{\cftdotsep}}
\addtolength{\cftsecnumwidth}{10pt}

% Redéfinition des commandes :

\renewcommand*\thesection{\arabic{section}}
\renewcommand{\ker}{\mathrm{Ker}}

% Nouvelles commandes :

\newcommand{\website}{https://agreg.skyost.eu}

\newcommand{\tr}[1]{\mathstrut ^t #1}
\newcommand{\im}{\mathrm{Im}}
\newcommand{\rang}{\operatorname{rang}}
\newcommand{\trace}{\operatorname{trace}}
\newcommand{\id}{\operatorname{id}}
\newcommand{\stab}{\operatorname{Stab}}

\providecommand{\newpar}{\\[\medskipamount]}

\providecommand{\lesson}[3]{%
	\title{#3}%
	\hypersetup{pdftitle={#3}}%
	\setcounter{section}{\numexpr #2 - 1}%
	\section{#3}%
	\fancyhead[R]{\truncate{0.73\textwidth}{#2 : #3}}%
}

\providecommand{\development}[3]{%
	\title{#3}%
	\hypersetup{pdftitle={#3}}%
	\section*{#3}%
	\fancyhead[R]{\truncate{0.73\textwidth}{#3}}%
}

\providecommand{\summary}[1]{%
	\textit{#1}%
	\medskip%
}

%\newcommand{\booklink}[1]{\website/bibliographie\##1}
\newcommand{\citelink}[2]{\hyperlink{cite.\therefsection @#1}{#2}}
\newcommand{\previousreference}{}
\providecommand{\reference}[2][]{%
	\notblank{#1}{\renewcommand{\previousreference}{#1}}{}%
	\marginnote{%
		\centering%
		\notblank{#1}{\cite{[\previousreference]}\\}{}%
		\citelink{\previousreference}{p. #2}%
	}%
}

\newcommand{\imagespath}{../images/}

\providecommand{\includelatexpicture}[1]{%
	\begin{center}%
		\input{\imagespath#1}%
	\end{center}%
	\medskip%
}

\definecolor{devcolor}{HTML}{00695c}
\newcommand{\dev}[1]{%
	\reversemarginpar%
	\marginnote[\bfseries\color{devcolor}\href{\website/developpements/#1}{DEV}]{}%
	\normalmarginpar%
}

% En-têtes :

\pagestyle{fancy}
\fancyhead[L]{\truncate{0.23\textwidth}{\thepage}}
\fancyfoot[C]{\scriptsize \href{\website}{\texttt{agreg.skyost.eu}}}

% Couleurs :

\definecolor{property}{HTML}{fffde7}
\definecolor{proposition}{HTML}{fff8e1}
\definecolor{lemma}{HTML}{fff3e0}
\definecolor{theorem}{HTML}{fce4f2}
\definecolor{corollary}{HTML}{ffebee}
\definecolor{definition}{HTML}{ede7f6}
\definecolor{notation}{HTML}{f3e5f5}
\definecolor{example}{HTML}{e0f7fa}
\definecolor{cexample}{HTML}{efebe9}
\definecolor{application}{HTML}{e0f2f1}
\definecolor{remark}{HTML}{e8f5e9}
\definecolor{demonstration}{HTML}{e1f5fe}

% Théorèmes :

\declaretheoremstyle[bodyfont=\normalfont]{theorem}
\declaretheoremstyle[headfont=\itshape]{remark}

\newcounter{thm}[section]

\newcommand{\newth}[3][style=theorem]{
	\declaretheorem[%
		name=#3,%
		%within=section,
		sibling=thm,%
		mdframed={%
			hidealllines=true,%
			backgroundcolor={#2!98!white},%
			innermargin=8pt,%
			splittopskip=18pt,%
			splitbottomskip=16pt,%
		},%
		#1%
	]{#2}%
}

\newth{property}{Propriété}
\newth{proposition}{Proposition}
\newth{lemma}{Lemme}
\newth{theorem}{Théorème}
\newth{corollary}{Corollaire}
\newth{definition}{Définition}
\newth{notation}{Notation}
\newth{example}{Exemple}
\newth{cexample}{Contre-exemple}
\newth{application}{Application}
\newth[style=remark]{remark}{Remarque}
\newth[style=remark, numbered=no, qed=\textsquare]{demonstration}{Démonstration}

% Environnements :

\NewEnviron{whitetabularx}[1]{%
	\renewcommand{\arraystretch}{2.5}
	\colorbox{white}{%
		\begin{tabularx}{\textwidth}{#1}%
			\BODY%
		\end{tabularx}%
	}%
}

% Maths :

\DeclareFontEncoding{FMS}{}{}
\DeclareFontSubstitution{FMS}{futm}{m}{n}
\DeclareFontEncoding{FMX}{}{}
\DeclareFontSubstitution{FMX}{futm}{m}{n}
\DeclareSymbolFont{fouriersymbols}{FMS}{futm}{m}{n}
\DeclareSymbolFont{fourierlargesymbols}{FMX}{futm}{m}{n}
\DeclareMathDelimiter{\VERT}{\mathord}{fouriersymbols}{152}{fourierlargesymbols}{147}


% Bibliographie :

\addbibresource{\bibliographypath}%
\defbibheading{bibliography}[\bibname]{%
	\newpage
	\section*{#1}%
}
\renewbibmacro*{entryhead:full}{\printfield{labeltitle}}%
\DeclareFieldFormat{url}{\newline\footnotesize\url{#1}}%

\AtEndDocument{\printbibliography}

\begin{document}
  %<*content>
  \development{analysis}{optimisation-dans-un-hilbert}{Optimisation dans un Hilbert}

  \summary{On prouve l'existence d'un minimum pour certains types de fonctions définies sur des espaces de Hilbert en utilisant les théorèmes hilbertiens classiques.}

  Soit $H$ un espace de Hilbert réel de norme $\Vert . \Vert$ et dont on note $\langle ., . \rangle$ le produit scalaire associé.

  \reference[I-P]{336}

  \begin{theorem}
    Soit $J : H \rightarrow \mathbb{R}$ une fonction convexe, continue et vérifiant
    %\[ \forall M \in \mathbb{R}, \, \exists r > 0 \text{ tel que } \forall x \in H \text{ de norme } \geq r, \, f(x) \geq M \]
    \[ \forall (x_k) \in H^{\mathbb{N}} \text{ telle que } \Vert x_k \Vert \longrightarrow_{k \rightarrow +\infty} +\infty \text{ alors } J(x_k) \longrightarrow_{k \rightarrow +\infty} +\infty \]
    Alors, il existe $a \in H$ tel que
    \[ J(a) = \inf_{h \in H} J(h) \]
  \end{theorem}

  \begin{proof}
    Soit $(x_k)$ une suite d'éléments de $H$ telle que $(J(x_k))$ converge vers $\inf_{h \in H} J(h)$. Supposons par l'absurde que $(x_k)$ n'est pas bornée. Il existe alors une extractrice $\varphi$ telle que
    \[ \Vert x_{\varphi(k)} \Vert \longrightarrow_{k \rightarrow +\infty} +\infty \]
    Or, par hypothèse, ceci entraîne $J(x_k) \longrightarrow_{k \rightarrow +\infty} +\infty$ : absurde. On en déduit que $(x_k)$ est bornée. Il existe alors $C > 0$ tel que $\Vert x_k \Vert \leq C$ pour tout $k \in \mathbb{N}$. On considère la suite $(\langle x_0, x_k \rangle)$. Elle est bornée car
    \[ \vert \langle x_0, x_k \rangle \vert \overset{\text{Cauchy-Schwarz}}{\leq} \Vert x_0 \Vert \Vert x_k \Vert \]
    donc, par le théorème de Bolzano-Weierstrass, il existe une extractrice $\varphi_0$ telle que la suite $(\langle x_0, x_{\varphi_0(k)} \rangle)$ converge. Par récurrence, supposons avoir construit $\varphi_0, \dots, \varphi_i$ des extractrices telles que $(\langle x_0, x_{\varphi_0, \dots, \varphi_i(k)}) \rangle)$ converge. Comme précédemment, la suite $(\langle x_{i+1}, x_{\varphi_0, \dots, \varphi_i(k)}) \rangle)$ est bornée. On en déduit, par le théorème de Bolzano-Weierstrass, qu'il existe une extractrice $\varphi_{i+1}$ telle que $(\langle x_{i+1}, x_{\varphi_0, \dots, \varphi_{i+1}(k)}) \rangle)$ converge. On crée donc comme cela une suite d'extractrices $(\varphi_i)$. On définit alors
    \[ \varphi : k \mapsto \varphi_0 \circ \dots \circ \varphi_k(k) \]
    et on a la convergence de $(\langle x_i, x_{\varphi(k)} \rangle)$ pour tout $i \in \mathbb{N}$ car $(\varphi(k))$ est une suite extraite de $(\varphi_0 \circ \dots \circ \varphi_k(k))$. On pose maintenant $F = \operatorname{Vect}(x_k)_{k \in \mathbb{N}}$. Par linéarité, $(\langle v, x_{\varphi(k)} \rangle)$ converge pour tout $v \in F$. De plus, comme $H$ est un espace de Hilbert,
    \[ H = \overline{F} \oplus F^{\perp} \tag{$*$} \]
    \newpar
    On définit la suite $(y_n)$ par
    \[ \forall n \in \mathbb{N}, \, y_n = x_{\varphi(n)} \]
    Montrons que pour tout $u \in H$, la suite $(\langle u, y_k \rangle)$ converge. Soient $u \in H$ et $\epsilon > 0$. Par $(*)$,
    \[ \exists (v,w) \in \overline{F} \times F^{\perp} \text{ tel que } u = v + w \]
    ainsi que $\widetilde{v} \in F$ tel que $\Vert v - \widetilde{v} \Vert \leq \epsilon$. Pour tout $k, l$ entiers, on a :
    \[ \vert \langle u, y_k - y_l \rangle \vert = \vert \langle v, y_k - y_l \rangle \vert \leq \Vert v - \widetilde{v} \Vert \Vert y_k - y_l \Vert + \langle \widetilde{v}, y_k - y_l \rangle \]
    Comme la suite $(\langle \widetilde{v}, y_k \rangle)$ converge, elle est de Cauchy. Il existe donc un entier $N$ tel que pour tout $k, l \geq N$, $\vert \langle \widetilde{v}, y_k - y_l \rangle \vert \leq \epsilon$. Ainsi, pour tout $k, l \geq N$,
    \begin{align*}
      \vert \langle u, y_k - y_l \rangle \vert &\leq \Vert v - \widetilde{v} \Vert \Vert y_k - y_l \Vert + \langle \widetilde{v}, y_k - y_l \rangle \\
      &\leq \epsilon (\Vert y_k \Vert + \Vert y_l \Vert) + \epsilon \\
      &\leq \epsilon 2 C + \epsilon
    \end{align*}
    On en déduit que $(\langle u, y_k \rangle)$ est une suite de Cauchy réelle, donc est convergente vers une limite $\ell_u \in \mathbb{R}$. On définit
    \[ \psi : u \mapsto \ell_u \]
    par linéarité de $u \mapsto \langle u, y_k \rangle$ et par unicité de la limite, $\psi$ est une forme linéaire. Comme $(x_k)$ est bornée, on a
    \[ \vert \psi(u) \vert \overset{\text{Cauchy-Schwarz}}{\leq} C \Vert u \Vert \]
    ce qui implique la continuité de $\psi$. On peut appliquer le théorème de représentation de Riesz, qui donne l'existence d'un unique $a \in H$ tel que
    \[ \forall u \in H, \, \psi(u) = \langle a, u \rangle \]
    Ainsi, pour tout $u \in H$, la suite $(\langle u, y_k \rangle)$ converge vers $\langle u, a \rangle$.
    \newpar
    Il reste à montrer que le minimum de $J$ sur $H$ est bien atteint en $a$. Soit $\beta > \inf_{h \in H} J(h)$. On définit
    \[ C_\beta = \{ x \in H \mid J(x) \leq \beta \} \]
    C'est un convexe fermé, non vide de $H$, et par le théorème de projection sur un convexe fermé, on peut définir la projection orthogonale $p_\beta : H \rightarrow C_\beta$. Comme $(J(x_k))$ converge vers $\inf_{h \in H} J(h)$, $J(y_k)$ aussi. Ainsi, il existe $N \in \mathbb{N}$ tel que $\forall k \geq N$, $y_k \in C_\beta$. Donc, d'après la caractérisation angulaire de la projection orthogonale,
    \[ \langle y_k - p_\beta(a), a - p_\beta(a) \rangle \leq 0 \]
    Or, $(\langle y_k, a -p_\beta(a) \rangle)$ converge vers $\langle a, a - p_\beta(a) \rangle$, donc en déduit que $\Vert a - p(a) \Vert^2 \leq 0$. Ce qui aboutit à $a = p_\beta(a)$ et $a \in C_\beta$. Ainsi, $J(a) \leq \beta$ pour tout $\beta \in \mathbb{R}$ tel que $\beta > \inf_{h \in H} J(h)$. On en déduit que $J(a) = \inf_{h \in H} J(h)$.
  \end{proof}
  %</content>
\end{document}
