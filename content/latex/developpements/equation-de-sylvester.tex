\documentclass[12pt, a4paper]{report}

% LuaLaTeX :

\RequirePackage{iftex}
\RequireLuaTeX

% Packages :

\usepackage[french]{babel}
%\usepackage[utf8]{inputenc}
%\usepackage[T1]{fontenc}
\usepackage[pdfencoding=auto, pdfauthor={Hugo Delaunay}, pdfsubject={Mathématiques}, pdfcreator={agreg.skyost.eu}]{hyperref}
\usepackage{amsmath}
\usepackage{amsthm}
%\usepackage{amssymb}
\usepackage{stmaryrd}
\usepackage{tikz}
\usepackage{tkz-euclide}
\usepackage{fourier-otf}
\usepackage{fontspec}
\usepackage{titlesec}
\usepackage{fancyhdr}
\usepackage{catchfilebetweentags}
\usepackage[french, capitalise, noabbrev]{cleveref}
\usepackage[fit, breakall]{truncate}
\usepackage[top=2.5cm, right=2cm, bottom=2.5cm, left=2cm]{geometry}
\usepackage{enumerate}
\usepackage{tocloft}
\usepackage{microtype}
%\usepackage{mdframed}
%\usepackage{thmtools}
\usepackage{xcolor}
\usepackage{tabularx}
\usepackage{aligned-overset}
\usepackage[subpreambles=true]{standalone}
\usepackage{environ}
\usepackage[normalem]{ulem}
\usepackage{marginnote}
\usepackage{etoolbox}
\usepackage{setspace}
\usepackage[bibstyle=reading, citestyle=draft]{biblatex}
\usepackage{xpatch}
\usepackage[many, breakable]{tcolorbox}
\usepackage[backgroundcolor=white, bordercolor=white, textsize=small]{todonotes}

% Bibliographie :

\newcommand{\overridebibliographypath}[1]{\providecommand{\bibliographypath}{#1}}
\overridebibliographypath{../bibliography.bib}
\addbibresource{\bibliographypath}
\defbibheading{bibliography}[\bibname]{%
	\newpage
	\section*{#1}%
}
\renewbibmacro*{entryhead:full}{\printfield{labeltitle}}
\DeclareFieldFormat{url}{\newline\footnotesize\url{#1}}
\AtEndDocument{\printbibliography}

% Police :

\setmathfont{Erewhon Math}

% Tikz :

\usetikzlibrary{calc}

% Longueurs :

\setlength{\parindent}{0pt}
\setlength{\headheight}{15pt}
\setlength{\fboxsep}{0pt}
\titlespacing*{\chapter}{0pt}{-20pt}{10pt}
\setlength{\marginparwidth}{1.5cm}
\setstretch{1.1}

% Métadonnées :

\author{agreg.skyost.eu}
\date{\today}

% Titres :

\setcounter{secnumdepth}{3}

\renewcommand{\thechapter}{\Roman{chapter}}
\renewcommand{\thesubsection}{\Roman{subsection}}
\renewcommand{\thesubsubsection}{\arabic{subsubsection}}
\renewcommand{\theparagraph}{\alph{paragraph}}

\titleformat{\chapter}{\huge\bfseries}{\thechapter}{20pt}{\huge\bfseries}
\titleformat*{\section}{\LARGE\bfseries}
\titleformat{\subsection}{\Large\bfseries}{\thesubsection \, - \,}{0pt}{\Large\bfseries}
\titleformat{\subsubsection}{\large\bfseries}{\thesubsubsection. \,}{0pt}{\large\bfseries}
\titleformat{\paragraph}{\bfseries}{\theparagraph. \,}{0pt}{\bfseries}

\setcounter{secnumdepth}{4}

% Table des matières :

\renewcommand{\cftsecleader}{\cftdotfill{\cftdotsep}}
\addtolength{\cftsecnumwidth}{10pt}

% Redéfinition des commandes :

\renewcommand*\thesection{\arabic{section}}
\renewcommand{\ker}{\mathrm{Ker}}

% Nouvelles commandes :

\newcommand{\website}{https://agreg.skyost.eu}

\newcommand{\tr}[1]{\mathstrut ^t #1}
\newcommand{\im}{\mathrm{Im}}
\newcommand{\rang}{\operatorname{rang}}
\newcommand{\trace}{\operatorname{trace}}
\newcommand{\id}{\operatorname{id}}
\newcommand{\stab}{\operatorname{Stab}}

\providecommand{\newpar}{\\[\medskipamount]}

\providecommand{\lesson}[3]{%
	\title{#3}%
	\hypersetup{pdftitle={#3}}%
	\setcounter{section}{\numexpr #2 - 1}%
	\section{#3}%
	\fancyhead[R]{\truncate{0.73\textwidth}{#2 : #3}}%
}

\providecommand{\development}[3]{%
	\title{#3}%
	\hypersetup{pdftitle={#3}}%
	\section*{#3}%
	\fancyhead[R]{\truncate{0.73\textwidth}{#3}}%
}

\providecommand{\summary}[1]{%
	\textit{#1}%
	\medskip%
}

\tikzset{notestyleraw/.append style={inner sep=0pt, rounded corners=0pt, align=center}}

%\newcommand{\booklink}[1]{\website/bibliographie\##1}
\newcommand{\citelink}[2]{\hyperlink{cite.\therefsection @#1}{#2}}
\newcommand{\previousreference}{}
\providecommand{\reference}[2][]{%
	\notblank{#1}{\renewcommand{\previousreference}{#1}}{}%
	\todo[noline]{%
		\protect\vspace{16pt}%
		\protect\par%
		\protect\notblank{#1}{\cite{[\previousreference]}\\}{}%
		\protect\citelink{\previousreference}{p. #2}%
	}%
}

\definecolor{devcolor}{HTML}{00695c}
\newcommand{\dev}[1]{%
	\reversemarginpar%
	\todo[noline]{
		\protect\vspace{16pt}%
		\protect\par%
		\bfseries\color{devcolor}\href{\website/developpements/#1}{DEV}
	}%
	\normalmarginpar%
}

% En-têtes :

\pagestyle{fancy}
\fancyhead[L]{\truncate{0.23\textwidth}{\thepage}}
\fancyfoot[C]{\scriptsize \href{\website}{\texttt{agreg.skyost.eu}}}

% Couleurs :

\definecolor{property}{HTML}{fffde7}
\definecolor{proposition}{HTML}{fff8e1}
\definecolor{lemma}{HTML}{fff3e0}
\definecolor{theorem}{HTML}{fce4f2}
\definecolor{corollary}{HTML}{ffebee}
\definecolor{definition}{HTML}{ede7f6}
\definecolor{notation}{HTML}{f3e5f5}
\definecolor{example}{HTML}{e0f7fa}
\definecolor{cexample}{HTML}{efebe9}
\definecolor{application}{HTML}{e0f2f1}
\definecolor{remark}{HTML}{e8f5e9}
\definecolor{proof}{HTML}{e1f5fe}

% Théorèmes :

\theoremstyle{definition}
\newtheorem{theorem}{Théorème}

\newtheorem{property}[theorem]{Propriété}
\newtheorem{proposition}[theorem]{Proposition}
\newtheorem{lemma}[theorem]{Lemme}
\newtheorem{corollary}[theorem]{Corollaire}

\newtheorem{definition}[theorem]{Définition}
\newtheorem{notation}[theorem]{Notation}

\newtheorem{example}[theorem]{Exemple}
\newtheorem{cexample}[theorem]{Contre-exemple}
\newtheorem{application}[theorem]{Application}

\theoremstyle{remark}
\newtheorem{remark}[theorem]{Remarque}

\counterwithin*{theorem}{section}

\newcommand{\applystyletotheorem}[1]{
	\tcolorboxenvironment{#1}{
		enhanced,
		breakable,
		colback=#1!98!white,
		boxrule=0pt,
		boxsep=0pt,
		left=8pt,
		right=8pt,
		top=8pt,
		bottom=8pt,
		sharp corners,
		after=\par,
	}
}

\applystyletotheorem{property}
\applystyletotheorem{proposition}
\applystyletotheorem{lemma}
\applystyletotheorem{theorem}
\applystyletotheorem{corollary}
\applystyletotheorem{definition}
\applystyletotheorem{notation}
\applystyletotheorem{example}
\applystyletotheorem{cexample}
\applystyletotheorem{application}
\applystyletotheorem{remark}
\applystyletotheorem{proof}

% Environnements :

\NewEnviron{whitetabularx}[1]{%
	\renewcommand{\arraystretch}{2.5}
	\colorbox{white}{%
		\begin{tabularx}{\textwidth}{#1}%
			\BODY%
		\end{tabularx}%
	}%
}

% Maths :

\DeclareFontEncoding{FMS}{}{}
\DeclareFontSubstitution{FMS}{futm}{m}{n}
\DeclareFontEncoding{FMX}{}{}
\DeclareFontSubstitution{FMX}{futm}{m}{n}
\DeclareSymbolFont{fouriersymbols}{FMS}{futm}{m}{n}
\DeclareSymbolFont{fourierlargesymbols}{FMX}{futm}{m}{n}
\DeclareMathDelimiter{\VERT}{\mathord}{fouriersymbols}{152}{fourierlargesymbols}{147}


% Bibliographie :

\addbibresource{\bibliographypath}%
\defbibheading{bibliography}[\bibname]{%
	\newpage
	\section*{#1}%
}
\renewbibmacro*{entryhead:full}{\printfield{labeltitle}}%
\DeclareFieldFormat{url}{\newline\footnotesize\url{#1}}%

\AtEndDocument{\printbibliography}

\begin{document}
  %<*content>
  \development{algebra, analysis}{equation-de-sylvester}{Équation de Sylvester}
  
  \summary{On montre que l'équation $AX+XB=C$ d'inconnue $X$ admet une unique solution pour tout $C \in \mathcal{M}_n(\mathbb{C})$ et pour tout $A, B \in \mathcal{M}_n(\mathbb{C})$ dont les valeurs propres sont de partie réelle strictement négative.}
  
  \reference[GOU21]{200}
  
  \begin{lemma}
    \label{equation-de-sylvester-1}
    Soit $\Vert . \Vert$ une norme d'algèbre sur $\mathcal{M}_n(\mathbb{C})$, et soit $A \in \mathcal{M}_n(\mathbb{C})$ une matrice dont les valeurs propres sont de partie réelle strictement négative. Alors il existe une fonction polynômiale $P : \mathbb{R} \rightarrow \mathbb{R}$ et $\lambda > 0$ tels que $\Vert e^{tA} \Vert \leq e^{- \lambda t} P(t)$.
  \end{lemma}
  
  \begin{proof}
    On fait la décomposition de Dunford de $A$ : $A = D+N$. Comme $D$ et $N$ commutent, on a $e^{tA} = e^{tD} e^{tN}$. Soient $P$ la matrice de passage donnée par la base de diagonalisation de $D$ et $\lambda_1, \dots, \lambda_n$ ses valeurs propres. En notant $\VERT . \VERT$ la norme subordonnée à $\Vert . \Vert_\infty$ sur $\mathbb{C}^n$, on a $\forall t \geq 0$,
    \begin{align*}
      \left \VERT e^{tD} \right \VERT &= \left \VERT e^{tP \operatorname{Diag}(\lambda_1, \dots, \lambda_n) P^{-1}} \right \VERT \\
      & = \left \VERT P e^{t \operatorname{Diag}(\lambda_1, \dots, \lambda_n)} P^{-1} \right \VERT \\
      & \leq \underbrace{\VERT P \VERT \left \VERT P^{-1} \right \VERT}_{= \alpha} \sup_{\Vert x \Vert_\infty = 1} \left \Vert \operatorname{Diag}(e^{t \lambda_1}, \dots, e^{t \lambda_n}) x \right \Vert_{\infty} \\
      & \leq \alpha \sup_{i \in \llbracket 1, n \rrbracket} e^{t\lambda_i} \\
      & \leq \alpha e^{-\lambda t}
    \end{align*}
    où $\lambda > 0$ par hypothèse. En dimension finie, toutes les normes sont équivalentes, donc il existe $\beta > 0$ tel que $\left \VERT e^{tD} \right \VERT \leq \beta e^{- \lambda t}$.
    \newpar
    Pour conclure, en notant $r$ l'indice de nilpotence de $N$,
    \begin{align*}
      \Vert e^{tA} \Vert & \leq \Vert e^{tD} \Vert \Vert e^{tN} \Vert \\
      & \leq e^{- \lambda t} \underbrace{\sum_{k=0}^{r-1} \beta \frac{\Vert N \Vert^k t^k}{k}}_{= P(t)}
    \end{align*}
  \end{proof}
  
  \reference[I-P]{177}
  
  \begin{theorem}[Équation de Sylvester]
    Soient $A$ et $B \in \mathcal{M}_n(\mathbb{C})$ deux matrices dont les valeurs propres sont de partie réelle strictement négative. Alors pour tout $C \in \mathcal{M}_n(\mathbb{C})$, l'équation $AX + XB = C$ admet une unique solution $X$ dans $\mathcal{M}_n(\mathbb{C})$.
  \end{theorem}
  
  \begin{proof}
    Comme l'application $\varphi : X \mapsto AX + XB$ est un endomorphisme de $\mathcal{M}_n(\mathbb{C})$, qui est un espace vectoriel de dimension finie, il suffit de montrer qu'elle est surjective pour obtenir l'injectivité (et donc l'unicité de la solution). Soit $C \in \mathcal{M}_n(\mathbb{C})$. On considère le problème de Cauchy suivant d'inconnue $Y : \mathbb{R} \rightarrow \mathcal{M}_n(\mathbb{C})$ :
    \[ \begin{cases} Y' = AY + YB \\ Y(0) = C \end{cases} \tag{$E$} \]
    Il s'agit d'une équation différentielle linéaire à coefficients constants (on peut voir cela notamment en calculant les produits $AY$ et $YB$ et en effectuant la somme ; l'égalité matricielle avec $C$ donnant le système d'équations voulu). D'après le théorème de Cauchy-Lipschitz linéaire, $(E)$ admet une unique solution définie sur $\mathbb{R}$ tout entier, que l'on note $Y$.
    \newpar
    On vérifie que la solution est définie $\forall t \in \mathbb{R}$ par $Y(t) = \exp(tA) C \exp(tB)$. En effet pour tout $t \in \mathbb{R}$, on a :
    \[ Y'(t) = A \exp(tA) C \exp(tB) + \exp(tA) CB \exp(tB) = AY + YB \]
    car toute matrice $M$ commute avec son exponentielle (puisque $\exp(M)$ est limite d'un polynôme en $M$) et donc $M$ commute aussi avec $\exp(tM)$ pour tout $t \in \mathbb{R}$.
    \newpar
    On va maintenant montrer que $X = - \int_{0}^{+\infty} Y(s) \, \mathrm{d}s$ est la solution de l'équation de Sylvester. Pour tout $t \geq 0$, on intègre $Y'$ entre $0$ et $t$ pour obtenir :
    \[ Y(t) - C = \int_0^t Y'(s) \, \mathrm{d}s = A \times \int_0^t Y(s) \, \mathrm{d}s + \int_0^t Y(s) \, \mathrm{d}s \times B \]
    Il ne reste donc plus qu'à montrer que $Y(t) \longrightarrow 0$ et que $Y$ est intégrable pour conclure. Par le \cref{equation-de-sylvester-1}, il existe $\lambda_1, \lambda_2 > 0$ et $P_1, P_2 : \mathbb{R} \rightarrow \mathbb{R}$ polynômiales tels que $\Vert e^{tA} \Vert \leq e^{- \lambda_1 t} P_1(t)$ et $\Vert e^{tB} \Vert \leq e^{-\lambda_2 t} P_2(t)$ pour tout $t \geq 0$. Ainsi, en posant $\lambda = \max(\lambda_1, \lambda_2)$ et $P = P_1 P_2$, comme $\Vert . \Vert$ est une norme d'algèbre :
    \[ \Vert Y(t) \Vert = \Vert e^{tA} C e^{tB} \Vert \leq \Vert C \Vert P(t) e^{-2 \lambda t} \]
    En particulier, on a bien $Y(t) \longrightarrow 0$. De plus, comme $C P(t) e^{-2 \lambda t}$ est intégrable sur $[0, +\infty[$ et domine $\Vert Y(t) \Vert$, alors $Y$ est aussi intégrable $[0, +\infty[$. Finalement, en faisant $t \longrightarrow +\infty$, on obtient :
    \[ -C = A \times \int_{0}^{+\infty} Y(s) \, \mathrm{d}s + \int_{0}^{+\infty} Y(s) \, \mathrm{d}s \times B \]
    Donc $\varphi(X) = X$ : $\varphi$ est surjective et $X$ est bien la solution de l'équation de Sylvester.
  \end{proof}
  
  \reference[GOU21]{189}
  
  \begin{remark}
    Pour dire que toute matrice $M$ commute avec $\exp(M)$, on aurait simplement pu dire que $\exp(M)$ est un polynôme en $M$ ie. $\forall M \in \mathcal{M}_n(\mathbb{C})$, $\exists P \in \mathbb{C}[X]$ tel que $\exp(M) = P(M)$.
  \end{remark}
  
  \begin{proof}
    Soit $M \in \mathcal{M}_n(\mathbb{C})$. L'ensemble $\mathbb{C}[M] = \{ P(M) \mid P \in \mathbb{C}[X] \}$ est un sous-espace vectoriel de $\mathcal{M}_n(\mathbb{C})$ qui est de dimension finie, donc $\mathbb{C}[M]$ l'est aussi et est en particulier fermé.
    \newpar
    Pour tout $n \in \mathbb{N}$, on pose $P_n = \sum_{k=0}^n \frac{M^k}{k!} \in \mathbb{C}[M]$ de sorte que $P_n \longrightarrow_{n \rightarrow +\infty} \exp(M)$. Comme $\mathbb{C}[M]$ est fermé, on en déduit que $\exp(M) \in \mathbb{C}[M]$. Donc $\exists P \in \mathbb{C}[X]$ tel que $\exp(M) = P(M)$.
  \end{proof}
  %</content>
\end{document}