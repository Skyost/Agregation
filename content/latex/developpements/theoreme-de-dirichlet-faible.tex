\documentclass[12pt, a4paper]{report}

% LuaLaTeX :

\RequirePackage{iftex}
\RequireLuaTeX

% Packages :

\usepackage[french]{babel}
%\usepackage[utf8]{inputenc}
%\usepackage[T1]{fontenc}
\usepackage[pdfencoding=auto, pdfauthor={Hugo Delaunay}, pdfsubject={Mathématiques}, pdfcreator={agreg.skyost.eu}]{hyperref}
\usepackage{amsmath}
\usepackage{amsthm}
%\usepackage{amssymb}
\usepackage{stmaryrd}
\usepackage{tikz}
\usepackage{tkz-euclide}
\usepackage{fourier-otf}
\usepackage{fontspec}
\usepackage{titlesec}
\usepackage{fancyhdr}
\usepackage{catchfilebetweentags}
\usepackage[french, capitalise, noabbrev]{cleveref}
\usepackage[fit, breakall]{truncate}
\usepackage[top=2.5cm, right=2cm, bottom=2.5cm, left=2cm]{geometry}
\usepackage{enumerate}
\usepackage{tocloft}
\usepackage{microtype}
%\usepackage{mdframed}
%\usepackage{thmtools}
\usepackage{xcolor}
\usepackage{tabularx}
\usepackage{aligned-overset}
\usepackage[subpreambles=true]{standalone}
\usepackage{environ}
\usepackage[normalem]{ulem}
\usepackage{marginnote}
\usepackage{etoolbox}
\usepackage{setspace}
\usepackage[bibstyle=reading, citestyle=draft]{biblatex}
\usepackage{xpatch}
\usepackage[many, breakable]{tcolorbox}
\usepackage[backgroundcolor=white, bordercolor=white, textsize=small]{todonotes}

% Bibliographie :

\newcommand{\overridebibliographypath}[1]{\providecommand{\bibliographypath}{#1}}
\overridebibliographypath{../bibliography.bib}
\addbibresource{\bibliographypath}
\defbibheading{bibliography}[\bibname]{%
	\newpage
	\section*{#1}%
}
\renewbibmacro*{entryhead:full}{\printfield{labeltitle}}
\DeclareFieldFormat{url}{\newline\footnotesize\url{#1}}
\AtEndDocument{\printbibliography}

% Police :

\setmathfont{Erewhon Math}

% Tikz :

\usetikzlibrary{calc}

% Longueurs :

\setlength{\parindent}{0pt}
\setlength{\headheight}{15pt}
\setlength{\fboxsep}{0pt}
\titlespacing*{\chapter}{0pt}{-20pt}{10pt}
\setlength{\marginparwidth}{1.5cm}
\setstretch{1.1}

% Métadonnées :

\author{agreg.skyost.eu}
\date{\today}

% Titres :

\setcounter{secnumdepth}{3}

\renewcommand{\thechapter}{\Roman{chapter}}
\renewcommand{\thesubsection}{\Roman{subsection}}
\renewcommand{\thesubsubsection}{\arabic{subsubsection}}
\renewcommand{\theparagraph}{\alph{paragraph}}

\titleformat{\chapter}{\huge\bfseries}{\thechapter}{20pt}{\huge\bfseries}
\titleformat*{\section}{\LARGE\bfseries}
\titleformat{\subsection}{\Large\bfseries}{\thesubsection \, - \,}{0pt}{\Large\bfseries}
\titleformat{\subsubsection}{\large\bfseries}{\thesubsubsection. \,}{0pt}{\large\bfseries}
\titleformat{\paragraph}{\bfseries}{\theparagraph. \,}{0pt}{\bfseries}

\setcounter{secnumdepth}{4}

% Table des matières :

\renewcommand{\cftsecleader}{\cftdotfill{\cftdotsep}}
\addtolength{\cftsecnumwidth}{10pt}

% Redéfinition des commandes :

\renewcommand*\thesection{\arabic{section}}
\renewcommand{\ker}{\mathrm{Ker}}

% Nouvelles commandes :

\newcommand{\website}{https://agreg.skyost.eu}

\newcommand{\tr}[1]{\mathstrut ^t #1}
\newcommand{\im}{\mathrm{Im}}
\newcommand{\rang}{\operatorname{rang}}
\newcommand{\trace}{\operatorname{trace}}
\newcommand{\id}{\operatorname{id}}
\newcommand{\stab}{\operatorname{Stab}}

\providecommand{\newpar}{\\[\medskipamount]}

\providecommand{\lesson}[3]{%
	\title{#3}%
	\hypersetup{pdftitle={#3}}%
	\setcounter{section}{\numexpr #2 - 1}%
	\section{#3}%
	\fancyhead[R]{\truncate{0.73\textwidth}{#2 : #3}}%
}

\providecommand{\development}[3]{%
	\title{#3}%
	\hypersetup{pdftitle={#3}}%
	\section*{#3}%
	\fancyhead[R]{\truncate{0.73\textwidth}{#3}}%
}

\providecommand{\summary}[1]{%
	\textit{#1}%
	\medskip%
}

\tikzset{notestyleraw/.append style={inner sep=0pt, rounded corners=0pt, align=center}}

%\newcommand{\booklink}[1]{\website/bibliographie\##1}
\newcommand{\citelink}[2]{\hyperlink{cite.\therefsection @#1}{#2}}
\newcommand{\previousreference}{}
\providecommand{\reference}[2][]{%
	\notblank{#1}{\renewcommand{\previousreference}{#1}}{}%
	\todo[noline]{%
		\protect\vspace{16pt}%
		\protect\par%
		\protect\notblank{#1}{\cite{[\previousreference]}\\}{}%
		\protect\citelink{\previousreference}{p. #2}%
	}%
}

\definecolor{devcolor}{HTML}{00695c}
\newcommand{\dev}[1]{%
	\reversemarginpar%
	\todo[noline]{
		\protect\vspace{16pt}%
		\protect\par%
		\bfseries\color{devcolor}\href{\website/developpements/#1}{DEV}
	}%
	\normalmarginpar%
}

% En-têtes :

\pagestyle{fancy}
\fancyhead[L]{\truncate{0.23\textwidth}{\thepage}}
\fancyfoot[C]{\scriptsize \href{\website}{\texttt{agreg.skyost.eu}}}

% Couleurs :

\definecolor{property}{HTML}{fffde7}
\definecolor{proposition}{HTML}{fff8e1}
\definecolor{lemma}{HTML}{fff3e0}
\definecolor{theorem}{HTML}{fce4f2}
\definecolor{corollary}{HTML}{ffebee}
\definecolor{definition}{HTML}{ede7f6}
\definecolor{notation}{HTML}{f3e5f5}
\definecolor{example}{HTML}{e0f7fa}
\definecolor{cexample}{HTML}{efebe9}
\definecolor{application}{HTML}{e0f2f1}
\definecolor{remark}{HTML}{e8f5e9}
\definecolor{proof}{HTML}{e1f5fe}

% Théorèmes :

\theoremstyle{definition}
\newtheorem{theorem}{Théorème}

\newtheorem{property}[theorem]{Propriété}
\newtheorem{proposition}[theorem]{Proposition}
\newtheorem{lemma}[theorem]{Lemme}
\newtheorem{corollary}[theorem]{Corollaire}

\newtheorem{definition}[theorem]{Définition}
\newtheorem{notation}[theorem]{Notation}

\newtheorem{example}[theorem]{Exemple}
\newtheorem{cexample}[theorem]{Contre-exemple}
\newtheorem{application}[theorem]{Application}

\theoremstyle{remark}
\newtheorem{remark}[theorem]{Remarque}

\counterwithin*{theorem}{section}

\newcommand{\applystyletotheorem}[1]{
	\tcolorboxenvironment{#1}{
		enhanced,
		breakable,
		colback=#1!98!white,
		boxrule=0pt,
		boxsep=0pt,
		left=8pt,
		right=8pt,
		top=8pt,
		bottom=8pt,
		sharp corners,
		after=\par,
	}
}

\applystyletotheorem{property}
\applystyletotheorem{proposition}
\applystyletotheorem{lemma}
\applystyletotheorem{theorem}
\applystyletotheorem{corollary}
\applystyletotheorem{definition}
\applystyletotheorem{notation}
\applystyletotheorem{example}
\applystyletotheorem{cexample}
\applystyletotheorem{application}
\applystyletotheorem{remark}
\applystyletotheorem{proof}

% Environnements :

\NewEnviron{whitetabularx}[1]{%
	\renewcommand{\arraystretch}{2.5}
	\colorbox{white}{%
		\begin{tabularx}{\textwidth}{#1}%
			\BODY%
		\end{tabularx}%
	}%
}

% Maths :

\DeclareFontEncoding{FMS}{}{}
\DeclareFontSubstitution{FMS}{futm}{m}{n}
\DeclareFontEncoding{FMX}{}{}
\DeclareFontSubstitution{FMX}{futm}{m}{n}
\DeclareSymbolFont{fouriersymbols}{FMS}{futm}{m}{n}
\DeclareSymbolFont{fourierlargesymbols}{FMX}{futm}{m}{n}
\DeclareMathDelimiter{\VERT}{\mathord}{fouriersymbols}{152}{fourierlargesymbols}{147}


% Bibliographie :

\addbibresource{\bibliographypath}%
\defbibheading{bibliography}[\bibname]{%
	\newpage
	\section*{#1}%
}
\renewbibmacro*{entryhead:full}{\printfield{labeltitle}}%
\DeclareFieldFormat{url}{\newline\footnotesize\url{#1}}%

\AtEndDocument{\printbibliography}

\begin{document}
  %<*content>
  \development{algebra}{theoreme-de-dirichlet-faible}{Théorème de Dirichlet faible}

  \summary{En raisonnant par l'absurde et en utilisant certaines propriétés des polynômes cyclotomiques, on démontre que l'ensemble des premiers congrus à $1$ modulo un certain entier $n$ est infini.}

  \reference[GOU21]{99}

  \begin{lemma}
    \label{theoreme-de-dirichlet-faible-1}
    Soient $a \in \mathbb{N}$ et $p$ premier tels que $p \mid \Phi_n(a)$ mais $p \nmid \Phi_d(a)$ pour tout diviseur strict $d$ de $n$. Alors $p \equiv 1 \mod n$.
  \end{lemma}

  \begin{proof}
    On a,
    \[ X^n - 1 = \prod_{d \mid n} \Phi_d = \Phi_n \underbrace{\prod_{d \mid \mid n} \Phi_d}_{= F} \]
    Comme $F \in \mathbb{Z}[X]$, en évaluant en $a$ :
    \[ a^n - 1 = \Phi_n(a) F(a) \implies p \mid a^n - 1 \text{ car } F(a) \in \mathbb{Z} \]
    Autrement dit, $a^n \equiv 1 \mod p$. En notant $m$ l'ordre de $\overline{a}$ dans $(\mathbb{Z}/p\mathbb{Z})^*$, on a $a^m \equiv 1 \mod p$. D'où $m \mid n$. Ainsi :
    \begin{itemize}
      \item Si $m = n$, alors $\overline{a}$ est d'ordre $n$. Donc par le théorème de Lagrange, $n \mid |(\mathbb{Z}/p\mathbb{Z})^*| = p-1$ ie. $p \equiv 1 \mod n$.
      \item Sinon, $m < n$. Comme $m \mid n$,
      \[ X^n-1 = \prod_{d \mid n} \Phi_d = \Phi_n \left ( \prod_{d \mid m} \Phi_d \right ) \left ( \prod_{\substack{d \mid \mid n \\ d \nmid m}} \Phi_d \right ) = \Phi_n (X^m - 1) \left ( \prod_{\substack{d \mid \mid n \\ d \nmid m}} \Phi_d \right ) \]
      Mais, $\overline{a}$ est racine de $\overline{\Phi_n}$ et $X^m - \overline{1} \in \mathbb{Z}/p\mathbb{Z}[X]$. En particulier, $\overline{a}$ est (au moins) racine double de $X^n - \overline{a}$. On peut donc écrire,
      \[ X^n - 1 \equiv (X-a)^2 G(X) \mod p \]
      Avec $X = Y+a$, cela donne :
      \[ (Y+a)^n - 1 \equiv Y^2 G(Y+a) \mod p \]
      Le polynôme de droite est de degré $\geq 2$, donc $p$ divise les coefficients des termes de degré $0$ et $1$ de $(Y+a)^n-1$, ie.
      \[ p \mid a^n - 1 \text{ et } p \mid \binom{n}{1} a^{n-1} = n a^{n-1} \]
      De la première égalité, on en tire $p \nmid a$. Ainsi, $a$ est premier avec $p$ (c'est donc également vrai pour $a^{n-1}$). Finalement, on tire de la deuxième égalité que $p \mid n$.
    \end{itemize}
  \end{proof}

  \begin{theorem}[Dirichlet faible]
    Pour tout entier $n$, il existe une infinité de nombres premiers congrus à $1$ modulo $n$.
  \end{theorem}

  \begin{proof}
    On suppose par l'absurde qu'il n'existe qu'un nombre fini de premiers de la forme $1+kn$, que l'on note $p_1, \dots, p_m$. On considère $N = \Phi_n(\alpha)$ où $\alpha = n p_1 \dots p_m$. On remarque en particulier que $N \equiv a_0 \mod \alpha$, où $a_0$ est le coefficient constant de $\Phi_n$ (cela se voit en écrivant $\Phi_n = \sum_{k=0}^{\varphi(n)} a_k X^k$, ce qui donne $N = a_0 + \alpha \sum_{k=1}^{\varphi(n)} a_k \alpha^{k-1}$ une fois évalué en $\alpha$).
    \newpar
    Or, $X^n - 1 = \prod_{d \mid n} \Phi_d$. En évaluant en $0$, on en tire :
    \[ -1 = \prod_{d \mid n} \Phi_d(0) \implies \pm 1 = a_0 \text{, car } \forall d \mid n, \, \Phi_d \in \mathbb{Z}[X] \]
    Ainsi, $N \equiv \pm 1 \mod \alpha$. Or $|N| = |\Phi_n(\alpha)| = \prod_{\xi \in \pi_n^*} |\alpha - \xi| \geq 2$. On peut en effet interpréter $|\alpha - \xi|$ comme la distance du complexe $\alpha$ au complexe $\xi$ ; le premier est sur l'axe réel et est $\geq 2$, le second est sur le cercle unité :
    \begin{center}
      \begin{tikzpicture}
        \draw[->] (-3, 0) -- (5, 0) node[right] {$x$};
        \draw[->] (0, -3) -- (0, 3.5) node[above] {$y$};
        \draw[orange, thick] (2, 0) -- (4, 0) node[below, shift={(-1,0)}]{$|1-\alpha|$};
        \draw[teal, thick] (45:2) -- (4, 0) node[above, shift={(-1.2,0.6)}, rotate=-27.5]{$|\xi-\alpha|$};
        \draw(0,2) node {$\bullet$};
        \draw(0,2) node[above right]{$i$};
        \draw(2,0) node {$\bullet$};
        \draw(2,0) node[below right]{$1$};
        \draw(0,0) circle (2);
        \draw(4,0) node {$\bullet$} node[below]{$\alpha$};
        \draw(45:2) node {$\bullet$} node[above right]{$\xi$};
      \end{tikzpicture}
    \end{center}
    En particulier, $\exists p$ premier tel que $p \mid N$. Par le \cref{theoreme-de-dirichlet-faible-1} :
    \begin{itemize}
      \item Ou bien $p \mid n$, dans ce cas $p \mid \alpha = n p_1 \dots p_m$.
      \item Ou bien $p \equiv 1 \mod n$, dans ce cas $p = p_k$ pour un certain $k \in \llbracket 1, m \rrbracket$. Et on a encore $p \mid \alpha$.
    \end{itemize}
    Pour conclure, on écrit $N = \alpha q \pm 1$ (par division euclidienne), et on a $p \mid N - \alpha q = \pm 1$ : absurde.
  \end{proof}

  \begin{remark}
    Si vous choisissez de présenter ce développement, il faut au moins connaître l'énoncé de la version forte du théorème.
  \end{remark}

  \begin{theorem}[Progression arithmétique de Dirichlet]
    Pour tout entier $n$ et pour tout $m$ premier avec $n$, il existe une infinité de nombres premiers congrus à $m$ modulo $n$.
  \end{theorem}
  %</content>
\end{document}
