\documentclass[12pt, a4paper]{report}

% LuaLaTeX :

\RequirePackage{iftex}
\RequireLuaTeX

% Packages :

\usepackage[french]{babel}
%\usepackage[utf8]{inputenc}
%\usepackage[T1]{fontenc}
\usepackage[pdfencoding=auto, pdfauthor={Hugo Delaunay}, pdfsubject={Mathématiques}, pdfcreator={agreg.skyost.eu}]{hyperref}
\usepackage{amsmath}
\usepackage{amsthm}
%\usepackage{amssymb}
\usepackage{stmaryrd}
\usepackage{tikz}
\usepackage{tkz-euclide}
\usepackage{fourier-otf}
\usepackage{fontspec}
\usepackage{titlesec}
\usepackage{fancyhdr}
\usepackage{catchfilebetweentags}
\usepackage[french, capitalise, noabbrev]{cleveref}
\usepackage[fit, breakall]{truncate}
\usepackage[top=2.5cm, right=2cm, bottom=2.5cm, left=2cm]{geometry}
\usepackage{enumerate}
\usepackage{tocloft}
\usepackage{microtype}
%\usepackage{mdframed}
%\usepackage{thmtools}
\usepackage{xcolor}
\usepackage{tabularx}
\usepackage{aligned-overset}
\usepackage[subpreambles=true]{standalone}
\usepackage{environ}
\usepackage[normalem]{ulem}
\usepackage{marginnote}
\usepackage{etoolbox}
\usepackage{setspace}
\usepackage[bibstyle=reading, citestyle=draft]{biblatex}
\usepackage{xpatch}
\usepackage[many, breakable]{tcolorbox}
\usepackage[backgroundcolor=white, bordercolor=white, textsize=small]{todonotes}

% Bibliographie :

\newcommand{\overridebibliographypath}[1]{\providecommand{\bibliographypath}{#1}}
\overridebibliographypath{../bibliography.bib}
\addbibresource{\bibliographypath}
\defbibheading{bibliography}[\bibname]{%
	\newpage
	\section*{#1}%
}
\renewbibmacro*{entryhead:full}{\printfield{labeltitle}}
\DeclareFieldFormat{url}{\newline\footnotesize\url{#1}}
\AtEndDocument{\printbibliography}

% Police :

\setmathfont{Erewhon Math}

% Tikz :

\usetikzlibrary{calc}

% Longueurs :

\setlength{\parindent}{0pt}
\setlength{\headheight}{15pt}
\setlength{\fboxsep}{0pt}
\titlespacing*{\chapter}{0pt}{-20pt}{10pt}
\setlength{\marginparwidth}{1.5cm}
\setstretch{1.1}

% Métadonnées :

\author{agreg.skyost.eu}
\date{\today}

% Titres :

\setcounter{secnumdepth}{3}

\renewcommand{\thechapter}{\Roman{chapter}}
\renewcommand{\thesubsection}{\Roman{subsection}}
\renewcommand{\thesubsubsection}{\arabic{subsubsection}}
\renewcommand{\theparagraph}{\alph{paragraph}}

\titleformat{\chapter}{\huge\bfseries}{\thechapter}{20pt}{\huge\bfseries}
\titleformat*{\section}{\LARGE\bfseries}
\titleformat{\subsection}{\Large\bfseries}{\thesubsection \, - \,}{0pt}{\Large\bfseries}
\titleformat{\subsubsection}{\large\bfseries}{\thesubsubsection. \,}{0pt}{\large\bfseries}
\titleformat{\paragraph}{\bfseries}{\theparagraph. \,}{0pt}{\bfseries}

\setcounter{secnumdepth}{4}

% Table des matières :

\renewcommand{\cftsecleader}{\cftdotfill{\cftdotsep}}
\addtolength{\cftsecnumwidth}{10pt}

% Redéfinition des commandes :

\renewcommand*\thesection{\arabic{section}}
\renewcommand{\ker}{\mathrm{Ker}}

% Nouvelles commandes :

\newcommand{\website}{https://agreg.skyost.eu}

\newcommand{\tr}[1]{\mathstrut ^t #1}
\newcommand{\im}{\mathrm{Im}}
\newcommand{\rang}{\operatorname{rang}}
\newcommand{\trace}{\operatorname{trace}}
\newcommand{\id}{\operatorname{id}}
\newcommand{\stab}{\operatorname{Stab}}

\providecommand{\newpar}{\\[\medskipamount]}

\providecommand{\lesson}[3]{%
	\title{#3}%
	\hypersetup{pdftitle={#3}}%
	\setcounter{section}{\numexpr #2 - 1}%
	\section{#3}%
	\fancyhead[R]{\truncate{0.73\textwidth}{#2 : #3}}%
}

\providecommand{\development}[3]{%
	\title{#3}%
	\hypersetup{pdftitle={#3}}%
	\section*{#3}%
	\fancyhead[R]{\truncate{0.73\textwidth}{#3}}%
}

\providecommand{\summary}[1]{%
	\textit{#1}%
	\medskip%
}

\tikzset{notestyleraw/.append style={inner sep=0pt, rounded corners=0pt, align=center}}

%\newcommand{\booklink}[1]{\website/bibliographie\##1}
\newcommand{\citelink}[2]{\hyperlink{cite.\therefsection @#1}{#2}}
\newcommand{\previousreference}{}
\providecommand{\reference}[2][]{%
	\notblank{#1}{\renewcommand{\previousreference}{#1}}{}%
	\todo[noline]{%
		\protect\vspace{16pt}%
		\protect\par%
		\protect\notblank{#1}{\cite{[\previousreference]}\\}{}%
		\protect\citelink{\previousreference}{p. #2}%
	}%
}

\definecolor{devcolor}{HTML}{00695c}
\newcommand{\dev}[1]{%
	\reversemarginpar%
	\todo[noline]{
		\protect\vspace{16pt}%
		\protect\par%
		\bfseries\color{devcolor}\href{\website/developpements/#1}{DEV}
	}%
	\normalmarginpar%
}

% En-têtes :

\pagestyle{fancy}
\fancyhead[L]{\truncate{0.23\textwidth}{\thepage}}
\fancyfoot[C]{\scriptsize \href{\website}{\texttt{agreg.skyost.eu}}}

% Couleurs :

\definecolor{property}{HTML}{fffde7}
\definecolor{proposition}{HTML}{fff8e1}
\definecolor{lemma}{HTML}{fff3e0}
\definecolor{theorem}{HTML}{fce4f2}
\definecolor{corollary}{HTML}{ffebee}
\definecolor{definition}{HTML}{ede7f6}
\definecolor{notation}{HTML}{f3e5f5}
\definecolor{example}{HTML}{e0f7fa}
\definecolor{cexample}{HTML}{efebe9}
\definecolor{application}{HTML}{e0f2f1}
\definecolor{remark}{HTML}{e8f5e9}
\definecolor{proof}{HTML}{e1f5fe}

% Théorèmes :

\theoremstyle{definition}
\newtheorem{theorem}{Théorème}

\newtheorem{property}[theorem]{Propriété}
\newtheorem{proposition}[theorem]{Proposition}
\newtheorem{lemma}[theorem]{Lemme}
\newtheorem{corollary}[theorem]{Corollaire}

\newtheorem{definition}[theorem]{Définition}
\newtheorem{notation}[theorem]{Notation}

\newtheorem{example}[theorem]{Exemple}
\newtheorem{cexample}[theorem]{Contre-exemple}
\newtheorem{application}[theorem]{Application}

\theoremstyle{remark}
\newtheorem{remark}[theorem]{Remarque}

\counterwithin*{theorem}{section}

\newcommand{\applystyletotheorem}[1]{
	\tcolorboxenvironment{#1}{
		enhanced,
		breakable,
		colback=#1!98!white,
		boxrule=0pt,
		boxsep=0pt,
		left=8pt,
		right=8pt,
		top=8pt,
		bottom=8pt,
		sharp corners,
		after=\par,
	}
}

\applystyletotheorem{property}
\applystyletotheorem{proposition}
\applystyletotheorem{lemma}
\applystyletotheorem{theorem}
\applystyletotheorem{corollary}
\applystyletotheorem{definition}
\applystyletotheorem{notation}
\applystyletotheorem{example}
\applystyletotheorem{cexample}
\applystyletotheorem{application}
\applystyletotheorem{remark}
\applystyletotheorem{proof}

% Environnements :

\NewEnviron{whitetabularx}[1]{%
	\renewcommand{\arraystretch}{2.5}
	\colorbox{white}{%
		\begin{tabularx}{\textwidth}{#1}%
			\BODY%
		\end{tabularx}%
	}%
}

% Maths :

\DeclareFontEncoding{FMS}{}{}
\DeclareFontSubstitution{FMS}{futm}{m}{n}
\DeclareFontEncoding{FMX}{}{}
\DeclareFontSubstitution{FMX}{futm}{m}{n}
\DeclareSymbolFont{fouriersymbols}{FMS}{futm}{m}{n}
\DeclareSymbolFont{fourierlargesymbols}{FMX}{futm}{m}{n}
\DeclareMathDelimiter{\VERT}{\mathord}{fouriersymbols}{152}{fourierlargesymbols}{147}


% Bibliographie :

\addbibresource{\bibliographypath}%
\defbibheading{bibliography}[\bibname]{%
	\newpage
	\section*{#1}%
}
\renewbibmacro*{entryhead:full}{\printfield{labeltitle}}%
\DeclareFieldFormat{url}{\newline\footnotesize\url{#1}}%

\AtEndDocument{\printbibliography}

\begin{document}
	%<*content>
	\development{analysis}{theoreme-central-limite}{Théorème central limite}

	\summary{En établissant d'abord le théorème de Lévy, on démontre le théorème central limite, qui dit que si $(X_n)$ est une suite de variables aléatoires identiquement distribuées admettant un moment d'ordre $2$, alors $\frac{X_1 + \dots + X_n - n \mathbb{E}(X_1)}{\sqrt{n}}$ converge en loi vers $\mathcal{N}(0, \operatorname{Var}(X_1))$.}

	\begin{notation}
		Si $X$ est une variable aléatoire réelle, on note $\phi_X$ sa fonction caractéristique.
	\end{notation}

	\reference[Z-Q]{536}

	\begin{theorem}[Lévy]
		\label{theoreme-central-limite-1}
		Soient $(X_n)$ une suite de variables aléatoires réelles et $X$ une variable aléatoire réelle. Alors :
		\[ X_n \overset{(d)}{\longrightarrow} X \iff \phi_{X_n} \text{ converge simplement vers } \phi_X \]
	\end{theorem}

	\begin{proof}
		\uline{Sens direct :} On suppose que $(X_n)$ converge en loi vers $X$. Pour tout $t \in \mathbb{R}$, la fonction $g_t : x \mapsto e^{itx}$ est continue et bornée sur $\mathbb{R}$. Donc par définition de la convergence en loi :
		\[ \lim_{n \rightarrow +\infty} \mathbb{E}(g_t(X_n)) = \mathbb{E}(g_t(X)) \]
		ce que l'on voulait.
		\newpar
		\uline{Réciproque :} Soit $\varphi \in L_1(\mathbb{R})$, on pose $f = \widehat{\varphi}$. Alors
		\[ \mathbb{E}(f(X_n)) = \mathbb{E} \left ( \int_{\mathbb{R}} e^{itX_n} \varphi(t) \, \mathrm{d}t \right ) \]
		Comme la fonction $(\omega, t) \mapsto e^{itX_n(\omega)} \varphi(t)$ est intégrable pour la mesure $\mathbb{P}_{X_n} \otimes \lambda$, on peut appliquer le théorème de Fubini-Lebesgue pour intervertir espérance et intégrale :
		\begin{align*}
			\mathbb{E}(f(X_n)) = \int_{\mathbb{R}} \mathbb{E} (e^{itX_n}) \varphi(t) \, \mathrm{d}t
		\end{align*}
		On définit maintenant la suite de fonction $g_n : t \mapsto \mathbb{E} (e^{itX_n}) \varphi(t)$. Alors :
		\begin{itemize}
			\item $\forall n \in \mathbb{N}$, $g_n$ est mesurable.
			\item La suite de fonction $(g_n)$ converge presque partout vers $g : t \mapsto \mathbb{E} (e^{itX}) \varphi(t)$ par hypothèse.
			\item $\forall n \in \mathbb{N}$ et p.p. en $t \in \mathbb{R}$, $|g_n(t)| \leq \mathbb{E} (|e^{itX_n}|) |\varphi(t)| \leq \mathbb{P}_X(\mathbb{R}) |\varphi(t)|$ avec $|\varphi| \in L_1(\mathbb{R})$.
		\end{itemize}
		On peut donc appliquer le théorème de convergence dominée pour obtenir
		\[ \mathbb{E}(f(X_n)) \longrightarrow \int_{\mathbb{R}} \mathbb{E} (e^{itX}) \varphi(t) \, \mathrm{d}t = \mathbb{E}(f(X)) \]
		Ainsi, le résultat est vrai pour toute fonction dans l'image de $L_1(\mathbb{R})$ par la transformée de Fourier. En particulier, il est vrai pour tout $f \in \mathcal{S}(\mathbb{R})$, dense dans $(\mathcal{C}(\mathbb{R}), \Vert . \Vert_\infty)$. Soient maintenant $f \in \mathcal{C}(\mathbb{R})$ et $(f_k)$ une suite de fonctions de $\mathcal{S}(\mathbb{R})$ qui converge uniformément vers $f$. Alors,
		\begin{align*}
			|\mathbb{E}(f(X_n)) - \mathbb{E}(f(X))| &= |\mathbb{E}(f(X_n)) - \mathbb{E}(f_k(X_n)) + \mathbb{E}(f_k(X_n)) \\
			&- \mathbb{E}(f_k(X)) + \mathbb{E}(f_k(X)) - \mathbb{E}(f(X))| \\
			&\leq 2 \Vert f - f_k \Vert_\infty + |\mathbb{E}(f_k(X_n)) - \mathbb{E}(f_k(X))| \\
			&\longrightarrow 0
		\end{align*}
	\end{proof}

	\reference[G-K]{307}

	\begin{lemma}
		\label{theoreme-central-limite-2}
		Soient $u, v \in \mathbb{C}$ de module inférieur ou égal à $1$ et $n \in \mathbb{N}^*$. Alors
		\[ |z^n - u^n| \leq n |z-u| \]
	\end{lemma}

	\begin{proof}
		$|z^n - u^n| = |(z-u) \sum_{k=0}^{n-1} z^k u^{n-1-k}| \leq n |z-u|$.
	\end{proof}

	\begin{theorem}[Central limite]
		Soit $(X_n)$ une suite de variables aléatoires réelles indépendantes de même loi admettant un moment d'ordre $2$. On note $m$ l'espérance et $\sigma^2$ la variance commune à ces variables. On pose $S_n = X_1 + \dots + X_n - nm$. Alors,
		\[ \left ( \frac{S_n}{\sqrt{n}} \right) \overset{(d)}{\longrightarrow} \mathcal{N}(0, \sigma^2) \]
	\end{theorem}

	\begin{proof}
		On a $S_n = \sum_{k=1}^n (X_k - m)$. Notons $\phi$ la fonction caractéristique de $X_1 - m$. Comme les variables aléatoires $X_1 - m, \dots, X_n - m$ sont indépendantes de même loi, la fonction caractéristique de $\frac{S_n}{\sqrt{n}}$ vaut $\forall t \in \mathbb{R}$,
		\begin{align*}
			\phi_{\frac{S_n}{\sqrt{n}}}(t) &= \mathbb{E} \left( e^{iS_n \left( \frac{t}{\sqrt{n}} \right)} \right) \\
			&= \mathbb{E} \left( \prod_{k=1}^n e^{i(X_k -m) \left( \frac{t}{\sqrt{n}} \right)} \right) \\
			&= \prod_{k=1}^n \phi_{X_k - m} \left ( \frac{t}{\sqrt{n}} \right) \\
			&= \phi \left ( \frac{t}{\sqrt{n}} \right)^n
		\end{align*}
		D'après le \cref{theoreme-central-limite-1}, pour montrer que $\frac{S_n}{\sqrt{n}}$ converge en loi vers $\mathcal{N}(0, \sigma^2)$, il suffit de montrer que
		\[ \forall t \in \mathbb{R}, \, \lim_{n \rightarrow +\infty} \phi \left ( \frac{t}{\sqrt{n}} \right)^n = e^{-\frac{\sigma^2}{2} t^2} \]
		car $t \mapsto e^{-\frac{\sigma^2}{2} t^2}$ est la fonction caractéristique de la loi $\mathcal{N}(0, \sigma^2)$.
		\newpar
		Comme $X_1$ admet un moment d'ordre $2$, $\phi$ est de classe $\mathcal{C}^2$ et
		\begin{itemize}
			\item $\phi(0) = 1$.
			\item $\phi'(0) = i^1 \mathbb{E}(X_1^1) = 0$.
			\item $\phi''(0) = i^2 \mathbb{E}(X_1^2) = - E(X^2) = -\sigma^2$ (car $m = 0$).
		\end{itemize}
		Ce qui donne le développement limité en $0$ de $\phi$ à l'ordre $2$ (par la formule de Taylor-Young) :
		\[ \phi(t) = \phi(0) + \frac{\phi'(0)}{1!} (t-0) + \frac{\phi''(0)}{2!} (t-0)^2 + o(t^2) = 1 - \frac{\sigma^2 t^2}{2} + o(t^2) \tag{$*$} \]
		Et, en appliquant le \cref{theoreme-central-limite-2} :
		\begin{align*}
			\left | \phi \left ( \frac{t}{\sqrt{n}} \right)^n - e^{-\frac{\sigma^2}{2} t^2} \right | &= \left | \phi \left ( \frac{t}{\sqrt{n}} \right)^n - \left( e^{- \frac{\sigma^2}{2n} t^2} \right)^n \right | \\
			&\leq n \left | \phi \left ( \frac{t}{\sqrt{n}} \right) - e^{-\frac{\sigma^2}{2n} t^2} \right |
		\end{align*}
		\[  \]
		On a d'une part, par développement limité :
		\[ e^{-\frac{\sigma^2}{2n} t^2} = 1 - \frac{\sigma^2}{2n} t^2 + o \left ( \frac{1}{n} \right) \]
		Et d'autre part, par $(*)$ :
		\[ \phi \left ( \frac{t}{\sqrt{n}} \right) = 1 - \frac{\sigma^2}{2n}t^2 + o \left ( \frac{1}{n} \right ) \]
		On obtient ainsi le résultat cherché, à savoir :
		\[ n \left | \phi \left ( \frac{t}{\sqrt{n}} \right) - e^{-\frac{\sigma^2}{2n} t^2} \right | = o (1) \]
	\end{proof}
	%</content>
\end{document}
