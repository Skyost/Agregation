\documentclass[12pt, a4paper]{report}

% LuaLaTeX :

\RequirePackage{iftex}
\RequireLuaTeX

% Packages :

\usepackage[french]{babel}
%\usepackage[utf8]{inputenc}
%\usepackage[T1]{fontenc}
\usepackage[pdfencoding=auto, pdfauthor={Hugo Delaunay}, pdfsubject={Mathématiques}, pdfcreator={agreg.skyost.eu}]{hyperref}
\usepackage{amsmath}
\usepackage{amsthm}
%\usepackage{amssymb}
\usepackage{stmaryrd}
\usepackage{tikz}
\usepackage{tkz-euclide}
\usepackage{fourier-otf}
\usepackage{fontspec}
\usepackage{titlesec}
\usepackage{fancyhdr}
\usepackage{catchfilebetweentags}
\usepackage[french, capitalise, noabbrev]{cleveref}
\usepackage[fit, breakall]{truncate}
\usepackage[top=2.5cm, right=2cm, bottom=2.5cm, left=2cm]{geometry}
\usepackage{enumerate}
\usepackage{tocloft}
\usepackage{microtype}
%\usepackage{mdframed}
%\usepackage{thmtools}
\usepackage{xcolor}
\usepackage{tabularx}
\usepackage{aligned-overset}
\usepackage[subpreambles=true]{standalone}
\usepackage{environ}
\usepackage[normalem]{ulem}
\usepackage{marginnote}
\usepackage{etoolbox}
\usepackage{setspace}
\usepackage[bibstyle=reading, citestyle=draft]{biblatex}
\usepackage{xpatch}
\usepackage[many, breakable]{tcolorbox}
\usepackage[backgroundcolor=white, bordercolor=white, textsize=small]{todonotes}

% Bibliographie :

\newcommand{\overridebibliographypath}[1]{\providecommand{\bibliographypath}{#1}}
\overridebibliographypath{../bibliography.bib}
\addbibresource{\bibliographypath}
\defbibheading{bibliography}[\bibname]{%
	\newpage
	\section*{#1}%
}
\renewbibmacro*{entryhead:full}{\printfield{labeltitle}}
\DeclareFieldFormat{url}{\newline\footnotesize\url{#1}}
\AtEndDocument{\printbibliography}

% Police :

\setmathfont{Erewhon Math}

% Tikz :

\usetikzlibrary{calc}

% Longueurs :

\setlength{\parindent}{0pt}
\setlength{\headheight}{15pt}
\setlength{\fboxsep}{0pt}
\titlespacing*{\chapter}{0pt}{-20pt}{10pt}
\setlength{\marginparwidth}{1.5cm}
\setstretch{1.1}

% Métadonnées :

\author{agreg.skyost.eu}
\date{\today}

% Titres :

\setcounter{secnumdepth}{3}

\renewcommand{\thechapter}{\Roman{chapter}}
\renewcommand{\thesubsection}{\Roman{subsection}}
\renewcommand{\thesubsubsection}{\arabic{subsubsection}}
\renewcommand{\theparagraph}{\alph{paragraph}}

\titleformat{\chapter}{\huge\bfseries}{\thechapter}{20pt}{\huge\bfseries}
\titleformat*{\section}{\LARGE\bfseries}
\titleformat{\subsection}{\Large\bfseries}{\thesubsection \, - \,}{0pt}{\Large\bfseries}
\titleformat{\subsubsection}{\large\bfseries}{\thesubsubsection. \,}{0pt}{\large\bfseries}
\titleformat{\paragraph}{\bfseries}{\theparagraph. \,}{0pt}{\bfseries}

\setcounter{secnumdepth}{4}

% Table des matières :

\renewcommand{\cftsecleader}{\cftdotfill{\cftdotsep}}
\addtolength{\cftsecnumwidth}{10pt}

% Redéfinition des commandes :

\renewcommand*\thesection{\arabic{section}}
\renewcommand{\ker}{\mathrm{Ker}}

% Nouvelles commandes :

\newcommand{\website}{https://agreg.skyost.eu}

\newcommand{\tr}[1]{\mathstrut ^t #1}
\newcommand{\im}{\mathrm{Im}}
\newcommand{\rang}{\operatorname{rang}}
\newcommand{\trace}{\operatorname{trace}}
\newcommand{\id}{\operatorname{id}}
\newcommand{\stab}{\operatorname{Stab}}

\providecommand{\newpar}{\\[\medskipamount]}

\providecommand{\lesson}[3]{%
	\title{#3}%
	\hypersetup{pdftitle={#3}}%
	\setcounter{section}{\numexpr #2 - 1}%
	\section{#3}%
	\fancyhead[R]{\truncate{0.73\textwidth}{#2 : #3}}%
}

\providecommand{\development}[3]{%
	\title{#3}%
	\hypersetup{pdftitle={#3}}%
	\section*{#3}%
	\fancyhead[R]{\truncate{0.73\textwidth}{#3}}%
}

\providecommand{\summary}[1]{%
	\textit{#1}%
	\medskip%
}

\tikzset{notestyleraw/.append style={inner sep=0pt, rounded corners=0pt, align=center}}

%\newcommand{\booklink}[1]{\website/bibliographie\##1}
\newcommand{\citelink}[2]{\hyperlink{cite.\therefsection @#1}{#2}}
\newcommand{\previousreference}{}
\providecommand{\reference}[2][]{%
	\notblank{#1}{\renewcommand{\previousreference}{#1}}{}%
	\todo[noline]{%
		\protect\vspace{16pt}%
		\protect\par%
		\protect\notblank{#1}{\cite{[\previousreference]}\\}{}%
		\protect\citelink{\previousreference}{p. #2}%
	}%
}

\definecolor{devcolor}{HTML}{00695c}
\newcommand{\dev}[1]{%
	\reversemarginpar%
	\todo[noline]{
		\protect\vspace{16pt}%
		\protect\par%
		\bfseries\color{devcolor}\href{\website/developpements/#1}{DEV}
	}%
	\normalmarginpar%
}

% En-têtes :

\pagestyle{fancy}
\fancyhead[L]{\truncate{0.23\textwidth}{\thepage}}
\fancyfoot[C]{\scriptsize \href{\website}{\texttt{agreg.skyost.eu}}}

% Couleurs :

\definecolor{property}{HTML}{fffde7}
\definecolor{proposition}{HTML}{fff8e1}
\definecolor{lemma}{HTML}{fff3e0}
\definecolor{theorem}{HTML}{fce4f2}
\definecolor{corollary}{HTML}{ffebee}
\definecolor{definition}{HTML}{ede7f6}
\definecolor{notation}{HTML}{f3e5f5}
\definecolor{example}{HTML}{e0f7fa}
\definecolor{cexample}{HTML}{efebe9}
\definecolor{application}{HTML}{e0f2f1}
\definecolor{remark}{HTML}{e8f5e9}
\definecolor{proof}{HTML}{e1f5fe}

% Théorèmes :

\theoremstyle{definition}
\newtheorem{theorem}{Théorème}

\newtheorem{property}[theorem]{Propriété}
\newtheorem{proposition}[theorem]{Proposition}
\newtheorem{lemma}[theorem]{Lemme}
\newtheorem{corollary}[theorem]{Corollaire}

\newtheorem{definition}[theorem]{Définition}
\newtheorem{notation}[theorem]{Notation}

\newtheorem{example}[theorem]{Exemple}
\newtheorem{cexample}[theorem]{Contre-exemple}
\newtheorem{application}[theorem]{Application}

\theoremstyle{remark}
\newtheorem{remark}[theorem]{Remarque}

\counterwithin*{theorem}{section}

\newcommand{\applystyletotheorem}[1]{
	\tcolorboxenvironment{#1}{
		enhanced,
		breakable,
		colback=#1!98!white,
		boxrule=0pt,
		boxsep=0pt,
		left=8pt,
		right=8pt,
		top=8pt,
		bottom=8pt,
		sharp corners,
		after=\par,
	}
}

\applystyletotheorem{property}
\applystyletotheorem{proposition}
\applystyletotheorem{lemma}
\applystyletotheorem{theorem}
\applystyletotheorem{corollary}
\applystyletotheorem{definition}
\applystyletotheorem{notation}
\applystyletotheorem{example}
\applystyletotheorem{cexample}
\applystyletotheorem{application}
\applystyletotheorem{remark}
\applystyletotheorem{proof}

% Environnements :

\NewEnviron{whitetabularx}[1]{%
	\renewcommand{\arraystretch}{2.5}
	\colorbox{white}{%
		\begin{tabularx}{\textwidth}{#1}%
			\BODY%
		\end{tabularx}%
	}%
}

% Maths :

\DeclareFontEncoding{FMS}{}{}
\DeclareFontSubstitution{FMS}{futm}{m}{n}
\DeclareFontEncoding{FMX}{}{}
\DeclareFontSubstitution{FMX}{futm}{m}{n}
\DeclareSymbolFont{fouriersymbols}{FMS}{futm}{m}{n}
\DeclareSymbolFont{fourierlargesymbols}{FMX}{futm}{m}{n}
\DeclareMathDelimiter{\VERT}{\mathord}{fouriersymbols}{152}{fourierlargesymbols}{147}


% Bibliographie :

\addbibresource{\bibliographypath}%
\defbibheading{bibliography}[\bibname]{%
	\newpage
	\section*{#1}%
}
\renewbibmacro*{entryhead:full}{\printfield{labeltitle}}%
\DeclareFieldFormat{url}{\newline\footnotesize\url{#1}}%

\AtEndDocument{\printbibliography}

\begin{document}
	%<*content>
	\development{algebra}{surjectivite-de-l-exponentielle}{\texorpdfstring{$\exp : \mathcal{M}_n(\mathbb{R}) \rightarrow \mathrm{GL}_n(\mathbb{R})$}{exp : Mn(R) -> GLn(R)} est surjective}

	\summary{Dans ce développement, on démontre que l'exponentielle de matrices est surjective en utilisant des théorèmes d'analyse.}

	\reference[I-P]{396}
	
	\begin{lemma}
		\label{surjectivite-de-l-exponentielle-1}
		Soit $M \in \mathrm{GL}_n(\mathbb{C})$. Alors $M^{-1} \in \mathbb{C}[X]$ (ie. $M^{-1}$ est un polynôme en $M$).
	\end{lemma}
	
	\begin{proof}
		D'après le théorème de Cayley-Hamilton, $\chi_M(M) = 0$. Or, en notant $\chi_M = \sum_{k=0}^n a_k X^k$, on a $a_0 = (-1)^n \det(M)$, d'où
		\[ 0 = M^n + \dots + a_1 M + (-1)^n \det(M) I_n \]
		En notant $Q = X^{n-1} + a_{n-1}X^{n-2} + \dots + a_2 X + a_1$, on en déduit que $(-1)^{n+1} \det(M) I_n = Q(M)M$. D'où
		\[ M^{-1} = \frac{(-1)^{n+1}}{\det(M)} Q(M) \in \mathbb{C}[M] \]
		ce qu'il fallait démonter.
	\end{proof}
	
	\begin{lemma}
		\label{surjectivite-de-l-exponentielle-2}
		Soit $M \in \mathcal{M}_n(\mathbb{C})$. Alors, $\exp(M) \in \mathrm{GL}_n(\mathbb{C})$.
	\end{lemma}
	
	\begin{proof}
		$M$ et $-M$ commutent, donc
		\[ \exp(M)\exp(-M) = \exp(M-M) = I_n = \exp(-M)\exp(M) \]
		Ainsi $\exp(M)$ est inversible, d'inverse $\exp(-M)$.
	\end{proof}
	
	\begin{lemma}
		\label{surjectivite-de-l-exponentielle-3}
		$\exp$ est différentiable en $0$ et,
		\[ \mathrm{d}\exp_0 = I_n \]
	\end{lemma}
	
	\begin{proof}
		Soit $H \in \mathcal{M}_n(\mathbb{C})$.
		\begin{align*}
			\exp(0+H) - \exp(H) &= \sum_{k=0}^{+\infty} \frac{H^k}{k!} \\
			&= I_n + H + \sum_{k=2}^{+\infty} \frac{H^k}{k!} \\
		\end{align*}
		Soit $\Vert . \Vert$ une norme d'algèbre sur $\mathcal{M}_n(\mathbb{C})$. On a :
		\begin{align*}
			\left\Vert \sum_{k=2}^{+\infty} \frac{H^k}{k!} \right\Vert &\leq \sum_{k=2}^{+\infty} \left\Vert \frac{H^k}{k!} \right\Vert \\
			&\leq \sum_{k=2}^{+\infty} \frac{\Vert H \Vert^k}{k!}
		\end{align*}
		Donc,
		\[ \sum_{k=2}^{+\infty} \frac{\Vert H \Vert^k}{k!} = o(\Vert H \Vert) \]
		ce qui donne le résultat annoncé.
	\end{proof}
	
	\begin{theorem}
		\label{surjectivite-de-l-exponentielle-4}
		$\exp : \mathcal{M}_n(\mathbb{C}) \rightarrow \mathrm{GL}_n(\mathbb{C})$ est surjective.
	\end{theorem}
	
	\begin{proof}
		Fixons $C \in \mathcal{M}_n(\mathbb{C})$ pour le reste de la démonstration. Comme $\mathbb{C}[C]$ est un sous-espace vectoriel de l'espace $\mathcal{M}_n(\mathbb{C})$, il est de dimension finie et est donc fermé. En particulier, $\exp(C) \in \mathbb{C}[C]$.
		\newpar
		Posons $\mathbb{C}[C]^* = \mathbb{C}[C] \, \cap \, \mathrm{GL}_n(\mathbb{C})$, et montrons que c'est un sous-groupe de $\mathrm{GL}_n(\mathbb{C})$.
		\begin{itemize}
			\item $I_n \in \mathbb{C}[C]$ et $I_n \in \mathrm{GL}_n(\mathbb{C})$, donc $I_n \in \mathbb{C}[C]^*$.
			\item Soit $M \in \mathbb{C}[C]^*$. Comme $M \in \mathrm{GL}_n(\mathbb{C})$, $M^{-1}$ existe, est inversible, et, par le \cref{surjectivite-de-l-exponentielle-1}, $M^{-1} \in \mathbb{C}[C]$.
			\item Enfin, $\mathbb{C}[C]^*$ est clairement stable par multiplication.
		\end{itemize}
		Ainsi, $\mathbb{C}[C]^*$ est un sous-groupe de $\mathrm{GL}_n(\mathbb{C})$, ce qui, combiné au \cref{surjectivite-de-l-exponentielle-2}, nous dit que $\exp : \mathbb{C}[C] \rightarrow \mathbb{C}[C]^*$ est bien définie. Il s'agit de plus d'un morphisme de groupes. En effet, $\forall A, B \in \mathbb{C}[C]$, on a $AB=BA$, d'où $\exp(A)\exp(B) = \exp(A+B) = \exp(B)\exp(A)$.
		\newpar
		Montrons que $\mathbb{C}[C]^*$ est un ouvert connexe de $\mathbb{C}[C]$. Notons qu'il s'agit bien d'un ouvert de $\mathbb{C}[C]$, car c'est l'intersection de $\mathbb{C}[C]$ avec $\mathrm{GL}_n(\mathbb{C})$ qui est ouvert dans $\mathcal{M}_n(\mathbb{C})$. Ensuite, soient $A, B \in \mathbb{C}[C]^*$. On pose
		\[ P = \det((1-X)A+XB) \]
		$P$ ne s'annule ni en $0$, ni en $1$ par inversibilité de $A$ et $B$. $P$ a un nombre fini de racines car n'est pas nul : on peut trouver une fonction continue $\gamma : [0,1] \rightarrow \mathbb{C}$ qui évite ces racines. Donc,
		\[ \forall t \in [0,1], \, \gamma(t) \in \mathbb{C}[C]^* \]
		donc $\mathbb{C}[C]^*$ est connexe par arcs, donc est connexe.
		\newpar
		Il s'agit maintenant de montrer que $\exp(\mathbb{C}[C]^*)$ est un ouvert-fermé de $\mathbb{C}[C]^*$. Par le théorème d'inversion locale appliqué à $\exp : \mathbb{C}[C] \rightarrow \mathbb{C}[C]$ (qui est bien $\mathcal{C}^1$ sur l'espace de Banach $\mathbb{C}[C]$ et, par le \cref{surjectivite-de-l-exponentielle-3}, $\det(\mathrm{d}\exp_0) \neq 0$) : il existe $U$ un voisinage de $0$ dans $\mathbb{C}(C)$ et un ouvert $V$ de $\mathbb{C}(C)$ contenant $\exp(0) = I_n$ tels que $\exp : U \rightarrow V$ soit un difféomorphisme de classe $\mathcal{C}^1$. Soit $A \in \mathbb{C}[C]$. Posons
		\[
			f_A :
			\begin{array}{ccc}
				\mathbb{C}[C] &\rightarrow& \mathbb{C}[C] \\
				M &\mapsto& \exp(A)^{-1}M
			\end{array}
		\]
		Pour tout $B \in V$, $f(\exp(A)B) = \exp(A)^{-1}(\exp(A)B) = B \in V$, donc $\exp(A)V \subseteq f^{-1}(V)$. Soit $B \in f^{-1}(V)$, alors $f(B) \in V$. Or, $f(B) = \exp(A)^{-1}B$, donc $B = \exp(A)f(B) \in \exp(A)V$. On en déduit que $\exp(A)V = f^{-1}(V)$ et que $\exp(A)V$ est un ouvert par continuité de $f$. Comme $V$ contient $I_n$, $\exp(A)V$ est un voisinage de $\exp(A)$. Or, $\exp(A)V$ est inclus dans $\mathbb{C}[C]$ car pour tout $B \in V$, il existe $M \in \mathbb{C}[C]$ tel que $\exp(M)=B$. Ainsi,
		\[ \exp(A)B = \exp(A)\exp(M) = \exp(A+M) \in \exp(\mathbb{C}[C]) \]
		On en déduit que $\exp(\mathbb{C}[C])$ est un ouvert.
		\newpar
		Montrons maintenant que
		\[ \mathbb{C}[C]^* \setminus \exp(\mathbb{C}[C]) = \bigcup_{A \in (\mathbb{C}[C]^* \setminus \exp(\mathbb{C}[C])} A\exp(\mathbb{C}[C]) \tag{$(*)$} \]
		Soient $A \in \mathbb{C}[C]^* \setminus \exp(\mathbb{C}[C])$ et $B \in \exp(\mathbb{C}[C])$. Alors $AB \in \mathbb{C}[C]^*$. Supposons par l'absurde que $AB \in \exp(\mathbb{C}[C])$. Il existe donc $M \in \exp(\mathbb{C}[C])$ tel que $AB = M$ et $A=MB^{-1}$. Comme $\exp(\mathbb{C}[C])$ est un groupe multiplicatif, alors $A \in \exp(\mathbb{C}[C])$ : absurde. On conclut que
		\[ \bigcup_{A \in \mathbb{C}[C]^* \setminus \exp(\mathbb{C}[C])} A\exp(\mathbb{C}[C]) \subseteq \mathbb{C}[C]^* \setminus \exp(\mathbb{C}[C]) \]
		Réciproquement, supposons que $M \in \mathbb{C}[C]^* \setminus \exp(\mathbb{C}[C])$. Comme $I_n \in \exp(\mathbb{C}[C])$, alors $M \in M\exp(\mathbb{C}[C])$. On en déduit $(*)$, d'où la fermeture de $\exp(\mathbb{C}[M])$.
		\newpar
		$\exp(\mathbb{C}[M])$ est un ouvert fermé non vide (car contient $I_n$) de $\mathbb{C}[M]^*$, alors $\exp(\mathbb{C}[M]) = \mathbb{C}[M]^*$. Pour conclure, si $C \in \mathrm{GL}_n(\mathbb{C})$, alors comme $C \in \mathbb{C}[C]$, $C \in \mathbb{C}[C]^*$. Donc $C \in \exp(\mathbb{C}[C])$, et $\exp$ est bien surjective.
	\end{proof}
	
	\begin{application}
		$\exp(\mathcal{M}_n(\mathbb{R})) = \mathrm{GL}_n(\mathbb{R})^2$, où $\mathrm{GL}_n(\mathbb{R})^2$ désigne les carrés de $\mathrm{GL}_n(\mathbb{R})$.
	\end{application}
	
	\begin{proof}
		Soit $M \in \mathcal{M}_n(\mathbb{R})$. Alors,
		\[ \exp(M) = \exp \left( \frac{M}{2} \right)^2 \]
		d'où $\exp(\mathcal{M}_n(\mathbb{R})) \subseteq \mathrm{GL}_n(\mathbb{R})^2$. Réciproquement, soit $A \in \mathrm{GL}_n(\mathbb{R})^2$. Posons $B = A^2$. D'après le \cref{surjectivite-de-l-exponentielle-4},
		\[ \exists P \in \mathbb{C}[X] \text{ telle que } A = \exp(P(A)) \]
		Comme $A$ est une matrice réelle, alors en passant au conjugué, on obtient $A = \exp(\overline{P}(A))$. Ainsi,
		\[ B = A^2 = \exp((P + \overline{P})(A)) \in \exp(\mathcal{M}_n(\mathbb{R})) \]
		d'où $\mathrm{GL}_n(\mathbb{R})^2 \subseteq \exp(\mathcal{M}_n(\mathbb{R}))$.
	\end{proof}
	%</content>
\end{document}
