\documentclass[12pt, a4paper]{report}

% LuaLaTeX :

\RequirePackage{iftex}
\RequireLuaTeX

% Packages :

\usepackage[french]{babel}
%\usepackage[utf8]{inputenc}
%\usepackage[T1]{fontenc}
\usepackage[pdfencoding=auto, pdfauthor={Hugo Delaunay}, pdfsubject={Mathématiques}, pdfcreator={agreg.skyost.eu}]{hyperref}
\usepackage{amsmath}
\usepackage{amsthm}
%\usepackage{amssymb}
\usepackage{stmaryrd}
\usepackage{tikz}
\usepackage{tkz-euclide}
\usepackage{fourier-otf}
\usepackage{fontspec}
\usepackage{titlesec}
\usepackage{fancyhdr}
\usepackage{catchfilebetweentags}
\usepackage[french, capitalise, noabbrev]{cleveref}
\usepackage[fit, breakall]{truncate}
\usepackage[top=2.5cm, right=2cm, bottom=2.5cm, left=2cm]{geometry}
\usepackage{enumerate}
\usepackage{tocloft}
\usepackage{microtype}
%\usepackage{mdframed}
%\usepackage{thmtools}
\usepackage{xcolor}
\usepackage{tabularx}
\usepackage{aligned-overset}
\usepackage[subpreambles=true]{standalone}
\usepackage{environ}
\usepackage[normalem]{ulem}
\usepackage{marginnote}
\usepackage{etoolbox}
\usepackage{setspace}
\usepackage[bibstyle=reading, citestyle=draft]{biblatex}
\usepackage{xpatch}
\usepackage[many, breakable]{tcolorbox}
\usepackage[backgroundcolor=white, bordercolor=white, textsize=small]{todonotes}

% Bibliographie :

\newcommand{\overridebibliographypath}[1]{\providecommand{\bibliographypath}{#1}}
\overridebibliographypath{../bibliography.bib}
\addbibresource{\bibliographypath}
\defbibheading{bibliography}[\bibname]{%
	\newpage
	\section*{#1}%
}
\renewbibmacro*{entryhead:full}{\printfield{labeltitle}}
\DeclareFieldFormat{url}{\newline\footnotesize\url{#1}}
\AtEndDocument{\printbibliography}

% Police :

\setmathfont{Erewhon Math}

% Tikz :

\usetikzlibrary{calc}

% Longueurs :

\setlength{\parindent}{0pt}
\setlength{\headheight}{15pt}
\setlength{\fboxsep}{0pt}
\titlespacing*{\chapter}{0pt}{-20pt}{10pt}
\setlength{\marginparwidth}{1.5cm}
\setstretch{1.1}

% Métadonnées :

\author{agreg.skyost.eu}
\date{\today}

% Titres :

\setcounter{secnumdepth}{3}

\renewcommand{\thechapter}{\Roman{chapter}}
\renewcommand{\thesubsection}{\Roman{subsection}}
\renewcommand{\thesubsubsection}{\arabic{subsubsection}}
\renewcommand{\theparagraph}{\alph{paragraph}}

\titleformat{\chapter}{\huge\bfseries}{\thechapter}{20pt}{\huge\bfseries}
\titleformat*{\section}{\LARGE\bfseries}
\titleformat{\subsection}{\Large\bfseries}{\thesubsection \, - \,}{0pt}{\Large\bfseries}
\titleformat{\subsubsection}{\large\bfseries}{\thesubsubsection. \,}{0pt}{\large\bfseries}
\titleformat{\paragraph}{\bfseries}{\theparagraph. \,}{0pt}{\bfseries}

\setcounter{secnumdepth}{4}

% Table des matières :

\renewcommand{\cftsecleader}{\cftdotfill{\cftdotsep}}
\addtolength{\cftsecnumwidth}{10pt}

% Redéfinition des commandes :

\renewcommand*\thesection{\arabic{section}}
\renewcommand{\ker}{\mathrm{Ker}}

% Nouvelles commandes :

\newcommand{\website}{https://agreg.skyost.eu}

\newcommand{\tr}[1]{\mathstrut ^t #1}
\newcommand{\im}{\mathrm{Im}}
\newcommand{\rang}{\operatorname{rang}}
\newcommand{\trace}{\operatorname{trace}}
\newcommand{\id}{\operatorname{id}}
\newcommand{\stab}{\operatorname{Stab}}

\providecommand{\newpar}{\\[\medskipamount]}

\providecommand{\lesson}[3]{%
	\title{#3}%
	\hypersetup{pdftitle={#3}}%
	\setcounter{section}{\numexpr #2 - 1}%
	\section{#3}%
	\fancyhead[R]{\truncate{0.73\textwidth}{#2 : #3}}%
}

\providecommand{\development}[3]{%
	\title{#3}%
	\hypersetup{pdftitle={#3}}%
	\section*{#3}%
	\fancyhead[R]{\truncate{0.73\textwidth}{#3}}%
}

\providecommand{\summary}[1]{%
	\textit{#1}%
	\medskip%
}

\tikzset{notestyleraw/.append style={inner sep=0pt, rounded corners=0pt, align=center}}

%\newcommand{\booklink}[1]{\website/bibliographie\##1}
\newcommand{\citelink}[2]{\hyperlink{cite.\therefsection @#1}{#2}}
\newcommand{\previousreference}{}
\providecommand{\reference}[2][]{%
	\notblank{#1}{\renewcommand{\previousreference}{#1}}{}%
	\todo[noline]{%
		\protect\vspace{16pt}%
		\protect\par%
		\protect\notblank{#1}{\cite{[\previousreference]}\\}{}%
		\protect\citelink{\previousreference}{p. #2}%
	}%
}

\definecolor{devcolor}{HTML}{00695c}
\newcommand{\dev}[1]{%
	\reversemarginpar%
	\todo[noline]{
		\protect\vspace{16pt}%
		\protect\par%
		\bfseries\color{devcolor}\href{\website/developpements/#1}{DEV}
	}%
	\normalmarginpar%
}

% En-têtes :

\pagestyle{fancy}
\fancyhead[L]{\truncate{0.23\textwidth}{\thepage}}
\fancyfoot[C]{\scriptsize \href{\website}{\texttt{agreg.skyost.eu}}}

% Couleurs :

\definecolor{property}{HTML}{fffde7}
\definecolor{proposition}{HTML}{fff8e1}
\definecolor{lemma}{HTML}{fff3e0}
\definecolor{theorem}{HTML}{fce4f2}
\definecolor{corollary}{HTML}{ffebee}
\definecolor{definition}{HTML}{ede7f6}
\definecolor{notation}{HTML}{f3e5f5}
\definecolor{example}{HTML}{e0f7fa}
\definecolor{cexample}{HTML}{efebe9}
\definecolor{application}{HTML}{e0f2f1}
\definecolor{remark}{HTML}{e8f5e9}
\definecolor{proof}{HTML}{e1f5fe}

% Théorèmes :

\theoremstyle{definition}
\newtheorem{theorem}{Théorème}

\newtheorem{property}[theorem]{Propriété}
\newtheorem{proposition}[theorem]{Proposition}
\newtheorem{lemma}[theorem]{Lemme}
\newtheorem{corollary}[theorem]{Corollaire}

\newtheorem{definition}[theorem]{Définition}
\newtheorem{notation}[theorem]{Notation}

\newtheorem{example}[theorem]{Exemple}
\newtheorem{cexample}[theorem]{Contre-exemple}
\newtheorem{application}[theorem]{Application}

\theoremstyle{remark}
\newtheorem{remark}[theorem]{Remarque}

\counterwithin*{theorem}{section}

\newcommand{\applystyletotheorem}[1]{
	\tcolorboxenvironment{#1}{
		enhanced,
		breakable,
		colback=#1!98!white,
		boxrule=0pt,
		boxsep=0pt,
		left=8pt,
		right=8pt,
		top=8pt,
		bottom=8pt,
		sharp corners,
		after=\par,
	}
}

\applystyletotheorem{property}
\applystyletotheorem{proposition}
\applystyletotheorem{lemma}
\applystyletotheorem{theorem}
\applystyletotheorem{corollary}
\applystyletotheorem{definition}
\applystyletotheorem{notation}
\applystyletotheorem{example}
\applystyletotheorem{cexample}
\applystyletotheorem{application}
\applystyletotheorem{remark}
\applystyletotheorem{proof}

% Environnements :

\NewEnviron{whitetabularx}[1]{%
	\renewcommand{\arraystretch}{2.5}
	\colorbox{white}{%
		\begin{tabularx}{\textwidth}{#1}%
			\BODY%
		\end{tabularx}%
	}%
}

% Maths :

\DeclareFontEncoding{FMS}{}{}
\DeclareFontSubstitution{FMS}{futm}{m}{n}
\DeclareFontEncoding{FMX}{}{}
\DeclareFontSubstitution{FMX}{futm}{m}{n}
\DeclareSymbolFont{fouriersymbols}{FMS}{futm}{m}{n}
\DeclareSymbolFont{fourierlargesymbols}{FMX}{futm}{m}{n}
\DeclareMathDelimiter{\VERT}{\mathord}{fouriersymbols}{152}{fourierlargesymbols}{147}


% Bibliographie :

\addbibresource{\bibliographypath}%
\defbibheading{bibliography}[\bibname]{%
	\newpage
	\section*{#1}%
}
\renewbibmacro*{entryhead:full}{\printfield{labeltitle}}%
\DeclareFieldFormat{url}{\newline\footnotesize\url{#1}}%

\AtEndDocument{\printbibliography}

\begin{document}
	%<*content>
	\development{analysis}{dual-de-lp}{Dual de \texorpdfstring{$L_p$}{Lp}}

	\summary{Avec les propriétés hilbertiennes de $L_2$ couplées à certaines propriétés des espaces $L_p$, on montre que le dual d'un espace $L_p$ est $L_q$ pour $\frac{1}{p} + \frac{1}{q} = 1$, dans le cas où $p \in ]1, 2[$ et où l'espace est de mesure finie.}

	Soit $(X, \mathcal{A}, \mu)$ un espace mesuré de mesure finie.

	\begin{notation}
		On note $\forall p \in [1, 2]$, $L_p = L_p(X, \mathcal{A}, \mu)$.
	\end{notation}

	\begin{lemma}
		\label{dual-de-lp-1}
		Soient $p \in ]1, 2[$ et $f \in L_2$. Alors $f \in L_p$ telle que $\Vert f \Vert_p \leq M \Vert f \Vert_2$ où $M \geq 0$.
	\end{lemma}

	\begin{proof}
		Comme $p \in ]1, 2[$, on a $\frac{2}{p} > 1$. Soit $r$ tel que $\frac{p}{2} + \frac{1}{r} = 1$. On applique l'inégalité de Hölder à $g = \vert f \vert^p \mathbb{1}_X$ de sorte que
		\[ \int_X \vert f \vert^p \, \mathrm{d}\mu = \Vert \vert f \vert^p \mathbb{1}_X \Vert_1 \leq \Vert \vert f \vert^p \Vert_{\frac{2}{p}} \Vert \mathbb{1}_X \Vert_r \leq \mu(X)^{\frac{1}{r}} \Vert f \Vert_2^p \]
		d'où le résultat.
	\end{proof}

	\begin{lemma}
		\label{dual-de-lp-2}
		Soit $p \in ]1, 2[$. Alors $L_2$ est dense dans $L_p$ pour la norme $\Vert . \Vert_p$.
	\end{lemma}

	\begin{proof}
		Soit $f \in L_p$. On considère la suite de fonction $(f_n)$ définie par
		\[ \forall n \in \mathbb{N}, \, f_n = f \mathbb{1}_{|f| \leq n} \]
		Clairement, $(f_n)$ est une suite de $L_2$. On va chercher à appliquer le théorème de convergence dominée à la suite de fonctions $(g_n)$ définie pour tout $n \in \mathbb{N}$ par $g_n = |f_n - f|^p$ :
		\begin{itemize}
			\item $\forall n \in \mathbb{N}$, $g_n$ est mesurable.
			\item $(g_n)$ converge presque partout vers la fonction nulle.
			\item Par convexité de la fonction $x \mapsto x^p$, on a
			\[ |f_n - f|^p = 2^p \left| \frac{f_n}{2} - \frac{f}{2} \right|^p \leq 2^{p-1} (|f|^p + |f_n|^p) \leq 2^p |f|^p \in L_1 \]
		\end{itemize}
		On peut donc conclure
		\[ \Vert f - f_n \Vert^p_p = \int_X |f - f_n|^p \, \mathrm{d}\mu \longrightarrow 0 \]
		ce qu'il fallait démontrer.
	\end{proof}

	\reference[Z-Q]{216}

	\begin{theorem}
		L'application
		\[
		\varphi :
		\begin{array}{ll}
			L_q &\rightarrow (L_p)' \\
			g &\mapsto \left( \varphi_g : f \mapsto \int_X f g \, \mathrm{d}\mu \right)
		\end{array}
		\qquad \text{ où } \frac{1}{p} + \frac{1}{q} = 1
		\]
		est une isométrie linéaire surjective. C'est donc un isomorphisme isométrique.
	\end{theorem}

	\begin{proof}
		Soit $g \in L_q$ et $f \in L_p$. L'inégalité de Hölder donne
		\[ \vert \varphi_g(f) \vert \leq \Vert g \Vert_q \Vert f \Vert_p \]
		donc $\varphi_g \in (L_p)'$ et $\VERT \varphi_g \VERT \leq \Vert g \Vert_q$. De plus, si $g = 0$, alors $\VERT \varphi_g \VERT = \Vert g \Vert_q = 0$. On peut donc supposer $g \neq 0$.
		\newpar
	 	Soit $u$ une fonction mesurable de module 1, telle que $g = u \vert g \vert$. On pose $h = \overline{u} \vert g \vert^{q-1}$. Comme $q = p(q-1)$, on a
		\[ \int_X \vert h \vert^p \, \mathrm{d}\mu = \int_X \vert g \vert^{(q-1)p} \, \mathrm{d}\mu = \int_X \vert g \vert^{q} \, \mathrm{d}\mu < + \infty \]
		d'où $h \in L_p$ et $\Vert h \Vert_p^p = \Vert g \Vert_q^q = \vert \varphi_g(h) \vert$. Comme, $\frac{\vert \varphi_g(h) \vert}{\Vert h \Vert_p} \leq \VERT \varphi_g \VERT$, on a en particulier,
		\[ \underbrace{\int_X \vert g \vert^{q} \, \mathrm{d}\mu}_{= \vert \varphi_g(h) \vert} \leq \VERT \varphi_g \VERT \underbrace{\left ( \int_X \vert g \vert^{q} \, \mathrm{d}\mu \right )^{\frac{1}{p}}}_{= \Vert h \Vert_p} \]
		et ainsi,
		\[ \VERT \varphi_g \VERT \geq \left ( \int_X \vert g \vert^{q} \, \mathrm{d}\mu \right )^{1 - \frac{1}{p}} = \left ( \int_X \vert g \vert^{q} \, \mathrm{d}\mu \right )^{\frac{1}{q}} = \Vert g \Vert_q \]
		donc $\VERT \varphi_g \VERT = \Vert g \Vert_q$ et $\varphi$ est une isométrie.
		\newpar
		Montrons qu'elle est surjective. Soit $\ell \in (L_p)'$. D'après le \cref{dual-de-lp-1}, on a $L_2 \subseteq L_p$, donc on peut considérer la restriction $\widetilde{\ell} = \ell_{| L_2}$.
		\[ \forall f \in L_2, \quad \vert \widetilde{\ell}(f) \vert \leq \Vert \ell \Vert \Vert f \Vert_p \leq M \Vert \ell \Vert \Vert f \Vert_2 \implies \widetilde{\ell} \in (L_2)' \]
		Comme $L_2$ est un espace de Hilbert, on peut appliquer le théorème de représentation de Riesz à $\widetilde{\ell}$. Il existe $g \in L_2$ telle que
		\[ \forall f \in L_2, \quad \widetilde{\ell}(f) = \int_X f \overline{g} \, \mathrm{d}\mu \]
		Pour conclure, il reste à montrer que $g \in L_q$ et que l'égalité précédente est vérifiée sur $L_p$. Comme dans précédemment, on considère $u$ de module $1$ telle que $g = u \vert g \vert$ et on pose $f_n = \overline{u} \vert g \vert^{q-1} \mathbb{1}_{\vert g \vert \leq n} \in L_\infty \subseteq L_2$. On a
		\[ \int_X \vert g \vert^q \mathbb{1}_{\vert g \vert \leq n} \, \mathrm{d}\mu = \ell(f_n) \leq \Vert \ell \Vert \Vert f_n \Vert_p = \Vert \ell \Vert \left ( \int_X \vert g \vert^q \mathbb{1}_{\vert g \vert \leq n} \, \mathrm{d}\mu \right )^{\frac{1}{p}} \]
		D'où
		\[ \left ( \int_X \vert g \vert^q \mathbb{1}_{\vert g \vert \leq n} \, \mathrm{d}\mu \right )^{\frac{1}{q}} = \left ( \int_X \vert g \vert^q \mathbb{1}_{\vert g \vert \leq n} \, \mathrm{d}\mu \right )^{1 - \frac{1}{p}} \leq \Vert \ell \Vert \]
		D'après le théorème de convergence monotone, on a
		\[ \lim_{n \rightarrow +\infty} \left ( \int_X \vert g \vert^q \mathbb{1}_{\vert g \vert \leq n} \, \mathrm{d}\mu \right )^{\frac{1}{q}} = \left ( \int_X \vert g \vert^q \, \mathrm{d}\mu \right )^{\frac{1}{q}} = \Vert g \Vert_q \leq \Vert \ell \Vert \]
		Et en particulier, $g \in L_q$. Ainsi, on a $\forall f \in L_2$, $\ell(f) = \varphi_g(f)$. Les applications $\ell$ et $\varphi_g$ sont continues sur $L_p$ et $L_2$ est dense dans $L_p$ (par le \cref{dual-de-lp-2}), donc on a bien $\ell = \varphi_g = \varphi(g)$.
	\end{proof}

	\reference[LI]{140}

	\begin{remark}
		Plus généralement, si l'on identifie $g$ et $\varphi_g$ :
		\begin{itemize}
			\item $L_q$ est le dual topologique de $L_p$ pour $p \in ]1, +\infty[$.
			\item $L_\infty$ est le dual topologique de $L_1$ si $\mu$ est $\sigma$-finie.
		\end{itemize}
	\end{remark}
	%</content>
\end{document}
