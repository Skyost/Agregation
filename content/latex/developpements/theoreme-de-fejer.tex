\documentclass[12pt, a4paper]{report}

% LuaLaTeX :

\RequirePackage{iftex}
\RequireLuaTeX

% Packages :

\usepackage[french]{babel}
%\usepackage[utf8]{inputenc}
%\usepackage[T1]{fontenc}
\usepackage[pdfencoding=auto, pdfauthor={Hugo Delaunay}, pdfsubject={Mathématiques}, pdfcreator={agreg.skyost.eu}]{hyperref}
\usepackage{amsmath}
\usepackage{amsthm}
%\usepackage{amssymb}
\usepackage{stmaryrd}
\usepackage{tikz}
\usepackage{tkz-euclide}
\usepackage{fourier-otf}
\usepackage{fontspec}
\usepackage{titlesec}
\usepackage{fancyhdr}
\usepackage{catchfilebetweentags}
\usepackage[french, capitalise, noabbrev]{cleveref}
\usepackage[fit, breakall]{truncate}
\usepackage[top=2.5cm, right=2cm, bottom=2.5cm, left=2cm]{geometry}
\usepackage{enumerate}
\usepackage{tocloft}
\usepackage{microtype}
%\usepackage{mdframed}
%\usepackage{thmtools}
\usepackage{xcolor}
\usepackage{tabularx}
\usepackage{aligned-overset}
\usepackage[subpreambles=true]{standalone}
\usepackage{environ}
\usepackage[normalem]{ulem}
\usepackage{marginnote}
\usepackage{etoolbox}
\usepackage{setspace}
\usepackage[bibstyle=reading, citestyle=draft]{biblatex}
\usepackage{xpatch}
\usepackage[many, breakable]{tcolorbox}
\usepackage[backgroundcolor=white, bordercolor=white, textsize=small]{todonotes}

% Bibliographie :

\newcommand{\overridebibliographypath}[1]{\providecommand{\bibliographypath}{#1}}
\overridebibliographypath{../bibliography.bib}
\addbibresource{\bibliographypath}
\defbibheading{bibliography}[\bibname]{%
	\newpage
	\section*{#1}%
}
\renewbibmacro*{entryhead:full}{\printfield{labeltitle}}
\DeclareFieldFormat{url}{\newline\footnotesize\url{#1}}
\AtEndDocument{\printbibliography}

% Police :

\setmathfont{Erewhon Math}

% Tikz :

\usetikzlibrary{calc}

% Longueurs :

\setlength{\parindent}{0pt}
\setlength{\headheight}{15pt}
\setlength{\fboxsep}{0pt}
\titlespacing*{\chapter}{0pt}{-20pt}{10pt}
\setlength{\marginparwidth}{1.5cm}
\setstretch{1.1}

% Métadonnées :

\author{agreg.skyost.eu}
\date{\today}

% Titres :

\setcounter{secnumdepth}{3}

\renewcommand{\thechapter}{\Roman{chapter}}
\renewcommand{\thesubsection}{\Roman{subsection}}
\renewcommand{\thesubsubsection}{\arabic{subsubsection}}
\renewcommand{\theparagraph}{\alph{paragraph}}

\titleformat{\chapter}{\huge\bfseries}{\thechapter}{20pt}{\huge\bfseries}
\titleformat*{\section}{\LARGE\bfseries}
\titleformat{\subsection}{\Large\bfseries}{\thesubsection \, - \,}{0pt}{\Large\bfseries}
\titleformat{\subsubsection}{\large\bfseries}{\thesubsubsection. \,}{0pt}{\large\bfseries}
\titleformat{\paragraph}{\bfseries}{\theparagraph. \,}{0pt}{\bfseries}

\setcounter{secnumdepth}{4}

% Table des matières :

\renewcommand{\cftsecleader}{\cftdotfill{\cftdotsep}}
\addtolength{\cftsecnumwidth}{10pt}

% Redéfinition des commandes :

\renewcommand*\thesection{\arabic{section}}
\renewcommand{\ker}{\mathrm{Ker}}

% Nouvelles commandes :

\newcommand{\website}{https://agreg.skyost.eu}

\newcommand{\tr}[1]{\mathstrut ^t #1}
\newcommand{\im}{\mathrm{Im}}
\newcommand{\rang}{\operatorname{rang}}
\newcommand{\trace}{\operatorname{trace}}
\newcommand{\id}{\operatorname{id}}
\newcommand{\stab}{\operatorname{Stab}}

\providecommand{\newpar}{\\[\medskipamount]}

\providecommand{\lesson}[3]{%
	\title{#3}%
	\hypersetup{pdftitle={#3}}%
	\setcounter{section}{\numexpr #2 - 1}%
	\section{#3}%
	\fancyhead[R]{\truncate{0.73\textwidth}{#2 : #3}}%
}

\providecommand{\development}[3]{%
	\title{#3}%
	\hypersetup{pdftitle={#3}}%
	\section*{#3}%
	\fancyhead[R]{\truncate{0.73\textwidth}{#3}}%
}

\providecommand{\summary}[1]{%
	\textit{#1}%
	\medskip%
}

\tikzset{notestyleraw/.append style={inner sep=0pt, rounded corners=0pt, align=center}}

%\newcommand{\booklink}[1]{\website/bibliographie\##1}
\newcommand{\citelink}[2]{\hyperlink{cite.\therefsection @#1}{#2}}
\newcommand{\previousreference}{}
\providecommand{\reference}[2][]{%
	\notblank{#1}{\renewcommand{\previousreference}{#1}}{}%
	\todo[noline]{%
		\protect\vspace{16pt}%
		\protect\par%
		\protect\notblank{#1}{\cite{[\previousreference]}\\}{}%
		\protect\citelink{\previousreference}{p. #2}%
	}%
}

\definecolor{devcolor}{HTML}{00695c}
\newcommand{\dev}[1]{%
	\reversemarginpar%
	\todo[noline]{
		\protect\vspace{16pt}%
		\protect\par%
		\bfseries\color{devcolor}\href{\website/developpements/#1}{DEV}
	}%
	\normalmarginpar%
}

% En-têtes :

\pagestyle{fancy}
\fancyhead[L]{\truncate{0.23\textwidth}{\thepage}}
\fancyfoot[C]{\scriptsize \href{\website}{\texttt{agreg.skyost.eu}}}

% Couleurs :

\definecolor{property}{HTML}{fffde7}
\definecolor{proposition}{HTML}{fff8e1}
\definecolor{lemma}{HTML}{fff3e0}
\definecolor{theorem}{HTML}{fce4f2}
\definecolor{corollary}{HTML}{ffebee}
\definecolor{definition}{HTML}{ede7f6}
\definecolor{notation}{HTML}{f3e5f5}
\definecolor{example}{HTML}{e0f7fa}
\definecolor{cexample}{HTML}{efebe9}
\definecolor{application}{HTML}{e0f2f1}
\definecolor{remark}{HTML}{e8f5e9}
\definecolor{proof}{HTML}{e1f5fe}

% Théorèmes :

\theoremstyle{definition}
\newtheorem{theorem}{Théorème}

\newtheorem{property}[theorem]{Propriété}
\newtheorem{proposition}[theorem]{Proposition}
\newtheorem{lemma}[theorem]{Lemme}
\newtheorem{corollary}[theorem]{Corollaire}

\newtheorem{definition}[theorem]{Définition}
\newtheorem{notation}[theorem]{Notation}

\newtheorem{example}[theorem]{Exemple}
\newtheorem{cexample}[theorem]{Contre-exemple}
\newtheorem{application}[theorem]{Application}

\theoremstyle{remark}
\newtheorem{remark}[theorem]{Remarque}

\counterwithin*{theorem}{section}

\newcommand{\applystyletotheorem}[1]{
	\tcolorboxenvironment{#1}{
		enhanced,
		breakable,
		colback=#1!98!white,
		boxrule=0pt,
		boxsep=0pt,
		left=8pt,
		right=8pt,
		top=8pt,
		bottom=8pt,
		sharp corners,
		after=\par,
	}
}

\applystyletotheorem{property}
\applystyletotheorem{proposition}
\applystyletotheorem{lemma}
\applystyletotheorem{theorem}
\applystyletotheorem{corollary}
\applystyletotheorem{definition}
\applystyletotheorem{notation}
\applystyletotheorem{example}
\applystyletotheorem{cexample}
\applystyletotheorem{application}
\applystyletotheorem{remark}
\applystyletotheorem{proof}

% Environnements :

\NewEnviron{whitetabularx}[1]{%
	\renewcommand{\arraystretch}{2.5}
	\colorbox{white}{%
		\begin{tabularx}{\textwidth}{#1}%
			\BODY%
		\end{tabularx}%
	}%
}

% Maths :

\DeclareFontEncoding{FMS}{}{}
\DeclareFontSubstitution{FMS}{futm}{m}{n}
\DeclareFontEncoding{FMX}{}{}
\DeclareFontSubstitution{FMX}{futm}{m}{n}
\DeclareSymbolFont{fouriersymbols}{FMS}{futm}{m}{n}
\DeclareSymbolFont{fourierlargesymbols}{FMX}{futm}{m}{n}
\DeclareMathDelimiter{\VERT}{\mathord}{fouriersymbols}{152}{fourierlargesymbols}{147}


% Bibliographie :

\addbibresource{\bibliographypath}%
\defbibheading{bibliography}[\bibname]{%
	\newpage
	\section*{#1}%
}
\renewbibmacro*{entryhead:full}{\printfield{labeltitle}}%
\DeclareFieldFormat{url}{\newline\footnotesize\url{#1}}%

\AtEndDocument{\printbibliography}

\begin{document}
	%<*content>
	\development{analysis}{theoreme-de-fejer}{Théorème de Fejér}

	\summary{Dans ce développement, on montre le théorème de Fejér, qui assure la convergence de la série de Fourier d'une fonction vers sa série de Fourier au sens de Cesàro.}

	\begin{notation}
		Pour tout $p \in [1, +\infty]$, on note $L_p^{2\pi}$ l'espace des fonctions $f : \mathbb{R} \rightarrow \mathbb{C}$, $2\pi$-périodiques et mesurables, telles que $\Vert f \Vert_p < +\infty$.
	\end{notation}

	\begin{notation}
		On note $\forall N \in \mathbb{N}^*$ :
		\begin{itemize}
			\item $e_n : x \mapsto e^{inx}$.
			\item $D_N : x \mapsto \sum_{n=-N}^N e_n$, le noyau de Dirichlet.
			\item $K_N = \frac{D_0 + \dots + D_{N-1}}{N}$, le noyau de Fejér.
		\end{itemize}
	\end{notation}

	\begin{notation}
		On note également, pour toute fonction $f \in L_1^{2\pi}$.
		\begin{itemize}
			\item $c_n(f) = \frac{1}{2\pi} \int_{- \pi}^\pi f(t) e^{-int} \, \mathrm{d}t$, le $n$-ième coefficient de Fourier de $f$.
			\item $S_N(f) : x \mapsto \sum_{n=-N}^N c_n(f) e_n$, la somme partielle d'ordre $N$ de la série de Fourier de $f$.
			\item $\sigma_N(f) : x \mapsto \frac{1}{N - 1} \sum_{n=0}^N S_n(f)(x)$, la moyenne de Cesàro des sommes partielles de la série de Fourier de $f$.
		\end{itemize}
	\end{notation}

	\reference[AMR08]{184}

	\begin{lemma}
		\label{theoreme-de-fejer-1}
		Soit $N \in \mathbb{N}$.
		\begin{enumerate}[label=(\roman*)]
			\item $D_N$ est une fonction paire, $2\pi$-périodique, et de norme $1$.
			\item \[ \forall x \in \mathbb{R} \setminus 2 \pi \mathbb{Z}, \, D_N(x) = \frac{\sin \left(\left( N + \frac{1}{2} \right) x \right)}{\sin \left( \frac{x}{2} \right)} \]
			\item Pour tout $f \in L_1^{2 \pi}, \, S_N(f) = f * D_N$.
		\end{enumerate}
	\end{lemma}

	\begin{proof}
		Soit $N \in \mathbb{N}$.
		\begin{enumerate}[label=(\roman*)]
			\item Soit $x \in \mathbb{R}$.
			\[ D_N(-x) = \sum_{n=-N}^{N} e_n(-x) = \sum_{n=-N}^{N} e_{-n}(x) = \sum_{n=-N}^{N} e_n(x) = D_N(x) \]
			Donc $D_N$ est bien paire. Elle est $2\pi$-périodique car $e_n$ l'est pour tout $n \in \mathbb{Z}$. De plus,
			\[ 1 = c_0(D_N) = \int_{-\pi}^{\pi} D_N(x) \, \mathrm{d}x = \Vert D_N \Vert_1 \]
			\item Soit $x \in \mathbb{R}\setminus 2\pi\mathbb{Z}$. On a :
			\begin{align*}
				D_n(x) &= e^{-iNx} \sum_{n=0}^{2N} e^{inx} \\
				&= e^{-iNx} \frac{e^{(2N+1)ix} - 1}{e^{ix} - 1} \\
				&= e^{-iNx} \frac{e^{(2N+1) i\frac{x}{2}} \left ( e^{(2N+1) i\frac{x}{2}} - e^{-(2N+1) i\frac{x}{2}} \right )}{e^{i\frac{x}{2}} \left( e^{i\frac{x}{2}} - e^{-i \frac{x}{2}} \right)} \\
				&= \frac{2i \sin \left( \left( N + \frac{1}{2} \right) x \right)}{2i \sin \left ( \frac{x}{2} \right)} \\
				&= \frac{\sin \left( \left( N + \frac{1}{2} \right) x \right)}{\sin \left ( \frac{x}{2} \right)}
			\end{align*}
			\item Soit $f \in L_1^{2 \pi}$.
			\[ f * D_N(f) = \sum_{n=-N}^N f * e_n = \sum_{n=-N}^N c_n(f) e_n = S_N(f) \]
		\end{enumerate}
	\end{proof}

	\begin{lemma}
		\label{theoreme-de-fejer-2}
		Soient $N \in \mathbb{N}^*$ et $f \in L_1^{2 \pi}$.
		\begin{enumerate}[label=(\roman*)]
			\item $K_N$ est une fonction positive et de norme $1$.
			\item \[ \forall x \in \mathbb{R} \setminus 2 \pi \mathbb{Z}, \, K_N(x) = \frac{1}{N} \left(\frac{\sin \left( \frac{Nx}{2} \right)}{\sin \left( \frac{x}{2} \right)}\right)^2 \]
			\item $K_N = \sum_{n=-N}^{N} \left(1 - \frac{\vert n \vert}{N}\right) e_n$.
			\item $\sigma_N(f) = f * K_N$.
		\end{enumerate}
	\end{lemma}

	\begin{proof}
		Soit $N\in\mathbb{N}^*$. Nous allons user et abuser du \cref{theoreme-de-fejer-1}.
		\begin{enumerate}[label=(\roman*)]
			\item La positivité résulte directement du point suivant. De plus,
			\[ \Vert K_N \Vert_1 = \frac{1}{2\pi} \int_0^{2\pi} K_N(x) \, \mathrm{d}x = 1 \]
			\item Soit $x\in\mathbb{R}\setminus 2\pi\mathbb{Z}$.
			\begin{align*}
				NK_N(x)&=\sum_{n=0}^{N-1} D_n(x) \\
				&= \sum_{n=0}^{N-1} \left( \sum_{\vert n \vert \leq j} e_j \right) \\
				&= \sum_{\vert n \vert \leq N-1} e_n \left( \sum_{\vert n \vert \leq j \leq N-1} 1 \right) \\
				&= \sum_{\vert n \vert \leq N-1} (N - \vert n \vert) e_n \\
				&= \sum_{n=-N}^{N} (N - \vert n \vert) e_n
			\end{align*}
			\item \begin{align*}
				NK_N&=\sum_{n=0}^{N-1}{D_n} \\
				&=\sum_{n=0}^{N}{\frac{\sin \left( \left( n + \frac{1}{2} \right) x \right)}{\sin \left ( \frac{x}{2} \right)}} \\
				&=\frac{1}{\sin \left ( \frac{x}{2} \right)}\operatorname{Im} \left( \sum_{n=0}^{N-1}{e^{i(n+\frac{1}{2})x}} \right) \\
				&=\frac{1}{\sin \left ( \frac{x}{2} \right)}\operatorname{Im} \left ( e^{\frac{ix}{2}}\frac{e^{iNx-1}}{e^{ix}-1} \right) \\
				&=\frac{1}{\sin \left ( \frac{x}{2} \right)} \operatorname{Im} \left ( e^{\frac{ix}{2}}\frac{e^{\frac{iNx}{2}}2i\sin \left( \frac{Nx}{2} \right)}{e^{\frac{ix}{2}}2i \sin \left ( \frac{x}{2} \right)} \right ) \\
				&=\frac{\sin \left( \frac{Nx}{2} \right)}{\sin \left ( \frac{x}{2} \right)^2}\operatorname{Im} \left(e^{\frac{iNx}{2}} \right) \\
				&=\frac{\sin \left( \frac{Nx}{2} \right)^2}{\sin \left ( \frac{x}{2} \right)^2}
			\end{align*}
			\item \[ N \sigma_N(f) = \sum_{n=0}^{N-1} S_n(f) = \sum_{n=0}^{N-1} f * D_n = f * \left (\sum_{n=0}^{N-1} D_n \right) \]
			Donc on a bien $\sigma_N(f) = f * K_N$.
		\end{enumerate}
	\end{proof}

	\reference[AMR08]{190}

	\begin{theorem}[Fejér]
		Soit $f : \mathbb{R} \rightarrow \mathbb{C}$ une fonction $2\pi$-périodique.
		\begin{enumerate}[label=(\roman*)]
			\item Si $f$ est continue, alors $\Vert \sigma_N(f) \Vert_\infty \leq \Vert f \Vert_\infty$ et $(\sigma_N(f))$ converge uniformément vers $f$.
			\item Si $f \in L_p^{2\pi}$ pour $p \in [1,+\infty[$, alors $\Vert \sigma_N(f) \Vert_p \leq \Vert f \Vert_p$ et $(\sigma_N(f))$ converge vers $f$ pour $\Vert . \Vert_p$.
		\end{enumerate}
	\end{theorem}

	\begin{proof}
		\begin{enumerate}[label=(\roman*)]
			\item On suppose $f$ continue.
			\begin{itemize}
				\item Sur l'intervalle compact $[0, 2\pi]$, $f$ est bornée et atteint ses bornes. En particulier, $\Vert f \Vert_\infty$ est bien définie. De plus, si $x \in \mathbb{R}$, par le \cref{theoreme-de-fejer-2} on a :
				\[ \sigma_N(f)(x) = (f * K_N)(x) \]
				Donc,
				\[ \vert \sigma_N(f)(x) \vert \leq \Vert f \Vert_\infty \underbrace{\Vert K_N \Vert_1}_{=1} = \Vert f \Vert_\infty \]
				d'où $\sigma_N(f)$ est bornée avec
				\[ \Vert \sigma_N(f) \Vert_\infty \leq \Vert f \Vert_\infty \]
				\item Soit $\delta \in ]0,\pi]$. Posons
				\[ \omega(\delta) = \sup_{\vert u - v \vert \leq \delta} \{ \vert f(u) - f(v) \vert \} \]
				le module de continuité de $f$. Pour tout $x \in \mathbb{R}$, on a :
				\begin{align*}
					f(x) - \sigma_N(f)(x) &= f(x) - (f * K_N)(x) \\
					&= f(x) - \frac{1}{2\pi} \int_{-\pi}^{\pi} f(x-t) K_N(t) \, \mathrm{d}t \\
					&= \frac{1}{2\pi} \int_{-\pi}^{\pi} (f(x) - f(x-t)) K_N(t) \, \mathrm{d}t \tag{*}
				\end{align*}
				d'où
				\begin{align*}
					\vert f(x) - \sigma_N(f)(x) \vert &\leq \frac{1}{2\pi} \int_{\vert t \vert \leq \delta} (f(x) - f(x-t)) K_N(t) \, \mathrm{d}t \\
					&\quad + \frac{1}{2\pi} \int_{\delta \leq \vert t \vert \leq \pi} (f(x) - f(x-t)) K_N(t) \, \mathrm{d}t \\
					&\leq \frac{\omega(\delta)}{2\pi} \int_{\vert t \vert \leq \delta} K_N(t) \, \mathrm{d}t + 2 \Vert f \Vert_\infty \frac{1}{2\pi} \int_{\delta \leq \vert t \vert \leq \pi} K_N(t) \, \mathrm{d}t \\
					&\leq \frac{\omega(\delta)}{2\pi} \int_{-\pi}^{\pi} K_N(t) \, \mathrm{d}t + \frac{2 \Vert f \Vert_\infty}{N \sin \left(\frac{\delta}{2} \right)^2} \\
					&= \omega(\delta) + \frac{2 \Vert f \Vert_\infty}{N \sin \left(\frac{\delta}{2} \right)^2}
				\end{align*}
				Donc, on a :
				\[ \Vert f(x) - \sigma_N(f)(x) \Vert_\infty \leq \omega(\delta) + \frac{2 \Vert f \Vert_\infty}{N \sin \left(\frac{\delta}{2} \right)^2} \tag{**} \]
				On peut passer à la limite supérieure dans $(**)$ pour obtenir :
				\[ \limsup_{N \rightarrow +\infty} \Vert f(x) - \sigma_N(f)(x) \Vert_\infty \leq \omega(\delta) \]
				Comme $f$ est continue sur le compact $[0,2\pi]$, elle y est uniformément continue par le théorème de Heine :
				\[ \forall \epsilon>0, \exists \delta>0, \forall(x,y)\in[0,2\pi]^2, |x-y|<\delta\Rightarrow |f(x)-f(y)|<\epsilon \]
				On peut donc faire tendre $\delta$ vers $0$ pour obtenir $\limsup_{N \rightarrow +\infty} \Vert f(x) - \sigma_N(f)(x) \Vert_\infty \leq 0$
				ie.
				\[ \limsup_{N \rightarrow +\infty} \Vert f(x) - \sigma_N(f)(x) \Vert_\infty = 0 \]
				Comme,
				\[ 0 \leq \liminf_{N \rightarrow +\infty} \Vert f(x) - \sigma_N(f)(x) \Vert_\infty \leq \limsup_{N \rightarrow +\infty} \Vert f(x) - \sigma_N(f)(x) \Vert_\infty = 0 \]
				On a bien,
				\[ \lim_{N \rightarrow +\infty} \Vert f(x) - \sigma_N(f)(x) \Vert_\infty = 0 \]
			\end{itemize}
			\item \begin{itemize}
				\item Par le \cref{theoreme-de-fejer-2}, on a :
				\[ \forall x \in \mathbb{R}, \, \vert \sigma_N(f)(x) \vert^p = \left \vert \frac{1}{2\pi} \int_{-\pi}^{\pi} f(x-t) K_N(t) \, \mathrm{d}t \right \vert^p \]
				On applique l'inégalité de Hölder à $\frac{K_N}{2\pi}$ :
				\[ \forall x \in \mathbb{R}, \, \vert \sigma_N(f)(x) \vert^p \leq \frac{1}{2\pi} \int_{-\pi}^{\pi} \vert f(x-t) \vert^p K_N(t) \, \mathrm{d}t \]
				Enfin, en intégrant par parties et en utilisant le théorème de Fubini-Tonelli :
				\begin{align*}
					\Vert \sigma_N(f) \Vert_p^p &\leq \frac{1}{4\pi^2} \int_{-\pi}^{\pi} K_N(t) \left( \vert f(x-t) \vert^p \, \mathrm{d}x \right) \, \mathrm{d}t \\
					&= \frac{1}{4\pi^2} \int_{-\pi}^{\pi} K_N(t) \left( \vert f(x) \vert^p \, \mathrm{d}x \right) \, \mathrm{d}t \\
					&= \Vert K_N \Vert_1 \Vert f \Vert_p^p \\
					&= \Vert f \Vert_p^p
				\end{align*}
				\item Par $(*)$ :
				\[ \Vert \sigma_N (f) - f \Vert_p^p \leq \frac{1}{4\pi^2} \int_{-\pi}^{\pi} K_N(t) \left( \vert f(x-t) \vert^p \, \mathrm{d}x \right) \, \mathrm{d}t \]
				En posant $g : t \mapsto \Vert f - \tau_t f \Vert_p^p$ (où $\tau$ est l'opérateur de translation) :
				\begin{align*}
					\Vert \sigma_N (f) - f \Vert_p^p &\leq \frac{1}{2\pi} \int_{-\pi}^{\pi} K_N(t) g(-t) \, \mathrm{d}t \\
					&= (g * K_N)(0) \\
					&= \sigma_N(g)(0)
				\end{align*}
				Comme $g$ est continue et $2\pi$-périodique, on a par le point précédent
				\[ \sigma_N(g)(0) \longrightarrow_{\rightarrow +\infty} g(0) = 0 \]
				Donc, on a bien,
				\[ \lim_{N \rightarrow +\infty} \Vert \sigma_N (f) - f \Vert_p = 0 \]
			\end{itemize}
		\end{enumerate}
	\end{proof}

	\begin{remark}
		Dans ce développement, il est courant de ne prouver que le premier point.
	\end{remark}
	%</content>
\end{document}
