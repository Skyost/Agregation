\documentclass[12pt, a4paper]{report}

% LuaLaTeX :

\RequirePackage{iftex}
\RequireLuaTeX

% Packages :

\usepackage[french]{babel}
%\usepackage[utf8]{inputenc}
%\usepackage[T1]{fontenc}
\usepackage[pdfencoding=auto, pdfauthor={Hugo Delaunay}, pdfsubject={Mathématiques}, pdfcreator={agreg.skyost.eu}]{hyperref}
\usepackage{amsmath}
\usepackage{amsthm}
%\usepackage{amssymb}
\usepackage{stmaryrd}
\usepackage{tikz}
\usepackage{tkz-euclide}
\usepackage{fourier-otf}
\usepackage{fontspec}
\usepackage{titlesec}
\usepackage{fancyhdr}
\usepackage{catchfilebetweentags}
\usepackage[french, capitalise, noabbrev]{cleveref}
\usepackage[fit, breakall]{truncate}
\usepackage[top=2.5cm, right=2cm, bottom=2.5cm, left=2cm]{geometry}
\usepackage{enumerate}
\usepackage{tocloft}
\usepackage{microtype}
%\usepackage{mdframed}
%\usepackage{thmtools}
\usepackage{xcolor}
\usepackage{tabularx}
\usepackage{aligned-overset}
\usepackage[subpreambles=true]{standalone}
\usepackage{environ}
\usepackage[normalem]{ulem}
\usepackage{marginnote}
\usepackage{etoolbox}
\usepackage{setspace}
\usepackage[bibstyle=reading, citestyle=draft]{biblatex}
\usepackage{xpatch}
\usepackage[many, breakable]{tcolorbox}
\usepackage[backgroundcolor=white, bordercolor=white, textsize=small]{todonotes}

% Bibliographie :

\newcommand{\overridebibliographypath}[1]{\providecommand{\bibliographypath}{#1}}
\overridebibliographypath{../bibliography.bib}
\addbibresource{\bibliographypath}
\defbibheading{bibliography}[\bibname]{%
	\newpage
	\section*{#1}%
}
\renewbibmacro*{entryhead:full}{\printfield{labeltitle}}
\DeclareFieldFormat{url}{\newline\footnotesize\url{#1}}
\AtEndDocument{\printbibliography}

% Police :

\setmathfont{Erewhon Math}

% Tikz :

\usetikzlibrary{calc}

% Longueurs :

\setlength{\parindent}{0pt}
\setlength{\headheight}{15pt}
\setlength{\fboxsep}{0pt}
\titlespacing*{\chapter}{0pt}{-20pt}{10pt}
\setlength{\marginparwidth}{1.5cm}
\setstretch{1.1}

% Métadonnées :

\author{agreg.skyost.eu}
\date{\today}

% Titres :

\setcounter{secnumdepth}{3}

\renewcommand{\thechapter}{\Roman{chapter}}
\renewcommand{\thesubsection}{\Roman{subsection}}
\renewcommand{\thesubsubsection}{\arabic{subsubsection}}
\renewcommand{\theparagraph}{\alph{paragraph}}

\titleformat{\chapter}{\huge\bfseries}{\thechapter}{20pt}{\huge\bfseries}
\titleformat*{\section}{\LARGE\bfseries}
\titleformat{\subsection}{\Large\bfseries}{\thesubsection \, - \,}{0pt}{\Large\bfseries}
\titleformat{\subsubsection}{\large\bfseries}{\thesubsubsection. \,}{0pt}{\large\bfseries}
\titleformat{\paragraph}{\bfseries}{\theparagraph. \,}{0pt}{\bfseries}

\setcounter{secnumdepth}{4}

% Table des matières :

\renewcommand{\cftsecleader}{\cftdotfill{\cftdotsep}}
\addtolength{\cftsecnumwidth}{10pt}

% Redéfinition des commandes :

\renewcommand*\thesection{\arabic{section}}
\renewcommand{\ker}{\mathrm{Ker}}

% Nouvelles commandes :

\newcommand{\website}{https://agreg.skyost.eu}

\newcommand{\tr}[1]{\mathstrut ^t #1}
\newcommand{\im}{\mathrm{Im}}
\newcommand{\rang}{\operatorname{rang}}
\newcommand{\trace}{\operatorname{trace}}
\newcommand{\id}{\operatorname{id}}
\newcommand{\stab}{\operatorname{Stab}}

\providecommand{\newpar}{\\[\medskipamount]}

\providecommand{\lesson}[3]{%
	\title{#3}%
	\hypersetup{pdftitle={#3}}%
	\setcounter{section}{\numexpr #2 - 1}%
	\section{#3}%
	\fancyhead[R]{\truncate{0.73\textwidth}{#2 : #3}}%
}

\providecommand{\development}[3]{%
	\title{#3}%
	\hypersetup{pdftitle={#3}}%
	\section*{#3}%
	\fancyhead[R]{\truncate{0.73\textwidth}{#3}}%
}

\providecommand{\summary}[1]{%
	\textit{#1}%
	\medskip%
}

\tikzset{notestyleraw/.append style={inner sep=0pt, rounded corners=0pt, align=center}}

%\newcommand{\booklink}[1]{\website/bibliographie\##1}
\newcommand{\citelink}[2]{\hyperlink{cite.\therefsection @#1}{#2}}
\newcommand{\previousreference}{}
\providecommand{\reference}[2][]{%
	\notblank{#1}{\renewcommand{\previousreference}{#1}}{}%
	\todo[noline]{%
		\protect\vspace{16pt}%
		\protect\par%
		\protect\notblank{#1}{\cite{[\previousreference]}\\}{}%
		\protect\citelink{\previousreference}{p. #2}%
	}%
}

\definecolor{devcolor}{HTML}{00695c}
\newcommand{\dev}[1]{%
	\reversemarginpar%
	\todo[noline]{
		\protect\vspace{16pt}%
		\protect\par%
		\bfseries\color{devcolor}\href{\website/developpements/#1}{DEV}
	}%
	\normalmarginpar%
}

% En-têtes :

\pagestyle{fancy}
\fancyhead[L]{\truncate{0.23\textwidth}{\thepage}}
\fancyfoot[C]{\scriptsize \href{\website}{\texttt{agreg.skyost.eu}}}

% Couleurs :

\definecolor{property}{HTML}{fffde7}
\definecolor{proposition}{HTML}{fff8e1}
\definecolor{lemma}{HTML}{fff3e0}
\definecolor{theorem}{HTML}{fce4f2}
\definecolor{corollary}{HTML}{ffebee}
\definecolor{definition}{HTML}{ede7f6}
\definecolor{notation}{HTML}{f3e5f5}
\definecolor{example}{HTML}{e0f7fa}
\definecolor{cexample}{HTML}{efebe9}
\definecolor{application}{HTML}{e0f2f1}
\definecolor{remark}{HTML}{e8f5e9}
\definecolor{proof}{HTML}{e1f5fe}

% Théorèmes :

\theoremstyle{definition}
\newtheorem{theorem}{Théorème}

\newtheorem{property}[theorem]{Propriété}
\newtheorem{proposition}[theorem]{Proposition}
\newtheorem{lemma}[theorem]{Lemme}
\newtheorem{corollary}[theorem]{Corollaire}

\newtheorem{definition}[theorem]{Définition}
\newtheorem{notation}[theorem]{Notation}

\newtheorem{example}[theorem]{Exemple}
\newtheorem{cexample}[theorem]{Contre-exemple}
\newtheorem{application}[theorem]{Application}

\theoremstyle{remark}
\newtheorem{remark}[theorem]{Remarque}

\counterwithin*{theorem}{section}

\newcommand{\applystyletotheorem}[1]{
	\tcolorboxenvironment{#1}{
		enhanced,
		breakable,
		colback=#1!98!white,
		boxrule=0pt,
		boxsep=0pt,
		left=8pt,
		right=8pt,
		top=8pt,
		bottom=8pt,
		sharp corners,
		after=\par,
	}
}

\applystyletotheorem{property}
\applystyletotheorem{proposition}
\applystyletotheorem{lemma}
\applystyletotheorem{theorem}
\applystyletotheorem{corollary}
\applystyletotheorem{definition}
\applystyletotheorem{notation}
\applystyletotheorem{example}
\applystyletotheorem{cexample}
\applystyletotheorem{application}
\applystyletotheorem{remark}
\applystyletotheorem{proof}

% Environnements :

\NewEnviron{whitetabularx}[1]{%
	\renewcommand{\arraystretch}{2.5}
	\colorbox{white}{%
		\begin{tabularx}{\textwidth}{#1}%
			\BODY%
		\end{tabularx}%
	}%
}

% Maths :

\DeclareFontEncoding{FMS}{}{}
\DeclareFontSubstitution{FMS}{futm}{m}{n}
\DeclareFontEncoding{FMX}{}{}
\DeclareFontSubstitution{FMX}{futm}{m}{n}
\DeclareSymbolFont{fouriersymbols}{FMS}{futm}{m}{n}
\DeclareSymbolFont{fourierlargesymbols}{FMX}{futm}{m}{n}
\DeclareMathDelimiter{\VERT}{\mathord}{fouriersymbols}{152}{fourierlargesymbols}{147}


% Bibliographie :

\addbibresource{\bibliographypath}%
\defbibheading{bibliography}[\bibname]{%
	\newpage
	\section*{#1}%
}
\renewbibmacro*{entryhead:full}{\printfield{labeltitle}}%
\DeclareFieldFormat{url}{\newline\footnotesize\url{#1}}%

\AtEndDocument{\printbibliography}

\begin{document}
  %<*content>
  \development{analysis}{theoreme-de-fejer}{Théorème de Fejér}

  \summary{Dans ce développement, on montre le théorème de Fejér, qui assure la convergence de la série de Fourier d'une fonction au sens de Cesàro.}
  
  \begin{lemma}
    \label{theoreme-de-fejer-1}
    Soit $f : \mathbb{R} \rightarrow \mathbb{C}$ une fonction continue et $T$-périodique. Alors $f$ est uniformément continue sur $\mathbb{R}$.
  \end{lemma}
  
  \begin{proof}
    Le théorème de Heine implique la continuité uniforme de $f$ sur $[-T, 2T]$, ce qui s'écrit :
    \[ \forall \epsilon > 0, \exists \eta > 0 \text{ tel que } \forall x, y \in [-T, 2T], \, \vert x - y \vert < \eta \implies \vert f(x) - f(y) \vert < \epsilon \tag{$*$} \]
    Soit $\epsilon > 0$ et soit le $\eta > 0$ correspondant donné par $(*)$, que l'on peut supposer strictement inférieur à $T$. Soient $x, y \in \mathbb{R}$ tels que $\vert x - y \vert < \eta$. Il existe $k \in \mathbb{Z}$ tel que $x' = x + kT \in [0,T]$. Alors,
    \[ y' = y + kT \in [x'-\eta, x'+\eta] \subseteq [-T, 2T] \]
    Comme $\vert x' - y' \vert < \eta$, on en déduit
    \[ \vert f(x) - f(y) \vert = \vert f(x') - f(y') \vert < \epsilon \]
    ce qu'il fallait démontrer.
  \end{proof}

  \reference[GOU21]{306}

  \begin{notation}
    On note $\forall n \in \mathbb{Z}$, $e_n : x \mapsto e^{inx}$ et, pour toute fonction $f$ continue et $2\pi$-périodique, $c_n(f)$ son $n$-ième coefficient de Fourier.
  \end{notation}

  \begin{theorem}[Fejér]
    Soit $f : \mathbb{R} \rightarrow \mathbb{C}$ une fonction continue et $2\pi$-périodique. On note pour tout $n \in \mathbb{N}$, $S_n$ le $n$-ième terme de sa série de Fourier et
    \[ C_n = \frac{1}{n+1} \sum_{k=0}^{n} S_k \]
    la suite des moyennes de Cesàro correspondante. Alors $(C_n)$ converge uniformément vers $f$ sur $\mathbb{R}$.
  \end{theorem}

  \begin{proof}
    On commence par noter, pour tout $n \in \mathbb{N}$, $D_n = \sum_{k=-n}^n e_k$ et $F_n = \frac{1}{n+1} \sum_{k=0}^{n} D_k$ les noyaux de Dirichlet et de Fejér. Comme, pour tout $k \in \mathbb{Z}^*$, $\int_{-\pi}^{\pi} e_n(t) \, \mathrm{d}t = 0$, on a pour tout $n \in \mathbb{N}$,
    \[ \frac{1}{2\pi} \int_{-\pi}^{\pi} D_n(t) \, \mathrm{d}t = \frac{1}{2\pi} \int_{-\pi}^{\pi} e_0(t) \, \mathrm{d}t = 1 \]
    et donc,
    \[ \frac{1}{2\pi} \int_{-\pi}^{\pi} F_n(t) \, \mathrm{d}t = \frac{1}{n+1} \left( \sum_{k=0}^n \frac{1}{2\pi} \int_{-\pi}^{\pi} D_n(t) \, \mathrm{d}t \right) = 1 \tag{$*$} \]
    Calculons le noyau de Dirichlet. Soit $x \in \mathbb{R}\setminus 2\pi\mathbb{Z}$. On a pour tout $N \in \mathbb{N}$,
    \begin{align*}
      D_N(x) &= e^{-iNx} \sum_{n=0}^{2N} e^{inx} \\
      &= e^{-iNx} \frac{e^{(2N+1)ix} - 1}{e^{ix} - 1} \\
      &= e^{-iNx} \frac{e^{(2N+1) i\frac{x}{2}} \left ( e^{(2N+1) i\frac{x}{2}} - e^{-(2N+1) i\frac{x}{2}} \right )}{e^{i\frac{x}{2}} \left( e^{i\frac{x}{2}} - e^{-i \frac{x}{2}} \right)} \\
      &= \frac{2i \sin \left( \left( N + \frac{1}{2} \right) x \right)}{2i \sin \left ( \frac{x}{2} \right)} \\
      &= \frac{\sin \left( \left( N + \frac{1}{2} \right) x \right)}{\sin \left ( \frac{x}{2} \right)}
    \end{align*}
    D'où, pour tout $N \in \mathbb{N}^*$ :
    \begin{align*}
      NF_{N-1}&=\sum_{n=0}^{N-1}{D_n} \\
      &=\sum_{n=0}^{N}{\frac{\sin \left( \left( n + \frac{1}{2} \right) x \right)}{\sin \left ( \frac{x}{2} \right)}} \\
      &=\frac{1}{\sin \left ( \frac{x}{2} \right)}\operatorname{Im} \left( \sum_{n=0}^{N-1}{e^{i(n+\frac{1}{2})x}} \right) \\
      &=\frac{1}{\sin \left ( \frac{x}{2} \right)}\operatorname{Im} \left ( e^{\frac{ix}{2}}\frac{e^{iNx-1}}{e^{ix}-1} \right) \\
      &=\frac{1}{\sin \left ( \frac{x}{2} \right)} \operatorname{Im} \left ( e^{\frac{ix}{2}}\frac{e^{\frac{iNx}{2}}2i\sin \left( \frac{Nx}{2} \right)}{e^{\frac{ix}{2}}2i \sin \left ( \frac{x}{2} \right)} \right ) \\
      &=\frac{\sin \left( \frac{Nx}{2} \right)}{\sin \left ( \frac{x}{2} \right)^2}\operatorname{Im} \left(e^{\frac{iNx}{2}} \right) \\
      &=\frac{\sin \left( \frac{Nx}{2} \right)^2}{\sin \left ( \frac{x}{2} \right)^2} \tag{$**$}
    \end{align*}
    Maintenant, on remarque que pour tout $n \in \mathbb{N}$ et pour tout $x \in \mathbb{R}$,
    \[ S_n(x) = \frac{1}{2\pi} \sum_{k=0}^n \left( \int_{-\pi}^{\pi} f(t) e^{-ikt} \, \mathrm{d}t \right) e^{ikx} = \frac{1}{2\pi} \int_{-\pi}^{\pi} f(t) D_n(x-t) \, \mathrm{d}t = f * D_n \]
    donc $C_n = \frac{1}{2\pi} \int_{-\pi}^{\pi} f(t) F_n(x-t) \, \mathrm{d}t = f * F_n = F_n * f$ par commutativité du produit de convolution. Soit $\epsilon > 0$. Le \cref{theoreme-de-fejer-1} assure l'existence de $\eta \in ]0, \pi[$ tel que
    \[ \forall x, y \in \mathbb{R}, \, \vert x - y \vert < \eta \implies \vert f(x) - f(y) \vert < \epsilon \]
    De plus, $\vert f \vert$ est continue sur tous les compacts de la forme $[2k\pi, 2(k+1)\pi]$, on peut donc la majorer par un réel $M > 0$. Alors, pour tout $x \in \mathbb{R}$,
    \begin{align*}
      \vert f(x) - C_n(x) \vert &= \left\vert \frac{1}{2\pi} \int_{-\pi}^{\pi} f(x-t) F_n(t) \, \mathrm{d}t - f(x) \times \underbrace{\frac{1}{2\pi} \int_{-\pi}^{\pi} F_n(t) \, \mathrm{d}t}_{= 1 \text{ par } (*)} \right\vert \\
      &= \left\vert \frac{1}{2\pi} \int_{-\pi}^{\pi} (f(x-t) - f(x)) F_n(t) \, \mathrm{d}t \right\vert \\
      &\leq \frac{1}{2\pi} \int_{\eta \leq \vert t \vert \leq \pi} 2MF_n(t) \, \mathrm{d}t + \frac{1}{2\pi} \int_{-\eta}^{\eta} \epsilon F_n(t) \, \mathrm{d}t \\
      &\leq \frac{2M}{2\pi} \int_{\eta \leq \vert t \vert \leq \pi} F_n(t) \, \mathrm{d}t + \epsilon
    \end{align*}
    Or, $(**)$ montre que
    \[ \forall n \in \mathbb{N}, \, \forall x \in [-\pi, \pi] \text{ tel que } \vert x \vert > \eta, \text{ on a } \vert F_n(x) \vert \leq \frac{1}{(n+1) \sin\left( \frac{\eta}{2} \right)^2} \]
    donc $(F_n)$ converge uniformément vers $0$ sur $[-\pi, \pi] \setminus [-\eta, \eta]$. Il existe ainsi $N \in \mathbb{N}$ tel que
    \[ \forall n \geq N, \, \int_{\eta \leq \vert t \vert \leq \pi} F_n(t) \, \mathrm{d}t < \epsilon \]
    de sorte que
    \[ \forall n \geq N, \, \forall x \in \mathbb{R}, \, \vert f(x) - C_n(x) \vert \leq \left(\frac{M}{\pi} + 1\right) \epsilon \]
    D'où le résultat.
  \end{proof}

  \begin{remark}
    Je préfère la preuve de \cite{[GOU21]}, qui est plus ``clés en main''. Il est possible de passer les calculs des noyaux de Dirichlet et de Fejér dans un premier temps, puis de les montrer à la fin selon le temps restant.
  \end{remark}
  %</content>
\end{document}
