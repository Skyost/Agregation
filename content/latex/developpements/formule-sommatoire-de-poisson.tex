\documentclass[12pt, a4paper]{report}

% LuaLaTeX :

\RequirePackage{iftex}
\RequireLuaTeX

% Packages :

\usepackage[french]{babel}
%\usepackage[utf8]{inputenc}
%\usepackage[T1]{fontenc}
\usepackage[pdfencoding=auto, pdfauthor={Hugo Delaunay}, pdfsubject={Mathématiques}, pdfcreator={agreg.skyost.eu}]{hyperref}
\usepackage{amsmath}
\usepackage{amsthm}
%\usepackage{amssymb}
\usepackage{stmaryrd}
\usepackage{tikz}
\usepackage{tkz-euclide}
\usepackage{fourier-otf}
\usepackage{fontspec}
\usepackage{titlesec}
\usepackage{fancyhdr}
\usepackage{catchfilebetweentags}
\usepackage[french, capitalise, noabbrev]{cleveref}
\usepackage[fit, breakall]{truncate}
\usepackage[top=2.5cm, right=2cm, bottom=2.5cm, left=2cm]{geometry}
\usepackage{enumerate}
\usepackage{tocloft}
\usepackage{microtype}
%\usepackage{mdframed}
%\usepackage{thmtools}
\usepackage{xcolor}
\usepackage{tabularx}
\usepackage{aligned-overset}
\usepackage[subpreambles=true]{standalone}
\usepackage{environ}
\usepackage[normalem]{ulem}
\usepackage{marginnote}
\usepackage{etoolbox}
\usepackage{setspace}
\usepackage[bibstyle=reading, citestyle=draft]{biblatex}
\usepackage{xpatch}
\usepackage[many, breakable]{tcolorbox}
\usepackage[backgroundcolor=white, bordercolor=white, textsize=small]{todonotes}

% Bibliographie :

\newcommand{\overridebibliographypath}[1]{\providecommand{\bibliographypath}{#1}}
\overridebibliographypath{../bibliography.bib}
\addbibresource{\bibliographypath}
\defbibheading{bibliography}[\bibname]{%
	\newpage
	\section*{#1}%
}
\renewbibmacro*{entryhead:full}{\printfield{labeltitle}}
\DeclareFieldFormat{url}{\newline\footnotesize\url{#1}}
\AtEndDocument{\printbibliography}

% Police :

\setmathfont{Erewhon Math}

% Tikz :

\usetikzlibrary{calc}

% Longueurs :

\setlength{\parindent}{0pt}
\setlength{\headheight}{15pt}
\setlength{\fboxsep}{0pt}
\titlespacing*{\chapter}{0pt}{-20pt}{10pt}
\setlength{\marginparwidth}{1.5cm}
\setstretch{1.1}

% Métadonnées :

\author{agreg.skyost.eu}
\date{\today}

% Titres :

\setcounter{secnumdepth}{3}

\renewcommand{\thechapter}{\Roman{chapter}}
\renewcommand{\thesubsection}{\Roman{subsection}}
\renewcommand{\thesubsubsection}{\arabic{subsubsection}}
\renewcommand{\theparagraph}{\alph{paragraph}}

\titleformat{\chapter}{\huge\bfseries}{\thechapter}{20pt}{\huge\bfseries}
\titleformat*{\section}{\LARGE\bfseries}
\titleformat{\subsection}{\Large\bfseries}{\thesubsection \, - \,}{0pt}{\Large\bfseries}
\titleformat{\subsubsection}{\large\bfseries}{\thesubsubsection. \,}{0pt}{\large\bfseries}
\titleformat{\paragraph}{\bfseries}{\theparagraph. \,}{0pt}{\bfseries}

\setcounter{secnumdepth}{4}

% Table des matières :

\renewcommand{\cftsecleader}{\cftdotfill{\cftdotsep}}
\addtolength{\cftsecnumwidth}{10pt}

% Redéfinition des commandes :

\renewcommand*\thesection{\arabic{section}}
\renewcommand{\ker}{\mathrm{Ker}}

% Nouvelles commandes :

\newcommand{\website}{https://agreg.skyost.eu}

\newcommand{\tr}[1]{\mathstrut ^t #1}
\newcommand{\im}{\mathrm{Im}}
\newcommand{\rang}{\operatorname{rang}}
\newcommand{\trace}{\operatorname{trace}}
\newcommand{\id}{\operatorname{id}}
\newcommand{\stab}{\operatorname{Stab}}

\providecommand{\newpar}{\\[\medskipamount]}

\providecommand{\lesson}[3]{%
	\title{#3}%
	\hypersetup{pdftitle={#3}}%
	\setcounter{section}{\numexpr #2 - 1}%
	\section{#3}%
	\fancyhead[R]{\truncate{0.73\textwidth}{#2 : #3}}%
}

\providecommand{\development}[3]{%
	\title{#3}%
	\hypersetup{pdftitle={#3}}%
	\section*{#3}%
	\fancyhead[R]{\truncate{0.73\textwidth}{#3}}%
}

\providecommand{\summary}[1]{%
	\textit{#1}%
	\medskip%
}

\tikzset{notestyleraw/.append style={inner sep=0pt, rounded corners=0pt, align=center}}

%\newcommand{\booklink}[1]{\website/bibliographie\##1}
\newcommand{\citelink}[2]{\hyperlink{cite.\therefsection @#1}{#2}}
\newcommand{\previousreference}{}
\providecommand{\reference}[2][]{%
	\notblank{#1}{\renewcommand{\previousreference}{#1}}{}%
	\todo[noline]{%
		\protect\vspace{16pt}%
		\protect\par%
		\protect\notblank{#1}{\cite{[\previousreference]}\\}{}%
		\protect\citelink{\previousreference}{p. #2}%
	}%
}

\definecolor{devcolor}{HTML}{00695c}
\newcommand{\dev}[1]{%
	\reversemarginpar%
	\todo[noline]{
		\protect\vspace{16pt}%
		\protect\par%
		\bfseries\color{devcolor}\href{\website/developpements/#1}{DEV}
	}%
	\normalmarginpar%
}

% En-têtes :

\pagestyle{fancy}
\fancyhead[L]{\truncate{0.23\textwidth}{\thepage}}
\fancyfoot[C]{\scriptsize \href{\website}{\texttt{agreg.skyost.eu}}}

% Couleurs :

\definecolor{property}{HTML}{fffde7}
\definecolor{proposition}{HTML}{fff8e1}
\definecolor{lemma}{HTML}{fff3e0}
\definecolor{theorem}{HTML}{fce4f2}
\definecolor{corollary}{HTML}{ffebee}
\definecolor{definition}{HTML}{ede7f6}
\definecolor{notation}{HTML}{f3e5f5}
\definecolor{example}{HTML}{e0f7fa}
\definecolor{cexample}{HTML}{efebe9}
\definecolor{application}{HTML}{e0f2f1}
\definecolor{remark}{HTML}{e8f5e9}
\definecolor{proof}{HTML}{e1f5fe}

% Théorèmes :

\theoremstyle{definition}
\newtheorem{theorem}{Théorème}

\newtheorem{property}[theorem]{Propriété}
\newtheorem{proposition}[theorem]{Proposition}
\newtheorem{lemma}[theorem]{Lemme}
\newtheorem{corollary}[theorem]{Corollaire}

\newtheorem{definition}[theorem]{Définition}
\newtheorem{notation}[theorem]{Notation}

\newtheorem{example}[theorem]{Exemple}
\newtheorem{cexample}[theorem]{Contre-exemple}
\newtheorem{application}[theorem]{Application}

\theoremstyle{remark}
\newtheorem{remark}[theorem]{Remarque}

\counterwithin*{theorem}{section}

\newcommand{\applystyletotheorem}[1]{
	\tcolorboxenvironment{#1}{
		enhanced,
		breakable,
		colback=#1!98!white,
		boxrule=0pt,
		boxsep=0pt,
		left=8pt,
		right=8pt,
		top=8pt,
		bottom=8pt,
		sharp corners,
		after=\par,
	}
}

\applystyletotheorem{property}
\applystyletotheorem{proposition}
\applystyletotheorem{lemma}
\applystyletotheorem{theorem}
\applystyletotheorem{corollary}
\applystyletotheorem{definition}
\applystyletotheorem{notation}
\applystyletotheorem{example}
\applystyletotheorem{cexample}
\applystyletotheorem{application}
\applystyletotheorem{remark}
\applystyletotheorem{proof}

% Environnements :

\NewEnviron{whitetabularx}[1]{%
	\renewcommand{\arraystretch}{2.5}
	\colorbox{white}{%
		\begin{tabularx}{\textwidth}{#1}%
			\BODY%
		\end{tabularx}%
	}%
}

% Maths :

\DeclareFontEncoding{FMS}{}{}
\DeclareFontSubstitution{FMS}{futm}{m}{n}
\DeclareFontEncoding{FMX}{}{}
\DeclareFontSubstitution{FMX}{futm}{m}{n}
\DeclareSymbolFont{fouriersymbols}{FMS}{futm}{m}{n}
\DeclareSymbolFont{fourierlargesymbols}{FMX}{futm}{m}{n}
\DeclareMathDelimiter{\VERT}{\mathord}{fouriersymbols}{152}{fourierlargesymbols}{147}


% Bibliographie :

\addbibresource{\bibliographypath}%
\defbibheading{bibliography}[\bibname]{%
	\newpage
	\section*{#1}%
}
\renewbibmacro*{entryhead:full}{\printfield{labeltitle}}%
\DeclareFieldFormat{url}{\newline\footnotesize\url{#1}}%

\AtEndDocument{\printbibliography}

\begin{document}
	%<*content>
	\development{analysis}{formule-sommatoire-de-poisson}{Formule sommatoire de Poisson}

	\summary{On démontre la formule sommatoire de Poisson en utilisant principalement la théorie des séries de Fourier.}

	\reference[GOU20]{284}

	\begin{theorem}[Formule sommatoire de Poisson]
		\label{formule-sommatoire-de-poisson-1}
		Soit $f : \mathbb{R} \rightarrow \mathbb{C}$ une fonction de classe $\mathcal{C}^1$ telle que $f(x) = O \left( \frac{1}{x^2} \right)$ et $f'(x) = O \left( \frac{1}{x^2} \right)$ quand $|x| \longrightarrow +\infty$. Alors :
		\[ \forall x \in \mathbb{R}, \, \sum_{n \in \mathbb{Z}} f(x+n) = \sum_{n \in \mathbb{Z}} \widehat{f}(2 \pi n) e^{2 i \pi n x} \]
		où $\widehat{f}$ désigne la transformée de Fourier de $f$.
	\end{theorem}

	\begin{proof}
		Comme $f(x) = O \left( \frac{1}{x^2} \right)$, il existe $M > 0$ et $A > 0$ tel que
		\[ \forall |x| > A, \, |f(x)| \leq \frac{M}{x^2} \tag{$*$} \]
		Soit $K > 0$. On a $\forall x \in [-K, K]$, $\forall n \in \mathbb{Z}$ tel que $|n| > K + A$ :
		\[ |f(x+n)| \overset{(*)}{\leq} \frac{M}{(x+n)^2} \leq \frac{M}{(|n| - K)^2} \]
		Donc $\sum_{n \in \mathbb{Z}} f(x+n)$ converge normalement sur tout segment de $\mathbb{R}$ donc converge simplement sur $\mathbb{R}$. On note $F$ la limite simple en question.

		\medskip
		On montre de même que $\sum_{n \in \mathbb{Z}} f'(x+n)$ converge normalement sur tout segment de $\mathbb{R}$. Donc par le théorème de dérivation des suites de fonctions, $F$ est de classe $\mathcal{C}^1$ sur tout segment de $\mathbb{R}$, donc sur $\mathbb{R}$ tout entier (la continuité et la dérivabilité sont des propriétés locales).

		\medskip
		Soit $x \in \mathbb{R}$. On a :
		\begin{align*}
			\forall N \in \mathbb{N}, &\, \sum_{n=-N}^N f(x+1+n) = \sum_{n=-N-1}^{N+1} f(x+n) \\
			\overset{N \longrightarrow +\infty}{\implies} & F(x+1) = F(x)
		\end{align*}
		ie. $F$ est $1$-périodique. On peut calculer ses coefficients de Fourier. $\forall n \in \mathbb{Z}$,
		\[ c_n(F) = \int_0^1 F(t) e^{-2i\pi n t} \, \mathrm{d}t = \int_0^1 \sum_{n=-\infty}^{+\infty} f(t+n) e^{-2i\pi n t} \, \mathrm{d}t \]
		Par convergence uniforme sur un segment, on peut échanger somme et intégrale :
		\[ c_n(F) = \sum_{n=-\infty}^{+\infty} \int_n^{n+1} f(t) e^{-2i\pi n t} \, \mathrm{d}t \]
		Or, la transformée de Fourier d'une fonction $L_1$ est convergente. On peut donc écrire :
		\[ c_n(F) = \int_{-\infty}^{+\infty} f(t) e^{-2i\pi n t} \, \mathrm{d}t = \widehat{f}(2\pi n) \]
		Comme $F$ est de classe $\mathcal{C}^1$, sa série de Fourier converge uniformément vers $F$. D'où le résultat.
	\end{proof}

	\begin{application}
		\[ \forall s > 0, \, \sum_{n=-\infty}^{+\infty} e^{-\pi n^2 s} = \frac{1}{\sqrt{s}} \sum_{n=-\infty}^{+\infty} e^{-\frac{\pi n^2}{s}} \]
	\end{application}

	\begin{proof}
		Soit $\alpha > 0$. On définit $G_\alpha : x \mapsto e^{-\alpha x^2}$ et on connaît sa transformée de Fourier :
		\[ \forall \xi \in \mathbb{R}, \, \widehat{G_\alpha}(\xi) = \sqrt{\frac{\pi}{\alpha}} e^{-\frac{\xi^2}{4 \alpha}} \]
		Soit $s > 0$. Appliquons le \cref{formule-sommatoire-de-poisson-1} à la fonction $G_{\pi s}$ :
		\begin{align*}
			& \sum_{n \in \mathbb{Z}} e^{-\pi s (x + n)^2} = \frac{1}{\sqrt{s}} \sum_{n \in \mathbb{Z}} e^{-\frac{(2 \pi n)^2}{4 \pi s}} e^{2 i \pi n x} \\
			\overset{x = 0}{\implies} & \sum_{n \in \mathbb{Z}} e^{-\pi s n^2} = \frac{1}{\sqrt{s}} \sum_{n \in \mathbb{Z}} e^{-\frac{\pi n^2}{s}}
		\end{align*}
	\end{proof}
	%</content>
\end{document}
