\documentclass[12pt, a4paper]{report}

% LuaLaTeX :

\RequirePackage{iftex}
\RequireLuaTeX

% Packages :

\usepackage[french]{babel}
%\usepackage[utf8]{inputenc}
%\usepackage[T1]{fontenc}
\usepackage[pdfencoding=auto, pdfauthor={Hugo Delaunay}, pdfsubject={Mathématiques}, pdfcreator={agreg.skyost.eu}]{hyperref}
\usepackage{amsmath}
\usepackage{amsthm}
%\usepackage{amssymb}
\usepackage{stmaryrd}
\usepackage{tikz}
\usepackage{tkz-euclide}
\usepackage{fourier-otf}
\usepackage{fontspec}
\usepackage{titlesec}
\usepackage{fancyhdr}
\usepackage{catchfilebetweentags}
\usepackage[french, capitalise, noabbrev]{cleveref}
\usepackage[fit, breakall]{truncate}
\usepackage[top=2.5cm, right=2cm, bottom=2.5cm, left=2cm]{geometry}
\usepackage{enumerate}
\usepackage{tocloft}
\usepackage{microtype}
%\usepackage{mdframed}
%\usepackage{thmtools}
\usepackage{xcolor}
\usepackage{tabularx}
\usepackage{aligned-overset}
\usepackage[subpreambles=true]{standalone}
\usepackage{environ}
\usepackage[normalem]{ulem}
\usepackage{marginnote}
\usepackage{etoolbox}
\usepackage{setspace}
\usepackage[bibstyle=reading, citestyle=draft]{biblatex}
\usepackage{xpatch}
\usepackage[many, breakable]{tcolorbox}
\usepackage[backgroundcolor=white, bordercolor=white, textsize=small]{todonotes}

% Bibliographie :

\newcommand{\overridebibliographypath}[1]{\providecommand{\bibliographypath}{#1}}
\overridebibliographypath{../bibliography.bib}
\addbibresource{\bibliographypath}
\defbibheading{bibliography}[\bibname]{%
	\newpage
	\section*{#1}%
}
\renewbibmacro*{entryhead:full}{\printfield{labeltitle}}
\DeclareFieldFormat{url}{\newline\footnotesize\url{#1}}
\AtEndDocument{\printbibliography}

% Police :

\setmathfont{Erewhon Math}

% Tikz :

\usetikzlibrary{calc}

% Longueurs :

\setlength{\parindent}{0pt}
\setlength{\headheight}{15pt}
\setlength{\fboxsep}{0pt}
\titlespacing*{\chapter}{0pt}{-20pt}{10pt}
\setlength{\marginparwidth}{1.5cm}
\setstretch{1.1}

% Métadonnées :

\author{agreg.skyost.eu}
\date{\today}

% Titres :

\setcounter{secnumdepth}{3}

\renewcommand{\thechapter}{\Roman{chapter}}
\renewcommand{\thesubsection}{\Roman{subsection}}
\renewcommand{\thesubsubsection}{\arabic{subsubsection}}
\renewcommand{\theparagraph}{\alph{paragraph}}

\titleformat{\chapter}{\huge\bfseries}{\thechapter}{20pt}{\huge\bfseries}
\titleformat*{\section}{\LARGE\bfseries}
\titleformat{\subsection}{\Large\bfseries}{\thesubsection \, - \,}{0pt}{\Large\bfseries}
\titleformat{\subsubsection}{\large\bfseries}{\thesubsubsection. \,}{0pt}{\large\bfseries}
\titleformat{\paragraph}{\bfseries}{\theparagraph. \,}{0pt}{\bfseries}

\setcounter{secnumdepth}{4}

% Table des matières :

\renewcommand{\cftsecleader}{\cftdotfill{\cftdotsep}}
\addtolength{\cftsecnumwidth}{10pt}

% Redéfinition des commandes :

\renewcommand*\thesection{\arabic{section}}
\renewcommand{\ker}{\mathrm{Ker}}

% Nouvelles commandes :

\newcommand{\website}{https://agreg.skyost.eu}

\newcommand{\tr}[1]{\mathstrut ^t #1}
\newcommand{\im}{\mathrm{Im}}
\newcommand{\rang}{\operatorname{rang}}
\newcommand{\trace}{\operatorname{trace}}
\newcommand{\id}{\operatorname{id}}
\newcommand{\stab}{\operatorname{Stab}}

\providecommand{\newpar}{\\[\medskipamount]}

\providecommand{\lesson}[3]{%
	\title{#3}%
	\hypersetup{pdftitle={#3}}%
	\setcounter{section}{\numexpr #2 - 1}%
	\section{#3}%
	\fancyhead[R]{\truncate{0.73\textwidth}{#2 : #3}}%
}

\providecommand{\development}[3]{%
	\title{#3}%
	\hypersetup{pdftitle={#3}}%
	\section*{#3}%
	\fancyhead[R]{\truncate{0.73\textwidth}{#3}}%
}

\providecommand{\summary}[1]{%
	\textit{#1}%
	\medskip%
}

\tikzset{notestyleraw/.append style={inner sep=0pt, rounded corners=0pt, align=center}}

%\newcommand{\booklink}[1]{\website/bibliographie\##1}
\newcommand{\citelink}[2]{\hyperlink{cite.\therefsection @#1}{#2}}
\newcommand{\previousreference}{}
\providecommand{\reference}[2][]{%
	\notblank{#1}{\renewcommand{\previousreference}{#1}}{}%
	\todo[noline]{%
		\protect\vspace{16pt}%
		\protect\par%
		\protect\notblank{#1}{\cite{[\previousreference]}\\}{}%
		\protect\citelink{\previousreference}{p. #2}%
	}%
}

\definecolor{devcolor}{HTML}{00695c}
\newcommand{\dev}[1]{%
	\reversemarginpar%
	\todo[noline]{
		\protect\vspace{16pt}%
		\protect\par%
		\bfseries\color{devcolor}\href{\website/developpements/#1}{DEV}
	}%
	\normalmarginpar%
}

% En-têtes :

\pagestyle{fancy}
\fancyhead[L]{\truncate{0.23\textwidth}{\thepage}}
\fancyfoot[C]{\scriptsize \href{\website}{\texttt{agreg.skyost.eu}}}

% Couleurs :

\definecolor{property}{HTML}{fffde7}
\definecolor{proposition}{HTML}{fff8e1}
\definecolor{lemma}{HTML}{fff3e0}
\definecolor{theorem}{HTML}{fce4f2}
\definecolor{corollary}{HTML}{ffebee}
\definecolor{definition}{HTML}{ede7f6}
\definecolor{notation}{HTML}{f3e5f5}
\definecolor{example}{HTML}{e0f7fa}
\definecolor{cexample}{HTML}{efebe9}
\definecolor{application}{HTML}{e0f2f1}
\definecolor{remark}{HTML}{e8f5e9}
\definecolor{proof}{HTML}{e1f5fe}

% Théorèmes :

\theoremstyle{definition}
\newtheorem{theorem}{Théorème}

\newtheorem{property}[theorem]{Propriété}
\newtheorem{proposition}[theorem]{Proposition}
\newtheorem{lemma}[theorem]{Lemme}
\newtheorem{corollary}[theorem]{Corollaire}

\newtheorem{definition}[theorem]{Définition}
\newtheorem{notation}[theorem]{Notation}

\newtheorem{example}[theorem]{Exemple}
\newtheorem{cexample}[theorem]{Contre-exemple}
\newtheorem{application}[theorem]{Application}

\theoremstyle{remark}
\newtheorem{remark}[theorem]{Remarque}

\counterwithin*{theorem}{section}

\newcommand{\applystyletotheorem}[1]{
	\tcolorboxenvironment{#1}{
		enhanced,
		breakable,
		colback=#1!98!white,
		boxrule=0pt,
		boxsep=0pt,
		left=8pt,
		right=8pt,
		top=8pt,
		bottom=8pt,
		sharp corners,
		after=\par,
	}
}

\applystyletotheorem{property}
\applystyletotheorem{proposition}
\applystyletotheorem{lemma}
\applystyletotheorem{theorem}
\applystyletotheorem{corollary}
\applystyletotheorem{definition}
\applystyletotheorem{notation}
\applystyletotheorem{example}
\applystyletotheorem{cexample}
\applystyletotheorem{application}
\applystyletotheorem{remark}
\applystyletotheorem{proof}

% Environnements :

\NewEnviron{whitetabularx}[1]{%
	\renewcommand{\arraystretch}{2.5}
	\colorbox{white}{%
		\begin{tabularx}{\textwidth}{#1}%
			\BODY%
		\end{tabularx}%
	}%
}

% Maths :

\DeclareFontEncoding{FMS}{}{}
\DeclareFontSubstitution{FMS}{futm}{m}{n}
\DeclareFontEncoding{FMX}{}{}
\DeclareFontSubstitution{FMX}{futm}{m}{n}
\DeclareSymbolFont{fouriersymbols}{FMS}{futm}{m}{n}
\DeclareSymbolFont{fourierlargesymbols}{FMX}{futm}{m}{n}
\DeclareMathDelimiter{\VERT}{\mathord}{fouriersymbols}{152}{fourierlargesymbols}{147}


% Bibliographie :

\addbibresource{\bibliographypath}%
\defbibheading{bibliography}[\bibname]{%
	\newpage
	\section*{#1}%
}
\renewbibmacro*{entryhead:full}{\printfield{labeltitle}}%
\DeclareFieldFormat{url}{\newline\footnotesize\url{#1}}%

\AtEndDocument{\printbibliography}

\begin{document}
	%<*content>
	\development{analysis}{caracterisation-reelle-de-gamma}{Caractérisation réelle de la fonction \texorpdfstring{$\Gamma$}{Gamma}}

	\summary{On montre que la fonction $\Gamma$ d'Euler est la seule fonction log-convexe sur $\mathbb{R}^+$ prenant la valeur $1$ en $1$ et vérifiant $\Gamma(x+1) = x\Gamma(x)$ pour tout $x > 0$.}

	\reference[ROM19-1]{364}

	\begin{lemma}
		\label{caracterisation-reelle-de-gamma-1}
		La fonction $\Gamma$ définie pour tout $x > 0$ par $\Gamma(x) = \int_0^{+\infty} t^{x-1} e^{-t} \, \mathrm{d}t$ vérifie :
		\begin{enumerate}[label=(\roman*)]
			\item \label{caracterisation-reelle-de-gamma-2} $\forall x \in \mathbb{R}^+_*$, $\Gamma(x+1) = x\Gamma(x)$.
			\item \label{caracterisation-reelle-de-gamma-3} $\Gamma(1) = 1$.
			\item \label{caracterisation-reelle-de-gamma-4} $\Gamma$ est log-convexe sur $\mathbb{R}^+_*$.
		\end{enumerate}
	\end{lemma}

	\begin{proof}
		\begin{enumerate}[label=(\roman*)]
			\item Soit $x \in \mathbb{R}^+_*$. Alors :
			\begin{align*}
				\Gamma(x+1) &= \int_0^{+\infty} t^x e^{-t} \, \mathrm{d}t \\
				&= \left[ e^{-t} t^x \right]_0^{+\infty} + x \int_0^{+\infty} t^{x-1} e^{-t} \, \mathrm{d}t \\
				&= x\Gamma(x)
			\end{align*}
			\item Comme $t \mapsto e^{-t} \mathbb{1}_{\mathbb{R}^+}(t)$ est la densité de probabilité d'une loi exponentielle de paramètre $1$, on a
			\[ \underbrace{\int_0^{+\infty} e^{-t} \, \mathrm{d}t}_{= \Gamma(1)} = 1 \]
			\item Soient $x, y \in \mathbb{R}^+_*$ et $\lambda \in ]0, 1[$. On applique l'inégalité de Hölder en posant $\lambda = \frac{1}{p}$ et $1-\lambda = \frac{1}{q}$ :
			\begin{align*}
				\Gamma(\lambda x + (1-\lambda) y) &= \int_0^{+\infty} e^{-t} t^{\lambda x} t^{(1-\lambda)y} \, \mathrm{d}t \\
				&= \int_0^{+\infty} (e^{-t} t^{x-1})^{\frac{1}{p}} (e^{-t} t^{y-1})^{\frac{1}{q}} \, \mathrm{d}t \\
				&\leq \left (\int_0^{+\infty} e^{-t} t^{x-1} \right)^{\frac{1}{p}} \left (\int_0^{+\infty} e^{-t} t^{y-1} \right)^{\frac{1}{q}} \\
				&= \Gamma(x)^\lambda \Gamma(y)^{1-\lambda}
			\end{align*}
			Donc $\ln \circ \Gamma$ vérifie bien l'inégalité de convexité sur $\mathbb{R}^+_*$ et ainsi, $\Gamma$ est log-convexe.
		\end{enumerate}
	\end{proof}

	\reference[RUD]{94}

	\begin{theorem}[Bohr-Mollerup]
		Soit $f : \mathbb{R}^+_* \rightarrow \mathbb{R}^+$ vérifiant le \cref{caracterisation-reelle-de-gamma-2}, le \cref{caracterisation-reelle-de-gamma-3} et le \cref{caracterisation-reelle-de-gamma-4} du \cref{caracterisation-reelle-de-gamma-1}. Alors $f = \Gamma$.
	\end{theorem}

	\begin{proof}
		Par récurrence, on a d'après le \cref{caracterisation-reelle-de-gamma-2} :
		\[ \forall n \in \mathbb{N}^*, \forall x \in ]0, 1], \, f(x+n) = (x+n-1) \dots (x+1)xf(x) \tag{$*$} \]
		Donc les valeurs prises par $f$ sur $\mathbb{R}^+_*$ sont entièrement déterminées par ses valeurs prises sur $]0, 1]$. Ainsi, pour démontrer le théorème, il suffit de vérifier $\forall x \in ]0, 1]$, $f(x) = \Gamma(x)$.
		\newpar
		Soient donc $x \in ]0, 1]$ et $n \in \mathbb{N}^*$ ; on applique le lemme des trois pentes à la fonction convexe $\ln \circ f$ (d'après le \cref{caracterisation-reelle-de-gamma-4} appliqué aux points $n-1$, $n$, $n+x$ et $n+1$ :
		\begin{center}
			\begin{tikzpicture}
				\draw (0.15,4) .. controls (1,1) and (4,0) .. (6,2) coordinate[pos=0.95] (F) coordinate[pos=0.05] (A) coordinate[pos=0.5] (B) coordinate[pos=0.75] (C) coordinate[pos=0.85] (D);
				\draw(A) node {$\bullet$} node[below left]{$n-1$};
				\draw(B) node {$\bullet$} node[below left]{$n$};
				\draw(C) node {$\bullet$} node[below]{$n+x$};
				\draw(D) node {$\bullet$} node[below right]{$n+1$};
				\draw(F) node[right,shift={(0.25,0)}]{$f(x)$};
				\draw (A) -- (B);
				\draw (B) -- (C);
				\draw (B) -- (D);
			\end{tikzpicture}
		\end{center}
		\[ \frac{(\ln \circ f)(n) - (\ln \circ f)(n-1)}{n - (n-1)} \leq \frac{(\ln \circ f)(n+x) - (\ln \circ f)(n)}{n+x-n} \leq \frac{(\ln \circ f)(n+1) - (\ln \circ f)(n)}{n+1-n} \]
		Mais, d'après $(*)$ et le \cref{caracterisation-reelle-de-gamma-3}, on a $f(n) = (n-1)!$. D'où :
		\begin{align*}
			&\ln(n-1) \leq \frac{(\ln \circ f)(n+x) - \ln((n-1)!)}{x} \leq \ln(n) \\
			\implies &\ln((n-1)^x) \leq (\ln \circ f)(n+x) - \ln((n-1)!) \leq \ln(n^x) \\
			\implies &\ln((n-1)^x (n-1)!) \leq (\ln \circ f)(n+x) \leq \ln(n^x(n-1)!) \\
		\end{align*}
		Par croissance de la fonction $\ln$, cela donne :
		\[ (n-1)^x (n-1)! \leq f(n+x) \leq n^x (n-1)! \]
		Et en appliquant $(*)$, on obtient :
		\[ \frac{(n-1)^x (n-1)!}{(x+n-1) \dots (x+1)x} \leq f(x) \leq \frac{n^x (n-1)!}{(x+n-1) \dots (x+1)} \]
		En ne considérant que la première inégalité, on peut remplacer $n$ par $n+1$ (car les deux inégalités sont vraies pour tout $n \in \mathbb{N}^*$) :
		\[ \frac{n^x n!}{(x+n) \dots (x+1)x} \leq f(x) \]
		Or, $\frac{n^x (n-1)!}{(x+n-1) \dots (x+1)} = \frac{n^x n!}{(x+n) \dots (x+1)x} \frac{x+n}{n}$, donc :
		\begin{align*}
			&\frac{n^x n!}{(x+n) \dots (x+1)x} \leq f(x) \leq \frac{n^x n!}{(x+n) \dots (x+1)x} \frac{x+n}{n} \\
			\implies & f(x) \frac{n}{x+n} \leq \frac{n^x n!}{(x+n) \dots (x+1)x} \leq f(x) \\
			\implies & f(x) = \lim_{n \rightarrow +\infty} \frac{n^x n!}{(x+n) \dots (x+1)x}
		\end{align*}
		en faisant $n \longrightarrow +\infty$ dans la deuxième inégalité. Comme $\Gamma$ vérifie le \cref{caracterisation-reelle-de-gamma-2}, le \cref{caracterisation-reelle-de-gamma-3}, et le \cref{caracterisation-reelle-de-gamma-4} ; le raisonnement précédent est a fortiori vrai aussi pour $\Gamma$. Donc
		\[ \Gamma(x) = \lim_{n \rightarrow +\infty} \frac{n^x n!}{(x+n) \dots (x+1)x} = f(x)  \]
		ie. $f$ et $\Gamma$ coïncident bien sur $]0, 1]$.
	\end{proof}

	\begin{remark}
		À la fin de la preuve, on obtient une formule due à Gauss :
		\[ \forall x \in ]0, 1], \Gamma(x) = \lim_{n \rightarrow +\infty} \frac{n^x n!}{(x+n) \dots (x+1)x} \]
		que l'on peut aisément étendre à $\mathbb{R}^+_*$ entier.
	\end{remark}
	%</content>
\end{document}
