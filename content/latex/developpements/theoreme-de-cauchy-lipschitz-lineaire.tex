\documentclass[12pt, a4paper]{report}

% LuaLaTeX :

\RequirePackage{iftex}
\RequireLuaTeX

% Packages :

\usepackage[french]{babel}
%\usepackage[utf8]{inputenc}
%\usepackage[T1]{fontenc}
\usepackage[pdfencoding=auto, pdfauthor={Hugo Delaunay}, pdfsubject={Mathématiques}, pdfcreator={agreg.skyost.eu}]{hyperref}
\usepackage{amsmath}
\usepackage{amsthm}
%\usepackage{amssymb}
\usepackage{stmaryrd}
\usepackage{tikz}
\usepackage{tkz-euclide}
\usepackage{fourier-otf}
\usepackage{fontspec}
\usepackage{titlesec}
\usepackage{fancyhdr}
\usepackage{catchfilebetweentags}
\usepackage[french, capitalise, noabbrev]{cleveref}
\usepackage[fit, breakall]{truncate}
\usepackage[top=2.5cm, right=2cm, bottom=2.5cm, left=2cm]{geometry}
\usepackage{enumerate}
\usepackage{tocloft}
\usepackage{microtype}
%\usepackage{mdframed}
%\usepackage{thmtools}
\usepackage{xcolor}
\usepackage{tabularx}
\usepackage{aligned-overset}
\usepackage[subpreambles=true]{standalone}
\usepackage{environ}
\usepackage[normalem]{ulem}
\usepackage{marginnote}
\usepackage{etoolbox}
\usepackage{setspace}
\usepackage[bibstyle=reading, citestyle=draft]{biblatex}
\usepackage{xpatch}
\usepackage[many, breakable]{tcolorbox}
\usepackage[backgroundcolor=white, bordercolor=white, textsize=small]{todonotes}

% Bibliographie :

\newcommand{\overridebibliographypath}[1]{\providecommand{\bibliographypath}{#1}}
\overridebibliographypath{../bibliography.bib}
\addbibresource{\bibliographypath}
\defbibheading{bibliography}[\bibname]{%
	\newpage
	\section*{#1}%
}
\renewbibmacro*{entryhead:full}{\printfield{labeltitle}}
\DeclareFieldFormat{url}{\newline\footnotesize\url{#1}}
\AtEndDocument{\printbibliography}

% Police :

\setmathfont{Erewhon Math}

% Tikz :

\usetikzlibrary{calc}

% Longueurs :

\setlength{\parindent}{0pt}
\setlength{\headheight}{15pt}
\setlength{\fboxsep}{0pt}
\titlespacing*{\chapter}{0pt}{-20pt}{10pt}
\setlength{\marginparwidth}{1.5cm}
\setstretch{1.1}

% Métadonnées :

\author{agreg.skyost.eu}
\date{\today}

% Titres :

\setcounter{secnumdepth}{3}

\renewcommand{\thechapter}{\Roman{chapter}}
\renewcommand{\thesubsection}{\Roman{subsection}}
\renewcommand{\thesubsubsection}{\arabic{subsubsection}}
\renewcommand{\theparagraph}{\alph{paragraph}}

\titleformat{\chapter}{\huge\bfseries}{\thechapter}{20pt}{\huge\bfseries}
\titleformat*{\section}{\LARGE\bfseries}
\titleformat{\subsection}{\Large\bfseries}{\thesubsection \, - \,}{0pt}{\Large\bfseries}
\titleformat{\subsubsection}{\large\bfseries}{\thesubsubsection. \,}{0pt}{\large\bfseries}
\titleformat{\paragraph}{\bfseries}{\theparagraph. \,}{0pt}{\bfseries}

\setcounter{secnumdepth}{4}

% Table des matières :

\renewcommand{\cftsecleader}{\cftdotfill{\cftdotsep}}
\addtolength{\cftsecnumwidth}{10pt}

% Redéfinition des commandes :

\renewcommand*\thesection{\arabic{section}}
\renewcommand{\ker}{\mathrm{Ker}}

% Nouvelles commandes :

\newcommand{\website}{https://agreg.skyost.eu}

\newcommand{\tr}[1]{\mathstrut ^t #1}
\newcommand{\im}{\mathrm{Im}}
\newcommand{\rang}{\operatorname{rang}}
\newcommand{\trace}{\operatorname{trace}}
\newcommand{\id}{\operatorname{id}}
\newcommand{\stab}{\operatorname{Stab}}

\providecommand{\newpar}{\\[\medskipamount]}

\providecommand{\lesson}[3]{%
	\title{#3}%
	\hypersetup{pdftitle={#3}}%
	\setcounter{section}{\numexpr #2 - 1}%
	\section{#3}%
	\fancyhead[R]{\truncate{0.73\textwidth}{#2 : #3}}%
}

\providecommand{\development}[3]{%
	\title{#3}%
	\hypersetup{pdftitle={#3}}%
	\section*{#3}%
	\fancyhead[R]{\truncate{0.73\textwidth}{#3}}%
}

\providecommand{\summary}[1]{%
	\textit{#1}%
	\medskip%
}

\tikzset{notestyleraw/.append style={inner sep=0pt, rounded corners=0pt, align=center}}

%\newcommand{\booklink}[1]{\website/bibliographie\##1}
\newcommand{\citelink}[2]{\hyperlink{cite.\therefsection @#1}{#2}}
\newcommand{\previousreference}{}
\providecommand{\reference}[2][]{%
	\notblank{#1}{\renewcommand{\previousreference}{#1}}{}%
	\todo[noline]{%
		\protect\vspace{16pt}%
		\protect\par%
		\protect\notblank{#1}{\cite{[\previousreference]}\\}{}%
		\protect\citelink{\previousreference}{p. #2}%
	}%
}

\definecolor{devcolor}{HTML}{00695c}
\newcommand{\dev}[1]{%
	\reversemarginpar%
	\todo[noline]{
		\protect\vspace{16pt}%
		\protect\par%
		\bfseries\color{devcolor}\href{\website/developpements/#1}{DEV}
	}%
	\normalmarginpar%
}

% En-têtes :

\pagestyle{fancy}
\fancyhead[L]{\truncate{0.23\textwidth}{\thepage}}
\fancyfoot[C]{\scriptsize \href{\website}{\texttt{agreg.skyost.eu}}}

% Couleurs :

\definecolor{property}{HTML}{fffde7}
\definecolor{proposition}{HTML}{fff8e1}
\definecolor{lemma}{HTML}{fff3e0}
\definecolor{theorem}{HTML}{fce4f2}
\definecolor{corollary}{HTML}{ffebee}
\definecolor{definition}{HTML}{ede7f6}
\definecolor{notation}{HTML}{f3e5f5}
\definecolor{example}{HTML}{e0f7fa}
\definecolor{cexample}{HTML}{efebe9}
\definecolor{application}{HTML}{e0f2f1}
\definecolor{remark}{HTML}{e8f5e9}
\definecolor{proof}{HTML}{e1f5fe}

% Théorèmes :

\theoremstyle{definition}
\newtheorem{theorem}{Théorème}

\newtheorem{property}[theorem]{Propriété}
\newtheorem{proposition}[theorem]{Proposition}
\newtheorem{lemma}[theorem]{Lemme}
\newtheorem{corollary}[theorem]{Corollaire}

\newtheorem{definition}[theorem]{Définition}
\newtheorem{notation}[theorem]{Notation}

\newtheorem{example}[theorem]{Exemple}
\newtheorem{cexample}[theorem]{Contre-exemple}
\newtheorem{application}[theorem]{Application}

\theoremstyle{remark}
\newtheorem{remark}[theorem]{Remarque}

\counterwithin*{theorem}{section}

\newcommand{\applystyletotheorem}[1]{
	\tcolorboxenvironment{#1}{
		enhanced,
		breakable,
		colback=#1!98!white,
		boxrule=0pt,
		boxsep=0pt,
		left=8pt,
		right=8pt,
		top=8pt,
		bottom=8pt,
		sharp corners,
		after=\par,
	}
}

\applystyletotheorem{property}
\applystyletotheorem{proposition}
\applystyletotheorem{lemma}
\applystyletotheorem{theorem}
\applystyletotheorem{corollary}
\applystyletotheorem{definition}
\applystyletotheorem{notation}
\applystyletotheorem{example}
\applystyletotheorem{cexample}
\applystyletotheorem{application}
\applystyletotheorem{remark}
\applystyletotheorem{proof}

% Environnements :

\NewEnviron{whitetabularx}[1]{%
	\renewcommand{\arraystretch}{2.5}
	\colorbox{white}{%
		\begin{tabularx}{\textwidth}{#1}%
			\BODY%
		\end{tabularx}%
	}%
}

% Maths :

\DeclareFontEncoding{FMS}{}{}
\DeclareFontSubstitution{FMS}{futm}{m}{n}
\DeclareFontEncoding{FMX}{}{}
\DeclareFontSubstitution{FMX}{futm}{m}{n}
\DeclareSymbolFont{fouriersymbols}{FMS}{futm}{m}{n}
\DeclareSymbolFont{fourierlargesymbols}{FMX}{futm}{m}{n}
\DeclareMathDelimiter{\VERT}{\mathord}{fouriersymbols}{152}{fourierlargesymbols}{147}


% Bibliographie :

\addbibresource{\bibliographypath}%
\defbibheading{bibliography}[\bibname]{%
	\newpage
	\section*{#1}%
}
\renewbibmacro*{entryhead:full}{\printfield{labeltitle}}%
\DeclareFieldFormat{url}{\newline\footnotesize\url{#1}}%

\AtEndDocument{\printbibliography}

\begin{document}
  %<*content>
  \development{analysis}{theoreme-de-cauchy-lipschitz-lineaire}{Théorème de Cauchy-Lipschitz linéaire}

  \summary{En construisant un raisonnement autour du théorème du point fixe de Banach, on montre le théorème de Cauchy-Lipschitz, qui garantit l'existence d'une solution répondant à une condition initiale et l'unicité d'une solution maximale.}

  Soit $\mathbb{K} = \mathbb{R}$ ou $\mathbb{C}$.

  \begin{lemma}
    Soit $I$ un intervalle compact. L'espace $(\mathcal{C} (I, \mathbb{K}^d), \Vert . \Vert_{\infty})$ est complet.
  \end{lemma}

  \begin{proof}
    Soit $(f_n)$ une suite de Cauchy de $(\mathcal{C} (I, \mathbb{K}^d), \Vert . \Vert_{\infty})$. Soit $x \in I$, on a
    \[ \forall p, q \in \mathbb{N}, \, |f_p(x) - f_q(x)| \leq \Vert f_p - f_q \Vert_\infty \]
    donc $(f_n(x))$ est de Cauchy dans $\mathbb{K}$. Comme $\mathbb{K}$ est complet, la suite $(f_n(x))$ converge vers une limite notée $f(x)$. Ainsi, la suite de fonctions $(f_n)$ converge simplement vers la fonction $f : I \rightarrow \mathbb{K}$ nouvellement définie. Il reste à montrer que la fonction $f$ est continue.
    \newpar
    Notons déjà que $(f_n)$ est de Cauchy, et est en particulier bornée :
    \[ \exists M \geq 0 \text{ tel que } \Vert f_n \Vert_\infty \leq M \]
    donc en particulier, si $x \in I$, $|f_n(x)| \leq M$. Par passage à la limite, on obtient $|f(x)| \leq M$. Donc $f$ est bornée et écrire $\Vert f \Vert_{\infty}$ a bien du sens.
    \newpar
    Soit $\epsilon > 0$. Par définition,
    \[ \exists N \in \mathbb{N} \text{ tel que } \forall p, q \geq N, \Vert f_p - f_q \Vert_\infty < \epsilon \]
    Donc,
    \[ \forall x \in I, \, \forall p, q \geq N, |f_p(x) - f_q(x)| \leq \Vert f_p - f_q \Vert_\infty < \epsilon \]
    En faisant tendre $p$ vers l'infini, on obtient :
    \[ \forall x \in I, \, \forall q \geq N, |f(x) - f_q(x)| < \epsilon \]
    Nous venons d'écrire exactement la définition de la convergence uniforme ! Ainsi, $(f_n)$ est une suite de fonctions continues qui converge uniformément vers $f$, donc $f$ est continue.
  \end{proof}

  \reference[DAN]{520}

  \begin{theorem}[Cauchy-Lipschitz linéaire]
    Soient $A : I \rightarrow \mathcal{M}_d(\mathbb{K})$ et $B : I \rightarrow \mathbb{K}^d$ deux fonctions continues. Alors $\forall t_0 \in I$, le problème de Cauchy
    \[ \begin{cases} Y' = A(t)Y + B(t) \\ Y(t_0) = y_0 \end{cases} \tag{$C$} \]
    admet une unique solution définie sur $I$ tout entier.
  \end{theorem}

  \begin{proof}
    Commençons par supposer l'intervalle $I$ compact. On va montrer l'existence d'une solution globale. On écrit l'équation sous forme intégrale :
    \[ Y \in \mathcal{C}^1 \text{ vérifie } (C) \iff Y(t) = y_0 + \int_{t_0}^t A(u) Y(u) + B(u) \, \mathrm{d}u \tag{$*$} \]
    et on introduit la suite de fonctions $(Y_n)$ définie par récurrence sur $I$ par $Y_0 = y_0$ et :
    \[ \forall n \in \mathbb{N}^*, \, Y_{n+1}(t) = y_0 + \int_{t_0}^t A(u) Y_n(u) + B(u) \, \mathrm{d}u \tag{$**$} \]
    Notons $\alpha = \sup_{t \in I} \Vert A(t) \Vert$ et $\beta = \sup_{t \in I} \Vert B(t) \Vert$. Montrons par récurrence que pour tout $n \geq 1$ et tout $t \in I$ :
    \[ \Vert Y_n(t) - Y_{n-1}(t) \Vert \leq (\alpha \Vert y_0 \Vert + \beta) \frac{\alpha^{n-1} |t-t_0|^n}{n!} \]
    Le résultat est clairement vrai pour $n = 1$, supposons donc le vrai à rang $n \geq 1$. Pour $t \geq t_0$ :
    \begin{align*}
      \Vert Y_{n+1}(t) - Y_n(t) \Vert &= \left| \left| \int_{t_0}^t A(u) \times (Y_n(u) - Y_{n-1}(u)) \, \mathrm{d}u \right| \right| \\
      &\leq \alpha \int_{t_0}^t (\alpha \Vert y_0 \Vert + \beta) \frac{\alpha^{n-1} |u-t_0|^n}{n!} \mathrm{d}u \\
      &\leq (\alpha \Vert y_0 \Vert + \beta) \frac{\alpha^n |t-t_0|^{n+1}}{(n+1)!}
    \end{align*}
    et on procède de même pour $t \leq t_0$, ce qui achève la récurrence.
    \newpar
    Soit $L$ la longueur de $I$. On obtient donc :
    \[ \forall n \in \mathbb{N}^*, \, \Vert Y_n - Y_{n-1} \Vert_{\infty} \leq (\alpha \Vert y_0 \Vert + \beta) \frac{\alpha^{n-1}}{n!} L^n \]
    Il en résulte que la série de fonction $\sum (Y_n - Y_{n-1})$ est normalement convergente. Comme $(\mathcal{C} (I, \mathbb{K}^d), \Vert . \Vert_{\infty})$ est complet, la série est uniformément convergente. On a donc l'existence d'une fonction $Y \in \mathcal{C} (I, \mathbb{K}^d)$ telle que
    \[ \left \Vert \sum_{n=1}^N (Y_n - Y_{n-1}) - Y \right \Vert_\infty = \left \Vert Y_n - (Y + Y_0) \right \Vert_\infty \longrightarrow 0 \]
    ie. $(Y_n)$ converge vers $Y + Y_0 = Y + y_0 = Z$. Par convergence uniforme sur un intervalle compact, il est possible de passer à la limite dans $(**)$. D'où :
    \[ \forall t \in I, \, Z(t) = y_0 + \int_{t_0}^t A(u) Z(u) + B(u) \, \mathrm{d}u \]
    et comme $Z$ est continue, elle est $\mathcal{C}^1$ et vérifie donc bien $(*)$.
    \newpar
    On peut maintenant montrer l'unicité. Soient $Y$ et $Z$ deux solutions de $(C)$ sur $I$. Par récurrence sur l'entier $n$, on montre comme ci-dessus que pour tout $t \in I$ :
    \[ \Vert Y(t) - Z(t) \Vert \leq \frac{\alpha^n |t-t_0|^n}{n!} \Vert Y-Z \Vert_{\infty} \longrightarrow 0 \]
    donc $Y$ et $Z$ coïncident bien sur $I$.
    \newpar
    Supposons maintenant $I$ quelconque. Il existe donc $(K_n)$ une suite croissante d'intervalles compacts telle que $I = \bigcup_{n = 0}^{+\infty} K_n$. En particulier, on définit bien l'application
    \[
    \theta :
    \begin{array}{ccc}
      I &\rightarrow& \mathbb{K}^d \\
      t &\mapsto& Y_n(t)
    \end{array}
    \]
    (où $Y_n$ est la solution de $(C)$ sur $K_n \ni t$). En particulier, $\theta$ est dérivable sur $I$ tout entier, vérifie $(C)$, et prolonge toute solution.
    \begin{comment}
      Supposons maintenant $I$ quelconque. On définit bien l'application
      \[
      \theta :
      \begin{array}{ccc}
        I &\rightarrow& \mathbb{K}^d \\
        t &\mapsto& Y(t)
      \end{array}
      \]
      en prenant $Y$ la solution de $(C)$ sur $J_{\alpha,\beta} = [t-\alpha, t+\beta]$ avec $\alpha, \beta \geq 0$ de sorte que $J_{\alpha, \beta} \neq \emptyset$ et $J_{\alpha, \beta} \subseteq I$. En particulier, $\theta$ est dérivable sur $I$ tout entier, vérifie $(C)$, et prolonge toute solution.
    \end{comment}
  \end{proof}

  Selon la leçon, on pourra préférer le théorème suivant (dont la démonstration utilise des arguments semblables).

  \reference[GOU20]{374}

  \begin{theorem}[Cauchy-Lipschitz local]
    Soient $I$ un intervalle de $\mathbb{R}$ et $\Omega$ un ouvert de $E$. Soit $F : I \times \Omega \rightarrow E$ une fonction continue et localement lipschitzienne en la seconde variable. Alors, pour tout $(t_0, y_0) \in I \times \Omega$, le problème de Cauchy
    \[ \begin{cases} y'=F(t,y) \\ y(t_0) = y_0 \end{cases} \tag{$C$} \]
    admet une unique solution maximale.
  \end{theorem}

  \begin{proof}
    Nous commençons par montrer l'existence en $4$ étapes.
    \begin{itemize}
      \item \uline{Localisation :} Fixons un réel $r > 0$ tel que le produit $P = [t_0 - r, t_0 + r] \times \overline{B}(y_0, r)$ soit contenu dans $I \times \Omega$. $F$ est continue sur $P$ qui est compact, donc est bornée par $M$ sur $P$.
      \item \uline{Mise sous forme intégrale :} Comme une solution de $y' = F(t, y)$ est de ce fait $\mathcal{C}^1$, on a
      \[ y \in \mathcal{C}^1 \text{ vérifie } (C) \iff y(t) = y_0 + \int_{t_0}^t F(u, y(u)) \, \mathrm{d}u \tag{$*$} \]
      \item \uline{Choix d'un domaine stable :} Soit $\alpha \in ]0, r[$. Introduisons l'intervalle $I_\alpha = [t_0 - \alpha, t_0 + \alpha]$, l'espace $A_\alpha = \mathcal{C}(I_\alpha, \overline{B}(y_0, r))$, puis l'application
      \[
      \Psi :
      \begin{array}{ll}
        A_\alpha &\rightarrow \mathcal{C}(I_\alpha, E)  \\
        \varphi &\mapsto \left(t \mapsto y_0 + \int_{t_0}^t F(u, \varphi(u)) \, \mathrm{d}u \right)
      \end{array}
      \]
      Le problème est ici de rendre $A_\alpha$ stable par $\Psi$. Pour tout $t \in I_\alpha$,
      \begin{align*}
        &\Vert F(t, \varphi(t)) \Vert \leq M \\
        \implies& \Vert \Psi(\varphi)(t) - y_0 \Vert \leq M |t-t_0| \leq \alpha M
      \end{align*}
      Par suite, en choisissant $\alpha M < r$, le domaine $A_\alpha$ est stable par $\Psi$.
      \item \uline{Détermination d'un domaine de contraction :} Ici, $A_\alpha$ est normé par la norme $\Vert . \Vert_{\infty}$, et on veut faire de $\Psi$ une contraction stricte. Soient $\varphi, \phi \in A_\alpha$, par définition, pour tout $t \in I_\alpha$,
      \begin{align*}
        \Vert (\Psi(\varphi) - \Psi(\phi)) (t) \Vert &= \left \Vert \int_{t_0}^t (F(u, \varphi(u)) - F(u, \phi(u))) \, \mathrm{d}u \right \Vert \\
        &\leq k |t-t_0| \Vert \varphi - \phi \Vert_\infty \\
        &\leq k \alpha \Vert \varphi - \phi \Vert_\infty
      \end{align*}
      où $k$ désigne le rapport de lipschitziannité de $F$. On choisit désormais $\alpha$ tel que $k \alpha < 1$ et $\alpha M < r$.
      \item \uline{Conclusion :} L'application $\Psi$ est, par choix de $\alpha$, une contraction stricte de $(A_\alpha, \Vert . \Vert_{\infty})$ dans lui-même. Le fermé $\overline{B}(y_0, r)$ de l'espace de Banach de $E$ est complet, par suite $(A_\alpha, \Vert . \Vert_{\infty})$ l'est aussi.
      \newpar
      Par le théorème du point fixe de Banach, $\Psi$ possède donc un point fixe $\varphi$ dans $A_\alpha$. $\varphi$ est alors de classe $\mathcal{C}^1$ et vérifie $(C)$ par $(*)$.
    \end{itemize}
    \medskip
    Il reste maintenant à montrer l'unicité. On note $\mathcal{S}$ l'ensemble des solutions de $(C)$. $\mathcal{S} \neq \emptyset$, donc peut définir $J$ comme la réunion des intervalles de définition des solutions de $(C)$.
    \newpar
    Soient $\varphi, \phi \in \mathcal{S}$ (on note $K$ et $L$ leur intervalle de définition). Une récurrence sur $n$ donne
    \begin{align*}
      \forall t \in K \, \cap \, L, \, \forall n \in \mathbb{N}, \, \Vert \varphi(t) - \phi(t) \Vert &\leq \left| \int_{t_0}^t \Vert F(u, \varphi(u)) - F(u, \phi(u)) \Vert \, \mathrm{d}u \right| \\
      &\leq \frac{|t-t_0|^n}{n!} k^n \sup_{t \in K \, \cap \, L} |\varphi(t) - \phi(t)| \\
      &\longrightarrow 0
    \end{align*}
    Donc $\varphi$ et $\phi$ coïncident sur $K \, \cap \, L$.
    \newpar
    Ainsi, on définit correctement l'application
    \[
    \theta :
    \begin{array}{ccc}
      J &\rightarrow& E \\
      t &\mapsto& \phi(t)
    \end{array}
    \]
    (où $\phi \in S$ tel que $t$ est dans son intervalle de définition). Si $t \in J$, il existe $\phi \in \mathcal{S}$ tel que $t$ soit dans son intervalle de définition $K$. Comme $\phi$ et $\theta$ coïncident sur $K$, $\theta$ est dérivable sur $K$ et
    \[ \forall t \in K, \, \theta'(t) = \phi'(t) = F(t, \phi(t)) = F(t, \theta(t)) \]
    Et comme $\theta(t_0) = y_0$, $\theta \in \mathcal{S}$ et prolonge toute solution. Donc $\theta$ est maximale et est bien unique.
  \end{proof}
  %</content>
\end{document}
