\documentclass[12pt, a4paper]{report}

% LuaLaTeX :

\RequirePackage{iftex}
\RequireLuaTeX

% Packages :

\usepackage[french]{babel}
%\usepackage[utf8]{inputenc}
%\usepackage[T1]{fontenc}
\usepackage[pdfencoding=auto, pdfauthor={Hugo Delaunay}, pdfsubject={Mathématiques}, pdfcreator={agreg.skyost.eu}]{hyperref}
\usepackage{amsmath}
\usepackage{amsthm}
%\usepackage{amssymb}
\usepackage{stmaryrd}
\usepackage{tikz}
\usepackage{tkz-euclide}
\usepackage{fourier-otf}
\usepackage{fontspec}
\usepackage{titlesec}
\usepackage{fancyhdr}
\usepackage{catchfilebetweentags}
\usepackage[french, capitalise, noabbrev]{cleveref}
\usepackage[fit, breakall]{truncate}
\usepackage[top=2.5cm, right=2cm, bottom=2.5cm, left=2cm]{geometry}
\usepackage{enumerate}
\usepackage{tocloft}
\usepackage{microtype}
%\usepackage{mdframed}
%\usepackage{thmtools}
\usepackage{xcolor}
\usepackage{tabularx}
\usepackage{aligned-overset}
\usepackage[subpreambles=true]{standalone}
\usepackage{environ}
\usepackage[normalem]{ulem}
\usepackage{marginnote}
\usepackage{etoolbox}
\usepackage{setspace}
\usepackage[bibstyle=reading, citestyle=draft]{biblatex}
\usepackage{xpatch}
\usepackage[many, breakable]{tcolorbox}
\usepackage[backgroundcolor=white, bordercolor=white, textsize=small]{todonotes}

% Bibliographie :

\newcommand{\overridebibliographypath}[1]{\providecommand{\bibliographypath}{#1}}
\overridebibliographypath{../bibliography.bib}
\addbibresource{\bibliographypath}
\defbibheading{bibliography}[\bibname]{%
	\newpage
	\section*{#1}%
}
\renewbibmacro*{entryhead:full}{\printfield{labeltitle}}
\DeclareFieldFormat{url}{\newline\footnotesize\url{#1}}
\AtEndDocument{\printbibliography}

% Police :

\setmathfont{Erewhon Math}

% Tikz :

\usetikzlibrary{calc}

% Longueurs :

\setlength{\parindent}{0pt}
\setlength{\headheight}{15pt}
\setlength{\fboxsep}{0pt}
\titlespacing*{\chapter}{0pt}{-20pt}{10pt}
\setlength{\marginparwidth}{1.5cm}
\setstretch{1.1}

% Métadonnées :

\author{agreg.skyost.eu}
\date{\today}

% Titres :

\setcounter{secnumdepth}{3}

\renewcommand{\thechapter}{\Roman{chapter}}
\renewcommand{\thesubsection}{\Roman{subsection}}
\renewcommand{\thesubsubsection}{\arabic{subsubsection}}
\renewcommand{\theparagraph}{\alph{paragraph}}

\titleformat{\chapter}{\huge\bfseries}{\thechapter}{20pt}{\huge\bfseries}
\titleformat*{\section}{\LARGE\bfseries}
\titleformat{\subsection}{\Large\bfseries}{\thesubsection \, - \,}{0pt}{\Large\bfseries}
\titleformat{\subsubsection}{\large\bfseries}{\thesubsubsection. \,}{0pt}{\large\bfseries}
\titleformat{\paragraph}{\bfseries}{\theparagraph. \,}{0pt}{\bfseries}

\setcounter{secnumdepth}{4}

% Table des matières :

\renewcommand{\cftsecleader}{\cftdotfill{\cftdotsep}}
\addtolength{\cftsecnumwidth}{10pt}

% Redéfinition des commandes :

\renewcommand*\thesection{\arabic{section}}
\renewcommand{\ker}{\mathrm{Ker}}

% Nouvelles commandes :

\newcommand{\website}{https://agreg.skyost.eu}

\newcommand{\tr}[1]{\mathstrut ^t #1}
\newcommand{\im}{\mathrm{Im}}
\newcommand{\rang}{\operatorname{rang}}
\newcommand{\trace}{\operatorname{trace}}
\newcommand{\id}{\operatorname{id}}
\newcommand{\stab}{\operatorname{Stab}}

\providecommand{\newpar}{\\[\medskipamount]}

\providecommand{\lesson}[3]{%
	\title{#3}%
	\hypersetup{pdftitle={#3}}%
	\setcounter{section}{\numexpr #2 - 1}%
	\section{#3}%
	\fancyhead[R]{\truncate{0.73\textwidth}{#2 : #3}}%
}

\providecommand{\development}[3]{%
	\title{#3}%
	\hypersetup{pdftitle={#3}}%
	\section*{#3}%
	\fancyhead[R]{\truncate{0.73\textwidth}{#3}}%
}

\providecommand{\summary}[1]{%
	\textit{#1}%
	\medskip%
}

\tikzset{notestyleraw/.append style={inner sep=0pt, rounded corners=0pt, align=center}}

%\newcommand{\booklink}[1]{\website/bibliographie\##1}
\newcommand{\citelink}[2]{\hyperlink{cite.\therefsection @#1}{#2}}
\newcommand{\previousreference}{}
\providecommand{\reference}[2][]{%
	\notblank{#1}{\renewcommand{\previousreference}{#1}}{}%
	\todo[noline]{%
		\protect\vspace{16pt}%
		\protect\par%
		\protect\notblank{#1}{\cite{[\previousreference]}\\}{}%
		\protect\citelink{\previousreference}{p. #2}%
	}%
}

\definecolor{devcolor}{HTML}{00695c}
\newcommand{\dev}[1]{%
	\reversemarginpar%
	\todo[noline]{
		\protect\vspace{16pt}%
		\protect\par%
		\bfseries\color{devcolor}\href{\website/developpements/#1}{DEV}
	}%
	\normalmarginpar%
}

% En-têtes :

\pagestyle{fancy}
\fancyhead[L]{\truncate{0.23\textwidth}{\thepage}}
\fancyfoot[C]{\scriptsize \href{\website}{\texttt{agreg.skyost.eu}}}

% Couleurs :

\definecolor{property}{HTML}{fffde7}
\definecolor{proposition}{HTML}{fff8e1}
\definecolor{lemma}{HTML}{fff3e0}
\definecolor{theorem}{HTML}{fce4f2}
\definecolor{corollary}{HTML}{ffebee}
\definecolor{definition}{HTML}{ede7f6}
\definecolor{notation}{HTML}{f3e5f5}
\definecolor{example}{HTML}{e0f7fa}
\definecolor{cexample}{HTML}{efebe9}
\definecolor{application}{HTML}{e0f2f1}
\definecolor{remark}{HTML}{e8f5e9}
\definecolor{proof}{HTML}{e1f5fe}

% Théorèmes :

\theoremstyle{definition}
\newtheorem{theorem}{Théorème}

\newtheorem{property}[theorem]{Propriété}
\newtheorem{proposition}[theorem]{Proposition}
\newtheorem{lemma}[theorem]{Lemme}
\newtheorem{corollary}[theorem]{Corollaire}

\newtheorem{definition}[theorem]{Définition}
\newtheorem{notation}[theorem]{Notation}

\newtheorem{example}[theorem]{Exemple}
\newtheorem{cexample}[theorem]{Contre-exemple}
\newtheorem{application}[theorem]{Application}

\theoremstyle{remark}
\newtheorem{remark}[theorem]{Remarque}

\counterwithin*{theorem}{section}

\newcommand{\applystyletotheorem}[1]{
	\tcolorboxenvironment{#1}{
		enhanced,
		breakable,
		colback=#1!98!white,
		boxrule=0pt,
		boxsep=0pt,
		left=8pt,
		right=8pt,
		top=8pt,
		bottom=8pt,
		sharp corners,
		after=\par,
	}
}

\applystyletotheorem{property}
\applystyletotheorem{proposition}
\applystyletotheorem{lemma}
\applystyletotheorem{theorem}
\applystyletotheorem{corollary}
\applystyletotheorem{definition}
\applystyletotheorem{notation}
\applystyletotheorem{example}
\applystyletotheorem{cexample}
\applystyletotheorem{application}
\applystyletotheorem{remark}
\applystyletotheorem{proof}

% Environnements :

\NewEnviron{whitetabularx}[1]{%
	\renewcommand{\arraystretch}{2.5}
	\colorbox{white}{%
		\begin{tabularx}{\textwidth}{#1}%
			\BODY%
		\end{tabularx}%
	}%
}

% Maths :

\DeclareFontEncoding{FMS}{}{}
\DeclareFontSubstitution{FMS}{futm}{m}{n}
\DeclareFontEncoding{FMX}{}{}
\DeclareFontSubstitution{FMX}{futm}{m}{n}
\DeclareSymbolFont{fouriersymbols}{FMS}{futm}{m}{n}
\DeclareSymbolFont{fourierlargesymbols}{FMX}{futm}{m}{n}
\DeclareMathDelimiter{\VERT}{\mathord}{fouriersymbols}{152}{fourierlargesymbols}{147}


% Bibliographie :

\addbibresource{\bibliographypath}%
\defbibheading{bibliography}[\bibname]{%
	\newpage
	\section*{#1}%
}
\renewbibmacro*{entryhead:full}{\printfield{labeltitle}}%
\DeclareFieldFormat{url}{\newline\footnotesize\url{#1}}%

\AtEndDocument{\printbibliography}

\begin{document}
	%<*content>
	\development{analysis}{connexite-des-valeurs-adherence-d-une-suite-dans-un-compact}{Connexité des valeurs d'adhérence d'une suite dans un compact}

	\summary{On montre que l'ensemble des valeurs d'adhérence d'une suite d'un espace métrique compact est connexe en raisonnant par l'absurde, puis on utilise ce résultat pour démontrer le lemme des grenouilles.}

	\reference[GOU20]{46}

	Soit $(E, d)$ un espace métrique.

	\begin{theorem}
		\label{connexite-des-valeurs-adherence-d-une-suite-dans-un-compact-1}
		On suppose $E$ compact. Soit $(u_n)$ une suite de $E$ telle que $d(u_n,u_{n-1}) \longrightarrow 0$. Alors l'ensemble $\Gamma$ des valeurs d'adhérence de $(u_n)$ est connexe.
	\end{theorem}

	\begin{proof}
		Pour tout $p \in \mathbb{N}$, on note $A_p = \{u_n \mid n\geq p\}$. On a $\Gamma = \bigcap_{p \in \mathbb{N}} \overline{A_p}$. $\Gamma$ est fermé (en tant qu'intersection de fermés) dans $E$ qui est compact, donc $\Gamma$ est compact.
		Supposons que $\Gamma$ soit non connexe ; on peut alors écrire $\Gamma = A \, \sqcup \, B$, où $A$ et $B$ sont deux fermés disjoints de $\Gamma$. Comme $\Gamma$ est compact, $A$ et $B$ le sont aussi. Notons $\alpha = d(A, B) > 0$ (car $A \, \cap \, B = \emptyset$). Posons :
		\[
		A'= \left \{ x \in E \mid d(x, A) < \frac{\alpha}{3} \right \} \text{ et } B'= \left \{ x \in E \mid d(x, B) < \frac{\alpha}{3} \right \}
		\]
		$A'$ et $B'$ sont ouverts (en tant qu'images réciproques d'ouverts par des application continues), donc $K = E \setminus (A'\cup B')$ est fermé dans $E$, donc compact.
		\newpar
		Montrons que $(u_n)$ admet une valeur d'adhérence dans $K$, ce qui serait absurde car $\Gamma \, \cap \, K = \emptyset$. Comme $\lim_{n \rightarrow +\infty} d(u_n, u_{n-1})=0$,
		\[ \exists N_0 \in \mathbb{N} \text{ tel que } \forall n \geq N_0, \, d(u_n, u_{n-1}) < \frac{\alpha}{3} \tag{$*$} \]
		Soit $N \geq N_0$.
		\begin{itemize}
			\item Soit $x_0 \in A$. Comme $x_0$ est valeur d'adhérence de $(u_n)$, $\exists n_1 > N$ tel que $d(x_0, u_{n_1}) < \frac{\alpha}{3}$. Donc $u_{n_1} \in A'$.
			\item Soit $y_0 \in B$. De même, $\exists n_2 > n_1$ tel que $d(y_0, u_{n_2}) < \frac{\alpha}{3}$. Donc $u_{n_2} \in B'$.
		\end{itemize}
		Soit maintenant $n_0$ le premier entier supérieur à $n_1$ tel que $u_{n_0} \notin A'$ (un tel entier existe car $u_{n_2} \in A'$). On a alors $u_{n_0 - 1} \in A'$.
		\begin{center}
			\begin{tikzpicture}[scale=1, pics/a/.style n args={3}{code={\draw [thick, fill=#1, fill opacity=0.3, scale=#2, shift={#3}]  plot[smooth, tension=.7] coordinates {(-3,2.5) (-3.5,2.2) (-4.2,1) (-2,0.8) (-1.2,2) (-3,2.5)};}}, pics/b/.style n args={3}{code={\draw [thick, fill=#1, fill opacity=0.3, scale=#2, shift={#3}]  plot[smooth, tension=.7] coordinates {(-1.95,-0.35) (-2.5,-0.7) (-3.4,-1.9) (-1,-2) (-0.1,-0.05) (-1.95,-0.35)};}}]
				\draw [thick, fill=black, fill opacity=0.05]  plot[smooth, tension=.7] coordinates {(-4,2.5) (-3,3) (-2,2.8) (-0.8,2.5) (-0.5,1.5) (0.5,0) (0,-2) (-1.5,-2.5) (-4,-2) (-3.5,-0.5) (-5,1) (-4,2.5)};
				\pic {a={black!20!green!60!white}{1}{(0,0)}};
				\pic {a={black!20!green}{0.75}{(-0.9,0.5)}};
				\pic {b={black!20!green!60!white}{1}{(0,0)}};
				\pic {b={black!20!green}{0.75}{(-0.6,-0.45)}};
				\draw(-3,1.3) node {$\bullet$} node[left]{\tiny $x_0$};
				\draw(-3.1,0.82) node {$\bullet$} node[left]{\tiny $u_{n_1}$};
				\draw(-1.2,-0.7) node {$\bullet$} node[right]{\tiny $y_0$};
				\draw(-1.3,-0.3) node {$\bullet$} node[left]{\tiny $u_{n_2}$};
				\draw(-2.5,0.75) node {$\bullet$} node[right]{\tiny $u_{n_0 - 1}$};
				\draw(-2.3,0.4) node {$\bullet$} node[right]{\tiny $u_{n_0}$};
				\node at (-2.5,1.5) {\small $A$};
				\node at (-1.35,1.85) {\small $A'$};
				\node at (-1.6,-1.2) {\small $B$};
				\node at (-1.6,-2.07) {\small $B'$};
				\node at (-0.7,0.7) {$K$};
				\node at (-2.5,3.3) {$E$};
			\end{tikzpicture}
		\end{center}
		D'après $(*)$, en appliquant l'inégalité triangulaire,
		\begin{align*}
			d(u_{n_0}, B) & \geq d(u_{n_0 - 1}, B) - d(u_{n_0 - 1}, u_{n_0})           \\
			& \geq d(A, B) - d(u_{n_0 - 1}, A) - d(u_{n_0 - 1}, u_{n_0}) \\
			& > \frac{\alpha}{3}
		\end{align*}
		ce qui prouve que $u_{n_0} \notin B'$. Comme $u_{n_0} \notin A'$, on a $u_{n_0} \in K$. On vient de montrer que,
		\[ \forall N \geq N_0, \, \exists n_0 \geq N \text{ tel que } u_{n_0} \in K \]
		On peut créer comme cela une sous-suite de $(u_n)$ dans $K$. Or $K$ est compact, donc $(u_n)$ admet une valeur d'adhérence dans $K$.
	\end{proof}

	\reference[QUE20]{142}

	\begin{application}[Lemme de la grenouille]
		Soient $f : [0, 1] \rightarrow [0, 1]$ continue et $(x_n)$ une suite de $[0, 1]$ tel que
		\[ \begin{cases} x_0 \in [0, 1] \\ x_{n+1} = f(x_n) \end{cases} \]
		Alors $(x_n)$ converge si et seulement si $\lim_{n \rightarrow +\infty } x_{n+1} - x_n = 0$.
	\end{application}

	\begin{proof}
		Le sens direct est évident. Montrons la réciproque. On suppose donc que $\lim_{n \rightarrow +\infty } x_{n+1} - x_n = 0$ et on note $\Gamma$ l'ensemble des valeurs d'adhérence de $(x_n)$. $\Gamma$ est non vide (car $(x_n)$ est bornée, donc admet une valeur d'adhérence par le théorème de Bolzano-Weierstrass) et est un connexe de $\mathbb{R}$ (par le \cref{connexite-des-valeurs-adherence-d-une-suite-dans-un-compact-1}), donc $\Gamma$ est un intervalle non vide.
		\newpar
		Soit $a \in \Gamma$. Il existe $\varphi : \mathbb{N} \rightarrow \mathbb{N}$ strictement croissante (on dit que $\varphi$ est une extractrice) telle que $x_{\varphi(n)} \longrightarrow a$. Mais alors,
		\[ x_{\varphi(n) + 1} - x_{\varphi(n)} = f(x_{\varphi(n)}) - x_{\varphi(n)} \longrightarrow f(a) - a \]
		et par hypothèse, le membre de gauche converge vers $0$. Donc $f(a) - a = 0$ ie. $a$ est un point fixe de $f$.
		\newpar
		Supposons par l'absurde que $(x_n)$ diverge. Alors $\Gamma$ n'est pas un singleton, donc est un intervalle d'intérieur non vide : on peut trouver $c \in \Gamma$ et $h > 0$ tel que $[c - h, c + h] \subset \Gamma$.
		\newpar
		Or, $c \in \Gamma$, donc
		\[ \exists N \geq 0 \text{ tel que } |x_N - c| \leq \frac{h}{2} \implies x_N \in \Gamma \]
		et en particulier, $x_N$ est un point fixe de $f$. Ainsi, $x_{n+1} = f(x_n) = x_n$ pour tout $n \geq N$ : absurde.
	\end{proof}
	%</content>
\end{document}
