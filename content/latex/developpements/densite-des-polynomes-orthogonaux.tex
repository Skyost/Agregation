\documentclass[12pt, a4paper]{report}

% LuaLaTeX :

\RequirePackage{iftex}
\RequireLuaTeX

% Packages :

\usepackage[french]{babel}
%\usepackage[utf8]{inputenc}
%\usepackage[T1]{fontenc}
\usepackage[pdfencoding=auto, pdfauthor={Hugo Delaunay}, pdfsubject={Mathématiques}, pdfcreator={agreg.skyost.eu}]{hyperref}
\usepackage{amsmath}
\usepackage{amsthm}
%\usepackage{amssymb}
\usepackage{stmaryrd}
\usepackage{tikz}
\usepackage{tkz-euclide}
\usepackage{fourier-otf}
\usepackage{fontspec}
\usepackage{titlesec}
\usepackage{fancyhdr}
\usepackage{catchfilebetweentags}
\usepackage[french, capitalise, noabbrev]{cleveref}
\usepackage[fit, breakall]{truncate}
\usepackage[top=2.5cm, right=2cm, bottom=2.5cm, left=2cm]{geometry}
\usepackage{enumerate}
\usepackage{tocloft}
\usepackage{microtype}
%\usepackage{mdframed}
%\usepackage{thmtools}
\usepackage{xcolor}
\usepackage{tabularx}
\usepackage{aligned-overset}
\usepackage[subpreambles=true]{standalone}
\usepackage{environ}
\usepackage[normalem]{ulem}
\usepackage{marginnote}
\usepackage{etoolbox}
\usepackage{setspace}
\usepackage[bibstyle=reading, citestyle=draft]{biblatex}
\usepackage{xpatch}
\usepackage[many, breakable]{tcolorbox}
\usepackage[backgroundcolor=white, bordercolor=white, textsize=small]{todonotes}

% Bibliographie :

\newcommand{\overridebibliographypath}[1]{\providecommand{\bibliographypath}{#1}}
\overridebibliographypath{../bibliography.bib}
\addbibresource{\bibliographypath}
\defbibheading{bibliography}[\bibname]{%
	\newpage
	\section*{#1}%
}
\renewbibmacro*{entryhead:full}{\printfield{labeltitle}}
\DeclareFieldFormat{url}{\newline\footnotesize\url{#1}}
\AtEndDocument{\printbibliography}

% Police :

\setmathfont{Erewhon Math}

% Tikz :

\usetikzlibrary{calc}

% Longueurs :

\setlength{\parindent}{0pt}
\setlength{\headheight}{15pt}
\setlength{\fboxsep}{0pt}
\titlespacing*{\chapter}{0pt}{-20pt}{10pt}
\setlength{\marginparwidth}{1.5cm}
\setstretch{1.1}

% Métadonnées :

\author{agreg.skyost.eu}
\date{\today}

% Titres :

\setcounter{secnumdepth}{3}

\renewcommand{\thechapter}{\Roman{chapter}}
\renewcommand{\thesubsection}{\Roman{subsection}}
\renewcommand{\thesubsubsection}{\arabic{subsubsection}}
\renewcommand{\theparagraph}{\alph{paragraph}}

\titleformat{\chapter}{\huge\bfseries}{\thechapter}{20pt}{\huge\bfseries}
\titleformat*{\section}{\LARGE\bfseries}
\titleformat{\subsection}{\Large\bfseries}{\thesubsection \, - \,}{0pt}{\Large\bfseries}
\titleformat{\subsubsection}{\large\bfseries}{\thesubsubsection. \,}{0pt}{\large\bfseries}
\titleformat{\paragraph}{\bfseries}{\theparagraph. \,}{0pt}{\bfseries}

\setcounter{secnumdepth}{4}

% Table des matières :

\renewcommand{\cftsecleader}{\cftdotfill{\cftdotsep}}
\addtolength{\cftsecnumwidth}{10pt}

% Redéfinition des commandes :

\renewcommand*\thesection{\arabic{section}}
\renewcommand{\ker}{\mathrm{Ker}}

% Nouvelles commandes :

\newcommand{\website}{https://agreg.skyost.eu}

\newcommand{\tr}[1]{\mathstrut ^t #1}
\newcommand{\im}{\mathrm{Im}}
\newcommand{\rang}{\operatorname{rang}}
\newcommand{\trace}{\operatorname{trace}}
\newcommand{\id}{\operatorname{id}}
\newcommand{\stab}{\operatorname{Stab}}

\providecommand{\newpar}{\\[\medskipamount]}

\providecommand{\lesson}[3]{%
	\title{#3}%
	\hypersetup{pdftitle={#3}}%
	\setcounter{section}{\numexpr #2 - 1}%
	\section{#3}%
	\fancyhead[R]{\truncate{0.73\textwidth}{#2 : #3}}%
}

\providecommand{\development}[3]{%
	\title{#3}%
	\hypersetup{pdftitle={#3}}%
	\section*{#3}%
	\fancyhead[R]{\truncate{0.73\textwidth}{#3}}%
}

\providecommand{\summary}[1]{%
	\textit{#1}%
	\medskip%
}

\tikzset{notestyleraw/.append style={inner sep=0pt, rounded corners=0pt, align=center}}

%\newcommand{\booklink}[1]{\website/bibliographie\##1}
\newcommand{\citelink}[2]{\hyperlink{cite.\therefsection @#1}{#2}}
\newcommand{\previousreference}{}
\providecommand{\reference}[2][]{%
	\notblank{#1}{\renewcommand{\previousreference}{#1}}{}%
	\todo[noline]{%
		\protect\vspace{16pt}%
		\protect\par%
		\protect\notblank{#1}{\cite{[\previousreference]}\\}{}%
		\protect\citelink{\previousreference}{p. #2}%
	}%
}

\definecolor{devcolor}{HTML}{00695c}
\newcommand{\dev}[1]{%
	\reversemarginpar%
	\todo[noline]{
		\protect\vspace{16pt}%
		\protect\par%
		\bfseries\color{devcolor}\href{\website/developpements/#1}{DEV}
	}%
	\normalmarginpar%
}

% En-têtes :

\pagestyle{fancy}
\fancyhead[L]{\truncate{0.23\textwidth}{\thepage}}
\fancyfoot[C]{\scriptsize \href{\website}{\texttt{agreg.skyost.eu}}}

% Couleurs :

\definecolor{property}{HTML}{fffde7}
\definecolor{proposition}{HTML}{fff8e1}
\definecolor{lemma}{HTML}{fff3e0}
\definecolor{theorem}{HTML}{fce4f2}
\definecolor{corollary}{HTML}{ffebee}
\definecolor{definition}{HTML}{ede7f6}
\definecolor{notation}{HTML}{f3e5f5}
\definecolor{example}{HTML}{e0f7fa}
\definecolor{cexample}{HTML}{efebe9}
\definecolor{application}{HTML}{e0f2f1}
\definecolor{remark}{HTML}{e8f5e9}
\definecolor{proof}{HTML}{e1f5fe}

% Théorèmes :

\theoremstyle{definition}
\newtheorem{theorem}{Théorème}

\newtheorem{property}[theorem]{Propriété}
\newtheorem{proposition}[theorem]{Proposition}
\newtheorem{lemma}[theorem]{Lemme}
\newtheorem{corollary}[theorem]{Corollaire}

\newtheorem{definition}[theorem]{Définition}
\newtheorem{notation}[theorem]{Notation}

\newtheorem{example}[theorem]{Exemple}
\newtheorem{cexample}[theorem]{Contre-exemple}
\newtheorem{application}[theorem]{Application}

\theoremstyle{remark}
\newtheorem{remark}[theorem]{Remarque}

\counterwithin*{theorem}{section}

\newcommand{\applystyletotheorem}[1]{
	\tcolorboxenvironment{#1}{
		enhanced,
		breakable,
		colback=#1!98!white,
		boxrule=0pt,
		boxsep=0pt,
		left=8pt,
		right=8pt,
		top=8pt,
		bottom=8pt,
		sharp corners,
		after=\par,
	}
}

\applystyletotheorem{property}
\applystyletotheorem{proposition}
\applystyletotheorem{lemma}
\applystyletotheorem{theorem}
\applystyletotheorem{corollary}
\applystyletotheorem{definition}
\applystyletotheorem{notation}
\applystyletotheorem{example}
\applystyletotheorem{cexample}
\applystyletotheorem{application}
\applystyletotheorem{remark}
\applystyletotheorem{proof}

% Environnements :

\NewEnviron{whitetabularx}[1]{%
	\renewcommand{\arraystretch}{2.5}
	\colorbox{white}{%
		\begin{tabularx}{\textwidth}{#1}%
			\BODY%
		\end{tabularx}%
	}%
}

% Maths :

\DeclareFontEncoding{FMS}{}{}
\DeclareFontSubstitution{FMS}{futm}{m}{n}
\DeclareFontEncoding{FMX}{}{}
\DeclareFontSubstitution{FMX}{futm}{m}{n}
\DeclareSymbolFont{fouriersymbols}{FMS}{futm}{m}{n}
\DeclareSymbolFont{fourierlargesymbols}{FMX}{futm}{m}{n}
\DeclareMathDelimiter{\VERT}{\mathord}{fouriersymbols}{152}{fourierlargesymbols}{147}


% Bibliographie :

\addbibresource{\bibliographypath}%
\defbibheading{bibliography}[\bibname]{%
	\newpage
	\section*{#1}%
}
\renewbibmacro*{entryhead:full}{\printfield{labeltitle}}%
\DeclareFieldFormat{url}{\newline\footnotesize\url{#1}}%

\AtEndDocument{\printbibliography}

\begin{document}
	%<*content>
	\development{analysis}{densite-des-polynomes-orthogonaux}{Densité des polynômes orthogonaux}

	\summary{On montre que la famille des polynômes orthogonaux associée à une fonction poids $\rho$ vérifiant certaines hypothèses forme une base hilbertienne de $L_2(I, \rho)$ (où $I$ est un intervalle de $\mathbb{R}$).}

	\reference[BMP]{140}

	Soient $I$ un intervalle de $\mathbb{R}$ et $\rho$ une fonction poids. On considère $(P_n)$ la famille des polynômes orthogonaux associée à $\rho$ sur $I$.

	\begin{lemma}
		\label{densite-des-polynomes-orthogonaux-1}
		On suppose que $\forall n \in \mathbb{N}$, $g_n : x \mapsto x^n \in L_1(I, \rho)$. Alors $\forall n \in \mathbb{N}$, $g_n \in L_2(I, \rho)$. En particulier, l'algorithme de Gram-Schmidt a bien du sens et $(P_n)$ est bien définie.
	\end{lemma}

	\begin{proof}
		On a $\forall n \in \mathbb{N}$,
		\[ \int_I \vert x^n \vert^2 \rho(x) \, \mathrm{d}x = \int_I \vert x^{2n} \vert \rho(x) \, \mathrm{d}x = \Vert g_{2n} \Vert_1 < +\infty \]
	\end{proof}

	\begin{theorem}
		On suppose qu'il existe $a > 0$ tel que
		\[ \int_I e^{a \vert x \vert} \rho(x) \, \mathrm{d}x < +\infty \]
		alors $(P_n)$ est une base hilbertienne de $L_2(I, \rho)$ pour la norme $\Vert . \Vert_2$.
	\end{theorem}

	\begin{proof}
		Soit $f \in \operatorname{Vect}(g_n)^\perp = \operatorname{Vect}(P_n)^\perp$. On définit
		\[
		\forall x \in \mathbb{R}, \quad \varphi(x) = \begin{cases} f(x) \rho(x) &\text{si } x \in I \\ 0 &\text{sinon} \end{cases}
		\]
		Montrons que $\varphi \in L_1(\mathbb{R})$. Remarquons tout d'abord que $\forall t \geq 0$, $t \leq \frac{1 + t^2}{2}$. Ainsi, on a
		\[ \forall x \in I, \quad |f(x)|\rho(x) \leq \frac{(1 + |f(x)|)^2}{2} \rho(x) \]
		Comme $\rho$ et $\rho f^2$ sont intégrables sur $I$, on en déduit que $\varphi \in L_1(\mathbb{R})$. On peut donc considérer sa transformée de Fourier
		\[ \widehat{\varphi} : \xi \mapsto \int_I f(x) e^{-i \xi x} \rho(x) \, \mathrm{d}x \]
		Montrons que $\widehat{\varphi}$ se prolonge en une fonction $F$ holomorphe sur
		\[ B_a = \left\{ z \in \mathbb{C} \mid \vert \operatorname{Im}(z) \vert < \frac{a}{2} \right\} \]
		\begin{center}
			\begin{tikzpicture}
				\draw[->] (-3, 0) -- (3, 0) node[right] {$x$};
				\draw[->] (0, -3) -- (0, 3) node[above] {$y$};
				\draw (-3, 2) -- (3, 2);
				\draw (-3, -2) -- (3, -2);
				\fill[opacity=0.3,blue!30] (-3,-2) rectangle (3,2);
				\draw(0,2) node {$\bullet$};
				\draw(0,2) node[above right]{$\frac{a}{2}$};
				\draw(0,-2) node {$\bullet$};
				\draw(0,-2) node[below right]{$-\frac{a}{2}$};
			\end{tikzpicture}
		\end{center}
		Définissons à présent $g : (z, x) \mapsto e^{-izx} f(x) \rho(x)$. Pour $z \in B_a$, on a
		\[ \int_I \vert g(z, x) \vert \, \mathrm{d}x \leq \int_I e^{\frac{a \vert x \vert}{2}} \vert f(x) \vert \rho(x) \, \mathrm{d}x \]
		En utilisant l'inégalité de Cauchy-Schwarz pour $\Vert . \Vert_2$, on obtient de plus
		\[ \int_I e^{\frac{a \vert x \vert}{2}} \vert f(x) \vert \rho(x) \, \mathrm{d}x \leq \left( \int_I e^{a \vert x \vert} \rho(x) \, \mathrm{d}x \right)^{\frac{1}{2}} \left( \int_I \vert f(x) \vert^2 \rho(x) \, \mathrm{d}x \right)^{\frac{1}{2}} < + \infty \tag{$*$} \]
		On définit la fonction $F$ par
		\[ \forall z \in B_a , \quad F(z) = \int_I e^{-izx} f(x) \rho(x) \, \mathrm{d}x = \int_I g(z, x) \, \mathrm{d}x \]
		L'inégalité $(*)$ montre que cette fonction est bien définie. De plus :
		\begin{itemize}
			\item $\forall z \in B_a$, $x \mapsto g(z, x)$ est mesurable.
			\item pp. en $x \in I$, $z \mapsto g(z, x)$ est holomorphe.
			\item $\forall z \in B_a$, $\forall x \in I$,
			\[ \vert g(z, x) \vert \leq h(x) = e^{\frac{a \vert x \vert}{2}} \vert f(x) \vert \rho(x) \]
			et l'inégalité $(*)$ montre que $h \in L_1(I)$.
		\end{itemize}
		Donc par le théorème d'holomorphie sous l'intégrale, la fonction $F$ est holomorphe sur $B_a$, et coïncide sur $\mathbb{R}$ avec $\widehat{\varphi}$.
		Ce théorème nous dit de plus que
		\[ \forall n \in \mathbb{N}, \forall z \in B_a, \, F^{(n)}(z) = (-i)^n \int_I x^n e^{-izx} f(x) \rho(x) \, \mathrm{d}x \]
		Ce qui donne, une fois évalué en $0$ :
		\[ \forall n \in \mathbb{N}, F^{(n)}(0) = (-i)^n \int_I x^n f(x) \rho(x) \, \mathrm{d}x = (-i)^n \langle g_n, f \rangle = 0 \]
		L'unicité du développement en série entière d'une fonction holomorphe montre que $F = 0$ sur un voisinage de $0$. Le théorème du prolongement analytique implique alors que $F = 0$ sur le connexe $B_a$ tout entier, et donc en particulier, sur $\mathbb{R}$. Ainsi, $\widehat{\varphi} = 0$. Comme $\varphi$ est une fonction intégrable, l'injectivité de la transformée de Fourier implique que $\varphi = 0$. Comme $\rho(x) > 0$, on en déduit que $f(x) = 0$ pp. en $x \in I$. On vient donc de montrer qu'une fonction orthogonale à tous les polynômes est nulle i.e. $\operatorname{Vect}(g_n)^\perp = \{ 0 \}$.
		En ajoutant le \cref{densite-des-polynomes-orthogonaux-1} à ceci, on a bien que les polynômes orthogonaux forment une base hilbertienne de $L_2(I, \rho)$.
	\end{proof}

	\begin{cexample}
		On considère, sur $I = \mathbb{R}^+_*$, la fonction poids $\rho : x \mapsto x^{-\ln(x)}$. On pose $\forall x \in I$, $f(x) = \sin(2 \pi \ln(x))$. On calcule
		\begin{align*}
			\langle f, g_n \rangle &= \int_I x^n \sin(2\pi \ln(x)) x^{-\ln(x)} \, \mathrm{d}x \\
			\overset{y = \ln(x)}&{=} \int_{\mathbb{R}} e^{(n+1)y} \sin(2 \pi y) e^{-y^2} \, \mathrm{d}y \\
			&= e^{\frac{(n+1)^2}{4}} \int_{\mathbb{R}} e^{- \left (y - \frac{n+1}{2} \right)^2} \sin(2 \pi y) \, \mathrm{d}y \\
			&= (-1)^{n+1} e^{\frac{(n+1)^2}{4}} \int_{\mathbb{R}} \sin(2 \pi t) e^{-t^2} \, \mathrm{d}t \text{, avec } t = y - \frac{n+1}{2} \\
			\overset{f \text{ impaire}}&{=} 0
		\end{align*}
		Ainsi, la famille des $g_n$ n'est pas totale. La famille des polynômes orthogonaux associée à ce poids particulier n'est donc pas totale non plus : ce n'est pas une base hilbertienne.
	\end{cexample}
	%</content>
\end{document}
