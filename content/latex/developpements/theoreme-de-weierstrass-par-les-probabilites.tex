\documentclass[12pt, a4paper]{report}

% LuaLaTeX :

\RequirePackage{iftex}
\RequireLuaTeX

% Packages :

\usepackage[french]{babel}
%\usepackage[utf8]{inputenc}
%\usepackage[T1]{fontenc}
\usepackage[pdfencoding=auto, pdfauthor={Hugo Delaunay}, pdfsubject={Mathématiques}, pdfcreator={agreg.skyost.eu}]{hyperref}
\usepackage{amsmath}
\usepackage{amsthm}
%\usepackage{amssymb}
\usepackage{stmaryrd}
\usepackage{tikz}
\usepackage{tkz-euclide}
\usepackage{fourier-otf}
\usepackage{fontspec}
\usepackage{titlesec}
\usepackage{fancyhdr}
\usepackage{catchfilebetweentags}
\usepackage[french, capitalise, noabbrev]{cleveref}
\usepackage[fit, breakall]{truncate}
\usepackage[top=2.5cm, right=2cm, bottom=2.5cm, left=2cm]{geometry}
\usepackage{enumerate}
\usepackage{tocloft}
\usepackage{microtype}
%\usepackage{mdframed}
%\usepackage{thmtools}
\usepackage{xcolor}
\usepackage{tabularx}
\usepackage{aligned-overset}
\usepackage[subpreambles=true]{standalone}
\usepackage{environ}
\usepackage[normalem]{ulem}
\usepackage{marginnote}
\usepackage{etoolbox}
\usepackage{setspace}
\usepackage[bibstyle=reading, citestyle=draft]{biblatex}
\usepackage{xpatch}
\usepackage[many, breakable]{tcolorbox}
\usepackage[backgroundcolor=white, bordercolor=white, textsize=small]{todonotes}

% Bibliographie :

\newcommand{\overridebibliographypath}[1]{\providecommand{\bibliographypath}{#1}}
\overridebibliographypath{../bibliography.bib}
\addbibresource{\bibliographypath}
\defbibheading{bibliography}[\bibname]{%
	\newpage
	\section*{#1}%
}
\renewbibmacro*{entryhead:full}{\printfield{labeltitle}}
\DeclareFieldFormat{url}{\newline\footnotesize\url{#1}}
\AtEndDocument{\printbibliography}

% Police :

\setmathfont{Erewhon Math}

% Tikz :

\usetikzlibrary{calc}

% Longueurs :

\setlength{\parindent}{0pt}
\setlength{\headheight}{15pt}
\setlength{\fboxsep}{0pt}
\titlespacing*{\chapter}{0pt}{-20pt}{10pt}
\setlength{\marginparwidth}{1.5cm}
\setstretch{1.1}

% Métadonnées :

\author{agreg.skyost.eu}
\date{\today}

% Titres :

\setcounter{secnumdepth}{3}

\renewcommand{\thechapter}{\Roman{chapter}}
\renewcommand{\thesubsection}{\Roman{subsection}}
\renewcommand{\thesubsubsection}{\arabic{subsubsection}}
\renewcommand{\theparagraph}{\alph{paragraph}}

\titleformat{\chapter}{\huge\bfseries}{\thechapter}{20pt}{\huge\bfseries}
\titleformat*{\section}{\LARGE\bfseries}
\titleformat{\subsection}{\Large\bfseries}{\thesubsection \, - \,}{0pt}{\Large\bfseries}
\titleformat{\subsubsection}{\large\bfseries}{\thesubsubsection. \,}{0pt}{\large\bfseries}
\titleformat{\paragraph}{\bfseries}{\theparagraph. \,}{0pt}{\bfseries}

\setcounter{secnumdepth}{4}

% Table des matières :

\renewcommand{\cftsecleader}{\cftdotfill{\cftdotsep}}
\addtolength{\cftsecnumwidth}{10pt}

% Redéfinition des commandes :

\renewcommand*\thesection{\arabic{section}}
\renewcommand{\ker}{\mathrm{Ker}}

% Nouvelles commandes :

\newcommand{\website}{https://agreg.skyost.eu}

\newcommand{\tr}[1]{\mathstrut ^t #1}
\newcommand{\im}{\mathrm{Im}}
\newcommand{\rang}{\operatorname{rang}}
\newcommand{\trace}{\operatorname{trace}}
\newcommand{\id}{\operatorname{id}}
\newcommand{\stab}{\operatorname{Stab}}

\providecommand{\newpar}{\\[\medskipamount]}

\providecommand{\lesson}[3]{%
	\title{#3}%
	\hypersetup{pdftitle={#3}}%
	\setcounter{section}{\numexpr #2 - 1}%
	\section{#3}%
	\fancyhead[R]{\truncate{0.73\textwidth}{#2 : #3}}%
}

\providecommand{\development}[3]{%
	\title{#3}%
	\hypersetup{pdftitle={#3}}%
	\section*{#3}%
	\fancyhead[R]{\truncate{0.73\textwidth}{#3}}%
}

\providecommand{\summary}[1]{%
	\textit{#1}%
	\medskip%
}

\tikzset{notestyleraw/.append style={inner sep=0pt, rounded corners=0pt, align=center}}

%\newcommand{\booklink}[1]{\website/bibliographie\##1}
\newcommand{\citelink}[2]{\hyperlink{cite.\therefsection @#1}{#2}}
\newcommand{\previousreference}{}
\providecommand{\reference}[2][]{%
	\notblank{#1}{\renewcommand{\previousreference}{#1}}{}%
	\todo[noline]{%
		\protect\vspace{16pt}%
		\protect\par%
		\protect\notblank{#1}{\cite{[\previousreference]}\\}{}%
		\protect\citelink{\previousreference}{p. #2}%
	}%
}

\definecolor{devcolor}{HTML}{00695c}
\newcommand{\dev}[1]{%
	\reversemarginpar%
	\todo[noline]{
		\protect\vspace{16pt}%
		\protect\par%
		\bfseries\color{devcolor}\href{\website/developpements/#1}{DEV}
	}%
	\normalmarginpar%
}

% En-têtes :

\pagestyle{fancy}
\fancyhead[L]{\truncate{0.23\textwidth}{\thepage}}
\fancyfoot[C]{\scriptsize \href{\website}{\texttt{agreg.skyost.eu}}}

% Couleurs :

\definecolor{property}{HTML}{fffde7}
\definecolor{proposition}{HTML}{fff8e1}
\definecolor{lemma}{HTML}{fff3e0}
\definecolor{theorem}{HTML}{fce4f2}
\definecolor{corollary}{HTML}{ffebee}
\definecolor{definition}{HTML}{ede7f6}
\definecolor{notation}{HTML}{f3e5f5}
\definecolor{example}{HTML}{e0f7fa}
\definecolor{cexample}{HTML}{efebe9}
\definecolor{application}{HTML}{e0f2f1}
\definecolor{remark}{HTML}{e8f5e9}
\definecolor{proof}{HTML}{e1f5fe}

% Théorèmes :

\theoremstyle{definition}
\newtheorem{theorem}{Théorème}

\newtheorem{property}[theorem]{Propriété}
\newtheorem{proposition}[theorem]{Proposition}
\newtheorem{lemma}[theorem]{Lemme}
\newtheorem{corollary}[theorem]{Corollaire}

\newtheorem{definition}[theorem]{Définition}
\newtheorem{notation}[theorem]{Notation}

\newtheorem{example}[theorem]{Exemple}
\newtheorem{cexample}[theorem]{Contre-exemple}
\newtheorem{application}[theorem]{Application}

\theoremstyle{remark}
\newtheorem{remark}[theorem]{Remarque}

\counterwithin*{theorem}{section}

\newcommand{\applystyletotheorem}[1]{
	\tcolorboxenvironment{#1}{
		enhanced,
		breakable,
		colback=#1!98!white,
		boxrule=0pt,
		boxsep=0pt,
		left=8pt,
		right=8pt,
		top=8pt,
		bottom=8pt,
		sharp corners,
		after=\par,
	}
}

\applystyletotheorem{property}
\applystyletotheorem{proposition}
\applystyletotheorem{lemma}
\applystyletotheorem{theorem}
\applystyletotheorem{corollary}
\applystyletotheorem{definition}
\applystyletotheorem{notation}
\applystyletotheorem{example}
\applystyletotheorem{cexample}
\applystyletotheorem{application}
\applystyletotheorem{remark}
\applystyletotheorem{proof}

% Environnements :

\NewEnviron{whitetabularx}[1]{%
	\renewcommand{\arraystretch}{2.5}
	\colorbox{white}{%
		\begin{tabularx}{\textwidth}{#1}%
			\BODY%
		\end{tabularx}%
	}%
}

% Maths :

\DeclareFontEncoding{FMS}{}{}
\DeclareFontSubstitution{FMS}{futm}{m}{n}
\DeclareFontEncoding{FMX}{}{}
\DeclareFontSubstitution{FMX}{futm}{m}{n}
\DeclareSymbolFont{fouriersymbols}{FMS}{futm}{m}{n}
\DeclareSymbolFont{fourierlargesymbols}{FMX}{futm}{m}{n}
\DeclareMathDelimiter{\VERT}{\mathord}{fouriersymbols}{152}{fourierlargesymbols}{147}


% Bibliographie :

\addbibresource{\bibliographypath}%
\defbibheading{bibliography}[\bibname]{%
	\newpage
	\section*{#1}%
}
\renewbibmacro*{entryhead:full}{\printfield{labeltitle}}%
\DeclareFieldFormat{url}{\newline\footnotesize\url{#1}}%

\AtEndDocument{\printbibliography}

\begin{document}
  %<*content>
  \development{analysis}{theoreme-de-weierstrass-par-les-probabilites}{Théorème de Weierstrass (par les probabilités)}

  \summary{On montre le théorème de Weierstrass en faisant un raisonnement sur des variables aléatoires suivant une loi de Bernoulli.}

  \reference[G-K]{195}

  \begin{theorem}[Bernstein]
    \label{theoreme-de-weierstrass-par-les-probabilites-1}
    Soit $f : [0,1] \rightarrow \mathbb{R}$ continue. On note
    \[ \forall n \in \mathbb{N}^*, \, B_n(f) : x \mapsto \sum_{k=0}^n \binom{n}{k} f \left( \frac{k}{n} \right) x^k (1-x)^{n-k} \]
    le $n$-ième polynôme de Bernstein associé à $f$. Alors le suite de fonctions $(B_n(f))$ converge uniformément vers $f$.
  \end{theorem}

  \begin{proof}
    Soit $x \in ]0, 1[$. On se place sur un espace probabilité $(\Omega, \mathcal{A}, \mathbb{P})$ et considère $(X_k)$ une suite de variables aléatoires indépendantes de même loi $\mathcal{B}(x)$. On note $\forall n \in \mathbb{N}^*$, $S_n = \sum_{k=1}^n X_k$. Ainsi, $S_n \sim \mathcal{B}(n, x)$ et donc par la formule de transfert,
    \[ \mathbb{E} \left( \frac{S_n}{n} \right) = \sum_{k=0}^n \binom{n}{k} f \left( \frac{k}{n} \right) x^k (1-x)^{n-k} = B_n(f)(x) \]
    La fonction $f$ est continue sur $[0,1]$ qui est un compact de $\mathbb{R}$, donc par le théorème de Heine; elle y est uniformément continue. Soit donc $\epsilon > 0$,
    \[ \exists \eta > 0 \text{ tel que } \forall x, y \in [0,1], \, |x-y| < \eta \implies |f(x) - f(y)| < \epsilon \]
    On a,
    \begin{align*}
      |B_n(f)(x) - f(x)| &= \left| \mathbb{E} \left( f \left( \frac{S_n}{n} \right) \right) - f(x) \right| \\
      &= \left| \mathbb{E} \left( f \left( \frac{S_n}{n} \right) - f(x) \right) \right| \\
      &\leq \mathbb{E} \left| f \left( \frac{S_n}{n} \right) - f(x) \right| \\
      &\leq \mathbb{E} \left( \mathbb{1}_{\left \{ \left| \frac{S_n}{n} - x \right| < \eta \right \}} \left| f \left( \frac{S_n}{n} \right) - f(x) \right| \right) + \mathbb{E} \left( \mathbb{1}_{\left \{ \left| \frac{S_n}{n} - x \right| \geq \eta \right \}} \left| f \left( \frac{S_n}{n} \right) - f(x) \right| \right) \\
      &\leq \mathbb{E} (\epsilon) + 2 \Vert f \Vert_\infty \mathbb{E} \left( \mathbb{1}_{\left \{ \left| \frac{S_n}{n} - x \right| \geq \eta \right \}} \right) \\
      &= \epsilon + 2 \Vert f \Vert_\infty \mathbb{P} \left( \left| \frac{S_n}{n} - x \right| \geq \eta \right) \tag{$*$}
    \end{align*}
    Comme $\mathbb{E} \left( \frac{S_n}{n} \right) = x$, on peut appliquer l'inégalité de Bienaymé-Tchebychev :
    \[ \mathbb{P} \left( \left| \frac{S_n}{n} - x \right| \geq \eta \right) = \mathbb{P} \left( \left| \frac{S_n}{n} - \mathbb{E} \left( \frac{S_n}{n} \right) \right| \geq \eta \right) \leq \frac{1}{\eta^2} \operatorname{Var} \left( \frac{S_n}{n} \right) \]
    Comme les $X_k$ sont indépendantes et de même loi :
    \[ \operatorname{Var} \left( \frac{S_n}{n} \right) = \frac{1}{n^2} \operatorname{Var} (S_n) = \frac{1}{n} \operatorname{Var}(X_1) = \frac{x(1-x)}{n} \leq \frac{1}{n} \]
    En réinjectant cela dans $(*)$, cela donne
    \[ |B_n(f)(x) - f(x)| \leq \epsilon + \frac{2 \Vert f \Vert_\infty}{n \eta^2} \]
    qui est une majoration indépendante de $x$. Comme la fonction $B_n(f) - f$ est continue sur $[0, 1]$, on peut passer à la borne supérieure :
    \[ \Vert B_n(f) - f \Vert_\infty = \sup_{x \in [0, 1]} |B_n(f)(x) - f(x)| \leq \epsilon + \frac{2 \Vert f \Vert_\infty}{n \eta^2} \]
    ce qui donne après un passage à la limite supérieure :
    \begin{align*}
      &\limsup_{n \rightarrow +\infty} \Vert B_n(f) - f \Vert_\infty \leq \epsilon \\
      \overset{\epsilon \longrightarrow 0}{\implies} &\limsup_{n \rightarrow +\infty} \Vert B_n(f) - f \Vert_\infty = 0 \\
      \implies &\lim_{n \rightarrow +\infty} \Vert B_n(f) - f \Vert_\infty = 0
    \end{align*}
  \end{proof}

  \begin{theorem}[Weierstrass]
    Toute fonction continue $f : [a,b] \rightarrow \mathbb{R}$ (avec $a, b \in \mathbb{R}$ tels que $a \leq b$) est limite uniforme de fonctions polynômiales sur $[a, b]$.
  \end{theorem}

  \begin{proof}
    On va avoir besoin du changement de variable suivant :
    \[ \varphi :
    \begin{array}{ccc}
      [0,1] &\rightarrow& [a, b] \\
      x &\mapsto& a + (b-a) x
    \end{array}
    \]
    Par le \cref{theoreme-de-weierstrass-par-les-probabilites-1}, la fonction $f \circ \varphi$ est limite uniforme d'une suite de fonctions polynômiales $(p_n)$. Donc $f$ est limite uniforme de la suite $(p_n \circ \varphi^{-1})$ où $\forall n \in \mathbb{N}$, $p_n \circ \varphi^{-1}$ est bien une fonction polynômiale car $\varphi$ (donc $\varphi^{-1}$ aussi) est affine.
  \end{proof}
  %</content>
\end{document}
