\documentclass[12pt, a4paper]{report}

% LuaLaTeX :

\RequirePackage{iftex}
\RequireLuaTeX

% Packages :

\usepackage[french]{babel}
%\usepackage[utf8]{inputenc}
%\usepackage[T1]{fontenc}
\usepackage[pdfencoding=auto, pdfauthor={Hugo Delaunay}, pdfsubject={Mathématiques}, pdfcreator={agreg.skyost.eu}]{hyperref}
\usepackage{amsmath}
\usepackage{amsthm}
%\usepackage{amssymb}
\usepackage{stmaryrd}
\usepackage{tikz}
\usepackage{tkz-euclide}
\usepackage{fourier-otf}
\usepackage{fontspec}
\usepackage{titlesec}
\usepackage{fancyhdr}
\usepackage{catchfilebetweentags}
\usepackage[french, capitalise, noabbrev]{cleveref}
\usepackage[fit, breakall]{truncate}
\usepackage[top=2.5cm, right=2cm, bottom=2.5cm, left=2cm]{geometry}
\usepackage{enumerate}
\usepackage{tocloft}
\usepackage{microtype}
%\usepackage{mdframed}
%\usepackage{thmtools}
\usepackage{xcolor}
\usepackage{tabularx}
\usepackage{aligned-overset}
\usepackage[subpreambles=true]{standalone}
\usepackage{environ}
\usepackage[normalem]{ulem}
\usepackage{marginnote}
\usepackage{etoolbox}
\usepackage{setspace}
\usepackage[bibstyle=reading, citestyle=draft]{biblatex}
\usepackage{xpatch}
\usepackage[many, breakable]{tcolorbox}
\usepackage[backgroundcolor=white, bordercolor=white, textsize=small]{todonotes}

% Bibliographie :

\newcommand{\overridebibliographypath}[1]{\providecommand{\bibliographypath}{#1}}
\overridebibliographypath{../bibliography.bib}
\addbibresource{\bibliographypath}
\defbibheading{bibliography}[\bibname]{%
	\newpage
	\section*{#1}%
}
\renewbibmacro*{entryhead:full}{\printfield{labeltitle}}
\DeclareFieldFormat{url}{\newline\footnotesize\url{#1}}
\AtEndDocument{\printbibliography}

% Police :

\setmathfont{Erewhon Math}

% Tikz :

\usetikzlibrary{calc}

% Longueurs :

\setlength{\parindent}{0pt}
\setlength{\headheight}{15pt}
\setlength{\fboxsep}{0pt}
\titlespacing*{\chapter}{0pt}{-20pt}{10pt}
\setlength{\marginparwidth}{1.5cm}
\setstretch{1.1}

% Métadonnées :

\author{agreg.skyost.eu}
\date{\today}

% Titres :

\setcounter{secnumdepth}{3}

\renewcommand{\thechapter}{\Roman{chapter}}
\renewcommand{\thesubsection}{\Roman{subsection}}
\renewcommand{\thesubsubsection}{\arabic{subsubsection}}
\renewcommand{\theparagraph}{\alph{paragraph}}

\titleformat{\chapter}{\huge\bfseries}{\thechapter}{20pt}{\huge\bfseries}
\titleformat*{\section}{\LARGE\bfseries}
\titleformat{\subsection}{\Large\bfseries}{\thesubsection \, - \,}{0pt}{\Large\bfseries}
\titleformat{\subsubsection}{\large\bfseries}{\thesubsubsection. \,}{0pt}{\large\bfseries}
\titleformat{\paragraph}{\bfseries}{\theparagraph. \,}{0pt}{\bfseries}

\setcounter{secnumdepth}{4}

% Table des matières :

\renewcommand{\cftsecleader}{\cftdotfill{\cftdotsep}}
\addtolength{\cftsecnumwidth}{10pt}

% Redéfinition des commandes :

\renewcommand*\thesection{\arabic{section}}
\renewcommand{\ker}{\mathrm{Ker}}

% Nouvelles commandes :

\newcommand{\website}{https://agreg.skyost.eu}

\newcommand{\tr}[1]{\mathstrut ^t #1}
\newcommand{\im}{\mathrm{Im}}
\newcommand{\rang}{\operatorname{rang}}
\newcommand{\trace}{\operatorname{trace}}
\newcommand{\id}{\operatorname{id}}
\newcommand{\stab}{\operatorname{Stab}}

\providecommand{\newpar}{\\[\medskipamount]}

\providecommand{\lesson}[3]{%
	\title{#3}%
	\hypersetup{pdftitle={#3}}%
	\setcounter{section}{\numexpr #2 - 1}%
	\section{#3}%
	\fancyhead[R]{\truncate{0.73\textwidth}{#2 : #3}}%
}

\providecommand{\development}[3]{%
	\title{#3}%
	\hypersetup{pdftitle={#3}}%
	\section*{#3}%
	\fancyhead[R]{\truncate{0.73\textwidth}{#3}}%
}

\providecommand{\summary}[1]{%
	\textit{#1}%
	\medskip%
}

\tikzset{notestyleraw/.append style={inner sep=0pt, rounded corners=0pt, align=center}}

%\newcommand{\booklink}[1]{\website/bibliographie\##1}
\newcommand{\citelink}[2]{\hyperlink{cite.\therefsection @#1}{#2}}
\newcommand{\previousreference}{}
\providecommand{\reference}[2][]{%
	\notblank{#1}{\renewcommand{\previousreference}{#1}}{}%
	\todo[noline]{%
		\protect\vspace{16pt}%
		\protect\par%
		\protect\notblank{#1}{\cite{[\previousreference]}\\}{}%
		\protect\citelink{\previousreference}{p. #2}%
	}%
}

\definecolor{devcolor}{HTML}{00695c}
\newcommand{\dev}[1]{%
	\reversemarginpar%
	\todo[noline]{
		\protect\vspace{16pt}%
		\protect\par%
		\bfseries\color{devcolor}\href{\website/developpements/#1}{DEV}
	}%
	\normalmarginpar%
}

% En-têtes :

\pagestyle{fancy}
\fancyhead[L]{\truncate{0.23\textwidth}{\thepage}}
\fancyfoot[C]{\scriptsize \href{\website}{\texttt{agreg.skyost.eu}}}

% Couleurs :

\definecolor{property}{HTML}{fffde7}
\definecolor{proposition}{HTML}{fff8e1}
\definecolor{lemma}{HTML}{fff3e0}
\definecolor{theorem}{HTML}{fce4f2}
\definecolor{corollary}{HTML}{ffebee}
\definecolor{definition}{HTML}{ede7f6}
\definecolor{notation}{HTML}{f3e5f5}
\definecolor{example}{HTML}{e0f7fa}
\definecolor{cexample}{HTML}{efebe9}
\definecolor{application}{HTML}{e0f2f1}
\definecolor{remark}{HTML}{e8f5e9}
\definecolor{proof}{HTML}{e1f5fe}

% Théorèmes :

\theoremstyle{definition}
\newtheorem{theorem}{Théorème}

\newtheorem{property}[theorem]{Propriété}
\newtheorem{proposition}[theorem]{Proposition}
\newtheorem{lemma}[theorem]{Lemme}
\newtheorem{corollary}[theorem]{Corollaire}

\newtheorem{definition}[theorem]{Définition}
\newtheorem{notation}[theorem]{Notation}

\newtheorem{example}[theorem]{Exemple}
\newtheorem{cexample}[theorem]{Contre-exemple}
\newtheorem{application}[theorem]{Application}

\theoremstyle{remark}
\newtheorem{remark}[theorem]{Remarque}

\counterwithin*{theorem}{section}

\newcommand{\applystyletotheorem}[1]{
	\tcolorboxenvironment{#1}{
		enhanced,
		breakable,
		colback=#1!98!white,
		boxrule=0pt,
		boxsep=0pt,
		left=8pt,
		right=8pt,
		top=8pt,
		bottom=8pt,
		sharp corners,
		after=\par,
	}
}

\applystyletotheorem{property}
\applystyletotheorem{proposition}
\applystyletotheorem{lemma}
\applystyletotheorem{theorem}
\applystyletotheorem{corollary}
\applystyletotheorem{definition}
\applystyletotheorem{notation}
\applystyletotheorem{example}
\applystyletotheorem{cexample}
\applystyletotheorem{application}
\applystyletotheorem{remark}
\applystyletotheorem{proof}

% Environnements :

\NewEnviron{whitetabularx}[1]{%
	\renewcommand{\arraystretch}{2.5}
	\colorbox{white}{%
		\begin{tabularx}{\textwidth}{#1}%
			\BODY%
		\end{tabularx}%
	}%
}

% Maths :

\DeclareFontEncoding{FMS}{}{}
\DeclareFontSubstitution{FMS}{futm}{m}{n}
\DeclareFontEncoding{FMX}{}{}
\DeclareFontSubstitution{FMX}{futm}{m}{n}
\DeclareSymbolFont{fouriersymbols}{FMS}{futm}{m}{n}
\DeclareSymbolFont{fourierlargesymbols}{FMX}{futm}{m}{n}
\DeclareMathDelimiter{\VERT}{\mathord}{fouriersymbols}{152}{fourierlargesymbols}{147}


% Bibliographie :

\addbibresource{\bibliographypath}%
\defbibheading{bibliography}[\bibname]{%
	\newpage
	\section*{#1}%
}
\renewbibmacro*{entryhead:full}{\printfield{labeltitle}}%
\DeclareFieldFormat{url}{\newline\footnotesize\url{#1}}%

\AtEndDocument{\printbibliography}

\begin{document}
	%<*content>
	\development{algebra, analysis}{nombres-de-bell}{Nombres de Bell}

	\summary{En utilisant les propriétés des séries entières, nous calculons le nombre de partitions de l'ensemble $\llbracket 1, n \rrbracket$.}

	\reference[GOU21]{314}

	\begin{theorem}[Nombres de Bell]
		Pour tout $n \in \mathbb{N}^*$, on note $B_n$ le nombre de partitions de $\llbracket 1, n \rrbracket$. Par convention on pose $B_0 = 1$. Alors,
		\[ \forall k \in \mathbb{N}^*, \, B_k = \frac{1}{e} \sum_{n=0}^{+\infty} \frac{n^k}{n!} \]
	\end{theorem}

	\begin{proof}
		Notons que clairement $B_1 = 1$. Soit $n \in \mathbb{N}^*$, exprimons $B_{n+1}$ en fonction des termes précédents. Pour tout $k \leq n$, on note $E_k$ l'ensemble des partitions $P$ de $\llbracket 1, n+1 \rrbracket$ tel que la partie de $P$ qui contient l'entier $n+1$ est de taille $k+1$. Choisir $P$ dans $E_k$, c'est choisir $k$ entiers de $\llbracket 1, n \rrbracket$ (ceux de la partition de $P$ qui contient $n+1$), puis construire une partition des $n-k$ éléments restants. Donc $|E_k| = \binom{n}{k} B_{n-k}$.

		\medskip
		Comme $E_0, \dots, E_n$ forment une partition de l'ensemble des partitions de $\llbracket 1, n+1 \rrbracket$, on obtient :
		\[ B_{n+1} = \sum_{k=0}^n \binom{n}{k} B_{n-k} = \sum_{k=0}^n \binom{n}{n-k} B_k = \sum_{k=0}^n \binom{n}{k} B_k \tag{$*$} \]

		À toute partition $P$ de $\llbracket 1, n \rrbracket$, on peut associer une permutation $\sigma_P \in S_n$, qui est le produit des cycles de chaque partition de $P$. On construit ainsi une application
		\[
		\begin{array}{ccc}
			\llbracket 1, n \rrbracket &\rightarrow& S_n \\
			P &\mapsto& \sigma_P
		\end{array}
		\]
		injective. D'où :
		\[ B_n = |\llbracket 1, n \rrbracket| \leq |S_n| = n! \]
		On en déduit en particulier que $\frac{B_n}{n!} \leq 1$. En vertu du lemme d'Abel, le rayon de convergence $R$ de la série entière $\sum \frac{B_n}{n!} x^n$ est supérieur ou égal à $1$. On peut donc définir
		\[
		B :
		\begin{array}{ccc}
			]-R,R[ &\rightarrow& \mathbb{R} \\
			x &\mapsto& \sum_{n=0}^{+\infty} \frac{B_n}{n!} x^n
		\end{array}
		\]
		et en dérivant, $\forall x \in ]-R,R[$ :
		\begin{align*}
			B'(x) &= \sum_{n=0}^{+\infty} \frac{B_{n+1}}{n!} x^n \\
			&\overset{(*)}{=} \sum_{n=0}^{+\infty} \frac{1}{n!} \left( \sum_{k=0}^n \binom{n}{k} B_k \right) x^n \\
			&= \sum_{n=0}^{+\infty} \left( \sum_{k=0}^n \frac{B_k}{k!} \frac{1}{(n-k)!} \right) x^n
		\end{align*}
		On reconnaît là le produit de Cauchy suivant :
		\[ B'(x) = \sum_{n=0}^{+\infty} \left( \sum_{k=0}^n \frac{B_k}{k!} \frac{1}{(n-k)!} \right) x^n = \left( \sum_{n=0}^{+\infty} \frac{B_n}{n!} x^n \right) \left( \sum_{n=0}^{+\infty} \frac{x^n}{n!} \right) = B(x) e^x \]
		Reste à résoudre cette équation différentielle linéaire homogène d'ordre $1$ :
		\[ B(x) = \lambda e^{e^x} \]
		Or, $B(0) = B_0 = 1 = \lambda e^1$. D'où $B(x) = \frac{1}{e} e^{e^x}$.

		\medskip
		La série entière définissant l'exponentielle a un rayon de convergence infini. On peut donc écrire, pour tout $z \in \mathbb{C}$ :
		\[ e^{e^z} = \sum_{n=0}^{+\infty} \frac{e^{nz}}{n!} = \sum_{n=0}^{+\infty} \sum_{k=0}^{+\infty} \underbrace{\frac{(nz)^k}{n!k!}}_{u_{n,k}(z)} \]
		On va appliquer le théorème de Fubini-Lebesgue à $u_{n,k}(z)$ (où $z \in \mathbb{C}$ est fixé) :
		\[ \sum_{n=0}^{+\infty} \sum_{k=0}^{+\infty} |u_{n,k}(z)| = \sum_{n=0}^{+\infty} \frac{e^{n|z|}}{n!} = e^{e|z|} < +\infty \]
		Donc on peut intervertir les signes de sommations. Pour tout $x \in ]-R,R[$,
		\begin{align*}
			f(x) &= \frac{1}{e} e^{e^x} \\
			&= \frac{1}{e} \sum_{n=0}^{+\infty} \sum_{k=0}^{+\infty} u_{n,k}(x) \\
			&= \frac{1}{e} \sum_{k=0}^{+\infty} \sum_{n=0}^{+\infty} u_{n,k}(x) \\
			&= \frac{1}{e} \sum_{k=0}^{+\infty} \left( \sum_{n=0}^{+\infty} \frac{n^k}{n!} \right) \frac{x^k}{k!}
		\end{align*}
		Par unicité du développement en série entière d'une fonction, on en déduit, par identification des coefficients :
		\[ \forall k \in \mathbb{N}^*, \, B_k = \frac{1}{e} \sum_{n=0}^{+\infty} \frac{n^k}{n!} \]
	\end{proof}

	\begin{remark}
		La partie sur le dénombrement (au début de la preuve) est un peu technique. N'hésitez pas à passer du temps dessus et à y réfléchir en faisant des exemples.
	\end{remark}
	%</content>
\end{document}
