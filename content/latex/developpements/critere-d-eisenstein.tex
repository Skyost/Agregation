\documentclass[12pt, a4paper]{report}

% LuaLaTeX :

\RequirePackage{iftex}
\RequireLuaTeX

% Packages :

\usepackage[french]{babel}
%\usepackage[utf8]{inputenc}
%\usepackage[T1]{fontenc}
\usepackage[pdfencoding=auto, pdfauthor={Hugo Delaunay}, pdfsubject={Mathématiques}, pdfcreator={agreg.skyost.eu}]{hyperref}
\usepackage{amsmath}
\usepackage{amsthm}
%\usepackage{amssymb}
\usepackage{stmaryrd}
\usepackage{tikz}
\usepackage{tkz-euclide}
\usepackage{fourier-otf}
\usepackage{fontspec}
\usepackage{titlesec}
\usepackage{fancyhdr}
\usepackage{catchfilebetweentags}
\usepackage[french, capitalise, noabbrev]{cleveref}
\usepackage[fit, breakall]{truncate}
\usepackage[top=2.5cm, right=2cm, bottom=2.5cm, left=2cm]{geometry}
\usepackage{enumerate}
\usepackage{tocloft}
\usepackage{microtype}
%\usepackage{mdframed}
%\usepackage{thmtools}
\usepackage{xcolor}
\usepackage{tabularx}
\usepackage{aligned-overset}
\usepackage[subpreambles=true]{standalone}
\usepackage{environ}
\usepackage[normalem]{ulem}
\usepackage{marginnote}
\usepackage{etoolbox}
\usepackage{setspace}
\usepackage[bibstyle=reading, citestyle=draft]{biblatex}
\usepackage{xpatch}
\usepackage[many, breakable]{tcolorbox}
\usepackage[backgroundcolor=white, bordercolor=white, textsize=small]{todonotes}

% Bibliographie :

\newcommand{\overridebibliographypath}[1]{\providecommand{\bibliographypath}{#1}}
\overridebibliographypath{../bibliography.bib}
\addbibresource{\bibliographypath}
\defbibheading{bibliography}[\bibname]{%
	\newpage
	\section*{#1}%
}
\renewbibmacro*{entryhead:full}{\printfield{labeltitle}}
\DeclareFieldFormat{url}{\newline\footnotesize\url{#1}}
\AtEndDocument{\printbibliography}

% Police :

\setmathfont{Erewhon Math}

% Tikz :

\usetikzlibrary{calc}

% Longueurs :

\setlength{\parindent}{0pt}
\setlength{\headheight}{15pt}
\setlength{\fboxsep}{0pt}
\titlespacing*{\chapter}{0pt}{-20pt}{10pt}
\setlength{\marginparwidth}{1.5cm}
\setstretch{1.1}

% Métadonnées :

\author{agreg.skyost.eu}
\date{\today}

% Titres :

\setcounter{secnumdepth}{3}

\renewcommand{\thechapter}{\Roman{chapter}}
\renewcommand{\thesubsection}{\Roman{subsection}}
\renewcommand{\thesubsubsection}{\arabic{subsubsection}}
\renewcommand{\theparagraph}{\alph{paragraph}}

\titleformat{\chapter}{\huge\bfseries}{\thechapter}{20pt}{\huge\bfseries}
\titleformat*{\section}{\LARGE\bfseries}
\titleformat{\subsection}{\Large\bfseries}{\thesubsection \, - \,}{0pt}{\Large\bfseries}
\titleformat{\subsubsection}{\large\bfseries}{\thesubsubsection. \,}{0pt}{\large\bfseries}
\titleformat{\paragraph}{\bfseries}{\theparagraph. \,}{0pt}{\bfseries}

\setcounter{secnumdepth}{4}

% Table des matières :

\renewcommand{\cftsecleader}{\cftdotfill{\cftdotsep}}
\addtolength{\cftsecnumwidth}{10pt}

% Redéfinition des commandes :

\renewcommand*\thesection{\arabic{section}}
\renewcommand{\ker}{\mathrm{Ker}}

% Nouvelles commandes :

\newcommand{\website}{https://agreg.skyost.eu}

\newcommand{\tr}[1]{\mathstrut ^t #1}
\newcommand{\im}{\mathrm{Im}}
\newcommand{\rang}{\operatorname{rang}}
\newcommand{\trace}{\operatorname{trace}}
\newcommand{\id}{\operatorname{id}}
\newcommand{\stab}{\operatorname{Stab}}

\providecommand{\newpar}{\\[\medskipamount]}

\providecommand{\lesson}[3]{%
	\title{#3}%
	\hypersetup{pdftitle={#3}}%
	\setcounter{section}{\numexpr #2 - 1}%
	\section{#3}%
	\fancyhead[R]{\truncate{0.73\textwidth}{#2 : #3}}%
}

\providecommand{\development}[3]{%
	\title{#3}%
	\hypersetup{pdftitle={#3}}%
	\section*{#3}%
	\fancyhead[R]{\truncate{0.73\textwidth}{#3}}%
}

\providecommand{\summary}[1]{%
	\textit{#1}%
	\medskip%
}

\tikzset{notestyleraw/.append style={inner sep=0pt, rounded corners=0pt, align=center}}

%\newcommand{\booklink}[1]{\website/bibliographie\##1}
\newcommand{\citelink}[2]{\hyperlink{cite.\therefsection @#1}{#2}}
\newcommand{\previousreference}{}
\providecommand{\reference}[2][]{%
	\notblank{#1}{\renewcommand{\previousreference}{#1}}{}%
	\todo[noline]{%
		\protect\vspace{16pt}%
		\protect\par%
		\protect\notblank{#1}{\cite{[\previousreference]}\\}{}%
		\protect\citelink{\previousreference}{p. #2}%
	}%
}

\definecolor{devcolor}{HTML}{00695c}
\newcommand{\dev}[1]{%
	\reversemarginpar%
	\todo[noline]{
		\protect\vspace{16pt}%
		\protect\par%
		\bfseries\color{devcolor}\href{\website/developpements/#1}{DEV}
	}%
	\normalmarginpar%
}

% En-têtes :

\pagestyle{fancy}
\fancyhead[L]{\truncate{0.23\textwidth}{\thepage}}
\fancyfoot[C]{\scriptsize \href{\website}{\texttt{agreg.skyost.eu}}}

% Couleurs :

\definecolor{property}{HTML}{fffde7}
\definecolor{proposition}{HTML}{fff8e1}
\definecolor{lemma}{HTML}{fff3e0}
\definecolor{theorem}{HTML}{fce4f2}
\definecolor{corollary}{HTML}{ffebee}
\definecolor{definition}{HTML}{ede7f6}
\definecolor{notation}{HTML}{f3e5f5}
\definecolor{example}{HTML}{e0f7fa}
\definecolor{cexample}{HTML}{efebe9}
\definecolor{application}{HTML}{e0f2f1}
\definecolor{remark}{HTML}{e8f5e9}
\definecolor{proof}{HTML}{e1f5fe}

% Théorèmes :

\theoremstyle{definition}
\newtheorem{theorem}{Théorème}

\newtheorem{property}[theorem]{Propriété}
\newtheorem{proposition}[theorem]{Proposition}
\newtheorem{lemma}[theorem]{Lemme}
\newtheorem{corollary}[theorem]{Corollaire}

\newtheorem{definition}[theorem]{Définition}
\newtheorem{notation}[theorem]{Notation}

\newtheorem{example}[theorem]{Exemple}
\newtheorem{cexample}[theorem]{Contre-exemple}
\newtheorem{application}[theorem]{Application}

\theoremstyle{remark}
\newtheorem{remark}[theorem]{Remarque}

\counterwithin*{theorem}{section}

\newcommand{\applystyletotheorem}[1]{
	\tcolorboxenvironment{#1}{
		enhanced,
		breakable,
		colback=#1!98!white,
		boxrule=0pt,
		boxsep=0pt,
		left=8pt,
		right=8pt,
		top=8pt,
		bottom=8pt,
		sharp corners,
		after=\par,
	}
}

\applystyletotheorem{property}
\applystyletotheorem{proposition}
\applystyletotheorem{lemma}
\applystyletotheorem{theorem}
\applystyletotheorem{corollary}
\applystyletotheorem{definition}
\applystyletotheorem{notation}
\applystyletotheorem{example}
\applystyletotheorem{cexample}
\applystyletotheorem{application}
\applystyletotheorem{remark}
\applystyletotheorem{proof}

% Environnements :

\NewEnviron{whitetabularx}[1]{%
	\renewcommand{\arraystretch}{2.5}
	\colorbox{white}{%
		\begin{tabularx}{\textwidth}{#1}%
			\BODY%
		\end{tabularx}%
	}%
}

% Maths :

\DeclareFontEncoding{FMS}{}{}
\DeclareFontSubstitution{FMS}{futm}{m}{n}
\DeclareFontEncoding{FMX}{}{}
\DeclareFontSubstitution{FMX}{futm}{m}{n}
\DeclareSymbolFont{fouriersymbols}{FMS}{futm}{m}{n}
\DeclareSymbolFont{fourierlargesymbols}{FMX}{futm}{m}{n}
\DeclareMathDelimiter{\VERT}{\mathord}{fouriersymbols}{152}{fourierlargesymbols}{147}


% Bibliographie :

\addbibresource{\bibliographypath}%
\defbibheading{bibliography}[\bibname]{%
	\newpage
	\section*{#1}%
}
\renewbibmacro*{entryhead:full}{\printfield{labeltitle}}%
\DeclareFieldFormat{url}{\newline\footnotesize\url{#1}}%

\AtEndDocument{\printbibliography}

\begin{document}
	%<*content>
	\development{algebra}{critere-d-eisenstein}{Critère d'Eisenstein}

	\summary{Ici, nous démontrons le célèbre critère d'Eisenstein que l'on utilise énormément en pratique pour montrer qu'un polynôme est irréductible.}

	Soit $A$ un anneau commutatif et unitaire.

	\begin{notation}
		Soit $P \in A[X]$. On note $\gamma(P)$ le contenu du polynôme $P$.
	\end{notation}

	\begin{lemma}
		\label{critere-d-eisenstein-1}
		Soit $p \in A$ tel que $(p)$ est premier. Alors $A/(p)$ est intègre.
	\end{lemma}

	\begin{proof}
		Soient $\overline{a}, \overline{b} \in A/(p)$. On suppose $\overline{a} \overline{b} = 0$. Comme $\overline{a} \overline{b} = \overline{ab}$, on a $ab \in (p)$. Donc par hypothèse,
		\begin{align*}
			&a \in (p) \text{ ou } b \in (p) \\
			\implies& \overline{a} = 0 \text{ ou } \overline{b} = 0
		\end{align*}
		et ainsi $A/(p)$ est bien intègre.
	\end{proof}

	\begin{lemma}
		\label{critere-d-eisenstein-2}
		Si $A$ est intègre, alors $A[X]$ l'est aussi.
	\end{lemma}

	\begin{proof}
		Soient $P, Q \in A[X]$ non nuls, de degrés respectifs $n \geq 1$ et $m \geq 1$ que l'on écrit $P = \sum_{i=0}^n a_i X^i$ et $Q = \sum_{j=0}^m b_j X^j$. Alors, le coefficient de $X^{n+m}$ dans le produit $PQ$ est $a_n b_m$. Comme $a_n \neq 0$, $b_m \neq 0$ et $A$ est intègre, ce coefficient est non nul. Donc en particulier, le produit $PQ$ est non nul.
	\end{proof}

	\reference[ULM18]{64}

	\begin{lemma}
		\label{critere-d-eisenstein-3}
		On suppose $A$ factoriel. Soit $a \in A$ irréductible. Alors $(a)$ est premier.
	\end{lemma}

	\begin{proof}
		On suppose que $a \mid bc$ avec $b, c \in A$. Alors, il existe $d \in A$ tel que
		\[ ad = bc \tag{$*$} \]
		Si $b$ est inversible, alors $a \mid c$. De même, si $c$ est inversible, alors $a \mid b$. Supposons donc que $b$ et $c$ ne sont pas inversibles. Comme $a$ est irréductible, on en déduit que $d$ est un élément non nul et non inversible de $A$. Il existe donc des décompositions en irréductibles
		\[ b = \beta_1 \dots \beta_n, \, c = \gamma_1 \dots \gamma_m \text{ et } d = \delta_1 \dots \delta_k \]
		avec $n, m, k \in \mathbb{N}^*$. Par conséquent, en injectant dans $(*)$ :
		\[ a \delta_1 \dots \delta_k = \beta_1 \dots \beta_n \gamma_1 \dots \gamma_m \]
		Comme la factorisation en irréductibles est unique à l'ordre près, il existe $\beta_i$ ou $\gamma_j$ qui est associé à $a$. Si bien que $a$ divise $b$ ou $c$ ; c'est ce que l'on voulait démontrer.
	\end{proof}

	\reference[GOZ]{10}

	\begin{lemma}[Gauss]
		\label{critere-d-eisenstein-4}
		On suppose $A$ factoriel. Alors :
		\begin{enumerate}[(i)]
			\item Le produit de deux polynômes primitifs est primitif.
			\item $\forall P, Q \in A[X] \setminus \{ 0 \}$, $\gamma(PQ) = \gamma(P) \gamma(Q)$.
		\end{enumerate}
	\end{lemma}

	\begin{proof}
		\begin{enumerate}[(i)]
			\item Soient $P, Q \in A[X]$ tels que $\gamma(P) = \gamma(Q) = 1$. Supposons $\gamma(PQ) \neq 1$. Alors, il existe $p \in A$ irréductible tel que $p$ divise tous les coefficients de $PQ$. Donc, dans $A/(p)$, $\overline{PQ} = \overline{P} \, \overline{Q} = 0$.
			\newpar
			Mais, par le \cref{critere-d-eisenstein-3}, $(p)$ est premier. Donc par le \cref{critere-d-eisenstein-1} $A/(p)$ est intègre, et en particulier, $A/(p)[X]$ l'est aussi par le \cref{critere-d-eisenstein-2}. Ainsi, $\overline{P} = 0$ ou $\overline{Q} = 0$ : absurde.
			\item En factorisant, on écrit $P = \gamma(P)R$ et $Q = \gamma(Q)S$ où $R, S \in A[X]$ avec $\gamma(R) = \gamma(S) = 1$. D'où $PQ = \gamma(P)\gamma(Q)RS$ avec $\gamma(RS) = 1$ par $(i)$. Ainsi, $\gamma(PQ) = \gamma(P) \gamma(Q).$
		\end{enumerate}
	\end{proof}

	\begin{theorem}[Critère d'Eisenstein]
		\label{critere-d-eisenstein-5}
		Soient $\mathbb{K}$ le corps des fractions de $A$ et $P = \sum_{i=0}^n a_i X^i \in A[X]$ de degré $n \geq 1$. On suppose que $A$ est factoriel et qu'il existe $p \in A$ irréductible tel que :
		\begin{enumerate}[(i)]
			\item $p \mid a_i$, $\forall i \in \llbracket 0, n-1 \rrbracket$.
			\item $p \nmid a_n$.
			\item $p^2 \nmid a_0$.
		\end{enumerate}
		Alors $P$ est irréductible dans $\mathbb{K}[X]$.
	\end{theorem}

	\begin{proof}
		Par l'absurde, on suppose $P = UV$ avec $U, V \in \mathbb{K}[X]$ de degré supérieur ou égal à $1$. Soit $a$ un multiple commun à tous les dénominateurs des coefficients non nuls de $U$ et $V$. On a
		\[ a^2 P = \underbrace{a U}_{\substack{= U_1 \\ \in A[X]}} \underbrace{a V}_{\substack{= V_1 \\ \in A[X]}} \]
		On applique le \cref{critere-d-eisenstein-4} pour obtenir :
		\[ a^2 \gamma(P) = \gamma(U_1) \gamma(V_1) \tag{$*$} \]
		En factorisant, on écrit $U_1 = \gamma(U_1) U_2$ et $V_1 = \gamma(V_1) V_2$ avec $U_2, V_2 \in A[X]$. Il vient :
		\[ a^2 P = \gamma(U_1) \gamma(V_1) U_2 V_2 \overset{(*)}{=} a^2 \gamma(P) U_2 V_2 \]
		Et comme $a \in A \setminus \{ 0 \}$ et que $A$ est intègre, on a $P = \gamma(P) U_2 V_2$ avec $U_2$, $V_2 \in A[X]$ de degré supérieur ou égal à $1$.
		\newpar
		On pose $U_2 = \sum_{i=0}^r b_i X^i$ et $V_2 = \sum_{j=0}^s c_j X^j$ avec $b_r c_s = a_n \neq 0$ par définition de $P$. Dans $A/(p)$, on a
		\[ \underbrace{\overline{P}}_{= \overline{a_n} X^n} = \overline{U_2 V_2} = \overline{U_2} \, \overline{V_2} \]
		et en particulier, le terme de degré $0$, $\overline{b_0 c_0} = \overline{b_0} \overline{c_0}$ est nul. Mais, $p$ est irréductible et $A$ est factoriel, donc au vu du \cref{critere-d-eisenstein-3}, $(p)$ est premier et $A/(p)$ est intègre par le \cref{critere-d-eisenstein-1}. Donc par le \cref{critere-d-eisenstein-2}, $A/(p)[X]$ est aussi intègre. D'où $\overline{b_0} = 0$ ou $\overline{c_0} = 0$ (mais pas les deux car sinon $p^2 \mid b_0 c_0 = a_0$, ce qui serait en contradiction avec $(iii)$).
		\newpar
		On suppose donc $\overline{b_0} = 0$ et $\overline{c_0} \neq 0$. Si on avait $\forall i \in \llbracket 0, r \rrbracket$, $\overline{b_i} = 0$, on aurait en particulier $\overline{b_r} = 0$, et donc $\overline{b_r} \overline{c_s} = \overline{a_n} = 0$ (exclu par $(ii)$). Donc,
		\[ \exists i \in \llbracket 0, r-1 \rrbracket \text{ tel que } \overline{b_0} = \dots = \overline{b_i} = 0 \text{ et } b_{i+1} \neq 0 \]
		Ainsi,
		\[ \overline{a_{i+1}} = \sum_{k=0}^{i+1} \overline{b_k} \overline{c_{i+1-k}} = \underbrace{\overline{b_{i+1}}}_{\neq 0} \underbrace{\overline{c_0}}_{\neq 0} \neq 0 \]
		ce qui est absurde au vu de la condition $(i)$ car $i \in \llbracket 0, r-1 \rrbracket$ avec $r-1 \leq n-1$.
	\end{proof}

	\reference[PER]{67}

	\begin{application}
		Soit $n \in \mathbb{N}^*$. Il existe des polynômes irréductibles de degré $n$ sur $\mathbb{Z}$.
	\end{application}

	\begin{proof}
		On applique le \cref{critere-d-eisenstein-5} au polynôme $P = X^n - 2$ avec le premier $p = 2$ qui nous donne l'irréductibilité du polynôme sur $\mathbb{Q}$. Reste à montrer qu'il est irréductible sur $\mathbb{Z}$.
		\newpar
		Or, en supposant $P$ réductible sur $\mathbb{Z}$, on peut écrire $P = QR$ avec $Q, R \in \mathbb{Z}[X]$ de degré supérieur ou égal à $1$ car $P$ est primitif. Mais à fortiori, $Q, R \in \mathbb{Q}[X]$ et ne sont pas inversibles donc $P$ est réductible sur $\mathbb{Q}$ : absurde.
	\end{proof}
	%</content>
\end{document}
