\documentclass[12pt, a4paper]{report}

% LuaLaTeX :

\RequirePackage{iftex}
\RequireLuaTeX

% Packages :

\usepackage[french]{babel}
%\usepackage[utf8]{inputenc}
%\usepackage[T1]{fontenc}
\usepackage[pdfencoding=auto, pdfauthor={Hugo Delaunay}, pdfsubject={Mathématiques}, pdfcreator={agreg.skyost.eu}]{hyperref}
\usepackage{amsmath}
\usepackage{amsthm}
%\usepackage{amssymb}
\usepackage{stmaryrd}
\usepackage{tikz}
\usepackage{tkz-euclide}
\usepackage{fourier-otf}
\usepackage{fontspec}
\usepackage{titlesec}
\usepackage{fancyhdr}
\usepackage{catchfilebetweentags}
\usepackage[french, capitalise, noabbrev]{cleveref}
\usepackage[fit, breakall]{truncate}
\usepackage[top=2.5cm, right=2cm, bottom=2.5cm, left=2cm]{geometry}
\usepackage{enumerate}
\usepackage{tocloft}
\usepackage{microtype}
%\usepackage{mdframed}
%\usepackage{thmtools}
\usepackage{xcolor}
\usepackage{tabularx}
\usepackage{aligned-overset}
\usepackage[subpreambles=true]{standalone}
\usepackage{environ}
\usepackage[normalem]{ulem}
\usepackage{marginnote}
\usepackage{etoolbox}
\usepackage{setspace}
\usepackage[bibstyle=reading, citestyle=draft]{biblatex}
\usepackage{xpatch}
\usepackage[many, breakable]{tcolorbox}
\usepackage[backgroundcolor=white, bordercolor=white, textsize=small]{todonotes}

% Bibliographie :

\newcommand{\overridebibliographypath}[1]{\providecommand{\bibliographypath}{#1}}
\overridebibliographypath{../bibliography.bib}
\addbibresource{\bibliographypath}
\defbibheading{bibliography}[\bibname]{%
	\newpage
	\section*{#1}%
}
\renewbibmacro*{entryhead:full}{\printfield{labeltitle}}
\DeclareFieldFormat{url}{\newline\footnotesize\url{#1}}
\AtEndDocument{\printbibliography}

% Police :

\setmathfont{Erewhon Math}

% Tikz :

\usetikzlibrary{calc}

% Longueurs :

\setlength{\parindent}{0pt}
\setlength{\headheight}{15pt}
\setlength{\fboxsep}{0pt}
\titlespacing*{\chapter}{0pt}{-20pt}{10pt}
\setlength{\marginparwidth}{1.5cm}
\setstretch{1.1}

% Métadonnées :

\author{agreg.skyost.eu}
\date{\today}

% Titres :

\setcounter{secnumdepth}{3}

\renewcommand{\thechapter}{\Roman{chapter}}
\renewcommand{\thesubsection}{\Roman{subsection}}
\renewcommand{\thesubsubsection}{\arabic{subsubsection}}
\renewcommand{\theparagraph}{\alph{paragraph}}

\titleformat{\chapter}{\huge\bfseries}{\thechapter}{20pt}{\huge\bfseries}
\titleformat*{\section}{\LARGE\bfseries}
\titleformat{\subsection}{\Large\bfseries}{\thesubsection \, - \,}{0pt}{\Large\bfseries}
\titleformat{\subsubsection}{\large\bfseries}{\thesubsubsection. \,}{0pt}{\large\bfseries}
\titleformat{\paragraph}{\bfseries}{\theparagraph. \,}{0pt}{\bfseries}

\setcounter{secnumdepth}{4}

% Table des matières :

\renewcommand{\cftsecleader}{\cftdotfill{\cftdotsep}}
\addtolength{\cftsecnumwidth}{10pt}

% Redéfinition des commandes :

\renewcommand*\thesection{\arabic{section}}
\renewcommand{\ker}{\mathrm{Ker}}

% Nouvelles commandes :

\newcommand{\website}{https://agreg.skyost.eu}

\newcommand{\tr}[1]{\mathstrut ^t #1}
\newcommand{\im}{\mathrm{Im}}
\newcommand{\rang}{\operatorname{rang}}
\newcommand{\trace}{\operatorname{trace}}
\newcommand{\id}{\operatorname{id}}
\newcommand{\stab}{\operatorname{Stab}}

\providecommand{\newpar}{\\[\medskipamount]}

\providecommand{\lesson}[3]{%
	\title{#3}%
	\hypersetup{pdftitle={#3}}%
	\setcounter{section}{\numexpr #2 - 1}%
	\section{#3}%
	\fancyhead[R]{\truncate{0.73\textwidth}{#2 : #3}}%
}

\providecommand{\development}[3]{%
	\title{#3}%
	\hypersetup{pdftitle={#3}}%
	\section*{#3}%
	\fancyhead[R]{\truncate{0.73\textwidth}{#3}}%
}

\providecommand{\summary}[1]{%
	\textit{#1}%
	\medskip%
}

\tikzset{notestyleraw/.append style={inner sep=0pt, rounded corners=0pt, align=center}}

%\newcommand{\booklink}[1]{\website/bibliographie\##1}
\newcommand{\citelink}[2]{\hyperlink{cite.\therefsection @#1}{#2}}
\newcommand{\previousreference}{}
\providecommand{\reference}[2][]{%
	\notblank{#1}{\renewcommand{\previousreference}{#1}}{}%
	\todo[noline]{%
		\protect\vspace{16pt}%
		\protect\par%
		\protect\notblank{#1}{\cite{[\previousreference]}\\}{}%
		\protect\citelink{\previousreference}{p. #2}%
	}%
}

\definecolor{devcolor}{HTML}{00695c}
\newcommand{\dev}[1]{%
	\reversemarginpar%
	\todo[noline]{
		\protect\vspace{16pt}%
		\protect\par%
		\bfseries\color{devcolor}\href{\website/developpements/#1}{DEV}
	}%
	\normalmarginpar%
}

% En-têtes :

\pagestyle{fancy}
\fancyhead[L]{\truncate{0.23\textwidth}{\thepage}}
\fancyfoot[C]{\scriptsize \href{\website}{\texttt{agreg.skyost.eu}}}

% Couleurs :

\definecolor{property}{HTML}{fffde7}
\definecolor{proposition}{HTML}{fff8e1}
\definecolor{lemma}{HTML}{fff3e0}
\definecolor{theorem}{HTML}{fce4f2}
\definecolor{corollary}{HTML}{ffebee}
\definecolor{definition}{HTML}{ede7f6}
\definecolor{notation}{HTML}{f3e5f5}
\definecolor{example}{HTML}{e0f7fa}
\definecolor{cexample}{HTML}{efebe9}
\definecolor{application}{HTML}{e0f2f1}
\definecolor{remark}{HTML}{e8f5e9}
\definecolor{proof}{HTML}{e1f5fe}

% Théorèmes :

\theoremstyle{definition}
\newtheorem{theorem}{Théorème}

\newtheorem{property}[theorem]{Propriété}
\newtheorem{proposition}[theorem]{Proposition}
\newtheorem{lemma}[theorem]{Lemme}
\newtheorem{corollary}[theorem]{Corollaire}

\newtheorem{definition}[theorem]{Définition}
\newtheorem{notation}[theorem]{Notation}

\newtheorem{example}[theorem]{Exemple}
\newtheorem{cexample}[theorem]{Contre-exemple}
\newtheorem{application}[theorem]{Application}

\theoremstyle{remark}
\newtheorem{remark}[theorem]{Remarque}

\counterwithin*{theorem}{section}

\newcommand{\applystyletotheorem}[1]{
	\tcolorboxenvironment{#1}{
		enhanced,
		breakable,
		colback=#1!98!white,
		boxrule=0pt,
		boxsep=0pt,
		left=8pt,
		right=8pt,
		top=8pt,
		bottom=8pt,
		sharp corners,
		after=\par,
	}
}

\applystyletotheorem{property}
\applystyletotheorem{proposition}
\applystyletotheorem{lemma}
\applystyletotheorem{theorem}
\applystyletotheorem{corollary}
\applystyletotheorem{definition}
\applystyletotheorem{notation}
\applystyletotheorem{example}
\applystyletotheorem{cexample}
\applystyletotheorem{application}
\applystyletotheorem{remark}
\applystyletotheorem{proof}

% Environnements :

\NewEnviron{whitetabularx}[1]{%
	\renewcommand{\arraystretch}{2.5}
	\colorbox{white}{%
		\begin{tabularx}{\textwidth}{#1}%
			\BODY%
		\end{tabularx}%
	}%
}

% Maths :

\DeclareFontEncoding{FMS}{}{}
\DeclareFontSubstitution{FMS}{futm}{m}{n}
\DeclareFontEncoding{FMX}{}{}
\DeclareFontSubstitution{FMX}{futm}{m}{n}
\DeclareSymbolFont{fouriersymbols}{FMS}{futm}{m}{n}
\DeclareSymbolFont{fourierlargesymbols}{FMX}{futm}{m}{n}
\DeclareMathDelimiter{\VERT}{\mathord}{fouriersymbols}{152}{fourierlargesymbols}{147}


% Bibliographie :

\addbibresource{\bibliographypath}%
\defbibheading{bibliography}[\bibname]{%
	\newpage
	\section*{#1}%
}
\renewbibmacro*{entryhead:full}{\printfield{labeltitle}}%
\DeclareFieldFormat{url}{\newline\footnotesize\url{#1}}%

\AtEndDocument{\printbibliography}

\begin{document}
  %<*content>
  \sheet{analysis}{autour-de-la-compacite}{Autour de la compacité}

  \summary{En utilisant la compacité, on montre diverses propriétés des espaces métriques et des espaces vectoriels normés, notamment de dimension finie.}

  \nocite{[DAN]}
  \nocite{[GOU20]}

  \begin{proposition}
    \label{autour-de-la-compacite-1}
    Soient $(E,d_E)$, $(F,d_f)$ deux espaces métriques et $f : E \rightarrow F$ continue. Si $E$ est compact, alors $f(E)$ est compact dans $F$.
  \end{proposition}

  \begin{proof}
    Soit $(y_n)$ une suite d'éléments de $f(E)$. On pose $\forall n \in \mathbb{N}$, $x_n = f(y_n)$. $E$ est compact, donc il existe une extractrice $\varphi : \mathbb{N} \rightarrow \mathbb{N}$ telle que $x_{\varphi(n)} \longrightarrow_{n \rightarrow +\infty} x$ où $x \in E$. Par continuité,
    \[ y_{\varphi(n)} = f(x_{\varphi(n)}) \longrightarrow_{n \rightarrow +\infty} f(x) \in f(E) \]
    $f(E)$ est ainsi séquentiellement compact, donc est compact.
  \end{proof}

  \begin{proposition}
    \label{autour-de-la-compacite-2}
    Soit $(E,d)$ un espace métrique. Si $A \subseteq E$ est compacte, alors $A$ est fermée et bornée.
  \end{proposition}

  \begin{proof}
    \begin{itemize}
      \item \uline{Fermée :} Soit $(a_n)$ une suite d'éléments de $A$ qui converge vers $a \in E$. Par compacité, il existe une extractrice $\varphi : \mathbb{N} \rightarrow \mathbb{N}$ telle que $a_{\varphi(n)} \longrightarrow_{n \rightarrow +\infty} a'$ où $a' \in A$. Par unicité de la limite dans un espace métrique, $a' = a \in A$. Par la caractérisation séquentielle des fermés, $A$ est bien fermée.
      \item \uline{Bornée :} Soit $a \in A$. On pose $B = \{ d(a,x) \mid x \in A \}$ et on suppose par l'absurde que $B$ est non borné. Il existe une suite $(a_n)$ telle que
      \[ \forall n \in \mathbb{N}, \, d(a, a_n) \geq n \]
      Par compacité, il existe une extractrice $\varphi : \mathbb{N} \rightarrow \mathbb{N}$ telle que $a_{\varphi(n)} \longrightarrow_{n \rightarrow +\infty} \ell$ où $\ell \in A$. Par continuité,
      \[ d(a, a_{\varphi(n)}) \longrightarrow_{n \rightarrow +\infty} d(a, \ell) \]
      Mais, pour tout $n \in \mathbb{N}$, $d(a, a_{\varphi(n)}) \geq \varphi(n) \geq n$ : absurde. Donc $B$ est borné : il existe $r \geq 0$ tel que $d(a,x) \leq r$ pour tout $x \in A$.
    \end{itemize}
  \end{proof}

  \begin{proposition}
    \label{autour-de-la-compacite-3}
    Soir $E$ un espace vectoriel de dimension finie $n \geq 1$ muni d'une norme infinie $\Vert . \Vert_\infty$. Les compacts de cet espace vectoriel normé sont les parties fermées et bornées.
  \end{proposition}

  \begin{proof}
    La \cref{autour-de-la-compacite-2} montre que les parties compactes sont fermées et bornées. Pour montrer la réciproque, prenons $r > 0$. Notons que l'intervalle $[-r,r]$ est compact : si $(a_k)$ est une suite d'éléments de $[-r,r]$, on peut extraire une sous-suite monotone et bornée qui est alors convergente dans $[-r,r]$ car $[-r,r]$ est fermé. Le théorème de Tykhonov nous dit que le produit $[-r,r]^n$ est alors compact.
    \newpar
    Posons
    \[
      \varphi :
      \begin{array}{ccc}
        ([-r, r]^n, \Vert . \Vert_\infty) &\rightarrow& (E, \Vert . \Vert_\infty) \\
        (\alpha_1, \dots, \alpha_n) &\mapsto& \sum_{k=1}^n \alpha_k e_k
      \end{array}
    \]
    où $(e_1, \dots, e_n)$ désigne une base de $E$ associée à la norme infinie $\Vert . \Vert_\infty$. Alors, par la \cref{autour-de-la-compacite-1}, $\varphi([-r, r]^n) = \overline{B}(0,r)$ est compact.
    \newpar
    Soit maintenant $A$ une partie fermée bornée de $E$. Alors il existe $r > 0$ tel que $A \subseteq \overline{B}(0,r)$. Donc, si $(a_n)$ est une suite d'éléments de $A$, par compacité de $\overline{B}(0,r)$, on a l'existence d'une sous-suite convergente vers $a \in \overline{B}(0,r)$. Comme $A$ est fermée, $a \in A$. $A$ est ainsi séquentiellement compacte, donc est compacte.
  \end{proof}

  \begin{theorem}
    \label{autour-de-la-compacite-4}
    Un espace vectoriel normé $E$ est de dimension finie $n \geq 1$ si et seulement si toutes ses normes sont équivalentes.
  \end{theorem}

  \begin{proof}
    \begin{itemize}
      \item \uline{$\impliedby$ :} Soit $\Vert . \Vert$ une norme sur $E$ et soit $\varphi$ une forme linéaire quelconque sur $E$. On définit la norme suivante sur $E$ :
      \[ \Vert . \Vert_{\varphi} : x \mapsto \vert \varphi(x) \vert + \Vert x \Vert \]
      Alors, pour tout $x \in E$, $\vert \varphi(x) \vert = \Vert x \Vert_{\varphi} - \Vert x \Vert \leq \Vert x \Vert_{\varphi}$ : $\varphi$ est continue pour $\Vert . \Vert_{\varphi}$ donc pour $\Vert . \Vert$ aussi par équivalence des normes.
      \newpar
      Supposons par l'absurde $E$ de dimension infinie. Soit $(e_n)_{n \in \mathbb{N}}$ une suite infinie de vecteurs linéairement indépendantes. On pose $V = \operatorname{Vect}(e_n)_{n \in \mathbb{N}}$. Soient $W$ un supplémentaire de $V$ dans $E$ et $p : E \rightarrow V$ la projection sur $V$ parallèlement à $W$. On définit $\psi$ une forme linéaire sur $V$ par $\forall n \in \mathbb{N}$, $\psi (e_n) = n \Vert e_n \Vert$. Alors, $\phi = \psi \circ p$ est une forme linéaire sur $E$ qui n'est pas continue. En effet :
      \[ \sup_{x \neq 0} \frac{\vert \phi(x) \vert}{\Vert x \Vert} = +\infty \]
      C'est absurde.
      \item \uline{$\implies$ :}
      Soient $(e_1, \dots, e_n)$ une base de $E$ et $x = \sum_{i=1}^n x_i e_i \in E$. Si $\Vert . \Vert$ est une norme sur $E$, on a :
      \[ \Vert x \Vert \leq \underbrace{\left( \sum_{i=1}^n \Vert e_i \Vert \right)}_{= \alpha} \Vert x \Vert_\infty \]
      Donc $\Vert . \Vert_\infty$ est plus fine que $\Vert . \Vert$.
      \newpar
      L'application $\Vert . \Vert : (E, \Vert . \Vert_\infty) \rightarrow (\mathbb{R}^+, \vert . \vert)$ est continue car lipschitzienne ($\forall x, y \in E, \, \vert \Vert x \Vert - \Vert y \Vert \vert \leq \Vert x - y \Vert$), donc est bornée et atteint ses bornes sur la sphère $S(0,1) = \{ x \in E \mid \Vert x \Vert_\infty = 1 \}$ (qui est fermée bornée, donc compacte par la \cref{autour-de-la-compacite-3}). On note $x_0 \in E$ ce minimum :
      \[ \forall x \in E \text{ tel que } \Vert x \Vert_\infty = 1, \text{ on a } \Vert x \Vert \geq \underbrace{\Vert x_0 \Vert}_{= \beta} \]
      Ainsi,
      \[ \forall x \in E, \left\Vert \frac{x}{\Vert x \Vert_\infty} \right\Vert \geq \beta \text{ ie. } \Vert x \Vert \geq \beta \Vert x \Vert_\infty \]
      Donc $\Vert . \Vert$ est plus fine que $\Vert . \Vert_\infty$ : les normes $\Vert . \Vert$ et $\Vert . \Vert_\infty$ sont équivalentes. Comme la relation d'équivalence sur les normes d'un espace vectoriel est transitive, on en déduit que toutes les normes sur $E$ sont équivalentes.
    \end{itemize}
  \end{proof}

  \begin{corollary}
    \label{autour-de-la-compacite-5}
    \begin{enumerate}[label=(\roman*)]
      \item \label{autour-de-la-compacite-5-1} Les parties compacts d'un espace vectoriel normé de dimension finie sont les parties fermées bornées.
      \item \label{autour-de-la-compacite-5-2} Tout espace vectoriel normé de dimension finie est complet.
      \item Tout sous-espace vectoriel de dimension finie d'un espace vectoriel normé est fermé.
      \item Soient $(E, \Vert . \Vert_E)$ et $(E, \Vert . \Vert_F)$ deux espaces vectoriels avec $E$ de dimension finie. Alors,
      \[ \mathcal{L}(E, F) = L(E,F) \]
      ie. toute application linéaire de $E$ dans $F$ est continue.
    \end{enumerate}
  \end{corollary}

  \begin{proof}
    \begin{enumerate}[label=(\roman*)]
      \item C'est une conséquence directe de la \cref{autour-de-la-compacite-3} et du \cref{autour-de-la-compacite-4}.
      \item Soit $(x_n)$ une suite de Cauchy d'un espace vectoriel normé $(E, \Vert . \Vert)$. Notons que :
      \begin{itemize}
        \item \textbf{$(x_n)$ est bornée.} En effet, il existe $N \in \mathbb{N}$ tel que $\forall p > q \geq N$, $\Vert x_p - x_q \Vert < 1$. Donc, $\forall p \geq N$, $\Vert x_p \Vert < 1 + \Vert x_N \Vert$. Ainsi,
        \[ M = \max(\Vert x_0 \Vert, \dots, \Vert x_{N-1} \Vert, \Vert x_N \Vert) \]
        majore la suite $(x_n)$.
        \item \textbf{$(x_n)$ admet au plus une valeur d'adhérence, et si c'est le cas, elle converge vers cette valeur d'adhérence.} En effet, si $(x_n)$ converge, alors sa limite est son unique valeur d'adhérence. Soit maintenant $x$ une valeur d'adhérence de $(x_n)$. Soit $\epsilon > 0$,
        \[ \exists N \in \mathbb{N} \text{ tel que } \forall p > q \geq N, \, \Vert x_p - x_q \Vert < \frac{\epsilon}{2} \]
        Soit $q \geq N$. Par définition de la valeur d'adhérence,
        \[ \exists p \geq q \text{ tel que } \Vert x_p - x \Vert < \frac{\epsilon}{2} \]
        Donc :
        \[ \Vert x_q - x \Vert \leq \Vert x_p - x_q \Vert + \Vert x_p - x \Vert < \epsilon \]
        ce que l'on voulait.
      \end{itemize}
      Supposons $E$ de dimension finie. Par le premier point, $(x_n)$ est bornée, donc incluse dans une boule fermée $B$, qui est compacte par le \cref{autour-de-la-compacite-5-1}, donc elle admet une valeur d'adhérence $\ell \in B$. Par le second point, $(x_n)$ converge vers $\ell$.
      \item Soient $(E, \Vert . \Vert)$ un espace vectoriel normé et $F$ un sous-espace vectoriel de $E$ de dimension finie. Soit $(x_n)$ une suite de $F$ qui converge vers $x \in E$. Notons que \textbf{$(x_n)$ est de Cauchy.} En effet, soit $\epsilon > 0$,
      \[ \exists N \in \mathbb{N} \text{ tel que } \forall p > q \geq N, \, \Vert x_p - x_q \Vert < \frac{\epsilon}{2} \]
      Soient $p > q \geq N$.
      \begin{align*}
        \Vert x_p - x_q \Vert &\leq \Vert x_p - x \Vert + \Vert x - x_q \Vert \\
        &< \epsilon
      \end{align*}
      Donc $(x_n)$ est une suite de Cauchy de $F$, qui est de dimension finie, donc complet par le \cref{autour-de-la-compacite-5-2}. $(x_n)$ converge donc dans $F$, et par unicité de la limite, on a $x \in F$. Par la caractérisation séquentielle des fermés, $F$ est bien fermé dans $E$.
      \item Soit $f \in L(E,F)$. On définit une norme sur $E$ par
      \[ \Vert . \Vert : x \mapsto \Vert x \Vert_E + \Vert f(x) \Vert_F \]
      Or, $\forall x \in E$,
      \begin{align*}
        \Vert f(x) \Vert_F &=\Vert x \Vert_E - \Vert x \Vert \\
        &\leq \Vert x \Vert_E - M\Vert x \Vert_E
      \end{align*}
      où $M > 0$, par le \cref{autour-de-la-compacite-4}.
      Ainsi,
      \[ \Vert f(x) \Vert_F = (1-M)\Vert x \Vert_E \]
      $f$ est une application linéaire bornée, donc continue.
    \end{enumerate}
  \end{proof}

  \begin{application}
    $\forall M \in \mathcal{M}_n(\mathbb{C})$, $\exists P \in \mathbb{C}[X]$ tel que $\exp(M) = P(M)$.
  \end{application}

  \begin{proof}
    Soit $M \in \mathcal{M}_n(\mathbb{C})$. L'ensemble $\mathbb{C}[M] = \{ P(M) \mid P \in \mathbb{C}[X] \}$ est un sous-espace vectoriel de $\mathcal{M}_n(\mathbb{C})$ qui est de dimension finie, donc $\mathbb{C}[M]$ l'est aussi et est en particulier fermé par le \cref{autour-de-la-compacite-5} \cref{autour-de-la-compacite-5-2}.
    \newpar
    Pour tout $n \in \mathbb{N}$, on pose $P_n = \sum_{k=0}^n \frac{M^k}{k!} \in \mathbb{C}[M]$ de sorte que $P_n \longrightarrow_{n \rightarrow +\infty} \exp(M)$. Comme $\mathbb{C}[M]$ est fermé, on en déduit que $\exp(M) \in \mathbb{C}[M]$. Donc $\exists P \in \mathbb{C}[X]$ tel que $\exp(M) = P(M)$.
  \end{proof}
  %</content>
\end{document}
